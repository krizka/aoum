Аоум. глава 15-я 04-04-1994г
Георгий Губин
  
 04-04-1994  ``– Поймите, многие истины к вам приходят сказками. И вы же говорите: сказка ложь, но в ней намёк. Далее, когда-нибудь, кто-нибудь из вас станет планетой. Согласитесь, что вначале она будет пустой''.
\people{**}
\people{Скажите, пожалуйста, как относиться к тому, что религиозным деятелями вводятся постоянно какие-то дополнения, изменения; отсюда явились староверы, реформаторы; одни крестятся двумя пальцами, а другие - тремя. Позволяя произвольное толкование Библии, часто происходит потеря авторитета. Не было бы лучше, если бы существовала одна церковь?}
\soul{Нет. Мы говорили вам о фанатизме. Представьте, одна церковь, и не имеет фанатов. Представьте одну веру. Мы говорили вам: Единство, - но не во взглядах.}
\people{Вот единство не во взглядах… Вы также говорили о благотворном влиянии щепоти, но ведь действительно крестятся и двумя (пальцами), это что …?}
\soul{Мы говорили вам о влиянии всего, во что вы верите. Да, даже физически есть разница между двумя и тремя перстами. И ранее было и пять.}
\people{Было и пять?! Любопытно. Этой версии я еще не знал.}
\soul{Вы не называли это щёпотью. Вспомните, вспомните первые религии, вспомните поклонение огню, вспомните поклонение солнцу, – разве там молились щёпотью?}
\people{Такой вопрос: Библия ведь вся пронизана иносказательностью. Что это – нежелание говорить правду в открытую? С момента написания Библии прошло 2000 лет, неужели наш уровень на том же месте, что и сейчас, в контактных ситуациях?}
\soul{Мы же говорили вам: пишется новая Библия. Пишется вашему времени, и мы приводили вам пример. Представьте: в те времена приходят и говорят об энергии. Поймут ли? Далее. У вас уже есть главы, но они так разбросаны… и противоречат друг другу.}
\people{Скажите, какой процент, извините за слово процент, о достоверности информации заложенной в Библии, вот на данный момент существующий? Ведь писали люди, а людям свойственно ошибаться. Могут переиначивать взгляды…}
\soul{Давайте спрашивать конкретно. Проценты. Как вам сказать проценты, если вы же говорили ``иносказательно''. ``Иносказательно''  всегда можно понимать по-разному.}
\people{Да, верно.}
\soul{Далее, мы говорили вам: вы сито. И вспомните, если к вам будут приходить и говорить так и так, говорить вам конкретно, – вы будете лишены пищи, и представьте: Христос пришел и сказал вам всё конкретно, хотя многое он делал так, и его не понимали. И ему тогда приходилось делать из вас сито.}
\people{Понятно.}
\soul{Далее, мы же говорили вам: Библия писана вами.}
\people{Вопрос такой: как относиться к предубеждению, что впоследствии будут появляться не пророки, а лжепророки? Как это относится к нашим изысканиям в разговорах с вами? И как вы оцениваете себя в свете слов Библии?}
\soul{Теперь представьте: вы сейчас создадите другую ветвь, и мы в этой ветви будем ложью, лжепророками. А в другой  мы можем быть просто пророками.}
\people{Так. Споры между христианами вызывают даже данные Моисею заповеди, к примеру, слова; ``не создай себе кумира'', ``не смей изображать ни живущих на небе, на Земле, и под Землёй''. Какой здесь истинный смысл? Почему их нельзя изображать?}
\soul{Это говорил не только Моисей. Вы поклоняетесь Христу, но умудряетесь не слушать его заповеди и нарушать, и при этом всегда находите идеологию, всегда находите оправдание, почему вы это делаете. Далее, вы говорите'' иносказательно''. Назовите мне человека, о котором говорили бы все одинаково. Вы можете мне назвать такого? Теперь представьте: Бога, и чтобы все его видели одинаково. Это что будет?}
\people{Не могу вам сказать… Что это будет – я не знаю. Ваше отношение к болгарской ясновидящей Ванге? Почему она сказала, что видит Гагарина, сидящего в коляске с зелёным забором, вы в курсе?}
\soul{Мы в курсе, но не будем отвечать на данный вопрос. Мы говорили вам о конкретных личностях.}
\people{Понятно.}
\soul{Далее, что даст это вам? Даже если мы скажем ``да, это было'' или ``нет''? С какой целью вы спрашиваете?}
\people{Понимаете, когда не ощущаешь мир и единственная связь с другим миром, с вами, хочется, всё-таки, узнать, как вы смотрите на это.}
\soul{Согласитесь, если мы вам скажем ``да'' или ``нет'', это ничего не изменит. И тот и другой ответ будет недоказуем, ничто вам не принесет.}
\people{Почему ``Земля была безвидна и пуста, бездонна и попросту вода… собрание вод назвал морями'' - цитата (фрагменты Бытие 1:2, 1:10). Здесь о какой воде идёт речь? О первом дне творения и втором дне… если своё название и конечный вид вода получила на третий день?}
\soul{Неужели вы представляете, что ваш мир был создан за 7 дней? Вы можете это  представить?}
\people{Ну как вам сказать…}
\soul{Поймите, многие истины к вам приходят сказками. И вы же говорите: сказка ложь, но в ней намёк. Далее, когда-нибудь, кто-нибудь из вас станет планетой. Согласитесь,  что вначале, она будет пустой.}
\people{Н-да. Я предполагал, конечно, что к этому идёт, но, вы подтвердили, честно говоря. -Ну, я пожалуй, по тетради закончу. Тут вопросы товарищ передавал. Меня такой вопрос волнует: а кто ж эта Мария была всё-таки? И кто она сейчас? В этой жизни мы с ней встретились? Вы-то видите, это мы пока ещё не можем узнать.}
\soul{А если мы вам скажем, что это было искусственное создание?}
\people{Искусственное создание? Это как? В прошлом – искусственное, или сейчас?}
\soul{В прошлом. Она была создана вами. Искусственно. Вы догадываетесь?}
\people{Нет.}
\soul{Подумайте. Или вы считаете, что в те времена нельзя было создать подобную технику?}
\people{Не понял. Это что, зомби что ли?}
\soul{Давайте скажем так, - механическая кукла.}
\people{Механическая - в буквальном смысле?}
\soul{В буквальном. Это будет вам подсказка, если вы займётесь поисками: где же вы  всё-таки были, что вы делали в те времена. Согласитесь, в те времена, вам называть года  или вы помните сами?}
\people{Это я помню.}
\soul{Механических кукол было достаточно мало.}
\people{Да, довольно редкая была штучка и дорогая.}
 1-2-3-4-5-6
\people{Многие ли стихи я повторяю буквально, как тогда?}
\soul{Вы порой буквально повторяете даже жизнь. Далее. Многие. Но, согласитесь, мы  говорим о вас, что вы повторяете, – это не плагиат, это ваше. Спрашивайте.}
\people{Скажите, а почему так долго переводчик выходил..?}
\soul{Борьба…}
\people{Понятно. Ага, насчёт борьбы… Скажите, если в нас 7 тел, и каждое борется друг с другом?}
\soul{Нет. Боретесь вы.}
\people{А, т.е. физическое с тонким?}
\soul{Давайте скажем так: самая первая одежда – это ваше сознание. Но не путайте – у  каждого из семи тел,  есть сознание. Мы говорим о самом первом. О самом грубом. Она -да, она  борется с более тонким. Также, она попирает более низшие.}
\people{Как это ``первое'' и ``более низшие''? Как сравнить?}
\soul{Мы говорили вам, что есть ``ниже'' вас, в вашем физическом понятии. Вы не ``первые'',  но и не последние. Да, вы имеете 7 тел. Но, если быть точнее, в вашем физическом мире  их сорок девять.}
\people{А, да-да-да. Скажите, это 49 смертей должно быть?}
\soul{Множество вы уже прошли.}
\people{А вы не скажете, сколько мы уже прошли?}
\soul{Нет.}
\people{Верна ли моя догадка: когда-то мы были в вашей сфере, во внефизическом мире, и потом нас что-то погубило, и мы опустились, так сказать, на дно?}
\soul{Давайте скажем так. Первое: это было не ваше. Далее. Нет, вы не правы.}
\people{Или может быть такую теорию принять за  более/менее истинную, так сказать, что все идут снизу вверх?}
\soul{Давайте скажем по-другому. Вы играете в жизнь и путешествуете. Вы пришли в}
 этот мир и забылись.
\people{Наверное.}
\soul{Теперь к вам приходят ``родственники'' и говорят ``вернитесь''. Вы их принимаете, как шёпот. Многие родились здесь.}
\people{Это как: ``родились здесь''? Расширьте тему, пожалуйста.}
\soul{Ну, тогда подумайте. Вы имеете детей?}
\people{Да.}
\soul{Почему бы здесь не иметь их? Далее, как вы понимаете понятие ``вечность''? Что, вы решили, что какое-то определенное число, когда-то возродилось и больше никогда не умирает и не рождается? И это число остаётся постоянным? В вашем понятии такова вечность?}
\people{В моём понятии сложно понять… у меня не одно понятие слова ``вечность''… Множество.}
\soul{Так вот представьте: вы путешественник, и играли в жизнь. Пришли сюда и забыли, -  ``Ищущие''.}
\people{Так.}
\soul{К вам приходят, и будьте внимательны: к вам приходят родные и зовут обратно. У вас же уже у многих есть дети, рождённые здесь, рождённые в этом мире. И вы их называете - ``землянами''.}
\people{Тогда вопрос такой: а вот, зовут нас – это значит, если мы соберёмся уйти, мы должны всё равно прожить жизнь… ну без самоубийства там, без ничего такого..?}
 (прим. имеет в виду грехи).
\soul{Что в вашем понятии ``самоубийство''?}
\people{Ну,  взял человек нож и зарезался…}
\soul{Это просто ``снял одежду и ушёл''? Извините, вас поймают, в прямом смысле, и оденут те же одежды или грубее.}
\people{А кто?}
\soul{Вам будет легче, если вы будете знать имена судей?}
\people{Имена? Что-то я не расслышал…}
\soul{Да.}
\people{Нет-нет, зачем имена. Вы так объясните кто. Не люди ж? Не земляне же? В смысле, на нашей Земле. Не физические?}
\soul{Будьте внимательны. Вы говорили, и мы говорили вам, о новой эре, о новых землянах. И мы вам говорим, что вы рождаете землян. Вспомните, мы говорили вам о детях и взрослых. Мы говорили вам, что мы ваши дети и мы ваши родители. И мы вам скажем: вы родители нового поколения. Но не понимайте это в буквальном смысле,- ваших сыновей и дочерей.}
\people{Вопрос по Библии. Почему сказано там, что второй раз замуж не выходить. Если ты стала вдовой, то будь вдовой.}
\soul{Мы говорили вам: у вас есть сердце, у вас есть любовь. Кто бы к вам не пришёл и будет указывать ``люби тех и люби тех'', у вас своё должно быть. Своё, поймите! И не путайте Бога и религию. Это разные вещи.}
\people{Ну, в принципе, у меня такая картина вырисовывается, что нет такого, что можно было бы осудить со стороны, потому что человек сам только о себе может всё знать, и он вправе себя судить. Так это?}
\soul{Вы вправе судить себя или весь мир?}
\people{Себя конечно; за что мир?}
\soul{Вы прежде научитесь судить себя, и, тогда, вы  не осудите мир.}
\people{Ну, я о том же говорю.}
\soul{Далее. Поймите: Бог, вера и религия – это разные вещи. Религия, по вашим понятиям, это проститутка, которая создана вами, людьми, для своих же удобств. Да, мы сказали это грубо, но это так. А теперь вы, я надеюсь, сможете понять разницу между Богом и верой? Бог есть Бог. Вера, это одно из видений Бога. Согласитесь, видений может быть множество, а Бог будет один. Мы говорили вам: назовите человека, о котором говорили бы все одинаково.}
\people{Такого   нет.}
\soul{Далее, мы говорили вам, что пишется новая Библия. Пишутся новые главы. И говорили вам о множестве глав, которые противоречат друг другу. Но будьте внимательны. Где вы видели противоречия там? Где?}
\people{Но я ещё не видел множества глав.}
\soul{Разве? У вас сейчас множество учений. Множество. И даже те, которые гонимы и не верны, имеют зерно истины. Согласитесь, на полной лжи никогда ничто не получается. Далее, мы говорили вам о религии, и говорили грубо. Да, если быть точнее, то люди делают это. Люди ведут себя, как мы говорили, и называют это ``религией''.}
\people{Под Богом можно ли подразумевать весь мир, всё?}
\soul{Да.}
\people{Это будет правильнее, чем другие высказывания? (о Боге. Прим.)}
\soul{Ну, представьте, что мы сейчас скажем ``правильно'' или ``нет''. И тогда мы создадим еще одну религию. И еще. Одну борьбу и множество. Согласитесь.}
\people{Т.е. у каждого свой бог получается?}
\soul{Да, но только не надо силой заставлять верить в своего Бога.}
\people{У нас тут сейчас был спор (прим. между собой) – вы знаете, о чём он, чтоб я не пояснял?}
\soul{Спрашивайте.}
\people{С одной стороны, конечно, переводчик говорит, что надо что-то делать. С другой стороны - как мы можем знать, что делать, если мы не знаем, к чему это приведет? Т.е. мы, получается, должны пройти дорогу сами? Потом вернуться…}
\soul{Простите. Когда путешественник собирается в дорогу открывать новые земли, – он знает куда идти? Он знает, что он увидит там? Но, он же, идёт! И если он человек, - он, в любом случае, остается человеком, что бы с ним ни случалось. В том ваше счастье, что вы не знаете дороги, по которой идете. Дороги многие были вам знакомы, и потому скучны. Вы выбрали новые. Вы создали себя в этом мире.}
\people{Скажите те 7 тел… ну 49, если быть точнее… как вы сказали…}
\soul{Давайте будем говорить о семи.}
\people{Хорошо. Эти 7 тел… сейчас идет борьба между физическим, данным, в котором мы себя осознаём и более, скажем, тонким? Так что ли? Это третий/четвёртый уровень, да?}
\soul{В более тонком, идёт борьба между этих тел.}
\people{Т.е. более тонкие борются, да?}
\soul{Более грубые. Хотят завоевать… В вашем понятии есть ``добро'' и ``зло''. Идёт борьба между двумя силами. У вас есть еще понятие ``отрицательное'' и ``положительное''.}
\people{Хорошо, я понял.}
\soul{Вы приняли отрицательное  – плохо, положительное – хорошо.}
\people{А у вас на это точка зрения какова?}
\soul{Мы же говорили вам, у нас нет ваших тел.}
\people{Я имел в виду ``отрицательное/положительное''. Или у вас нет даже понятия ``отрицательного - положительного'', а просто…}
\soul{В вашем понятии, мы уже множество раз умерли.}
\people{Интересно. Вы сказали, что мы должны с вами соединиться, дойдя до вас, и не сделав ваших ошибок.}
\soul{Вы не сможете их сделать, даже желая этого. Мы же говорили о незнании дороги.}
\people{А не можем ли мы случайно повторить один к одному? (ошибки. прим.)}
\soul{Нет.}
 (прерывается связь)
\people{Уже дважды на…}
\soul{Уже, нет.}
\people{Это хорошо или плохо, с вашей точки?}
\soul{Нет, это ни хорошо и не плохо. Это одна из дорог. Далее, вы же сами говорили: нет  ничего плохого, есть только уроки.}
\people{Ну да.}
\soul{Согласитесь, придя в спортивный зал, вы насилуете себя, и со стороны для новичка  будет казаться пыткой.}
 1-2-3
\soul{Нам трудно работать. Мы не можем  и не должны и не имеем права отключать вас полностью. В вашем понятии ``транс'' – это что-то плохое. Да, вы правы. Это действительно плохо – полный транс. В данном случае, сознание работает. Работает, но на себя.}
\people{На себя? }
\soul{В вашем понятии – это работать по кольцу. Замкнутому.}
\people{А вы, значит, разрываете, потихонечку. Да?}
\soul{Нет, мы создаём это кольцо. Поймите, сознание занято самим собой. Ей не много дел до нас.}
\people{Хорошо.}
 1-2-3-4-5-6
\soul{Всё чаще и чаще. Уже не мы корректируем, а корректирует оно. Мы воздействовали на вас физически только в плане речи. Но согласитесь, это физический план. И сознанию приходится отвечать за него. Сознание действует вашим языком.}
\people{Скажите, а вот, в не физической области, ну в вашей области, она тоже имеет свои пути, дорожки, разные миры?}
\soul{Множество.}
\people{А в чём их коренное отличие от физических? По принципу, я имею в виду.}
\soul{Мы же сказали вам, что уже множество раз, в вашем понятии, умерли.}
\people{Т.е. это уменьшение времени идёт, да? По нашим понятиям?}
\soul{Нет. Это утоньшение материи до той степени, когда она просто исчезает.}
\people{Но она в принципе не должна исчезнуть никогда, как я понял. Просто она должна вечно  утончаться?}
\soul{Нет, вы неправы. Она трансформируется в энергию. Согласитесь, что любая материя, это, всё-таки, энергия.}
\people{Да.}
\soul{Но, ваша материя, ваш камень, – это застывшая. Да, он обладает энергией, но столь ничтожной, что она - застывшая. И, чем она ``застывшей'', тем твёрже. А теперь, смотрите  аналог - камень и вы.}
\people{Так, вот вопрос такой: -  Вы сказали, что физически мы вас не интересуем. Т.е.  вообще не интересуем. А почему же тогда от нашего физического состояния зависит ещё и наше эмоциональное состояние?}
\soul{Мы же говорили вам: эмоции – это физический план.}
\people{Т.е. всё, на что действует физика, это физика.}
\soul{Нет. Так нельзя говорить. Но вы живёте в физике. Есть планы, не имеющие физики, но, всё-таки, воздействуют. Подумайте сами. Душа, и всё же она воздействует. Она движет вашими  телами.}
\people{Ну, да. С другой стороны, вы говорили, что мы не имеем даже понятия, что такое душа.}
\soul{Простите, если мы будем уметь понять что такое душа, то нам просто скучно будет жить.}
\people{Да, это верно.}
\soul{В том и смысл, в вашем понятии, как вы говорите – Божий промысел. Это вечный поиск. Вечный поиск себя. Поиск души. Вы подумайте сами, рассудите логически. Душа, по вашим религиям попавшая в рай или в ад… Подумайте, значит, она уже делима в этих понятиях, - ``рай'' и ``ад''.}
\people{Скажите, а, вот, уроки этой жизни в будущем… мы будем как-то… Они повлияют на нас?}
\soul{Вы не слышали начала.}
\people{М- м- м. ``Душа  хочет воплотиться..''. Скажите, а есть ли иное – не наше и не ваше ``мирожитие'', так скажем?}
\soul{Множество.}
\people{Т.е. и физическое и нефизическое.}
\soul{Множество. Есть множество физических, в вашем понятии, они тоже будут не физическими, ибо они будут отличаться и не содержать ваши таблицы. Есть множества, подобные вам. И мы говорили вам о молекулах. Есть множества в духовном плане. Но, начало одно.  И это начало ищут все. Вы это называете Богом.}
\people{Ага, всё-таки, все ищут.}
1-2-3
\soul{Что говорили вам о богах? Вы персонируете. Вы можете представить персону – весь мир?}
\people{Ну что ж, вполне можно, только возникает вопрос…}
\soul{Да? Тогда, где же вы?}
\people{Внутри. Мы - частицы, молекулы этого мира.}
\soul{Молекулы этого мира? Значит, тогда вы уже не все. Тогда в вас  уже нет всего мира, раз вы только молекулы.}
\people{Ну, если взять теорию ``всё отображается во всём'', о которой вы говорили, значит, весь мир отображается на нас? Т.е. у нас в каждой молекуле есть ДНК, а из неё возникает такой же мир, подобный. Значит, можно судить, что вся молекула, одна молекула – это информация обо всём мире. Мы – отображение всего мира. У меня другой вопрос. Если этот мир очень грубо представить в форме человека, каждая молекула, осознавая себя, продвигается к мозгу, чтобы…}
\soul{Нет. Вы опять персонифицируете, в вашем понятии вся Вселенная – это человек. И соответственно, вы уже создали органы. Но, мы же говорили вам о мире едином. В вашем же понятии, существуют различные органы.}
\people{Различные функции одного организма?}
\soul{Тогда что вы скажете: мозгу не хватает вашей молекулы, значит мозг неполноценный?}
\people{Нет.}
\soul{Да, была такая теория и даже пыталась (прим. стать) религией. Нет, вы не правы.}
\people{Но у меня возникает второй вопрос: вы знаете, что для вас тоже есть - недоступное, т.е. представления обо всём вы не имеете. Полного.}
\soul{Поймите, чем больше вы познаёте, тем больше вы хотите знать, и тем больше будет неизведанного. Мы же говорили вам о бесконечности, и вы не можете понять: как можно знать всю бесконечность и при этом оставаться самой бесконечностью.}
\people{Да-а-а, вопрос, конечно, сложный. Ладно,  допустим, нам просто не дано понять  нашим мозгом это.}
\soul{Вы не хотите. Это будет точнее. Вы боитесь.}
\people{Я ещё, насчёт другого. Вы сказали: мы можем стать планетой, а галактикой мы можем быть? Т.е. совокупностью планет?}
\soul{Да.}
\people{Значит, допустим, конкретный человек станет галактикой, т.е. дорастёт сознанием до неё. Мало ли что случается – может и такое быть, то включающие в эту галактику планеты – это тоже будут какие-то люди?}
\soul{А вы подумайте и проведите аналогию. Если человек может стать планетой, может быть Земля, может быть, когда-то была человеком, и вы – её составляющие? Будьте логичны.}
\people{Я всё к другому клоню. Если кто-то галактика, а кто-то в этой галактике планеты, то вот и получается, что всё больше и больше осознавая, но так сказать, менее осознаваемое, находится в его ведении…}
\soul{Давайте скажем так: в вашем понятии ``стать планетой'' – это не быть планетой, это осознавать, что вы планета, осознавать и жить планетой, но это не значит, что вы физически будете планетой. У вас даже не хватит массы, чтобы стать ею. Будьте логичны. Да, в вашем понятии, это можно назвать дух. Духом планеты. Согласитесь, на этой планете есть множество не доросших. Они по сравнению с этим духом будут называться детьми.}
\people{А у вас там тоже есть планеты?}
\soul{Мы же говорили вам: мы иной мир. У нас нет физического плана. Мы - совершенно иное. Поймите это. Вы же спрашивали, мы вам только что отвечали: есть множество физических и не физических. Хорошо, если вы не можете понять, что такое ``иное'', тогда представьте совершенно иная таблица Менделеева. Может быть тогда вам будет легче?}
\people{Ну, ладно. Вы сказали, что прошли стадию нашу и мы ваши воспоминания, это значит, что вы были на какой-то планете такими же людьми.}
\soul{Будьте внимательны. Мы говорили вам о детях. О родных.}
\people{Еще вопрос. Сейчас такая пошла жизнь, что каждый вольно или невольно старается урвать кусок пирога, урывая его, но отрываясь от другого. Как вот жить в этой жизни? Сейчас очень сложные времена настали…(1994г)}
\soul{Простите, мы вам говорили ``сито'' и мы вам говорили, что наступило время, в вашем понятии, экзамена. Согласитесь, если мы вам скажем, как жить, это будет всего лишь только шпаргалка. И что она вам даст? Будет ли выучен вами урок? И что вы будете делать в следующей жизни, где мы можем и не придти. Хотя, такого еще не было.}
\people{Дай Бог.}
\soul{Вы не внимательны. Мы дали вам понять, что в любых реинкарнициях мы были с вами и разговаривали и, в вашем понятии, ``вели контакт''.}
\people{Вопрос такой: Вы говорите, что нам хорошо бы добраться до 7-го тела ( прим. говорили в прошлом контакте об этом), чтобы уйти выше. В принципе, мы тут уже ``загулялись'' похоже, да?}
\soul{Как вы понимаете ``загулялись''? Как вы загулялись, если вы не знаете по каким целям вы здесь гуляете?}
\people{Да, цели мы не помним…}
\soul{А как же тогда вы можете рассуждать, что вы уже слишком много здесь гуляете или нет? Вы, сперва  найдите ту цель, сделайте, что вам надо, тогда вы поймёте: много или мало. Многие остаются воспитывать новых детей. Вспомните начало.}
\people{Да. Вы сегодня подтвердили свои слова о том, что вы пришли породить в нас сомнения – сомнений вы родили очень много.}
\soul{Нет. Мы просто будим вас. Они в вас были, и новых мы не рождаем. Мы делаем так, чтобы ничего нового вы не получили, а смогли разбудить всё старое. И всё это старое для вас ново, ибо было крепко забыто.}
\people{Скажите, а реинкарнации существуют только для того, чтобы дать ещё один шанс?}
\soul{Нет, вы не внимательны. Вам мы уже говорили, что многие остаются.}
\people{Нет… дать шанс вспомнить, зачем ты здесь?}
\soul{Мы вам говорили, что многие остаются, чтобы воспитывать детей. Новых детей. Но ещё более – забывшие, и так и не нашедшие и не вспомнившие.}
\people{Да…}
\soul{Спросите, кто вы: вспомнившие или воспитатели? Вы хотите это спросить?}
\people{Да я не знаю, корректно это или нет…}
\soul{А вы подумайте.}
\people{Да… ``Зачем родился я на свет я помнил не всегда''.}
\soul{Тогда подумайте. Если мы вам скажем: мы пришли, вы пришли, - воспитателем. Чему вы учите детей, если вы не помните? Мы вас когда-то спрашивали. Почему вы не помните? И вы мне не дали ответ.}
\people{Я сказал ``я не помню, потому что я не помню''. Как ещё можно сказать почему?}
\soul{А теперь подумайте, как вы ответили: ``я не помню, потому, что я не помню''. А мы говорим вам, что вы боитесь вспомнить.}
\people{М-м-м. Ну, вот, не уверенность нас останавливает…}
\soul{Неверие.}
\people{Может быть неверие.}
\soul{Далее, – страх. И мы когда-то говорили вам о страхе. Вспомните.}
\people{Страх может быть разный. Страх ошибиться – тоже страх.}
\soul{Тогда ищите в себе. Поймите, и мы просим вас: не верьте нам. Не верьте, как фанаты. Ищите, ищите, что говорим мы, как говорим вам. Ищите. Ищите в себе.}
\people{Хорошо.}
\soul{Вы же, слушаете, и не более. Вы не помните, вы не хотите помнить. Мы же говорили вам, и даже в прошлой жизни, мы говорили вам: помните только то, что хотите. Вам вначале было прослушано 5 реинкарнаций. Вы же, не заметили.}
\people{В начале этого контакта?}
\soul{Да.}
\people{Записано там… было плохо слышно и не всегда понятно. Просто куски какие-то, а к чему относящиеся… единственное, понял я это начало 1200 какой-то год, и вот 16… остальное не понял ничего. Ну ладно.}
\soul{Скажем вам, и вы сами уже поняли, вы и мы были всегда с вами вместе. И мы всегда говорили с вами. И когда-то вы находились на месте переводчика. И он задавал вопросы вам.}
\people{Я, вот, только не могу понять, как это так происходит. В принципе, душа имеет  своё лицо, и только своё, правильно? А вы не можете подсказать, в каком году  я находился на его месте?}
\soul{Нет. Это будет всего лишь шпаргалка. Мы и так уже слишком много вам. Далее, главные цели – вспомнить не прошлые жизни, а вспомнить все те чувства, все те переживания, все те эмоции, всё то, что вы называете движением души. А если вы будете просто вспоминать физически, то ваше тело вы можете вспомнить, даже, когда вы не были человеком. Просто вспомнить – это ничего не даст.}
\people{Скажите, на одном сеансе человека загипнотизировали, и вспомнил он прошлое, как он 5000 лет назад был волком. Он стал на четвереньки и завыл. Он действительно был в теле волка?}
\soul{А вы подумайте. Почему бы нет, если вы можете менять тела мужские, женские. И даже в этой жизни вы несколько раз меняете тело. Вспомните. Вы учили биологию.}
\people{Ну, да. В течение там 7-ми лет…(тело полностью обновляет все клетки физич. тела. прим.)}
\soul{Почему бы вы не могли поменять другие одежды?}
\people{Но вы сказали, что человечество образовалось 15000 лет назад, а никак не 5000.}
\soul{Мы говорим вам об одежде. Почему вы не можете сопоставить? Далее. 15000 лет назад пришли вы.}
\people{Т.е. конкретная личность? Или как? ``Вы”… Это что вы подразумеваете под словом ``вы''?}
\soul{Мы говорили вам о ``путешествующих'', мы говорили вам об ``ищущих''. Вспомните.}
\people{Угу. Путешественники они же путешествуют и наблюдают, у них никаких таких целей нет. Они восхищаются, смотрят, узнают…}
\soul{Тогда подумайте, путешественник, попавший на остров, в вашем понятии, необитаемый. Кто он становится? Он путешествует или он уже просто начинает жить, существовать, чтобы продлить свою жизнь? Подумайте. И ждёт помощи извне, вместо того, чтобы построить самому и уйти с этого острова. Вы сейчас похожи на подобных. Вы ищите помощь извне, но не в себе.}
\people{А построить и уйти можно только найдя в себе что-то? Может, есть другие пути какие-то?}
\soul{Есть. Есть другие пути. Но бойтесь – к вам могут приплыть рабовладельцы. И тогда, они будут вашими спасителями, а вы будете их рабом.}
\people{Ещё вопрос. О питании. Каждый питается нижестоящим?}
\soul{Нет.}
\people{Ну, я грубо говорю ``нижестоящим''.}
\soul{Нет. В любом понятии нет.}
\people{Но вы, вроде, говорили…}
\soul{Подумайте, как тогда можно говорить о вашей духовности, если вы только питаетесь низшим. Где же тогда ваша духовность? И подумайте, рассудите логически. Если вы бы питались только низкими, низкие питались бы ещё более низкими, – сколько это может продолжаться?}
\people{Вы как-то обмолвились, что и вами питаются, и вы питаетесь, так что нельзя говорить, что… и, вроде как, так мир создан, что каждый питается каждым.}
\soul{В вашем понятии, ``питается'', это, почему-то, грубо.}
\people{Ну, почему же…}
\soul{Хорошо. Давайте сделаем так. Вы приходите в кинотеатр и смотрите фильм. Согласитесь, что вы питаетесь. В этот момент вы питаетесь. Разве можно сказать о плохом? И согласитесь, если фильм для вас тяжёл – вы уходите. Согласны?}
\people{Да.}
\soul{Вот вам и пожалуйста, кто кем питается. Если к вам приходят и питаются злом, вы будете опустошены и наоборот. Но согласитесь, подобное - притягивает подобное.}
 (теряется связь ``переводчика'')
\soul{Придёт время, и вы не сможете нас слышать. Наступит скоро. Но, обещаем вам, что вы услышите нас, ибо, мы будем стараться, и подготавливать почву. Поймите, мы посадили вам зерно, и оно будет расти. Согласитесь, что в начальный рост -  его надо беречь. И потому мы вас покинем, в вашем понятии, и лишь только  в вашем, на время.}
\people{Мы, даже внутри себя, вас не будем слышать?}
\soul{Нет, вы будете слушать и будете на связи. Но мы будем молчать.}
\people{Т.е. нам придётся самим выкарабкиваться…}
\soul{Мы принесли вам зерно, пусть растёт. Согласитесь, вы должны научиться выращивать его. Да, мы можем вырастить за вас, но что это будет? Придёт другая жизнь, другая реинкарнация и начинать всё с начала? Поймите, мы говорили вам о памяти, о различных памятях. К сожалению, в физическом мире носителем вашей информации является мозг. Он приёмник и он кладовая этой памяти. А если быть точнее – ключ, столь грубый, что не всегда открывает то, что надо. Не это мне говорить вам.}
\people{Вопрос такого рода: а вот если мы, уйдя из жизни, ну, рано или поздно нам придётся это сделать, из тела…}
\soul{А почему вы так торопитесь уйти из этой жизни?}
\people{Ну, я говорю ``рано или поздно''.}
\soul{Так вот, когда вы уйдёте, тогда и спрашивайте, как идти далее. Куда вы торопитесь? Вы даже не знаете этой жизни, а уже говорите о другой.}
\people{Не-е, я не о том хотел сказать.}
\soul{Нет, давайте не будем говорить о том, что будет далее и далее. Научитесь жить здесь. Научившись жить здесь, вы сами выберете, когда вам идти или не уходить.}
\people{Ну, реинкарнировать мы будем всё равно? В смысле, если мы решим не уходить, мы будем реинкарнировать или жить в этом теле?}
\soul{Давайте будем говорить тогда о прошлом, если вы не можете оставить эту тему, давайте говорить о прошлых инкарнациях. Но, запомните: не пройдя путь, нельзя говорить, что будет далее. У вас есть пословица `` делить шкуру не убитого медведя''. Далее, мы говорили вам от 11-ти до 19-ти. (счет. прим.) И наоборот. Вы можете пользоваться этим, но помните. Нет, не бойтесь, что будут какие-либо осложнения.}
\people{А о чём же тогда вы нас предупреждали?}
\soul{Мы вас предупреждали о злоупотреблении. Это первое. Второе. Да, ему будет сложно, но если будет очень, то мы сможем остановить его. Но только в том случае, если это происходит по нашей вине. Если же будете виноваты вы, мы не будем делать ничего. Потому что это будет ваша ошибка – вы же и исправляйте. Вы же, чаще, делаете их, но не знаете, как исправить.}
\people{Ну да. Тогда, лучше не надо.}
\soul{Это касается не только этого счёта. Далее. Мы работаем с переводчиком, и нам приходится выполнять кое-какие его условия, хотя, он даже не знает об этом.}
\people{Странно устроено сознание, что, вроде, вы выполняете его условия, но оно, в том же роде, не даёт…}
\soul{Подумайте, мы же говорили вам: мы играем в ваши игры. Согласитесь, что лучшее обучение – обучение в игре. Сознание видит, что это игра и чувствует себя сильным, позволяет нам некоторые шалости в его понятии. Далее. Если брать, в вашем понятии, то, конечно, мы выше этого сознания. Но, согласитесь, если вы будете только бить ребёнка и не играть с ним,- что из этого выйдет?}
\people{Ну, знаете, это вопрос довольно двоякий. Я видел, которых били и вышли}
хорошие люди… 
\soul{Да?}
\people{Да.}
\soul{Скажите, их постоянно били? Всегда? Или были, всё-таки, времена, когда их ласкали, когда с ними, всё-таки, играли?}
\people{Ну, не без этого, конечно.}
\soul{Ваше сознание и так достаточно ``бьётся'' и без нашей помощи.}
\people{Хорошо. Так, что бы спросить? Скажите, А вот, логически… А, в принципе, нет…}
 (счет)
\soul{Спрашивайте.}
\people{Так. Скажите такое… Всё-таки, как вот прожить жизнь, чтобы никому не ``насолить''? Ведь, чтобы выжить в этом мире, приходится ``рвать зубами кусок мяса'' у другого'',  такого же нищего?}
\soul{Да? Вы представляете это только так - ``рвать кусок мяса''?}
\people{Я образно говорю.}
\soul{Образно? В вашем образе всё и есть. В вашем образе и есть ошибка. Вы представляете мир, как борьба. Борьба, не больше. И вы хотите быть сильным и хотите выиграть, вместо того, чтобы вести диалог, вместо того, чтобы научиться понимать. Много ли с вас вырывали кусков? И подумайте, много ли ваша мать ``вырывает'' с вас?}
\people{Нет, мать, как раз, и  даёт всё.}
\soul{А тогда подумайте, и сделайте далее.}
\people{Да.}
\soul{Рано или поздно вы опомнитесь, но будет поздно. Далее мы говорили вам о времени, экзамене. Почему вы не помните  то, что сказано? Сказано.}
\people{Экзамен - в какую сторону? Что тот самый умный, который может себя прокормить, выжить в этой ситуации или, наоборот, - кто последнее отдаст?}
\soul{Вот и подумайте, это один из вопросов.}
\people{Вот и думаем.}
\soul{Мы пришли к вам не шпаргалками. Далее, мы говорили вам, для вас всё старое как новое, и вы не помните. Вы так увлеклись игрой в жизнь, что забыли, что такое сама жизнь. Вы любите бороться, странно, вы пришли увидеть мир и сделать его другим, и стали бороться. Вы боретесь даже в себе, вы боретесь даже во снах, вы боретесь между своими телами, между своими одеждами. Удивительно, не правда ли? Теперь представьте, как вы несёте мир, если вы забыли, что это такое? Вы пришли сюда в шлеме, а взяли в руки меч. И вы, это называете жизнь? Тогда скажите, на поле битвы есть жизнь, она рождается или убивается?}
\people{Убивается.}
\soul{А теперь скажите, кто создал такое? ``Боги'' пришли и столкнули вас?}
\people{Понятно.}
\soul{К вам приходят ваши родственники.}
 (Конец контакта)
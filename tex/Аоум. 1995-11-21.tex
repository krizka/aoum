Аоум. глава 28-я 21-11-1995г
Георгий Губин
\people{**}
 
21-11-1995г   
\people{Из серии вопросов о пространстве-времени.  Что такое пространство? Какова его мерность?  Пожалуйста, ответьте. }
  
\soul{У вас - всего три. Вы говорите, о безмерности. Нет такого пространства, которое не имело бы мерности, в вашем понятии. Любое физическое тело обладает мерами. И, в зависимости от сложности его, будет обладать или более или менее этими мерами. }
\people{Скажите, а постоянна ли мерность пространства на Земле и в космосе? И от чего она зависит?}
\soul{Ничто не может быть постоянным. Космические реки, в вашем понятии, изменяют. Даже вы среди себя говорите: ``у него плоское мышление''. }
\people{Скажите, а какой наилучший способ для перемещения в пространстве? }
\soul{Мы говорили вам: научитесь доверять снам, научитесь доверять чувствам. Тогда вы будете знать гораздо больше, чем просто физикой. Физически, перенести плоть не сложно. Даже сейчас  у вас есть все знания, чтобы сделать это, но вы не можете их соединить. Все науки ваши разбиты по частям. И потому, не соберёте. }
\people{А когда-нибудь соберём, поймём?}
\soul{Это зависит от  вас. Если вы будете только ``наукой'' собирать – не соберёте или соберёте, но сделаете множество ошибок. Соедините. Нельзя рассматривать картины отдельно кусочками. Пока вы не создадите одну науку. Тогда, стоит ли говорить о перемещение вашего в  плоти?}
\people{Но, наверное, для этого на Землю должен прийти столь же гениальный человек, как Эйнштейн или кто-то в этом роде? }
\soul{Только вы придумали эти термины. }
\people{Ну, боюсь, что не способны, кто-то из нас, такую теорию придумать. Общую науку.  }
\people{Тут нужно обладать синтетическим мышлением. }
\people{Да.}
\soul{Давайте скажем так: вы должны перестать бояться. Это первое. И это удивительно – первый шаг сделает тот, кто совсем не относится к наукам. Ибо наука ваша – это уже тормоз для вашего мышления. Ибо вы установили законы и веруете в них, хотя, не знаете даже, действуют ли они и когда они действуют. У вас есть множество исключений из правил. И всё это вы называете ``законами'', выдуманными вами. Неужели вы думаете, что природа подчиняется вашим законам? И первый, кто сделает шаг, будет тот, кто не будет знать этих законов. В вашем понятии,- совсем не гениальный человек. }
\people{Ну, логично. Скажите, а для перемещения в пространстве, верна ли гипотеза, что верное перемещение – это путём искривления пространства?}
\soul{Нет. Есть множество путей. И давайте уточним: мы будем говорить, о плоти? О физическом перемещении или, о каком-либо другом?}
\people{Мы, пока, о физическом. Есть ли способ перемещения физического тела путём искривления пространства?}
\soul{Множество.}
\people{Как?}
\soul{Есть. Но можно сделать и проще. Можем найти недостающее, в вашем понятии, ``четвёртое измерение''. И тогда, вы можете перемещаться, уже не завися от первых трёх. Можно и наоборот – уйти от одного от одного измерения. И остаться в двух. И тогда, вы тоже сможете перемещаться. И только вами выдумано - о плоскости. Один из лучших способов, пока, для вас – это или уйти от одного измерения или получить другое. Искривление пространства? Что вы называете искривлением пространства, если вы не знаете что такое пространство? Что вы собираетесь искривлять? В вашем понятии пространство – это пространство, которое обладает конечными, заметьте, конечными координатами. И даже путь и цель, куда вы хотите переместиться, вы определяете координатами. А можно сделать проще. Уйти от координат. Но для этого вы должны забыть, где находитесь вы сейчас. Иначе, если вы знаете, где вы находитесь – это уже координаты. Вы согласны?}
\people{Да. }
\soul{Уйдите от них. Представьте, что вы не знаете, где вы находитесь. Или давайте скажем так: заблудшему путнику скажите, что он находится там-то и там-то, хотя, он находится от этого места очень далеко. Для него – да, он действительно будет находиться в этом месте. Ибо для него, нет лжи. Вы были искренны, когда говорили:'' влезть в рай''. Да, постороннему наблюдателю трудно будет доказать, что он находится в другом. Вы согласны? }
\people{Согласен. Но…}
\soul{А теперь, представьте следующую ситуацию: путник, которому определили ложные координаты. Он уже отмеряет свой путь относительно этих ложных координат. Правильно?}
\people{Да.}
\soul{В вашем понятии, он сделает ложный путь. Его дорога будет ложна. Для него – нет. Что можно сделать?}
\people{Ну, он же не придёт в требуемую точку.}
\soul{Для постороннего наблюдателя – да. Но вы хотите переместиться? Вы, перемещаясь, считаете себя посторонним наблюдателем?  Нет. Итак, давайте будем говорить о волновых<процессах> в природе. Путник, настроившись, не зная, что он находится здесь – он уже создал свои координаты, которые полностью не совпадают с посторонним. И, относительно координат, скажем так -  относительно колебаний, своих колебаний, – не посторонних, он может перемещаться в любое пространство относительно своей координаты. Но, для постороннего наблюдателя, он остаётся на месте. Ибо, посторонний наблюдатель не менял своих координат. Не настроился на него. И, потому, можно сделать проще. Если вы хотите перенестись на какую-то из планет, вы должны настроиться на неё. Физически? Значит, физически и настраивайтесь. Значит, изменяйте все ваши колебания, чтоб они были синхронны с этой планетой. И тогда не будет понятия, о времени, – как долго вам перемещаться туда, ибо вы уже будете находиться там. }
\people{Ну, мы к этому не скоро придём? Человечество.}
\soul{Время - относительно.}
\people{Ну, как инопланетные корабли тогда перемещаются к нам? Таким же способом? }
\soul{По такому принципу. Есть и множество других способов. Искривление пространства – это всего лишь только один из шагов, и довольно мелких. Вы не можете искривить пространство так, чтобы попасть в любую точку. }
\people{Так у нас, всё-таки, бывают летательные корабли с других планет? }
\soul{У вас есть передвижные средства сейчас. Они не далеко ездят. То же самое будет, если вы создадите и начнёте управлять искривлённым пространством. Это тоже будет не далеко, в космических масштабах. И это вас не устроит. Далее, вы придете к волновой системе и будете перемещаться, куда вам угодно. Одна беда – вы не знаете колебаний той планеты, на которую вы должны попасть. Иными словами, вы не знаете координат, космических координат, а не относительно вашей Земли, куда вы должны попасть. Только в этом сложность. Определите координаты – то вы будете уже там. И физически, и духовно. }
\people{А, бывает, называют шифр Земли или другой планеты. Это, как раз, имеет этот смысл? Определение координат в пространстве.}
\soul{Кто даёт вам этот шифр? Землянин? Относительно себя?}
\people{Нет. Инопланетяне. Ну, или они, по крайней мере, так себя называли. И просили сообщить им шифр Земли. }
\soul{Давайте скажем так: в данном случае – он посторонний наблюдатель. Вы согласны? Как он может дать вам верные координаты относительно вас, если он не может находиться в ваших координатах? Вы скажете, к вам приходят и находятся. Во-первых, это не долго, и чаще всего, это случайность. Чаще всего, в вашем понятии - инопланетяне, приходят к нам, заблудившись. Или методом случайности или методом перебора. И потому, они не могут попасть в одну и ту же точку по желанию.}
\people{Скажите, на Земле есть места, где исчезают люди. Может быть, таким способом ``они'' забирают людей? Или люди перемещаются в другое пространство?}
\soul{Нет. Для этого не обязательно попасть в другое пространство. Достаточно попасть в другое время, которое будет отличаться от вашего  лишь на долю секунды,  и вы уже не будете его видеть. Не будете видеть его у вас. Есть временнЫе изменения. Есть пространственные. Чаще – временнЫе. Человек остаётся здесь, на этом же месте. Но он, относительно ваших координат сместился. Не на много, чтобы уйти с Земли, но достаточно, чтобы попасть на другую Землю. Только не думайте, что время связано с годами или веками. Достаточно нескольких микросекунд, чтоб исчезнуть с этого времени. А если быть точнее – достаточно мгновения,  чтобы потерять друг друга. }
\people{Скажите, возможно ли в пространстве, сверхсветовое перемещение?}
\soul{Конечно.  Неужели вы думаете, что вы только начали и уже нашли? }
\people{А мгновенное перемещение в пространстве, вообще, возможно?}
\soul{Мы же говорили вам: если вы будете знать координаты, вам не нужно будет понятие времени. }
\people{А если не брать другие планеты, а только рассматривать перемещение в пределах Земли? Это возможно?}
\soul{В первую очередь, это будет болезненно для вас, и  болезненно для Земли. Ибо вы будете сеять хаос, вы будете искажать поле, что было дано космосом. Вы будете вредить не только себе, но и всей планете. Потому  планета и не даёт вам всех этих возможностей. Потому и ставит вам препятствия. Потому и приходят  к вам и называют ложные шифры. Настоящих шифров – их просто нет, ибо не существует такого понятия ``шифр планеты'' или шифр чего-то другого. Вами придуманы координаты. Ибо придут другие, и у них совершенно другой аппарат. И совершенно другое понятие, о времени и пространстве. И потому, они не смогут пользоваться вашими координатами, как и вы.}
\people{Одна из московских групп, уже лет 30 назад вышла на разум Земли и умеет с ним контактировать, и получают интереснейшие сведения. Известно вам это?}
\soul{Давайте скажем так: если бы вы могли слушать Землю, вы бы не могли оставаться человеком.  Вы, знаете ли, что такое Земля? Знаете ли вы ``шёпот Земли''? В данном случае, вы не можете выдержать этого ``шёпота''. }
\people{Ну, они не звуковыми, конечно, а другими видами контактов разговаривают.}
\soul{Мы же не говорим вам, о речевом. Вы даже не слышите дыхание Земли. Вы, даже не слышите дыхание себе подобных и хотите разговаривать с Высшим! Другие вообще не признают Землю живой. Значит, неживые они. Другие обожествляют Землю и говорят, что она единственная и более ничего. Вот болезнь ваших многих, что только на Земле Бог создал людей, а другие? Это всего лишь ваша гордость.  Далее. Земля никогда и не переставала говорить с вами, вы – дети ее, и она вас не покидала. Вы покидаете её. Вы уходите от неё.  И она вам кричит, но вы не слышите, ибо вы заняты совсем другим. Вы хотите перемещаться. Зачем? Зачем? Вы можете представить себе мать, по которой вы хотели бы перемещаться? }
\people{Нет.}
\soul{Зачем? Зачем же вы тогда хотите перемещаться? Зачем? Только для того, чтобы ``убить время''? Чтобы чем-то себя занять? И всё?!}
\people{Но мы не видим альтернативы этому. Значит, мы должны, как квочка, сидеть на одном месте и думать, о чем-то? Ни творить ничего? Человек, видимо, создан в движении. Это, значит, его основные функции?}
\soul{Да. Только движение – не ногами, а подвергаться всему. Если вы хотите путешествовать – путешествуйте, но чтобы был смысл, духовный смысл, а не просто ``убивание времени''. Многие из вас умеют перемещаться, но никогда этим не пользуются. Только начинающие, только делающие первые шаги, прыгают туда и сюда – им становится интересно. И когда они уже понимают смысл движения, они понимают его бессмысленность. И уже не пытаются сделать этого. Удивительно, не правда ли – обладать силой и чем больше обладаешь силами, тем больше у тебя нежелание ими пользоваться? }
\people{А контакты с Землёй этих москвичей – вы скептически относитесь? Они накопили много информации.}
\soul{Нет. Мы не говорили вам, о лжи. Мы говорили вам только, о шёпоте Земли. Мы говорили вам, что Земля никогда не покидала вас – вы уходите от неё. Если кто-то слышит – слава ему. Но, как и мы не можем передать точно, так и они не передадут всё. Ибо, всегда существует ``переводчик''. Вот если бы каждый из вас мог слышать, и не нужен бы вам был ``переводчик'', даже там была бы ложь. Ибо, аппарат ваш, – мозг ваш, - уже переводчик. Вы же пользуетесь услугами сотни ``переводчиков'', пока до вас дойдёт. Это - испорченный телефон. }
\people{То есть, надо осторожней быть в выводах и полученных знаниях? }
\soul{Мы говорим вам не об учении, а о беседе. Принимайте это за беседу и никогда и никому не доверяйте. Всегда ищите и вдумывайтесь. Мы здесь с вами не для того, чтобы дать знания, а цель только одна: растормошить вас, чтобы, как вы говорите,'' не стояли на месте'', чтобы начали думать. Вы же, пока, только сомневаетесь во всём и вся. Вы ничего не принимаете на веру. То же самое говорим и мы: не принимаете. Но вы, для того, ничего и не делаете! Вы отвергаете всё или соглашаетесь, со всем. Соглашаетесь только тогда, когда совпадает с вашими понятиями. Вы не слышите противника, ибо он говорит не вашими словами и не поддерживает ваших идей. О каком тогда можно говорить ``движении''? О каком  можно говорить продвижении … (теряется)}
\people{Следующий вопрос. Изменяется ли при сверхсветовом перемещении знак времени? }
\soul{Нет. }
\people{Останавливается ли время при перемещении со скоростью света?}
\soul{Нет. Поймите, какой бы вы ни придумали совершенный аппарат в будущее или в прошлое, относительно вас – время всегда идёт вперёд. В какую бы эпоху ни попали, вы, именно ваше лично время, будет идти только вперёд. Вы не можете придумать машину времени, чтобы помолодели на 40 лет назад. Не можете этого сделать, но вы можете вернуться на 40 лет назад и посмотреть на себя. Но будете смотреть глазами 40-летнего. }
\people{Скажите, а остановилось ли время для фотонов?}
\soul{Нет. Вы повторяетесь. Вы говорите о скорости света. И тут же задаёте подобный вопрос, только изменяя… (теряется)}
\people{А скажите, а чем объясняются перерывы у переводчика, что ему мешает?}
\soul{Мы не можем вмешиваться, мы всего лишь помогаем. И не более. Да, есть многие из вас, кто является чистыми контактёрами, иными словами, – одержимые, когда извне приходят и владеют ими. Мы же, не делаем того. В любой момент переводчик может уйти или прийти к нам – это его желание. Его же желание подкрепляется вашими. }
\people{Могут ли фотоны существовать при скорости близкой, но не равной скорости света? }
\soul{И, да и нет. Давайте скажем так: вы считаете, что фотон является частицей света. Так?}
\people{Так.}
\soul{И тут же говорите о скорости света. Вы можете совместить или отделить ``фотон'' от ``света''? }
\people{Н-да, трудно. }
\soul{Далее. Фотон обладает свойством волновой частицы. Согласны?}
\people{Да.}
\soul{Вот, когда он проявляет свойства частицы, он не может, именно в вашей, на вашей планете и в вашей физике, превысить скорость света. Ибо ваши приборы не регистрируют того, ибо ваши приборы состоят из тех частиц, которые движутся со скоростью света и не более. Вы  согласны?}
\people{Да.}
\soul{Как вы тогда можете зарегистрировать бОльшую скорость? }
\people{Правильно.}
\soul{Потому, вы будете замерять только те фотоны, которые близки к этим скоростям, т.е. к изменению электрического тока. На этом же созданы ваши приборы. Те же, которые будут ниже или выше – они вы падут из поля действия этих приборов. }
\people{Ну, а нами, землянами, наблюдались такие фотоны, которые двигались бы со скоростями, намного отличающимися от световой? }
\soul{Да. }
\people{Но мы их не осознавали, да?}
\soul{Почему же? Вы знаете, даже их, но не признаёте их за частицы, вы не признаёте их за фотоны. Мы когда-то говорили вам, о реакции переводчика. Помните? О том, что мы не имеем физики, и всё остальное – это всего лишь реакция физики. Если вы скажете, что видели нас, вы, одновременно, солжёте или нет. Потому, что вы не можете нас увидеть - вы увидели только реакции. Но вы будете правы, что видели нас, ибо реакция материи это, в принципе, то же, что и мы. Если вы видите себя в зеркале – нельзя же сказать, что это не вы? Вы согласны?}
\people{Да. Скажите, заполнено ли пространство полностью? }
\soul{Да. Нет, в вашем понятии, вакуума. Нет пустоты. Вы никогда не найдёте её и нигде. И то, что вы называете пустотой – как раз самая населённая. }
\people{А чем, чем населена?}
\soul{Ищите. Всем. В этой частице пустоты, если можно сказать так, - есть всё. Туда спокойно можно вместить всю вашу планетную систему и останется ещё множество… (теряется)}
\people{То есть, пустота может быть заполнена и электромагнитными, гравитационными полями, какими-то иными вихрями, энергиями, иными частицами?}
\soul{Давайте скажем так: пустота – это  идеальное пространство. Значит, нет конечности. А раз так, то можно вместить туда очень много. Ибо, это неограниченное пространство. И ваш космос, ваша Вселенная, для кого-то - всего лишь только частица пустоты или вакуума – как хотите. Есть множество Вселенных, где есть своя жизнь, свои понятия, о пустотах, о вакуумах.  И вы – всего лишь маленькая доля. }
\people{А скажите, верна ли гипотеза, что пустоту (или вакуум в нашем понятии) можно использовать, как источник энергии? Или для создания энергии? Это верное предположение?}
\soul{Если даже в одной частице пустоты, в вашем понятии, есть вся Вселенная – почему бы нет?}
\people{Теперь серия вопросов об аппаратах для перемещения в пространстве и времени. Они важны для моего коллеги из Москвы.  Чем являются  наблюдаемые нами НЛО? Какую часть из них составляют техногенные аппараты, а какую неизвестные нам явления природы, или проявления неизвестной нам жизни?}
\soul{Давайте договоримся в терминах. Техногенная – значит, искусственная.}
\people{Искусственная, да.}
\soul{1-2% - не более. }
\people{А 98% это относится к явлениям природы?}
\soul{К явлениям природы, к явлениям лжи. Или желания. }
\people{Это, как раз, наверное, о том, что чего люди хотят, то они и видят. То есть, создают своим воображением такие объекты, а потом их видит кто-то?}
\soul{Тогда вспомните историю, о Жюль Верне. Вспомните. Тысячи людей видели, как марсиане завоёвывали города, но ведь этого не было. А тысячи видели! }
\people{Неужели все НЛО – это плод воображения? }
\people{Т.е. это и есть как раз и есть астральный мир, да? Иллюзии?}
\soul{Нет. Ни в коем случае. Это то, что вы хотите увидеть. Это, созданное вами. И потому, мы говорили вам о техногенном, и о термине. Множество - вы выдумываете сами, и это множество может и должно воплотиться. Когда вы один и мечтаете о чём-то - это трудно сделать. Когда у вас тяжелое время, вы призываете внешние силы, чтобы помогли вам, ибо сами боитесь сделать первые шаги. И поэтому множество людей ждут помощи от инопланетян. Пожалуйста, они появляются тут же! Ибо тысячами думают об одном - и вот вам итог.  Многократно умноженная энергия изменяет, в вашем понятии, пространство и создаёт вами желаемое. }
\people{Так, человек может что угодно натворить, и творит наверное, да?}
\soul{А вы подумайте. Вспомните Библию. Как созданы были вы? А вы ведь - подобие божие. }
\people{Всё ясно.}
\people{А как объяснить те периоды, когда прилетало очень много НЛО?}
\soul{Вы не внимательны. Мы вам только что говорили. Когда вы говорите, о иных или желаете множество - вы ленивы, вы боитесь всего. Вы боитесь сделать даже то, что было бы вам на пользу. И потому, вы начинаете мечтать. И заметьте, в вашем понятии НЛО приходит только тогда, когда для всего человечества тяжёлый период. Когда всё прекрасно и хорошо, и цветёт ``коммунизм'', НЛО не было. }
\people{Так может быть, тогда и не надо изучать НЛО?}
\soul{Разве? Давайте скажем так: если вы хотите изучать только истинные НЛО, то этих 2% вам вполне хватит.  Далее. Вы должны изучать все. Ибо вы, в первую очередь должны изучить себя, свою психологию. И если множество из них выдуманы вами, так это прекрасно.  И, значит, вы, всё-таки, обладаете силами и можете ими владеть. Но только если вы будете желать что-то другое – будет и это другое. Если вы скажете ``надоело, скорей бы конец света'', то он и наступит.}
\people{Давайте поговорим о 2% техногенных НЛО. Откуда и кто прилетает на таких НЛО?}
\soul{Мы, когда-то говорили вам, что они подобны вам. Многие – тоже люди, но только изменены. Вспомните прошлый контакт, мы говорили вам, об общих точках и расхождении по времени и делам. Помните?}
\people{Угу.}
\soul{Вот, представьте, когда-то давным-давно появилось расхождение и тысячи лет вы развивались отдельно. И один из ваших… (обрывается)  прилетя сюда, не знает, ибо он не изменился за эти тысячи лет. И будет говорить и называть вас ``инопланетянами'', и вы ему – говорить то же самое. Представьте теперь такую ситуацию: вы ведете магнитофонную запись, вы выступаете по телевидению, иными словами вы заменяете электрические поля Земли. Эти изменения никогда не исчезнут, ибо было изменено окружающее пространство. И вы помните, мы говорили вам о картинках?}
\people{Н-да.}
\soul{Вот вам – ``картинки'', это - первое. И второе,- каждый из вас видит ложно, относительно другого. Для себя вы правдивы, себе вы не лжёте, но относительно другого, вы можете видеть ложно.}
\people{Нет, но…}
\soul{Вот вам теория относительности. Мы не обвиняем вас во лжи, нет. Но поймите, каждый видит только одно, что-то своё. Грубо говоря, настроен только на какую-то одну из частот. Другой - настроен на другие частоты. Другой скажет, что он видел совершенно другое, и  вы лжёте, и вы оба будете правы, и оба будете ``нет'', потому, что каждый увидел только кусочек. Если бы вы могли видеть всё, вы бы уже не задавали вопросов, а спрашивали бы - у вас. И даже мы  бы пришли бы к вам и спрашивали бы у вас. Вы же видите только кусочки картины, потому вы совершаете множество ошибок, потому вы занимаетесь лечением и не столь успешно, потому вы называетесь экстрасенсами, потому, что вы видите только что-то одно. Даже в самом термине – экстрасенс, уже есть понятие, что вы видите только что-то одно, ограниченное. Если вы будете кого-то обвинять, что он видел не то, вы будете не правы. И даже умение не видеть, а это тоже, согласитесь, надо уметь, что бы не видеть ничего. Это тоже умение, и оно тоже заслужено.}
\people{Скажите, последний вопрос, откуда земляне получили цифры? Это подаренное?}
\soul{Опять вы мечтаете о подарках.}
\people{А это изобретение землян, разве? Нам кажется, что нам, всё-таки, другие цивилизации дали…}
\soul{Давайте скажем так, что вы, всё-таки, здесь не первые. Давайте скажем тогда так - осталось.}
\people{Ага. Осталось от других цивилизаций. Ну, всё, спасибо. }
\people{(Ольга) Действительно ли существовала Атлантида, Лемурия, так сказать?}
\soul{Вы же сами согласились, что раньше люди более правдивые были.}
\people{Да. ( Белимов -Ольге) Они говорили нам,  да, всё это правильно. Мы хотим определить темы для будущих разговоров. Мы хотели бы, что бы вы ответили ``да'' или ``нет''. Например, как вы смотрите на такую тему, в дальнейшем, в будущем, `` растительный мир земли, особенности жизни растений'' Можно с вами это обсудить?}
\soul{Вы  никогда  нас не спрашивали заранее о темах, спрашивайте.}
\people{Но вы однажды отказались отвечать на одну из тем, вот мы сейчас хотим разрабатывать одну из тем, например, про луну вы отказались. Но разрабатывать ли нам другие темы? Например, ``животный мир земли'', тоже ряд вопросов есть интересных.}
\soul{Давайте скажем по-другому, хотя это, в какой-то мере, может  и обидит вас, но мы будем отвечать на все ваши темы ``да''.}
\people{Хорошо.}
\soul{А потом будем отказываться. Почему? Потому, что разрабатывая, вы уже будете думать. И это хорошо.}
\people{А-аа. Ну, хорошо. А вот нас интересует ситуация с эльфами. Выяснилось, что переводчик, например, не знаком с романами Джона Толкина. Вам известно его творчество?  }
\soul{Переводчик тоже читал.}
\people{То есть, и вам известно. Так, есть предположение, что его книги – это  вовсе не выдумка, а реальный пласт знаний об ином мире квэнди, эльфов, который как-то проявляется на плане Земли. Так ли это?}
\soul{Помните, мы когда-то говорили вам, что человек не совершенен тем, что он не может придумать то, что не существует. Вспомните.}
\people{Да, значит, это существует. Так.}
\soul{Всё, что бы вы ни сказали, это существует, или одна из возможности реальности. Поймите, вы не можете выдумать то, чего нет.}
\people{Значит, если человек сказал, что он эльф, это так и есть?}
\soul{Это значит, что он или был им, или будет им. А сейчас, будучи в человеческом облике, он просто играет в него.}
\people{Или вспоминает прошлую свою жизнь, так ведь?}
\soul{В вашем понятии, вспоминает только о прошлом, а можно вспоминать и будущее.}
\people{А вообще, кто такие эльфы?}
\soul{Давайте не будем говорить о высших и низших мирах. Это один из миров.}
\people{А так - в человека может воплотиться  любое существо из всех миров, или нет?}
\soul{И да, и нет. Вы никогда во снах не были кем-то другим?}
\people{Были.}
\soul{Зачем же тогда спрашиваете?}
\people{Ну, получается, у них есть карта местности…?}
\soul{Ну, зачем говорить о картах местности, если мы вам говорим – да, существует. И не надо говорить о параллельности миров. Это не только это, это может быть сдвинуто во времени, уже это вы называете параллельностью, хотя не верны. }
\people{Ну, скажите…}
\soul{Далее, существует множество миров. И здесь, на этом месте есть множество миров, они пронизывают вас, вы пронизываете их, и не замечаете друг друга. Вы же не замечаете воздух, пока он есть у вас, вы продвигаетесь сквозь него, вы не замечаете микробов, хотя дышите ими, продвигаетесь среди них, вы не замечаете их. Вы не замечаете множество молекул, веществ материи, которые пронизывают вас, вы их не замечаете. Лишь заметите только тогда, когда они исчезнут, или когда их будет слишком много.}
\people{Скажите, допустим, любое животное, даже, допустим, моя собака, в общем - может ли она в следующей жизни быть человеком? }
\soul{И да, и нет. ``Да'', только в том случае, если она сможет ``достигнуть'', в вашем понятии, или наоборот  - ``упасть'', в вашем понятии, до человека. Всё относительно, ибо для неё сейчас совершенно другие меры, и нельзя говорить, что она выше человека, или ниже его. Ибо это совершенно другие меры. Вы же не можете обвинить стул, или какую-то мебель в том, что она глупа. Вы согласны? Далее, она не может стать человеком, ибо слишком далёкий путь. Скажем так, слишком далёк этот параллельный мир. Почему? Потому, что нужно слишком много изменений. Мозг, физический мозг, на данный момент, имеет химическую память, мы как то говорили вам  о множестве видах памяти, помните? Имеет физическую память, имеет ментальный, и многие, многие планы памяти. Все ваши семь полей, хотя гораздо их более, имеют свою память, и где-то из этих памятей может быть человеческая матрица, но она должна…(срыв)}
1,2…
\people{(Белимов)… Какая кукла?}
\soul{…странно..}
1,2..
\people{(Ольга) Давайте продолжим беседу о животных, о мирах параллельных. Вы помните, на чём мы закончили? (Белимов Ольге) Про животных я хотел бы отдельно.}
\soul{Спрашивайте.}
\people{Я так поняла, что у существ, которые имеют мозг, множество ячеек памяти, да?}
\soul{Нет - планов памяти.}
\people{Планов памяти, да.}
\soul{Вы же говорили, что у вас есть множество полей, множество уровней, и даже дали им математическое определение, и дали им количество – семь, хотя между ними нет чёткой границы, но пусть будет по-вашему ``семь''. Значит, есть и семь уровней памяти. Это можно даже доказать логически, вы можете сделать это сами.}
\people{А у животных - тоже?}
\soul{Да.}
\people{И у растений?}
\soul{Ваш мир един. В прошлый раз мы хотели выяснить, чем же отличаетесь вы от животных…}
\people{Да. }
\soul{…если физика у вас одна, потребности у вас одни и, даже, мечты у вас одни. Вы говорите, что вы отличаетесь тем, что у вас есть разум. Да, но у животного он есть тоже. У него есть тоже, множество тех же планов, и как он, так множество из вас не знает о них и не мечтает попасть на любой из них. Многие вы, как и они, просто существуют, для того чтобы прожить свой срок, и конечно желательно прожить его хорошо. Чем вы отличаетесь от животных и от себя? Чем?}
\people{Но, всё равно…}
\soul{Одно из отличий то, что вы уже не считаете себя животными, хотя физиологически вы те же, что и они.}
\people{А, допустим, интеллектом обладают животные?}
\soul{Да.}
\people{У них тоже интеллект есть?}
\soul{Ну, а как вы думаете?}
\people{Ну, тогда чем человек отличается от животного?}
\soul{Вы говорите ``божественная монада''. Вами сказано, что человеку дано было. Чем отличается? Давайте скажем, чем изначально отличался человек,- когда он только появился, Адам - от животных? Чем?}
\people{Чем он отличался?}
\soul{Правом дать имена. Вы должны знать это, вы помните, что Адам давал имена животным, не Бог?}
\people{А это не означает то, что человек создал всё?}
\soul{Давайте скажем далее, мы говорили вам о множестве богов. Вы должны помнить, что если существует какая-то сфера труда, и в ней есть такое понятие – энергия. Именно направленная энергия, именно на исполнение этого труда. Именно эту силу, именно эту энергию вы можете назвать богом. Есть энергия, которая создала вас, именно вас, имеется именно вашу душу, именно ваше ``Я'', которое является человеком, само зерно человека. Человек – это не ваша плоть, нет. Почему говорят, что собака как человек, или что-то другое? Помните? И так же говорят, ``как зверь'' о человеке. Вы помните?}
\people{Да, конечно.}
\soul{Почему? Потому, что может человек, в вашем понятии ``плоть'', быть человеческой, но ничего человеческого в нём не будет. Это уже не человек, это зверь, который случайно получил и использовал матрицу, физическую матрицу именно плоти человека, и та же собака стала человеком, именно плотью. Но, она же не стала самим человеком, вы согласны?}
\people{Да.}
\soul{Просто произошла ошибка. Вы думаете, что в мире всё закономерно ``от'' и ``до''? Нет. Тоже есть ошибки, но ошибки только в том смысле, что в них нельзя найти смысл пока на этом уровне. Может, где-то гораздо ``выше'', это не ошибка, а было закономерно, но мы этого  не знаем и сами.}
\people{А так же, допустим, эльфы тоже могут, да?}
\soul{Это могут быть все.}
\people{Все абсолютно, да?}
\soul{Это может быть любое, любое живое. Мы когда-то говорили вам, что живое всё, тот же стул, он тоже живой. Всё, что имеет движение, имеет энергию – это живое. Он вполне может стать человеком, ибо он не имеет матрицы, не имеет матрицы, строения, чертежей, или, как вы хотите, той матрицы, которая создаст физическую плоть человека. Он не имеет её имеет более близкий к человеку, более близкий к формам. Стул, в следующей жизни, если можно так сказать грубо, может стать, но он станет только одной  частью какой-то другой мебели. Поймите – ``подобное к подобному'' сохраняется и здесь.}
\people{(Ольга) Ясно, а…, о, господи…, мысль потерялась. А растения? В  некоторых источниках, в книгах пишется, что растения, по-другому  эволюционируют, чем человек и животное на земле. То есть, немножко другой путь у них, у растений?}
\soul{Давайте скажем так, что и среди людей есть разные пути эволюционирования. Одни - уходят на другие планеты, другие - остаются здесь, третьи - опускаются вниз, четвёртые - идут в рай, пятые - идут в ад - и так далее, далее. Множество. И это совершенно разные виды эволюции, и это нормально. Тем более между классами должно быть различия больше, но цель одна. Цель одна, но этой цели не знает никто, её все ищут. И чем отличаются растения от животных? Только тем, что в вас больше жаждет найти эту цель.}
\people{Скажите, в Индии йоги, но не будем говорить о таких йогах, которые Хатхой занимаются, или что-то в этом роде, а действительно, будем говорить о йогах, которые обладают определёнными силами, владеют ими, допустим, и имеют какую-то свою теорию, о эволюции человека. Они никогда не заканчивали институтов, не знают о ядерном распаде, или о каком-то синтезе и так далее, но они знают больше, чем все наши учёные, допустим - так. Вот они ближе, может быть, к естественному процессу?}
\soul{Ну, давайте скажем так, пришёл Иисус Христос, он преподавал… }
(сбой контакта)
\soul{И когда-то, через тысячи лет, вы же, сидя на какой то планете и поймав эти изменения и сумев расшифровать их, но не поняв, что это инопланетяне, а скажете так, что это какое-то природное явление не понятое вами. Оно приведёт вас к беспокойству, к смутным воспоминаниям. Но это быстро пройдёт, ибо вы не найдёте причин, ибо вы не сможете вспомнить то, что было тысячу лет назад, и не можете соединить, что эти колебания были сделаны вами, вашим изображением, или даже вашим голосом, вашим движением. Лишь только тревога, что что-то знакомое,- согласитесь, у вас часто бывает это, когда вы видите что-то и не можете вспомнить, хотя вы знаете, что когда-то видели, хотя и знаете, что этого вы не могли видеть.}
\people{Но мы так никогда и не осознаем это, такое объяснение?}
\soul{Пока не вырастете, в вашем понятии ``духовно'' -  нет. Этого нельзя делать. Это будет для вас всего лишь одна из пыток. Ну, представьте, вы сейчас вспомните все свои прошлые жизни. Что будет? Ничего, просто вы бросите писать, и вам нужно будет совсем другое место.}
\people{Нет, но мы хотели бы поговорить, когда-нибудь с вами о прошлых жизнях более подробно.}
\soul{Это можно сделать, ибо вы не понимаете это. Да, вы допускаете, что это есть, но доля сомнения сидит в вас и спасает вас. }
\people{Угу.}
\soul{Если вы же всё увидите это и ещё переживёте, вы можете не выдержать. И потому, если придут к вам и предложат вам ``вспомнить будущее'', или ``прошлое'', и не просто вспомнить, а заново пережить, бойтесь того, ибо пришли погубить вас.}
\people{Угу. Хорошо, мы это учтём. Продолжим о техногенных НЛО. На каких физических принципах работают двигатели таких НЛО?}
\soul{Вы говорили об ``искривлении пространства'',- пожалуйста. Есть в вашем понятии - ртутный двигатель. Мы когда-то говорили вам о ртути и о более тяжёлых материалах. Вы помните?}
\people{Да, припоминаю, - виманы и прочее.}
\soul{Когда-то вы делали опыты. Чем более раскручиваете, заметьте, не просто изменяете скорость, а именно делаете… (срыв)}
1,2…
\people{Вы не можете эту мысль продолжить? … }
\soul{Ибо только при вращательных движениях вы можете достичь подъёма, исключения гравитационных сил, это один из способов, когда вы стараетесь избавиться от гравитационных сил, применяя более другие силы. Пожалуйста, вы же можете сейчас электромагнитными полями поднять что-то. Согласны?}
\people{Да. }
\soul{Это значит, это поле было сильнее гравитационного. И любым полем вы можете сделать это, любым. Вам нужно, всего лишь только, превысить силу гравитационного поля. Можно сделать и наоборот. Вечно падать, используя гравитацию, вечно падать, но управлять этим падением так, чтобы, всё-таки, не достичь поверхности. И математики уже создают это, но только на бумаге, только на пере.}
\people{Хорошо.  Движетели НЛО могут находиться  внутри аппарата, или вне аппарата чаще всего? Внутри оболочки, может быть?}
\soul{Давайте скажем так, если вы решили падать, то вам не нужен аппарат. Вы согласны? А может находиться внутри, если вы будете использовать какие-то другие поля, только разрозненные поля. Если же вы когда то придёте к теории единого поля, вам вообще не нужно будет понятия о передвижении, ибо уже вы  можете находиться где угодно и когда угодно. Пока же ваши поля разъединены, пока вы пытаетесь сделать электрическим, или другим путём. Давайте скажем так. Вы придумали летательные аппараты легче воздуха, то есть использовали архимедову силу, а не можете  вы найти те силы, которые могли бы использовать гравитационные поля, или электромагнитные поля, и тот же принцип архимедовых сил. Теперь, давайте сделаем так, - давайте, вы будете падать на ближайшую массу от земли, но управляя этим падением, и вы тоже будете над поверхностью этой земли. Вы согласны?}
\people{Трудно понять, но может быть так.}
\soul{Ну, давайте скажем так, вы решили падать на Солнце. Но если вы сумеете уравновесить и силы гравитационные, и солнца, то вы будете, в вашем понятии, плавать от поверхности земли. Когда-то у вас была теория эфира, вы отказались от неё. А почему бы вам не поплавать и в нем?}
\people{Угу.}
\soul{Вы выдумываете очень множество таких аппаратов. От чего? От того, что вы думаете - инопланетяне, это значит что-то ``ого-го'', по аналогии зарубежного. Всё гораздо проще. Спрашивайте далее.}
\people{Скажите, двигатели НЛО способны изменять ``пространство-время'' вокруг себя?}
\soul{Изменяете его и вы.}
\people{Даже без двигателя?}
\soul{Даже тепловой двигатель ваш уже изменяет пространство и искажает время, в малых долях, но это происходит, ибо вы изменяете поля. В данном случае - тепловые, а значит и электромагнитные. Вы согласны?}
\people{Да.}
\soul{Меняется поток воздуха, меняется направление света, меняется множество, множество, меняются все поля. Почему и говорят вам о теории единого поля. Но приборы ваши не чувствительны, поэтому не могут заметить того. Давайте скажем так, ``время'' - это одно из полей.}
\people{Мы фиксируем, иногда, в точках посадки НЛО изменение времени,- значит, мы правы - время меняется?}
\soul{Да. Время всегда и никогда не было стабильным. Вы всегда изменяете его, но изменяете в малых долях, чтобы не потерять друг друга. Помните, мы говорили, вам достаточно мгновенья, чтоб потерять, и тут же говорим, что время не постоянно? Вы изменяете, но мгновенье столь большое…(срыв)}
\people{1,2..}
\people{Скажите, изменяет ли двигатели НЛО вокруг себя время, для того, чтобы перемещаться в пространстве?}
\soul{Да. Это тоже один из способов, но им пользуются гораздо реже, чем думаете вы. Вы представьте,- вы изменили время, но вы останетесь в той же точке, вы согласны? Вы можете просто прийти в будущее, или прошлое. И если вы будете рассуждать, что изменив время, мы подождём, когда Земля повернётся, - это глупо.}
\people{Угу. Скажите, разные виды НЛО пользуются различными принципиальными типами двигателей?}
\soul{Природа одна и физика одна. Вы можете сказать, что используете разные способы теплового двигателя, ракетного двигателя, или какого либо другого, электрический, реактивный и так далее? Вы можете назвать существенный? ( признак отличия. Прим.)}
\people{Нет, это близкие, действительно. Угу. Скажите, является ли свечение вокруг НЛО обычным светом,- например, светом прожекторов,-  или это является следствием работы их двигателей? Часто замечают такие свечения.}
\soul{Давайте скажем так, прожектора - выдуманное вами.}
\people{Угу. Но значит это проявление работы какой-то, то ли двигателей, то ли полей вокруг аппаратов, так?}
\soul{Это может быть изменением полей, это может просто датчики, замеряющие, или снимающие какие либо показания, в конце концов, это может быть насос, который высасывает из вас, или наоборот, наполняет чужеродной энергией.}
\people{А это могут быть плазменные образования, как следствие работы двигателя плазменного типа?}
\soul{Нет.}
\people{Не то, да? А вследствие чего это? Вы уже сказали, да? И насос, допустим, и…}
\soul{Давайте скажем так, плазменный двигатель, это тоже, что и реактивный. Далеко ли вы уйдёте на нём, и далеко ли вы ушли? Вами, даже, были сделаны модели фотонных двигателей, но вы не смогли их использовать и вряд ли их используете. Вы представьте, двигатель, в вашем понятии -  плазма. Грубо говоря, вы хотите посмотреть рыбок, шумя и гремя. Как вы думаете, будут они ждать вас?}
\people{Наверное, нет. Хорошо, энергетические следы от посадок НЛО, являются следствием работы двигателей, у них же остаются следы на местах…?}
\soul{Не только. В вашем понятии ``экстрасенсы'' могут чувствовать, что здесь был когда-то человек, но в человеке нет ``двигателя''.}
\people{Угу. Да.}
\soul{Давайте скажем так: и то и другое. Ибо человек,- в вашем понятии ``экстрасенс'',- может сказать какая здесь лежала вещь, какой здесь был человек и, уж тем более, он заметит изменения физических полей, если здесь работал какой-то двигатель.}
\people{Угу. А скажите, может ли являться следствие воздействия на почву поля изменённого времени, связанного с прилётом НЛО?}
\soul{Да. Поймите, всё это чужеродно. И эти два процента чужеродны, и поэтому, - они приводят к изменениям, и чаще, плохим. А остальное - придумано вами, и земля помогла вам, ибо вы желали того.  Мать исполняет прихоти ребёнка, и потому, эти выдуманные вами не причиняют никакого вреда, если, конечно вы не сочините, что они вредные.}
\people{Скажите, в районе Жирновска наблюдают нередко трёх-звёздное НЛО, явно техногенного типа, - мы не ошибаемся, это не наши иллюзии, это настоящие техногенные аппараты?}
\soul{Вы думаете, ответим вам?}
\people{Это нельзя отвечать?}
\soul{Давайте мы будем делать за вас всё. Будем за вас жить, будем за вас учиться ходить, или будем постоянно водить вас под ручку. И что, - вы научитесь?}
\people{Ясно.}
\soul{Да, поймите, что многое ложно, или истинно не влияет на смысл поиска. Даже если это было ложно, но вы-то искали настоящее, истинное, ну, не нашли вы его, зато найдёте многое другое. Поймите, нельзя разделять так. Главное, вы  должны искать, искать, учиться, не стоять на месте. Если мы будем говорить вам – ``ложное'', ``да'', или ``нет'', представьте, скольких людей мы можем обвинить во лжи, даже если они не виновны? Поймите, если вы увидели, в вашем понятии, из этих восьмидесяти процентов, нельзя назвать это ложью, для вас она была истина. Вы согласны?}
\people{Да. Хорошо, вот нам нужно соблюдать технику безопасности при обследовании мест посадки. Скажите, там есть воздействие радиации?  }
\soul{Всё зависит от конструкции.}
\people{То есть, может быть. Да?}
\soul{Вы боитесь только радиации?}
\people{Ну, нет, а воздействие плазмы, допустим.}
\soul{Поймите, вы даже у себя в квартире имеете множество точек, где нельзя находиться вам, вы их не боитесь, а тут - боитесь.}
\people{Ну, просто мы…}
\soul{Почему? Потому что вы настраиваетесь, ибо страх ваш входит в резонанс и всё. Это не значит, что если вы не будете бояться, вам не будет вреда. Нет. Но если вы будете бояться, то это не приведёт ни к чему хорошему, это только усилит и не больше. Дальше, вы всегда и везде должны быть осторожны, вы должны быть осторожны даже думая, даже размышляя. Вы же, чаще всего, говорите, и лишь только потом думаете. Вы сперва делаете, а потом, смотрите - ``что же я наделал''.}
\people{Угу. Ну, на что обратить внимание при обследовании мест посадки, какую защиту всё-таки вы предлагаете, если мы не знаем какие виды энергии там воздействуют? Возможно, они опасны и губительны для людей? По крайней мере - микроорганизмы не выживают там.}
\soul{А вот, смотрите, смотрите, как реагирует природа. Если была уничтожена земля,- в вашем понятии были уничтожены микроорганизмы,- зачем же лезете тогда вы? Если вы состоите из тех же микроорганизмов. Вы согласны?}
\people{Ну, да.}
\soul{Какое тогда же может быть соблюдение безопасности? Хорошо, давайте мы вам дадим чертежи скафандров, или ещё что-нибудь  подобное, - и что, что вы в них нанаблюдаете? И спасут ли они вас? Смотрите, как реагирует земля, как реагирует растительный, животный мир.}
\people{Угу. Понятно, спасибо.}
\soul{Если при лесном пожаре бегут звери, разве вы будете изучать этот пожар?}
\people{Н-да.}
1,2,3
\soul{Спрашивайте.}
\people{Хорошо, продолжим. Скажите, сможет ли наша цивилизация, т.е. человечество построить летательный аппарат, подобный НЛО в ближайшие десятилетия?}
\soul{Спрашивайте далее, мы отвечали вам уже на подобный вопрос и отвечали уже сегодня.}
\people{Ага. Сумеет ли облететь планета…( обрыв)}
\people{(Белимов)…да, да.}
\soul{Потому, что контактёр не находится под постоянным гипнозом. Вы должны помнить это. Мы не можем этого делать, мы не имеем на то права. И потому, мы не можем держать это под контролем, и мы, тоже не имеем этого права.  Далее. Когда переводчик встаёт и говорит, что ничего не помнит, он просто, может быть, не хочет помнить. Или - это одна из реакций, одна из видов защиты. Или это - просто ``театр'', чтоб показаться более глупым, чтоб наши ответы показались более умными. Вы любите играть в театр. Вы всю жизнь проводите в нём. Вы лжёте себе, лжёте нам, лжёте всем. И здесь нельзя винить вас. Ибо вы играете, но не знаете. Ибо вы бываете марионетками в чьих-то руках. Да, вы виновны в том, что вы - марионетки. Но, нельзя винить вас в том, что вы являетесь ими, ибо вы - ещё дети.  Далее, вы говорили ``когда же можно будет разговаривать, в вашем понятии, с нами, не входя в это состояние?''. Поймите, это состояние, всего лишь только для того, чтобы отключить другие каналы связи, в вашем понятии. Если быть точнее, не отключить (ибо на это мы тоже не имеем права), а всего лишь затормозить или (теряется).}
\people{Вы сказали ``затормозить каналы'' вы пытаетесь. Мы сейчас что-то нарушили? }
\soul{Давайте скажем так: реакции. Мы не имеем права вмешиваться, ни в вашу, ни в его. Мы всего лишь приходим и беседуем с вами. Мы не говорили вам, об нарушении. Нет. Но ``переводчик'', мы уже говорили вам, что непомнящее его – это лишь всего лишь его выборка. Поймите, мы когда-то говорили вам, что мозг ваш – преграда для нас. Вы помните? }
\people{Помним.}
\soul{Ибо мы не можем прийти и вмешаться в него, вмешаться в его работу. Ибо это будет такая бурная реакция защиты, что не только ``переводчик'', но и рядом сидящие не выдержат того. И потому, мы не можем держать контроль. Если смеетесь вы и становится смешно ему, - он будет смеяться. Если вы будете переживать, будет переживать и он. Поймите, мы говорили вам, что он более чувствителен, в вашем понятии ``экстрасенсорики''. Помните? }
\people{Да.}
\soul{И тут же говорим, что мы стараемся блокировать реакцию.  Но, пожалуйста, не смешивайте слух с внутренним видением. Это совершенно разные вещи. Да, он может видеть вас. Он может слышать вас, не только теми органами, которыми вы привыкли делать это, но и многими другими.  И каждый из вас делает это всегда и везде. Пример? Вы встречаете человека, и он вам уже нравится или нет.  И ищете всех бед причину: - почему? Чтобы объяснить себе. Вы просто видите не только этими органами, но и другими. Но мы сейчас должны уменьшить реакцию этих органов. Вы поняли?}
\people{То есть, реакции из-за магнитофона могли помешать или повредить?…}
\soul{Мы не говорили о ``помешать и повредить''. Мы просто объясняем вам или пытаемся объяснить, что происходит. Далее,- это даже, когда-то  нас и успокаивает. И мы не хотим, чтобы каждый из вас был роботом. Если вы совершаете ошибки, значит вы ещё человек. }
\people{А кто не совершает ошибок?}
\soul{Есть ли такие? И вы хотели бы вы иметь таких? И хотели бы сами стать такими? Не совершая ошибок – это действовать только строго по расписанию. Или не делать ничего. Оттого вы и ``человек'', что вы ``идёте''. Чаще, идёте, не зная куда.  Ошибаетесь, но оттого вы и человек, что можете делать из ошибок выводы, а не просто вырабатывать очередной инстинкт. Оттого вы человек, что вы можете и решаете всё сами. И даже к вам, когда приходи в вашем понятии ``бог'', вы имеете право выбора. Оттого вы и ``человек''. }
\people{Так, всё-таки, что такое духовный путь? Хотя бы в общих чертах. Потому что у нас по этому поводу споры, как раз таки, и происходят. Духовенство, которое присвоило себе…}
\soul{Давайте скажем так: опишите мне картину, висящую на стене в общих чертах.}
\people{Сейчас?}
\soul{Да, сейчас. В общих чертах, чтоб было понятно. И чтобы передать смысл этой картины хотя бы в общих чертах.}
\people{На картине нарисован…}
\soul{Вы сможете это сделать?}
\people{Это сложно, конечно… Всё не учтёшь, особенно передачу…}
\soul{Поймите, если вам будут говорить что такое ``духовный путь'' - это вам или  лгут или вводят в заблуждение. Пока вы не найдёте эту дорогу, вы не поймёте ничего. Можно ли слепому объяснить, что такое Солнце? Можно ли глухому объяснить, что такое музыка? Можете?  Нет. И мы не берёмся объяснять вам это. Тем более, мы сами – тоже ищем.}
\people{Я продолжу. Скажите, может ли человечество совершать полёты во времени? Пользуясь машиной времени, или что-то в этом роде?}
\soul{Вы уже спрашивали.}
\people{Вы уже спрашивали.}
\people{Хорошо. А совершить полёты между параллельными мирами?}
\soul{Что в вашем понятии ``параллельные миры''?  Если вы только за один контакт уже могли  поменять их множество, что и сделали.}
\people{А существуют ли какие-то запреты на этот счёт? (путешествия во времени, параллельных миров?)}
\soul{Пока, у вас запретов больше, чем возможностей.}
\people{А эти запреты созданы человечеством, нашим незнанием или же идут от творца, от Бога, независимо от нас?}
\soul{Ну, давайте скажем так: мать бережёт сына. Отец учит, даёт ему знание и даёт ему силу. И если ребёнок хочет сделать что-то, которое приведёт его к боли или смерти, ни отец, ни мать не позволят вам  того. Вы же – всего лишь ребёнок. Дальше. Вы говорите о Творце, говорите о Боге? Как вы думаете – Бог один или, всё-таки, надо вернуться к язычеству?}
\people{Думаю, что, всё-таки, один.}
\soul{Хорошо. Бог один. Как вы думаете, существует ли что-то или кто-то который отвечает, допустим, за электронику? Или, скажем так, Бог виноделия? Как вы думаете – были ли они, существуют ли они?}
\people{Вряд ли. }
\people{Не знаю.}
\soul{Разве? Если какое-то общество людей занимается только одним делом, вы согласитесь, что есть уже понятие, о поле, о направленном поле. Вы согласны? Вот вам и - Бог виноделия. Или Бог кораблестроения. Или ещё что-то, ещё, ещё… Вот вам и язычество. Вы согласны?}
\people{Ну, в какой-то мере, но ведь это…}
\soul{Ну, давайте скажем так, что  вначале, люди были менее лживы, чем сейчас. Вы согласны? Почему же мы тогда не говорим, о язычестве, о множестве богов? Да, их множество. Ибо вы, создали их столь мощной аурой, что они действительно существуют. Но есть Бог, который дал начало всему этому. И именно духовный Бог, в вашем понятии. Именно духовный. Всё остальное – физика. И этот духовный Бог и сделал вас человеком. Именно его… Вы должны стремиться идти к нему, чтобы быть развитыми духовно. Для того, что бы придти к богу Бахусу, достаточно быть винной бочкой. Вы согласны?}
\people{Угу.}
\people{Скажите, для чего, всё-таки, Бог создал человека?}
\soul{А для чего вы создаёте себя? Только ли инстинкт, что вы создаёте себе подобных? Что вы выращиваете их, жалеете их, мучаетесь вместе с ними?  Для чего? Вы можете это объяснить?}
\people{Наверное, жажда…}
\soul{Наверное? Я думаю, если что если мы придём к богу и спросим, для чего же он сделал человека, он тоже будет отвечать наверное, примерно так же: ``да, наверное''.}
 
\people{Наверно, он будет более категоричен, в этом смысле.}
\soul{Нет. Он более человечнее вас. И если вы не можете фактами или математикой объяснить, что такое любовь и что такое ``себе подобных'', и пока не можете это объяснить, вы – человек. Именно тогда, вы духовны. А когда вы найдёте формулу смеха, формулу любви – вы будете ``машинами'' и только лишь.  Бог же - более человечнее вас. Это вы приписали ему все качества и все… (теряется).}
\people{Но, как же мы приписали ему все качества… Но нам непонятно, как же Бог допускает гибель невинных детей?}
\soul{Как допускаете вы? Что делаете вы? Вы сидите, смотрите телевизор, читаете газеты: ``Ах, какие нехорошие! Смотрите, что творится!''. А что вы делаете для того, чтобы этого не творилось? Что? ``Один в поле не воин'' - ваша фраза.}
\people{Наверное, эта песня - ``если бы парни всей Земли…'', да? Хором однажды, что-то там смогли. Вот это было бы здорово. Наверное, это всё-таки…}
\people{Скажите, другие цивилизации доброжелательно ли относятся к стремлению Землян овладеть новыми для себя знаниями, о пространстве и времени?}
\soul{Когда-то мы вам уже отвечали на подобный вопрос. Помните?}
\people{Да уже забыли, наверное.}
\soul{Вспомните об аборигенах. Вспомните, как вы приходите изучать новые земли, и что делаете вы. Вспомните, как вы изучаете новых животных. Вы сперва убиваете, а потом изучаете. Вспомните, как вы изучаете сейчас гуманным способом, придуманным вами. Вы опускаетесь в сферу жизни, допустим, тех же рыб, и вокруг них крутитесь. Именно крутитесь, потому что, вы не можете их изучить. Потому, что вы уже изменили их нормальную жизнь. Далее. Они такие же, как и вы. Есть путешественники, которые приходят и приносят добро, и приносят истинные знания. А есть и те, кто просто приходит поохотиться и поразвлечься.}
\people{А вот, для других цивилизаций, наши работы, землян, могут оказаться преждевременными и напрасными? Они сдерживают эти работы? Или наоборот?}
\soul{Зачем? Вы сами прекрасно справляетесь с этим. Вы сами прекрасно губите идеи.}
\people{А что следует сделать, чтобы другие цивилизации перестали опасаться неразумного использования нами новых для нас знаний?}
\soul{Ответ простой,- стать разумными. И тогда, вы, будучи разумными, не будете использовать их неразумно.}
\people{А вот, что нам нужно сделать для ускорения работ по созданию релаподобной аэрокосмической техники?}
\people{А для нахождения общего языка с вами, что стоит предпринять?}
\soul{А на каком языке говорите вы?}
 (Между собой): -  Он имел в виду информацию, наверное. (о коллеге из Москвы).
\soul{Поймите, мы говорили вам, что мы не имеем вашей ``физики''. Для того, чтобы разобраться в ней, мы должны глубже, в вашем понятии, войти в переводчика. Зачем нам это делать? У переводчика достаточно знаний, чтобы открыть самому. Но мы не можем воспользоваться его знаниями, ибо мы тогда вмешаемся в него. Далее. Нужно ли вам это? Нужно ли? Неужели вы думаете, что мы позволим себя использовать в качестве справочника, для того, чтоб, потом, вы натворили кучу дел?}
\people{(задумчиво) Ведь все изобретения используются военными …}
\people{Скажите, вот сейчас существует проблема: урон или наоборот, серьезные политические дивиденды получит то правительство, которое первым заявит о контактах или хотя бы о том, что на их территории существуют базы представителей иных ВЦ? Какой видится вам эта проблема?}
\soul{Первое, что вы узнаете – это страх. Ибо вы сразу же будете думать, что ваша держава теперь ослаблена.  Вы сразу же переведёте на военные цели: ``Ага! У них инопланетяне, значит, у них новое оружие''.}
\people{То есть, правительству опасно заявлять, о таких контактах?}
\soul{Первое, что будет – это страх.}
\people{У других стран? Если мы заявим, что у нас контакты…}
\soul{Да.}
\people{Понятно.}
\soul{Далее. Вы из этой вещи постараетесь высосать всё для военных целей. А лишь только потом объявят, что ``да, мы их увидели''. Далее, есть вариант, когда вы объявите сразу. Но только с тем условием, если вы сможете доказать. И что это приведёт? Опять приведёт к страху. Ибо вы привыкли,  придя в чужую страну, завоёвываете её, свои порядки и далее, далее. Вы привыкли, что, придя к вам, чужие тоже заведут свои порядки.  И вы также перенесёте это на других, на инопланетян. И вы будете уже бояться вторжения их. Согласитесь, не зря же тысячи людей видели ``нападение марсиан''? Значит, это уже было накоплено, значит, уже было подготовлено. Вы согласны? Тем более - сейчас, когда каждый ребёнок знает, о существовании НЛО. И любой из вас поверит, тем более, если об этом будет говорить вся периодика. Поверят, и первое, что ощутят – страх. Или радость, что, наконец-то, наша партия победила!}
\people{Мы ожидаем, что сейчас, всё-таки, радость будет больше. Тогда следующий вопрос. А вот если правительство, объявившее о контактах с представителями другой цивилизации, оно же может быть тут же обвинены в подчинённости этой внеземной цивилизации, в порабощении.}
\soul{Мы вам только что об этом говорили.}
\people{То есть, это страх вызовет такую реакцию, да?}
\soul{Поймите, самое первое, чем вы начнёте заниматься – это политическими играми. И лишь только потом уже будете спрашивать о духовном,  и меняться культурой. В первую очередь вы будете стараться узнать; - `` какого вида оружие, каким могуществом обладаете вы, какая развитая у вас техника  и науки?''  И, лишь только потом, вы будете спрашивать: - `` А не могли бы вы нам показать картины или, что-то сыграть?”}
\people{А вот, какой ваш совет по правилам поведения в данной ситуации? Объявлять или нет о контактах? Как правильно ввести? Ведь это очень трудный политический вопрос.}
\soul{Ну, давайте скажем так: вы когда-то спрашивали о наших контактах. Вы помните ответ?}
\people{Уже, может быть, забыли.}
\soul{Мы тогда говорили: для многих ли мы будем авторитетом?}
\people{Да, это было.}
\soul{И много ли будет иметь авторитета та страна, которая объявит о том, что она ``контактирует''?}
\people{Ну, ведь когда-то же, всё равно придется об этом говорить. Ну, не сейчас, не в этом тысячелетии, значит, в следующем.}
\soul{Да, конечно. }
\people{Но, сейчас идёт подготовка. Представьте, произошло бы это 30 лет назад. Прилетели бы инопланетяне. А вы о них знаете только из фантастических романов. И сейчас, когда их уже множество. И уже многие сейчас, уже говорят о грозящей инопланетной войне. О том, что НЛО прилетают, крадут и далее, далее, далее. Ваши же. Вспомните, что сейчас вы уже объявили войну. Зачем же тогда прилетать и убивать себя? Их устраивает  и  то, что их не признают.}
\people{А нам, как правильно вести себя? Приучать людей к НЛО, вырабатывать миролюбие по отношению к НЛО? Так?}
\soul{Нет. Вы приучайте себя, в первую очередь, как стать человеком. (теряется)}
\people{И всё-таки, мне не ясна роль уфологов, лично моя роль. Так что, нам приучать людей постепенно к уфологической ситуации? Запугивать? Или, некоторые уфологи бросают, уходят вообще из уфологии. Может это правильный путь? Надо уйти? Бросить?}
\soul{Вы выбираете цели и идёте к ним. Есть множество причин, когда вы оставляете их. Одна из причин – когда вам уже кажется, что вы уже знаете всё. Вы останавливаетесь. Хотя, по-настоящему, это говорит как раз о том, что вы не знаете ничего. Вы уже останавливаетесь: ``зачем? Я это уже знаю'', но не применяете того. Чаще всего, вы так и делаете. И никогда вы не видите конца. И вам кажется, что это бесполезно, ибо, это столь бесконечный путь!  И бросаете тоже. Бросаете и не даёте уже идти другим. Есть и другие причины. Их множество. И все они – человеческие. И множество из них – лень или страх.  Чаще же, вы выбираете цели из-за жажды. Жажды знать. Из-за страха жажды. А это плохие путеводители. Да, надо стремиться знать более, но не забывайте о качестве. А у вас только жадность. Да, надо бояться, чтобы не быть последним. Надо бояться, а то ведь можно не успеть. Но не должно быть страха, иначе вы повернёте назад или проскочите эту цель, не заметив её. Далее. Выбрали одну дорогу – так идите по ней. И не надо  блуждать.}
\people{Но если она не верная?}
\soul{Даже если она неверна, другие будут знать, что эта дорога была ложной. А так – на кого вы похожи, если вы сегодня одно, а завтра – другое? Что выйдет из того? Ничего. Вы ни там, ни там не будете знать. И бойтесь, самое страшное, бойтесь, когда вы нашли ложные цели. Когда вы уже решили, что всё знаете и дошли. Ибо это самое страшное. Лучше уж бойтесь. Лучше идите. Проскочите, - вы вернётесь потом рано или поздно. Но никогда не говорите: ``Я уже пришёл''.}
\people{Скажите, в одном из контактов нам подсказали, и, наверное, правильно, что смысл жизни человека в познании. Вы согласны с этим?}
\soul{И – да, и - нет. О каком познании говорите вы?}
\people{Вообще, эволюционном.}
\soul{Эволюционном? Простите, ваши машины тоже эволюционируют. Сперва были телеги, теперь - корабли. Это тоже эволюция.}
 
\people{Ну, всеобщие познания.}
\people{Познания, как будто мы дети. С этой точки зрения.}
\soul{Давайте скажем так: ребёнок познаёт в первую очередь мир, который находится вокруг него, который создан вами. Вы согласны? Через эти вещи он познаёт вас. Но ведь помимо этого ребёнок познаёт ещё что-то, что-то то, неуловимое, которое вы не можете поймать сейчас. Почему вы говорите, что надо быть ребёнком? Вспомните Библию.}
\people{”Будьте как дети, но не умом''.}
\soul{А теперь, давайте далее. Вы можете стать энциклопедией, знать всё и вся. А толку? Если вы не можете придумать что-то иное или открыть что-то новое? А, вот, многие из вас, считают, что уже эволюционные… (теряется)  … стать библиотекой.}
\people{Нет, не в этом смысле ``познания''. Не в сборе информации, как для отгадывания кроссвордов. А именно в ``хорошем'' смысле этого слова.}
\soul{В хорошем смысле слова?}
\people{Познания жизни, окружающего мира.}
\soul{Что вы называете ``жизнью''? Что за смысл жизни - ``познание жизни''? Как это можно объяснить?}
\people{Ну, человек живёт, он задумывается, зачем он живёт и почему.}
\soul{Вот это уже и есть смысл. В том, чтоб задуматься, в чём же смысл. А если вы уже не ищете смысла жизни, то вы уже и не живёте. И пока вы его ищете – в этом и есть смысл. А всё остальное – дело наживное. Все остальные знания, и далее, и далее - не являются критерием.}
\people{Ну, знания, это, как отправная точка.}
\soul{Хорошо, тогда почему же многие из вас, называемые больными, всё-таки попадают в рай? Ведь они не могут знать. Они не обладают вашими знаниями, вы согласны? Но они, всё же, в раю.}
\people{Дело не в том, какой человек: сумасшедший или нет, там, попадает куда-то. Дело не в этом.}
\soul{Потому и говорят вам: ``дело не в знаниях''.  Смысл жизни вашей – в том, чтобы искать этот смысл. И когда вы найдёте его – вы уже ``машина''. Ибо вы уже придумали формулы жизни, смеха, чувств, любви. И пока вы не придумали того, вы – человек. И пока вы ищете смысл жизни, вы – человек. И смысл жизни - в искании этого смысла. И не больше. Всё остальное, говорят вам… (теряется)}
\people{Скажите, числовой математический аппарат. Он одинаков в разных мирах и цивилизациях?}
\soul{Нет. Ну как он может быть одинаков, если другие пространства, другое понятие времени и так далее? Да, можно сказать, что просто будут другие цифры. Нет. Это совершенно разное. И ваш аппарат – всего лишь только ваш аппарат. И, придя на другую планету, не настроившись, в вашем понятии, на неё, вы будете видеть только именно, как говорится, один из параллельных миров, и именно связанный именно с этим аппаратом.}
\people{А математика должна быть универсальна для всех цивилизаций? Ведь математические законы должны быть общие. Они общие действительно?}
\soul{Конечно, для вас это звучит гордо, что вы придумали математические законы, которые будут действительны во всех Вселенных.}
\people{Но это неправда? Теорема Пифагора в других цивилизациях и  Вселенных неверна?}
\soul{Да. Говорят же вам: только, в именно обладающих вашей физикой, будет верна математика. В других Вселенных она будет уже иной. И в других, скажем так, в ``параллельных мирах'', она уже будет отличаться.  Не столь сильно, но, всё-таки, уже будет отличаться. И если есть понятие другого времени, понятие другого пространства, - математика будет совершенно иной, совершенно другой. Как вы думаете, если вы сейчас откроете четвёртое и пятое измерение – останутся ``пифагоровы штаны'' на месте?}
\people{Наверное, нет. Вы правы. Но математика, как наука, она присутствует во всех цивилизациях, во всех мирах?}
\soul{Да. Это одна из мер.}
\people{А у вас есть мерность? Вы меряете? Вы математически…}
 (разрыв записи. Начинается ответ на другой вопрос)
\soul{… который через миллионы лет кто-то уловит и будет думать: ``кто же это был такой неразумный?''. И это, может быть, будете вы же. (о сигнале SETI посланном в космическое пространство. прим.)}
\people{А вы знакомы с термином ``дева- эволюция''? }
\soul{Спрашивайте.}
\people{Человечество и дэвы идут,  как бы рядом друг с другом, но не могут разделиться, скажем так.}
\soul{Давайте скажем так: да, вы говорили о язычестве и множестве богов. И продолжим эту же тему.  (разрыв) Здесь есть, которые обязаны выполнять что-то определённое. Допустим, создавать ветра, третьи – создавать для вас штормы, четвёртые – волновать что-то - те же самые землетрясения, любые движения любой физики. Скажем так: институт существ, в котором каждый занимается своим делом. Да, это есть.}
\people{А вот, о единстве тогда. Вот, допустим, на Земле настолько много разных существ, но,  всё же, говорят, что всё это едино.}
\soul{Это едино. Но только вы не знаете об этом, вы не объединяетесь. И те же дэвы не могут объединиться.}
\people{У них тоже существуют какие-то дифференциации?}
\soul{Ну, вы же говорили, сами говорили, что самое совершенное – это Бог.}
\people{Ну, хорошо, если так сказать: вот, человек овладевает определёнными силами природы.  Вот, если таким термином пользоваться - элементалами. И, допустим, ими управляет. Ну, по каким-то делам. Стоит вообще этим заниматься и зачем всё это нужно?}
\soul{Мы говорили вам и повторимся. Чем больше вы имеете сил, тем больше будет нежелание пользоваться ими. И если вы получили возможность управлять дождём или чем-то другим – управляйте. Но что это даст вам? Только утешит вашу гордость, что мол, ``вот я какой''. Если только гордость будет управлять вами – ничего доброго не приведёт. Если же что-то иное, действительно истинное, то вы поймёте, что вам  незачем управлять этим дождём. И лишь только тогда – отступитесь от правила, если действительно нужно будет полить засохшие земли, а не просто показывать свои возможности. Многие из вас получают силы. Их получить очень легко. Для этого существует множество заклятий, и множество-множество всего. И всё это верно. Лишь бы человек веровал в это. Если он верит, что если он положит крестик под ногу и будет три дня его носить и станет обладать большими силами, ибо будет сатана управлять им,- да, это произойдет, ибо он этого хотел. И наоборот, если он крестик повесит на шею и будет мечтать о том, что наконец-то Христос поймёт его – этого не произойдёт, ибо он будет только мечтать, а не делать этого. Вот вам - чем отличается сатана от Бога. Что - сатана? – Это ваш страх, это ваша лень, это ваша жадность и множество-множество. А как легко поддаться соблазнам! И так тяжело пойти праведному…}
\people{Так, значит, сатана – это совокупность отрицательных качеств  всех людей?}
\soul{Давайте скажем по-иному. Нет такого понятия ``качества'', ``отрицательные или положительные''. Есть только понятие, как вы будете использовать. Нет такого понятия ``эволюция'' и наоборот. Нет таких понятий. Есть только понятие, что вы хотите стать богом или хотите остаться, кем вы есть.}
\people{Но если человек говорит, что его устраивает, какой он есть. Это значит…}
\soul{Это значит, что он остановился.}
\people{Это страшно.}
\soul{И если он будет кричать, что истинно верующий, что он Иисус Христос или Сатана – это не значит, что стал он им. Нет. Это есть, как вы говорите, одержимость. То есть, когда мечта владеет его разумом – только и всего. Когда ваши мечты сильнее вас. Ведь, чаще, мечты ваши, склонны к жадности и страху. Можете ли вы мечтать только, чисто о духовном? Можете ли? Нет. Вы обязательно перейдёте на физическое, на материальное. Вы уже забыли, как это делается, вы уже не умеете этого делать. Ранее, вы были менее лживы.  Да, это действительно так. В вашем понятии ``неандерталец”-  был чище, чем многие сейчас из вас, ибо он был ближе к животным. Но и не отличался силой – он был тем же животным. Он был чище только в том смысле, что он не осознавал своих ошибок. Он их не творил – инстинкт владел им. Вам же - был дан разум. И этот разум губит вас. И не из-за того, что разум вреден, а из-за того, что вы нашли способ услаждать себя с помощью разума, вместо того, чтобы ставить новые цели и работать. Спрашивайте.}
\people{Вот, есть у Киплинга стихотворение: ``наполни смыслом каждое мгновенье, часов и дней неумолимый бег. Тогда ты будешь человек''. Про себя я не могу сказать, что я могу наполнить смыслом каждое мгновение. Не знаю, как это сделать.}
\soul{Давайте скажем так: если вы будете специально стремиться наполнять, то это будет опять проявление жадности. Или гордости. Вы, делающи первые шаги, и тогда вам подсказывают. Тогда вас учат и ведут вас. Появляются учителя, подсказывают вам. Дальше? Дальше вы должны идти сами. Если вы совершаете доброе дело и тут же признаётесь себе, что вы совершили это доброе дело, - считайте, что оно уже не доброе. Что только жадность и гордость говорила вами, если вы подали кому-то милостыню и тут же похвалили себя, считайте, что вы погубили себя и человека, взявшего эту милостыню.}
 Скажите, вот в былые времена, до 19 века, в основном мужчины несли в мир учения. А далее, в основном, женщины.
\soul{Это связано не только с вашим понятием ``мужчина'' и ``женщина'', это не связано с тем, что женщина была более или менее чувствительна. Нет. Просто мужчина считался главенствующим, и женщина не могла поднять голову выше головы мужчины. Только и всего.}
\people{Скажите, а переводчик упоминает всё время, о какой-то кукле. Это плод его воображения? Или что-то связанное с ним?}
\soul{У вас плоха память. Возьмите и поднимите прошлые контакты. Вот и поймёте об искусстве прошлого.}
\people{Наша собеседница, действительно, не помнит эти контакты. Она не участвовала в них.}
\soul{Да, она не участвовала в них, но она знала.}
\people{Знала? А наша собеседница как-то связана с группой четверых, о которых вы давно говорили?}
\soul{И вы опять ищете. Спрашивайте.}
\people{Я попробую сейчас вопросы, которые нам по переписке присылали. В связи с идущей сейчас всеобщей, якобы, транс-мутацией людей Земли и мира тонкого в теле и духе истинном, в связи сдачи,  как бы ``выпускных экзаменов'' на всей планете, хотелось бы узнать;}
Для тех, кто успешно сдаст экзамен и будет переведён в следующий класс, т.е. новую эру на Земле, - Сатья-йогу,-  как быстро будет меняться существующие сейчас социально-экономические, политические отношения на планете? Каковы свойства этих предстоящих перемен будут? Какие у них будут задачи, у этих людей?
\soul{А теперь представьте, как переводчик должен перевести ваши слова и нарисовать нам картинку, чтобы мы могли понять, и чтобы ему было не сложно это сделать? Далее, давайте скажем, о классах, - что это ложно.}
\people{Ну а сам переход в новую эпоху. Он грядёт? Сатья-йога будет?}
\soul{Как вы можете представить? Резкое? Перелом?}
\people{Нет, не резкое…}
\soul{Да вы всегда шли. Вы всегда  менялись. Но, только мы говорили вам, что среди вас есть и первые и шестые. Как можно тогда сказать, что пришло это время или не пришло? Если только брать по процентам, что ``наконец-то, 80% стало желтых, а осталось 20% только таких. Только тогда можно.'' -  будем говорить?}
\people{Скажите, тело, через миллион лет,  допустим, претерпит какие-то изменения?}
\soul{Давайте скажем - через миллион лет, ваше тело уже не будет этим.}
\people{Оно будет другим, или оно не будет ``твёрдым'', будем говорить так?}
\soul{Будет оно твёрдым или нет, это зависит от вас. Но именно этих тел уже не будет.}
\people{Ну, это понятно. Просто, про человечество мы говорим…}
\soul{Мы говорим именно про человечество. Мы знаем, что ваше тело не может прожить более определённого срока. Именно тело.}
\people{Я говорю…}
\soul{Ваших форм, и этих форм, уже не будет.}
\people{А, допустим, в атлантическую эпоху, люди имели сходство с нашей, с арийской расой?}
\soul{Отдалённо – да. }
\people{Ну, а если бы встретились, мы могли бы…}
\soul{Вы бы назвали  бы его уродом. Что сказал бы он и о вас.}
\people{Значит, мы будем считать друг друга уродами…}
\soul{Вы были бы удивлены, в вашем понятии, ихней ``тупости'', их незнаниям. Ибо они не знают многое, что знаете вы. Ибо они были менее технократичны, чем вы. Они не смогут понять вас, потому, что не смогут понять многих ваших целей. Что не сделаете и вы.}
\people{Скажите, а вот какая-либо система солнечная, планетарная, имеет ли какую-то цель существования? Так же, как человек? Или она тоже ищет?}
\soul{Все эволюционируют. Да. Вы только поймите - Солнце не старается сохранить свою плоть, иначе бы оно поменьше бы пекло вам. Далее. Если есть начало, если что-то рождено, оно будет иметь и конец. Рано или поздно ваше Солнце остынет. Ваша Вселенная будет пуста и темна. И физика будет потеряна. Рано или поздно произойдёт наоборот. Когда ваша Вселенная или сожмётся или разбежится столь далеко, что даже планеты, молекулы планет будут ``разбежены'' так же, как сейчас звёзды. Тогда  уже нельзя будет говорить, о Вселенной. Это время наступит. И наступит время, когда это будет сжато в одну точку и когда нельзя будет уже говорить о понятии, о времени и пространстве, и что вообще это есть. И тогда, уже можно будет смело сказать, что эта Вселенная исчезла. Любая физика конечна. Любая физика исчезает. Вы говорите, о трансплантации. Да, появится новая. Что такое смерть? Что такое смерть?}
\people{Вы у нас спрашиваете?}
\soul{Да.}
\people{Смерть… Смерти, наверно, нет.}
\soul{Давайте скажем мягче. Зачем опадают листья?}
\people{Чтобы новые выросли.}
\soul{Вот вам и эволюция. А вы говорите ``смерть''. Вы боитесь её. Давайте скажем так: если вы будете стремиться сохранить свою плоть - это бесполезно. Вы должны сохранить себя, своё истинное ``Я''. Если хотите – хотя бы свои знания, чтобы были переданы другим. Ваши тела будут изменяться. Изменяться в зависимости от той же активности Солнца. Рано или поздно, Солнце будет менее и менее греть вас. А если вы ещё и устроите какие-нибудь войны, и вы можете исчезнуть совсем – на смену вам придут другие. И тогда, вас уже будут называть ``дэвами'' и далее, далее. Но только вы, в последнем миге, испугавшись,- согласитесь, вы получите очень больший страх, - для тех, новых рождённых, вы будете являться ``злыми духами''. Будете приходить к ним, и многие из вас, от зависти и ненависти к ним будут мешать им. Вот вам и ``ад'', вот вам и ``черти''. Но не все. Другие, из вас же, придут и будут подсказывать, чтобы не повторилось то, что было с вами. Вот вам и ``ангелы''.}
\people{Скажите, а солнечная система тоже имеет какие-то периоды?}
\soul{Да.}
\people{И в прошлой ``жизни'' солнечной системы что-то…}
\soul{Давайте скажем так: о прошлой жизни давайте не будем говорить, Солнца, или чего-то другого. Не берите планетарные понятия прошлого, ибо они столь большИ, что множество ваших жизней не хватает. И если кто-то говорит, что: ``вот, в прошлых жизнях Луны…'' или что-то подобное – это всё ложное, это, всего лишь ваши фантазии. Никак вы не могли помнить прошлые жизни той же Луны. Никак! Ибо в прошлой жизни Луны не было вас. Не было даже вас, и не было понятия о человеке.}
\people{Но ведь вся память существует где-то… в чём-то.}
\soul{Да. Та же самая Луна помнит всё. Вы же, увидев её, и услышав её, причём многие не слышат её сейчас, ибо её голос ослабевает, а вы становитесь на ухо всё туже и туже; вы же, услышав,  как испорченный телефон, вы думаете многое своё, более вам подходящее.  Пример - Библия одна, а сколько множество вер! Простите, вы ругаете мусульман. Но у них даже меньше разногласий, чем между вами. Вами выдуманы чёрные и светлые силы. Кришнаиты ушли пока от этого, но придут скоро и к тому. Они сделали хитрее - они чёрное и светлое объединили в одно. Вами выдуманы все понятия. Вами выдуманы прошлые жизни. Только вами, ибо вы прошлые жизни считаете только и меряете только плотями. Своими плотями. У вас была иная плоть, значит, и была прошлая жизнь. Да вы и не переставали жить. Разве вы помирали? Вы же сами говорите, что не существует смерти, и говорите о ``прошлой'' жизни!}
\people{Ну, это условности.}
\soul{Условности? Говорят вам, что все ваши знаки, все ваши алфавиты, все ваши слова уже влияют. И если они названы неправильно – неправильно будут и понято. И неправильно будут и работать. Если вы неправильно применяете термины, значит, всё будет неправильным. Ибо каждый слог, ибо каждое движение - влияет.}
\people{Тогда, может быть лучше - молчать?}
\soul{Зачем? Вы – человек. Ищите. Но только если вы будете применять только именно этот термин, и не использовать другие, обозначающие это же - это беда ваша.}
\people{А вот скажите. Вышла книга ``Диагностика кармы''. Автор приводит такой пример: у женщины болезни, и она не может от них избавиться. Муж её бьёт, оскорбляет, унижает и т.д. Автор говорит, что это ей во благо, потому, что она имеет большую гордость (гордыню), а муж ей дан для того, чтобы эту гордость из неё выбить. Что вы можете сказать на этот счёт?}
\soul{Ничего.}
\people{Это глупость?}
\soul{Вы подумайте сами. Разве это может быть действительным? Неужели вы думаете, что так грубо карма будет работать с вами? Если вы были горды, то следующий раз вас надо побить, унизить? Вы и придумали эти меры. Вы поймите, вы гляньте, вы посмотрите, что все-все ваши науки, какие бы ни были - духовными или физическими,- все они пронизаны вашими слабостями. Вашими характерами черт. Вы согласны? Зачем? Да, конечно, легче обвинить кого-то, обвинить в этом прошлую жизнь, которую нельзя исправить. ``Это прошлое, это было в прошлой жизни'', вроде как, не виноват сейчас, за что, бедный, карается? Это, всего лишь только один из признаков слабости. И выдумано вами, чтобы оправдаться. Ибо вы можете только силой или …(теряется)}
\people{Хорошо. Скажите, вы имеете какие-то творческие контакты с нашими ушедшими соотечественниками: Рерихами, Блаватской, Гагариным, Александром Менем?}
\soul{И вы снова мир измеряете плОтями.}
\people{Это те, которые умерли здесь, на Земле, ну, грубо, так скажем.}
\soul{Давайте будем так и говорить. Хотя, заметили вы, что мы не используем имён?}
\people{Да.}
\people{Да, мы знаем, но как их не назвать? Тут приходится их обозначать. Ну, вы с ними связи не имеете?}
\soul{Мы говорили вам, что мы не смотрим мир кусками. Мы стараемся охватывать всё. Но мы не имеем права вмешиваться и входить в кого-то, если он не хочет того или не знает того, или не слышит. Поймите, мы разговариваем с переводчиком просто из-за того, что мы вошли в резонанс. Только и всего. Мы можем, конечно, подстроиться под любого из вас, но это будет не выгодно ни вам, ни нам. Ибо мы станем рабами вашими, или вы – нашими рабами. Зачем?}
\people{Скажите, в Агни-йоге существует такое утверждение: есть Белое братство, так называемое. Существуют учителя, которые помогают людям. Это так?}
\soul{Представьте, приходят и обвиняют переводчика одержимым, что в нём сидит сатана. Приходят истинные верующие, в вашем понятии, и начинают ``изгонять сатану''. Что должны сделать мы? Что мы должны сделать? Если мы уйдём, чтобы не нарушить веру их, не поколебать их веру в бога, мы уйдём – значит, признаем себя сатаной. И, значит, всё, что мы говорили – ложно. Если же мы останемся, мы поколебаем веру. Что делать нам? Как нам быть? Как бы были бы вы на этом месте? Что сделали бы вы? Или уйти и признать себя виновным, или остаться, но, значит, поколебать чью-то веру? Поймите, мы не имеем права, в кого бы вы ни верили, даже если вы будете верить в то, что действительно существуют или наоборот, очень вредно, – мы не будем разговаривать с вами. Это ваш путь, ваш. Рано или поздно вы придёте. Вы должны придти. Ибо вы уже разговариваете, вы - уже думаете. И если мы скажем: ``это ложное'' или ``нет'', это значит, что множество-множество жизней будут погублено нами. Представьте, если мы скажем, что: ``нет, это не действительно''. Сколько множества людей может свернуть с этого пути, и, как говориться, ``от горя'' запить? А если мы скажем: ``да, это верно'', это не значит, что побежит множество сразу к Белому братству. Нет. Но это поднимет гордость множества, кто верит в него – что, вот, мы оказывается, правы, а остальные – нет. И что в итоге? Мы поддержали битву среди людей.}
\people{У нас есть одно письмо, которое необычное по тематике. ``Что ожидает город Чапаевск в ближайшие 8-10 лет? Наш городок славится наркоманией, больными детьми и химзаводами. Даже сейчас построили завод по уничтожению химических продуктов. Хотя он и не работает, но люди в это не верят. Если возможно узнать судьбу нашего города, то прошу ответить на вопрос''.}
\soul{Давайте спросим их, почему именно 80 лет?}
\people{Нет. В ближайшие 8-10 лет.}
\soul{И что должно измениться? Что?}
\people{То есть, наверное, ухудшения будут. Какая судьба этого городка?}
\soul{Нас больше волнует судьба всей вашей страны, всей вашей Земли. Какова будет судьба страны, будет такова и судьба этого городка. Отвечая на судьбу городка, мы, значит, ответим на судьбу всей страны. Страна же ваша… нельзя сказать, что она находится в состоянии покоя или движется. Скорее, она просто колеблется и не знает куда  идти.<1995год>  И если мы вам скажем, что у вас будет всё хорошо – мы солжём. И наоборот, если мы вам скажем, что будет всё плохо – мы опять будем лживы.  Всё будет так же, как и было. Ибо, вы остановились. Всего лишь только три года назад вы активно двигались, вы верили в светлое будущее и далее, далее. Сейчас вы остановились. Сейчас среди вас хаос. Вас больше волнуют набитые карманы. И не больше. Три года назад вы были честнее. Всего лишь только три года назад. За эти три года вы свернули, вы переломили, переломили всё, что было создано веками. Вы разрушили. А новое? Что вы построили нового?  Ничего. Если только не считать очередного ``храма для мамоны''.}
\people{Вы это говорите, о храме Спасителя?}
\soul{Мы говорим, о храме мамоны.}
\people{О банках, наверное, это.}
\people{Мне кажется, мы, всё-таки, прозреваем, видя такую паскудную жизнь, и, наверное, когда-то найдутся силы воспрепятствовать нашему разложению.}
\soul{Вот то и страшно, что вы говорите, что ``прозреваете”… (теряется)}
\people{Не прозреваем?}
\soul{Если вы начинаете верить, что вы чего-то достигли, это говорит только о том, что вы уходите от этой цели. Уходите и начинаете себе лгать, чтобы незаметно отступить от этого, и чтоб это не было больно. Нет - вы остановились. Остановились столь сильно, что очень трудно сдвинуть вас. И многие из вас, как вы говорите, экстрасенсы, вместо того, чтобы, как вы говорите в вашем понятии, лечить (только в вашем понятии),- стали губить. Очередной способ наживы, только и всего. Вы придумываете множество, множество ложного, лишь бы прославиться. Вы сочиняете новые, новые миры, сочиняете новые, новые … для чего? Чтобы выделиться и прославиться. Появилось множество контактёров, которые разговаривают с кем угодно, лишь бы их только слышали, лишь бы они были заметны. Появилось множество вер. Для чего?  Только для того, чтобы опять выделиться или стать первым, или получить власть. Появилось множество одержимых, которые называют себя ``Иисусами'' и наоборот. Есть множество пророков. И самое страшное – что не всех можно обвинить во лжи. Нет, они не пророки. Они просто больны. Но они не знают о том и кричат, что они прозрели. Они теперь знают всё, они знают весь мир, ибо они – пророки. Самое страшное, что они не признают себя виновными. Что не признают возможность ошибки. Даже Иисус, придя, сомневался. А множество, из ваших, дают множество клятв, что только они истинны, и только они верны.}
 (что-то про страны в ступоре, и он спрашивает, что правильно ли он делает, что помогает молодому человеку)
\soul{Вы снова повторяетесь. Вы снова хотите, чтобы вели вас за руки. Да поймите же, не будем мы делать того. Не будем! Вы должны выбирать! Поймите, если мы дадим вам сейчас ответ, хотя,  мы уже можем рассчитать, или, иначе, так, ``что может быть и не может, если будет заниматься тем или иным'', но если вы, сейчас, послушаетесь нашего совета, тем более,  если мы подтвердим фактами, которые потом подтвердятся, вы будете задавать всё более и более вопросов, и что в итоге? Вы станете нашей машиной, которая будет выполнять наши прихоти. Нам это нужно? Нет. Хорошо, возьмем другой вариант. Мы вам говорим, что надо сделать то-то, то-то или  будет то-то, то-то. И этого не происходит. Представляете? Вы говорите о доброте. Давайте скажем так: один из экстрасенсов говорит о доброте… (обрыв записи)}
 … у него очень хорошо развитая сердечная чакра, в вашем понятии. Он любит всё человечество, тут же предсказывает человечеству, что - вот, будет вот такая-то, такая-то беда, столько-то, столько-то всего погибнет.  Приходит время - это ничего не погибает, и он огорчается в том, что его предсказания не исполнились. О какой сердечной чакре говорите вы?
\people{Да.}
\soul{А вас множество таких. Большинство. Большинству, когда говорят ``вы умрёте через столько-то, столько-то'',- проживает человек, приходит – ``Да нет, я не умер''. И вы огорчаетесь: ``Ах! Как жаль!''. Вот она, ваша любовь.}
\people{Да. Вот из-за плохо проводимых реформ  и, возможно, предательской политики властей по отношению к народу, несёт нас в спячку, неверие и прочее.}
\soul{Давайте скажем так, что это не только ваша страна. Весь мир ваш сейчас остановился. Да, есть понятие, и оно верно, понятие экзамена. А если быть точнее, то это не экзамена. Просто надоело. Надоел  этот театр… (обрыв плёнки)}
    …в конце концов, и ужаснулись, - неужели это я?
\people{Ну, мы ужаснёмся? И, значит, начнём очищение?}
\soul{Будем надеяться.}
\people{Только `` будем надеяться''…}
\people{По-моему, сейчас нет повода надеяться.}
\soul{Мы хотим прекратить сеанс. Как вы относитесь к этому?}
\soul{Спрашивайте.}
\people{Мы хотим дать обратный отсчёт. Вы не возражаете?}
\soul{Вы приходите разговаривать – не мы.}
\people{Ну, сегодня хватит уже…}
\people{Вот, наши контакты, беседы… в принципе… мы же, в принципе – одно?}
\soul{Да, но вы не знаете того.}
\people{Мы не осознаём, а вы сознаете?}
\soul{Мы пытаемся это делать. Поймите, мы не говорим вам, что мы ``Высшие'', что мы знаем ``Всё''. Нет. Мы подобны вам, только у нас - иная физика. Только и всего. Мы тоже ищем. Мы тоже ищем смысл жизни. Только наше преимущество в том, что мы – посторонний наблюдатель по отношению к вам. И поэтому нам лучше видно. Согласитесь, что со стороны всегда виднее. }
\people{Ну, да.}
\soul{Только в том наша сила. Но это не заслуга. Просто - иная физика.}
\people{Я прочитала в книге из серии ``Трактат о семи лучах'' – есть такая серия книг 30х годов,- что ``недалеко то время, когда человечество будет выходить на контакты''.  И что это должно поощряться.  В общем – это приветствуется.  Это необходимо человеку и вам, и вообще, по-моему, всем.}
\soul{Один из способов познания.}
\people{Один из способов познания, действительно. То есть, вы оцениваете положительно? И человечеству стоит оценивать такие моменты положительно, а не пугаться и запрещать это?}
\soul{Пугаться нельзя никогда. Страх никогда не должен быть во главе. Даже если пришло, действительно, что-то истинное и ужасное, вы должны не бояться, а искать способ борьбы. Чаще же, вы убегаете и только больше вредите себе этим. Чаще, к вам приходят и пугают вас, чтобы накормиться вашим страхом. Мир эмоций – столь большой мир, в нём столько много хищников, и все хотят полакомиться вами. Многие же из вас, чуть более чувствительны, в вашем понятии - экстрасенсы, тем более питательны, ибо они больше дают пищи, ибо они больше реагируют на окружающее. Ибо их легче обидеть, или наоборот. Множество хищников бродит в вашем мире и питается вами. А вы же, чаще всего, с радостью:  `` Во как! Я знаю то-то, то-то, я что-то чувствую…'', нет – это, всего лишь  множество для вас, приманка, чтобы вы раскрылись.  И, в то же время, мы говорим вам: ``Будьте раскрыты. Раскрывайте себя полностью''. Но, если есть в вас страх, что натопчут, – натопчут.}
\people{Можно личный вопрос задать?}
\soul{Спрашивайте.}
\people{У нас есть знакомый. Он экстрасенсом себя считает и контактёром. Год назад я ложилась спать и услышала внутри себя его голос: ``помогите, помогите''. В общем-то, страха не было, я просто стала наблюдать и  увидела картину, не очень приятную. Вы можете считать её с меня? }
\soul{Давайте скажем так:  вы, считая, его неправым, создаёте свою картину. Это не значит, что это была истина. Нет. Просто вы создали эту картинку, и вы считаете, что он там-то и там-то не прав. Потому, нарисовали сами себе. В том и беда ваша, что вы, чаще, видите мир, выдуманный вами. Вы должны помнить песню о любимой: когда ожидаешь её – весь мир будет прекрасен, когда не  пришла –  и мир пахнет кислой капустой. Вы помните? Вот ваша иллюзорность.}
\people{Да, но дело в том, что после этого случая уснула и ночью слышу голос: ``Проснись и выключи грелку'', а я легла с электрической грелкой. А другой голос, такой ехидный, говорит: ``Не отвечай, спи дальше''.}
\soul{Давайте скажем так, что изучая Фрейда, вы совершите множество ошибок. Может, он был прав, но ваш мир слишком скучен. Всего лишь только три меры двигают вашим миром. Это скучно. Дальше. Чаще - вы не пользуетесь своей памятью как надо. Вы тут же забываете. Тут же. Мгновенно. Хотя, только что приказали себе это всё запомнить. Ваш мозг отключён, приходит время отдохнуть. Сон? Сон для того, чтобы, наконец, дать мозгу отдохнуть и навести у себя порядок. То, что нужно – запомнить, то, что не нужно – отбросить. И вот, пожалуйста – он делает свою работу. И находит какие-то ошибки, которые ранее были не замечены вами. Или какие-то недоделки, или какая-то мысль, которая только началась и тут же была прервана только тем, что кто-то позвонил или ещё что-то подобное. Он продолжил эту мысль. И если есть возможность, он проявит её во сне, чтобы закрепить. Дальше. Ваши чувства обострены, когда вы спите. Да, зрение выключено, слух  притуплён, осязание – тоже. Но у вас помимо этого есть ещё подчувства, скажем так. Под-зрение, хотя вы, наоборот, вы называете его ``третьим глазом'' или как-то ещё по-иному. И каждое ваше чувство,- в вашем понятии, их шесть, хотя вы не все признаёте,- имеет подчувства. Они никогда не отключаются – они всегда работают. С помощью их, вы можете реагировать на мир. Ведь когда вы спите, вы ж не отключаетесь полностью, вы согласны? Далее и далее… (теряется)}
\people{Вы сказали, что мы не отключаемся… и далее прервалось на этом. Вы можете продолжить мысль?}
\soul{И далее… Вы больше можете и имеете право назвать себя экстрасенсом, чем когда были наяву. Ибо вы более чувствительны и менее лживы.  Множество чувств ваших приносит информацию лживую или ненужную, которая забивает то зёрнышко, которое  именно должно было быть получено вами. И в эти моменты, вы очень чувствительны, и в эти моменты к вам приходят, или во снах, или как-то по-иному. И в эти мгновения вы более человечны и менее лживы. Много ли вы лжёте во снах? И, заметьте, во снах вы не знаете страха. А если и знаете, то он столь смешон. Согласны, что у вас нет понятия, что это невозможно? Вы не удивляетесь тому, что что-то такое, если бы было наяву, привело бы вас в ужас и вы в это не верили.  Во снах всё нормально, всё прекрасно. Вы не верите снам. Вы не верите им. Вы говорите, что это ложно. Вот вам - Фрейд. А если бы верили в них или хотя бы делали выводы из них, - это было бы множество. Вы же, чаще всего, понимаете чувствами или мыслями. Если вы приходите, вам надоедают какие-то определённые люди, то вы тут же найдёте множество причин и, тем более, если вы  экстрасенс, вы тут же скажете, что ``у него такое-то, такое-то поле, здесь что-то плохо, здесь что-то не то'', и вы уже сможете солгать себе более обоснованно.}
\people{Просто я хотела сказать, что в связи с этим случаем,  я ему на следующий день позвонила, и он мне сказал, что на него было какое-то нападение, и что, действительно, это случилось. Что-то такое было ужасное… вот что интересно.}
\soul{Давайте скажем так: эта история была переводчику известна, ибо вы ему уже рассказывали. Дальше. Множество ли бывает совпадений? Множество ли бывает совпадений? }
\people{Да нет, наверное, это всё чувство, сверх-чувство, о котором мы ещё не отдаём себе отчёт.}
\soul{Зачем же тогда спрашиваете? Вы не доверяете себе, а хотите дальше выяснить ``есть или нет''. Но поймите же, что вы не можете солгать себе. И если вы что-то подумали, даже если относительно другого это ложно, относительно вас это будет верно. И если было совершено нападение, это не значит, что было совершено нападение. Нет. Потому, что вы сами могли напасть. А будете говорить, что было нападение, от другого. Вы не допускали ни разу этой мысли, что вы совершили нападение, а не кто-то иной? Или он может напасть на вас, и вы, защищаясь, сделаете то же самое. Нет, вы сразу находите силы извне. Так почему вам не разобраться в себе? Кто из  вас совершил нападение? Подумайте.}
\people{Хорошо, она подумает. Скажите, а у нашей собеседницы каков потенциал неиспользованных возможностей? Он велик? Можно ли идти в развитие дальнейшего?}
\soul{Покажите мне хотя бы одного человека, который не мог бы идти.}
\people{Ну, она правильным путем движется? Тем, что читает книги…}
\soul{И мы снова повторяемся.}
\people{(Белимов – девушке): - Ну, хотелось бы узнать твоё назначение в нашем окружении. И творческий потенциал.}
\soul{Вот вам - материальность. Вот вам - чисто физика, когда вы должны конкретизировать всё. Математически рассчитать всё ``от и до''. Хотя, тут же забудете о математике и будете выполнять всё это хаотически.}
\soul{А вообще, человек должен организовывать каким-то образом не только свою жизнь внешнюю, будем говорить…}
\soul{Обязательно. Иначе, он не будет тогда делать ничего. Только не надо быть строгим. Ибо многие создают режимы и соблюдают его строго независимо от человеческих факторов – лишь бы был соблюдён этот режим. Дальше, чаще всего старайтесь не обвинить себя, а посмотреть себя - ``не я ли виноват в этом?''. Вы же стараетесь искать всё извне. Поймите, если с вами случилось что-то, то в первую очередь виноваты вы, ибо вы позволили себе, ибо, если на вас напали, значит, вы хотели того нападения. Почему? Потому что – подобное к подобному. Резонанс. Если у вас очень прекрасное весёлое настроение и истинно весёлое настроение не оттого, что просто у вас кошелёк был набит, а действительно - вы слились с природой или что-то,- в это время ничто не случится с вами плохого. Если же вы идёте и радуетесь чему-то, и, в то же время, где-то сидит червячок сомнения ``не слишком ли много я смеюсь?'' - что-нибудь да случится, ибо вы хотели того.}
\people{То есть, это изречение Христа: ``Вера ваша спасла вас'' - в этом. Да? Очень многое можно найти… смысл…}
\soul{Нет, смысл здесь один. Воспринимать вам надо именно так, как было сказано. И не искать подсмысла. Вы выбираете. Вы имеете свободу выбора. Вам и решать.}
\people{А вот, вы,  сомнения - считаете грехом? Вот, человек, делающий все дела и не имеющий сомнения – он будет прав?}
\soul{Нет. Вспомните, даже Иисус сомневался. Вы должны это помнить.}
\people{То есть, это черта характера человека не самая осуждаемая, не плохая?}
\soul{Ну, если всё доводить до крайности, то будет осуждаемо всё.}
\people{Ага.}
 (конец контакта)
Аоум. глава 18-я 16-04-1994г
Георгий Губин
  
\people{**}
\people{Вы слышите нас? Сегодня 16 апреля, суббота, мы сегодня впервые без Геры участвуем.}
\soul{Когда-то мы вам говорили о параллельных мирах. Как вы думаете, умирая здесь, вы умираете в иных мирах?}
\people{Нет, мы не знаем. Скорее всего - нет.}
\soul{Значит, не поняли, что было сказано вам. Мы говорили вам, что рождаетесь, и вы рождаете миры. Да, есть миры, где вы уже давно умерли. Есть миры, где вы не родились. Есть миры, где вы ещё живёте. Где-то вас нет.  И многие-многие. И вы их называете `` множество вариантов  конечного''.}
\people{Нам это слишком не привычно, и совершенно ново. Хотелось бы поверить, но нет подтверждений. }
\soul{Какие  будут вам подтверждения? Только умерев, и уйдя в те миры, вы скажете:  ``Да, было''.  А может быть, тот мир,  мир умерших, будет для вас тоже, что и этот, и вы будете искать подтверждения в том же мире и  будете называть ``наш мир'', мир в котором вы живёте теперь, ``извне''. Как и другие.  Неужели вы думаете, что уйдя, вы изменитесь? Если б изменились, вы бы не возвращались.}
\people{Но, наверно, у нас другое видение уже мира. Некоторые осознают, что они, перейдя в другой мир,  больше видят и своих грехов, свои не совершенства на земной жизни. Или это даже не учитывается?}
\soul{Ошибка ваша в том, что говорите: ``не совершенство земной жизни''. И было сказано сейчас о параллельных мирах. Вспомните сказанное сейчас. Как вы можете говорить о не совершенстве, когда допускаете, что другие не совершенны. И в других мирах вы будете так же искать то же, что ищете и здесь. Только эти миры будут для вас уже ``иные''.}
\people{А несовершенство имеете в виду человеческое? Каждого индивидуума?}
\soul{Нет. Несовершенство - ваше сознание.}
\people{Да, видимо так. Но, всё-таки, о параллельных мирах. Мы так полагали, что это не связано с человеком в земном воплощении. Вы же говорите, что это земное воплощение и никакой смены нет. Тогда, получается, нет разных цивилизаций?}
\soul{А вы подумайте, мы же говорили вам в начале,  ``где-то, вы родили миры, где-то - ещё вы живёте. Где-то, вы умнее, где-то – глупее. Где-то, совершеннее, где-то – нет. Вот и различия ваши. Далее. Вы говорите:“Иные миры, – иные жизни''. Посмотрите! Вы – меняетесь! И вы - не постоянны. В каждую минуту – вы другой. В каждую минуту, можно сказать, что вы иной. Подумайте! Вспомните себя, каковы вы в разных обстоятельствах? Какие вы в  транспорте, дома, на работе? Вот вам – ``всё во всём''. Теперь представьте, если различий  больше? Конечно же, вы будете больше ``иные''. Далее. Вы говорите, и поняли так, что инопланетяне, это же вы. Вспомните. Мы говорили, что вы едины. Помните? Как тогда вы можете сказать, что если вы едины – то они – не вы? }
\people{Получается, - между нами… То есть, ничто ниоткуда не возникает и ничто никуда не девается? Типа закона физики. И мы варимся в одном котле?}
\soul{В вашем понятии, если вы когда-то родились, возьмём точку начала, то больше ничего не рождается, ничего не умирает.  Как вы поняли единство? Как вы поняли бесконечность? Мы ж говорили вам,  вы пришли сюда и рождаете. Рождаете и духовно, и, тем более,  физически.}
\people{Ну, всё-таки, с чем нам сравнить? С круговоротом воды, например, здесь; вода, лёд, облако?}
\soul{Нет. В вашем понятии `` круговорот''? Скажите, вода изменяется? Нет. Она имеет только разные агрегатные состояния, и не больше. Мы же говорим о вас, как об изменяющих. }
\people{А вот, параллельные миры, которые рождает человек высокоинтеллектуальный, конечно должны отличаться от миров, которые (рождает) духовно убогий человек, агрессивный? В тех мирах, наверно, жутко, страшно жить.}
\soul{А вы подумайте. Какие миры родите вы, собеседник, ваш знакомый и далее? Каждый, из всех вас, родит миры по своим понятиям. Как вы понимаете? Вы понимаете буквально? ``Родить новое''? А может быть, просто, в том мире вы сделали немного другое? Здесь вы  сказали одно слово, там - другое. Вот вам – ``параллель''.  Вот вам – ``начало''. А далее, - всё больше и больше. Сделанная одна ошибка или сделанное одно движение – влияет на будущее. В вашем понятии – это будущее будет всё больше и больше меняться, уходить от вас, которого вы считаете ``эталоном''. Вы говорите: `` иные''.  И не можете представить, что кто-то говорит о вас, как о ``иных''.}
\people{Ну, это мы можем представить. А вот, вы? Вы во всех Мирах ведёте вместе с этими…или только с нашим  с земным, c  данными индивидуальностями так себя ведёте? А в других может, вас и нет? Вы не вхожи в их сферу сознания, или вхожи? (имеется ввиду контакты с иными мирами помимо нашего. прим.)}
\soul{Давайте скажем так… Вы находитесь в физическом мире. И есть множество миров, физических. ``Иных'', в вашем понятии. Но есть миры, отличающиеся физикой, но тоже остающиеся физическими. Скажем по-вашему: ``иная'' таблица Менделеева, или проще - другой порядок той таблицы. Но, есть миры, где нет физики. Как мы вам можем объяснить, если у вас нет даже понятий?}
 1-2-3-4-5-6
\people{Ну, а сквозь понятия, такие, как ``мыслеформа'', ещё что-нибудь,  можно как-то при… Если сказать, что это ``мыслемир'' какой-то? Будет ли это хотя бы немножко правильно?}
\soul{Нет. Это будет, только немножко ближе. И только. }
\people{Но, всё-таки?}
\soul{Вы находитесь в физическом мире. И ваше сознание столь мало в этом мире, но считает себя царём. И потому, не даёт другим. Мы говорили о подсознании. И говорили, что оно знает более. Более, чем ваше сознание. И, вспомните, мы говорили вам - проценты. Мы говорили вам ``доли'' сознания< доли процента> и до 3-х,5-ти – подсознания. Тогда подумайте, для чего даны иные?}
\people{Проценты, да? Количество?}
\soul{Ваша задача, вспомните,- энергия! Умение управлять энергией! Вас пронизывает множество энергий! Вы обладаете огромными могуществами! Но не умеете управлять этим.  Хаос! Если хоть кусочек придёт порядка, вы говорите: ``Ясновидение''! ``Озарение''! Всего лишь, кусочек!  Представьте теперь, если вы сможете охватить всё? Если сейчас вы сможете охватить всё, что будет с вами? Что будет с вашим сознанием? Оно было ``доли'' и будет уничтожено. И что это даст? В вашем понятии,  вы умрёте. И потому, нельзя, нельзя объять всё, если не родился со знанием всего. Как вы можете представить дитё, знающее всё? Знающее все ваши науки. Да и посмотрите, что случается в таких случаях.  У вас были и есть, и будут. Вспомните, вы их называете ``юные гении''. И что далее? Что дальше? Где их сознание?}
\soul{1-2-3-4}
\people{А вот, как раз в Библии часто говориться, что надо родиться ``свыше'', ``от бога''. Во время крещения происходит, как бы, ``рождение''. Есть какая-нибудь связь с тем, что вы говорите?}
\soul{Если вы приходите, просто, ``модой'', или что ещё, или насильно креститься, то - нет, ничего вам не будет. Вы должны прийти не просто желанием… Необходимостью! Необходимостью'' родиться заново''.  И тогда, крещение даст вам. Но, подумайте,  во всех ли религиях есть ли крещение?}
\people{Практически, оно, конечно, нет, в таком виде, как в христианстве, но поклонение воде идёт кругом. Даже в древних цивилизациях.  Майя поклонялись воде.}
\soul{Везде есть крещение. Везде. Не обязательно водой.  Что такое крещение? Крещение, это, всего лишь, начало. Начало,  помогающее вам родиться. Если хотите, это просто ``акушерка'', в вашем понятии. Вы должны прийти необходимостью, необходимостью.}
\people{Значит, неспроста Иисус крестился в 30 лет?}
\soul{Вы уже ответили сами.}
\people{А как же нам относиться к тому, что православные любят крестить детей с рождения? Это ``добро'' или ``зло''? Или что?}
\soul{А вы можете назвать хоть что-то, где вы действительно нашли золотую середину? Далее. Подумайте, как вы противоречите себе. Если младенец не имеет греха, зачем же тогда крещение? Далее. Вы приходите креститься с младенцем – понимаете ли, что делаете? И будет ли это началом? Для многих – нет. Для многих, будет просто забыто. Для многих. И подумайте. Связь дитя и матери… Если хотите, то ближе будет,- мать, приводящая младенца креститься, - крестится сама.}
\people{Ну, я смотрю, мы опять перешли опять на вопросы библии и религии. Нас это интересует. И вот, из этих вопросов, что мы заранее готовили, есть такой…  Из множества притч, нас заинтересовали две; когда Иисуса спрашивали о Царствии божьем, то он начал вспоминать о дрожжах и о горчичном зёрнышке, которое разбухает или прорастает. Объясните, зачем нужна эта иносказательность, которая больше похожа на игру слов, чем на попытку доступно объяснить суть вопроса. Что вы можете ответить нам, теперяшним, спустя 2000 лет, на этот же вопрос?}
\soul{А теперь, подумайте и вспомните. Мы сегодня говорили вам, как вам сказать о том мире, и у вас даже нет понятий, и у вас даже не отчего оттолкнуться, чтобы поняли, что это? Далее. Вспомните. Вы - сито! Сито. Далее. Вы иносказательно понимаете более. Ибо  начинаете думать и договариваете сами. Поймите, если придут к вам и скажут ``от'' и ``до'', скажут словами,  иные слова уже не придут к вам в голову. Хотя, быть может, те слова, что могли бы найти сами,  помогли бы вам более. Далее. Сказано вам: `` Отворите!'' А если вам будет сказано всё, вы разве отворите?}
\people{Ну, вот иносказательность и недоговорённость в информационном плане, получилось, разделила верующих на разные группировки, по своему толкующих притчи, заповеди. Как вы относитесь к такой многоликости верующих? Ведь она не в пользу, как христианству, так и Учению.}
\soul{Не в пользу, только тогда, когда вы идёте ``мечём''. У вас есть – `` В спорах рождается истина''. Вы помните?}
\people{Помним.}
\soul{Если вам не приходиться никому ничего  доказывать, много ли вы будете знать об этом? Вы же забудете. Иначе - вы не будете искать… }
 1-2-3-4-
\soul{Спрашивайте.}
\people{Значит, будущее будет за объединением, всё-таки,  отдельных направлений христианства, союз, за истину какую-то? Солидарность в поисках, а не разобщение.}
\soul{Мы говорим, что вы едины, то значит, рано или поздно,  вы все осознаете то. И тогда у вас не будет религий никаких. Вы будете жить в том мире , в котором вы называете ``религией''. И тогда у вас не будет понятия ``бог''. Ибо вы будете жить с ним. Будете знать истинного бога.}
 1—4
\soul{Спрашивайте.}
\people{Это положительный момент, когда люди откажутся от религии? Поймут…}
\soul{Поймите, мы говорим вам, когда вы придёте в мир, о котором говорите в религии, нужна ли будет вам та религия, если вы уже будете жить в том мире? Если сейчас откажетесь, а у вас уже было то,  вспомните, много ли хорошего в том?}
 1-2-3-4
\soul{Спрашивайте.}
\people{А какие основные ошибки делают те или иные секты, или церкви, в изучении библии, и в выполнении каких-либо обрядов и их толкований? Есть такие ошибки?}
\soul{Ошибки? Это создание обрядов.  Вы должны жить, жить по вашим заповедям. По заповедям, что даны вам. А вы же, придумали обряды!  Что это? Это попытка уйти от реальности, попытка ``купить''.}
\people{Понятно. Вот, существуют споры, насчёт дат, не указанных в библии, как ,например, день рожденья Иисуса Христа, его крещения, Пасха, как день его воскрешения. Насколько эти даты верны и необходимы человечеству?}
\soul{Даты не верны. Не назвали вы тех дат, которые были.  Сначала было множество версий. Вы даже зацепились за затмение.}
\people{А эти даты, видимо, не важны, не играют такую уж большую роль для человечества?}
\soul{Нет, они играют. И играют больше даже, чем вы думаете. Потому и будете искать.}
\people{А вы не можете назвать дату рождения Христа?}
\soul{Вы должны искать, или мы должны принести вам?}
\people{Ясно. }
\people{На что нам обратить внимание в поиске этой даты? На текст в библии или на гороскоп какой астрологический, или ещё на что-нибудь?}
\soul{Сделайте проще. Гораздо проще. Уйдите от математики, уйдите от физики и попробуйте пережить. Пережить всё, что сказано.  Пережить до той степени, что следующую страницу вы будете повторять сами, не читая. Тогда можно будет говорить, что бог в вас. Христос пришёл к вам.  Далее. Вы можете сказать это о любой религии.}
\people{Ну, это очень сложный путь. Боюсь, что …}
\soul{Сложно? Сложность в том, что вы боитесь. Боитесь потерять себя. Вы боитесь потерять те `` доли процента''. Вот ваша сложность.}
\people{Но есть люди, которые уходят вообще в монастыри в изучение религии и то, мы на них смотрим, они не достигают этого желаемого.}
\soul{Подумайте, что делаете. Вы уходите в монастыри и нарушая, нарушая законы. Вы пришли в этот мир, что бы жить. Жить среди всех. А вы уходите. Уходите в монастыри. От кого вы уходите? От себя? От всех? Чтоб уйти к богу? Тогда почему вы забыли, что в каждом бог? Тогда зачем вы ушли от бога?}
\people{Но они всё хотят посвятить богу.  Полное служение. Не отвлекаясь на мирское. Так, по крайней мере.}
\soul{Всё - мирское? Тогда, ваше будущее, это будут кельи. Пустые кельи, где нет бога. Подумайте. Вы пришли в мир, чтобы жить в нём. Жить так, чтобы другие шли за вами. Жить примером. А вы уйдёте. Уходя, вы предаёте. Предаёте имя божье, ради которого уходите. Ведь сказано: `` В каждом бог!''.  Вдумайтесь! В каждом! А вы уйдёте. Вы уйдёте от множества разных видений бога. Каждый видит бога по-своему. Вы согласны?}
\people{Согласны.}
\soul{Теперь представьте, что вы ушли, чтоб увидеть только в своего, уверовав, что вы верны, что вы правы, а остальные – нет. Так что же?  Этим, вы уменьшаете. Этим вы оскорбляете бога, во имя которого творите. И многое, многое вы творите ради себя. Нужны ли богу ваши обряды? Может быть, они нужнее ВАМ, а не ЕМУ? Подумайте.}
\people{Скорее всего, так и есть.}
\soul{Вы же, говорите:  `` Богу угодно то-то и то-то.'' Как вы можете говорить то, если не видели его? Как вы можете говорить и делать за него? А вы, только этим и занимаетесь.}
\people{Но, вот ваши слова, они выглядят еретически для серьёзных церковных деятелей, митрополитов и так далее. Вас это не страшит, что вы можете получить в лице их – врагов, отрицающих ваше правильное, может быть, воздействие. То есть, стоит нам только опубликовать  наш с вами разговор, как церковники могут преследовать такие мысли и говорить, что это ``от бесов''.}
\soul{Вспомните. Вспомните! Земля – вертится! Вспомните. То же, было неугодно ``верхам''. Но, Земля то - ВЕРТИТСЯ! Далее… Что же говорим против бога? Что? Разве бога обвиняем? Или обвиняем ваши ``верха''?  Разве бога обвинили мы? Разве мы сказали, что нет его? Разве мы сказали, что он…}
 1-2-3-4—
\soul{Спрашивайте.}
\people{Но из ваших разговоров получается, что сейчас больше достаётся нашей православной церкви.  А в этом плане лучше, например, Свидетели Иеговы, которые, непосредственно поклоняются имени бога Иеговы? И они утверждают, что это его имя. }
\people{И у них нет обрядов. }
\people{И нет  у них обрядов, и ничего такого нету.}
\people{Может быть они, действительно, ближе?}
\people{Они ближе, хотя бы на один шаг?}
\soul{Нельзя говорить ``ближе'' и ``дальше''. Нельзя, поймите. В любой религии есть те, кто ``ближе'' или ``дальше''. Вы же хотите, чтоб я сказал вам о всех. Согласитесь, что в любой религии есть множество людей, и они все по-разному видят, и кто-то видит дальше, кто-то видит ближе. Дальше… Вы поклоняетесь имени божьему, или, всё-таки, богу? Есть в том разница? И знаете ли вы ИМЯ?}
\people{Вот, споры между христианами вызывают даже такие данные Моисею заповеди, к примеру, слова – ``Не сотвори себе кумира'',  ``Не смей изображать ни живущих на земле и на небе и под землёй''. }
\people{У меня лично складывается впечатление, что нельзя Иисуса изобразить как свет, или, что-то параллельное, вот, вы говорите, чисто духовное. Вот как, в этом плане, действительно, истинный смысл этих слов?}
\soul{Истинный смысл? Да, вы не сможете изобразить его никак. Вы можете отобразить только его маленькую, маленькую сторону в вашем физическом мире. Если хотите – отражение, тень. Не более. Далее. Вы создаёте иконы и говорите, на ней изображён бог. И подумайте, что ваша икона, в вашем понятии, имеет ауру. Ауру, которая создана, создана художником, рисующим его. Вы не думали о том? Далее. Подумайте. Вы говорите о плачущих. И вы говорите: плачет икона. То обман. Обман, созданный вашими ``верхами''. Неужели вы думаете, что бог идёт на обман? Вы хотите сказать… Вы сравниваете его с вашими вождями, которые хотят набрать просто…}
\soul{1-2-3-4-}
\soul{Мы хотим вас поздравить.}
\people{С чем?}
1-2-3…
\people{С чем  вы нас хотели поздравить? Что-то у нас получается хуже или лучше?}
\soul{Мы говорим лично о вас.}
\people{Так.}
\soul{Вы уже можете разговаривать…}
\people{Спокойно и понятливо для переводчика, так что ли?}
\soul{Нет. Вам назвать имя, или назовёте сами, без кого не могли обойтись?}
\people{А-а… ясно.  Ну, это хорошо. Мы учимся, значит. А вот моя попытка лечь и попробовать самому контакт, не вышла. Вы говорите: `` не было желания''. Ну, было оно. Только я, может быть…}
\soul{Поймите. Можно веровать, но не знать. Можно знать, но не веровать. Вы знаете, что есть другие страны. Знаете ведь, но не были там. Не были. Вот и приведите аналогию. }
\people{То есть, мне не хватает, видимо, веры? Да? В возможность этого контакта. Ну, ладно. Спасибо.  Продолжим. Какие особенности нас ждут на пути развития цивилизации и на пути эволюции? Какая конечная цель цивилизации и эволюции человечества? Это трудный вопрос, но попробуйте нам вкратце.}
\soul{Нет. Трудность в том, что вы не можете понять, что не будет конца. Не будет. Даже если вы уничтожите себя, вы уничтожите только физически. Вы уничтожите возможность жить в этих мирах. Вы  уйдёте в другие. В более тяжёлую, в вашем понятии, материю. Быть может, вы станете камнем. Камнем где-то у ``иных''. Но есть и другая. Вы станете, в вашем понятии, птицей. Вот вам - иносказание. Вы станете летать всё выше и выше. И не будет тому конца. Не будет тому предела. Какой может быть предел, если вам будет легче? Хорошо, пусть вашим пределом будет бесконечность.}
\people{А значит следующий шаг, примерно, какой, всё-таки, ждёт нас?}
\soul{Следующий шаг? }
\people{В нашей эволюции.}
\soul{Следующий шаг у вас – это кровь.}
\people{Вы не в первый раз нам об этом говорите. Неужели нельзя это избежать?}
\soul{Можно. Можно, и потому приходим ко многим, и не только мы. А что вы понимаете? Если вы где-то услышали о контактах и там говориться – ``Люди, живите мирно'',- вы уже иронически улыбаетесь и у вас уже меньше веры в тот контакт. Подумайте, разве не так? Потому что, у вас это уже набило оскомину. Вы привыкли к тому, что можете себя легко уничтожить и потому перестали бояться.}
\people{Ну, это грехи нашей цивилизации. А вот по вопросам эволюции? Это немножко разные…}
\soul{Вы можете жить здесь?}
\people{Вынужден, и живу.}
\soul{Вы не знаете, что делать. Вы ищете множество ответов и задаёте больше несущественных вопросов. Вы говорите: иные миры. Вы ищите иное, а для вас `` иное'' – ваши ближние. Вы не можете разобраться в них, а уже ищете Иные Миры! Ищите иные цивилизации! Вы хотите увидеть Шамбалу? Вдумайтесь,  что вы увидите? Что вы увидите там своим взглядом? И чем вы будете там? Чем? Вы придёте – завоевателями. Вы – будете сеять свою пропаганду. Свои понятия религии. Вы придёте в вашу Шамбалу и скажете: ``Вы - не верны''. Возьмёте все их знания и поставите свою религию. И разница только в том, что это будет синтез их и ваш. Синтез, но не Истина.}
\people{Скажите, вот мы связываем эволюцию человечества с такими понятиями как парапсихология. Парапсихология – это действительно наше прошлое?  Или будущее?}
\soul{И да и нет. Да, в прошлом, и даже в вашей жизни, вы обладали тем. И обладаете сейчас. Но мы же, говорили вам - хаос! Далее. Вы будете уметь всё более и более  управлять. Но сейчас управляют вами.}
\people{В связи с этим, вот такой вопрос:  Вы говорите, что можете видеть любые события. Расскажете нам вот о возможностях в старину таких единоборств, как ниндзя или ещё что-нибудь? До нас доходят некоторые данные, что они обладали там парапсихологией  человеческой очень сильно, Использовали левитацию, удары на расстоянии. Астральное каратэ.}
\people{Расскажете  нам, как это, что это? Напомните нам наше прошлое в таком случае.}
\soul{Хорошо. И заметьте, вот вам  ваше подобие. Вы говорите о психическом. Вы говорите о духовном…  И о применении, как оружие.  Вот вам пример. Вот вам, то, что мы говорили о Шамбале и иных. Пожалуйста!  Вы представляете, как оружие. Но, многое - ваши легенды. Ваши легенды, ваши мечты. Вы говорите – было. Было. Но гораздо меньше, чем слышите. Да. В вашем понятии есть, и мы не отрицаем, и уже говорили много вам, - вы обладаете столь могучей энергией, что способны затушить галактику. А вы  говорите – каратэ! Мы ответили вам?}
\people{Ну, понятно. Вот парапсихологи никак не могут найти, какая часть черепа человека отвечает, например, за способности к ясновидению. Вы можете подсказать, где, например, искать части мозга?}
\soul{Тогда вспомните, в вашем понятии, чакры.}
\people{Так. Именно они играют роль при ясновидении, да?}
\soul{В вашем понятии что это? Физически, это сгусток нервных волокон. Что вам назвать?}
\people{То есть, это не в головном мозге заложено ясновидение, а в развитых чакрах,  да?}
\soul{}
  В вашем понятии, третий глаз. Но, есть и выше. Выше. Вспомните. И будьте внимательны, есть чакры над головами. Как можно говорить, в какой части мозга находиться она? Дойдя до неё, и умея управлять ею, вы уйдёте, в вашем понятии, в миры, о котором мечтаете сейчас в религиях. И вспомните, вы говорили о шестой расе. 
\people{О чём? О шестой расе?}
\people{Да, мы помним.}
\soul{Вот и подумайте. Подумайте, ваши чакры… И за что отвечают они? Подумайте. Сравните. Есть люди, которые  владеют сомнениями.  Есть люди, кто рождается всё более и более, которые владеют уже больше, в вашем понятии, сердцем. Есть чакра ума.  Но и беда ваша в том, что если вы обладаете большой мощью чакры ума, значит, остальные у вас задушены. И, заметьте, так, в вашем понятии, больше ``ума'' – меньше ``сердца''. Ибо вы взяли энергию, которой умеете управлять, кусочек, не будем говорить о процентах, - кусочек, и распределяете из жизни в жизнь. В одной реинкарнации вы становитесь зверем и пользуетесь только нижней чакрой. А остальные будут задушены, ибо вся энергия ушла  на неё. Вы становитесь человеком, обладающим той же порцией энергии, но только уже распределяете  на сердце. И тогда, вы любите. Любите, и у вас мало ума и всего остального.}
\people{А не возможно ли совместить?}
\soul{Вы, сперва, возьмите. Возьмите эту порцию энергии, что взяли, распределите поровну. Распределите, и тогда не будет у вас разногласий. Тогда, сердце не будет кричать мозгу, а мозг - сердцу. И вы будете обладать единым взглядом, ибо они будут равны по силе. Тогда, уже можно будет сказать, что вы едины и сердцем и умом. Вы поняли?}
\people{Да, спасибо. }
\people{Ещё осмысливать придётся. }
\soul{И когда вы сможете распределить, в вашем понятии, поровну, чтоб не были обижены все, только тогда вы получите новые. Новые порции. И это вы будете называть ``новым рождением''.  И далее и далее. И снова, и снова - всё сначала. Вы получили порцию, и будете распределять где-то больше, где-то   меньше. Потому и говорим вам, что:  Да, есть другие миры, и, в вашем понятии, есть Рай. Но, подумайте, если был Рай, и там были святые - откуда же тогда пришли ``дьяволы''? Не те ли ``падшие ангелы''? Откуда взялись они? Вот вам – Библия! Подумайте логически. Рассудите, что такое Рай? Сборище ``святых''! Почему же тогда святые могут ``падать''? Ибо была получена ещё порция. Ещё была получена порция. И снова, и снова вы раскидали и обидели все чакры, оставив какую-то одну, более сильную. Вот вам и ``грехопадение''! Вот вам – ``иносказание''! Вот вам и ``яблоко''! И далее – вот вам и Змей(искуситель - библ. прим.) Спрашивайте.}
\soul{А вот что имеется в виду, когда говорят, что человек пошёл не по тому пути развития? Как это отражено в мифологии, библии? Не на это ли намекает рассказ о Каине и Авеле?}
\soul{Как вы понимаете – `` рождение''? Мы говорили вам только что о чакрах. Да. Просто, есть чакры, отвечающие за половые функции, отвечающие за желудок. И представьте, если желудочная чакра будет обладать большей энергией, кем вы станете? Кем? Хищником, пытающемся прокормить себя – и не больше! Вам не будет дела до остальных. Сердце, отвечающее, по вашим понятиям, за любовь, будет кричать вам. Но кричать тихо, ибо не будет обладать той энергией. Вот вам и - разность характеров! Вот вам и – ``разные люди''. А вы говорите: ``не то направление''! Да. Ошибка ваша в том, что вы назвали себя богом, не осознав. Вы возвеличили себя, поставили выше, хотя, оставались ещё животными.}
\people{Что стоит за библейским призывом к смирению и кроткости? }
\soul{Смирению и кроткости? Мы говорили вам: есть религия, есть вера, и есть бог. Если хотите,  религия – это та же политика. Та же политика. Мы ответили вам?}
\people{Понятно. Я хочу вернуться к предыдущему разговору. Значит, неспроста на востоке многие учения против такого понятия, как ``секс'', в нашем понятии, в западном? Для них, ``секс”- это, для  продолжение рода человеческого, и все учения говорят, о том, что это впустую просто тратить космическую силу, эту энергию.}
\soul{Вы называете это ``кундалини''. Подумайте. Мы говорили вам, что вы хотите в какую-то чакру подать энергии больше. И здесь вы входите в крайность тоже. Простите, той, нижней чакре, тоже нужна энергия. Тоже. Вы же пошли б без крайности, но не знаете середины и идёте до крайности. Вы теперь хотите оголить. Оголить эту чакру, оставить без энергии. И что будет? Что будет? Уже были попытки и что? Вымирали те цивилизации. Далее. Вы придумали ``плохое'' и ``хорошее''. Разум ваш постарался. Теперь подумайте, с какой целью вы делали то или то? С какими мыслями? Мысль, которые движет этим поступком – они должны отвечать, а не за сам поступок. Ибо…}
1-2…
\people{Сознание… Вот вы нам говорите, что не умеем пользоваться. Это наша ли в том вина? Природа ведь не могла сделать что-то не разумное. Почему это так получилось?}
\soul{Если б была не ваша вина, как тогда мог бы прийти бог к вам и говорить вам? }
\people{А вы считаете, что мы обладали тогда этими энергиями более чем сейчас?}
\soul{Давайте скажем так, - да, вы обладали. Но, как ребёнок, обладали, но не ценили и потеряли. А вы, чаще, цените только тогда, когда теряете. То же, что имеете, не имеет цены у вас. }
\people{Это, наверно, вопрос про Каина и Авеля? Авель, он олицетворял чисто такую… духовную сторону символизировал,  а Каин, желудочные чакры, так сказать. Вот нас это и убило.}
\soul{Ваша беда, что многие из вас, как Каин, хотят купить. Кого вы хотите купить? Что вы хотите купить? Мир, которому не нужна ваша физика, ибо он не имеет её? Вы, придя в тот мир, будете цениться не за свои заслуги или намерения, не за ваши богатства или пустые карманы. Вы будете цениться за то, что имеете то, что не можете назвать словами, что, в вашем понятии, говорите ``душа''. Но, если будете понимать правильно и говорить ``трудно войти в рай богатым'' и будете бедным… С какой целью вы будете бедным? Что бы войти в рай? Не войдёте. Не войдёте, ибо поставите цель и будете идти только вы, толкаясь локтями и создавая всё, лишь бы остаться бедными с этими целями. Мы же говорим вам, - вам надо жить. Жить. Жить! Не нарушать, в вашем понятии, Десять Заповедей. Надо жить, а не быть постоянным контролем и смотреть ``нарушил я или нет''. Это будет похоже на сделку.}
\people{Скажите, среди людей есть такие люди, которые живут по всем десяти христианским заповедям?}
\soul{Есть.}
\people{Они счастливы?}
\soul{Есть. Что б это был бы за мир, если б были только неугодные? Разве пришли б  тогда к вам?}
\people{Скажите, они в большинстве своём счастливы или гонимы обществом?}
\soul{Для посторонних наблюдателей, чаще, гонимы.}
\people{Понятно.}
\soul{Но, согласитесь, что у вас понятие счастья – разное. Вы можете быть счастливы, можете быть счастливы, а со стороны будете казаться - несчастным. Подумайте, ведь когда-то говорили вам о рабовладельческом строе, и доказывали, что и там можно было бы жить хорошо. Вы помните?}
\people{Помним, да.}
\soul{В ваших понятиях – не были рабы счастливыми. Не было ни одного. Да, была неволя, но была и любовь. Была любовь, значит, была и вера и счастье. Для посторонних – как вы были рабами, так и остались.}
\people{Понятно. Вот тут вопрос  возник: Что происходит,  когда девушка привораживает своего возлюбленного? Что это за воздействие такое?}
\soul{Воздействие? Это одна из попыток, одна из попыток ``купить''. Это одно из способов. Это то самое ``яблоко'', которое было искушено. ( в райском саду. прим.)}
\people{Но это не может одобряться. Это насилие над чужой волей?}
\soul{Да, вы знаете такую операцию. И вспомните. Вспомните историю начала грехопадения. Вспомните, с какой целью вы…}
 1-2-
\people{Что происходит с переводчиком? }
\soul{Вы очень интересны. С вами тяжело, но интересно работать. Вы непостоянны.  Вам нельзя предсказать, что будет так или так. Вы можете меняться. Меняться столь быстро, что нельзя предсказать вам что…}
1-2-3..
\soul{В том и сила ваша.}
\people{Вы говорите лично обо мне, или о всех людях?}
\soul{О всех вас.}
\people{Ага. Понятно. Ясно.}
\soul{Спрашивайте.}
\people{Ну, вы даже чем-то меня обрадовали, когда я задавал  этот вопрос. То примерно такой ответ я и имел в виду. Вот, скажите в этом плане, - одна из заповедей гласит: ``Не возжелай жены ближнего своего, ни имущества его'' и так далее. Как раз это и имелось в виду – не наводить порчу, не наводить сглаз? У нас такие понятия есть.  Даже просто, просто, пожелав даже добро, вроде бы, позавидовав слегка, - можно нанести вред этому человеку. Вот, в этом плане как? Сглаз и всё остальное. }
\soul{Будьте внимательны, там же сказано:  ``Даже  мыслями грешите'', тем более делами. Что вы делаете? Вы не знаете силы, которые хотите включить. Вы идёте и привораживаете, не зная даже, что привлекаете. Не знаете, чьими рабами становитесь. Далее. Сказано Христом, нельзя, нельзя идти на поводу ваших желаний. Нельзя. И не путайте с желанием прийти к богу. То - не желание.}
\people{Скажите, вот в  библии написано: - ``Почитай отца и мать, что бы продлить дни свои.''  Какая связь между этими словами? }
\soul{Вы подумайте. Подумайте, что раньше было? Какие были законы ваши? Вспомните. Вспомните,- оскорбящий мать или отца – был казнён. Здесь всё гораздо прозаически.  Он изгонялся. Изгонялся из города. Изгонялся в пустыню. Согласитесь, там труднее выжить. Вот откуда взялось понятие ваше. Есть и другое. В вашем понятии, иносказание. Вы меряете - длительностью жизни. Но, если это было сказано иносказательно, то почему бы вы тогда, что и длительность жизни – иносказательна? Почитая мать и отца, вы будете знать более, а значит, в вашем понятии, вы как бы дольше прожили. Вы поняли меня?}
\people{Угу. Понятно. А вот что даёт такой  призыв, как ``молитесь за проклинающих вас''?}
\soul{А вы подумайте. Вы повторили вопрос. Повторили.}
1-2-3..
\people{Скажите, что происходит с переводчиком?}
\soul{Защита. Защита мозга. Ибо видит то, что не может понять и потому…}
1-2-3…
\soul{Спрашивайте.}
\people{Может это быть потому, что у нас не присутствует Гера?}
(пауза)
\people{Тогда значит, всё-таки, получается из этого такой вопрос неожиданно: Так где ж хранится наша память? То есть, наша информация, которую мы знаем.}
\soul{Мы говорили вам, что ваш мозг, всего лишь ``ключ''. Ключ.}
\people{Или приёмник, другими словами. Да?}
\soul{Да.}
\people{А где же всё ж храниться? }
\soul{Везде. В каждом атоме храниться всё. Мы говорили о полях, и говорили о вещах. Простите, любая вещь имеет поле. И если вещь будет варьировать, - будет варьировать и поле. И это поле будет варьировать и на другие вещи. Вот вам и ваше доказательство, что в любом атоме записано всё о прошлом, о будущем и настоящем. В любом атоме, и даже менее. Всё и вся. Вы же, ваш же мозг – ключ. Ключ. Если хотите, машина, которая раскодирует.}
\people{Скажите, какое самое тяжёлое преступление может сотворить человек? Я имею в виду космические масштабы?}
\soul{Вы, нам когда-то задавали подобный вопрос. Мы вам отвечали. С какой целью задаёте? С какой? }
\people{Ну, не для того,  что бы сотворить. Просто знать градации…}
\soul{Подумайте! Подумайте! Вы скажете, - для того, чтобы не совершить это преступления. Узнаете это, и вы, где-то, скажете кому-то другому, другой – другому, и тот сделает то уже сознательно.}
\people{Хорошо, спасибо. В Спарте болезненных и слабых новорождённых бросали в море. Это должно было сохранить здоровье нации. Почему же всё-таки Спарта выродилась?}
\soul{А вы подумайте. И вспомните о чакрах. Вспомните их определение. Помните, что чаще – ``чем умнее, тем слабее''. Вспомните. И вспомните ваши пословицы. ``Сила есть – ума не надо''. Так  представьте себе общество сильных, но не имеющих ума. Что будет? Не вымрет ли оно?}
\people{То есть, забвение духовных ценностей…}
\soul{Поймите, вы стали человеком тогда, когда поняли, что слабого надо не убивать, а беречь! Беречь и пользоваться его знаниями. Тогда, вы стали человеком. Тогда, вы ушли от животных. Тогда.}
\people{Хорошо. А скажите, почему выполняют последнюю волю умирающего?}
\soul{То – ваше. То - ваш обычай.}
\people{А он имеет большой смысл для человека? Для людей.}
\soul{Всё имеет смысл. С какими мыслями будет твориться. Подумайте лучше. Человек – умирает… и он хочет… Он говорит тем – сокровенное и наболевшее. Конечно, ему угодно, чтоб то исполнилось. }
\people{Но это никак не связано с тем…}
\soul{Представьте теперь, человек умирает с проклятиями.  И что бывает? Проклятья. Куда попадёт он?}
\people{А он действительно попадёт не туда, куда…}
\soul{Вспомните Голгофу! Вспомните.}
\people{Скажите, вот сейчас, после публикаций, и моих лично, и других, о жизни после смерти, люди умирали. Наши знакомые. Они пытаются выйти на меня, на нас, и как-то так подсказать, что – ``Да, мы читали, мы не верили, или верили, а теперь можем подтвердить'' – были ли такие попытки? Вы не ``в курсе''? Может, я просто не слышу их обращений?}
\soul{Простите, когда вы приходите, скажем так в вашем понятии, аналогия – переезжаете  в другой город, - многое ли вы вспоминаете? Торопитесь, спешите, и далее и далее, - первое. Второе, - когда человек умирает и приходит в тот мир, то у него нет понятий `` Как же? Всё это есть, а я не верил''! Он просто пришёл и стал здесь жить. Потому он и не представляет, что когда-то не верил в это. Подумайте. Ребёнок родился со словами `` О! А я в это не верил!''? Нет. Он просто пришёл и живёт. Далее. Приходят, в вашем понятии, ``ИЗВНЕ'', те, кто находится в ``пограничной зоне''. Не пришедшие ТУДА, но уже ушедшие ОТСЮДА.}
\people{И, как правило, это люди не слишком высокого интеллекта, да?}
\soul{Почему же? Нет.}
\people{Но, вот оспаривается, мы вели диалоги с Игорем Тальковым. Оспаривается это. Хотя, переданы стихи его, вроде бы интересные.}
\soul{Поймите. Мы говорили вам о ``ключах''.  Ваш мозг - ``ключ''. Неужели вы думаете, что в прямом смысле – к вам приходят с ``того света''? Нет. Это, всего лишь, ваш ``ключ'' немножко открыл чью-то ячейку.}
\people{То есть, подключился к информационному полю, да? }
\people{Тогда, в таком плане, получается, после смерти Цоя появился человек, который был похож на него, имел ту же самую национальность и пел песни очень схожие с теми, которые пел Цой?  (Виктор Цой. Прим.)}
\soul{Ну, почему, почему вы говорите, что ``ко мне пришло что-то''? ``Ко мне пришёл другой, ко мне пришёл Будда''? Почему вы не можете подумать о себе? Что рождённое, всё-таки, вами. И что, всего лишь, в вашем понятии, было крещение. Вы должны в церковь прийти креститься, чтобы заново родиться, и чья-то смерть может быть для вас крещением, и вы заново родитесь и станете подражать кому-то.}
 1-2-3-4-5-6-7-8
\people{Почему не говорят плохо о мёртвых? Это имеет какой-то смысл?}
\soul{Нельзя говорить плохо о всех. О мертвых?  Потому, что он не может ответить вам. По вашим понятиям – ``дать сдачи''. И вы, пользуясь безнаказанностью, творите, что хотите. Вы же, в последнее время, взяли это в привычку.}
\people{Это плохая черта, да? Ясно.}
\soul{В вашем понятии - ``удар в спину''. В вашем понятии - попытка ``свалить'' на ушедшего. `` Он виноват! Он! А нам ещё жить!'' Это ваши слова.}
\people{Ну,  это может как-нибудь может сказаться на последующей его жизни в параллельных  мирах? То, что о нём говорят плохо или ещё что-нибудь.}
\soul{У вас была версия, что человек умирает, но душа его живёт до тех пор, пока хоть один о нём помнит. Была и такая религия.}
\people{Была,  была…}
\soul{Да. И мы говорили вам, что все религии имеют правдивые и лживые стороны.  Правдивые в том, что сказано. Да, - душа будет умирать, пока о ней будут помнить. А вы подумайте, ведь сказано было, что, сперва, ``было слово''.  Мысль творит всё. Вы же, вспоминая, даёте энергию. Даёте энергию ему или ей. }
\people{Расскажите тогда немножко поподробнее, два таких варианта;  один человек умирает и о нём никто не вспоминает, а о другом – вспоминают. Так как это сказывается на его последующее перевоплощение или, вы говорили,- параллельные жизни?}
\soul{Давайте так, - человек умер, и о нём не вспоминают. Первое,- он будет обладать малой энергией. Ибо, придя,(в иной мир. Прим.)  спросят его: ``Много ли ты принёс с собой''? А что принёс, если о нём никто не вспомнит? Ничего. Значит, пуста была его жизнь. Вы согласны?}
\people{Да.}
\soul{И подумайте, что станет с ним? Тогда он может уйти в более низшие планы. Или он может вернуться. Вернуться человеком. Но, как правило, это будет человек-одиночка. Человек, страдающий одиночеством в окружении всего. У вас много таких. Теперь представьте, человек умерший, но вспоминают его. И он придёт, и спросят его: ``Что взял с собой''? Но, о нём ведь могут вспоминать и хорошо и плохо. Подумайте. Тогда уже будут говорить: ``Ты принёс много, но каково качество''? И уже по качеству будут судить – куда. И здесь, то же самое, -  он может уйти и ниже, а может уйти и выше, в отличие от человека, не принесшего ничего. Или может остаться.}
\people{По религии, должно делаться что-то там Иисусом непосредственно - суд нам устраивать или ещё что-нибудь. А выходит, от нас всё зависит? Суд-то устраиваем мы сами?}
\soul{Подумайте. Подумайте и вспомните, что сказано: - ``Придёт судный день!'' А вы ждёте. Вы ждёте конкретного дня! Вы, каждый миг – осуждаете. Вы, каждый миг – судья. Вы говорите - ``то-то'' и ``то-то''. Разве не судьи вы? Судьи. Вы, даже, судьи над умершими, хотя, не имеете на то право. Потому и сказано вам: ``Не судите и судимы не будете''.}
\people{Скажите, какой механизм стоит за таким феноменом, как определение по фотографии жив или нет человек, изображённый на ней?}
\soul{Мы же сказали вам о духовном, и говорили об ауре. Вспомните. Любое изображение, рисунок, пускай это даже будут детские каракули, содержит в себе множество информации. Подумайте! Ребёнок рисует, в вашем понятии, каракули не несущие ничего, но, мысль двигала его! Согласны? И это, всего лишь ваше не понятие, что нарисовано им. А вы поставьте эксперимент. Пусть ребёнок нарисует любое, что вы ему скажете. И дайте другому. И тот назовёт то же. Хотя, вы не увидите сходства. Другой же ребёнок найдёт. Вот и подумайте, умели ли вы когда-то, в вашем понятии, ``мыслить''?}
 1-2-3-4-5-6-7
\people{Именно этим, наверно,  определяется полезность картины, её отдача? Есть картины, которые ``сосут'' энергию, а есть картины, которые одухотворяют, придают энергию. Мы правильно понимаем? Есть такие различия?}
\soul{Как вы понимаете - правильно? Вы не понимаете. Если картина нарисована, в вашем понятии, хорошим человеком и к нему придёт плохой человек,  что скажет о той картине? Он будет говорить же самое – ``сосёт''. Вот и подумайте. }
\people{А вот, картина, допустим, известнейшего нашего советского художника, русского , ``Мистерия''. Я слышал о ней, что она очень ``сосущая'' такая. Энергетическим вампиризмом обладает. Это действительно так?}
\soul{Мы вам ответили только что.}
\people{А. Ну, то есть, по-разному действует.}
\soul{Подумайте. Вы  можете подойти к картине, нарисованной, по вашим понятиям, святым. Но, если вы грешен, вы не воспримите её и вы будете говорит, как о сосущей. Сейчас. А представьте человека, который не знает понятия ``сосать энергию'' или ``отдавать''. Что он скажет? Он сделает проще. Он скажет: Бесовщина. Или что-то другое. Он назовёт то же самое другими словами. Будет время, когда, в вашем понятии, ``отдать'' или ``взять энергию'', изменится, и вы уже будете говорить по-другому, более близкими к истине.}
\people{А определять по фотографии,  жив человек или уже умер, могут только экстрасенсы или  вообще, могли бы и другие?}
\soul{Нет. Это может любой. Любой. Вы все обладаете, и вас пронизывает одна и та же энергия. Что у экстрасенса, что у вас, что у младенца. Разница только в том, что каждый управляет больше или меньше. И вспомните, вспомните себя, разве не было у вас, когда вы смотрите на фотографию и думаете – жив человек? Что вас заставило задать тот вопрос?}
\people{У меня не получалось это.  Всегда хочется думать, что человек ещё жив. Мы гадали по фотографии, я, по крайней мере, и я не мог почувствовать. Так, ну ладно.}
\people{В связи с этим, возник вот такой вопрос. Сейчас наши учёные подошли к тому, что действительно стали по-настоящему задумываться, что мысль материальна и даже её пытаются искать. У них даже есть какие-то подвижки, какое-то материальное что-то. Идут разговоры о лептонах, как о носителе мысли. А другие настаивают, что всё-таки, свет, одно из разновидностей света – красные лучи там, и есть носитель. Всё-таки скажите, что вернее? Где больше искать истины?}
\people{Давайте скажем так, что если вам кварки, ультра, короткие и иное,  то вы уже делите. Вы уже делите. Тогда, что-то другое было рождено не мыслью? Спрашивайте. И вы же спрашивали нас, что такое - ``слово''.  Душа пришла в мир. Со всеми вашими полями, материей. Мы же говорим вам, что в каждом, в каждом – находится всё и вся. Вот и подумайте. Зачем вам говорить конкретно, о каких-то о частотах? Этим – вы уже делите. Делите. Ваш глаз уже поделил и видит столь малое! Вы же, из столь малого, хотите ещё, ещё делить, делить и делить. И что останется? Вы хотите увидеть через приборы, не доверяя себе. А вами же было сказано, сердце, сердце – лучший прибор! Вы же, хотите заменить его. }
\people{Я опять о фотографии.  Смотрите, на какой день уже можно определить о смерти человека?}
\soul{В вашем понятии, мгновенно.}
\people{Но, я читал… }
\people{Что на сороковой – лучше всего.}
\soul{Мгновенно. Но, согласитесь, ничто не исчезает ``сразу'', в вашем понятии. Ничто.  И вы, всего лишь, можете определить это через какой-то определённый срок, когда, в вашем понятии, энергия ауры истощается до той степени, когда  вы уже можете это заметить. Хотя, вы бы могли заметить это и без фотографии. Мгновенно. Когда человек не только умер или только умирает, или только будет умирать. Когда механизм смерти уже запущен. Запущен. И, к сожалению, чаще всего, он запускается в момент рождения.}
\people{С момента рождения? То есть, предопределена его судьба? Окончание? Интересно…}
\soul{Всё зависит от вас. Мы вам говорили о времени. Вы бы могли заметить, что, в каждом из вас, время разное. Вспомните время ожидания, время нетерпения, время любви? Всё разное. И вспомните, мы говорили вам, вы спрашивали о родителях. Чем больше знаете, значит, тем больше прожили. Вы же говорите о сроках, говорите о сроках:  50, 70… и далее. Но, подумайте! Можно прожить ваши 70,  и остаться глупцом. А можно прожить и 10, но стать мудрецом.}
 1-2-3
\people{Но, здесь есть ещё разговор о карме? Непосредственно – рождается человек, умирает. Как бы отрабатывает свою карму.}
\soul{То не отрабатывает. То, всего лишь причина. Когда-то вы начали, и ваше рождение и смерть – это всего лишь попытка, попытка напомнить вам, что делаете. Попытка остановить вас, если идёте не верно.  И когда вы рождаетесь,  вы забываете и не помните, но не можете понять что. И живёте, живёте. И это, всего лишь, бег причин. }
\people{Скажите, а есть ли разница – умер человек сам, от старости или болезни, или его убили, если это определять по фотографии?}
\soul{Да.}
\people{Это заметно экстрасенсам?}
\soul{Ну, вы подумайте. Мы же сказали вам - в каждом  атоме. В каждом атоме есть информация. Даже ауре, вмещающей в себя всё. Вы же можете по ауре определять характер человека. Многие могут определять болезни и, даже, определять будущее. Тем более, если наступает момент, когда кто-то пытается уничтожить вас.}
 1-2
\soul{Спрашивайте.}
\people{Скажите, вот нам кажется, что если включатся парапсихологические возможности, экстрасенсорные, то вполне можно и с преступностью покончить. То есть, человек будет знать, что он обязательно раскрываем будет.}
\soul{Нет… Просто, у вас будет другая преступность. В психических планах. Только и всего. Ибо мы говорили вам о грехопадении. Вспомните.  Да, сейчас вы скрываетесь только физически. Дальше – через время, - вы будете это делать психически. Согласитесь, что если вы обладаете энергией, вы её можете повернуть и на `` чёрное'' и на `` белое''.}
\people{Ну, вот, если бы человечество бы выработало элементы того, что преступление всегда наказуемо, не только на ``небесах”…}
\soul{Потому вам и не даётся. Потому и не знаете всё, ибо, не умеете управлять, по той простой причине, что вы можете столько бед натворить! И если вы, будете не подготовлены и получите, что будете делать? Представьте, что вы сейчас обладаете способностями. Представьте, что вы будете делать? А вы уверены, что вы не будете использовать  их на выгоду себе?}
\people{Хм… Да-а…}
\soul{Ну, и подумайте, в чем разница, если этим будут обладать все? Просто, в вашем понятии, будут более хитрые. Более на другом уровне, только и всего. }
\people{Но преступность, это что? Обязательный ли элемент человеческой цивилизации или это какой-то атавистический?}
\soul{Идёт борьба ``светлых'' и ``чёрных''.}
\people{“Чёрные'' это обязательно преступные замыслы, да?}
\soul{В вашем понятии, да.}
\people{Но, может быть преступник и со светлыми замыслами?}
\soul{Ну, подумайте. Называется ``чёрным'' то, что только себе, себе и себе.}
\people{Вредит? А! ``Себе и себе'', – тянет на себя! Ясно. А скажите, ``благими намерениями путь в ад выстлан'', как вообще понять это в таком случае?}
\soul{Вам привести пример? Ну, давайте представим, - к вам пришли и говорят:- давайте своруем то-то и то-то. Есть хорошая возможность ``подработать''. Что вы будете делать? }
\people{Ну, да… Благие намерения улучшить своё положение приведёт… }
\people{Ну, явно, не это имелось в виду.}
\soul{Нет. Подумайте дальше, что вы будете делать? Вы скажете: ``Нет, не пойду''. А вы идите!}
\people{Как?!}
\soul{Подумайте.  Вот - благое намерение! Вы бы пошли, но не остановили. Далее. Возьмемте другой пример. Вы идёте, и нищий просит подаяния. Вы отдаёте ему? Как называете это? Хорошо вы сделали или плохо?}
\people{Считается богоугодно, хотя, не каждый нищий вызывает положительные эмоции. Есть – спекулируют на этом. }
\people{Я всё время подаю. Сегодня вот, буквально, десять рублей отдал.}
\people{Как вы оцениваете; вот – человек всё время подаёт, а я, наоборот, избегаю.}
\soul{Подумайте. Мы приведём вам грубый, грубый пример. Тонкий вы найдёте сами. Представьте, вы подали, а ему не хватало, всего лишь, в вашем понятии, ``выпить''. Вот, ваши деньги пошли на то. И человек…}
\soul{1-2-3-4-5-6-7}
\soul{Вы поняли? }
 
\people{Мы поняли. Но в библии явно не эти примеры приводились. Там приводились примеры…}
\soul{У вас называется ещё это – ``мысли вслух''.  То же самое, когда вы понятием, - своим понятием, подумайте, СВОИМ понятием о благе,- творите. Но, вы уверены, что ваше понятие верно? Вы уверены ли в том? И вспомните. Вспомните историю, об ангеле. Вспомните. Когда приходил он, и убивал. И когда спросили его: ``Что делаешь?'', -  что сказал он? Вспомните! }
\people{Закончите мысль сами.}
\soul{Вы не помните… Ибо, вы не знаете ни прошлого, ни настоящего, ни будущего. ``И был убит ребёнок ангелом. И было сказано: ``Ибо, вырастет он, и будет, в вашем понятии, преступником”''.   Но, сейчас, он же сделал, для вас, преступление. Вы согласны? }
\people{Да.}
\soul{Вы не знаете будущего и у вас понятие своё о благородстве. Подумайте, какое благородство сейчас у вас, какое оно было ранее и ране и ранее? Вы же сами говорили, благородным считалось убить старого и бессильного. Сейчас считаете вы это благородным? И теперь представьте, - то, что вы спросили, вы спросите в те времена, когда благородство было иным.}
\people{Но, сейчас мы убедились, что вы сейчас явно процитировали библию, не  подчерпнув это из знаний переводчика. Поскольку он говорил, что библию не читал, по крайней мере, только несколько глав осилил. Значит, получается, что вы с нами говорите, не затрагивая знания переводчика. Так?}
\soul{Давайте так, - пусть, задавший вопрос, всё-таки вспомнит, читал он или не читал.}
\people{Я не читал.  Переводчик треть библии прочитал.}
\people{А кто задавал вопрос, я просматривал библию, но всю не прочитал, конечно, но такого места я что-то не припомню там. Где  оно есть-то. }
\people{Так, ну, мы вам не задали вопрос наверно? Вот вы нам и не отвечаете? Да?}
\soul{Спрашивайте.}
\people{Скажите, а вот, будут ли поддерживаться семейные узы. Она против аскетизма…}
 1-2- 
\people{… И, всё же, многие христиане добровольно уходят в монастыри, отказываясь от связи даже со своими родственниками. Так, например, аскетами были некоторые православные святые. Здесь есть какая-то связь? Одобряется ли она в вашем мире?}
\soul{Нет! Мы говорили вам, и вы уже спрашивали сегодня.  Мы говорили вам, если вы пришли в этот мир, то живите в этом мире, и боритесь в этом мире. Вы же, пытаетесь уйти. Многие, вашем понятии - аскеты,  уходят от трусости, ибо не знают, что делают. Но мы и говорили вам,  если в каждом -  бог, а вы хотите видеть этих каждых, значит, не хотите видеть Бога! Бога, который проявлен во всём! Вот вам. (Истина.)}
\people{Но, именно их возвеличивает и церковь и духовенство. И им дают даже сан посвящённости. Значит, они, получается, шли праведным путём?}
\soul{Вы всегда возвеличиваете то, что не понято вами. То, что говорите – ``нужно обладать силой.''  Вы же… Вам легче признать кого-то героем только ради того, что он ушёл и живёт один. А тот, кто живёт и мучается вами, страдает вами, помогает вам, вы не называете героем, а чаще, называете наоборот.  }
\people{Хорошо. Скажите, что стоит за выражением библейским – ``Браки заключаются на небесах''? Это действительно так?}
\soul{Иносказательно, да.  Здесь говориться о любви. Об истинной любви, а не о  вашем понятии. Подумайте.  Многие могут любить, не заключая браков ваших. Вспомните. Неразделённая любовь. Вот вам и иносказательность, вот вам и ``Небеса''! Ибо ``Небеса'' вы называете…. Вспомните ранее и возьмите словари. Вспомните, что называлось Небом? Вспомните! Ранее, называлась небом – ВАША Душа!}
\people{А скажите, это означает, что человек должен искать упорно свою половинку, так выполняя брак на небесах, или же это не самое важное?}
\soul{В том и беда ваша, что чем больше ищете, тем больше мешаете. Чем ищете? Вы ищете сознанием. Мы же говорим вам, сознание ваше мало. Вспомните! Мы вам сейчас сказали вспомнить и вы мучаетесь и не можете этого сделать, и вспоминаете только тогда, когда не мучаете своё сознание и не пытаетесь вспоминать. Вот вам и ответ, чем больше вы будете искать, тем меньше шансов, что вы найдёте.}
\people{Ну, здесь есть связь с тем, что многие учёные сделали свои открытия во сне?}
\soul{Да. Ибо сознание отдыхает. Подумайте. Вы скажете, к вам пришли и дали во снах. Мы же говорим вам, подобное – подобным.  Вы говорите, что человек, будучи в сознании, мучается этой задачей, или нашёл ответ и не может сформулировать. И подсознание уже выдаёт сформулированный ответ. Да. Нет, здесь другой мир. Но вы забываете, что вы ``ключ''. Вы ``ключ''. Ключ ко всему, что находится ``извне''. И ваш ключ более открывает, когда сознание спит.}
\people{Примерно, то же самое, что происходит сейчас во время контакта. Я прав?}
\soul{Если хотите, то - да.}
\people{Скажите, если уж мы о разводах начали и о браках. Ваше мнение о разводам на основании измен и вообще.  Как вы относитесь к разводам? Как к положительным факторам или отрицательным для человека?}
\soul{Неужели можно сказать конкретно ``да''  или ``нет''?}
\people{Кому-то ``да'', а кому-то, может и `` нет''? Да?}
\soul{Подумайте, - вы разводитесь… Значит, вы допускаете, что вы ошиблись. Правильно?}
\people{Да.}
\soul{Но, чаще, вы обвиняете не себя. Значит, ошиблись не вы, а другой. Согласны?	}
\people{Да. Согласен.}
\soul{В том и беда. Вот в чём несчастье ваше. Вот в чём грех ваш. Что вы считаете себя обиженным, считаете обманутым, и потому разводитесь, не заметив, что в том и вина есть ваша.}
\people{Но, бывает, жить не возможно с человеком. Он тебя не понимает полностью или изменяет. }
\soul{Простите, а как вы раньше не знали того?  }
\people{В каком смысле?}
\soul{Вы ошиблись. Не  заметили? Не поняли?}
\people{В каком смысле?}
\people{До свадьбы.}
\people{А-а… До свадьбы! Да?}
\soul{Вот вам и `` благими намерениями''! Вот вам и незнание будущего. Вот вам и неумение управлять. И вы никогда, или редко, сможете допустить, что в вашем поведении виноваты вы. Вы тоже виноваты в том разводе. Потому и говорят вам: `` не разводитесь''. Ибо, имеете – живите. Живите дружно, но, не мучая друг друга. Если же не можете жить и не мучать, то лучше уйдите. Уйдите, и тогда это не будет грехом, что вы ушли. Грех только в том, что вы ранее не заметили и не обвинили себя.}
\people{Да. Наверно. А вот, скажите…}
1-2-3-4…
\soul{Спрашивайте.}
\people{* Скажите, вот наш общий друг, Гера, сейчас, в общем-то, на грани свадьбы, но у него есть сомнения. Вы можете подсказать что либо, по его судьбе?}
\soul{Когда-то мы говорили вам: Если сомневаешься - нельзя говорить о любви.}
\people{То есть, эти слова, желательно, чтобы Гера узнал. Да. Но бывает…}
\soul{Он слышал их. Он слышал их… Слышал не раз, и называл сам. И подумайте, что приведёт, что приведёт ``нелюбовь''? Вы говорите: ``Привыкнется - слюбится''. В том и ваша ошибка, что вы пытаетесь привыкнуть друг к другу. И чем больше вы пытаетесь это сделать, тем больше не любите.}
\people{То есть, это можно ожидать и от его свадьбы? Очередной…}
\soul{Теперь подумайте. Мы сейчас скажем ``да, ожидается'', - и всё, - мы дали установку. И тогда обязательно это случится. Ибо он, уже где-то внутри себя скажет: ``Это должно случиться'' и будет идти к тому.}
\people{Ну, он сейчас… Он как в обязательность… Он как обещал и…}
\soul{Поймите меня. Вы постоянно находитесь в гипнозе, самогипнозе… самообмане… вспомните… в обмане… Вы даёте себе установки всегда и постоянно. И вы выполняете их, даже не зная о том. Вот вам и ``рабы''.}
\people{Скажите, пожалуйста…}
 1-2-3-4
\soul{Спрашивайте.}
\people{Как вы относитесь к эмансипации женщин?}
\soul{Простите, это уже ваши проблемы. Вы всегда входите в крайность. Почему вы не можете жить, жить, не попирая? Почему вам всегда нужны ``верхи'' и ``низы''? Почему? Этим, вы уже…}
\people{Понятно. А как вы относитесь…}
 1-2-3-4-5-6-7-8-9
\soul{Спрашивайте.}
\people{А как вы относитесь к сексу несовершеннолетних девушек и парней, или к незамужним девушкам и сексу? Есть осуждение, или вы считаете это вполне нормальное явление?}
\soul{Простите, это ваши проблемы. Далее… У нас нет, нет вашего физического мира. Потому у нас другие понятия.}
\people{Но мы задаём этот вопрос, не имея в виду вас, а имея ввиду себя.}
\soul{Тогда вы должны спросить у себя и у всего человечества. Ибо, вы живёте  в этом мире, и этот мир должен отвечать вам. И если мы скажем вам, то скажем от имени вашего мира. Но, подумайте… В вашей стране… в другой стране… другой… И везде ответы будут разными.}
\people{Ну, а со стороны, так сказать ``космоса'', всё-таки, на эту проблему взглянуть? На последующую жизнь и реинкарнацию… Или в параллельный мир когда он будет уходить, то - может сказаться как-то?}
\soul{Подумайте. Если любовь ваша - всего лишь только ``секс'', то, простите, ничего хорошего в том нет.}
\people{Это мы и хотели услышать.}
\soul{И вспомните, вспомните, - ранее, был ли ранний секс? Раньше вы были ближе духовно. Раньше, влюблённые думали не о сексе. Подумайте. вы же, сейчас, говорите ``ускорение'' и далее и далее… Разве? А может - просто унижение чувств? Вы говорите - ``акселерация'', вы говорите - ``ускорение''… Быть может, просто, разучились? Разучились чувствовать, разучились любить? }
 1-2-3
\soul{Спрашивайте.}
\people{Хорошо. Спасибо. Скорее всего, так оно и есть. Вы правильно оцениваете, - мы разучились любить, мы слишком животными чувствами живём.  Скажите, какое место в мироздании в нашей жизни занимает эфир? Древние называли его основой мироздания, Энштейн - ``мозгом бога''. Современные учёные считают, что он может наращивать материю, увеличивая массу атома, то есть, - ``эфирный ветер''. как вы относитесь…}
\soul{Вспомните, мы говорили об иерархии. Вспомните, мы говорили вам о силе, говорили о проматерии… Эфир - проматерия.}
\people{Так оно и есть? Эфир, это проматерия, да?}
\soul{Да. Но есть и выше. Вспомните иерархию. Есть и силы, которые родились… Вы же сами спрашивали нас. Сперва, было слово, и лишь потом была материя.}
\people{Ну, а как относиться к высказыванию Энштейна на этот счёт, что это ``мозг бога''? Проматерия содержащая всё. Вы подумайте,подумайте, если мы говорили вам, что в каждом атоме есть обо всём… Пусть будет ``мозг''. Пусть будет так. Подумайте и возьмите в аналогию приведите себя,- у вас есть мозг? Есть. Но вы подозреваете, что вы, всё-таки, работаете не мозгом.}
\people{Да, после сегодняшнего разговора и тем более.}
\people{Так… Почему наша Земля приплюснута по полюсам? Это эфирный ветер влияет, или тяжесть льдов, или что-либо? Правда, что наша Земля была меньше в размерах и материки касались друг друга?}
\soul{Подумайте. Подумайте, но касались они не от той причины, что была меньше, нет. Мы говорили вам, что для многих, ваша планета - солнце. А где-то в других параллельных мирах она обладает уже столь огромной массой, что стала солнцем. Где-то - дыра(чёрная дыра. прим.), а где-то - яблоко. Адамово яблоко. Н-да… Когда-то была малой и растёт. И не мне вам доказывать. Вы же знаете это сами.}
\people{Угу. А приплюснутость - это обязательное условие или это случайное совпадение какое-то?}
\soul{Нет. Это, всего лишь, вращение.}
\people{Вращение, да?}
 1-2-3
\people{Мы хотим, в принципе, скоро закончить. Мы благодарим вас, конечно, за сегодняшний вечер. Были интересные ответы. Но вот попробуем ещё одно… Одно время ученые считали, что межзвёздное пространство - чистый вакуум. То что это не так стало понятно после такого представления, что в вакууме не могли бы распространяться никакие волны. Если говорить о мысли, как о средстве чуть ли не передвижения, источнике информации обо всём, то что же лежит в основе мысли, среде обитания? Возможен ли здесь разговор аналогичный разговору о вакууме, который заставил бы серьёзнее задуматься о наших мыслях, их реальности в материальном плане? Что бы не оставалось у нас сомнений, чтобы не воспользовались ``чёрными'' мыслями для ухудшения обстановки на Земле, в параллельных, во всех мирах.}
\soul{Давайте скажем так… Природа боится пустоты. Нигде, ничто вы никогда не найдёте пустое, чтобы ничто не находилось там. Да, одно из доказательств ``непустоты'' - распространение радиоволн. Здесь вы правы. Пустого нет нигде. Везде есть жизнь. Далее, вы говорили о проматерии. Вот вам, пожалуйста. Вы называете их ``лептоны'' и многое-многое разное… И всегда, вспомните, даже в древней религии всегда вспоминалась, в вашем понятии, проматерия. Вы же, если хотите, то вы состоите из той же материи, только более организованной. Проматерия пронизывает её энергией. Энергия названа ``силой''. Сила воздействует на материю…Извините, - праматерию. И создаёт материю. Материю, в вашем понятии - подчиняющуюся таблице Менделеева. И осталось немножко, чтобы создать жизнь. Осталось лишь только немножко, это - взять ``время''. Ибо, если есть проматерия, то есть понятие ``пространство'', и есть понятие о ``времени''. У вас когда-то была теория - ``Материя, это искривление пространства и времени''. Мы же говорили вам, что все теории ваши верны. Возьмите и сложите всё, и будет вам ответ. Далее. Вы говорите…}
\people{}
1-2-3-4…
(конец записи)
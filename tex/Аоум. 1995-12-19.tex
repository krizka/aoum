Аоум. глава 29-я 19-12-1995г
Георгий Губин
 
 19-12-1995г  
\people{(Белимов) Сегодня 19 декабря 1995 года, мы делаем очередной сеанс. Вы слышите нас?}
\soul{(Чужой)Здравствуйте…}
\people{Здравствуйте!}
\soul{(Чужой) Спрашивайте, мы будем отвечать вам.}
\people{(Белимов) Но переводчик ведёт себя необычно, это ему не помешает?}
\soul{(Чужой) Спрашивайте.}
\people{(Белимов) Хорошо. Мы первый вопрос задаём…}
\soul{Чужой!}
\people{Кто чужой?}
\people{(Ольга) Идёт чужой, или здесь кто-то чужой?}
\people{(Белимов) Скажите, в прошлый раз ``переводчик'' вёл себя необычно, он застыл после того, как мы задали мысленный вопрос. Мы не правильно как-то повели себя?  Объясните эту ситуацию. Как надо было себя вести, вдруг она повторится?}
\soul{Что было сейчас?}
\people{Сейчас было, долгое молчание переводчика и непонятная его реакция. }
\soul{Вы же слышали, ``Чужой!”}
\people{Да, слышали.}
\soul{И не только мы можем прийти к вам.}
\people{Ааа… }
\people{То есть, это какая-то другая сила на нас …?}
\soul{Вы должны были бы помнить, ибо мы когда-то говорили вам, что никогда не здороваемся с вами.}
\people{(Гера) Да, кстати. Хотел спросить.}
\people{(Белимов) Так, а на сей раз вы поздоровались. Что это означает?}
\soul{Мы ли? Вы не внимательны.}
\people{Ахх.}
\people{(Ольга) А чужой- это кто?}
\people{(Белимов Ольге) Ну, другая сила. }
\soul{Есть множество, множество желающих разговаривать с вами. Множество хотят придти к вам помочь по доброте или по глупости, или навредить. Хотя это одно и то же.}
\people{(Гера) А вы по доброте нам помогаете?}
\soul{Мы отвечаем на ваши вопросы, и мы говорили вам, что вы отвечаете на них. Вы помните?}
\people{(Гера) Да.}
\people{(Белимов) А скажите, а как вам удалось сейчас чужого убрать, нейтрализовать или попросить его уйти?}
\soul{Страх. Страх ``переводчика''.}
\people{Нам вообще-то ``чужие'' не нужны. Мы к вам привыкли и хотели бы, действительно, продолжать общение именно с вами.}
\people{Скажите, как в прошлый раз надо было закончить сеанс? Почему ``переводчик'' долго не выходил из своего состояния?}
\soul{Ошибка переводчика. Он желал вмешаться, он желал проверить свои способности. }
\people{Ага. Но при этом надо было как-то помочь ему. Счёт провести с 11 до 19-ти, но мы это не делали. Мы забыли это.}
\soul{Вы многое забыли.}
\people{Это вас разочаровывает, да?}
\soul{Мы говорили вам, что когда-то покинем вас. Мы будем вынуждены сделать это,  для того чтобы вы остановились, и у вас было время подумать. Хотя у вас всегда есть время, но вы не пользуетесь им.}
\people{Пока вы нас не покинули, мы хотим задавать и продолжать задавать вопросы. Их у нас много.}
\people{(Ольга) Скажите, может быть, моя реакция, когда мысленный вопрос задавала, была негативной?}
\soul{Переводчик. Мы же, не имеем права вмешиваться, как в ваши, так и в его действия. Вы можете увидеть нас только реакциями, реакциями поведения вашего, переводчика. Сами же, не хотим и не будем.}
\people{У нас сегодня вопросы подготовили: отношение между людьми, отношение между полами и некоторые медицинские вопросы, по медицине. }
\people{Первый вопрос такой. Необходимо ли подавлять в себе эмоции, желания которые не вредят другим людям?}
\soul{Нет. Вы должны жить, а не подавлять себя, иначе - вы посадили себя в клетку, иначе - вы не хотите идти. И что заставляет давить вас?  Что? - Разум ваш, ум. Мы же говорили вам, что эмоции ваши, чаще, - чище. Вы же скажете: ``Есть множество эмоции отрицательных''. Что создало их? - Ум ваш. Много ли искренности в ваших эмоциях, особенно в тех которые вы хотите подавить? В тех, подавленных, больше честности, чем в тех, в которых вы играете, которые вы создаёте для окружающих, и чтобы обмануть себя. }
\people{Скажите,  на востоке, есть такое выражение: ``убей желание'', то есть, каким-то образом убивая в себе  эмоциональные реакции, желания,  человек может достигнуть какого-то…?}
\soul{Тогда вспомните, сколько вы сегодня уже убили. Вы это делаете постоянно. Вы убиваете самих себя. ``Желание желанию - рознь'' - скажете вы. Нет, мысль - мысли – рознь''. Чаще, мысль рождает ваши желания. В тех же чувствах, которые приходят изнутри, именно от человека, а не от плоти вашей, те чувства, очень легко подавить, ибо они слабы. И чаще, вы их давите, даже не замечая этого. }
 это ваше вдохновение. Вы сейчас найдёте множество причин, чтобы отказаться от этого. Вам не хватает времени, или что-то ещё, и неудобно, и многое-многое. Лишь бы только не хотеть. Те же, как вы говорите, прихоти, от плоти вашей, вы подавляете гораздо реже. Вы, просто, их перекрашиваете, перекрашиваете в другие цвета и желания о чём-то превращается в другое, но всё - то же самое. Вот вам и обман. 
\people{Вообще, как отличить желания плоти от желаний мысли?}
\soul{Этому вы должны учиться. Этому никто не научит вас. Даже Бог, придя к вам, не сможет помочь вам, ибо вы будете глухи и слепы, а мы уже говорили вам, мы говорили вам, что вы дети. Но вы дети, но, вы достаточно взрослые, чтоб отвечать за себя. Вы же, превращаетесь в младенца: ``я не виноват. Господь или ещё кто-то. Дьявол – попутал! При чём здесь я? Я ещё маленький'' -  ваше… Ваше. Вот вам одно из подавлений чувств. И вот вам, наоборот, одно из желаний: превратиться в маленького и незаметненького, и чтобы кто-то за вас думал и совершал, а вы будете только плыть по течению. А если вы будет говорить: ``нет, я иду против'' - это одно из желаний обмануть себя, ибо вы выбираете просто одно из течений, которое вам по пути. Только и всего. }
 Вы берёте книги и читаете меж строк. И всё зависит от настроения, что прочитали вы. Плохое настроение – и книга плоха. Хорошее – и книга иначе. Так где же тогда истина в книге той? Где? Тогда вы скажете, что все книги ложны или истинны. И там и там вы будете правы. Ибо всё, что говорится или слышится вами, должно быть переработано вами. Вами же принято или отказано. Чаще, вы это делаете только умом, разумом, логикой. Всё, что не подходит вам – требует  доказательства. Вы не принимаете тех истин и не хотите принять их, которые не могут быть доказаны. 
\people{Часто ли встречается сейчас удержание сущностями других планов или развоплощёнными людьми? }
\soul{Все, все вы одержимы какими-то идеями. Вы привыкли к словам, привыкли к их содержанию, к их обозначениям. И не можете догадаться, расширить их. Все вы одержимы, все. У вас есть идеи, и даже сейчас, в разговоре с нами, вы одержимы, ибо, вы хотите задать множество вопросов и получить ответы. И в то же время, мы можем сказать вам, что нет среди вас одержимых,- это нормально. Иначе, если б у вас не было желаний – не было б ничего. Если вы хотите знать точный ответ об одержимости, так и скажите: ``одержимость – что называет церковь''. Тогда мы сможем вам сказать именно это. Всё остальное – (перебивают)}
\people{Да, что это такое? Одержание другими…бесами. Именно с церковной точки зрения. С религиозной.}
\soul{Чаще, чаще всего, не происходит бес к вам. Чаще всего, вы хотите играть в чужую игру. Если вы неудачник, то есть 2 пути одержимости. Или считать себя счастливчиком, или полностью неудачником. Полностью неудачником – гораздо проще, согласитесь. И тогда, вы создаёте свой имидж. Лишь только церковь придумала одержимость. Нет, это желание. Ваше желание, ваша лень, ваш страх. Когда вы превращаетесь в ребёнка, которому позволено всё. Тогда все ваши чувства выходят наружу и тут же выполнять все ваши прихоти. Вот вам одержимость, вот вам и ``вселился бес''. Чаще всего, происходит именно так. Именно так. Посмотрите, и будьте внимательны: человек, потерявший или (теряется)}
\people{Вы сказали о том, что он получил опьянение,  наркотик, наркотическое, возможно. }
\soul{Нет. Мы хотим сказать, что эти люди тоже похожи, в вашем понятии, на одержимых, ибо они не контролируют себя, у них нет границ. Вот вам и одержимость. Реже, когда действительно приходит к вам что-то инородное и чужое, - это столь редко и столь редко встречающееся, и столь редко замечаемое вами, что об этом, пожалуй, не стоит говорить. Потому что вы выдумываете. Вы должны найти себе оправдание, что натворили не вы, а какой-то бес. Множество, множество, но выдуманное вами, именно вами, или что-то непонятое, или просто разукрашенное, или увеличенное временем. Ибо вы желаете, и всегда лжёте. Что-то доказывая кому-то другому, вы всегда будете приводить факты, которых не было. И будете приводить до такой степени, что уже сами уверуете в них, уж тем более другие. И лишь что-то непонятое, неверно сделанное вами - и вы это увеличиваете, увеличиваете и превращаете в религию. Любое учение, ставшее религией, а вы слышали это часто, становится никчемным. Ибо, подумайте сами, быть с Богом и говорить об учении?  Согласитесь, что это глупо. Глупо, ибо, будучи у Бога, вам не нужно будет ничему учиться, и ничего не нужно будет знать, потому что вы будете жить во всём этом. Вы будете во всём этом жить! Понимаете?! Вы же, только существуете. Вы знаете множество миров, что пронизывает вас. Ну и что? Вы же не видите ни одного из них. Вы даже не видите тот мир, в котором вы живёте. Вы даже не знаете себя. Вы не знаете свой мир, который внутри вас. Вы приходите и спрашиваете: ``скажите о моём здоровье. Скажите, кто я?'', и,  даже, просите характеристику на себя! Не является ли это признаком вашей слепоты?}
\people{Конечно, является. }
\soul{Вот вам - одно из одержимостей. Человек, приходящий в церковь и молящий, молящий, молящий и забывший, что у него мать лежит больная без присмотра… а ему некогда, ибо ему надо, утром пойти помолиться, потом собрать пожертвования для монастыря. И он говорит о вере в Бога!? Вот вам - одержимость. Поймите, Сатана не придёт открыто к вам. Он примет облик Бога и будет внушать вам прекрасные мысли, прекрасные, но поступки будут другими.}
\people{(Гера) Значит, мысль не должна отделяться от дела? Она должна быть единой, в выражении себя, в этом мире?}
 
\soul{Представьте, что все ваши мысли, - все будут исполняться реально. Что будет? Подумайте. Сколько вы сегодня всего надумали, и если бы это исполнилось. Что было бы? Что? Научитесь, сперва, управлять мыслью, а потом уже - всё остальное. Если вы научитесь и уберёте хаос с головы, уже всё остальное будет для вас просто. Очень просто. Вы говорите, вы не можете лечь и разговаривать с нами. Причина? Страх. Страх хаоса. Ибо хаос мешает вам сосредоточиться. Неверие. В чём неверие? Ибо есть множество  мыслей, которые говорят ``нет, нельзя'' или ``не получится''. Вот ваш хаос. А вы говорите об управлении мыслью. Вы не можете управлять и долей, долей процента. И когда-то мы говорили вам об этом. А вы хотите все стать сверхчеловеком! Зачем? Научитесь сперва, быть просто человеком. Вы забыли, что человек - это не плоть, что находится в вас, а та искорка, что называете ``божьей''. Спрашивайте.}
\people{А должен ли человек быть пассивным по отношению к политическим, социальным событиям? И заниматься только своими личными делами, делами своего окружения? }
\soul{Мы когда-то говорили вам об отшельниках. Вы помните? Вы скажете, что ``нас не было''. Это не оправдание вам. Так вот, отшельники были эгоистами. Просто религиозными эгоистами. Они уходили от мира, отходили, чтоб говорить с Богом. Слепцы. Они ушли от людей. Они говорили, что Бог в каждом из них и тут же ушли от них. Где же вера, где? Вы говорите, что знаете и тут же не выполняете, ибо не верите. Для вас это просто факт. И не больше. Вы говорите только о себе. Только лишь? Если вы будете думать только о себе, вы только и останетесь в себе и больше нигде. Вы так и останетесь в этой плоти. И всё. Но не надо говорить, что ``давайте думать только о всех, а не о себе''. Если вы будете делать так, значит, вы уже и о всех не заботитесь, ибо в эти ``все'' входите и вы. Вы же говорите о единстве и тут же отделяете себя от всех. Где же, где же ваши знания? Вы же знаете это всё. Почему же не выполняете? И не завидуем вам, ибо знающие – это предатели. }
\people{Скажите, на Алтае община живёт, которая себя считает перевоплощением Христа, может вы поняли, о ком мы говорим? }
\soul{Спрашивайте.}
\people{Вот он считает себя перевоплощением Христа. И многие люди этому верят. }
\soul{Плох солдат, что не мечтает стать генералом.  Вы же говорите, что в каждом из вас Бог. Вы должны разбудить в себе Христа. Или, как вы говорите: ``пустить'', ``открыть врата'', ``открыть душу''. Пожалуйста, в каждом из вас Христос, в каждом из вас Бог! Вы же – подобие его! Вы же, и есть - боги. Но, только, если вы будете говорить об этом с гордостью, с величием: ``Я! – а всё остальное –  фу-у!'' ,  это будет говорить не Бог. Будет иное. Чаще – величие вами, величие только вами глаголит. Вы, чуть чего-то набравшись, чуть чего-то научившись или заметивши, уже говорите: ``Я! А все остальные - фу-у-у…'', ``Я уже это прошёл'', ``Я уже знаю''  Я!-Я!-Я!- и только одно ``Я''. Это говорит уже не Бог. Просто вы нашли ещё одну колыбельную песенку для него, чтоб, не дай бог, он не проснулся в вас. }
\people{Но представитель из Минусинска …}
\soul{Давайте не будем говорить конкретно о каждом. Это ваша, это ваша задача. Вы должны разобраться во всём. И сами вы. Мы же только беседуем с вами. И мы говорили и будем говорить вам. И мы будем сеять в вас сомнения. Мы пришли не давать вам знания, ибо они уже есть все. Мы не даём вам ничего нового. Мы хотим, чтобы вы видели, видели то, что уже знаете. Использовали то, что уже знаете. Зачем же вам новое, если вы не знаете пройденного материала? Зачем?  Вы хотите о конкретностях? Нет. Вы должны разбираться в них. И мы должны спрашивать у вас. Мы. Мы же спрашиваем у вас… И что? Вы начинаете жаловаться на головные боли или на сомнения. Мы же, если хотите, мы можем сказать так: ``мы – ваши сомнения''. }
\people{А является мафиозная структура альтернативой существующей власти? Или это то же желание иметь власть?}
\soul{А вот и подумайте, играет ли здесь большая разница? }
\people{Две, наверное, стороны, которые рвут друг у друга кусок пирога. }
\soul{Это, всего лишь, зеркало. Зеркало одного, и другое зеркало - другого.  Если хотите, то корень один. Мы говорили вам, говорили вам о жадности, говорили вам о страхе. Вы боитесь быть последними. Но, вы же, и боитесь быть первыми. Вы предпочитаете спрятаться в толпу, быть такими как все, не выделиться. И, если вдруг, что-то стало выделять вас – вы сразу придумываете кучу эпитетов, чтобы оправдаться. Вот один из них: -``экстрасенс'',- ваши слова. Почему же мы не произносим имён? Да потому, что не выделить вас, ибо зачем, зачем пугать вас ещё более? Далее. Каждый звук имеет множество, множество последствий. Вы же, чаще всего, теряете энергию на что? }
\people{На разговоры.}
\soul{Да просто на звуки. Просто на звуки, и даже не ведаете, что эти звуки эхом вернуться к вам, вернутся к вам же, и будут бить вас же. (теряется). Спрашивайте.}
\people{Вы сказали, что если бы мысли все сразу воплотились, то был бы хаос. Но, ведь ни одной мысли не теряется в пространстве.}
\soul{Да. Но не материализуется в вашем мире. А есть те миры, где все мысли ваши превращаются в реальность. Вот туда вы и попадёте. Вот это вот и будет вам называться ``адом''. Если же вы создаёте мир прекрасный – вот вам и рай. Вот - и каждому – своё. Вот вам - одна из трактовок. }
\people{А нужно ли попадать туда, куда ты придумал?}
\soul{Вы должны туда попасть, ибо вы должны посмотреть на мысли свои. Хотите вы того или нет. Вы должны увидеть себя. Вы не можете себя сейчас, но, зато, вы увидите свои мысли в деле. Тогда, уже сможете сказать, хороши были они или плохи. Но только не говорите, что наказывает вас Бог. Это, только всё ваше. Ваше. А наказание божье – это всего лишь оправдание, выдуманное вами, чтобы опять не оказаться виновными, ибо вам легче, легче сказать, что ошиблись не вы, а кто-то другой. И вот вам ``ад'' и ``рай''. Или – ``ничто''. Есть и такое – ``ничто''. Это, когда вы или слишком много создаёте и столько противоречиво, что нельзя создать ни один из миров, или вы пусты. Оболочка. Как вы говорите,- не имеющий души. Зомби. Мертвецы. Среди вас их множество. Множество и рождается сейчас. Множество. Ибо вы начинаете обладать множественными способностями и что делаете? – воруете друг у друга. Человек только родился – вы уже воруете у него. Вот вам и сглаз! А матери надо восполнять. Чем? Собой! Хорошо, если, в вашем понятии, если говорить вашими словами, есть ``связь с космосом''. Хорошо, если так. Но, не у всех же, это есть. Согласны? }
\people{Угу.}
\soul{Вы как-то сказали, что у нас всё уже есть, только мы не умеем пользоваться этим. Это просто другими словами – то же самое ?}
\soul{Всё. Вы живёте, вы живёте в едином мире. Одно это должно говорить обо всём. Если вы поймёте о единстве, действительно поймете, действительно прочувствуете это - у вас больше не будет никаких вопросов. У вас не будет никаких ответов, вам не нужны будут никакие учения и религии и далее, далее. Вам не нужен будет даже Бог! Ибо вы не будете желать его, ибо, вы уже придёте к нему. Вы же, только говорите, что знаете. Вы говорите о единстве. И для вас это пустой звук. Пустой. И потому, чаще, вы приходите в пустое, приходите в пустые миры. И потому, вы, умирая, возвращаетесь сюда голосами, привидениями и многим другим, потому что вы испугались пустоты. Вы хотите  придти в мир, где бы вы жили, и там не было пусто. Вот одна из причин, когда приходят к вам извне. Вы же, радуясь тому, что есть возможность поговорить и изучить, начинаете усиленно заниматься спиритизмом. И что? Человек, который умер, остаётся здесь. Ибо он нашёл себе собеседника, друга и не хочет идти дальше. И тогда - вы уже являетесь сатаной. И потому, говорят вам: ``не занимайтесь этим''. Любая религия всегда верна, но отчасти. Есть множество примет или верований… (теряется)}
\people{А если будет стоять человек с большим духовным потенциалом. Сможет ли он что-то изменить, ну, вот, ситуацию в нашей стране? Ведь раньше царей выбирали ``от бога''.}
\soul{Давайте скажем так: никогда царя не выбирали ``от бога''. Всегда выбирали вы сами. Если хотите – ваши эмоции. Если хотите – ваши чувства, ваши чувства выбирают, ваши желания выбирают. Вот вам и ``выборы''. Ибо ваше желание -  и приходит к вам деспот, или кто-то другой. И правит вами, ибо вы желали того. Вы же, говорите - Бог. Бог поставил вам царя, Бог прислал нам ещё кого-то. Нет, это вы, вы пожелали того или другого. Вы желаете себе правителей. Вы создаёте себе королей. И, если хотите, это ваше, ваше ЭГО. Ваше коллективное ЭГО, которое пришло и командует вами. Вы поняли нас?}
\people{Поняли. А вот, если кто-то уходит с головой в работу, но эта работа никому не нужна. Да и ему самому только для того, чтобы просто деньги зарабатывать. Как вот…}
\soul{Простите, есть разные понятия ``уйти с головой''. Да, можно уйти с головой только зарабатывая деньги. Но можно уйти с головой туда, а чаще всего у вас это называется искусство, когда его, искусство, не принимается никем, оно никому не нужно. И, видя, что оно не нужно никому, он понимает, что оно не нужно и ему. Ибо, он заряжается вами. Мы когда-то говорили вам о детях и возможности научить их летать. Вы помните? }
\people{Да.}
\soul{То же самое. И это разные вещи. Зарабатывая деньги? Нет, мы не скажем, что в деньгах  зло. Зло в том, как вы добываете и в каком количестве. Только в этом зло. И не больше.  ``Уйти с головой'', - так часто ли вы это делаете? Если бы вы умели это делать, какое было бы счастье для вас и для нас. Вам кажется, что вы ушли. Вам кажется, ибо вы уже не замечаете, что-то другое. Вот вам один из признаков одержимости. И, причём, с отрицательной стороны. Человека, который считает себя гением, если он действительно гений, в вашем понятии, это не значит, что он будет гением где-то там, куда ему предстоит идти далее. Просто, он нашёл струнку, ноту, которую вы все признаёте. Истинные же гении, чаще всего, не признаются. Лишь только по смерти их, они становятся великими, ибо только по смерти вы понимаете, кто были они. Почему? А вот и подумайте, подумайте. И есть уже ответ. И вы читали его множество раз. Но, не знали, как применить его. Подсказка:'' О пророках''.}
\people{Некоторые пишут, что деньги – это выражение праны на физическом плане. И некоторые экстрасенсы утверждают, что чем больше человек зарабатывает денег, тем больше он может тратить их, и тем самым, имеет большее количество этой энергии. }
\soul{Да. Когда-то, кто-то из вас говорил: был бы богатым, у меня б было больше времени заниматься духовным. А всё остальное время мне приходится заниматься добычей денег, поэтому, мне не хватает времени на литературу, подумать о духовном и далее, далее. Мы же говорим вам и повторим: вы – сосуд. Вы должны помнить это. Вы – сосуд. Так вот, этот сосуд заполнен материальным и духовным. И чего нужно вам более? Материального или духовного? Сосуд же ваш - един. И он не может увеличиться или уменьшиться. Что ж, набивайте себя материальным, набивайте себя праной! Многие ли из вас умеют ей пользоваться? И что, вы называете их чистыми? Подумайте, Сатана глуп? А он, вашими словами, - экстрасенс, и столь мощный! Но мы же, не можем говорить о нём, как о светлом. А ведь он имеет множество в вашем понятии; бесконечное множество он может создавать сам. Как вы говорите, ``материальной праны''. Только и всего. Сосуд, наполненный материей, - и нет духовного. Вот, кто вы. И если что-то всплывает наверх, вы тут же, тут же засыпаете новое. Если к вам пришло сомнение, ``а так и делаю я?'', вы тут же глушите его логикой, и новым способом добычи той же ``праны''. }
\people{В религиях говорят, что сомнение – большой грех. А я, вот, только и делаю всю жизнь, что сомневаюсь. }
\soul{“Блажен сомневающийся''. Будьте внимательны при чтениях. }
\people{Да? Хм. Многие источники, которые сейчас создаются: Тайная доктрина, Агни-йога,- призывают расширять сознание. Что это может означать? Приобретение знаний, или приобретение способностей аномальных?}
\soul{Ни в том и ни в другом. Что такое приобретение знаний? Это можно стать просто энциклопедией. Энциклопедия, имеет ли она множество знаний? Да, она имеет. Но умна ли она от этого?  Нет. Это, всего лишь, факты. Это, всего лишь, знания и множество фактов, но не воспользование ими. Вы говорили о повышении своих способностей. Ну что же, давайте будем повышать. Давайте, мы станем могучими, и мы говорили только что о Сатане. Да, вы при желании можете сдвинуть горы. Но зачем? Зачем нужно это вам? Зачем? Чаще всего, вам привлекательно то, чтобы создать вокруг пупка какой-то огонь. Вы помните?}
\people{Да.}
\soul{А зачем?  Чтобы `` уверовать в себя''. Чтобы понять, что ``мы растём духовно''. Оттого, что у вас горит пупок, это не значит, что вы духовны. Это, всего лишь, возможность проявлять материю и не больше. }
\people{Да, вы как-то сказали, что мы должны учиться управлять, а не просто иметь.}
\soul{Мы говорили вам учиться управлять? Мы не говорили, чтоб управлять чисто материей. А вы не можете управлять даже и ею. И слава богу, в вашем понятии. Вы не можете управлять даже своим собсвенным. Вы не можете управлять своей памятью. Той самой физической памятью, той самой химией, которой в вас. И вы не помните или всё извращаете. Почему? Вы скажете ``память слаба''. Разве?  Да, химическая память слаба. И есть множество, множество факторов внешних и внутренних, которые мешают, мешают проходить этим химическим реакциям. Но, простите, вы же, всё-таки, считаете себя человеком не только от-того, что у вас есть химия? Вы согласны? Вы же, зная об этом, не делаете больше ничего. Вы занимаетесь только …(теряется)}
\people{Вот я хочу о лидерах спросить. Всегда ли группе нужен лидер? Или можно находить общие решения наравне?}
\soul{К сожалению, вам нужен всегда лидер сейчас. Вы привыкли, привыкли. И вы даже Христа сделали лидером! Не говорил Христос о том, что вы бараны и вы его стадо. Не говорил! Это вы желали того. Вы! Вам нужен лидер, потому что вам легче,- меньше думать, меньше решать. Вам очень трудно решать, потому что вы боитесь ошибиться. Что такое лидер? Вы скажете, что это человек, не боящийся ошибиться? Нет. Ему просто наплевать на эти ошибки. И потому, у вас есть множество лидеров, которые достаточно глупы, и вы знаете об этом, но он всё равно остаётся у вас лидером. Потому что где-то в глубине, где-то в глубине - ``хорошо, что это - не я''. И, тут же, вспоминаете о ``козле отпущения''. }
\people{Н-да. }
\people{Если целитель не может помочь себе в каких-то случаях, может ли он помочь другим?}
\people{Имеет ли право?}
\people{Да.}
\soul{Давайте не будем говорить о праве. О праве решаете вы сами. Только вы сами. И больше никто. Вы должны решать, сделать это или нет. Ваше сердце, ваши чувства. И никто не имеет права запретить вам или настоять, ибо вы должны быть хозяином. Вы и только вы. Всё остальное – это зомби. Богу - зомби не нужны.  Далее. (теряется)}
\people{Скажите, а что сейчас чувствовал переводчик, почему такая интересная реакция была?}
\soul{Мы же говорили вам: чужой. }
\people{Он, всё-таки, врывается в разговор, да?}
\people{Скажите, а как мы чужого можем в сознательном состоянии слышать? Это как разговор внутри себя?}
\soul{Мы говорили вам о состоянии переводчика. }
\people{А-а-а. }
\soul{И к нему могут придти любые. Любые. И он вправе выбирать кого-то из них. И если у него не хватит сил от вмешательств, в него могут войти. Мы говорили вам когда-то, хоть и грубо, но мы сказали, что вы нас не волнуете физически, и ваши болезни не волнуют нас, помните? Мы вынуждены делать это. Вынуждены. Ибо, иначе, вы будете просто идти нашей дорогой. Дорогой, которую желаем мы. И хотя мы будем говорить вам: ``мы хотим, чтобы вы выбрали дорогу сами, идите своей'', - всё-таки, в вашем понятии, подсознательно, даже не замечая сами, мы будем направлять вас. Мы же не хотим того. Другие, как переводчик называет их - чужими,- будут очень упорны. Почему слаб голос божий? Почему? Потому, что он не настаивает, он не требует, он просто - шепчет вам, просит вас. Он не имеет права на насилие, потому и слаб. А всё остальное довольно грубо приходит с силой. И чаще, у вас не хватает сил, чтобы ответить ему. Потому и говорим вам: будьте внимательны при прочтении и получении знаний. Ибо многое, многое написано так хорошо, что вы даже не заметите что вы уже стали, в вашем понятии - запрограммированы, в вашем понятии - зомбированы, в вашем понятии - к вам придут вампиры. Или вы позовёте их сами или создадите себе какое-то фэнтези. Прочитав что-то, вы найдёте множество подтверждений, что это было с вами. И рано или поздно это с вами сотворится. Дай бог если о хорошем. Но плохое вам воспринимать легче, ибо страх сильнее других чувств пока что и к сожалению. И потому, плохое вы воспринимаете легче и запоминаете лучше. И обиды запоминаются гораздо лучше, чем радости. И человек, которого вы всю жизнь любили, который никогда плохого вам не делал, сделав одну ошибку относительно вас, - и вы уже сразу забудете, что он плохой… Извините - хороший. И вы уже будете в обиде на него. Почему? - Ибо зло запоминается легче, ибо шёпот слышать трудно в хаосе ваших мыслей и желаний других поговорить с вами. Иными словами, жить в вас в вашей плоти.}
Вам говорят: бойтесь не того, кто осквернит плоть вашу, а душу. Помните? В каждой религии есть зерно, ценное зерно. В каждом слове есть смысл. И даже если это было дурное слово, в нём всегда найдётся смысл. А вот какой смысл, добрый или хороший, зависит от вас. Только от вас. Только вы выбираете. Только вы. Приходил к вам Христос, говорил вам. Что? Что поняли вы? – ничего. Почему? Потому, что он не настаивал и не требовал. Всё остальное - выдумано вами. Вы сказали, что он требует. Вы сказали, что он вселится. Вы сказали, что он может убить, наказать. Почему? Потому что вы ищете лидера. А в вашем понятии лидер - это обязательно человек без компромиссов и очень жестокий. Почему? Потому, что, чаще всего, вы управляемы страхом. И потому, требуете, чтобы лидер воздействовал на вас страхом. Вот ваша беда! Потому и Боги ваши все - страшны. Ибо вы хотите того. Спрашивайте. 
\people{(Ольга) Скажите, имеет ли право целитель брать деньги с того человека которого он пытается излечить? И даже вперёд.}
\soul{Мы говорили вам о деньгах. Но, если быть честней, в вашем понятии, самое страшное, когда вы берёте заранее. Этим вы уже обязываете себя и обязываете его. Хорошо если только это, но у дающего появляется страх: ``А не пропали ли деньги даром?'' Это непроизвольно, но это уже мешает лечению. Когда же он придёт, и будет чувствовать бескорыстие от вас, ему будет, гораздо легче поверить что вы действительно можете его лечить. Всё остальное время, будет сомневаться только в том, что, не один ли это из способов выкрасть у него эти деньги. Вы согласны? Поэтому, говорим вам: берите только, что дадут вам. Вами же написаны сказки о горшках продающих, но не имеющих цены. Сколько даст? Кто-то возьмёт бесплатно, а кто-то и даст вам. И всё это вернётся, всё вернётся. Есть же люди, которые хотят помощи вашей, но не могут придти к вам, ибо у них нет тех денег. И когда вы придёте…хорошо, давайте скажем к Христу. Вы придёте к нему, и он скажет: ``А вы отказали тому человеку!''. Вы же: ``Не знал его, не видел никогда!''. И вы будете лживы, ибо он не мог придти к вам. Почему? Потому, что у него не было денег, которые требуете вы. А значит, вы уже отказали, а значит, вы уже рассудили. И самая страшная мера, когда вы судите по деньгам. Когда товарищей себе выбираете по деньгам или по выгоде. Вот что самое страшное. Вот это и есть то – мамона. А вы говорите: ``Материальн…'' (теряется)}
\people{(Ольга) Что такое - грех? Очень много всяких интерпретации по поводу греха и первородного греха особенно. Тут можно и запутаться. Может ли считаться грехом любовь одного человека к нескольким особам противоположного пола, если он действительно искренне их любит?}
\soul{Вы знаете, давайте спросим вас. Вера в Сатану, искренняя вера – это грех? }
\people{(Белимов) Скорее всего - грех.}
\people{(Ольга) Нет, наверное.}
\soul{Давайте скажем по-другому: достаточно ли искренности для веры в Бога?}
\people{(Ольга) Этого не достаточно, но это необходимое условие наверно. Человек должен быть искренним.}
\soul{Одно из необходимых условии, но недостаточных. Достаточно ли искренности любить? И простите, можно искренно любить и деньги, можно искренно любить и кровь. Есть множество способов любви и множество объектов любви и все они будут искренними. Человек в вашем понятии маньяк - одержимый, он искренне любит убивать. Вы назовёте его святым?}
\people{(Ольга) Да нет.}
\soul{Есть те, которые сомневаются в своей любви: ``Достаточно ли хорошо я люблю?''. Они грешны? В них нет искренности, они сомневаются. Так вот они будут более честными. Скажем по-другому. Если вы из множества любимых выбираете себе самую любимую, значит, вы не любите ничего. }
\people{(Гера) Тогда, получается - нельзя ничего выбирать?}
\soul{Вы подумайте, если у вас есть возможность выбора, значит уже, вы не любите ни то, и ни другое.}
\people{(Ольга) А если нет такой возможности? Допустим, женщина зачастую выходит замуж за человека, который первый предложит ей. Она выходит, боясь остаться одинокой.}
\soul{Ну и как? Это искренность страха, а не любви.}
\people{(Белимов) Можно ли любить нескольких человек?}
\soul{Надо любить всех. Но только ``любовь любви - рознь''. Если только говорить о любви плотской - это грех. Почему? Ибо вы, энергетическии раскидываете себя и заражаете других. Вы забываете, что вы носители энергии и носители информации. Вы забываете о том. И когда вам говорят о любви плотской, имеется в виду не что-то, а именно то, что вы создаёте новую форму энергии. Чаще всего – отрицательную, ибо всё-таки в вас есть страх. Страх, что будете судимы. Вы чувствуете, что вы грешите. И всё это создаёт столь грязную смесь, которую к вам же и вернётся, и вас же будет бить. И когда, в вашем понятии, вы попадаете в ад, вы попадаёте именно в то, что создали сами. Когда же вы говорите о любви духовной, искренней, то, простите, искренняя любовь - истинная духовная, это любовь всех, когда вы уже не можете сказать, что я не люблю того-то и того-то. Это любовь даже самого падшего. Ибо вы, познаёте уже единство, и вы будете знать, что падший - это кусочек вас, и, значит, вы виноваты, что потеряли этот кусочек. Вы, и никто другой. Потому, давайте будем различать любовь.}
\people{(Гера) Скажите, а вот некоторые люди  на других воздействуют  энергетически там, или, как-то эмоционально, не знаю, ну, энергетически, -  очень отрицательно. И, хотя, вроде бы, против этого человека ничего не имеется, но, однако, люди не желают с ним иметь контактов. Ну, стараются ограничивать.}
\soul{Есть множество причин. Одна из причин, это старая память. Что вы, когда-то уже знали его и были врагами. И вам есть возможность исправиться. А что делаете, чаще всего, вы? Вы тут же объявляете его врагом и здесь. Если вы скажите: ``Нет, он не враг, но я просто не люблю его, не хочу разговаривать с ним'' - это уже враг. Это уже значит, вы назвали его врагом. Есть и другое. В вашем понятии, ``полярность''.  ``Не тот'' заряд.  Да. Это говорит только о вашей слабости. Это говорит о том, что вы ограничены, что вы имеете только один маленький заряд, а не общий, а не единый. Только и всего. Христос пришёл и не говорил, что этот человек плох или хорош. Вы помните? Он никогда не осуждал никого. Вы же, говоря, что у него отрицательная энергия… А вы уверены в этом? Помните, мы вам говорили о картине?}
\people{Да.}
\soul{Вспомните, придёт человек с чистой душой, картина даст ему энергию. Или, он же придёт, и картина будет отбирать у него. Но, это не значит, не значит, что она даёт отрицательную энергию. Она, просто забирает её. Как вы можете сказать, отрицательна энергия или положительна?  Только относительно себя.  Если вы отрицательны, все остальные будут для вас положительны, но вы будете называть их отрицательными. Потому что вы не хотите признать себя плохим героем, вы согласны?}
\people{Угу.}
\soul{И потом будете говорить, что вы положительный, а все остальные - отрицательные. И наоборот… }
 (счёт)1…
\soul{Мы говорили, что ошибка ваша - в желании зла. Проклиная  другого, вы проклинаете и  себя. Мы говорили вам, что вами питаются в эмоциональных мирах, питаются вашими энергиями. Соответственно, злые приходят и злят вас. Злят, и радуются, когда вы их ругаете, ибо они этим питаются. Вы делаете это всегда и постоянно. Вы, вместо того чтоб хотя бы отнестись к нему равнодушно, не замечая, а ещё лучше б если вы просто пожелали добра и ему хочешь-нехочешь, а пришлось бы уйти от бессилия, и мы когда то говорили вам, что самое главное…}
 1-2-3… (обрыв связи).
\people{(Белимов) Интересные мысли говорили нам об отношении чужой энергетики. О том, что не надо злиться и прочее. Вы можете продолжить, чтоб мы, хоть поняли, как нам вести себя в этой ситуации?}
\soul{Сказано было же вам: ``ударили по одной щеке, поставь другую'', как вы вспоминаете это? Со смехом!}
\people{(белимов) Со смехом, да.}
\soul{Потому, что вы сразу представляете о щеках, и о бьющих. Присмотритесь, вглядитесь глубже, почитайте меж строк. Если бьют вас, и вы даёте сдачи - бьют вас ещё более. Ах… Если бы вы умели, если бы умели отличать добро от зла, вы бы никогда не обвиняли огульно. Чаще, чаще вы добро объявляете злом, ибо оно приходит к вам, непонятое вами. Почему? Потому, что ``не так, как хотелось бы'', ибо мозг  отрицает: ``нет, это не правильно, надо было по-другому'' и тогда, добро превращается в зло. Мы говорили вам относительно добра и зла, говорили о единстве его. И что имеется? Что имеется? То, что зло вы принимаете за добро. Когда приходят к вам и предлагают, что-то тёмное, вы тут же соглашаетесь и говорите: ``Как повезло!'' И, даже, иногда вспоминаете бога, творя грязные дела. ``Бог помог''!  Когда приходит добро, что делаете вы? Вы не понимаете его, ибо оно мешает. Мешает. Ибо, чтобы стать добрым, это надо отказаться от многого. Но, логически, зачем? Это же глупо, если это моё. Вы, подавая милостыню, вы тут же похвалите себя: ``вот какой я добренький'', а не давая, скажите: ``да он же алкоголик, он всё равно их пропьёт''.  И вы будете в этом правы? Простите, это его проблема. Ваша проблема дать. Когда то вы спрашивали, и вы должны помнить, спрашивали о милостыне. Помните? И каков был ответ? Вы не можете повторить его?}
\people{(Гера) Буквально? Не могу.}
\people{(Ольга) Смысл,хотя бы, скажи.}
\people{(Гера) Ну, смысл тот, что каждому, так сказать,- по его мыслям. Т.е. кто-то даст от чистого сердца, а тот употребит во зло, ну, там или плохое дело, скажем, то это уже будут его проблемы. Получается, человек, который даёт от доброго сердца, скажем так, он не несёт ответственности, что дальше будет делать его там деньги. }
\soul{Да. И даже вы привели пример о ``дороге''. Мы помним, потому что помните вы. Но, вы же играете игру непомнящих, игру  ребёнка. Вы помните и знаете всё. Это даже можно доказать физически. Доказать вашими же законами. Ну, а вы не делаете этого. Почему? Потому что вы тут же говорите о множестве неизвестных, подразумевая, конечно душу. Любой учёный может сделать множество расчётов, но он всегда будет подразумевать душу. И, если даже не признаёт её, он будет подразумевать её. Только назовёт её ``интуиция'' или ещё чем-нибудь другим. Или гениальностью. Ну, какая разница? Вы же делаете всё наоборот. Вы математику превращаете в орудие и в оружие. Орудие, - чтобы прожить. Вы, даже в математике хотите вычислить и найти формулу любви и счастья. Дай бог, чтобы вы не нашли! Иначе, тогда вы забудете, что такое счастье и останетесь просто машинами. А к этому и идёте. Вы развиваете, какие науки? Физические. Физические. Мы же говорим о единстве, а это значит, что и физическое - тоже духовно. Это значит, что все науки ваши, - тоже духовны. Но, вы не видите в них души. В каждом есть душа. И мы говорили вам, и вы даже спрашивали: ``есть ли душа в стуле или в столе?''. Вы должны помнить. И мы отвечали вам: `` в каждом есть, в каждом из вас, и в каждой. И нет пустоты''. Вы же, берёте линейку и измеряете. Вот учение ваше. И даже беря, в вашем понятии, ``Святую книгу'', вы  тут же измеряете её штангелями, циркулями, цифрами, и что-то хотите там найти, что-то зашифрованное или что-то ещё.}
\people{Скажите…}
\soul{Вы, даже объясняясь в любви, как вы говорите – ``искренне'', -  вы тут же подсчитываете ``за'' и ``против''. Вот ваша математика, вот ваша логика. Те же чувства, что говорят в вас, что дают вам сомнения, вы опровергаете, потому что есть множество, множество доказательств, что это чувство просто ложно, это суеверие, это просто глупость и множество, множество причин. Вот ваша жизнь, вот ваш обман. Вот мир, в котором вы живёте.Вмире обмана, в мире лжи.Мир, в котором вы играете в большой театр.И вы привыкли, чтоб у вас был режиссёр, и назначили эту должность богу - бог управляет вами, без негоне упадётни один волос… Правильно!- Случай чего - бог виноват! Не вы… Вы даже не представляете, как это мы может…(обрыв записи)}
 1-5-6-7-1
\soul{Спрашиваете. }
\people{(Ольга) Скажите, вот я вижу иногда над переводчиком такие тёмные пятна, может быть, они ему мешают? Что это?}
\soul{Что вы можете сделать с этими пятнами?}
\people{(Ольга) Да ничего. Просто спросила.}
\soul{А вы видите пятная ещё где-то?}
\people{Чёрные?}
\soul{В каких-то других местах?}
\people{Ну, около головы. Только не пятна, а…}
\soul{Слушайте, давайте скажем так: вы можете увидеть себя, не посмотря в зеркало?}
\people{(Ольга) Могу.}
\soul{Представьте себя, рядом.}
\people{Ну, представила.}
\soul{Вы можете это сделать? Вы представили? Что вы увидели там?}
\people{Ну, я может быть, не себя представила, а просто, так сказать, свою внешность…}
\soul{Хорошо, давайте сделаем так…  Вы говорили, о множестве тел, которые находятся в вас. Ну, давайте представьте не себя, а представьте астральное тело. Или, представьте ментальное тело. Вы можете это представить?}
\people{Астральное проще, а ментальное… тут уже я затрудняюсь.}
\soul{Хорошо, давайте сделаем по-другому…  Постройте своё биополе рядом. Вы знаете, что такое биополе.  Вы часто просите. Постройте его.}
\people{Рядом?  Ну, построила.}
\soul{Построили?}
\people{Да.}
\soul{И что вы увидели там?}
\people{Свет.}
\soul{Каков?}
\people{Голубой и зелёный.}
\soul{Прекрасно. Спрашиваете далее.}
\people{(Ольга) Вот, теперь, несколько вопросов об отношении между полами, т.е. вот именно… Ну, не будем говорить о любви вообще между людьми, потому что это, конечно, вопрос такой… Ну, отделим, разделим опять немножко так, потому, как у нас очень часто бывает затруднение именно в отношении, как раз такие. Вот утверждают некоторые йоги, что женщина имеет слабую витальную силу, и поэтому ей очень нужен мужчина, т.е. она берёт от мужчины эту силу витальную. Действительно это так?}
\soul{Можно сказать и наоборот. Что такое, мужчина без женщины? Что?}
\people{Вот,  зачастую…}
\people{Свободный человек.(Белимов перебил))}
\people{Т.е. это необходимо, ведь женщина мужчина - это одно целое, в принципе. }
\soul{Давайте скажем так: мужчина без женщины Ничто, и женщина без мужчины - будет то же самое. Давайте не будем говорить, что больше, что меньше. Не было б мужчины,- не было бы и Мира. Не было бы женщины, - не было бы его тоже. Только в вашем мире. Есть миры, где нет полов, и мы же говорим вам, что вы имеете пола только здесь. Пол и род это совершенно разные вещи. Вы можете всегда материально рождаться женщиной, хотя род ваш мужской. И  наоборот. Мы не можем объяснить этих причин, потому что для каждого своя, и для нас - это тоже тайна. Мы можем сказать только одно, что род играет вами. Род влияет на вас, но не пол. Пол,- это свойство реакции, реакции воздействия рода. Только и всего. Вы можете быть мужчиной, можете быть женщиной. Почему, в вашем понятии, появились бисексуалы? Почему? Причина проста  - вы теряетесь. Вы, не хотите матриархат, ибо боитесь и говорите: ``женщина во главе? ''.  И, тут же… и, тут же, -  хотите женщину, хотите ласки, хотите любви. И чаще, в вашем понятии, ``би'', всего лишь, только те, которые были лишены одной из сторон, или матери, или отца. И, тогда, вы хотите занять, восполнить эту потерю. Вот вам, пожалуйста.  Почему человек, всю жизнь проживший мужчиной, вдруг, хочет стать женщиной. Почему? Очень просто. Может, он просто потерял любимую женщину и не хочет признаться в этом?  И потому, не сумев создать образ или найти подобный, вы хотите вылепить его сами. Вы хотите умереть сами, но что бы жила ваша любимая. И тогда, вы,  для большей убедительности, создаёте себе женское тело. Хотя, не меняется в вас ничто. Только тело, только плоть. Неужели вы думаете, что женщина и мужчина, думают по-разному? Нет. Они думают по-разному. Ибо женщина более восприимчива. Почему? Только не из-за слабости энергетики, нет. Просто, у неё более обострённые чувства. Просто, у неё более  ``хуже'', в вашем понятии, логика. Вы же принимаете за отрицательную черту. Нет… Нет… Просто женщина ближе, ближе к сердцу.  И ``искорка божья'' светит ей ярче. А если быть точнее, просто она видит её ярче. Но и женщина, обладая теми свойствами, может быть страшна. Ибо, правильно говорят у вас: ``Женщина в гневе, что может быть страшнее?'', ибо обладает слабостью своей, столь могущей, что мир уничтожить - в её праве. Что делаете вы? Что делаете вы, мужчины? Вы говорите, что вы сильны… и покоряетесь женщинам. Сильны ли вы перед женщинами? Нет, слабы. Слабы и беспомощны. Но и успокоим вас - женщины слабы и пред вами. И потому говорим вам: вы равны. Но,только не надо то понимать в прямую, как делаете это вы - женщина-тракторист или мужчина-дояр. }
\people{Ну, а почему бы и нет, если человеку нравится?}
\soul{Вот вам и бисексуализм.}
\people{Скажите, вот сейчас, вообще-то, очень распространёно это - ``бисексуалы''. Т.е. это даже модой стало. Дело в том, что некоторые, действительно, может быть вот и внутренне такие у них проблемы или переживания, а вот другие просто следуют какой-то моде. Сейчас, особенно среди людей искусства очень распространено это, или…}
\soul{Есть воспоминания, если вы когда-то в прошлой жизни, вы были другим полом, и где-то в вас это стало инстинктом, вы стараетесь  вернуть этот пол. И тогда уже с детства, уже с детства вы не признаёте себя мальчиком или девочкой. Но, чаще, это всего лишь ваша прихоть - выделиться, изменить себя. Потому, что вам уже надоело, вы уже боитесь.  Вы хотите что-то новое, но вы находитесь в тупике и не видите выход и думаете: ``если поменять тело, -  это же совершенно что-то иное, это другой поворот, это крутой поворот в моей жизни, это что-то изменит''. Конечно, если вы так резко различаете мужчину и женщину, для вас это будет ``крутой”поворот! А вот изменит ли? Нет. И вы опять пойдёте по кругу. Опять и снова. Один из вас написал сказку о возможности мальчика возвращаться в прошлое. И что же? Он стал повторять то же самое. Он знал, что будет через пять минут, но не мог сделать шаг, чтобы изменить это. Как вы думаете, он прошёл двойную, тройную жизнь? Он прошёл ту же самую жизнь. Да, длительность её конечно была иная, но что… И что это дало ему, что? Если он прожил тоже самое, он совершил те же самые ошибки. Да, конечно, он совершил те же самые и хорошие. Но, простите, у вас есть понятие: ``Я же не сделал ничего плохого''. Но и не забывайте  добавлять – ``Но и  ничего хорошего''. Вот  тааааак….}
 1-2-3..
\people{Скажите, вот сейчас…  А как отличить, вот допустим, вот, что мы общаемся с вами или с кем-то другим?}
\soul{Вспомните, мы говорили вам. }
\people{Только ``здравствуйте''? и всё?}
\soul{Разве? Вспомните, когда вы ещё не вели записи. Когда-то было, в вашем понятии, как ведёте счёт - второй контакт. Вспомните, как мы говорили вам? И, как даже вы посоветовали, и мы теперь делаем это. Вы забыли?}
\people{Да, честно говоря, было давненько.}
\soul{Давненько?}
\people{Ну, относительно конечно. Не помню. Вы лучше повторите.}
\soul{И сколько мы будем повторять вам? Мы только и делаем, что каждый раз повторяем вам одно и то же, только другими словами. И тратим множество красок, чтоб нарисовать одну и ту же полосу. Только что разные цвета. Вопросы вокруг да около, но все те же самые. Соответственно, те же самые будут ответы. Но, это не в упрёк вам, нет. Задавайте, ищите. Вы совершаете ошибки, но это счастье ваше, что вы можете совершать их. Их не совершает только те, кто ничего не делает. Мы же говорили вам. Помните - о движении? И были бы вы, если внимательны, что было сегодня. Что было сегодня?  Было вам:``здравствуй''. И было, когда спрашивали о вас, не  делая движение руками. Вы помните? И тут же прекратилось, и тут же прервалось. Вы не  заметили того? И это ничто не напомнило вам? Когда-то вы говорили о щёпоти, и спрашивали о ней. Вы помните? }
\people{Да.}
\soul{А говорите ``забыли''!  Почему нет её сейчас?}
\people{Вот, сказали, что уже, уже по-другому. Ну, я так подумал -  мало ли что изменилось.}
\soul{Придут к вам чужие, скажут ``по-другому'', и вы будете продолжать. Вам было ``здравствуй'', и вы стали спрашивать.}
\people{Хотя и удивились.}
\soul{У вас были сомнения, но вы не послушались их. Почему?}
\people{Я просто не успел сказать. Я заметил это - что ``здравствуй'' и сразу вспомнил, что вы не говорили никогда. Но меня перебили. Я не оправдываюсь, просто - реальная вещь.}
\people{Можно дальше спрашивать?}
\soul{Давайте далее. Что вы знаете о шёпоте?}
\people{О шёпоте Земли? О щёпоти?}
\soul{Нет, о шёпоте.}
\people{О шёпоте Земли? Или о шёпоте…души?}
\soul{Вообще. Что такое шёпот?}
\people{Ну, вообще-то…  Интуиция, видимо.  Если, вот, говорить о, допустим, общении между людьми, когда люди говорят шёпотом, замирают. Т.е. это самое, будем говорить, такое… Высшая стадия общения. Да. Собственно, здесь не нужны какие-то громкие слова или что-то, а просто, допустим, два человека любимых, они шепчут друг другу что-то…}
\soul{Высшая стадия общения?}
\people{Нет?}
\soul{Два вора, чтоб не услышал сторож.}
(Смеются.)
(Коенц контакта)
(обсуждение после контакта)
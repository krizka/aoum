Аоум. глава 3-12-95г
Георгий Губин
\people{**}
3-12-95г
\people{**}
VG - 1995.12.03
\soul{…цифра 7. Другие говорят, что можно прожить и сто жизней. Зачем? Вы живите этой жизнью. Зачем вам знать прошлые жизни, будущие? Вы живите этой. Если вы будете вспоминать слишком много о прошлом, вы не успеете сделать в настоящем. Если нет настоящего, то не будет и будущего. Если вы будете знать все прошлые жизни, что это даст вам? Вы растеряетесь и не сможете уложить всё это по полочкам, ибо мозг ваш, всё-таки, физика, химия. И вы имеете…}
\people{(Белимов) Скажите, если о реинкарнации говорить, какова дальнейшая судьба тех, кто закончил на Земле цикл инкарнации? }
\soul{Мы когда-то говорил вам о мирах, куда вы попадаёте. Вспомните, мы говорили вам о богатстве. Если вы богаты, вы попадаёте в мир не богатства и не нищеты, как вы считаете -  противоположное, а в мир страха, ибо здесь вами управляет страх потерять это богатство. Если вы всю жизнь прожили бестолково, в следующей жизни вы берёте старт от прошлого. И будете также начинать бестолково. И дай Бог, если вы поймёте это и будете жить далее, как следует. От того прожили вы хорошо или нет, праведно или нет не зависит, будете вы на Земле или нет. Здесь совершенно другие меры. Если вы были здесь грешником, это не значит, что вы попадёте на другую планету, где более страшные физические муки. Нет, вы можете остаться здесь же на Земле, а можете уйти и выше в вашем понятии. Ибо, опять же, высшие миры и низшие миры вы мерите физически, мерите климатом планеты, на которую попадёте. Если вы прожили себя плохо, значит, вы попадёте на планету, где очень-очень плохо живётся физически. - Нет, здесь нет меры физики. Далее, вы можете не попасть ни на какую из планет, вы можете остаться здесь. Это есть понятие как - вы заблудились. Вы, не верите ни в то и ни в другое. Или вы слишком много представили и сочинили миров, а в какой из них попасть вы уже не знаете. Вы теряетесь, ибо тот мир ваш и тот мир ваш, и все они придуманы вами. Истинный же мир, который должен был прийти к вам, вы не знали его. Вы только и делали тем, что сочиняли другие миры. И потому, вы можете придти в один из созданных вами миров. Или остаться на Земле, чтобы повторить всё снова или уйти на другую планету, но не связанную с физикой. Ибо там меряют вас не вашими возможностями физическими, а вашими накопленными знаниями, вашей душой.}
\people{Наш коллега из Екатеринбурга спрашивает:  ``Сколько людей ждет очередных Земных воплощений''? }
\soul{Спрашиваете далее.}
\people{Скажите, сколько инопланетных душ подселенных после начала большого эксперимента развиваются в земных условиях? }
\soul{Мы когда-то говорили вам, что вы пришли 15000 лет назад. Вы помните это? }
\people{Да }
\soul{А теперь возьмите и попробуйте, соедините или хотя бы найдите, что было 15000 лет назад. И тогда этот вопрос отпадёт сам, но появятся множество других. И тогда вы будете задавать их. Сейчас мы можем сказать вам:   Что в вашем понятии, инопланетянин? Что? }
\people{Ну, это посланцы с других планет, мы так полагаем? }
\soul{Давайте скажем так: из параллельного мира пришли к вам инопланетяне?}
\people{Ну, это иномиряне, мы их так называем. }
\soul{А как вы отличите их, если они не будут похожи на вас?}
\people{Трудно будет. }
\people{(Ольга) Да не как, это внутренняя сущность…или как? А внешне, они будут такими же? }
\soul{Не обязательно. Физику вы можете менять. Скажем далее. Душа меняет и выбирает тела ваши, но не наоборот. И потому, если вы духовно изменитесь, изменится и ваше тело. Если вы ничего не делали и оставались на месте, вы опять станете и останетесь в том же теле. Вы опять будете человеком. Вы можете сотни раз в вашем понятии оставаться человеком. Это не значит, что вы продвинулись или остановились. Это только говорит о том, что вы разучились идти.}
\people{(Белимов) Скажите, верно ли наше предположение о проводящейся мягкой коррекции человечества со стороны внеземных цивилизации?}
\soul{И опять вы ищете виновных на стороне.}
\people{Мы ощущаем, что всё-таки может быть воздействие, если мы неразумно себя ведём по отношению к природе. Или такого нет вообще? }
\people{(Ольга) Мы сами себя наказываем. }
\soul{Давайте скажем так: природа может скорректировать и делает это. К сожалению, вы пытаетесь сделать тоже с природой. Вам не хватает знаний, и тогда вам легче её уничтожить, чем и занимаетесь вы. Далее, вы говорите об инопланетянах. Почему вы боитесь их так? Почему вы их создали? Для того, чтобы обвинить их. Или вы ищете птицу счастья, которая принесёт вам. Ибо вы сами уже не хотите, и вам надоело искать его. И вы хотите уже помощи на стороне. Для этого вам нужны инопланетяне или Бог или ещё кто-то другой. Бог есть. Но не будет вам давать счастья. Вы должны прийти к нему. Вы сами должны его найти. Только и всего. Вы же приходите к Богу, расшибаете лбы в молитвах и просите: ``Подари''. Не ``дай'', а ``подари''. Не то что заслужили, а просто, хотите получить подачку. Бог же их не дает, а вы потом говорите: ``Бог не услышал моих молитв''. Хорошо если так, чаще же: ``Нет Бога! Ибо он не слышат нас.”}
\people{(Белимов) Скажите, тут вопрос такой трудноватый… }
\people{(Ольга) 1-2-3-4….}
\people{(Белимов) Скажите переводчик, может быть, требует перерыва по физиологическим данным? }
\soul{Давайте, скажем так: что делает ваше сознание, когда бодрствует? Она старается подчинить тело. Не обращая внимания, что телу что-то требуется другое или что-то иное. Иными словами, вы теряете здоровья тем больше, чем больше осознаете себя сознанием. Когда вы спите, происходит процесс очищения, ибо сознание занято снами, для чего и дано. Тело же справляет свои нужды, подлечивается, ибо уже не дает ей кто-то поправок ``сверху''. И оно знает само, что делать. В данном случае, ``переводчик'' ваш, относительно тела, спит.}
\people{Скажите, как часто люди злоупотребляют пользованием информационным полем Земли? }
\soul{А вы только и живёте тем, что пользуйтесь полем Земли. Что - мозг ваш? Только ``приемник'' и не больше.}
\people{(Ольга) Это естественный процесс? }
\people{(Белимов)  То есть, мы, может быть, наоборот - ``недобираем'' использованием поля Земли? }
\soul{А что было бы, если бы вы ``добирали''? Вы уже сейчас умудряетесь делать духовные преступления. Уже сейчас, вы заходите в иные миры с целью нажиться или, что ещё хуже…}
\people{Скажите, переводчик резко оборвал сегодняшний сеанс. Чем это на ваш взгляд объясняется? Чем это можно объяснить? }
\soul{Вы прерываете, мы не уходили. Давайте, скажем так: переводчик спешит.}
\people{То есть, ему надо домой вообще-то? }
\soul{Нет. Переводчик спешит, хочет опередить нас. Он пытается нарисовать свои картины. Мы не мешаем ему. Его картины тоже красивы, хотя, чаще, не верны.}
\people{(Ольга) Как нам тогда различить? }
\people{(Белимов) Вы как-то пробовали, отключали всё-таки его вмешательство. Наверно надо так и поступать. }
\soul{Нет, ибо это уже насилие.}
\people{Хорошо. Спасибо вам, за такое отношение к  нашему ``переводчику''. Мы продолжим вопросы.}
\people{Скажите, как проходит подготовка человечества к переходу в шестую расу? Проходит ли она? }
\soul{Давайте договоримся, что в вашем понятии ``шестая раса''? Только вы придумали наименования и даже умудрились дать им нумерацию. Каждый из вас, как вы говорите, ведёт подготовку. И почему только в шестую? Если среди вас есть, и ``первые'' и ``вторые'' и ``седьмые'', если пользоваться вашей нумерологией. Среди вас есть те, которые не подлежат вообще никаким вашим понятиям и нумерологиям, ибо они пришли извне. И здесь в качестве гостей или в кого-то либо других. Каждый из вас готовится. И даже если кто-то из вас уже перестал надеяться, или как вы говорите: ``деградируете'', это всего лишь один из уроков, только и всего. Ибо, что такое жизнь? Это движение вперёд. Вы когда-то спрашивали о машине времени, и мы вам говорили, что относительно вас, лично вас, вы не можете уйти в прошлое. Вы можете посмотреть на прошлого себя, но именно для вас время идет вперёд. Именно для вас - всегда есть движение. Даже тогда, когда, в вашем понятии, вы остаётесь на месте и не совершенствуетесь. Далее, нельзя говорить о скорости, когда придёт время то. Нельзя. Ибо резко не будет оно, потому что каждый из вас должен прийти своей дорогой, и никто не будет вас подталкивать. Есть варианты, когда природа помогает, иначе в вашем понятии - катастрофы. Вами развита теория катастроф. Отчасти она верна. Дальше, вы переходите через боль, через страх. Мы говорили вам, что это одно из самых сильных чувств ваших, и вы должны опуститься столь низко, чтобы испугаться самих себя. Но это не значит, что вы опуститесь определенного уровня. Нет, ибо человек в вашем понятии ``пустой'', имеет свою глубину падения. Священный человек тоже падает, но у него своя ``глубина''. Вы поняли?}
\people{Угу. Ясно. Скажите так называемый синдром хронической усталости - это не воздействие тёмных цивилизации?}
\soul{Нет. Это не умение пользоваться вашей энергией. Мы когда-то говорили вам, что значит умение. }
\people{Хорошо. Последний вопрос, извиняюсь. Управляема ли человеческая цивилизация, как мы подозреваем всё-таки?}
\soul{Да. Но управляют вами не те, о ком вы думаете.  }
\people{(Ольга) Наши родители? }
\people{(Белимов) А кто тогда управляет? }
\soul{Давайте скажем так, мир, в котором вы живёте, и в котором вы являетесь. Мир. А в мир входите и вы. Значит, и вы управляете собой, и значит, управляют и иные, ведь вам приходится подстраиваться, как вы говорите ``под кампанию''. Разве это не управление? Вы хотите управление ``выше'', а много ли вы слышите высшее? Чаще - вы заняты,  чаще - вы в бегах,  вы работаете. Чаще - вы боитесь остаться наедине с самим собой и посмотреть? что же вы натворили что вы наработали. ``Зачем? - Будет ещё время''. Вы боитесь старости, ибо, опять - плоть ваша. Вы боитесь, молодости и мечтаете стать взрослыми, ибо, опять - плоть ваша. Ребенок, мечтающий стать взрослым - вот что самое грустное. И старый человек мечтающий стать младенцем, это тоже, грусть.}
\people{Хорошо. Вопросы нашего коллеги, который он обозначил как общекосмические темы. И вот, первый из них. Как понять утверждение о бесконечности вселенной, но, в то же время, о конечной численности цивилизации, что их якобы около 69 миллионов? Это получено по контактной информации.}
\soul{Давайте скажем так: что такое бесконечность? Что в вашем понятии бесконечность? Вселенная не знает бесконечности. Любая физика - конечна. Мы говорили вам о том. Значит, и вселенная ваша конечна. Если вы бежите по кругу, это не значит, что она бесконечна. А что находится за вселенной, это не можем мы вам объяснить, ибо нет у вас этих слов. И вы не сможете понять. Если вы попадёте туда, в вашем понятии, это и есть смерть. Ибо там нет вашей физики. Там совершенно иная вселенная совершенно иное всё. И потому, вы не можете попасть туда. Вот вам понятие анти-материи. В высшем понятии, а не в том, что представляете вы. Далее. Вселенная, можно сказать: бесконечна. И бег по кругу, это, всё же, бег вперёд, для вас, ибо, вы не можете управлять временем. И придя, вернувшись, откуда вы пришли, для вас - это будет другой мир. Ибо он изменился за то время, которое вы бегали. Вот вам и бесконечность. Далее, есть множество вселенных, и все они конечны. И эта конечность – бесконечна. Ибо их столь великое множество, что уже нельзя говорить о цифрах. И скажем вам обратное: ``конечны ли в вашем понятии цивилизации?''. Как же вы тогда совместите: ``мир един'', если вы уже разделили на ``цивилизации''?  Так и скажите: ``множество ли цивилизаций имеют вашу физику, именно вашу плоть или понятие, входящее в вас -  инопланетяне?''. Спросите тогда так. -  Множество. И в конечной вселенной они, естественно, конечны.}
\people{Но, не обязательно 69 миллионов да? Это дезинформация?}
\soul{Да поймите же вы, что даже вас на Земле множество! Множество, в вашем понятии ``инопланетян''. Те же животные, для вас, ``инопланетяне''. Каждый из вас, для вас, ``инопланетянин''. Даже жена вам ``инопланетянин'', вы не можете её понять. ``Она устроена совершенно по-другому, у неё совершенно другие понятия о мире'' Ну, давайте её посчитаем тоже. И в то же время, можно сказать, что мир един. Вы же живёте рядом с женой и рядом с другими и животными. Вы согласны? Разделите здесь бесконечность и конечность.}
\people{(Ольга) Скажите, тогда, вот допустим, человек познавая мир… или как? Открывая, расширяет, будем говорить, свой круг действии что ли… или как ? Жизни…  или, открывая для себя жизнь и познавая, допустим, всю вселенную, став тем, кем хочет стать, наверное, будет хотеть познать и другие вселенные?}
\soul{Что в вашем понятии ``конечная цель познания''?}
\people{Ну, вот допустим, мы исследуем тело, да? Только это тело.}
\soul{Давайте скажем так, исследуя только это тело, вы примитесь за другие тела - только и всего. Дальше? Став вселенной, вы сможете продвинуться далее. Когда вы станете самой вселенной, вы поймёте бесплодность попыток разделить мир, ибо вы уже ``вселенная''. И тогда, вы уже можете назвать себя ``частицей космоса''. И тогда, вы уже будете пытаться познать космос, но не другие вселенные, ибо это уже будет насилие над другими.}
\people{(Белимов) Хорошо. Скажите, разум это постоянное и характерное свойство материи? А могут ли существовать материи без разума, а разум без материи?}
\soul{Определите термин ``разум''. Ибо здесь уже заключается ваша ошибка. Что в вашем понятии ``разум''? Вы объявляете, животные обладают разумом, и тут же доказываете, что животные не обладают им. Вы говорите: ``человек'', и тут же различаете ``человек'' от ``природы''. Где ваш разум? Вы говорите: ``природа разумна'', и тут же говорите: ``стихия''. Где разум? Каково ваше понятие разума? Именно вашего? Тогда, можно сказать, что вы единственны.}
\people{Скажите, первичны ли от природы энергетические цивилизации, или они эволюционировали из плотно-телесных цивилизации?}
\soul{И, да и нет. Есть множество, и множество имеют разные пути.}
\people{Но не обязательна ли стадия прохождения в плотном теле?}
\soul{Нет. Согласитесь, что ``свет'' в вашем понятии - тонкая материя.}
\people{(Белимов Ольге) Задавай.}
\people{(Ольга) Скажите, вот вы говорите об истинном мире и об иллюзорном, напридуманном человеком. Человек может отличить мир, который он придумал, и другие, допустим, которые он чувствует? Тот истинный мир, который создал Бог, будем так говорить. То есть, ту истину, которую мы должны понять.}
\soul{Давайте скажем, что вы не живёте в полном цивилизованном мире. Вы видите истинные картины, но искажаете их через себя. И, по вашему - вы живёте в иллюзорном мире. Относительно других, вы можете сказать,  что вы живёте в истинном. Ибо он будет видеть опять искажённо - своим. Бог создал истинный мир, но вы не помните его. Вы не знаете его. И когда вы придёте в него, вы опять будете говорить, что это иллюзорный мир. И действительно, к тому времени, когда вы придёте, он действительно будет иллюзорен. Ибо всё эволюционирует в вашем понятии. Поймите, Бог не остаётся на месте, он также познаёт и себя. И потому, мир, созданный вами, иллюзорен, и мир созданный Богом, Бог назовёт иллюзией тоже, ибо он идёт дальше, ибо он тоже растёт.}
\people{(Белимов) Скажите, объясняется ли существование параллельных вселенных так называемым парадоксом скрытой массы вселенной, что якобы 39/40 это скрытая масса невидимого вещества?}
\soul{Физически, да.}
\people{Верны ли наши предположения о том, что чем ближе к центру галактики, тем более старшие и более высокоразвитые цивилизации там живут?}
\soul{Нет.}
\people{Это подчиняется закону случайных чисел?}
\soul{Нет. Не зависит геометрия от разума. Вы можете находиться в любой точке и можете быть и разумны и нет.}
\people{Скажите, верны ли наши догадки, что так называемые чёрные дыры и вспышки сверхновых звёзд это единые явления, массы энергетического обмена между параллельными вселенными?}
\soul{Нет.}
\people{Мы ошибаемся тут, да?}
\soul{Давайте скажем так, что это червоточины пронизывающие ваши вселенные. Именно через эти червоточины вы можете увидеть космос. Всё остальное - ваши вселенные. Имеет понятие ``космос'' и ``вселенная''. В космосе есть множество вселенных. Вы же говорите наоборот, или совмещаете вселенную и космос вместе. Хотя, тут же доказываете, что среди них может быть много, и приводите множество примеров. У вас даже есть теория образования вселенных пузырьков, ибо это пузырьки. Ну, что же, отчасти это верно. Но только отчасти, ибо космос велик, и значит, велики множества созданний вселенных. Ваша вселенная - физика. Ваша вселенная, если хотите, чем ближе к центру,  то тогда, логически, было бы сказать наоборот, ибо, чем ближе к центру, тем больше плотность. Тогда, что в вашем понятии: ``чем плотнее тело, тем мир…''. }
\people{Ясно. Скажите, вечна ли информация в хранении, не исчезает ли она со временем?}
\soul{Никогда. И мы говорили вам. Вспомните, мы проводили пример, о вас. Когда эфир, излучающий ваше изображения и вашу речь, и когда-то через множество лет на другой планете вы же, поймав их… Вы помните, что говорили вам?}
\people{Помним. Надо же, это огромные…}
\soul{Ну, вы же, сказали, что никогда и ничто бесследно не исчезает, и тут же говорите об исчезновении информации.}
\people{Нам трудно понять.}
\soul{Вам трудно вычислить. Вам трудно найти из этого хаоса. И потому, множество миллионов, в вашем понятии, лет, можно уже было предположить и сказать, что вы родились, и когда вы родитесь, что вы будете делать и заниматься, ибо всё это закономерность. Если хотите - математика. Та самая математика, которую вы не хотите признать, ибо она не касается духовного. Но, поймите, в данном случае, мы говорим о физике.}
\people{Скажите, на какой энергетической основе существует информация и в каком виде она хранится?}
\soul{Давайте скажем так, вы придумали понятие ``информационное поле''. Вы примерно предполагаете, что это тонкие тела. Мы же, сегодня говорили вам о колебаниях, которые вы будете принимать за постоянный из-за отсутствия их. Вы помните?}
\people{Да.}
\soul{Только мир ваш, информационный, состоит из мельчайших, в вашем понятии, частиц, где состояние покоя - это уже скорость света.}
\people{Скажите, есть ли тот, кто заведует всей информацией?}
\soul{И да, и нет. Да. Это все вы. Если хотите, коллективный разум. Само информационное поле. Вы и есть то ``информационное поле''. Вы - носители его. Вы - создаёте его. Мы вам говорили о множестве богов. Вы помните?}
\people{Помним.}
\soul{И проводили вам примеры. Вы создали этих богов. Вы. Это ваши фантомы. Ибо, человек, мечтающий о чём-то? определенной профессии, его мечты об идеале совпадут с другим идеалом этой же профессии. Вы согласны? Если вы мечтаете, стать великим журналистом, согласитесь, что любой другой журналист будет мечтать о том же самом, и будет представлять то же самое. Вы согласны?}
\people{Да.}
\soul{И этот великий журналист и есть Бог журналистики, если хотите. И вы можете сказать, что он управляет вами? – Да. Ибо вы продвигаетесь к этому идеалу. Значит, он управляет вами. Вы согласны?}
\people{Да. Можно..(согласиться прим.)}
\soul{И, в то же время, мы можем сказать, что вы управляете этим идеалом, ибо вы создали его. Вы можете разграничить эти понятия? Нет. Вот вам и ``единство''.}
\people{Скажите, это предположение или действительно, что якобы сбор людей в количестве 12-ти человек рождает какое-то уже другой разум, более объёмный, более умный чем, если бы их было три, пять?}
\soul{Давайте скажем так, что это уже говорит вам память предков. Когда-то существовало такое верие: если съесть мозг кого-то, то станешь обладать его умом. Вы помните?}
\people{Да.}
\soul{Каннибализм. Вот вам, и остатки его.}
\people{Но почему именно 12 человек, 12 заседателей? Это ошибка? Это атавизм человечества?}
\soul{Почему у вас 12 месяцев?}
\people{(Ольга) 12 знаков зодиака.}
\people{(Белимов) В общем, это не влияет? Но с другой стороны, мы знаем, что сообщество саранчи допустим, грызунов, вдруг рождают очень правильные действия мыслительные. Получается - каждый маленький умишка объединяясь в большую, рождает довольно интересный разум.}
\soul{И приходит один человек и может изменить весь мир.}
\people{Ну, да. }
\soul{Откуда же он взял эти силы? Вы дали ему, или он взял у вас?}
\people{Наверное, он в себе вобрал весь разум коллективный. }
\soul{Вы будете правы при любом ответе. Ибо он использовал вас, ибо вы хотели того. }
\people{(Ольга) Так лидера выбирают, да? По такому принципу? Скажите, а верно ли такое предположение, что неиспользованная энергия мыслей людей рождает колонию, допустим, насекомых, таких, как саранча или ещё каких-то?}
\soul{Нет.}
\people{(Белимов) Скажите, как связано развитие высшего разума с возникновением новых вселенных?}
\soul{Зачем вам этот вопрос?}
\people{Это наш коллега задаёт.}
\soul{Вы разберитесь в своей, вы разберитесь с самим собой, а уж потом думайте, как создавать вселенные. А каком ``высшем'' разуме говорите вы?  Вы обладаете этим разумом, а толку? }
\people{(Белимов) Мы не сказали бы, что им обладали.}
\people{(Ольга) Мы обладаем, мы не умеем его… это…}
\people{(Белимов) Да.. .использовать. КПД низкий.}
\people{В процессе развития разумных существ постоянно происходит укрупнение их сообществ, так ли это?}
\soul{Да. Вы же говорили только что, о коллективизме.}
\people{Какой объём материи в процентах перерабатывается и осваивается разумом в масштабах вселенной?}
\soul{Именно вселенной или только вашей Земли? Если Земли, то Земля изменяется с каждым мгновением. Если хотите в процентах, то пусть будет сто. Здесь нельзя разделить, или вы влияете на Землю, или Земля влияет на вас. Всё это столь тесно связано, что нельзя говорить о разделениях. Согласитесь, экологическая катастрофа создана вами. Вы согласны? Природу изменили  вы. Но если совершается катастрофа, она меняет и вас. Вплоть до уничтожения. Вы согласны? Так, кто кого изменяет?}
\people{(Ольга) Взаимная зависимость, наверно.}
\soul{Мы говорили вам о шёпоте Земли. Вы помните?}
\people{Да}
\soul{Мы говорили, что каждый из вас должен слышать этот шёпот. И каждый из вас, если услышит его, в вашем понятии, физически будет уничтожен. Духовно же, вы никогда не покидали Землю. Ибо это мир, в котором вы живёте. Мы же говорим - о единстве мира. Зачем тогда Земле разговаривать с вами? Зачем? Если вы частица Земли. Зачем говоря с Землей, вы уже хотите отвергнуть себя от неё. Уже вы говорите: ``это Земля, а это я''. Это совершенно разные вещи. Вы живёте на Земле. Давайте не будем говорить о душе. Плоть ваша с  Земли? Так и живите на Земле.}
\people{Но московские исследователи говорят, получают очень интересные результаты от общения с Землёй, и они моделируют новые ситуации и убеждаются, что Земля выполняет даже их просьбы. Как к этому отнестись? Это фантазия их?}
\soul{Это желания,- первое, второе,- это ваш фантом. Но нельзя назвать это ложью. Нельзя, ибо это опять одно из иллюзий. И относительно себя, нет лжи. Относительно постороннего, это ложь, и да, ибо посторонний не может услышать шёпота, и потому, он будет сомневаться, и будет обвинять вас во лжи, хотя постоянно разговаривает с Землёй и сам. И каждый из вас разговаривает с Землёй. Примеры? Пожалуйста: Гомеопаты.}
\people{Ясно. Скажите, можете ли вы перечислить и охарактеризовать известные типы духовных энергий - материи тончайшего уровня?}
\soul{Это вы придумали их названия. Когда-то мы вам говорили о памяти вашей. Вы помните? Поднимите кассеты и послушайте. Мы говорили вам о химической памяти. Говорили вам о биологической памяти. Говорили…(обрыв записи)}
\people{Какими видами энергий пользуются энергетические цивилизации? Каким образом они управляют состоянием материи?}
\soul{Зачем вы спрашиваете так далеко? Зачем? Вы хотите узнать о других мирах, не зная о своём. Вы не можете сказать, как управляетесь вы, а хотите узнать, как управляются соседи. Зачем? Любопытство?}
\people{Может он пробрался уже в далёкие понимания…Наверное, любопытство, да…}
\soul{И к чему приведёт ваше любопытство? Вы хотите узнать на стороне. Хотите узнать всё о соседях, и не знаете, что творится у вас дома. Согласитесь, что это глупо.}
\people{Не можете объяснить нам, с какими процессами связаны полиморфные превращения кораблей фантомов цивилизации высшего звена? Это часто замечают, когда корабль вдруг превращается в диск, из него разбивается на сектора и грушевидное и всякое разное.}
\soul{Это делаете и вы. Давайте представим так: Ребенок, взявший в руки пластилин. Он может пластилину придать любую форму. Вы согласны?}
\people{Да.}
\soul{Ибо для него, он мягок. Теперь, представьте тонкий мир, обладающий более тонкими телами. Для них ваша материя является тем пластилином, потому, её легко сформировать. И поймите, всё наоборот, чем тоньше мир, тем легче ему управлять более плотным. }
\people{Скажите, существует ли иерархия высокоразвитых цивилизации, и иерархия высших разумов?}
\soul{Да.}
\people{Взаимоподчинённость есть такая?}
\soul{Да.}
\people{(Белимов) Она положительную роль играет?}
\people{(Ольга) Конечно.}
\soul{Да.}
\people{(Белимов) А какова их структура?}
\soul{А чем отличается от вас? Да ничем. }
\people{Осуществляется ли связь с высшим разумом в том случае, когда человек просто подумал о нём? Если осуществляется, то в каком виде?}
\soul{Чаще, нет. А если быть точнее, вы всегда в своём роде контактёры. Вы никогда не теряли контаков. Вы же говорите: ``осенило''. Вот вам, пожалуйста. Но, это не значит, что вы просто машина, которая принимает осенения. Нет. Это заслуга ваша, что вы смогли настроиться и услышали. И потому хвала тем поэтам, иначе без этих поправок вы скажете, что нет знаменитых людей. Ибо любой поэт, это всего лишь ``высшие силы надиктовали ему''. Нет. Для того чтобы услышать тот диктант, тоже нужны силы. И это тоже надо заработать.}
\people{Какова скорость перемещения мысли в пространстве? На сколько порядков отличается от скорости света?}
\soul{Мы говорил вам, и повторимся - состояние покоя это скорость света. А значит, есть и выше.}
\people{Существует ли цивилизации, перемещающиеся со скоростью мысли? Что это за цивилизации?}
\soul{Множество. Столь огромное множество, что вы не можете это представить. Если говорить вашими числами, то нет ещё вашего числа. Давайте скажем так, ваши двойники, те самые фантомы, которые создаёте вы, не самые образы, которые создаёте вы, не миры ли это? }
\people{Нам трудно это всё понять. Слишком новое для нас это.}
\people{Скажите, каким образом разум может фокусироваться и проникать в микромир? Изучать строение атома?}
\soul{Вот, если бы вы были бы разумом, действительно разумом, вам не надо было бы никуда проникать, ибо вы уже были бы везде. Вы есть везде, но не можете осознать того. И чем отличается микромир от макро? Чем?}
\people{(Ольга) Ничем.}
\people{Каким образом осуществляется телепортация плотнотелесных индивидов - представителей цивилизации первого типа?}
\soul{Вы уже задавали такие вопросы.}
\people{(Белимов) Ну, ладно.}
\people{(Ольга) Скажите, а можно вот мысленно задать вопрос вам, а вы можете мне, ну, не мне – любому, ответить?}
\soul{Можно, но не желательно это делать всегда. Вы должны были бы заметить, что иногда мы делаем то. Но, это зависит от ``переводчика'', ибо вы, каждый из вас, а мы говорили и повторим, есть контактёры. Каждый из вас живёт в едином мире, и каждый из вас ощущает его. Лишь только сознание ставит границы. Мы говорим вам - сознание спит. Но оно спит и, всё равно, охраняет. Охраняет, каждого из вас. Вы же можете проснуться, если вас будят, вы согласны?}
\people{(Ольга) Да}
\soul{Далее, есть, когда сознание мертво. Вы это называете ``твёрдыми телами'', когда нет сознания в камне. Сознание мертво, но относительно вас. В камне тоже есть жизнь, и относительно вас, он тоже мечтает, он тоже хочет, он тоже ищет. Но, в вашем понятии, он мёртв. Для высших миров, ваше сознание тоже мертво, как и мёртв камень для вас. Поэтому многие высшие не могут видеть вас. Да, они понимают и допускают, что вы живы, что сознание ваше живо, как и вы допускаете, что камень жив. Вы согласны?}
\people{Да.}
(сбой контакта)
\soul{…независимо – высшее/низшее, ибо относительно каждой точки можно только замерить высшее и низшее. Те же высшие, как вы считаете, тоже говорят о высших и низших, и они тоже считают себя средними и тоже ищут и тоже бояться упасть, или тоже хотят подняться. У них тоже есть физика, они тоже хотят уйти. Вы говорите ``огонь'', ``огненный мир''… Вы приходите в огненный мир, и вы видите, что надо идти дальше, что здесь тоже физика, и вы тоже хотите потерять плоть. Там, тоже есть плоть, там, тоже есть физика, которая держит вас. А вы говорите ``энергетика''!}
1995-12-03_02
\soul{Всё. Вы должны помнить. Вспомните.}
\people{Да, да..}
\soul{И он испугался.}
\people{Особенно политическую деятельность.}
\soul{Ибо, вы всегда хотите понять всё сознанием, умом, и не хотите применять чувства. А если применяете, то это, в основном, страх, ибо он самый сильный у вас - страх и жадность.}
\people{Да. Ну, хорошо, мы тогда задаём вопросы нашего коллеги. Я заранее извиняюсь, если что-то есть повторяющееся. Вот, его интересует, каков процент жителей Земли, по вашему мнению, который реально оценивает сегодняшнее состояние человечества и в целом самой Земли?}
\soul{Ни одного. Все вы живёте в мире иллюзий. Каждый из вас живёт в своём мире, каждый, и никто не может оценить. И как вы можете оценить, если вы слепы, если не видите? Если видите, то только маленькие, маленькие точки на большой картине. Вы не можете сказать о всей картине, а потому,  не можете видеть реально. Вы, мир ваш преобразовываете относительно себя, потому нет согласия меж вами, потому и войны, потому не понимаете друг друга. И если когда-то находится человек, с которым вы входите, в вашем понятии, в резонанс, вы это называете любовью, но очень быстро теряете.}
\people{Угу. Скажите, но если в данный момент появится вдруг кто-то, действительно  кто понял, вот 0,1% там, тысячные доли, - то он сразу приобретёт статус пророка, так что ли?}
\soul{Нет.}
\people{Будут ли такие люди?}
\soul{Нет. Вы не можете привести никакого статуса, кроме плохого, ибо вы будете говорить, что он слеп, врёт, или ещё хуже.}
\people{Скажите, а когда человек покидает, будем говорить, свое тело физическое, он тоже частично реальность видит, тоже так, да?}
\soul{Да.}
\people{А продвигаясь дальше, допустим, некоторые…}
\soul{Давайте скажем так: герои Шамбалы тоже видят не всё реально.}
\people{Но видят больше, конечно, да?}
\soul{Да.}
\people{Реально, наверное, может только оценить всё  только бог наш?}
\soul{Когда вы не будете в физике, тогда вы сможете увидеть всё реальней. Иначе - вами управляет физика, материя. Материя же груба, потому и не даёт вам полных картин. Подумайте, чем массивней тело, тем более низкие колебания создает оно. Вы согласны?}
\people{Да.}
\soul{Теперь, представьте: частица столь малая и колеблется столь большей скоростью, что вы уже не можете принять её, ибо она колеблется с большей частотой, чем ваша скорость света. Как вы воспримите это? Вы воспримите, что она вообще не колеблется.}
\people{Да.}
\soul{Разве это правдиво?}
\people{Вы правы. Скажите, мешают ли эмоции и рационально-чувственное восприятие мира развитию человечества как цивилизации?}
\soul{Только крайние эмоции. А вообще-то, весь мир ваш устроен на эмоциях, все вы делаете через эмоции, и крайности могут вас сделать героем, или наоборот.}
\people{Но у нас ощущения, что они где-то, может быть, мешают нашему развитию, поскольку иные цивилизации, другие, они, как правило, лишены эмоционального плана.}
\soul{И тогда вы будете машинами.}
\people{Ааа, то есть если, если другая крайность…}
\soul{Мы же говорили вам. Почему вы невнимательны и не помните? Мы же говорили вам, чем отличаются машины:  тем, что вы умеете ошибаться и умеете запоминать ошибки, и исправлять их. Чем люди отличаются от животных? Тем, что вы умеете мечтать и стараетесь воплотить эти мечты. Беда только в том, что мечты ваши в детстве честнее и чище, чем сейчас. И идеал ваш непостоянен, ибо меняется относительно вас, образованием и… и… и…}
\soul{1… }
\people{Скажите, как человеку преодолеть страх перед неизвестным?}
\soul{Как преодолеть страх?}
\people{Да.}
\soul{Уйти от страха. Что для этого нужно? Страх, чтобы испугаться этого же страха, – только так вы сможете.}
\people{Только так? А никакого особого вида кодирования там нету?}
\soul{Это всего лишь установка. Если хотите, из вас делают ``машины'', `` роботов''. И если вам приходят и открывают третий глаз, вы становитесь всего лишь машиной, придатком открывающего. И что увидит этот глаз? Только то, что вы захотите, и не больше.}
\people{А скажите, может быть   для развития общества, хорошего, целеустремленного  дальше, только бесстрастие и холодный ум нужны?}
\soul{Вы повторяетесь. }
\people{Но нам кажется, что это действительно обедняет человечество. Хорошо, мы повторяемся. Тогда откуда, по вашему мнению, в человеке возникло агрессивное миро-приятие… мировосприятие?}
\people{Страх?}
\people{Тоже - всё страх, да? И желание поработить?}
\soul{Страх и жадность, основанные на страхе. Только не подумайте, что мир ваш состоит только из жадности и страха, нет. У вас есть множество прекрасных эмоций, но вы забываете о них, и когда они просыпаются в вас, вы пугаетесь: ``не сошел ли я с ума?'',- и стараетесь вернуться на прежнее место: то бишь опять - бояться, то есть, опять - не выделяться.}
\people{А скажите, можно ли считать источником жестокости современное общество, суровые условия борьбы за существования в первобытном обществе?}
\soul{Нет, это всего лишь только ваше оправдание.}
\people{Но, оно же реально было, действительно.}
\soul{Хорошо, есть же множество детей, которые воспитывались в прекрасных семьях, и что? Они были прекрасны?}
\people{Не всегда.}
\people{Значит, выходит, что человек сам, в общем-то, творец собственной жизни, будем говорить,  и вот, что сейчас происходит, – это то, что мы натворили в прошлом, по всей видимости, да?}
\soul{Только вы и больше никто, всё остальное – лишь оправдание.}
\people{А скажите, был ли жизненной необходимостью культ оружия, провозглашаемый людьми во всю свою историю?}
\soul{Для вас - да, ибо вами руководит страх: страх быть побежденным, страх быть последним и страх быть первым тоже. И потому, вам легче защищаться. Если не хватает мозгов, применяется сила.}
\people{Ну, так это что, до скончания человеческой цивилизации будут войны и …}
\soul{Нет. Когда вы станете человеком, вы это забросите.}
\people{Ну, мы, вроде, уже - человеки…}
\soul{Не похоже.}
\people{Не похожи? Н-даа… Можно ли постичь энергии природы, развиваясь только духовно, на чём вы, в принципе, настаиваете?}
\soul{А вы только и должны это делать. Можете ли вы постичь природу инструментом, можете ли вы познать Землю, копая её экскаватором?}
\people{Ну, мы же ряд энергий, всё-таки,  добились именно не духовным, а материалистическим развитием.}
\soul{Всего лишь только ряд. Если быть точнее - всего лишь только маленькие кусочки, которые губят вас, ибо вы открываете их, не смотря и не видя те дыры, что натворили вы, и вы стараетесь забить их. Чаще, вы забиваете их, опять же – эмоциями. Эмоциями страха. Согласитесь, что эмоции – это тоже физический мир, он так же реален, как и вы сами. Ибо, что - ваши эмоции? Это изменения вашей химии. Вы согласны?}
\people{Да…}
\soul{И тогда, любой физик и химик скажет вам, что - да, вы не замкнутая система. И потому, мир эмоций – тоже ваша физика. Если вы добыли какую-то энергию и взяли только кусочек, а если быть точнее, украли у природы, и этим сделав ей больно,- она не будет вам мстить, нет,- она будет стараться залатать эту дыру. И тогда уж, простите, виноваты вы.  Но вы, опять же, обвините не себя, а других: нечистый попутал или что-то ещё. Если к вам приходит ``нечистый'', вы же виноватый! Кто же ещё?}
\people{Но не слишком ли природа, действительно, гуманна, если она так не… старается залатать дыры, а не ``дать по рукам'' человеку?}
\soul{Вы говорите о гуманности. Вы даже не знаете, что это такое, а говорите о гуманности. Вы чувствами своими - измеряете природу. Неужели вы думаете, что природа имеет те же самые эмоции, что и вы?}
\people{Скажите, переводчик, в общем-то, с некоторой иронией сказал о героях Шамбалы. Это связано с его личным отношением, так сказать? Нет?}
\soul{Зачем же? Мы говорим вам, что никто не может видеть реальность, никто, пока находитесь в физике. Астральные тела ваши – тоже физика, какой бы она тонкой ни была, это, все же физика, и физика ваша, земная.}
\people{Но ментальный мир ведь тоже, наверное, физика?}
\soul{Давайте скажем так. Вы говорите ``огненный мир''. Что - огонь? Огонь, это тоже - материя, только более тонкая и всё. И потому, здесь нет иронии. Печаль в том, что вы не видите даль. Мечты ваши слишком малы. Вы хотите достичь Шамбалы? А что будете делать там? Установите свои законы? Ну, хорошо, давайте так: вы духовны, сможете попасть в Шамбалу – и что? Страх в том, что вы можете остановить и сказать: ``Я уже достиг мечты своей. Что же ещё дальше?'' А чаще всего вы так и делаете. Или вы разочаровываетесь, потому что вы ожидали большего, и опять уходите, а потом говорите, что в Шамбалу поднимается всего один человек в сто лет, остальных изгоняют. Нет, вы уходите сами, потому что вы хотели одно, а получаете другое. Вы думали, что придёте туда отдыхать – наоборот. Вы мечтаете о рае и не знаете, что попотеете там более, чем здесь.}
\people{Интересно. Скажите, фатальна и неизбежна ли революция потрясений в развитии человеческой цивилизации?}
\soul{Да, для вас, да.}
\people{Но это прогресс означает, да?}
\soul{Не обязательно.}
\people{О революциях, можно…?}
\people{Известны ли в мире цивилизации с плавным, не ступенчатым ходом эволюции? У нас же в основном ступенчатая, резкая, а в мире есть ли?}
\soul{Нет.}
\people{Нет? Не бывает?}
\people{Плавная она, как раз, идёт…}
\soul{Поймите. Что в вашем понятии ``революция''?}
\people{Ну, какой-то резкий поворот, потрясение с изменением…}
\soul{…что заставляет встряхнуть вас, что заставляет вас сбиться с курса, которым вы идёте и искать вам новый курс – вот что ``революция''. А что вы представляете ``плавно''?}
\people{Ну, это эволюционное развитие, постепенное набирание…}
\soul{Даже природа не знает плавности, даже в природе существуют катастрофы, как вы говорите, ``перерывы истории''.}
\people{Ну, хорошо. Ясно. Скажите, а каков предел терпимости природы по отношению к человеку в период его, ну, грубо-материалистического познания? Мы же познаём, мы невольно наносим природе ущерб. Долго она будет терпеть?}
\soul{Что даст вам ответ?}
\people{Ну, предел терпимости, наверное, трудно сказать.}
\people{Да он уже на пределе, по-моему.}
\people{На пределе, да?}
\soul{Спрашивайте далее.}
\people{Так, хорошо, а наступил ли этот предел?}
\soul{Спрашивайте далее.}
\people{Можно ли утверждать, что на Земле начались неотвратимые процессы деградации биосферы?}
\soul{Прекрасно! Опять виноваты кто-то, но не вы.}
\people{Нет, ну, через нас, через нашу деятельность. Мы не снимаем с себя этого. Это необратимый процесс, или всё-таки ещё можно что-то поправить?}
\soul{Поймите, вы боитесь  -  что? Смерти? Физической или духовной?}
\people{И то, и то, наверное.}
\soul{Поймите, вы привыкли находиться в своём теле, и любая его потеря или хотя бы травма, причиняет вам боль. И заметьте, боль духовную. Почему? Что такое ``духовный мир''? Как можно придти в него? Всё зависит от вас. Духовный мир есть не только ``чист''. Есть множество, в вашем понятии, уровней, и ``грязного'', и ``чистого''. И в какой уровень попадёте вы, зависит чисто от вас. И если вы считаете, что плоть ваша – это жизнь ваша,  соответственно и в духовный мир придёте, и будет только боль и не более. Но, если же вы скажете: ``Я не боюсь боли'' –  и будете сжигать себя, что уже делали неоднократно - и куда вы придёте? В мир боли, всё туда же, и будете говорить, что вы духовно выросли. Нет, вы упали, ибо вы насилуете, насилуете себя, а значит, и природа, что создала ваши плоти. Вот вам ``уважение'' природы, вот вам и ``терпение'' её.}
\people{Скажите, восполнима ли утрата живого мира Земли для космоса, если такое произойдёт?}
\soul{И да, и нет.}
\people{Скажите, когда наступает период  пролайи? То есть, вот, исчезает физический мир.}
\soul{Давайте скажем – ``перерывы в истории''. Ваш термин.}
\people{Истории? Пусть будет…}
\soul{Вы же разделили – ``доисторический'' мир ``историческим''. И основывается он у вас всего лишь только на умении писать. То есть, ``древним миром'' вы считаете мир, где не было писания.}
\people{Нет, нет…}
\soul{Далее. }
1-2-3-4
\soul{Спрашивайте.}
\people{Хорошо. Какова уникальность живого мира на Земле?}
\soul{Вы невежливы.}
\people{Что? В чём это выразилось? Прервал вопрос моей коллеги?}
\people{Задавать вопрос можно дальше?}
\soul{Спрашивайте.}
\people{Какие преобладают связи на Земле между живыми существами: фагиальные или симбионтные?}
\soul{Мы когда-то говорили вам о единстве мира, а вы спрашиваете о связях.}
\people{А по терминам нельзя, вот, разделить? Всё-таки, чего больше…?}
\soul{Термины придуманы вами. Мы говорили вам, о едином, вы тут же размежёвываете. Мы говорим вам, что вы никогда не были отделены от природы, вы тут же придумали множество тел, назвали их эфирными, дали им уровни. Интересно, как вы только распределяли это? Мы же говорим вам, что мир един, а значит, нет связей, ибо нет их названий, ибо их множество, а вы же хотите разделить, вы опять хотите распилить всё на кусочки и разбросать. Вы уже когда-то это сделали и погибли. Не пошёл вам урок впрок.}
\people{Скажите, а если вам, переводчику, задать вопрос по-английски или на другом языке, он сможет ответить?}
\soul{Мы говорили вам, что мы находимся, вашим термином, всего лишь только в ``мире его эмоций''. Мы не имеем права входить в него, хотя мы можем сделать то. Вот только это будет не переводчик, а, как вы говорите, зомби. И тогда, он сможет сделать всё. Всё, что вы хотите, всё, что вы пожелаете. Это сделаем мы через него, но что это даст? И тогда, контакты ваши, как вы говорите, приведут к ``печальным последствиям''. Вами сказано.}
\people{Нет, мы этого не хотим…}
\soul{Вы не знаете, что такое ``неудача'' в контактах. Практически, вы всегда разговаривали, вы согласны?}
\people{Согласны.}
\soul{Причина? Причина в том, что мы не хотим пугать ни вас, ни его. Для того мы должны придти собеседником, но не в него. Потому и говорим вам - ``беседа''. Мы же, всего лишь советуем, если хотите, подсказываем, но не имеем права указывать. Мы такие же ищущие, как вы, – это первое. Второе: мир ваш – не наш, и потому, мы здесь ``гости'' и многое не знаем и делаем множество ошибок, как это было в прошлый раз, в прошлой жизни, когда вы помнили всё. Больше мы не хотим повторяться, мы не хотим ошибаться,  мы не хотим тревожить вас, мы не хотим прерывать, в вашем понятии, контакт, потому что это зависит только от вас. И потому, да, теоретически, вы можете изучать и разговаривать на любом  языке, но, чаще, мы стараемся использовать словарный запас, запас, который не заставляет напрягать, в вашем понятии, его мозги. Потому говорим вам, не применяя терминов, не говоря имён и чисел.}
\people{Скажите, значит, если бывают такие, ну, будем говорить опять, ``контакты'' или ``медиумы'' здесь некоторые, которые там спиритические сеансы проводят, если они, вот, начинают разговаривать на других языках или что-то там применяют? Термины, которые не свойственны, допустим, в жизни этого человека, то, значит, это можно говорить об одержании?}
\soul{Нет, это говорит о том, что с вашим же миром говорят с вами, мы же - пришли из другого. И мы же говорили вам, что мы инородны, инородны для вас. Мозг ставит защиты, зачем же будем нарушать его, зачем мы будем пугать его? Те же, которые могут войти, в вашем понятии, глубже, необязательно одержимы, просто говорят из вашего же мира, в вашем понятии, из параллельности или ещё откуда, но все с вашего мира, с вашей физики. Мы же создаем только реакцию, мы не можем изменить что-либо в вас и вашем мире, мы можем только изменить относительно медиума, но тогда это, мы говорили вам, – вмешательство.}
\people{Скажите, а когда-нибудь может наступить такой момент, когда, ну… мы же едины, вы сами говорили об этом, соединимся что ли, или как, то есть…?}
\soul{Когда вы осознаете, что вы едины. А пока вы говорите: ``Я знаю, да, мы едины''. И что из того, что вы знаете?}
\people{Только знаем, да? Выходит так.}
\soul{Тогда не нужны будут контакты, тогда вы будете знать и видеть всё. Сейчас же вы - страх, страх, неверие. Вы же не верите ни во что! Вы пытаетесь только, вы только обманываете себя, что вы верите. Если бы вы верили, вы бы сделали. Когда-то один из вас ложился – и что из того? Неверие.}
\people{А что помешало?}
\soul{Неверие. Только это. Вы говорите - ``сомнение''…}
\people{Вы знаете, мне сон приснился, что такая вот ситуация, мне предлагали войти в такое же состояние.}
\people{Ну, у тебя получилось?}
\people{Я не помню.}
\people{Может быть у коллеги попробовать? Когда-то.}
\soul{Пробуйте. Мы же говорили вам, что мы никогда не уходили. И, в вашем понимании, ``контакт'' – это просто умение переводчика услышать нас.}
\people{Скажите, а сознанием он может нас… в общем-то, необязательно входить в это состояние?}
\soul{Мы же говорили о единстве мира. Можете, но вы не верите в эту возможность.}
\people{И переводчик не верит?}
\soul{Первое. Второе - вы слишком загружены, в вашем понятии, вы так заняты, что вам некогда. Дальше? Дальше, - сознание ваше не хочет, ибо оно хочет главенствовать миром, ваше сознание, потому и делаете множество ошибок. И сознание… }
\people{Скажите… Один, два… Скажите, а вера вообще, вот о вере говорится в религиозных кругах везде, вера… Что такое вера? Вот, как принять..}
\people{Мы бы рады добиться веры, вот сейчас вот нам мешает…}
\people{Может – глубокое приятие…? Я не знаю, как уже…}
\soul{Вера – это когда вы чувствуете, а ваши мечты – когда вы можете воплотить её, когда вы можете создать, хотя это грубо, давайте скажем, фантом, и можете вы увидеть и почувствовать его и пережить вместе с ним все события – вот тогда можно сказать о вере. Это всё - приблизительно. И ``фантом'' – это такое грубое слово, а вы любите его так часто применять!}
\people{А более тонкие есть слова? Мы же не знаем, терминология у нас очень бедна.}
\soul{Терминология ваша останавливает вас, ибо каждое слово имеет, в вашем понятии, жесткие объяснения, и дальше этих объяснений вы не можете уйти. И говоря о чём-то, вы уже создаёте программу, словесную программу, и дальше её вы не можете шагнуть. Дети, не знающие смысл слов ваших, потому и более легко контактируемы, ибо они не знают ограничений. Вы же всю жизнь учили, что можно и что нельзя. Вы всю жизнь использовали слова в строгих определениях, и сейчас вам не хватает их, и вы ищете новые. Сочинить - лень, чтоб подходили. Вам легче взять с других языков.}
\people{А скажите, вот есть мнение, что преобладание симбиозов – это именно обязательное условие для возникновения человека разумного. Так ли это?}
\soul{Нет.}
\people{А что тогда для разумности человека помогло? Что?}
\soul{Что такое ``разум''? Можете ли вы найти клетки, отвечающие за разум?}
\people{Не знаем. Мы боимся, что это…  Я полагаю, что это ``над-сознание'' какое-то.}
\soul{Давайте скажем так: вы даже вошли в тайны природы, называемые генОм…}
1-2-3-
\soul{… несколько слов. Даже слово, содержащее имя, уже останавливает. (объяснение сбоя в контакте близким звучанием слов генОм и Гена. Прим.)}
\people{Скажите, есть у человека имя… ну, будем…. или как…  ``вибрация'' называется, которая, вот, присущая его душе?}
\soul{Да, у вас два имени: имя, которым вы пользуетесь всю жизнь, но оно недействительно.}
\people{Временное?}
\people{… и имя, которое носит каждый из вас. И вы, только вы знаете это имя, но не можете произнести его, ибо нет перевода на ваш язык. И это имя говорит  многое о вас. То же имя, что носите вы, или, что даётся вам в церквях – всего лишь только грубый перевод, подобие. Да, оно тоже имеет значение, по нему тоже можно определить, кто вы, ваши наклонности, ваши болезни, ваше будущее, но это всего лишь приблизительно, что-то ``около да вокруг''. Настоящее имя не принесёт никто, ибо не можете этого сделать, – и вот тогда, принеся его, откроется так, что каждый, каждый может войти в него. Вы же - слишком грязны, чтобы посещать друг друга.}
\people{Но смена имен нежелательна в течение жизни?}
\soul{Пожалуйста, меняйте.}
\people{Но, говорят, что это даже губительно действует на человека. Мы читали такие очерки, что погибал человек.}
\soul{Если вы чувствуете потребность – пожалуйста.  Хуже будет, если вы будете терпеть имя, которое вам не нравится. Это будет гораздо хуже. Дальше, это будет всего лишь видоизмененный перевод. Истинное имя вы не меняете. Сколько бы жизней вы ни прожили, вы их не меняете. И там, в вашем понятии, ``у бога'', вы будете называться этими именами, а не вами выдуманными.}
\people{Скажите, вот церковные деятели, в монахи стригутся или ещё, им дают имена, это тоже связано с этим?}
\soul{Да.}
\people{Это важный ритуал? Нужный?}
\soul{Да.}
\people{Он нужен?}
\people{То есть,  отрекаются от мирского, будем так… от прошлого. Да? Выходит - так?}
\soul{Поймите, что такое ложь? Ложь – это относительно другого. Относительно вас - вы можете не лгать. Если вы будете всем говорить, что у вас есть машина, дача и многое другое, другие могут поверить, для них это не будет ложью, обманули вы только себя. Вы согласны? И если кто-то иной, сторонний наблюдатель, скажет: ``Это глупо. Здесь не вижу ничего. Это глупые ритуалы, суеверщина, и так далее''  – это его право. Для того же, кто совершает это, уже не ложь, ибо это, для него важно. Какая разница постороннему, какое носите вы имя?  Никакой. Ну, что ж, привыкнет и к этому. Для вас же, это важно. Меняя имя, вы уже хотите поменять, если можно так сказать, забыть старый мир, уйти от него с надеждой, что в этом новом мире будет всё иное, лучше или как вы хотите. Вот это и есть ваша нереальность.}
\people{Скажите, получается, что у людей, у многомилиардного населения огромное разнообразие имён, которое мы не может произнести. Так? Так ли это?}
\soul{У каждого. У каждой вещи есть своё имя, у каждого животного есть имя.}
\people{(Ольга) У каждого атома.}
\soul{Везде и всегда существовало имя. То имя, которое носит всё в вас. Если хотите, то вспомните,- сперва было ``слово''. Слово – это и есть имя ваше.}
\people{Так, а если человек каким-то образом узнал свое истинное имя, почувствовал его и хотя бы мысленно его произносит, это делает его более защищенным, счастливым человеком, или же не влияет совершенно?}
\soul{Вы не можете произнести его. Это и есть то сокровенное, что делает вас человеком, это и есть то сокровенное, что вы ищете, что вы хотите понять, что такое разум и что такое жизнь. Вы можете создать многое, но, ни один из вас ещё не создал жизнь. Даже самое простейшее. Ибо он не знает имён, и не может и не умеет их давать. }
\people{На прошлом контакте вы говорите, что в библии сказано, что Адам нарёк имена животным. Было такое?}
\soul{Да.}
\people{А это говорит…}
\people{Получается - не бог.}
\people{Это получается, что человек создал животных?}
\soul{Давайте скажем так: бог дал вам душу и дал вам знания. Бог дал вам жизнь. Остальное вам подарила природа. Дальше, ребёнок в утробе знает имя своё, ибо он ещё не умеет говорить вашим языком, ибо он ещё не придумал и не услышал ваших слов. Потому, в утробе, он знает имя своё, и потому, он может родиться. И с первым криком он теряет множество знаний. С первой улыбкой он научится любить, но забудет прошлое, ибо тогда он научился бы и ненавидеть. Проходит сорок дней, в вашем понятии, – ребёнок забывает практически всё, и он пытается произнести имя, перевести его на этот язык, но он не может этого сделать, не может. Вы же называете это ``криком''. Вы же, это называете…}
1-2-
\soul{Несущие молчание пять и более лет, они не хотят говорить, ибо они не хотят с первым словом потерять последние частицы знаний, и молчат. }
\people{Это дети особые?}
\soul{Но проходит время и им приходится говорить…  Об особенности?  Тогда вспомните ваших знаменитостей, многие из них начинали говорить, ходить достаточно поздно.}
\people{Это интересное замечание! То есть к таким детям стоит как-то особенно присматриваться, прислушиваться, да?}
\soul{Надо присматриваться, прислушиваться ко всем детям, а вы избираете.}
\people{Ну, человечество ведёт вперед гениев, всё-таки, действительно.}
\people{Чтобы его использовать в своих целях, вот так.}
\people{Хорошо. Действительно ли говорят, что паспорта, вот, номера паспортов, цифровой там ряд, неслучайно дается людям, это особая кодировка, буквально высших сил?}
\soul{Ну, если мир един, почему бы и нет? И почему ``высших сил''? Почему?}
\people{А каких?}
\soul{Вы же, вы же! Только сознание ваше не знает ничего и не хочет знать, но всё остальное-то, - от природы, всё остальное-то  - знает, но не может перевести на язык сознания. И потому, интуитивно, вы создаёте то, что вам надо, интуитивно, вы не совершаете ошибок. Совершаете только тогда, когда вы долго думаете и разумом делаете поступки. Только тогда. Самое первое случайно произнесённое слово невпопад – это и есть истинное слово, всё остальное уже – выдуманное вами, подстроенное под данную ситуацию. Вы говорите о цифрах. Да, в вашем мире одна математика, вы же говорили об этом. Да, вы подчиняетесь математике. И пока, вы - физически,-  вы будете подчиняться математике. И если вы слишком глубоко принимаете к сердцу номер вашего телефона и считаете, что он влияет на вас, простите, он будет влиять, ибо вы настроились на это.}
\people{Но ведь известный афганский математик по номеру паспорта, по серии определяет буквально судьбу и назначение человека. Как он это ухитряется? У него есть формула выведения? Он прав или это шарлатанство?}
\soul{Давайте скажем так: вы родились, вам присваивают номер, чтобы не потерять вас,- помимо бирки вы существуете ещё и в журнале,- но вы младенец , и вы не знаете того номера. Повлияет ли он на вас?}
\people{Нет, не думаю, что повлияет.}
\soul{Вот и ответ вам.}
\people{Так, а афганский ученый паспортом очень сильно оперирует, это, буквально, для него…}
\soul{Ну, разве мы отрицаем того? Мы говорим вам, что вы, вы настраиваетесь на эти цифры. Простите, если бы была возможность вас получить паспорт намного раньше, был бы совершенно другой номер. Вы согласны…}
\people{Конечно.}
\soul{…что здесь уже все взаимосвязано? Если вы родились именно такого числа, то уже можно вычислить, какого числа вы получите паспорт, и какой примерно будет его номер. Вы согласны, что это можно сделать одной статистикой, не применяя никаких ``высших'' сил?}
\people{Ну, может быть. А если я продиктую номер своего паспорта, можно мне… можете что-либо сказать обо мне?}
\soul{Видите, как все интересно, вы не знаете себя и спрашиваете от других.}
\people{Человек, познай себя.}
\people{То есть, не стоит мне паспорт…?}
\soul{Почему же? Пожалуйста! Задавайте, мы ответим. Но, поймите же, вы должны знать, прежде, себя.}
\people{Вот, я задаю номер паспорта переводчику и не спрашиваю дату своей смерти, а спрашиваю хотя бы, много лет я проживу или не очень много.}
\soul{Согласитесь, что это просто одна из уловок обмануть нас.}
(Смех Ольги)
\people{Ну…  думайте так. Мне любопытно узнать, много мне ещё предстоит жить, и могу я что-то создать творчески и так далее, или мне надо поторопиться? Вот у меня номер 3РКххххх2. Это что-либо говорит переводчику или вам?}
\soul{У вас есть понятие такое ``цифрология'', вы помните?}
\people{Да, и мы никак не можем вам курс вопросов задать, это очень хотелось бы.}
\soul{Когда-то мы говорили вам о цифрах. Когда-то даже переводчик пытался сделать это один. Давайте начнём так, а вы уже сами определите, кто вы и как долго и чего. Итак, цифра 1 – это начало счёта, вы согласны?}
\people{Да.}
\soul{Это значит, что начало рождения, и значит начало физики. Далее, цифра 2 – это говорит о том, что физика не может существовать бездуховно.  Есть понятие ``дух'', ``энергия''  – цифра 2 отвечает за энергию. Цифра 3 – это значит, что это не слепая энергия, если хотите, это энергия, несущая жизнь, в вашем понятии, растения, животные, но не человек. Цифра 4…}
1-2-
\soul{… Пять – это желание найти середину, то есть цифра обмана. Ибо вы хотите найти самый лёгкий путь, ни туда и ни сюда, пристроиться – вот вам цифра 5. Цифра 6 – это, если хотите, умение обманывать себя так прекрасно, что даже сами себя можете обмануть, потому цифра 6 вам не нравится нигде. Цифра 7 –  когда, в конце концов, вам надоедает обман, вы стараетесь найти истину. Цифра 8 – вы находите истину, но есть два варианта: или вы разочаровываетесь и возвращаетесь, или идёте вперёд. Хорошо, если вы идёте вперёд, тогда цифру 9 вы пропускаете, и вы попадаете сразу на цифру 10. В вашем понятии, что такое 10? Это начало нового счёта, но давайте вернёмся к 8 и 9. Восемь, мы говорили вам, – когда вы наконец-то прозреваете, когда вы пытаетесь уже сделать первые шаги, но вам не хватает уже времени, вам уже не хватает энергии, ибо, чаще всего, цифра 8 обозначает старость, физическую старость. Вы же, чаще всего, физическую старость принимаете ``старость'' за свою, окончание жизни. Потому говорили вам, что цифра 7 может решать, и цифра 8…}
1-2-3
\soul{Мы вас просили не называть цифры 0, вы помните?}
\people{Да.}
\soul{Почему? Ибо она и в вашем понятии обозначает ``смерть''. Ноль, несущий цвет белый или красный, ибо всегда белое и красное несло смерть. Только, почему-то, в последнее время вы считаете смерть и черным цветом – вы не правы! Черный цвет – это не рождение и не смерть, ибо это ничто. В этом ничто нет ни жизни, ни смерти.  В этом ничто находились все вы, пока не пришёл, в вашем понятии, бог и не создал мир. Каждой цифре соответствует свой цвет, если хотите, он совпадает с радугой, вот только одно ``но”: для других стран существуют и другие меры, ибо даже оттенки черного существуют тысячи. Для вас, их всего лишь только два. В том и есть цвет. Кто-то из вас сказал, что это цвет энергии. Мы говорили вам только, что это цвет смерти, это не значит, что только это. Красный цвет – это цвет огня. Огонь может согреть вас, дать вам жизнь, но может и убить вас. Любой другой цвет имеет те же противоположности. Голубой? Пожалуйста, – это небо, дающее вам жизнь, это море, дающее вам жизнь, но небо и море могут погубить вас. Назовите любой цвет – и он всегда будет противоположен. Весь мир ваш дуален.}
\people{А вот, про девятку не сказали. Девятка означает что-то?}
\soul{Девять?}
\people{Да.}
\soul{Если хотите, это глубокая старость. Не обязательно физическая, нет. Старость несёт в себе усталость, старость несёт в себе болезни, но та же старость несёт в себе и знания. А что, выбирайте сами.}
\people{А вот если в дате рождения, если в паспорте присутствуют те или иные цифры, семерки там… другие там… то есть, отсюда можно делать вывод и о направленности человека, и афганский математик так именно считает?}
\soul{Давайте скажем так, что попытка по номеру посчитать, какую жизнь живёте вы – это глупо.}
\people{Ага.}
\soul{Всё остальное, вы можете сосчитать. Да, это голая математика с теорией вероятности.}
\people{Скажите, если мы вступим в 1999 год, это не означает ли то, что… Ну, да, это… переход.}
\people{Глубокая старость нынешней цивилизации.}
\people{Столетие заканчивается, тысячелетие, и так далее, переход в новое. Это может означать и то, что? вполне возможно, что произойдёт…}
\people{Катаклизмы серьезные, так сказать.}
\people{Нет, смена, во всяком случае, чего-то…}
\soul{Ну, мы же говорили вам, что ``девятка'' – это старость. Да, здесь век заканчивается. Да, есть множество болезней, есть усталость, но старость несёт в себе и знания, а что изберёте вы, зависит от вас. Поймите, далее идёт, в вашем понятии, цифра 10 – это рождение. Это новое рождение, и если вы прожили все эти ваши цифры пустыми, и, в вашем понятии, пришли ``девятки'' пустыми, то 10 – рождение, но тоже будет пустое, ибо вы пришли к нему пустыми, пустое и родите. Всё зависит от вас. Только никогда не обвиняйте цифры – цифры выбираете вы, мы проводили вам в пример статистики, и если вы слишком много придаете внимания цифрам, вы даёте себе установку. Может, вы всю жизнь обладали одним характером, а к вам пришли и сказали: ``Да нет, по цифрам вы другие!'' - вы посмеётесь, но где-то в глубине вы это запомнили, и постепенно будете меняться и устраиваться. Потому, говорим вам, не приходите к гадалкам, ибо они уже настраивают вас и дают вам установку, как жить. Если они говорят вам, что вы умрёте тогда-то и тогда-то, есть множество шансов, что это исполнится, ибо вы будете готовить себя к этой дате. Потому, не даётся вам знание будущего.}
(начало третьей части 3-12-95)
\soul{…Что бы придти, в вашем понятии, в конечную точку – хотя её не существует – вам нужно избавиться от всех материй, а значит, избавиться от энергий. И если мир энергетический, это не значит, что он ``высокий'', он просто немного ``выше'' вас. Относительно. И для постороннего наблюдателя это будут одинаковые миры, только что разные, имеющие физику. Только и всего. Потому и говорим вам, не называйте нас ``высшими'' или ``низшими''. Мы никогда не говорили вам, что мы выше или ниже нас. У нас – иное, у вас – иное. Только и всего. И когда вы придёте в энергетический мир, это не значит, что вы ``выросли'' во что-то духовное и так далее… Нет, просто пришло время поменять плоть. И в том же энергетическом мире, в том же мире Огня, есть тоже, в вашем понятии, чистое и нечистое. И не на много-то он отличается от этого мира. Не на много. Там тоже есть войны, там тоже идёт борьба. Оттуда приходят к вам, чтобы собрать друзей… и плохие и хорошие стороны. Отсюда – и есть множество религий, которые направлены против или за бога, которые направлены против или за человека. Ибо, любой мир, который вы назовёте, любые иерархии  - это мир такой же, как и ваш. Только что разная физика и другое понятие. Здесь вы можете сделать преступление, но - физически. В том мире, вы сделаете преступление, но в духовном плане. Поэтому говорят, что в энергетических мирах ответственности больше. Ребёнок, совершивший что-то – разве его обвинят в преступлении? Нет. И сделай то же самое взрослый – он уже обвинён. И потому, когда придёт к вам кто либо из ``высших'', в вашем понятии, и скажет, что вы сделали преступление – не верьте ему. Не верьте. Он просто ищет себе сторонников. Больше или меньше. Далее… Вы должны искать и ни кого не спрашивать. Вы должны видеть себя. Вы же, чаще, приходите к другим и говорите: ``Посмотри меня''. Вы создали зеркала, чтоб увидеть свою физику и не можете создать зеркало, чтоб посмотреть свою душу. А зеркало простО, его и делать-то не надо. Это – отношение людей к вам. Вот ``зеркало''. Посмотрите, как относятся к вам, и вы увидите себя. Посмотрите, как вы относитесь к другим – и вы увидите себя. Но… зеркало кривО, ибо вы будете смотреть сознанием. И будете искажать, и будете себе врать или в угоду, или наоборот. Вы никогда не можете поставить себе правильной оценки. Вы всегда её или занижаете или повышаете. Вы ищете высшие миры, мечтаете о боге – и что же? Приходит к вам бог, и вы обвиняете его во лжи. И вы тут же лишаете его жизни… Было? Было… и будет. Ибо – бог пришёл не по вашим мерам. Он не понравился вам.}
1-2-3
\people{Если вам вопрос задать мысленно… Можно так?}
\soul{Спрашивайте.}
\people{Можете ответить? (был задан мысленный вопрос)}
\soul{Мысленно?}
\people{Нет.}
\soul{Вы можете сконцентрироваться на мысли? Только на мысли.}
\people{Попробую.}
\soul{Сконцентрируйтесь только на этой мысли.}
(конец кассеты)
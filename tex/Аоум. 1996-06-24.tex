Аоум. глава 24-06-96г
Георгий Губин
1996.06.24
\people{11-12-13}
\soul{В принципе, тебя сейчас вижу - аж троих. Именно там, где ты сейчас находишься, потом - выше, и, наконец, знаешь, где-то, кажется, на потолке.}
\people{(Гера) Меня? На потолке?}
\soul{Да. И это что интересно, краски - чем выше, тем чернее, скажем так. Тут тебя руки  - одного цвета, а вот на потолке ты - чёрный. В данном случае, получается, я бы так это сказал, что… ну, как… ты сейчас ``расстроИлся''.}
\people{(Гера) Чёрные полоски в каком смысле?  - Черные мысли? Или как?}
\soul{О, нет. Я говорил же, что цвета – не значит, что ``плохо'' там… что ``чёрный'' – это именно ``плохо''. Нет. В данном случае, возникает мысль, знаешь, а вот… как бы, что у тебя сейчас количество, так это больше, чем качество этих мыслей… У тебя в голове сейчас большой бардак. Очень много, знаешь, незаконченного, скажем так вот, мыслей, да? А они, соответственно,  значит, что, ну, как сказать… бесполезны, понимаешь. Они вот только ``пожирают'', понимаешь, вот? То есть,  поглощают, да? Вот такое, в данном случае, объяснение ``черного цвета'', да? А вот, взять, допустим… Ну, как мне сказать, только не называя имени, да?}
\people{(Белимов) В белой рубахе человек, или…}
\soul{Хорошо. Так и скажем, - ``в белой рубашке'', да? Тот же самый чёрный цвет, понимаешь? А вот говорит как раз-то уже о другом. Вот здесь надо просто знать что ли, да? Может быть, тот же самый цвет, да? Но, вот, характеристика совершенно другая. Вот, допустим, взять ту же собаку. Понимаешь? Вот… если не ошибаюсь, то был ``сбой'' при её приходе. Вначале.}
\people{(Гера) Нет.}
\people{(Белимов) В начале - да.}
\soul{Понимаешь, именно, что здесь, вот, это… меня удивило - резкое дополнение к фону. Понимаешь? И причём, довольно-то нестандартное дополнение, потому что аура того же животного всё-таки, в конце концов, она отличается от ауры человека. Понимаешь? Или того же предмета.}
\people{(Гера) Ну, ты, по ауре сможешь сказать, где хищное животное, вообще-то, а где, допустим, собака?}
\soul{Ты знаешь… только на данный момент. }
\people{(Гера) Ага…}
\soul{Нельзя сказать, что - вот этот человек ``грязный'', да? Он может быть сейчас ``грязный'', как ты говоришь, да, но, может быть, он всю жизнь был добрейший человек, но вот только-только что ему нагрубили, понимаешь, и у него плохое настроение. Это вот поверхностно. А вот в глубину заглядывать… ты знаешь, здесь надо иметь чувство такта. Знаешь, это всё равно, что залазить, копошиться в чужом белье, понимаешь, вот… Вот, можно… При желании, я могу заглянуть к тебе внутрь, глубже, то есть. Но, знаешь, существуют, вот, как ты говоришь, такое понятие ``такта''. И даже если будешь просить, я не всегда смогу это сделать… Понимаешь? Почему ещё… Потому, что человек силён тем, что вот он, как раз-то…  любовью к  тайнам. Если я тебя сейчас буду знать ``от и до'', то, понимаешь,  я боюсь, что мы просто потеряем друг друга и всё.}
\people{(Гера) То есть, то, что сейчас в голове у тебя, насчёт меня, не будет соответствовать действительности,- и всё, ты разочаруешься…?}
\soul{Нет. Наоборот! Когда я о тебе всё полностью узнаю… Знаешь, вот, скажем так: женщина – это ``книга'', да? Давайте её внимательно ``читать'', - а можно заглянуть и в оглавление, а можно сразу и в конец. Правильно?}
\people{(Гера) Да.}
\soul{И вот, когда вся эта ``книга'' будет прочитана, понимаешь, текст книги ``пропадает'', и уже знаешь что. И когда ты будешь её заново ``читать'', ты уже будешь знать ``продолжение''. Правильно? Теряется вот это чувство… первое чувство.}
\people{(Гера) Да. Верно, }
\soul{Ну, вот, - то же самое.}
\people{(Гера) Интересно, тогда никакой любви, получается, не существует – просто, чисто удивление там чё-то… }
\people{(Ольга) Тайна.}
\people{(Гера) Тайна, да. И всё? Больше ничего?}
\soul{О нет. Это другое.}
\people{(Ольга) Это другое. Другое.}
\people{(Гера) Я имею в виду - на земном уровне, а не в ``высших материях''.}
(Ольга вышла. Сбой контакта. )
\people{(Белимов)  Шесть пятнадцать. (время. прим)}
\soul{Идёт сейчас, да? (Ольга возвращается. прим.) Я знаю, что… Ну, сейчас, ладно. Ушла, да, - и сидит в кресле и, в то же время, я вижу, что она  наклонилась надо мной. (переводчик наблюдал 2 самостоятельных копии Ольги во время её отсутствия . прим)}
\people{(Гера) Кто?}
\people{(Белимов Гере) Ну - ты.}
\people{(Ольга Гере)Ты.}
\people{(Гера Белимову) Я? Нет - она.}
\soul{Да.}
\people{(Гера) По-моему, она пришла не через три минуты, а сейчас. }
\people{(Ольга) Пиво! (объясняет почему она выходила. Было лето, стояла жара. Ольга выпила бокал холодного пива. )}
\soul{Она только что уходила, но я её видел здесь. Понимаешь? Что она наклонена была надо мной.}
\people{(Гера)”Батюшки''! (это был возглас удивления. Прим.)}
\soul{И вот, как вот это вот объяснить?  Да, я понимаю, что это можно как-то объяснить, но для меня это будут пустые слова, что вот, аура там, астральное тело…  Понимаешь?}
\people{(Гера) Смысла нет?}
\soul{Да! Я не могу пока найти, вот это… именно в корне объяснить, вот это вот, что именно конкретно. Понимаешь, в чём дело? – Это дальше счёт, что ли просто, может… Нет, я же говорю, дальше, к примеру, что я видел тебя - троих, да? Вот! А вот объяснить - как? Я знаю только - почему.}
\people{(Гера) Почему?}
\soul{Точнее - что это обозначает. Понимаешь?}
\people{(Гера) Что это?}
\soul{Но я только что это говорил! }
\people{(Гера) А-а!}
\soul{Понимаешь? А вот, почему именно так… То есть, я не вижу начала, я вижу только готовый ответ. Это плохо, конечно.}
\people{(Ольга) ОНИ говорят: ``Ищите начало мысли''. И не только мысли, а всего-всего начало искать. То есть, ниточку не терять  - ``нить Ариадны''. Да?}
(сбой контакта)
\people{(Белимов) Это уже не Геннадий?}
\people{(Гера) Конечно, нет.}
\people{(Белимов) Ну, мы сейчас в растерянности. Мы давали и прямой, и обратный счёт, потому что было долгое ``зависание''. Скажите, как нам сейчас поступать? С 19-ти считать или с 9-ти до… Если придётся  возвращаться.}
\soul{А вы уже не помните, какой вы давали счёт? Мы же говорили вам, начинайте и заканчивайте все счета в обязательном порядке!}
\people{(Белимов) Думали, что, вроде, дошли до конца. Ну…}
\people{(Ольга) Скажите, может быть, я неправильно поступила тут, ушла так вот просто… неожиданно.}
\soul{Это ваши проблемы.}
\people{(Ольга) Да? Не влияет?}
\people{(Белимов) Не повлияло ничего? Ну, хорошо. }
\people{(Ольга) А то, может быть как-то…}
\people{(Белимов)Скажите, вот, среди землян много сейчас обсуждается вопросы, есть ли, действительно, половые контакты с иномирянами? И что они влекут? Какую цель они имеют? Вы можете нам это подсказать?}
\people{(Гера) Эзотерические контакты.}
\soul{Есть множество причин. Первое: это желание прославиться. Второе: это нежелание других. Конечно, тогда можно выдумать себе ``героя'', пусть это будет инопланетянин, пусть сразу убивает два зайца: вами пользуются, и – вы знамениты.}
\people{(Белимов) Кто пользуется?}
\people{(Гера) Выдумки, короче.}
\people{(Белимов) Это разве выдумка?}
\soul{Пойдёмте дальше.}
\people{(Белимов) Так.}
\soul{А как вы думаете… Или - хотя бы ответьте на такой вопрос, действительно, который был уже задан вам,- но вам было лень на него отвечать или хотя бы задуматься, почему, когда ``стальные границы'' ваши были нарушены, и сразу, после объявления гласности, появилось множество ``половых контактов'', как вы говорите? А не кажется ли вам, что вы просто становитесь ``обезьянками''?}
\people{(Ольга) Ну, да, повторяем всё то.}
\soul{Это - первое. Давайте дальше. Вы любите цифры? Хорошо, давайте перейдём на цифры, - 0,98 – ложь. Вас устраивает это?}
\people{(Белимов) Что?… В общем, 98 процентов лжи? Но ведь, смотрите, есть… созданы специальные институты, за рубежом в основном, где исследуют и находятся и физически бывают, и просто вынуждены, их замечают, делают аборты, то есть… вполне развивается клетка, потом - она куда-то исчезает…}
\soul{А совсем недавно, простите, вы жили ``при коммунизме''. Вы забыли уже об этом?}
\people{(Ольга) Да-да, всё верно… Что мы придумываем, то и получаем. В конце концов, через какое-то количество лет или как…}
\soul{Причём то? Мы говорим вам, что совершенно недавно вы читали в газетах, и вы считали, что живёте ``лучше всех''.}
\people{(Ольга) Да-да…}
\soul{Вы считали, что вы единственные, которые верны. А теперь - отмена власти, и пять - всё это было ложь, и у вас теперь ``новая правда''. И как часто будут у вас ``новые правды''?}
\people{(Ольга) Да это ``кому – как'', наверно. Когда, действительно, смена режима, и тогда уже появляется ``другая правда'', как будто ``правд'' много. Это уж мы такие…}
\people{(Белимов) Вот, вы говорите, видимых радостей там, преимуществ, человек, признающийся в половых контактах, особенно женщина, - она же не получает никаких дивидендов, кроме, даже… Ну, то, что она может  ``странной'' быть названа, а вы говорите, что она придумала!}
\soul{Мы говорили вам об ``обезьянах''. А теперь, возьмите - та же самая американская женщина объявляет об этом. А вы знаете её культуру? Вы знаете, как она живёт? И не видите ли вы разницу между нею и нашей, как вы говорите, ``советской'' женщиной? Не замечаете разницу?}
\people{(Белимов) Ну, нашей, по-моему, есть что скрывать. Она в дом умалишённых может попасть при таких признаниях.}
\people{(Гера) Да и, обычно, у нас  мораль  не такая.}
\soul{Раньше - да. А сейчас?  А сейчас, то же - приносит славу.}
\people{(Гера) А-аа, чисто ради славы…}
\soul{А, простите, ``нашей русской'' женщине  не очень-то сильно интересует, как ей мы ей будем пользоваться, ``лишь бы в доме был достаток'', а уж только потом - остальное. И заметьте, большинство, почему-то насилует одиноких…}
\people{(Белимов) Угу.Так.}
\soul{…или с каким-либо другим недостатком. Простите, миллионершу ещё никто не насиловал. А почему? Вы можете объяснить?}
\people{(Ольга) Да.}
\people{(Белимов) Ну, вы скажите.}
\soul{Ну, вы подумайте.}
\people{(Ольга) Да-да, конечно.}
\people{(Гера) Мужиков полно. Зачем ей инопланетяне?}
\soul{Как вы примитивно рассуждаете.}
\people{(Гера) Возможно.}
\people{(Белимов) А что? Ну, что вот это? Ведь признания в этих… наоборот…унижает…}
\soul{Хорошо, давайте скажем так. Как вы думаете, сколь много историй вы услышали об инопланетянах, как вы думаете, большой ли процент верности? И что заставляет людей лгать вам? Они же  от этого ничего не получают, как вы говорите. }
\people{(Белимов)  Ничего. Так. Что заставляет лгать?  Тёмные силы, может быть? Или иные?}
\soul{Ну, хорошо. Кому-то просто интересно обмануть – ``Как же! САМ ….'' - и ваша фамилия…}
\people{(Белимов) И так часто бывает?}
\soul{…и похвастаться перед другими. Другой, может быть, всю жизнь мечтал стать артистом, но не получилось, и вот он себя теперь проверяет на вас. Вы поверили – ``И всё-таки зря, какой артист погиб во мне!''. Так? Продолжать?}
\people{(Белимов) Ну, хорошо, и мотивы этого? Некоторые очень искренне, так сказать, ищут ответа…}
\soul{Мы не говорим уже о тех, кто выдумал и поверил сам в это. Мы вам говорим именно о лжи. Итак, другие. Ради славы. Простите, в те времена ``попасть в газету!'' - было довольно-таки сложно, а господин Б.,- пожалуйста! Приди, расскажи ему ``сказку'', при этом предупреди всех друзей, что это ложь, и над гражданином Б. смеются, а ты уже ``герой''! Вам нравятся эти ``раскладки'' ?}
\people{(Белимов) Да нет, мы…}
\soul{А сколь много было?}
\people{(Белимов) Много было?}
\soul{А мы говорили вам: то не ваши проблемы,-  он отвечает за ложь. Для вас, это была правда, когда-то мы говорили о лжи и правде. Для вас, это правда, значит, пишите! Значит, вы всё-таки не лжёте, ибо вы уверовали. Для вас, это не является ложью. Он же - будет отвечать за ложь свою. Он же… И что интересно, он может столь сильно раздразнить тех, о ком сочиняет, что они придут и устроят ему всё то же самое! (стоит задуматься над фразой, что ``нет лжи и правды'' в определенном смысле. Прим.)}
\people{(Гера) Ну, всё-таки, пускай не на 98%, а 2% всё-таки было на самом деле?}
\soul{Ну, говорят - да.}
\people{(Белимов) А если, я бы взял, допустим, и 17-ть глав своей книги написанной, стал с вами советоваться, то есть…}
\soul{Это было бы глупо.}
\people{(Белимов) …правда или не правда?}
\soul{Сколь вы много раз уже спрашивали, и мы отказывали, и скажем снова.}
\people{(Ольга) Сами должны.}
\soul{Ну, поймите же! Вы должны научиться ``ходить'' сами! Не будете же вы всю жизнь ходить ``за ручку''?!}
\people{(Гера) Тогда теряет  смысл всякой вообще помощи кому либо  по такому поводу .}
\soul{Разве?}
\people{(Белимов) Да, и тут вообще теряется смысл изучения…}
\soul{А ``первые шаги''? Вы заметьте… Вы заметьте, что чем развитее существо, тем трудней ей начать жить.}
\people{(Гера) Да. Иным не хватает всей жизни, чтобы начать ``правильно ходить'', так скажем.}
\soul{Мы не о том. Мы - чисто о биологическом уровне.}
\people{(Гера) А-а! Понятно.}
\people{(Ольга) Скажите, а вот, действительно, есть человеки, которые не могут сейчас воплотиться в тела?}
\people{(Гера) Подходящих нет.}
\people{(Ольга) Нет. (5 сек. тишины.прим.) Всё… }
(Переводчик завис. )
(Выход на Маску. Прим.)
\soul{Хитрые! Что только `` хорошая, хорошая''. Не только.}
(Переводчик завис. )
(Выход на связь с сознанием переводчика. Прим.)
\people{(Ольга) Счёт дать? Счёт дать?}
\people{(Белимов) Скажите, какой счёт дать? Откуда? Обратный?}
\soul{Обычный.}
\people{(Ольга) Обычный? Назад?}
\people{(Гера) Прямой? Назад?}
\people{(Белимов) Гена, какой у тебя сейчас голос? Ну-ка, скажи}
\people{(Ольга) Какой ,какой? Как всегда! Сегодня.}
\people{(Гера) Нормальный. Громче или тише?}
\people{(Белимов) Ген,скажи. Ну, скажи!}
\people{(Ольга) Громче всех. (смех. Прим.)}
\people{(Белимов) Скажи.}
\soul{Боюсь.}
\people{(Ольга) Всё нормально, ты глянь!}
\people{(Белимов) Ну-ка, ещё раз!}
\people{(Ольга) ``Боюсь''.}
\people{(Белимов) Надо сделать  магнитофон тише.}
\people{(Ольга) Ты говори.}
\soul{Всё равно болит.}
\people{(Ольга) Ну, не важно.}
\people{(Белимов) Ты видишь, голос наладился! (у переводчика болело горло. Прим.)}
\people{(Ольга) Ты сейчас говорил таким же – `` Не только - говорит -  хорошее''. (о голосе при выходе на Маску.прим)}
\soul{Болит, вот здесь вот болит.}
\people{(Белимов) Понятно, что повредил горло.}
\people{(Ольга) Ну, ты скажи, ты насчёт зрения ничего не помнишь? Насчёт зрения ничего не помнишь? }
\soul{Решил.}
\people{(Ольга) Решил? }
\soul{Решил.}
\people{(Гера) Это твоя тайна, да?}
\soul{-Нет, почему?}
\people{(Белимов) Гена, что ты сейчас припоминаешь? Какой разговор?}
\people{(Ольга) Насчёт ``Решил'' - что?}
\soul{Не, вы теперь более объёмные всё-таки.}
\people{(Гера) А что, мы плоские -то были?}
\soul{Ну, нет, только почему-то вы теперь такие… больше, может, плоские ещё.}
\people{(Белимов) А что насчёт зрения? Чё ты там, Оль?}
\people{(Ольга Гене) Ты говорил нам.}
\people{(Белимов) Горло восстановилось? Ты там хрипел, а сейчас спокойненько говоришь.  Они тебя вылечили!?}
\people{(Гера ) Да потише вы!}
\people{(Ольга) А ты сказал, вот, ``я боюсь, и не знаю, до конца контакта нужно решить'' Скажи, решил или нет?}
\people{(Гера) Ты решил в каком зрении ты будешь видеть?}
\soul{Вообще-то, я не понял, о чём вы говорите сейчас, честно говоря…}
\people{(Ольга) Они…}
\soul{Сейчас и устроит недавнее. }
\people{(Ольга)  Вот, понимаешь, ты когда вошёл(в контакт. Прим.), ты был сиплый, а потом, там, ну, кое-что прошло. Сказали… счёт дали от 1 до 35, мы дали, - и обратно, и ты заговорил. Сказал, что … Нормальным голосом.}
\people{(Белимов) Хорошим голосом. Представляешь? То хрипел…}
\people{(Ольга) А потом, говоришь… }
\people{(Белимов) Включи свет, а то…}
\people{(Ольга) А потом, говоришь: ``Мне предложили со зрением, но я испугался'' – и, вот, ну, чтобы зрение исправить, нормальное чтобы было.}
\people{(Белимов) Нет, они же не так, Они же…}
\soul{Я, примерно, догадывался, между прочим, где они примерно должны быть. Но они сантиметра на 3-4, получается, длиннее. Я б не излучил волны, понимаете, получается.  Ну, в смысле как… полутени  что ли…}
\people{(Белимов) Ты знаешь чего… это… Оля, они говорили не о коррекции зрения, а о том, хочет ли он видеть…}
\people{(Гера) В том зрении, вот в инфракрасном…}
\people{(Белимов) В инфракрасном.  Он больше будет видеть.}
\people{(Ольга) Да? Именно это?}
\people{(Белимов) Больше. То ли ауру, то ли что…}
\soul{Тепло?}
\people{(Белимов) Да нет, }
\people{(Гера) Нет. Ну, это я так сказал, невидимый там…}
\people{(Белимов) То есть…}
\people{(Гера) Аурические свечения…}
\people{(Ольга Белимову) Надо капнуть вначале. Первая плёнка прям вначале…}
\people{(Белимов Ольге) Ну, сейчас мы будем искать там!}
\soul{Мне не понятно то, что который контакт я уже никуда не хожу. (после контакта в уборную по ``мелкому делу'' )}
\people{(Белимов) Не понятно, а как энергетика? Что?}
\people{(Гера) Потом выходит.}
\soul{Да каким потом!}
\people{(Белимов) Не хочешь?}
\people{(Ольга) Потому, что ты, в основном, в сознании в своём. (в контакте. прим)}
\people{(Гера) Да, кстати…}
\people{(Белимов) Может, вообще, прекратить всю деятельность ?}
\people{(Гера) Нет, просто его ``дУет'' в разные дни.}
\people{(Ольга) Не-ет! Геннадий Степаныч, они вам правду сказали, нам нужно просто, действительно…}
\people{(Гера) Отсеивать.}
\people{(Ольга) Помните, разговор с вами вначале ещё был, что вы должны различать уметь? Помните?}
\people{(Гера) Да-да.}
\people{(Белимов) Ну, я, вроде, отсеиваю. Я не так уж часто…(об умении различать правду и ложь.прим.)}
\people{(Ольга) А вы мне говорите: ``Научи меня'', – а я говорю: ``А как научу? Это надо самому.'' – вот, они там,  то же самое сказали.}
\soul{Дело в том, что…}
(Сбой контакта. Начало следующего диалога идёт не сначала)
\soul{Именно вот этот ``авось”…}
\people{(Гера) ``В волчьей стае –  по-волчьи выть'' только не ``на авось''..}
\people{(Ольга) Авось, авось, точно.}
\soul{Это авось! Это есть самый натуральный авось! Вы ``плывёте по течению''! Вы не стараетесь ни барахтаться…-  ``А! Куда приплывём, туда и приплывём! Нам везде хорошо, где нас нет''. Понимаете? Вот тот принцип – это и есть тоже, как говориться, соблазнение. И вот, представьте, я сейчас могу приобрести возможность, допустим… левитации, скажем так, да?}
\people{(Гера) Вот именно сейчас?}
\soul{Ну, я говорю - допустим. Вот, представьте себе, я имею эту возможность, я ей воспользуюсь, и в первую очередь, для чего? – Чтобы…}
\people{(Ольга) Показать нам, какой ты… (Бэтмен. Прим.)}
\soul{Показать, да? Именно вот этот соблазн - стать ``выше'', понимаешь.}
 
\people{(Гера) То есть, достаточно…}
\soul{То же самое яблоко. И, понимаешь, вот что тебя заставляет сейчас вот рассказать, понимаешь. Сейчас в этом положении меня очень трудно, вообще-то, обмануть - я вижу тебя. Я вижу тебя, понимаешь? И я вижу, что тебя заставило задать этот вопрос именно сейчас о левитации. Тебя сейчас волнует не сам интерес, как я взлечу.}
\people{(Гера) Да, не сам.}
\soul{Вот именно! А вот что,- я не буду говорить, чтобы не портить, но ты уже понял. Понимаешь? И вот это  Адамово яблоко, оно всегда, всегда рядом с нами. Всегда с нами этот ``змий'', понимаешь? Он постоянно соблазняет. Любая попытка, любой соблазн – это вот именно как раз и есть выбор добра и зла, на данном уровне жизни. А то сейчас скажешь, `` Как? Всё едино!''.- Нет, пока я вот живу здесь, да, пока вот ты здесь живёшь, в тебе нет того единства, потому что ты не ощущаешь этого единства, ты не живёшь в нём, в этом единстве. У тебя есть жена, с которой ты постоянно ругаешься, у тебя есть работа, на которую ты хочешь или не хочешь идти, понимаешь? У тебя есть много обязанностей, и ``хочу'' и ``не хочу''. О каком единстве может быть речь? Ты не живёшь в нём. И поэтому, для тебя, существует добро и зло, для тебя, соответственно, существует, значит, и дьявол, существуют и те же ангелы-хранители – всё это есть для тебя.}
\people{(Гера) А на самом деле их нет?}
\soul{На самом деле, всё гораздо по-другому. Понимаешь? Ну, а как я тебе это объясню? Ты же не ощутишь. Ну, а как я тебе могу объяснить? Но вот, понимаешь, я, допустим, возьму сейчас какую-нибудь картину. Вот - сейчас у меня картина, так? Перед собой. Как бы я тебе её не разрисовывал, понимаешь, ты сам её не поймёшь. Я могу тебе сейчас по каждому сантиметру, по каждому миллиметру рисовать её точно-точно. Да, ты всё это запомнишь, ты её так же попытаешься нарисовать, понимаешь, но ты не сможешь её нарисовать полностью! Самый максимум, что ты сделаешь, это - из семи, понимаешь, из семи вот этих, как говорится,  ``квадратиков'', нарисуешь ``семь на семь'', но не больше. Вот почему по этому же принципу взято не больше семи человек, потому что тут уже - внимание. Понимаешь? Хоть я и вижу чаще всего вот вашу общую ауру, да? И в то же время, если поднапрячься, я могу увидеть отдельно. Но, понимаешь, дело в том, что я изначально, где-то я не знаю именно этих основ, - это дальше счёт дать надо, чтобы знать именно корни вот этих знаний, откуда я знаю. Но я знаю, что, в принципе, описание цветов, литературы – это очень-очень примитивно, и она, понимаете, больше, как говорится, ``от чёрных сил'' написана. Это то же самое ``Адамово яблоко'', соблазн, понимаешь - возродить как можно больше экстрасенсов, понимаешь? И даже не экстрасенсов, а, вот именно, что, как бы сказать, играть на эгоизме человека. А вот смотри: ``Ах! Как? Вот, Кашпировский может, может быть, и я могу?'' – понимаешь? Подняться на уровень Кашпировского. Ну, конечно, мы в оправдание сразу скажем: ``Плох тот солдат, что не мечтает стать генералом'' – правильно? Ну, вот значит, плох тот экстрасенс, что не мечтает быть Кашпировским.}
\people{(Белимов) Угу, ясно.}
\soul{Но, понимаешь, здесь идёт чувство эгоизма, именно, что ``стать, так Кашпировский'', а не первая мысль, чтобы лечить людей…хотя бы, как Кашпировский. Понимаешь? Вот это вот – ``Адамово яблоко''.}
\people{(Ольга) Кашпировский, он… ну, как… вот…}
\soul{Ну, вот мы затронули теперь личности. Давайте, не будем трогать их.}
\people{(Ольга) Ну, хорошо, давайте.}
\people{(Белимов) Ну, ладно. А вот можно проверить твою память вот, именно с работой свободного сознания? Скажи, ты мог десятки раз читать или слышать о смерти или рождении Маяковского. Назови сейчас даты.}
\soul{Нет, не могу.}
\people{(Белимов) Значит, не отложилось, да?}
\soul{Нет, отложилось…}
\people{(Белимов) …Или она не востребована сейчас этим сознанием свободным? Другой, да, нужен? Другой счёт?}
\soul{Нет, я этого не знаю, другой счёт или нет. Я, в принципе, об этом не думал. Да, тут, естественно, конечно, знаю дату смерти, понятно, я даже знаю, как он умер – всё это я знаю, но я не могу сейчас этого почему-то сделать.}
\people{(Белимов) Ну, ладно, не будем. При других счетах мы это проверим. А скажи, пожалуйста…}
\soul{Это – первое. Потом, что ещё… В принципе… Хотя нет, это не та причина. Я думал, что просто не тот объём памяти взят, в принципе, нет. Дело в том, что, как бы вот взять вот, моё состояние в данном случае, да, это вот, какое-то поле, пространство… да, скажем так, именно вот, информации,  да?}
\people{(Гера) Уровень?}
\soul{Нет, не уровень, а именно поле объемное.  Дело в том, что всё это в объёме. Вот объём, который содержит информацию, а я всего лишь маленький кубичек, да, - вот ``я'' передвигается, окошечко, можно так сказать, да? Вот передвигается по этому пространству, вот что в этом окошечке, как говорится, видно, вот этим пространством и владею. Понимаете?}
\people{(Гера) А свободно… То есть, в данную точку попадать выборочно ты не можешь, ты чисто сканируешь,так… случайно?}
\soul{Ну, почему? Нет, не случайно, конечно…}
\people{(Гера) Но для это надо знать путь к этой точке, куда попасть надо.}
\soul{Но в принципе, понимаете, вот вы задаёте вопросы, а в принципе-то, я…}
\people{(Гера) А! Вопрос является паролем?}
\soul{Паролем является, да. Но, видишь, ещё дело в том, что… данная тема, тема сейчас ближе. Тема духовная, да, вот… И поэтому…}
\people{(Белимов) Ну, хорошо, ладно, давайте отложим для других. А вот скажи, пожалуйста…}
\soul{Зато я могу помнить. Ну, здесь вот надо ,действительно, видно ставить эксперименты, как можно больше, потому что даты смерти Маяковского я не вспомнил, но зато прекрасно помню рождение своё же, понимаете?}
\people{(Белимов) Ты нас удивил тогда, что кормление помнишь. Это поразительно.}
\soul{И что интересно, что я даже… а вот здесь, я тогда не знаю, честно говоря. Дело в том, что, как родилась… А! Ну, вообще-то, это объясняется, это легко объясняется. Я помню, как родилась моя мать.}
\people{(Гера) Да ты что?!}
\soul{Это всё прекрасно объясняется, потому что…}
\people{(Белимов) А как? Как? Почему ты… почему ты помнишь? Что, передалось, её впечатления в твой мозг?}
\soul{Ну, наверное, да. Я не знаю точно. Наверное. Дело в том, что я не знаю именно истоков, да, понимаете? Вот, как ОНИ говорят, ``начало мысли'', да? Я не могу сейчас найти начало этих мыслей, потому что, видно…. Но чем я сейчас отличаюсь от обычного, вот, когда я сейчас проснусь, да? Ну, вот  только тем, что просто, скажем так, более развязан язык…}
\people{(Белимов) Раскован.}
\soul{Раскован, да. Сняты какие-то вот…}
\people{(Белимов) Ограничения.}
\soul{…барьеры, да! И, в принципе, не больше. В принципе, даже память уже не больше, не намного больше этого. Просто более свободно передвигаюсь. Но существует какое-то вот… ограничение существует что ли. Вот туда мне можно, конечно, но  вроде как не нужно.}
\people{(Гера) Рано, может?}
\people{(Белимов) А что…}
\soul{Не ``рано'', а вроде как незачем, понимаешь?}
\people{(Белимов) Особенности рождения матери - какие? Можешь это сказать?}
\soul{Особенностей никаких.}
\people{(Белимов) Никаких… Скажи, какие впечатления детства, на твой взгляд, особенно повлияли на твой выбор профессии? Оно же с детства, наверное, идёт?  Вот, наши…}
\soul{Ну… знаете …дело в том, что я помню…}
\people{(Гера) Может - камень, просто? (намёк на камень рождения. Прим.)}
\soul{Нет. Я помню первый полёт Гагарина. Даже помню, что он был в 61-м году, и я ещё не родился.}
\people{(Белимов) О! Ты не родился ещё?!}
\people{(Ольга) Да.}
\soul{Ну, естественно, я же 62-го года рождения, да?}
\people{(Гера) Да.}
\soul{А вы помните, какого числа?}
\people{(Белимов) 12 апреля, я помню. Меня приняли в пионеры.}
\soul{А теперь для интереса посчитайте, была ли моя мать беременна в этом сроке, как говорится?}
\people{(Ольга) 12 апреля?}
\soul{Да.}
\people{(Гера) Не должна была…}
\soul{Если я, как говорится…}
\people{(Гера) Нет, не должна быть.}
\people{(Белимов Гене) Откуда ж ты мог знать?}
\soul{Вот, понимаете, а получается… И дело в том, что я видел это собственными глазами, получается. Да?}
\people{(Гера) Ну, опиши. Чё ?}
\soul{Но я знаю, что мать, например, не имела этой возможности, потому что не было телевизора. Понимаете? Вот. И что… А здесь вот происходит уже накладка. Дело в том, что получается,- да, мать не имела на тот срок телевизора, а я помню телевизор тот очень хорошо, когда он потом появился - он появился, в принципе, уже…когда я уже, как говорится… 64-й год.}
\people{(Гера) Слушай, ну, ведь позже показывали Гагарина же.}
\soul{Да… 18 мая 64-го года.(дата появления телевизора в семье. Прим.)}
\people{(Белимов) Ты помнишь дату приобретения телевизора?!}
\soul{Я это помню очень хорошо потому, что я был очень недоволен этим ящиком, потому, что на меня никто не обращал внимания, - как бы - не был нужен, понимаете? Все водились вокруг этого ящика, я был очень так обижен. Мне дали петушка, причём с отломанной головкой, тут вообще я уже просто был разочарован в жизни, понимаете? Петушок, и то бракованный! Я в слёзы – а никому не нужен, я сижу, понимаете, в уголку сижу плачу… И я  этот телевизор, грубо говоря, этот ящик,- я даже не знал, что это такое,- я его просто возненавидел, понимаете? Вот самая первая реакция – это просто я возненавидел его, я приревновал к нему, потому что все были именно заняты им. Даже та самая бабушка прыгала вокруг него, но не вокруг меня, понимаете? И я вот лез под ноги, единственно, что получал, так это подзатыльники: ``Отойди да не мешай!'' – понимаете? Вот. Ну, а что мне там было-то? Посчитать – сколько мне было?. Вот, видите… И вот это, как говорится, ``первое впечатление'' к электронике, да, вот к машине - именно негативное вот.}
\people{(Белимов) А почему ты выбрал к своей…}
\soul{Понимаете, в чём дело? И, вдруг, вот этот ящик, и вдруг, понимаете, он стал показывать картинки. Странно, мне вдруг стало интересно, я подбежал посмотреть, да? Вот. И вот так я смотрел на эту картинку – меня не трогал никто, я подошёл, решил пощупать ``тётю'', так? Я начал её щупать, во-первых, она меня уколола…}
\people{(Гера) Статика. (статическое электричество. Прим.)}
\soul{… это уже сразу мне не понравилось, а во-вторых - подбежал отец, отпихнул меня оттуда, потому что вроде как тут телевизор… новенький… – понимаете, – ``Тут ребёнок, мало ли чё!'' – понимаете, – ``Надо отучать!'', а то я потом приду  с погремушкой, начну по этой ``тёте'' стучать.}
\people{(Гера) Короче – запретный плод сладок, - так ты стал  радиолюбителем.}
\soul{Нет, нет, не то!}
\people{(Ольга Гере) Это совсем другое.}
\soul{Дело в том, что я эту тётю тоже не полюбил. Потому что я из-за неё ``получил'', как говориться… Как собаку приучают, понимаете, - ``Чужой!'' – всё! Она всех гостей, вот которые пришли в гости… А её запирают в ванную, да? Получается, акт - она знает, что ванна – это плохо, ей не нравится. Теперь любой, кто бы ни пришёл, она будет его ненавидеть, потому что ей, значит, в ванну надо идти. По тому же принципу. Все, кто появлялись в телевизоре, я их ненавидел. Вот, понимаете, с детской ненавистью… Она следов, вообще-то, не оставляет. Я имею в виду, она такая вот безвредная, скажем так, понимаете? Если она не будет закрепляться, конечно. Но, дело в том, что дальше-то пошли потом ``Спокойной ночи'', понимаете? Самое первое осознанное, я помню, когда я просто запомнил слово ``фильм'', я не умел читать, но я запомнил ``фильм''. Именно что ``фильм'', потому что, когда именно вот эта вот картинка: буква Ф большая была, понимаете? Вот эта картинка, именно что слово ``фильм'', я знал, что вот вся семья приходит и начинает… Собирается и сидит в этот ящик смотрит. А вот что там ``документальный фильм'', понимаете, или ``художественный фильм'' – это для меня было всё равно. Я вот как увижу этот ``фильм'', я бегу: ``Папа! Мама! Там кино!'' – понимаете? Они приходят: ``Тьфу! Это ж документальный фильм!'' – понимаете? И всё, уходят. И я, вот, тогда стал задумываться, почему, почему так, что вот, вот ``фильм'', да… и тот - ``фильм''. На тот фильм они прибежали, да? А на этот они наплевали и не хотят. Я стал задумываться, почему – картинка-то та же, да? И тёти те же, дяди те же… Потому что, в принципе-то, понимаете, я не очень-то различал. В принципе, ребёнок не очень-то отличает одну тётю от другой тёти, да? Вот как, для нас, негры все одинаковые, как говорится, да. Есть некоторые с различиями… И вот этот вот интерес - почему так разница? - заинтересовала меня, и я стал, как говорится… }
\people{(Белимов) Твоя дежа.}
\soul{Да, я стал в вот этот первый момент, как говорится, когда начал думать. И причём, заметьте, я стал думать-то с логической точки зрения, понимаете? Уже - логикой. Вот почему, я, как говорится, в технические сферы более…}
\people{(Белимов) Да, почему?}
\soul{Почему? – Потому, что я начал именно с логического мышления. А если б с гуманитарного, это значит, надо было вначале толчок именно с гуманитарного, допустим, чтоб я возненавидел скрипку или пианино, которое прищемило бы мне палец. Понимаете? Вот, с того бы - началось, и я может быть стал бы пианистом. Понимаете в чём дело? Вот. Ну, в принципе, вот – начало, как говорится. А потом, меня уже стало интересовать, когда я был… я стал, понимаете…  В 2 года ребёнок ещё не соображает объем: влезет - не влезет. Вот он берёт огромный мяч и пытается засунуть его в стаканчик. У него ещё нет понятия, что это не влезет туда. Понимаете? Это потом к нему приходит именно вот это понятие ``физическое'', именно ``на уровне физики'', как говорится, да? ``Физический план'', что вот это большое, а это маленькое, да? И, соответственно, большое в маленькое влезть не может. Правильно?}
\people{(Гера) Ну, первая ``установка'' для сознания.}
\soul{Да, но вот именно, что… Но, дело в том, понимаешь, что, когда я вот дошёл до этого момента, я понял, что большое в маленькое не влазит, а как же тогда эта тётя туда залезла?! }
\people{(Гера) Логично.}
\soul{И я, естественно, решил посмотреть, куда она, как она могла туда сложиться. Мне надо было посмотреть, какие у неё ноги. А получилось как? Что здесь стало, как говорится, допингом, да, вот это вот, ускорителем, этот вот, доминантой? Именно то, что мать купила себе новые туфли. Понимаете? И она хвасталась там, показывала, ``то да сё'' показывала, и, что интересно, я почему-то решил, что у тёти тоже должны быть туфли, раз они есть у мамы. А почему я их ни разу не видел? И я решил посмотреть на её туфли вот эти, понимаете, вот эта вот  реакция: куда она эти ноги спрятала, да? Ну, что я сделал? – Я естественно ходил вокруг телевизора, так? Он стоял в углу тумбочки. Как мне туда добраться? Ну, самая первая мысль – это, естественно - через телевизор. Нет бы  сбоку – там же ниже тумбочка, как говорится, да? Но я, почему-то,  решил…}
(сбой контакта.)
\soul{Я слышу.}
\people{(Белимов) Ну, как с телевизором, мы это… что получилось? Ты полез через него?}
\soul{Да, я полез через него.}
\people{(Гера) Что ты тормознулся сейчас, а?}
\soul{Дело в том, что, получается, я.. не могу, не имею права в этом состоянии дотрагиваться до самого себя. }
\people{(Белимов)И, сейчас…}
\soul{А я…}
\people{(Ольга) Да-да. До носа.}
\soul{… дотронулся…}
\people{(Гера) Да. А что, это большие потери для тебя? Или это восстанавливается как-нибудь?}
\soul{Я даже не знаю, честно говоря.}
\people{(Белимов) То есть, ты сейчас опять через туннель улетел куда-то, да? И с трудом вернулся?}
\soul{Нет, здесь сейчас не было туннеля, просто… Ну, вообще-то я не знаю толком что произошло.}
\people{(Белимов) Но мы тоже будем это знать и, наверное, в методике тоже надо указывать, что надо быть осторожнее.}
\people{(Гера Белимову) Надо сверху веревку натянуть, чтоб он не смог.}
\people{(Белимов Гере) Да нет, это не обязательно.}
\soul{Но, может быть, знаете, может быть, тот же самый принцип? В принципе, ребёнок боится своих рук. Может быть это? Потому что, в принципе, я на уровне сейчас…}
\people{(Белимов) Сознания, бестелесного…}
\soul{Ну, как? Нет. В принципе, я вспоминал, вообще-то, что… Две/три версии было… Ну, не знаю… Почему-то нельзя…}
\people{(Ольга) Гена, скажи, мы вот все, да, ведь должны сначала хотя бы хоть приобрести вот сознание, такое, как вот свободное, вот сейчас у тебя, хотя бы, по жизни  чтобы дальше продвигаться?}
\soul{Но это очень тяжело. Вот почему я не могу сейчас проснуться и стать таким же свободным, да? Вы же сами меня ``затюкаете''. Ну, возьмём, допустим… ну, как вот… Мне не привычно… Я не знаю, как ОНИ обходятся без имён?}
\people{(Белимов) Странно, да.}
\soul{“Только что пришедший'' скажем так. Вот он же меня в первую очередь и ``затюкает'', понимаете? Вот, в принципе, чем я сейчас отличаюсь от того, когда я вот просто наяву ``разошёлся'', понимаете? В принципе - ничем же?}
\people{(Ольга) Да-да-да.}
\soul{Ничем. Но, дело в том, что, вот как ОНИ говорили, что вам нужно необычное – вот это махание руками, да? Принцип тот же самый,- психологический, то есть, как говорится, самообман. Вот существует самообман, понимаете? Если б я всё то же самое начал бы рассказывать без махания, а просто вот…}
\people{(Гера) Ну, да-да-да.}
\soul{…вы бы не поверили в это, да? Или просто – ``во как интересно!”}
\people{(Белимов) Не тот эффект… Нам нужна загадочность.}
\soul{Вот видите!}
\people{(Ольга) Да ну! Наоборот.  Мне, например, это претит.}
\soul{Но, понимаете, вот и получается, вот эти тормоза, внешние тормоза, вот эта свобода… Ну, и представьте свободу? Ну, вот, пожалуйста, сколько ходит людей сейчас с вот таким освобождённым сознанием, не имеющим, знаете, границ: ``это можно, а это нельзя'' – да?}
\people{(Гера) А есть такие?}
\soul{Да множество! Вот они ходят - вы их дураками называете.}
\people{(Гера)А-а!}
\soul{Понимаете? Но просто… ну, как тебе объяснить? - Не были именно заданы те ключи, вот в данном случае, да, вот вы дали… Точнее - ОНИ дали вам как бы… ну, скажем так, гипнотическую вот эту установку о духовном, да? Вот. Соответственно, вы тут же о духовном и спрашиваете, соответственно, знаете, вот эту вот… совокупность вот эта вся позволяет именно мне, вот, именно вот эту область пространства взять, да? А возьми подойди к какому-нибудь дурачку, да, задай ему те же самые вопросы… ну, он просто ещё не настроен. Он не вошёл с вами в резонанс, понимаете? И вот, если, в принципе, если сейчас задали бы мне вопрос, именно я смог бы войти в резонанс с тем же Маяковским, допустим, да? Но почему-то я могу его имена назвать, а ваши - нет. Ну, ладно. Всё же - Маяковский… А с вопросом, как говориться, если б я вошёл в резонанс, я бы вошёл в эту область, да? И, знаете, очень много интересного можно сказать, честно говоря, о том же Маяковском. Очень много. Я знаю, что я знаю…}
\people{(Гера) До сих пор не знают, убили его или нет.}
\soul{Ну, понимаешь, я знаю, что я знаю, а вот вытащить бы это. Но, скажем так, да - я знаю таблицу… Не, не таблицу… Скажем так: когда-то я учил какое-то стихотворение, да? Я знаю, что я его учил. Получается, что я знаю, что я его знаю, а сейчас я не могу его повторить. Понимаете? То же самое!}
\people{(Белимов) Другое… Счёт другой.}
\soul{Тот же самый принцип. Ну, а что о телевизоре? – Ничего. Я, просто, получил, как говорится, ``ата-та'' и всё. Не больше. Я  так и не узнал. Ну, ничего, зато я потом узнал о приёмнике.}
\people{(Белимов) Гена, а к технике… А может быть, ты, всё-таки, с механической куклой… с какой-то памяти, с внутренней, захотел…}
\soul{Нет, я ничего не могу сказать о кукле. Я же говорил, что в прошлой жизни, в принципе, на данном счёте, я не могу ничего  вам сказать. И у меня нет доказательств даже существования прошлой жизни. А вот дальше счёт давать…}
\people{(Ольга) Ну, скажи…}
\people{(Белимов) Ну, мы ещё не готовы. Не будем торопиться.}
\people{(Ольга) Ген, вот о таблице Менделеева, да? Там содержится 108 у нас этих элементов открытых, которые мы открыты…}
\people{(Гера) Сто десять.}
\people{(Ольга) Уже сто десять? Да?}
\people{(Белимов) Ну, ладно.}
\soul{Ну, тогда давайте возьмёт сразу 114.}
\people{(Белимов) Да, 114!}
\people{(Ольга) Вот! А вот, в принципе, вот чем отличается одно… один элемент от другого? Вот, только набором? Вот, допустим, вот, или…}
\people{(Гера) Орбитами электронов?}
\soul{Нет, ни в коем случае! Если мы говорим о духовной точке зрения, то только, как говорится, о реакции. Понимаете? Реакции. Но чем… Ну, как объяснить…}
\people{(Ольга) У водорода одна реакция, да?}
\soul{Ну, давайте скажем так. Вы смотрите, как интересно получается, да? Здесь, получается, противоположность, а именно - мы говорим об утончении тела, да? А в итоге наоборот получается,- масса-то  - увеличивается!}
\people{(Гера)Да.}
\soul{Получается, водород, мы же взяли-то примитивный, правильно?}
\people{(Белимов) Угу…}
\soul{А  он оказывается самым лёгким, самым утончённым, в нашем понятии, да?}
\people{(Ольга) Да.}
\soul{А возьмём, давайте, скажем, дальше. А там идёт уже радиоактивные. Практически они все уже радиоактивные пошли, правильно?}
\people{(Гера) Ну, да.}
\soul{И сейчас уже пошли искусственные ещё, а они более утяжелённые. А реакция у них, скажем так… ну, они больше, как говориться, подходят к  солнечным, правильно?}
\people{(Гера) Ну, да.}
\soul{Понятие ``солнечное человечество'', как говориться – да, больше подходит. Вот почему это? – Да потому, что, понимаете… да между ними, в принципе, в духовном плане, реакции нет никакой: что там – атом, что там – атом. Просто - здесь – больше, здесь - меньше. Прости, а какая разница между, вот, допустим, тобой, да? – и… другим, который имеет большую массу? Понимаешь? Просто, чисто по физическим размерам, да?}
\people{(Ольга) Ну, понятно, да.}
\soul{Ну, а какая разница? Просто он тяжелее, а ты легче. Но так же, в принципе, любой атом, разницы в духовном плане не имеет никакой. Что тот, что этот. Это не значит: утолщится – значит, что ты должен превратиться в водород – нет.}
(Ольга) Вот, скажи…
\people{(Гера) Минутку! У нас была тема такая, если помнишь контакт, на счёт водорода, отрицательных частот и отрицательных, как говорится, водородах, скажем так. То есть, таблица не вперёд, а назад. Что ты можешь об этом сказать? Знания у тебя есть, на счёт этих…? Соображения? - отрицательных частот, в частности. Вот, как их создать? Что это такое? Я до сих пор представить не могу.}
\soul{Ну, вообще-то, я не думал об этом, честно говоря. Но это можно сделать проще.}
\people{(Ольга) Зеркальное отражение?}
\soul{Нет.}
\people{(Гера) Нет.}
\soul{Здесь, знаете что? Давайте попробуем.}
\people{(Гера) Давайте.}
\soul{Попробуем, знаете, как сделать… Ну, хотя бы до 23-х.(дать счёт до 23-х.прим.)}
\people{(Ольга) А что ты шёпотом?}
\people{(Белимов) Ему сейчас виднее. Это у нас случайные вопросы возникли. Лучше давайте, это…}
\soul{Она вроде бы страшновата, в то же время, так хочется.}
\people{(Ольга) Хочется?}
\people{(Белимов) Нет, прямой сейчас дать счёт?}
\people{(Ольга) Нет-нет! Ты знаешь, всё-таки надо… Я не знаю, может, я перестраховщица, да? Но мне кажется, нам хотя бы здесь надо научиться так себя вести и быть готовыми ко всему…}
\people{(Белимов) А то нам следующий вопрос захочется… на 50 выйти.}
\people{(Гера) Кстати, а ты говоришь до 23-х, а ОНИ  сказали же, что надо заканчивать до девятки.}
\people{(Ольга) Не важно.}
\soul{Нет, понимаешь, здесь оказывается имеет ещё важность именно… Ну, как объяснить… Вот они сказали до 25-и, да? И причём, в категоричной это было форме или нет?}
\people{(Белимов) В категоричной.}
\people{(Ольга) Да.}
\soul{Понимаете, получается, как говорится, я вроде как подглянул, что ли, да… попал туда, то есть затронул какие-то струны, да? – но не заставил колебаться в полную мощность, понимаете?}
\people{(Гера) Угу. Безболезненно там ``прошёлся”…}
\soul{Ну, как… Ну, не знаю. В данном случае получилась реакция наоборот, что - да, именно о болезнях речь-то и шла. Но было не в полной мощности всё это запущено. А полная мощность - значит, если я не умею этим управлять. Ну… ну, тот же самый одержимый, да? Скажем, тот же самый библейский герой, с которым был, как говорится, вызван легион. Помните?}
\people{(Ольга) Угу. Да.}
\soul{Но вот, в принципе, он находился как раз где-то на счёте вот с 42-х…-47-ми, если так вот по отношению ко мне, допустим, было бы взять. 42-47. Именно, когда вот, именно, чисто подсознание, да? А вот почему 42-47, а  не полный счёт? - 41 пропущен. Почему?- Потому, что отсутствует сознание, понимаете? Он же, всё-таки был, безумным.}
\people{(Ольга) Да-да-да! Вот теперь уже яснее становится, ага.}
\soul{И поэтому, никогда нельзя начинать счёт, допустим, так вот, знаете… }
\people{(Ольга) До скольки хочешь.}
\soul{Да.  35, 36… Нет! Давать надо начинать с единицы, чтобы оставалось вот именно сознание, потому что даже, когда вот речь идёт об утоньчении, да, - о солнечном человечестве, - там же не говорится об отказе сознания же, правильно?}
\people{(Ольга) Да. Наоборот.}
\people{(Гера) Скажи, -  ещё вопрос… Ты слышишь нас, да?}
(сбой контакта)
\soul{Вместо того, чтобы взять, как говорится, сбоку – там же пространство же есть, да? Я решил сразу прямо через телевизор-то и полезть. Ну, естественно, ну, чё… Это мне надо было стульчики нагромоздить, понимаете? Вот нагромоздил всё это хозяйство, а сколько шуму? После этого мать сразу прибегала: куда я стулья тащу-то? }
\people{(Гера) Ну, да.}
\soul{Ну, увидела когда это дело… Ну, что… получил я просто по заднице, да? А вот с этими стульями, что ещё интересно было. Решил я… Мне было…четыр… да, мне было четыре года, и это было… это было, как раз, 12 апреля.}
\people{(Гера) Ага, День космонавтики.}
\soul{И вот, смотрите. Всё вот… под впечатлением всего этого я решил, естественно, в космонавты, понятное дело. Я взял табуретку, одел себе на голову, а снять не могу. Понимаете? И я ходил с этой табуреткой…}
\people{(Ольга) Теперь понятно. Вот теперь понятно. (смеется.)}
\soul{…полдня, пока не пришёл отец, просто-напросто взял эти ножки разломал, вытащил меня оттуда, понимаете? Вот. Но мне это показалось мало, потому что тогда-то это было как: если уж праздник, то на целую неделю, да? По телевизору и по радио: ``Внимание всем! Говорят все…'', как говорится, да? Ага. Ну, и вот. Я, буквально 16-го беру и одеваю на голову кастрюлю. И, естественно, она опять у меня там застревает. Ну… как может кастрюля застрять, я вообще не понимаю? Видно, ушами что ль просто там заклинил. Понимаете? Ну, здесь… А когда с меня… меня снимали табурет,- соответственно ещё и трёпку получал, потому что табуретку-то жалко было. А отец так говорил: ``На заднице всё заживёт!''. Я представил ситуацию, как приходит отец, раздирает эту кастрюлю, понимаете? А за кастрюлю-то я получу ещё больше по шее, потому что она новая! Мне стало это так страшно. А получилось как… почему я не знаю? И вдруг…}
\people{(Белимов) Ну!}
\people{(Гера) Давай-давай!}
\soul{Гера начал вопросы задавать и вы счёт с чего-то начали.}
\people{(Белимов) Нет- нет, тут был некоторый сбой. Ты, вдруг, вспомнил о кастрюле, которую одевал в День космонавтики…}
\people{(Гера) На голову.}
\people{(Белимов) На голову. И чем кончилась история, мы хотели бы вообще-то узнать.}
\soul{Кастрюля?}
\people{(Белимов) Да. Но это был сбой. Мы потом тебе объясним.}
\soul{16-го, да?}
\people{(Белимов) Да, 16 апреля ты одел кастрюлю, и очень боялся, что получишь наказание. Вспомни, пожалуйста, до конца эту историю.}
\people{(Гера) Чем она кончилась?}
\people{(Белимов) Чем она кончилась?}
\people{(Гера) Ты снял кастрюлю или отец опять пришёл?}
\soul{Нет, я помню, что я снял эту кастрюлю. По-моему, от испугу что ли.}
\people{(Белимов) Ну, уши не оторвал?}
\soul{Да нет.}
\people{(Ольга) Наверное, поэтому все дети хотят быть космонавтами, и одевают себе, кто вазы, кто горшок, кто угодно, наверное… Или какими-то героями в шлемах.}
\soul{Ну, да, конечно.}
\people{(Белимов) Гена, но…}
\soul{Но я не помню, чтоб я говорил… Я, получается, как… Не знаю… По-моему, получилось как: Гера начал ``А вот скажи…'' – и тут же вдруг счёт  вы даёте.}
\people{(Белимов) Нет, был перерыв. Ты ещё минут 5 рассказывал…}
\people{(Гера) Я хотел задать вопрос - я не сказал.}
\people{(Белимов) Давай, Гера продолжит.}
\people{(Гера) Давай, пока на кассете у нас немного есть… Ты можешь сейчас вторую страницу этой книги найти? Ты помнишь, какую, да? Во сне которую…}
\soul{Можно попробовать, но это должны вы пробовать теперь, а не я.}
\people{(Ольга) Мы пробовать?!}
\soul{Ну, получается, как? -  вместе…}
\people{(Гера) Мы должны задать правильно вопрос, да? Чтоб ты нашёл её сразу.}
\soul{Нужен опять тот же самый настрой, понимаете? Потому, что вот этот квадрат, этот визирчик потом носится, да? Соответственно, он от ваших эмоций носится. Я вспоминаю такие моменты, которые мне, в принципе-то, не нужны. Вы вроде бы их не задаёте, а это оказывается эмоции ваши, где-то что-то перемкнуло вас потому, что вы… Понимаете, я вам рассказываю что-то, а вы начинаете своё вспоминать, понимаете? И я тут же, на основе ваших воспоминаний, как говорится, на этом, на подсознательном уровне, да, начинаю вспоминать что-то у себя, хотя у вас не было заданного вопроса. И вы тут же задаёте вопрос, вот мне надо умудриться, чтобы не раздвоиться, всё-таки досмотреть, как говорится, что я вспомнил, да, благодаря вам, в то же время ещё и вопрос…}
\people{(Ольга) Вот, поэтому и сбой был. Вот ты сам сказал нам, почему у тебя сбой был. То есть, ты вот начал рассказывать, а когда что-то… это вот ещё… до табуретки, и дотронулся до носа, и произошёл сбой. Потом, мы начали  задавать другие вопросы, ты уже на другие вопросы отвечал, а сейчас ты опять вернулся к этому же, начал дорассказывать то, что недорассказал…}
\people{(Гера) Короче – ты перескочил, потом вернулся назад, и начал заново.}
\people{(Ольга) То есть, надо всегда дорас… ну, до конца мысль довести.}
\people{(Гера) Ладно. Вопрос. Вот эту первую страницу ты можешь запросто представить, да, сейчас? В принципе, 5-го… Какого числа? Апреля, что ли, ты её видел в первый раз?}
\soul{Первую страницу.}
\people{(Гера) А! Пять лет тебе было! Да?}
\soul{Да боюсь, что нет. Что-то не так.}
\people{(Гера) Кто-то не пускает? Не даёт?}
\people{(Ольга) Наверно, не надо.}
\people{(Белимов) Нет, ты был на другом уровне…}
\soul{Дело в том, что я вижу книгу. Да, вижу, но такое ощущение, что я, вроде как …то ли подойти к ней не могу, понимаешь? Такое… что-то не даёт, то есть смутные воспоминания.}
\people{(Гера) Другой массив, скажем так.}
\soul{Ну, может…}
\people{(Белимов) Да-да-да. Можно вспомнить}
\soul{Я знаю, что это есть, я знаю, что я знаю, но не могу идти к ней.}
\people{(Гера) Ты хотя бы описать суть можешь второй страницы? ``Что было бы, если бы не друзья'',- я вот это хочу спросить, что ж они такого-то сделали с  тобой?}
\soul{Суть?}
\people{(Гера) Да.}
\soul{Ну, понимаешь, дело в том, что там говорится именно не о вас.}
\people{(Гера) А о каких друзьях?}
\soul{О мужчине, женщине. Вот о них. Это, как бы,  друзья.}
\people{(Гера) А-а-а… Ну, хоть что-то такое проясняется.}
\people{(Ольга) Ген, ну, вот скажи, те, которых мы называем дурачками, да? Или… Ну, вот всё-таки… или просто мы ненормальные, так может быть, - нормальные-ненормальные… не знаю, это уж как назвать. Вот, между нами разницы, конечно, никакой, кроме того, что они, вот,  более свободные, да? Но… для того, чтобы вот иметь, будем говорить, такое сознание, как сейчас хотя бы,- будем хотя бы об этом говорить,- и… ну, уметь себя выражать… но не то, чтобы уметь выражать, а как-то ну… быть более правдивым, наверное, всё-таки, потому что мы себя ограничиваем, - вот это чёт нам этого нельзя,- наверное, вот,  чувство такта, что ли нужно какое-то иметь. Или внутреннего какого-то…}
\soul{Да нет, скорее всего, нужно иметь терпение, потому, что находясь в таком состоянии его, просто-напросто очень быстро приобретёте популярность идиотки или идиота. Понимаете?}
\people{(Ольга) Угу. То есть, смотри…}
\soul{И поэтому, надо будет вот какое-то вот… независимость, да?}
\people{(Ольга) Понимание иметь, да?}
\soul{Вот чем сумасшедший, да… Вот, в нём что хорошее… ну, скажем так…}
\people{(Белимов) Положительная черта.}
\soul{Да, положительная черта – вот это вот… независимость.}
\people{(Гера) От внешних причин.}
\soul{Ну, как - от внешних причин? – Внешние причины-то влияют, но он более свободен. Понимаешь? Он более свободен от них. А мы что? Мы сразу к внешним причинам стараемся измениться, обмануть себя. Вот мы есть такие-то на самом деле, понимаешь? А внешние обстоятельства нам говорят: ``Нет, ты должен быть такой-то, такой-то…'' – да?}
\people{(Гера) Потому, что тебя восприняли таким.}
\soul{Ну, вот смотри. Хорошо, давай, я вспомню именно конкретный твой пример.}
\people{(Ольга) Давай.}
\people{(Гера) Мой?}
\soul{Именно твой.}
\people{(Гера) Давай.}
\soul{Я надеюсь, что ты его вспомнишь. Вспомни. 74-й год. Не помнишь?}
\people{(Гера) Ну-у…}
\soul{Линейка. Ты стоишь на линейке. И как раз…}
\people{(Гера) Погоди, на какой линейке? Где?}
\people{(Ольга) В школе.}
\soul{В школе, пионерская линейка.}
\people{(Гера) В школе я не мог стоять в пионерской, тем более, что я в школу пошёл в 76-м. Так что ты, братец, что-то тут напутал.}
\soul{Да, тогда я напутал. А почему 74-й взял?}
\people{(Белимов) Ну, вспомни, когда… что эта ситуация…}
\soul{Нет, дело в том, что я чётко знаю 74-й год.}
\people{(Гера) В 84-м, может, что было?}
\soul{Нет.}
\people{(Ольга) Это ты про меня говоришь?}
\people{(Гера) Может, про неё?}
\people{(Ольга) Может, про меня говоришь?}
\soul{Да нет, вижу-то я его, в принципе.}
\people{(Гера) Я ещё в школе не был-то.}
\soul{Нет.}
\people{(Гера) Может быть, вот, рядом сидящий?}
\soul{Коридор. Коридор,- так? И часы. Вот эти квадратные часы, вот эти вот, стрелка там вот… секундная ходит, да? И газета, стенд-газета, а на ней – дата…}
\people{(Гера) Какая?}
\soul{74-й год.}
\people{(Гера) И что там было написано?}
\people{(Ольга) Слушай, может, эту газету вытащили с 74-го года, что-то там подправили и повесили?}
\soul{Да нет, она вроде-то довольно-то свежая…}
\people{(Белимов) Вспоминай, что там с этой ситуацией было. Ладно…. Может, разберёмся.}
\people{(Гера) Чё было-то? Ну, стоял я там, допустим. Пусть будет 74-й, Бог с ним. Хотя это быть не может.}
\soul{Понимаешь, вот эти вот прыжки, они довольно-таки тяжёлые. Я теперь… Вы меня спросили, что сейчас в газете, теперь я должен вспомнить, что в газете. И тут же просите, что было на линейке. Давайте что-то одно…}
\people{(Гера) Давай, что было на линейке, не важно, что было в газете. Что там говорилось? О чём?}
\soul{Три дня назад было принятие в пионеры, и поэтому директриса вышла и стала рассказывать… Директриса…}
\people{(Гера) Да, Мусиенко.}
\people{(Белимов) Понятно.}
\soul{И она стала рассказывать о Владимире Ильиче Ленине и пионерский значок, помнишь же, да? И вот… я не знаю, тут накладка почему, но вот… ты, да, стоишь и создаёшь… стараешься создать… Тебе смешно. Но, так как говорится о Ленине, ты, вдруг, сделал очень хитрый способ нашёл, ты стал говорить: ``Ленин умер, Ленин умер, как жалко!'' – то есть, чтобы, как говорится, сбить свой вот этот смех, понимаешь? Тебе это почти удалось, но тебя сзади толкают ранцем…  И всё-таки это был 74-й год.}
\people{(Белимов) И он взрывается смехом, да?}
\soul{Да, вот этот настрой, что у него полу… он уже настроился, уже всё… ``Ленин умер”- грустно, да…}
\people{(Белимов) Да-да-да}
\soul{Но вот почему 74-й год, вот этого я всё-таки не могу понять. Но газета свежая, если попробовать пойти по коридору, то это надо…. это зависит…  Вы должны пройти.  }
\people{(Гера) Это школьный коридор?}
\soul{Да, это школьный коридор.}
\people{(Гера) В школьном коридоре, да?}
\soul{Да, второй этаж.}
\people{(Гера) Второй этаж… школьный коридор… Нас никак не могли принимать в пионеры, потому что я помню, когда нас принимали в пионеры, мы были в актовом зале. Меня, правда, не приняли в этот раз, и я получил галстук…}
\soul{Тогда в таком случае вы невнимательны, потому что я сказал, три дня назад было принятие в пионеры, и была линейка – это первое. Второе…}
\people{(Гера) Слушай, ты слышишь меня? Слушай, во-первых, я уж никак не через три дня был принят в пионеры, а где-то как бы на следующий год – это первое. Во-вторых – ранцев у нас уже не было…}
\soul{Тогда давай не будем ссориться. Мы говорим о разных вещах, понимаешь? Я тебе говорю, что ты был принят три дня назад в пионеры, а сейчас линейка, а ты мне говоришь, что за три дня…}
\people{(Гера) А-а-а}
\soul{Вот это ``Ага!'' я не смогу…}
\people{(Гера) Слушай, официально я не был принят в пионеры, мне просто выдали галстук и заставили его одеть. Ну, как ``заставили''? Естественно, сказали, чтобы я его одел.}
\people{(Белимов) Да нет, тут воспоминания были о Ленине, о твоей первой, так сказать, реакции артистизма.  }
\people{(Гера) Ну, возможно…}
(сбой контакта)
\soul{Я просто попробовал… ну, как… погулять по свободному сознанию, значит, попробовать от вас освободиться, да?}
\people{(Белимов) Угу.}
\soul{Погулять. Меня действительно заинтересовало вот это вот, почему так произошло именно вот так… нестыковка.}
\people{(Белимов) Разница во времени.}
\soul{Я?… Если это я, ну… то, может быть, я, конечно, почему именно тогда увидел именно его? Может, потому,  что -  спрашивающий?}
\people{(Гера) Может быть. }
\soul{Но это, тогда было, чтобы уточнить, это надо просто спросить тогда  меня будет. Именно задать точно такой вопрос, чтобы я мог такой же рассказывать. Но это тогда - не сейчас, наверное, скорее всего.}
\people{(Гера) Слушай, а сколько тебе было в 74-м году?}
\soul{Если я 62-го был - это же очень просто. Но меня в пионеры, простите за выражение, принимали в 72-м.}
\people{(Гера) Тогда вообще интересно.}
\people{(Белимов) Н-да. Ну, вот, наверное…}
\people{(Гера) Опять что-то…}
\people{(Белимов) Сбой? Нам хочется определить твой выбор профессии. Влияло ли на это, допустим, любовь твоя к школе? Какими предметами ты интересовался? Ты можешь ответить?}
(Сбой. Далее идёт разговор с неизвестным персонажем по названию Внутренний хранитель времени)
\soul{Взял я эту карточку, так? Составил. Всё, хорошо! Естественно, эта карточка завизирована вами. Вы же собрались, правильно?}
\people{Угу.}
\soul{Выйти с автобуса. Я составляю. Ага, вы выходите – и соответственно, дальше идут план, что и как будет дальше. И довольно-то подробно, хотя бы в общих чертах иметь представление, а вот я уже должен всё это назвать подробно. И вот представьте, вдруг вмешивается кто-то, и вы не успеваете выйти с этого автобуса. Как вы думаете, я уже ложную карточку нарисовал?}
\people{(Белимов) Не знаем, наверное, следующую карточку нарисовал?}
\soul{Я беру, рисую следующую. А куда я деваю ту карточку? Что, просто аннулирую и всё?}
\people{(Гера) Другому кому-то наверно…}
\people{(Белимов) Не знаю, нет. А куда девает, мы не знаем.}
\soul{И я её подсовываю любому другому. Только лишь бы хоть немножко был бы похож на вас, понимаете? Я возьму эту карточку и подсуну же ему. И хотя ему надо было выходить на следующей остановке,  он, паразит, выйдет на этой.}
\people{(Белимов) Угу, всё ясно.}
\soul{Понимаете? Потому что я, в отличие от вас, должен ценить свой труд, и поэтому каждая любая карточка она никогда не пропадёт.}
\people{(Белимов) А у того человека же свой есть хранитель времени, он же будет возражать.}
\soul{Знаете, меж собой мы всегда договоримся.}
\people{(Белимов) Договориться, да?}
\soul{Это вы между собой не можете договориться! А мы всегда договоримся – это первое. Второе: я же могу, я это чаще делаю. Понимаете, как вы делаете? ``Примечание'', так?}
\people{(Белимов) Угу. Так.}
\soul{Или есть такие ещё графы ``Особые отметки''. А почему бы мне в графе ``Особые отметки'' или ``Примечание'' не сразу бы… как говорится, дать себе шанс эту карточку отдать другому? ``Особые отметки'' - случай того-то, того-то. Всё, как у вас – чистая канцелярия.}
\people{(Белимов) А может…}
\soul{Так что всё элементарно, всё очень просто. А теперь представьте: это всё происходило, когда вы просто едете. У вас мысль толкнула в голову: ``Ой, я выйду здесь или на следующей''. То есть это твёрдо ещё не решено. А теперь представьте: у вас есть мечта. Вы мечтаете: ``Я буду  тем-то, тем-то'' – и вы мечтаете это с самого детства. Тут уже не просто какая-то паршивая карточка, которую можно запихнуть её кому угодно,- уже заключается договор, понимаете? И вот этот договор гласит, что вы…}
\people{(Белимов) Угу.}
\people{(Гера) Мы слушаем.}
\soul{Как вы думаете, какая маска сейчас больше всего вами владела?}
\people{(Белимов) Маска? Ну, мне кажется…}
\people{(Гера) Ума.}
\people{(Белимов) Нет - интереса.}
\soul{Не-е.}
\people{(Белимов) А какая?}
\soul{Интереса, но никак не ума. Потому что, когда есть интерес, и когда маска просто выбивает интерес, то вряд ли маской ума можно здесь что-нибудь сделать. Понимаете? Существует такое понятие, как азарт.}
\people{(Белимов) Угу.}
\people{(Гера) Ну, да…}
\soul{Вы скажете, есть маска азарта? – Нет, есть маска интереса, так?  Вот, представьте: вы азартный игрок – какая маска будет владеть здесь вами?}
\people{(Белимов) Ну, азарта, наверное.}
\people{(Гера) Желания, наверное.}
\soul{Мы только что сказали, что маски азарта нет. Маска желания? Верно. Но достаточно ли это?}
\people{(Белимов) Ну, наверное, ещё маска алчности.}
\soul{Зачем же? Маска желания и маска алчности – это одна и та же маска. Всё зависит только от того, как она овладела вами. Вот. Так вам достаточно этой маски или, всё-таки, надо добавить маску ума?}
\people{(Белимов) Да, наверное, тоже ум надо.}
\soul{А почему ж тогда, чаще, маска желания преобладает над вами, и вы проигрываетесь полностью? А где же маска ума? Где ваш ум, чтобы во время остановиться, а? Потому что вы скажете: ``Есть ещё маска надежды, что вы выиграете'' – да?}
\people{(Белимов) Угу. Всё верно.}
\people{(Гера) Да.}
\soul{А нет, это не верно. Есть ли вообще маска надежды? А, может быть, это та же маска желаний? Давайте посмотрим, может быть, это - маска желаний? Вы тогда мне объясните, чем отличается маска желаний от маски надежды? Вы можете это объяснить?}
\people{(Гера) Ну, тем отличается? Тем же, наверное, чем жажда отличается от желания попить воды.}
\soul{Ой, как, как вы хитрите! Это говорит в вас маска ума или маска лени?}
\people{(Гера) Нет, маска фантазии, наверное.}
\soul{Разве? И всё-таки, давайте не будем ``вокруг да около'', ну, попробуйте! Дайте мне различие между маской желаний и маской надежды.}
\people{(Белимов) Ну, надежда – это более такое объективное, не стремящееся к корысти, а маска желаний  - более такая ``приземлённая''.}
\soul{А если вы надеетесь выиграть лотерею? А если вы надеетесь, что вы удачно украдёте миллион или ограбите банк?}
\people{(Белимов) Нам хотя бы сейчас ``пятьсот”найти. (рублей. Прим.)}
\soul{Ну, вот, ответьте, пожалуйста . Пожалуйста. Пожалуйста. Это надежда или…}
\people{(Белимов) Надежды маска… Надежды… Ну, надежда, конечно, более светлое понятие. Желание, всё-таки, –  более приземлённое. Наверное, только этим различается.}
(конец кассеты)
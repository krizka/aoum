Аоум. глава 37-я 30-11-1996г
Георгий Губин
 
Контакт 30.11.96
Места отмеченные ``(…)'' 
нуждаются в прояснении. 
;	(Гера) Тридцатое ноября 1996 года.
;	(Белимов) Суббота!
;	(Гера) Суббота.
;	(Белимов) Одиннадцать часов утра.
;	(Гера) Да, нам работа…
;	(Подсознание) Должно быть состояние покоя Логического.
;	(Гера) Одного полушария.
;	(Подсознание) И в тоже время, логическое полушарие отвечает за сон. Вы представьте сложность, чтобы как бы отключить его и, в тоже время, не уснуть.
;	(Гера) А-аа… да.
;	(Подсознание) И здесь вот есть интересные моменты, как бы ``ноу-хау'' подсказки, или счёт или счёт, что некоторые дают счёт, да? Или мантра.
;	(Гера) Угу.
;	(Подсознание) Понимаете: Зацепка за какую-то одну мысль и множественное её повторение. Потому, что как бы теряется бдительность и, в то же время, для повторения нужны силы. Понимаете?
;	(Гера) Да, да..
;	(Подсознание) В котором участвует именно Логическое полушарие. И вот как у меня это множество красок, калейдоскоп красок. И что еще интересно, что где-то на подсознательном уровне я понимаю значение этих красок, а логически - нет. Понимаете, если я пытаюсь логически разобраться в этой гамме, да, и вспоминая прошлое о цветах…то, тут же я получаю,  как бы, наказание. Это… головокружение, или, вот, понимаете — потеря ориентации. Что довольно-то чувствительно в моём состоянии. И я тогда, отказываюсь от этого занятия, просто рассматриваю, понимаете, как цветы. Когда я не даю задачи рассматривать пестики тычинки, понимаете, а просто наслаждаюсь красотой. Я понимаю, что это красиво, но логически мне даже как бы запрещается рассуждать, в данном случае.
;	(Гера) Ну, понятно.
;	(Белимов) А тогда какую информацию мы можем получить? И мы тогда никакую же не получаем информацию и так далее… Это одна из наших целей.  
;	(Подсознание) Ну, все относительно понятие информации. Понимаете, даже  цель может быть одна, а вот поиски её могут быть множественные. И даже бездействие может привести к цели. Но здесь нет участия конкретно Вашего сознания. Да? И вы не осознаете цену этой цели. Как вы говорите - даром досталось, да,  - даром и потеряете. А вот смотрите - глаз. И вы должны, наверное, помнить, что там есть слепое место, именно соединение нервных окончаний к глазному яблоку. Да? То есть, получается, мы видим картину, и в середине вот это слепое пятно, которое не воспринимается. Но мы-то видим картину полностью, мы не замечаем же этого пятна, правильно? Почему? Это колебания, постоянные колебания и дорисовка картины. Понимаете? То есть, мозг как бы дорисовывает, ну, если хотите,  фантазирует эту картинку и довольно-то с большой реальностью. А вот, заметьте… Чек..
;	(Гера) Да мы слушаем…
(Контакт прерывается, идет счет)
;	(Подсознание) Напомните.
;	(Гера) Как…
;	(Белимов) Вы про глаз говорили…
;	(Ольга) Это Он…[Белимову]
;	(Белимов)…и реальность картинки которую мозг фантазирует.
;	(Подсознание) Да. И в состоянии гипноза или каком либо, человек может вспомнить даже, что не видел. Да? Ну, представьте, вот вы идёте по дороге, и вы практически не обращаете ни на что внимание ``Вы задумались'', но если ввести вас в какое-нибудь состояние, вы можете вспомнить, что встречалось и даже то, что находилось позади вас, и вы не могли увидеть. Понимаете? Почему? Потому что мозг дорисовывает. Для чего? Потому что это ещё осталось со старых времён когда вы были дичью и инстинкт, именно мозг, дорисовывал картины. Как бы расширял ваше зрение. Понимаете? А вот и заметьте, что человек теряет угол зрения, оно становится всё уже и уже, потому что ему уже не надо бояться удара со спины. Понимаете? И вот, заметьте, ещё, одно из признаков душевной болезни - это резкое сужение угла зрения. То есть,  иными словами - мозг теряет способность дорисовывать, как вы говорите, фантазировать. А это очень важно, это очень важно, даже если вам уже не надо вся эта цель. Почему? Потому, что мозг, это та же самая мышца и её тоже можно тренировать. И вы довольно-то активно должны это делать. Чтобы научить вашу лень, мозгу приходится фантазировать и ставить вас в различные, как бы тестируемые, положения. 
;	(Гера) Угу.
;	(Подсознание) Понимаете? Это понятие инстинкта. Мы стараемся, как бы уйти от него. Понимаете, для нас[имеет ввиду людей]инстинкт - это что-то животное, не естественное для нас. Но мы забываем, что это естественное для природы. И всё отрицательное, да, все отрицательные эмоции мы воспринимает именно, и обзываем это словом ``инстинкт''. Понимаете? Чтобы, как бы облагородить себя. 
;	(Гера) Ну, понял. Ну, да.
;	(Подсознание) И если вы чего-то испугались и совершили поступок из-за страха, да, вы будете говорить, что это просто проснулся в вас инстинкт, и это как-то вас оправдывает, понимаете? И даже вас, получается, будут не ненавидеть, а жалеть. Понимаете? 
;	(Гера) Это даже в судебной практике используется.
;	(Подсознание) И вот, смысл в том, что мы создаём множество, понимаете - понятий чувств. Чувства у нас остались, но мы стараемся расширить их, понимаете,  чтобы и оправдать себя, и обмануть себя и других. Вот вы говорите Астральные мирs, да? И разделяете их на Низшие и Высшие. Ну, вообще-то, это не верно. Понимаете? Разделяется именно эмоциями, раздел эмоций ``низкий'', ``высокий''.
;	(Гера) Понятно.
;	(Подсознание) И то - это тоже не правильно. Почему? Потому, что одна и та же эмоция может иметь и положительный и отрицательный характер. Понимаете? В зависимости от обстоятельств. И потому, было бы самое точно более близкое распределение этих вот миров именно на — не низкие и высокие, а на негативные и позитивные разделение эмоций. Понимаете? Тот же страх, он может дать негативное, понимаете, и, в тоже время, положительную реакцию. И заметьте, одно и то же чувство в разных условиях имеет, опять же, полярность — негативную и позитивную сторону. Ниже…
;	(Гера) Да..
(Контакт прерывается, идет счет)
;	(Подсознание) Напомните.
;	(Гера) Про эмоции. Положительное, отрицательное. Про Астральные миры.
;	(Подсознание) Астральные миры?
;	(Белимов) Разделение. На низкие, на высокие делим.
;	(Ольга) Тот же страх может и в положительную, так и отрицательную… 
;	(Белимов) А на самом деле, может быть это надо вибрации Низкие/ Высокие?.
;	(Подсознание)  Нет. Понимаете… Мы… Нам понравилось это слово ``вибрация'' и мы постоянно его употребляем, где нужно и где нет. Понимаете, разделение Астрала заключается не именно в понятии Низкий или Высокий, таких понятий нет. Это вы придумали Низкий план, Высокий план для своего же оправдания, понимаете? Если вам появились ``зеленые человечки'' или ``чертики'', да, от алкоголя, вы скажете это низкий план, да? Но если вы в том же алкогольном опьянении увидите божественное, вы уже скажете, что это Выше план, правильно? 
;	(Гера) Ну да, да…
;	(Подсознание) То есть, это уже выбор, как поставить вот эти вешки, мерки, да? – это ``Низшее'', - это ``Высшее''. То есть, по вашему восприятию, по вашему, понимаете, по вашей морали.
;	(Гера) Ну, да в принципе.
;	(Подсознание) И видите, дело в том, что эффект был один и тот же. Тоже алкогольное опьянение, но почему два таких, в вашем понятии, огромных различия? 
;	(Гера) Значит, было положительное алкогольное опьянение или отрицательное.
;	(Подсознание) И как это вы понимаете? Понимаете, дело в том, что именно, вот как бы объяснить вам…  проявление этого инстинкта заключается и в чувствах. Понимаете? Инстинкт не имеет языка, и ему приходится воздействовать на чувства, создавать эти чувства. Помните состояние когда вы все что-то хотите, а что не знаете…Вы хотите что-то сделать и, в тоже время, вами одолела огромная лень, и вы не можете найти себе места, понимаете? И самое страшное вы не знаете, что вам делать. Вот это инстинкт создаёт новые эмоции, новый уровень понимаете, эмоций. Почему? Потому, что если вы будете постоянно… как говориться, вами будет владеть Страх,  то у вас будет привычка, и вы уже перестанете обращать внимание на этот страх, понимаете? То есть, эти реакции, реакции логические, физические, эмоциональные реакции, они будут притупляться постепенно. Понимаете, всю жизнь прожить в Раю - это будет очень скучно, это будет действительно очень скучно, потому, что вы будете испытывать только одно чувство и больше никакое, понимаете, в Вашем понятии ваших Религий. А вы хотите этого, вы боитесь, понимаете, изменения, именно изменений эти чувств, понимаете? Почему? Потому, что у вас есть понятие это - лени, да? И эта лень даже в чувствах проявляется. Вы хотите радоваться, понимаете, и малейшее изменение - вы тут же… как говориться, начинаете заниматься самокопанием и искать причину, а почему изменилось это настроение, и тут же выдумаете целую кучу новых. Понимаете?
;	(Гера) А почему оно на самом деле изменилось если вот так взять? Не  ни с того ни с сего же? Ведь внешние причины действуют. Да? 
;	(Ольга) Он ушёл….Время
;	(Белимов) Да.
(Контакт прерывается, идёт счёт)
;	(Ольга) Тебя можно назвать по имени в этом…
;	(Гера) В этом состоянии твоём можно назвать?
;	(Подсознание) Имена?
;	(Гера) Да.
;	(Подсознание) Видите, дело в том, что имя… имя -  это, как бы, понимаете… частица вибрации относящаяся именно к этому человеку. И хотя вы можете иметь множество, понимаете… людей носящих одно и то же имя, а характеристика звучания этого имени будет, всё-таки, иная. Возьмем, давайте попробуем любое имя, только не будем называть его. И вот это имя, допустим, носит сотни человек, понимаете?
;	(Гера) Угу.
;	(Подсознание) И вот, представьте,- плюс ещё совпадение и даты рождения. Так? Ну, и давайте ещё, чтоб всё полностью совпадало, то давайте возьмём и чтобы это был близнец, чтобы и место рождения было одним и тем же. 
;	(Гера) Угу.
;	(Подсознание) Вот и мы тут же, понимаете, решим, что у них будут одинаковый характер, потому, что мы взяли одинаковые… имена, а имена описывают характеры… так?
;	(Гера) Ну, да.
;	(Подсознание) В Астральном плане есть, конечно, изменения, почему? Потому что одновременно не могут родиться два человека. Кто-то позже, кто-то раньше, кто-то будет самый старший, кто-то будет младший. Понимаете?
;	(Гера) Да.
;	(Подсознание) А мы не знаем этих тонкостей, и поэтому, в принципе, скажем, что эти люди одинаковые.
;	(Гера) Ну, они близки будут, да?
;	(Подсознание) И понимаете, даже ведется статистика, что близнецы влюбились в одних и тех же, работали по одной и той же специальности, а вот статистику несовпадений почему-то никто не ведёт. Понимаете? Как в отношении снов. Понимаете? Вот, совпал сон, и мы его запомнили. А множество, что не совпало, мы тут же забудем. Почему? Потому, что наша память избирает, понимаете, то, что Нам нужно, и то что Нам нравиться, то есть, вот здесь происходит проявление чувств. И вот, произнеся имя, - а произнести мы его можем по-разному - с чувствами, без, просто так. Правильно?
;	(Гера и Белимов)  Угу.
;	(Подсознание) Вот, вспомните навязчивые слова какой-нибудь песенки - они так и крутятся в вас, и крутятся в голове, и вы не можете от этого избавиться. А вот откуда это взялось?
;	(Гера) Откуда?
;	(Подсознание) Откуда? А это защита! Именно защита, понимаете, это может быть проявление защиты. Ибо мозг должен отвлечься от ваших проблем, от личных ваших проблем и заняться своими проблемами. Понимаете? Или  может быть извне, чтобы не дать вам совершить какую-то умственную работу. 
;	(Гера) А почему?
;	(Подсознание) Но вы же сами создали себе врагов и довольно-то неплохих, понимаете?
;	(Ольга) Конечно.
;	(Подсознание) И, причём, у вас понятие Враг, - чем выше этот враг, да, тем вам даже как-то приятнее. И вот, когда какой-нибудь вражонок, маленький такой, противный, да, он, как вы говорите, и ``выеденного-то не стоит'', и вам даже будет обидно, что он является вам врагом. Понимаете? Вам нужен Враг, так такой Враг - чтобы Ух, какой! 
;	(Гера) Ну, когда трудности…
;	(Подсознание) И что интересно - вы к врагам относитесь более серьёзно, чем к друзьям. Понимаете, что вы самое страшное, что с друзьями вы допускать вольности, множество вольностей, которые могут оскорбить и вас и его. Понимаете? И вот тогда дружеские узы будут теряться. А вот с врагом вы это не позволяете потому, что вы нас стороже и постоянно боитесь его. Понимаете? Или, как правило, дорисовываете множество ему качеств, которым он даже-то не обладает. Вам кажется, что этот враг умеет читать ваши мысли, понимаете, и вы найдете множество подтверждений этому…..[завис]
(Контакт прерывается, идет счет задается вопрос)
;	(Подсознание) Что именно в алкоголе? Алкоголь, это, в принципе, тот же наркотик, что и любой другой, понимаете?  Эйфория по картине — это тоже алкоголь. Только, понимаете, здесь вызывает алкоголь э-э….[задумался] другими, эмоциональными планами. А вот при распитии - чисто химией. Понимаете? Вот что интересно, тут же поставлены вешки - положительные/отрицательные. Если эта эмоция была вызвана алкоголем, то это отрицательно, понимаете? Когда пьяница объясняется  вам в любви, вы ему не очень-то верите, если вы сами-то трезвый. Понимаете?
;	(Гера) Да, да..
;	(Подсознание) Вот, а тут же, в принципе-то, и забываете - что у пьяного на уме, то у него на языке. Понимаете?
;	(Гера) Что у пьяного на уме…
;	(Белимов) Угу.
;	(Ольга) Что у трезвого на уме - то у пьяного на языке!
;	(Подсознание) Пусть будет так… но было бы точнее — ``что у пьяного на уме''. Понимаете, почему? Потому, что здесь совершенно другой план эмоций.
;	(Олгьа) Угу
;	(Подсознание) Понимаете, и поэтому, в принципе, было бы точнее - что у пьяного на уме. Именно на данный момент.
;	(Олгьа) Ну, ясно.
;	
;	(Подсознание) Потому, что у него нет такого понятия, как контроль, контроль за собой. Понимаете? И кто бы вы ни были, он будет с вами более откровенен. Почему? Потому, что логическая машина… она притуплена сейчас, понимаете, той химией, алкоголем и…. не обязательно алкоголем…
;	(Гера) Понятно.
;	(Ольга) Ну, ясно.
;	(Переводчик) … Ведь когда вы доверяете друг другу, то вы и трезвые - тоже теряете контроль. Правильно?
;	(Ольга) Да.
;	(Белимов) Угу.
;	(Переводчик) Понимаете, вот это вот логическое мышление - вот это можно, это нельзя, вот эта граница, вот эта мораль, которая сложена в вас, она потеряла власть над эмоциями. Понимаете? И поэтому, нельзя сказать ``что у трезвого…''. Трезвый и пьяный - это совершенно разные люди, понимаете, и совершенно разное восприятие мира. Это совершенно всё абсолютно разное. 
;	(Гера) Здесь личности разные…
;	(Переводчик) Личность одна, а вот восприятие этой личности на окружающий план, на создание эмоций, на создание чувств, а точнее - отражение  вот этих чувств на личности, будут совершенно разное. И тогда, вы, конечно, можете сказать что — трезвый человек - как человек, а пьяный – чёрте что. Понимаете? ``Совершенно другой человек'', как вы говорите.
;	(Гера) Ну, да бывает, такое иногда.
;	(Переводчик) Ну, вообще-то, в этом плане - да. Почему? Потому что… что в нашем понятии человек - для другого человека?
;	(Гера) Это - отношение к нему.
;	(Подсознание) Отношение к нему! Отношение к окружающей среде…Понимаете? И вот, как вы, увидев человека, ставите ему сразу характеристику правильно, положительную или отрицательную и, и…
(Контакт прерывается, идет счет задается вопрос)
;	(Переводчик)… человек ставит характеристику к отношению….?
;	(Белимов) Нет. Как меняется, насколько меняется под влиянием алкоголя восприятие других людей, характеристик его…
;	(Гера) Именно людей.
;	(Белимов) И насколько мы правы в этом… при этом, бываем. Может действительно раскованность - она больше дает правдивости в оценках человека, нежели….
;	(Переводчик) Ну, понимаете, это всё относительно. Всё относительно именно к окружающему. Пьяный человек в пьяной компании ведёт себя нормально.
;	(Гера) Ну, да.
;	(Переводчик) А вот пьяный среди трезвой компании - это уже не адекватно. И, понимаете, характеристика… Уточните, характеристика пьяного человека?
;	(Гера) Да, да, пьяного.
;	(Белимов) Пьяный по-другому как-то оценивает других людей. Насколько эта оценка, ну, справедливее, может быть, раскованнее? Может он действительно больше видит в этом состоянии? Как наркотическое средство…
;	(Переводчик) Справедливости нет. Понимаете? Здесь, вот вспомните юродивых -они не всегда говорили истину, но, всё-таки, она проскакивала, понимаете? А трезвый человек может никогда не повторить этой истины, никогда не сказать её, потому что она ему не потребна. Понимаете? И потому, тот же юродивый, и те же пьяные, они,чаще, произносят именно истину, но не всегда. Почему не всегда.   Потому, что у них не…. [пауза 5 сек тишина,]Вы, видите один и тот же предмет… Bли давайте так  - ``на вкус и цвет товарища нет'', понимаете?
;	(Белимов) Угу.
;	(Переводчик) То есть, один и тот же вкус - а восприятие будет другим. Может, тогда вкус всё-таки разен? В химическом плане, даже в химическом плане вкус будет разен. Почему? Потому, что тело одного человека обладает какими-то другими, более другими витаминами, металлами, и поэтому, восприятие вкуса, разложение этих составляющих на вкус будет совершенно иным. Даже в химическом плане вкус будет другой. Цвет. Ну, самый тяжелый случай, это дальтоник. Они не различают вообще цвета, поэтому он не сможет понять вашего восхищения о какой-то картине, понимаете, или о закате Солнца. Ну, а если взять -два нормальных человека — ну, кому-то нравиться, а кому-то нет. Вот представьте, муж и жена и присутствуют, допустим, на восходе Солнца - а восприятие будет разным. Почему? Потому, что до этого было наложение других эмоций, других планов. Может жена была до этого чем-то недовольна, устала или что-то, а вы - наоборот, понимаете? И восприятие будет совершенно другое. Почему? У Вас сейчас другой план чувств, понимаете, другой набор чувств. Давайте не будет говорить о что-то Астральных мирах там, о какие-то… нет, именно вот чисто физическом плане, именно у вас, сейчас, другая химия, - химия чувств. Понимаете? И поэтому, у вас не будет одинаковых реакций. А заметьте, когда оба и… [завис]
;	(Ольга) Говори, говори.
;	(Гера) 1,2,3,4,5
;	(Ольга) Мы тебя слушаем, слушаем. Слушаю.
;	(Гера) 6,7,8,9 [ведёт счёт]
;	(Переводчик)  А?
;	(Ольга) Слушаем тебя.
;	(Переводчик)  Напомните. 
;	(Белимов) Когда у…
;	(Гера) Химия чувств.
;	(Белимов) Химия чувств. Когда оба оценивают закат. Допустим, муж и жена, в разном состоянии эмоциональном.
;	(Ольга) Он уже другой!
;	(Переводчик)  Понятно.  Но видите, дело в том, что есть и реакции, когда они всё- таки,- муж и жена,- воспринимают одинаково. Понимаете? А что для этого нужно? Понимание друг друга. Настрой на друг друга. Резонанс. Понимаете? Тот самый резонанс, тот самый резонанс душ. А мы сейчас говорим как? Вот биополе мужа, вот биополе жены - и по привычному нашему физическому, простите за выражение -“сексу'', воспринимает также и биополе. Понимаете? Совместим только тогда, когда вот в этом биополе есть дырка, а вот в этом биополе наоборот, есть выступ. Понимаете? И тогда они совмещаются и всё нормально. Понимаете, но это же глупо! Это мы, попытка физического, именно физического восприятия переносим на вот именно эти тонкие чувства. То есть, мы здесь стараемся дорисовать именно логикой, понимаете, той самой логикой, картину.  И мы, конечно, за аналогию берём самое примитивное….[завис]
;	(Ольга) …примитивное. Слушаем.
;	
(Контакт прерывается, идет счет)
;	(Гера) Один.
;	(Белимов) Собачка пришла. А может, ты чувствуешь ещё кого-то? Кто прибавился? Ты не видишь никого? Может кто-то из невидимого мира к нам пришёл?
(Контакт прерывается, идет счет)
;	(Белимов) Мне хотелось бы узнать о сужении зрения,  в связи с этим и сужении мыслительной способностью восприятия мира. Насколько это эволюционный процесс полезен человечеству, или это уже технократическое какое-то ограничение? Это хорошо или плохо? Мы к чему идём?  
;	(Подсознание) Нет, понимаете… Как можно сказать о полезности или о вредности? Дело в том, что когда вы  общались более близко с природой и были среди природы, и для вас было множество противников, которых вы боялись и были вынуждены боятся и защищаться от них. А самая лучшая защита это, всё-таки, увидеть первым.
;	(Белимов) Угу.
;	(Подсознание) И потому, угол рения был больше, понимаете? Можно сказать, что дикий человек практически видел на все 360 градусов. Понимаете? Физически - нет, конечно, но мозг, мозг, понимаете, способствовал. Именно он включал все чувства. Понимаете? Все. И если зрение не могло увидеть задний сектор — помогало…. 
;	(Гера) Кожное зрение.
;	(Подсознание) Кожное зрение - оно всегда было малоразвитым, и это ещё впереди. Помогали другие чувства. Тот же слух. Тот же малейший шорох, понимаете? И заметьте, что в принципе-то, все наши органы обращены-то в одну сторону. Правильно? Физически, куда смотрит глаз, туда практически, как говориться, и слушают уши, понимаете? И даже нос там же. То есть, все чувства собраны практически в одной точке. Понимаете? Основные чувства, которые дают нам максимум информации. А почему? Почему? Потому, что нам нужно было увидеть противника первым, и поэтому, нам нужно было, как бы обратить внимание всех этих чувств в одном направлении, усилить эти чувства. Понимаете? А вот, среди животных …
;	(Ольга) Один
(Контакт прерывается, идет счет)
;	(Ольга) Семь, восемь, девять.
;	(Белимов) …Знания.
;	(Подсознание) Да и у вас есть животные, у которых, как говориться, чувства разнесены. Понимаете?  И..[завис]
;	(Гера)Да, мы слушаем.
;	(Ольга) Ещё раз.
;	(Белимов) Ты в этом состоянии чувствуешь, что ждёт человечество впереди, - сужение или угла зрения там, мыслительных способностей, или какие-то новые чувства будут приходить на смену?
;	(Подсознание) Безусловно. Безусловно, сужение будет происходить, к сожалению.
;	(Белимов) Так, так.
;	(Подсознание) Понимаете… дело в том, что нам уже не требуется вот этот кругозор. Понимаете? Не, ну конечно можем сказать, что как это было бы удобно  если бы я видел вокруг, понимаете? Я же мог и шпаргалочки, как говориться, и списывать у соседа. Но видите, дело в том, что это не жизненно необходимо. Да, мозг рисует, это нужно, а вот жизненной необходимости в этом нету, и поэтому - теряется, уходит.
;	(Белимов) Уходит, да?. А что взамен идёт?
;	(Подсознание) Уходит. И, понимаете, дело в том, что…[завис]
;	(Ольга) Один, два, три…
;	(Белимов) Что идёт взамен этим чувствам, которые мы теряем?
;	(Гера) Семь, восемь, девять…
;	(Подсознание) Понимаете, дело в том, что вы…как бы вам объяснить… Вы превышаете децибелы, вы превышаете болевой порог восприятия. Понимаете? То есть, мы хотели бы попросить вас говорить тише. 
;	(Белимов) А-аа. Пожалуйста.
;	(Подсознание) Видите, дело в том, что… Лучше напомните. 
;	(Белимов) Ну, мы разговаривали с вами о том, что на смену старым чувствам: мыслительной способности, уму, зрению,- идут новые. А вот какие новые? Может быть это третий глаз? Вот, нас интересует, допустим, третий глаз.
;	(Подсознание) Вы знаете, третий глаз, вообще-то, это не новое, вы потеряли его когда-то. Понимаете? Вспомните, -  тот же самый гипофиз. Вы сейчас привязались к гипофизу и говорите: вот это был третий глаз потому, что он чувствительный к свету. Понимаете? И тут же говорите, приводите доказательства, что, раз он чувствителен к свету, то значит… - а зачем это мозгу видеть свет внутри? Понимаете? А зачем? А вот мозгу-то надо и внутри видеть, потому, что множество химических реакций происходит именно понимаете, при ээээ….[задумался] выделении этого света. Вот, тот самый ``третий глаз'' вы, как правило, его рисуете, опять же, впереди, собирая все органы вместе. Ну, где ему само удобно стоять? Раз это третий глаз, то, наверное, пусть он будет, простите за выражение, и между глазами. Понимаете? Третий…
;	(Белимов)  Угу.
;	(Подсознание) …чисто по инерции мышления. На пятку вы его не поставите. Понимаете? Хотя сейчас ученые приходят к множеству фактов, что, практически, глаз может находиться, практически, в любом месте, понимаете? Потому, что важен именно не сам глаз, не сама его анатомия, хотя это и способствует, конечно, лучшему видению, но и кожное зрение - это тот же самый глаз. Вы согласны?  
;	(Гера)  Угу.
;	(Подсознание) То есть, что основное в глазе? Оптика? Нет! 
;	(Гера)  Нервы.
;	…Именно вот эти вот нервы, отвечающие… эти колбочки, отвечающие за восприятие света, - а кожа воспринимает свет. Понимаете?
;	(Белимов) Угу.
;	(Гера)  Да, понимаем.
;	(Ольга) Счёт.
;	(Гера)  Один, два …
(Контакт прерывается, идет счет)
;	(Гера)  Что виделось, что зналось? Пока лежишь-балдеешь.
;	(Белимов) Да не надо выключать пока…[про магнитафон]
;	(Подсознание) Цвета какие-то…
;	(Белимов) Цвета горят?
;	(Ольга) У меня тоже.
;	(Белимов) [что-то невнятное]
(Продолжение контакта)
;	(Подсознание) Практически, я могу ответить на множество вопросов, но зависит именно вот, от вашего…
;	(Гера) Угу. Настроя, да?
;	(Подсознание) Ваши вопросы… Понимаете? И причём, так как сейчас - логическая часть более отключена, но, всё-таки, она находится в активном состоянии, вот. В активном пассиве. И, в принципе, это позволяет, как бы избавиться от ваших эмоций в плане, вот, мысленном плане, потому что чувствительность повышена. Чувствительность повышена, и вот все физические составляющие воспринимаются довольно-таки на большом уровне. И, например, громкий звук - он уже принимается, как болевой. Понимаете?
;	(Белимов) Хорошо. 
;	(Подсознание) Яркий свет, который был бы нормальным, допустим, воспринимается уже, как болевой.
;	(Гера) А, и…
;	(Подсознание) И понимаете, вот это вот, как вы говорите, электромагнитные составляющие  - именно работа вашего мозга - она воспринимается так же, как и остальные, и это же тоже физическое. И мы обладаем этими органами, чтобы воспринимать, как говориться ``читать мысли'', но сейчас желательно бы этого не делать, потому, что мы не умеем управлять своими мыслями. И, как правило, чаще всего, в нашей голове хаос и множество-множество вопросов, которые стараются быть первыми - и… на какой отвечать? Понимаете? И какой больше… Конечно я могу, я могу сейчас отсортировать, понимаете, отсортировать и найти тот вопрос, который вам действительно важен, но тогда мне нужно будет применить всё- таки логическую часть. Понимаете, мы слишком уж разделили Логическое/Эмоциональное, мы разделили половинки до такой степени, что кажется, что они независимы друг от друга. Нет, дело в том, что видеть тем же третьим глазом, да, или читать мысли, -  для этого тоже нужен логический блок. То есть, мозг не имеет лишнего, понимаете?
;	(Гера) Угу.
;	(Подсознание) И когда мы говорим, что мозг работает всего лишь только на 3 процента, а это уже хорошо, потому что, как правило, среднее число, это, всё-таки, 0.3 . Представляете, как это ужасно иметь то и пользоваться всего лишь только 0.3? И хотя бы эти 0.3 или эти 3 они, всё-таки, работают благодаря оставшимся процентам. Понимаете?
;	(Гера) Ну, понятно.
;	(Подсознание) Давайте вспомним, когда в виде какой-то физической травмы или болезни было уничтожено практически большая половина мозга, и человек при этом умудрялся, как вы говорите, функционировать нормально. Понимаете?
;	(Гера) Да, было такое.
;	(Подсознание) И вот, смотрите, что интересно, здесь есть две крайности, когда мозг практически полностью разрушен и человек обладает ещё способностью мыслить и о своем разрушении узнает только тогда, когда  об этом говорят ему врачи. И маразматик, обладающий нормальным мозгом, и, в то же время - им не пользуется. Понимаете?
;	(Гера) Угу.
;	(Подсознание) А именно в чём смысл? Смысл -  именно в этих связях. Понимаете? Эти связи… Вот вы говорите каналы, да? Закупорка каналов — это у вас такое любимое выражение и, в то же время, понимаете, вы не можете представить и нарисовать эти каналы, вы даже не можете найти эти каналы в человеке, понимаете, хотя допускаете что их множество, и вы, тут же придумали множество понятий. Раз есть биополе, то, значит, есть, конечно, и биоточка. Понимаете? И вы стараетесь… [обрывается]
(Контакт прерывается, идет счет)
;	(Гера) Да.
;	(Белимов) Вы о биотоках говорили. О том, что это может каналы, забитые каналы.
;	(Гера) И, кстати, вопрос сразу, вот - ученые не могут найти их на теле, но луч лазера входящий именно в точку проходит, как по световоду. То есть, он выходит из всех точек этого канала или каналов вообще.
;	(Подсознание) Но, понимаете, вы ищете, в принципе, вы ищете чисто на физическом уровне.
;	(Гера) Ну, да
;	(Подсознание) И хотя уже создана наука — Информатика, вы ещё не воспринимаете её, и не прикладываете её к человеку, к природе человека. А надо бы сказать, что, в принципе, важны не части механизма. Ну, вот представьте: механизм, множество шестеренок, но они не связаны между собой, они не совершат никакой полезной работы. Вы согласны?
;	(Гера) Да.
;	(Подсознание) Это всего лишь будет груда металлолома. А вот связь между ними, если хотите сказать — компьютер. Понимаете? Это груда металлолома пока нет программиста, пока нет программы. Есть программа -  и эта груда металлолома уже будет работать. Чисто на информационном… правильно? И, заметьте, что интересно, что мы получаем физическое, хотя воздействуем чисто информацией. 
;	(Гера) Да.
;	(Подсознание) И вот, мы, зная и изучая эту науку, не можем применить его к человеку, и потому мы не сможем множество найти, что есть в нас. Да мы понимаем, что мы состоим из множества клеток, но когда мы будем соединять эти клетки, мы не заставим их взаимодействовать между собой и создать жизнь. Мы можем взять множество химических продуктов и соединить их, получится эмбрион, но почему-то он не сможет жить. Понимаете? Мы не дадим ему жизнь. Именно то, что мы называем ``душой'' и то, что нам даёт бог. Понимаете? А почему? Потому, что мы не можем применить информационные поля. Мы знаем их существование, знаем, но не можем их применить. Понимаете?
;	(Гера) А как их применить?
;	(Подсознание) А вот смотрите, давайте вернемся, опять же, к биополям и биоточкам.
;	(Гера) Угу.
;	(Подсознание) Да, конечно, человек состоит из множества, как говорится, биоточек. Было бы точнее сказать, что не человек содержит биоточки, а биоточки строят человека. Именно из них создан человек,  а иначе, - а иначе, это значит, что мы имеем не только их на теле, но и внутри него. Понимаете?
;	(Гера) Угу.
;	(Подсознание) Именно каждая любая клетка является именно этой точкой, но не каждая клетка способна воспринять информацию на тонком уровне. Понимаете? То есть, мы можем на любую клетку воздействовать, допустим, тем же электрическим током и, соответственно, будет исходить какая-то реакция, тот же электролиз. Правильно? 
;	(Гера) Угу.
;	(Подсознание) То есть, точка отреагирует. Клетка отреагирует на этот сигнал, но на более тонкий сигнал она не сможет реагировать. Почему? Почему? Потому что это тоже защита. Вот, представьте, если все клетки будут работать сами по себе -  нет, для этого нужен координатор, понимаете? Именно то первое звено, которое является началом, если хотите, началом вот этого вот чувства -  осязание, слуха, зрения чего угодно, тактильного ощущения, понимаете, и вот именно вот эта точка, которая имеет начало, то бишь вход, вход информации, вот это и есть и называется биоточка. Понимаете? И вы говорите о каналах — нет, в таковом понятии каналов не существует, существует просто цепочка. Цепочка тех точек, которые способны передать информацию, перенести её, переработать на уровень других клеток. Вот, смотрите, мы взаимодействуем на какую-то точку, так? Чтобы, допустим, вылечить какой-то организм. Мы физически с этим организмом не контактируем. Вы согласны? Мы, чисто, придём только информационно и, опять-таки, что вы называете каналом - это цепочка, цепочка этих клеток ``биоточек'' преобразовывает, как вы говорите, создаёт нужную вибрацию для этого организма. Понимаете, как….
;	(Гера) Угу. Понимаем.
(Контакт прерывается, идет счет)
;	(Белимов) Вы говорили о том, что цепочки создают нужную информацию для организма…вносят нужную информацию.
;	(Гера) Воздействуют.
;	(Подсознание)  Но, видите, дело в том, что мы сейчас стараемся это делать как бы извне,- рассуждая логически. Понимаете? Мы уходим от природы, к сожалению, мы уходим от природы. К сожалению, меняются все наши чувства восприятия к окружающему миру, а значит и меняется восприятие этих ``биоточек''. Иными словами - они грубеют. Почему они грубеют? Потому, что для того, чтобы чисто в природном состоянии взаимодействовать с этой точкой, нужно было достаточно мало энергии. Сейчас, когда вы окружены электромагнитными полями - а это часть природы, и причём они промодулированы вашим разумом, а не природным, - в итоге получается - точки грубеют потому, что они должны защищаться. Вспомните радиацию. Тело должно закрыться, оно как бы одевает на себя кокон. Как оно это делает? Вы говорите ``биополе'', и ей приходится уменьшать размеры этого биополя, чтобы увеличить её плотность. Понимаете? Чтобы сделать её более, как вы говорите, непробиваемой.
;	(Гера)  Угу.
;	(Подсознание) Вот, представьте - животное. Животное может найти хозяина даже  за тысячи и тысячи. Вы согласны? 
;	(Белимов) Угу.
;	(Подсознание) Почему? Потому, что оно не отрывалось от природы и, хотя ей тоже губительна ваша атмосфера, ведь вы-то окружающую среду-то засоряете, и довольно-то неплохо, всё-таки, логически, тоже животное не воспринимает их. Понимаете? Потому что оно не знает ваших слов. Вот, представьте, идёт радиопередача, а их множество радиопередач, и вот этот хаос слов тело ваше и воспринимает. Вспомните тогда ``зубного врача и германий''. 
;	(Гера) Угу, угу.
;	(Подсознание) И вот, понимаете, ваша логика старается расшифровать, промодулировать и превратить в слова, понимаете, все эти сигналы, и потому оно отвлекается и старается определенные точки, как бы ``обострить''. А за счёт чего? За счёт других точек. И вот, тогда, вы можете говорить об опухолях, понимаете? О тех злокачественных опухолях, которые вы боитесь. А ведь раньше их было гораздо меньше! А почему? Вы сейчас говорите - это инопланетяне, инопланетное что-то входит в ваше тело. Понимаете? Есть у вас такая версия?
;	(Белимов) Угу
;	(Ольга) Есть.
;	(Подсознание) Есть! Нет, дело в том, что ваша…Ваш мозг, именно логическая часть, понимаете, оно обострило, обострило эту точку для каких-то своих целей за счёт других, а это значит, что она все функциональные каналы, -всё-таки будем пользоваться вашим термином ``каналы”- перегнала именно на эту точку и, тем самым огрубив её….[контакт обрывается]
;	(Белимов) Вы говорили о том, что все информационные каналы, подчас, у нас перегоняются в одну точку, и это огрубляет восприятие видимо мира…
;	(Подсознание) И это создаёт опухоли. Понимаете?
;	(Белимов) Опухоли?
;	(Подсознание) Очаги воспаления. Очаги воспаления. И видите, дело в том, что мы это делаем, как говорится, подсознательно. Понимаете? Подсознательно, именно подсознанием. А что такое подсознание!? Это, то же самое сознание, но только не воспринимаемое и не переводимое на мышечные игры. Понимаете?
;	(Гера) Угу.
;	(Подсознание) То есть, чисто информационное. А так как мы рассказали, что информация - это самое главное, понимаете, именно та вот программа. Вот именно то понятие вот  души, вот всё это есть информационное, и оно, как таковое, не имеет понятие физики, понимаете? Это сейчас очень трудно это понять,  потому что информатика… Мы всё-таки информацию переносим физическими носителями,  правильно?
;	(Гера) Угу.
;	(Подсознание) И мы не можем пока… [контакт обрывается]
(Контакт прерывается, идёт счёт)
;	(Подсознание) Беда в том, что вы, имея какую-то одну идею. Так? Какое-то желание одно, да? И вы стараетесь его подавить, потому что вроде как сейчас не к месту, да? И, в тоже время, вот эта вот борьба, вот эта борьба, вот это вот несоответствие вашим планам, очень действует болезненно, как и на вас, так и на других. Понимаете? Если вы хотите и не можете побороть, тогда уж спрашивайте именно что, тогда, вас волнует, потому, что мы чувствуем вашу…
;	(Белимов) Понятно, понятно!  Это, видимо, про меня всё идёт. В принципе, я, конечно, ожидал больше информативного такого, но и эта тема интересна. Например, вот сейчас мы говорили о переносе физическими носителями, а вот интересно в этой связи, как - рак. Вот вы можете оценить в этом состоянии, это что? Создание рака, возникновение рака, это как раз, может перенос информации в одну точку и странное разрастание организма? Но, если не можете отвечать - не отвечайте. Я даже не знаю что… если перейти на вопросы по Тибету, по горам, по экспедиции? Вот, сможешь ли ты в этом состоянии ответить больше, чем я, допустим, ожидаю? Я ожидал от наших контрагентов  ответов, от наших первых, те, кто выходит на первом счёту,- но если мы сейчас на твоем подсознании работаем, на твоем сознании… то, наверное, ты не слишком много знаешь об экспедиции в горы. Можешь что-нибудь ответить?
;	(Подсознание) Но видите, вы опять поставили вас на стадию учения. Они, другие… Вы забываете, что это, всё-таки, всё в одном.
;	(Гера) Ну, так как понять? Вот, допустим, у него в этом, так сказать…на это тело скажем, действуют одни, допустим, ну, личности там, или, я не знаю, как определить. разных планов, да.  Вот на меня, допустим, действуют другие… советы там.
;	(Подсознание) Вот, понимаете в чём, в чём отличие одержимого от, допустим, от меня? Понимаете, дело в том, что одержимый - это когда приходят действительно извне, приходят чужие. Понимаете? Как вы говорите ``инопланетяне''. И они взаимодействуют с ним, как бы добиваясь своих целей, понимаете?
;	(Гера) Угу, то есть грубость.
;	(Подсознание) Здесь же все ``герои'' находятся в одном. А как это объяснить?  Ну, знаете, грубая аналогия - вы. Вы сами. В одно мгновение чувствуете себя одним, в другое мгновение вы чувствуете себя другим, правильно?
;	(Гера) Да, да, да.
;	(Подсознание) Но это всё вы воспринимаете логически. Понимаете? И обвиняете чувства. Понимаете? То есть, вы не допускаете, что в вас сидит множество, живёт в вас множество, понимаете? А вас же… В каждом из вас - ``множество''. Понимаете? И в чём это проявляется? Ну, давайте скажем так - вот вы сейчас в одном настроении, так..?
;	(Гера) Угу
;	(Подсознание) …и вы, значит, уже как совершенно другой. 
;	(Гера) Ну, да.
;	(Подсознание) Изменилось настроение и вы стали опять другим, да? И даже любимый вам человек, он может вас не узнать, испугать вас, отойти от вас. Понимаете?
;	(Гера) Да.
;	(Подсознание) Почему? Потому, что вы, как бы вам объяснить… Вы стараетесь контролировать чувства, так? Это не всегда получается, иногда чувства преобладают над вами, понимаете, и вы этого момента стыдитесь. И даже когда чувства преобладали над вами и дали положительную реакцию, вы всё равно этого момента потом стыдитесь, даже перед близким человеком. Понимаете? Почему? Потому, что вы почувствовали себя, как бы раскрытым, незащищенным. И вот объяснение в любви, действительно в порыве, именно без игры чувств, вы этого момента потом стесняетесь, вы боитесь этого момента.  Так же боитесь, как и момента, когда вы где-то, что-то испугались и, грубо говоря, вели себя неправильно. Понимаете? Логически, вы поставили, что вот здесь неправильны, окружающая среда сказала что - да вы были трусом, понимаете, то же самое. И когда вам говорят, что самое сильное чувство в вас — страх, а что его родило?  - Эгоизм, понимаете? К сожалению, первопричина — эгоизм. Но, вы сами же говорите — Эго первопричина всех ваших чувств, к сожалению. Вспомните древние индейские мифологии. От плохих богов рождались и хорошие. Вы помните? 
;	(Гера) Да, да.
;	(Подсознание) И вот, давайте скажем так, что вот этот именно бог, если хотите сказать - тот самый Зевс. Это есть ва…[контакт обрывается]
;	(Гера) Один…
(Контакт прерывается, идёт счёт)
;	(Подсознание) Задавайте вопросы.
;	(Белимов) Гера, ты хотел задать вопросы.
;	(Гера) А, да-да. А, вот, ты можешь сейчас, в этом состоянии, сказать, всё-таки, и это мне интересно, почему ты, когда ты вот на одном из контактов нас сканировал: Ольгу, меня, мою жену Лену, - ты сказал Лене, что она… ощущаешь её, как ``пустая''? Что за пустоту? В каком смысле пустота? Можешь сказать? 
;	(Подсознание) К сожалению, да, это - действительно ощущение пустоты. Понимаете, человек накапливает годами информацию, перерабатывает её, иными словами - эта информация идёт ему в пищу и в пользу. Но есть люди, которые…  как вы говорите - ``об стену горохом''. То есть, вот эта информация она приходит и уходит, практически не оставляя следов. Понимаете? То есть, как бы вам объяснить?  Это очень вообще-то плохо.  Происходит именно вот этот момент -  начальной деградации. Понимаете? А почему, почему? Потому что это влияет отсутствие интереса. Отсутствие интереса. Почему нет интереса? Потому,  что, вот, эта информация пришла  и ушла. Как вы говорите: ``влетело в одно ухо, -  вылетело в другое''.  Ну, должно же что-то задержаться, должны быть уроки жизни! Уроков жизни вы получаете вдоволь, но не воспринимаете это как ``урок'' и не делаете ``выводы''. Вы не решаете задачи. Когда-то, к сожалению, родители подготовили множество ответов для решения задач. Но, они готовили-то это - для своих задач! Правильно?
;	(Гера) Угу
;	(Подсознание) Вы же, запоминаете эти ответы и стараетесь их воткнуть в любую другую задачу, которая более/менее похожа.
;	(Гера)  Понятно.
;	(Подсознание) Понимаете? А в итоге получается -  вот эти уроки проходят даром…[контакт обрывается]
;	(Гера) Один…
(Контакт прерывается, идет счет)
;	(Подсознание) Напомните.
;	(Белимов) Вы говорили о том, что…
;	(Гера)  О пустоте.
;	(Белимов) ..о пустоте, и информация может уходить из человека и не оставаться. А что он никогда не появиться, этот интерес, у человека?
;	(Гера) Как сделать, чтобы задержалась? Может  помочь там…
;	(Подсознание) Нет. Понимаете, дело в том, что информация….. 
(Контакт прерывается, идет счёт, обрыв ленты)
;	(Подсознание) …и тут нужно задумать о чём-нибудь ином. Ну, и каждый, конечно,-  Отец: ``Ага! Я шофер, значит, ты у меня будешь шофёром!'' Ну, соответственно, значит, ты будешь крутить баранку, чтоб, вроде, как я уже за где-то привык к бензину. Его не интересует, буду я шофером, хочу я им быть или не хочу. Но, в какой-то мере, в то время ребёнок, вообще-то, ``пластилин''. Что хочешь с него, то и вылепишь. Почти. Потому, что это всё-таки не вода, понимаете, а пластилин, его довольно-то тоже трудно обрабатывать. Ну, а так как это всё было пущено на самотёк, то я не стал и шофёром. И, причём, даже на оборот, - я до сих пор не разбираюсь в машинах… Жигули… какие-то, номера какие-то, понимаете…01-02. Для меня это всё пустое.[контакт обрывается]
(Контакт прерывается, идёт счёт, меняется интонация переводчика)
(Краски) Этот переход красок, мы просто, понимаете, вот этот плавный переход мы не замечаем, потому, что мы не можем контролировать свои мысли. Понимаете? И вот, только тогда, когда мы сможем перевести эту мысль на физический уровень, а именно - в движении, движение тем же языком… А многие люди почему-то считают, что они думают именно только тогда, когда шевелят именно языком ``про себя''. Понимаете? Вот.  И вот это мы и называем ``границей'' . Понимаете? То есть, чувствительностью, если хотите. Чувствительностью.  И чем чувствительнее будет человек, тем больше он будет видеть, понимаете, тем больше он увидит, что граница-то, всё-таки, не вот эта вот одна черта, резкая, а вот именно,.. как вам объяснить - это безмерное. Как дальтоник.  Дальтоник - он может увидеть лишь резкое различие цветов, плавные… рисунок состоящий из цветов и..и…и….[контакт обрывается]
(Контакт прерывается, меняется интонация переводчика)
;	(Белимов) Какие у тебя эти…
;	(Гера) Как после контакта?
;	(Белимов) Говори тише.[Гере]
;	(Ольга) А он… нет, это он снова вошёл. Выключай свет. 
;	(Гера) Выключай.
(идёт счёт)
;	(Маска) …Ах, если бы ещё и тараканов считали!
;	(Спрашивающие)[смеются]
\soul{А?}
\soul{(Белимов) Так кого ж тогда считать? Там мы начали голову ломать: живые существа… Или, вот, с нашей девочкой могут прийти три существа других,.. Из других планов. Вы их учитывали? Вы, присутствие, их ощущали как-то?}
;	(Маша) Ну, вообще-то, знаете… мне всё равно сколько вас тут сидит. Куча, там, или один. Понимаете?  Вы чё? Опять забыли, кто я?
;	(Ольга) Не забыли.
;	(Гера) Маска, Маска.
;	(Маша) Во! А что разве Маска вам сказала, что вас много или мало? Я только послушала, что вы тут жалуетесь и всё. 
;	(Белимов) А, подслушивала.
;	(Ольга) [смеется]
;	(Белимов) Маска, ну, вот мы хотели бы у тебя спросить: среди многих чувств, которые у человека есть: зрение там, слух. А третий глаз? Вы что-нибудь знаете о нём? Какое это качество, хорошее или плохое? На сколько человечеству он надо владеть этим?
;	(Маска) Ну, давайте рассмотрим, попробуем, отдельно. 
;	(Белимов) Так.
;	(Маска) Вот, отдельно - зрение. Вот, что это такое? Можете объяснить, что такое, отдельно, зрение?
;	(Гера) Отдельно?
;	(Белимов и Ольга) Отдельно.
;	(Маска) Ну, просто зрение…одно из чувств. Так?
;	(Ольга) Да.
;	(Гера) Угу.
;	(Маска) Ну, объясните, пожалуйста, что это такое? 
;	(Ольга) Зрение иных миров что ли?
;	(Маска) [тяжело вздыхает]
;	(Белимов) Давайте не загадками, давайте -  что такое третий глаз? 
;	(Маска) Нет, а мы не загадками. Это вы нам загадками говорите. 
;	(Ольга) Ну, пространство зрения.
;	(Маска) Пространство зрения. А что такое зрение? Вы мне можете объяснить? Вы мне объясните, что…
;	(Ольга) Ну, это…глаза.[смеется]
;	(Маска) Глаза![удивленно] А третий глаз - тоже глаз. 
;	(Гера) Нет. Тихо, тихо, тихо! ``Базар'' развели! [было шумно. прим.]
;	(Белимов)  Вибрация сигналов и передача в мозг. Адекватная картина. Ну, по крайней мере, так можно объяснить. 
;	(Гера) Может - осознавание того, что есть, - по другому.
;	(Белимов) Осознание, что видишь…
;	(Маска) О, нет, подождите. Вы говорите – глаза, это зрение. Третий глаз -  это тоже глаз, это тоже - зрение. 
;	(Ольга) Ну, да.
;	(Гера) Знание, может?
;	(Маска) Как мне вам объяснить, о третьем глазе, если вы не знаете о своих ещё чувствах, тех которыми обладаете и пользуетесь всю жизнь? Ну, что такое слух? Колебание воздуха, да? Причем от 16герц  до 20 000. так?
;	(Белимов) Да, отлично.
;	(Гера) Некоторые до 25 000.
;	(Маска) Хорошо. Некоторые… Но вы к этому не относитесь, не переживайте. Итак…
\people{(Гера) Бог миловал.}
;	(Маска) Ой! Бог миловал? Да нет, бог, наоборот, вас обидел! Вы представляете, что вы бы могли слышать, если бы могли до 30 килогерц?
;	(Гера) Ну, это же элементарно можно прибором сделать, в принципе, сейчас.
;	(Маша) Да? И чего этим прибором-то услышите-то? Но это всё равно, простите, если вместо руки вам поставить культяпку, какой-нибудь механизм. Ну и что?
;	(Гера)Что-то услышим.
;	(Маска) Да?
;	(Гера) А интересно, дети, что слышат, вот, на 25 000? Собаки на 30 000?
;	(Маска) Какое интересно сравнение, заметьте! Дети и собаки! Ну, хорошо, давайте упустим… Итак, слух это — колебание воздуха. А колебание повыше, что это? Это свет!
;	(Гера) Да.
;	(Маска) И уже для этого нужен более сложный аппарат. Правильно?
;	(Ольга) Правильно.
;	(Гера) Глаза, это тоже уши.
;	(Маска) Хорошо, а что такое кожное, ваше любимое кожное? Что такое? 
;	(Гера) Нервная система.
;	(Маска) Это - то же самое колебание! Вы согласны?
;	(Ольга) Да.
;	(Маска) Но для этого нужны менее грубые. Правильно? Хорошо, аналог с техникой? А то сейчас скажете, что с кожей тоже очень сложно. Весы — тактильное ощущение,-  одно и то же, почти. Для того чтобы вам уже обладать слухом вы должны собрать уже довольно сложный аппарат, правильно? Так? А для того чтобы ещё и увидеть изображение - это ещё сложнее аппарат. Ну, а Третий Глаз — у нас должен аппарат ещё быть сложнее. Итак, для того чтобы слушать вам было достаточно всего лишь три уха. Так? 
;	(Гера) Да.
;	(Маска) Вспомните анатомию.
;	(Гера) Ну, да.
;	(Маша) Для того чтобы видеть, вам нужно иметь миллионы колбочек! Правильно? А для того… и развитая нервная система прямой, если хотите сказать, ввод в сам мозг, в зрительный нерв. Правильно?
;	(Белимов) Всё верно, да.
;	(Маска) То здесь уже вы получаете больше информации, а значит, и нужна более информационный переноситель/носитель правильно?
;	(Ольга) Да.
;	(Гера)  Правильно.
;	(Маска) А третий глаз? 
;	(Белимов) Очень мало о нём знаем.
;	(Маска) Очень мало знаем…[обиженно]
;	(Ольга) Это - ещё более…
;	(Белимов) Подключается мозг, и резервы мозга, нейроны, дополнительные. Мы же используем на пять процентов мозг.
;	(Маска)[глубоко вздыхает] Ой какие мы невнимательные! Мы говорим о сложности, так давайте про сложности и продолжим. Вот, третий глаз должен быть ещё сложнее? Правильно?
;	(Белимов) Конечно. Согласен.
;	(Маска) Значит он… А Вы его рисуете всё время на лбу, а заметьте - уши внизу, подальше от мозга, глаза - уже прям на самом, понимаешь ли, лбу,
;	(Гера)  Угу.
;	(Маша) …а третий глаз должен быть уже…?
;	(Гера)  В мозге.
;	(Маска) В мозге! Пра…[обрыв контакта]
          Конец контакта.
;	(Белимов) Что ты помнишь? Что, был где? Ну… Не, ну вспомни. Вспомни, какие…
;	(Гера) Может и не сразу вспомнишь.
;	(Ольга) Ты как себя чувствуешь?
;	(Белимов) …картинки или что, да.
;	(Гера) Может быть - покурить? 
[обрыв ленты]
;	(Белимов) Совершенно не помнишь или (…)
;	(Переводчик) Помню.
;	(Белимов) А что помнишь? Ну, давай, что помнишь-то?
;	(Переводчик) Про вас вспомнил.
;	(Белимов) Ну, вот включай… и что ты помнишь?
;	(Переводчик) Вопросы ваши помню.
;	(Белимов) Помнишь?
;	(Переводчик) Да.
;	(Белимов) Ну, Гена, вот можешь ты в таком сейчас состоянии… мне сейчас действительно, это должен ответить. Даже ``Москва, так сказать, запрашивает'' уже. Что предполагается? - А мы ничего не можем! То – может долетит, то - может нет.
;	(Ольга) Ну, ведь сказали сразу - не долетят.
;	(Белимов) Не долетят. Так это я уже понял. Но все равно это, я не понял, вот, смотрите, один даже не стартанёт, а второй, вроде что-то будет, тоже, что с первым спутником.
;	(Ольга) Да. Ну, с ``Proserpina''.
;	(Лена) С третьим.
;	(Ольга) С первым, ну с ``Proserpina'' наверное. То есть, он улетит куда-нибудь в другое…
;	(Белимов) А! Как с ``Proserpina''!
;	(Гера) ``Pioneer 10'' до этого…
;	(Белимов) А ты сейчас как? Это… давай твое видение. Что будет с Марсом, с полетами на Марс? И что? Почему они не дают? Они…
;	(Гера) И что ты там видел? Что тебе там показали?
;	(Переводчик) Я просто вопрос вспомнил, - Марс мне ``до лампочки'' вообще, честно говоря.
;	(Белимов) А-а… Вполне логично.(…). Но у тебя внутри уверенность, что долетят с (…)
;	(Переводчик) Да я вообще не задумывался. Я, честно говоря, вот на собрании раз услышал об этом и всё.
;	(Ольга) И я тоже. Да. Что запускают. Просто вы интересуетесь, наверное, и… 
;	(Белимов) Ну, да, мы публикацию сделали такую… довольно шумную - ``смените тактику полета на Марс'' а как сменить, мы ответа не дали.
;	(Ольга) Так. А подождите, они говорили о 70 часах… каких-то. 
;	(Белимов) Что-то они там про 70 часов…
;	(Ольга) Про часы, а вы…  ну, то, что про какие-то 70 часов.
;	(Белимов) А вот эта дыра..
;	(Гера) А! Кассету потеряли. Ну…
;	(Ольга) Нет! Это уже потом, Это потом Геннадий Степанович спросил что-то про кассеты. Почему-то он перевел на кассеты, ну, а Они говорят — ``Ну и это тоже''.
;	(Белимов) Ну, да, что мол…А! Ну, там был вопрос такой, что… Ты уже не записывай, это уже не надо. Там был вопрос такой, что я говорю — Ну мы никому не скажем.  А они, вроде, сказали что…
;	(Ольга) А! - Уже потеряно 70 часов. А! Всё! Да!
;	(Белимов) А!
;	(Лена) Потеряно.
;	(Ольга) Уже потеряно 70 часов.
;	(Белимов) А! Ребята, да…. Тут что, вот ещё вот только произошла катастрофа, мне были определенные звонки…
(Обрывается контакт меняется кассета)
;	(Девочка) Взял вот так это…
;	(Белимов) Он взял в руки Гена белого кролика?
;	(Девочка) Да кролика.
;	(Белимов) Белого кролика взял?
;	(Девочка) Да, в руки.
;	(Белимов) А он что похож на кролика большой, маленький.
;	(Девочка) Ну, он…
;	(Белимов) Он взял его так и что?
;	(Ольга) Он родился в год кота или кролика да, это одно и тоже.
;	(Девушка) И он как-то кролик у него исчез. Кто-то, как будто бы забрал. А тут Маска на..[показывает]
;	(Белимов)  Типа очков. Она шептала мне. Говорит у него очки какие-то на глазах…
;	(Девушка) Как, вот, маска.
;	(Белимов) Не очки, а как полиэтиленовые какие-то.
;	(Девушка) Да.
;	(Гера) И после этого маску?
;	(Белимов) А потом, глаза тоже сняли? Сняли маску они?
;	(Девушка) Нет.
;	(Белимов) А потом, ты что, перестала видеть, что с ним происходит? Были какие-то еще моменты?
;	(Девушка) Ну, нет. Я видела, какие-то люди прорывались к нему. Вот он лежал, а вот тут вот такая линия,  она до сих пор. По-моему, как-то ну…как, вот, золото.
;	(Белимов) Так…
;	(Ольга) Ну.
;	(Девочка) И вот на нём люди, а Они как-будто прорываются к нему.
;	(Белимов) Прорывались да?
;	(Ольга) Ну, да, все же хотят.
;	(Белимов) Ну… то есть, не высоко, не под потолком, а именно в центре?
;	(Ольга) А кто-нибудь проходил из них?
;	(Белимов) А кто-нибудь проходил, да? И тогда менялся его разговор. Тот Мабуу? Этот, да?
;	(Гера) Нет, он срывался не в те моменты.
;	(Белимов) А когда он срывался, почему он срывался, как раз не заметила, да? Так…

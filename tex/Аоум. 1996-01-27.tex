Аоум. глава 32-я 27-01-1996г
Георгий Губин
\people{**}
 (Запись не сначала)
 (Сергей) … Ну, она же из купцов.
 (Ольга)   А-а!
 (Сергей) Она -купеческая дочь.  Поэтому, конечно,  я понимаю, что её оставили голой, и она теперь называет нас ``быдлами''.
 (Ольга)   А-а. Ну, это её проблемы.
 (Сергей) Ну, не знаю. Ну, я хожу сейчас в кружок ленениады и.. .я не нашёл ничего плохого.
 (Ольга)   Ты Маркса не изучаешь ещё, ты, просто, Ленина?
 (Сергей) Нет, нам его читают. Потом разъясняют. 
 (Гера)   Ну, тебе нравится Маркс?
 (Сергей) Вы знаете, я его тяжело понимаю.
 (Ольга)  У тебя такое  детство было… не очень, будем говорить так. 
 (Гера)    Не барское, не очень, как это сказать….
 (Сергей) А причём здесь детство?
 (Гера)   Отпечаток накладывает на мировозрение…
 (Ольга)  Дело в том, если бы ты получил воспитание… у тебя бы родители… если бы у тебя… ты в школу рано стал ходить тебе легче бы было всё-таки воспринимать всё это. А тебе приходилось бороться за жизнь.
 (Сергей) Подождите! Он кажется, побежал!
 (Ольга)  Кто? Андрей? Андрей?
 (Гера)   Пусть побегает!
 (Ольга)  А-а, за скорой помощью, наверное, побежал!
 (Гера)   Может, ему ``приспичило''?
 (Ольга)  Ты, по-моему, заговорился, смотри.
 (Сергей) Можно ещё вас спросить?
 (Гера)   Да, пожалуйста. Сколько угодно. 
 (Сергей) Мне во сне постоянно снится,… ну, как бы вам сказать…
 (Ольга)  Ну, говори, говори,  не стесняйся.
 (Гера)   По-любому - говори.
 (Сергей) Только не смейтесь. Серьёзно. Мне снится кукла.
 (Гера)   Кукла?
 (Ольга)  Ну. Ну-ну.
 (Гера)   И она говорит.
 (Сергей) Да.
 (Гера)   Механическая.
 (Сергей) Не знаю.
 (Гера)   Ну, кукла, естественно она неживая.
 (Ольга)  Она разговаривает.
 (Лена)   Тихо! Вы даже слова не даёте сказать.
 (Сергей) Нет, она не живая.
 (Гера)   Так-так, ну и что? Какие факты? Малейшие нюансы, тут очень важно это, кстати.
 (Сергей) Ну, разговаривает со мной. Я ей задаю какие-то вопросы, и она отвечает.
 (Гера)   Какие? Не помнишь?
 (Сергей) Какие?
 (Гера)   На какую тему?
 (Сергей) Не помню.
 (Гера)   Ну, ладно. Не важно. Что ещё с этой куклой связано? Вокруг кто-нибудь есть ещё?
 (Ольга)  А кто эту куклу сделал?
 (Сергей) Понимаете, я просто…как бы вам объяснить… У меня мать – это больной вопрос. Я помню отца плохо, но всё-таки помню. А мать я не помню вообще. И я почему-то иногда думаю, что эта кукла, ну, что…  отожествляю её с матерью. 
 (Гера)   Ну, это понятно, в принципе.
 (Сергей) И вы говорите, что  ничего не знаете…
 (Ольга)  О кукле?
 (Гера)   О кукле знаем немножко. А о матери твоей нет. К сожалению, нет.
 (Сергей) Что знаете о кукле?
  
 (Генра )Кукла…Вот мы сами там разбираемся. Со временем что-то у нас… Мы думали, ты в то время сделала её. Потому что, насколько мы знаем, ещё ранее было у нас четыре или три друга, что ли… В прошлом. Уже и в твоём и в нашем прошлом, так сказать. И кто-то один сделал эту куклу, и продали какому-то графу. Где-то во дворе жили, мы толком не знаем. Ну, короче, это было изобретение твоё в прошлом. У тебя, видать, голова варила здорово.
 (Сергей) В прошлом?
 (Гера)   В твоём прошлом.
 (Сергей) В каких годах?
 (Гера)   Не знаем.
 (Ольга)  Скорее всего это XIX век.
 (Гера)   Тебе ничего не говорит 16-ый год?
 (Сергей) 16-й год…
 (Гера)   Что ты делал в 16-том году?
 (Сергей) 16-ый год… в июле меня…Нет, в августе… Где-то в начале августа… я пришёл в детдом.
 (Гера)   Пришёл в детдом?
 (Сергей) Да. Я был при церкви…
 (Гера)   А в 14-ом - ничё? Не случалось так? 
 (Сергей) 14-ый…
 (Гера)   Никакого убийства не случалось?
 (Сергей) В 14-ом  я болел.
 (Ольга)  Сергей, а не ты своровал часы в учителя и решил там посмотреть их устройство их? Не ты?
 (Сергей) А вы откуда знаете? 
 (Ольга)  Ты нам сам говорил! (Смеется)
 (Сергей) Я говорил?
 (Ольга)  Да.
 (Гера)   Ну, ты мог не знать.
 (Сергей) Как я мог вам говорить?
 (Ольга)  Ой, ты много чего не помнишь! Вот ты сейчас с нами говоришь.. ты запомнил…а ты раньше говорил нам. Ну, видно не запомнил, что ты нам говорил, или у тебя память была закрыта.
 (Сергей) Да, но я не могу теперь восстановить их…
 (Ольга)  Нет.
 (Сергей) Я их просто не соберу.
 (Ольга)  Тебя мальчишки в детдоме били?
 (Сергей) В детдоме,  наверное,  всех бьют. 
 (Гера)   Да.
 (Ольга)  Да вообще-то мальчишки, наверное, девчонки редко дерутся. Ну, у нас родители, понимаешь… и у тебя родители будут…и отец и мать.
 (Гера)   В будущем. То есть, в нашем ``сейчаcешнем'' настоящем.
 (Ольга)  Да. И у тебя жена и дети будут. И всё у тебя будет  хорошо. И ты очень умный парень.
 (Гера)   Да, тут ты очень умный парень. Тут ты изобретаешь такие вещи, которые…два института тут  разбирают…
 (Ольга)  Ну, мы тешим твое самолюбие.
 (Гера)   Ты ``голова'', парень.
 (Ольга)  Андрей там не …нигде?
 (Смех)
 (Гера)   Где он там?
 (Ольга)  Не бегает там, рядом, Андрей?
 (Сергей) Нет. Не знаю. Я сейчас посмотрю.
 (Ольга)  Посмотри.
 (Гера)   А, чё ты о нём можешь сказать? Андрей – это я в будущем. Вот я, конкретно.
 (Сергей) Ну, я вообще-то с ним только  второй год знаком. В 20-м мы с ним познакомились. Когда меня принимали на ``Раб-фак'', и он первый, с кем я познакомился. Ну, вообще-то мы с ним, ну, как бы вам сказать…
 (Ольга)  ``Не очень'', да?
 (Сергей)  Не очень. Понимаете, он – сноб. 
 (Ольга)   А-а. А он не из интеллигентной семьи?
 (Сергей)  Нет.
 (Гера)    Тоже детдомовец?
 (Сергей)  Нет, у него есть отец и мать. Отец работает здесь же, в моем цеху. Он мастер. А мать – она  какая-то учетчица, тоже где-то у нас, но я её ни разу не видел. 
 (Ольга)   А скажи, внешность, опиши внешность его? Может быть они похожи сейчас?
 (Гера)    И отчество скажи его.
 (Сергей)  Викторович.
 (Ольга)   А фамилия?
 (Сергей)  Мацута, по-моему.
 (Ольга)   Мацута? 
 (Сергей)  Да, как-то так.
 (Ольга)   Ну, а внешность у него какая? Рост какой? Ну, вообще, опиши его внешность тогда.
 (Сергей)  Ну, выше меня…
 (Ольга)   Угу. Волосы какие у него?
 (Сергей)  Где-то 1,82, наверное.
 (Ольга)   А волосы, волосы кудрявые?
 (Сергей)  Волосы? Да нет.
 (Ольга)   Прямые?
 (Сергей)  Да, знаете, - ``ёжик''.
 (Ольга)   А цвет - светлый?
 (Сергей)  Рыжий. И у него уже плешина позади.
 (Гера)    А сколько ему лет-то?
 (Ольга)   Постарше он тебя?
 (Сергей)  Ой, сейчас скажу… Знаете, по-моему… по-моему, он 98-й или 97-й .
 (Гера)    Рождения.
 (Ольга)   Ну, наверное, постарше, немного.
 (Сергей)  Да, он старше.
 (Ольга)   Он - сухощавый?
 (Сергей)  Да.
 (Ольга)   И у него такие резкие, быстрые движения?
 (Сергей)  Да нет, не сказал бы.
 (Лена)    Нет, он наоборот, спокойный, медлительный, да?
 (Сергей)  Да он никогда не был спокойный, если затронуть что-нибудь.
 (Ольга)   В общем, он такой…
 (Лена)    Эмоциональный? 
 (Сергей)  Он начинает прыгать, как обезьяна. Доказывает всё своё. Поэтому-то и с ним нечего спорить, потому что он может в горячке сделать что угодно.
 (Лена)    Ого!
 (Ольга)   А ты знаешь, что он и сейчас такой? Почти что.
 (Сергей)  Ну, это ничего хорошего.
 (Ольга)   Ну, это ясно. А ты о себе расскажи.
  (Сергей) Что я могу о себе рассказать?
  (Ольга)  Ну, какой у тебя рост примерно, внешность приблизительно. Может, мы сравним…
 (Сергей)  Ну, последний раз я взвешивался – у меня было 56.
 (Ольга)   Худенький парень.
 (Сергей)  Метр семьдесят два.
 (Ольга)   А глаза, какие у тебя?
 (Сергей)  Голубые.
 (Ольга)   Ясно. И сейчас у тебя тоже голубые.
 (Лена)    А волосы?
 (Сергей)  Не знаю. Волосы у меня белые сейчас.
 (Ольга)   А, да-да, седые у тебя. Ну, вообще то, а характер свой как ты охарактеризуешь? Ну-ка,  давай! Теперь о тебе говорить. О  себе трудно разговаривать, да?
 (Сергей)  Ну…
 (Лена)    Ну, вот, как ты? Тоже спокойный или, может, туда поближе…
 (Гера)    К беспокойным?
 (Лена)    Как Андрей?
 (Сергей)  Да мне, как Андрей, не хотелось быть, конечно. Ну, не знаю, как сказать… спокойный… Я просто привык. 
 (Лена)    Тебя ничего уже так сильно не удивляет?
 (Гера)    Инфантильный, да? 
 (Сергей)  Не знаю, что это за слово.
 (Ольга)   Скажи, ты, наверное, терпеливый. Вот я так чувствую, что ты терпеливый парень. У тебя столько в жизни было… Ты привык, да?
 (Сергей)  Ну, вы знаете… как Степаныч сказал, хорошо, что я очень многого не помню. Иначе б я сошел бы уже с ума. 
 (Ольга)   Какой Степаныч?
 (Сергей)  Это мой доктор.
 (Ольга)   Ты у него лечишься, что ли?
 (Сергей)  Нет,  я у него  лечился. Просто мы с ним поддерживаем связи. Это он же мне помог устроиться в детдом.
 (Ольга)   Хороший он, наверное?
 (Гера)    Ну, ты сейчас не болеешь?  Всё нормально?
 (Сергей)  Да нет, я не болею.
 (Ольга)   Ясно. Ну, в общем, мы о тебе можем сказать, что ты парень нормальный, очень умный. Умница. Интеллект у тебя высокий…
 (Сергей)  Ну, как вы можете сказать, если вы не знаете ничего?
 (Ольга)   Ну, ``привет''! Как же мы не знаем!? Мы знаем тебя тут…
 (Сергей)  А-а… Вы говорите о будущем?
 (Ольга)   А ты думаешь  изменился очень сильно?
 (Сергей)  Подождите, а вот вопрос: мне  Петровна говорит, что она в прошлой жизни была мужчиной. Это может быть?
 (Ольга)   Может!
 (Гера)    Может, запросто!
 (Ольга)   Может пол меняться.  А внутренняя суть остается всё той же. Это называется ``эволюция''. Она тебе говорила?
 (Сергей)  Да, говорила. 
 (Гера)    Есть цивилизация, есть эволюция. Это разные вещи.
 (Сергей)  В цивилизацию она не верит. Она, наоборот, сказала, что со смертью царя мы перестали быть цивилизованными людьми. 
 (Ольга)   Да нет!
 (Сергей)  Мы стали теперь ``дикари'' и ``быдло''.
 (Ольга)   Ну, ты знаешь, вообще-то многие такие - богатые, такие люди… и будем говорить, не интеллигентные внутренне, они всегда называли нас, народ,  быдлом. Русский народ. Поэтому и сейчас такие есть. И нас называют точно так же, ты не волнуйся, и здесь точно такие же есть, те, которых, знаешь, ну, называют ``из грязи в князи'', и  вот они теперь всех остальных так же, как  и Петровна называла. Вас называет.
 (Сергей)  А она не нас называет…
 (Ольга)   А не важно. А кого она подразумевает? А вы - не такие? Вы - не такие, как все?
 (Сергей)  Пока здесь нет Андрея…  В общем - существующую власть,. Она говорит, что эта власть долго не продержится. Я не любитель спорить, конечно, но знаете… теперь, наверное, она права.
 (Ольга)   Ну, 70 лет – это для страны, конечно, не так много…
 (Сергей)  Семьдесят?!
 (Гера)    72-71 вот так.
 (Ольга)   Ну, советская власть…
 
(обрыв)
 (Ольга)  …у нас строй.
 (Гера)   Рыночная экономика. 
 (Ольга)  Да. Вот называется ``рыночная экономика''.  И, в общем-то,  наверное, вообще-то состояние наше сейчас такое же, как у вас.(1996г. прим.)  Ну, дело в том, что вам, наверное, было тяжелее, чем нам…
 (Гера)   У нас хоть производство хоть какое-то есть там, ну…
 (Ольга)  У них тоже есть. У нас войны, просто, не было сейчас вот…
 (Сергей) У вас идёт сейчас война?
 (Гера)   Нет. Ближайшие 50 лет у нас войны не было. 
 (Сергей) У нас-то - идёт.
 (Ольга)  Гражданская? Ну, у нас в принципе, тоже идёт… война.
 (Сергей) А когда она закончится?
 (Ольга)  Гражданская война?
 (Гера)   Да, по-моему,  в 22-ом и закончится,.. Да. По истории мы учили - 22-ой год.
 (Ольга)  Да, вот 22-ой – 23-ий… Да, она скоро закончится уже.
   
 (Сергей) А-а… Меня успокаивает, что вы тоже не очень хорошо знаете историю. Не один я такой.
 (Гера)   Нет, я просто что-то этот период ``не очень''. Я более глубокие ``копал''.
 (Сергей) Более глубокие? Мне сейчас приходится писать о Петре I. Могли бы мне подкинуть что-нибудь такое, что я мог бы не знать?
 (Гера)   Запросто!
 (Ольга)  Не-е… ты хитрый парень! 
 (Гера)   Ты в курсе, что Петр I завез табак  в Россию?  Перенял у немцев.
 (Сергей) Конечно.
 (Гера)   Ну, ничего нового тогда нет. 
 (Сергей) Никто не знает, как он умер. 
 (Ольга)  И у нас тоже тайна покрытая мраком, нам не говорят. 
 (Гера)   Я знаю одно, не знаю, знаешь ли ты, вот чё новенькое, недавно слушал радио,  и там передавали про Петра I…
 (Сергей) У вас есть радио?
 (Гера)   О, у нас и телевизор есть!
 (Сергей) Вы богатые?
 (Ольга)  Мы? Ну, по сравнению с вами – богатые. Понимаешь, дело в том, что всё идёт же…люди…
 (Гера)   Так… Дайте я расскажу о Петре. В общем, можешь смело заявлять, такой факт. В общем, когда Петр женился на Катерине, она была дочерью отставного солдата, шведа, по-моему.
 (Сергей) Ну, и?
 (Гера)   Вот. И, значит, в течение нескольких месяцев или даже года, её официально не называли ни королевой, никак.  Он старался умалчать тот факт, что он женатый.
 (Сергей) Ну, это мы знаем.
 (Гера)   А-а, знаете? Ну, тогда ``извини''.
 (Ольга)  Сергей, давай тогда чё-нибудь… Задавай нам вопросы, или мы тебя что-нибудь будем спрашивать.
 (Гера)   Мы на патефон записываем твой голос, что ты нам отвечаешь, что мы вам говорим.  У вас есть записывающие патефоны?
 (Ольга) Нет, они просто слушают. Ты слышал патефон, слышал? Пластинки слушал?
 (Гера) 1…
 ( сбой)
\soul{На какой букве остановились мы?}
\people{На букве ``щ''.}
\people{Подождите, мы никогда не спрашивали кто вы. Конечно, может у нас память короткая. У нас алфавит, мы разговариваем с кем-то и не знаем, кто нам алфавит. (даёт. Прим.) Вы можете назваться, или как-то там представится?}
\people{Может, мы знаем вас? Кто вы? Предохранители?}
\soul{У нас много имён. }
\people{Вы назывались?}
\soul{Другие.}
\people{А-а… другие. }
\people{Цвета?}
\soul{Если быть точнее, то мы, как бы - ваши желания. Желания, но, как бы вам объяснить это поближе…не просто желания, а вот желание…}
\people{Что-то знать? Интерес? Нет?}
\soul{Желание  желания.}
\people{А-а. Желание  желания. Так…  Мы, приблизительно, уловили.}
\people{Хотение?}
\soul{О, нет! Это совершенно разные вещи. Вы, прежде, чем начать мечтать, должны увидеть подобное. И лишь  только к этому подобному вы добавляете свои желания,  и тогда это возвращается мечтой. Вы должны увидеть идеал. Мы же только помогаем вам создать этот идеал, только и всего. Почему мы сейчас говорим об алфавите вашем? Потому, что вы желали того.  У вас были какие-то проблемы. Его нет среди нас сейчас. }
\people{Почему?}
\people{Кого?}
\soul{Это вы должны знать,  почему он не пришёл. Он желал – мы отвечали. И, если быть точнее, то каждый из вас, особенно вы, говорящая.}
\people{Я?}
\soul{Да, вы. Вас всегда интересует - гороскопы.}
\people{Да, да. Мне хочется просто докопаться. Всё-таки, это связано и завязано.}
\soul{Это всё связано. Безусловно! Но вы никогда не сможете развязать, если вы будете смотреть лишь только именно это, если только один этот узелочек развязывать,  всё  остальное у вас не получится.}
\people{Я не считаю, что гороскопы – это единственное, что может объяснить человека.}
\soul{Понимаете, узелочек – это ваше любопытство, гороскопы и числа – это всего лишь один узелочек, который находится далеко в середине. А внешние узелки, начальные и конечные, вы даже не затрагиваете. Мы говорим вам о буквах. Но предупреждаем, что буквы имеют множество смыслов. Мы уже пытались объяснить вам, что нельзя на одушевленное /неодушевленное – это совершенно разные вещи. И, хотя многое, в вашем понятии неодушевленное – это, всё-таки, имеющее душу, но примерно, примерно будете правы если будете хотя бы по этой, по этой держаться рамке. Дальше. Мы говорим относительно человека растущего,  начиная с младенческого возраста и до старости. Если же мы начнем говорить о его других характеристиках, то те же сами буквы уже будут звучать и расшифровываться по- другому. }
\people{Так. Вопрос: Что вы называете ``человеком'' и ``другими характеристиками''?}
\soul{А вы попробуйте … ``Человек'' – мы дали уже вас все буквы.}
\people{Угу. Ну, практически.}
\soul{Попробуйте и найдите. Нам говорили, чтобы мы старались говорить вам, как можно меньше  на ваши слова и лишь только давали начало, основу. Нас предупредили сразу, и предупредили, что мы будем наказаны. Наказаны, как - мы ещё не знаем, да и не хотим, честно говоря, об этом знать. }
\people{Нет, ну они правы, не надо нам… Может быть, мы как-то обленимся совсем.}
\soul{Вы как-то спрашивали у них, как они выглядят. Вы знаете, нам стало это тоже очень интересно. И хотя мы не можем обладать всеми чувствами, что обладаете вы, нам пришлось разбудить в вас желание. Мы же и есть - желание желания? И мы решили, немного нечестно по отношению  к вам, но мы всё-таки разбудили желание, чтобы вы узнали, кто они,  да по-подробней. Но они оказались сильнее нас, они не стали отвечать на эти вопросы. Тогда мы решили сделать это сами. Мы стали призывать, как они вам потом представились, ``носителей цветов''. Да, они как-то говорили уже с вами. Мы пришли к ним, но оно отказались. Они отказались нам отвечать и даже не стали с нами связываться. }
 (Ольга) Почему?
\soul{Почему? Потому, что мы сразу же меняем их цвета. }
\people{А-а. Ну, это же желания, да… А с теми, с кем мы раньше разговаривали, ну, это наши друзья старые, будем так говорить…}
\people{Первые.}
\people{…вы с ними контактируете, каким-то образом?}
\soul{Вы знаете, контактируете только вы. Мы - все вместе. Мы же в одном человеке, как мы можем? Единственное что,- это когда вы ``давите'' одних и ``возвышаете'' других. }
\people{Ну, да. Мы это уже поняли это.}
\soul{Вы знаете, есть у вас множество имён, но больше всего было бы ближе ``эго''. Эго. Есть такие. Можно было бы конечно назвать их, чтобы вам польстить, можно назвать их такими черненькими,  лохматенькими  чертёнками. Это метафора.  Эго, это, знаете… что-то такое с острыми углами…}
\people{Колючее наверно, да?}
\soul{Нельзя это назвать колючим. Это что-то такое, знаете, бесформенное, и в то же время, имеет множество углов. Углы эти довольно-таки острые и они  постоянно вас ранят. И каждое ребро отвечает за одно из проявлений эгоизма. Ведь проявления  эгоизма тоже разные. Когда есть в вас жадность, жадность к деньгам – это одно ребро, а когда вы ревнуете – это другое ребро. А всё равно, это тот же самый эгоизм. И их множество. И знаете, порой их становится больше/меньше… Ни разу мы не видели, что бы их не было. }
\people{Короче, на земле, нет,  наверно…  ну, среди людей, вернее,  нет  не эгоистичных…}
\soul{Это не только среди людей. Это есть у всех. У растений  и у камня. Каждый камень желает стать во главе храма. Каждый. Каждое растение желает, чтобы его поливали, и чтобы оно было единственным. Та же самая роза. Это же - ваше. }
\people{Да.}
\people{Капризная.}
\soul{Та же самая капризная роза – она хочет быть единственной и самой красивой. И конечно, мальчик никогда не скажет, что он видел множество других роз, для него она будет одна. Вы знаете, очень мало граней у ребенка. И хотя его поведение очень эгоистичное, очень, любой ребёнок это - практически ``голый эгоизм''. Вы знаете, у него нет этих граней. Этот эгоизм у него на основе инстинкта что ли… Незнание, что  можно это или нет. Если ему нравится какая-то вещь, он просто хочет её, но не больше. Он не выдумывает ничего, чтобы как-то ею завладеть. Он просто будет брать это силой, откровенностью. И  у него нет, поэтому, этих ран,  ран которые наносят эти ребра. А в чём разница, когда ребёнок становится взрослым и хочет уже ту же самую вещь, допустим?  Он уже начинает хитрить, он уже начинает обманывать. Он уже понимает, что другому - будет больно. Но он всё равно на это идёт. Понимаете разницу? Ребёнок не знает, что он кому-то делает больно. И если он и делает кому-то больно, его будет только глодать любопытство: ``А что это он орёт?''  и будет заглядывать ему в рот. Понимаете,  но это не эгоизм, это, просто, интерес. Один из способов познавания мира, только и всего. Так что зло – злу - рознь …}
\people{Можно и так сказать сделать,  чтобы человек сам отдал ту вещь, которую он хочет.}
\soul{Вот это и есть проявление этой грани, когда вы вынуждаете человека это сделать. Когда вы требуете – он не отдаёт, вы не хотите взять это силой, потому что боитесь или ещё что-то, какие-то устои не дают вам, вы тогда пытаетесь у него купить. Но простите, это то же самое - вынудить его отдать. Какая разница? Какая разница?  Поменяете или купите, вы заставляете его войти в своё положение. Вы предлагаете ему какую-то, обмен на эту вещь, сделку. Для чего? Для того чтобы удовлетворить себя. И, в то же время, чтобы много не потерять. Слишком много-то хочется тоже не отдавать. Вот тогда эти грани появляются. Не раньше.}
\people{Нет, ну, он может отдать это, как сказать… ну, рад будет, что отдал. }
\soul{Бывает. Бывает и такое. Бывает вы до того надоедите, что он рад вам отдать, лишь бы отстали. И это - тот же самый способ. Тот же самый – вынудить. А вы говорите ``не то''! Как часто вы встречали это ``не то''? Как часто? }
\people{Ну, редко, но было. }
\soul{Было ли? Было ли? Поймите, когда вы делаете доброе дело, где-то внутри, вы успеваете себя уже похвалить! А это уже не доброе дело. Это, уже просто попытка - показаться добрым. Вы подаете милостыню, и тут же: ``Какой я хороший!''.  Или приходите к другу и: ``Знаете, я сегодня отдал просящему, и мне стало легче!'' Этим вы уже хвастаетесь. Хвастаетесь. Вами же было сказано, что ``не должна знать левая, что делает правая''(библ. Прим.). Вами же было сказано. А вы забываете об этом! Вы всегда любите себя хвалить. Заметьте, с вами легче говорить, когда вас хвалят. Даже если не заслужено. Конечно, вы любите, когда вас ругают. Потому что если вас хвалят столь неумело, то вы уже чувствуете, что вам лгут. Вам приятна эта ложь, но, опять же, до определенной степени, потому что вы тогда можете заметить, что этот человек вас призирает и просто над вами смеется. Ну, давайте уберём эти крайности и попробуем провести диалог.}
\people{Давайте.}
\soul{Нет, вы не поняли, попробуйте провести диалог со своим врагом. Вот, вы идёте  сегодня к человеку, которого не любите или к человеку, которому должны или что-то ещё. Что первым делом делаете вы? Вы совершаете грандиозный монолог ``про себя”…}
\people{Ну, да.}
\soul{…как вы будете себя с ним вести: ``да вот я ему скажу, да вот он паразит'' и так дальше.}
\people{Это точно.}
\soul{Правильно? Правильно. Вы приходите к этому человеку, и ничего этого не происходит.  }
\people{Всё правильно.}
\soul{Почему? Потому что вы соблюдаете ``технику безопасности''.  }
\people{Боимся, да?}
\people{Опасаемся. }
\people{1..}
\soul{Спрашивайте. }
 (Перепад)
 (Ольга)  Подожди, кто это?
 (Мабуу)  Кто-то вошёл, а я… а я… а я его стукнул!
 (Ольга)  Кто?
 (Мабуу)  Думал, что это ``время''  пришло, а это оказался монах.
 (Ольга)  Ага!
 (Мабуу)  А он сказал, он сказал, что я его убил,  и он теперь будет… Вот, кем - я не помню.
 (Ольга)  Подожди, ну-ка, ну-ка ещё раз.  Мы ничего не поняли. 
 (Гера)   А когда ты убил его?
 (Лена)   Что случилось?
 (Мабуу)  Я же говорил вам.
 (Ольга)  Ну…
 (Лена)   Мы не поняли. Объясни.
 (Мабуу)  Я говорил вам, что я хотел подглядеть.
 (Лена)   Куда?
 (Мабу)   Зашёл в пещеру, а меня там заперли.
 (Ольга)  Ну.
 (Гера)   А! Да-да-да!
 (Мабуу)  Вот. И монах сказал, что ``придёт время, и ты будешь такой же, как все''.
 (Гера)   Да.
  
 (Мабуу)  А я думал, что это пришло ``время''. 
 (Гера)   А это монах зашёл, да?  А ты его палкой!
 (Ольга)  Испугался его?
 (Мабуу)  Нет, я не знал, что это монах. Я думал, что это ``время''.
 (Ольга)  И чё ты ему сделал?
 (Мабуу)  Ударил.
 (Гера)   И он упал?
 (Лена)   Чем ударил?
 (Мабуу)  Да.
 (Гера)   А он больше не поднялся?
 (Ольга)  Рукой ударил?
 (Мабуу)  Он сказал, что я его… убил. И я теперь буду… вот где  буду, я не помню.
 (Лена)   Так чем ты его ударил?
 (Гера)   В аду, наверное, да?
 (Мабуу)  Нет. Что я теперь стану… не знаю.
 (Ольга)  Ну, ладно. Сам подумай хорошенько над своим… Так значит, ты ``время'' ударил?
 (Мабуу)  Монаха.
 (Ольга)  Ну, ты хотел…
 (Гера)   Он, потом, встал, пошёл? Он потом живой встал, пошёл? Он живой остался, да?
 (Ольга)  Он встал? Ты его ударил - он встал потом?
 (Мабуу)  Нет. Он спит.
 (Ольга)  Он спит?
 (Гера)   И навсегда, да, будет спать?
 (Мабуу)  Не знаю. Он мне сказал, что я… время придёт, и я буду, как все. А это некрасиво – они  без шкур. Я их, сперва, боялся, а теперь привык. Я ещё к вам плакал, а вы…Вы ничё не сказали…
 (Ольга)  Нет, ты заплакал, мы хотели тебя пожалеть, а ты ушёл от нас…
 (Гера)   Да. Мы тебя потеряли.
 (Ольга)  Спрашивали, а ты не говорил  ничего. 
  
 (Мабуу)  Да. Я теперь не знаю, что делать. Я хочу кушать. 
 (Гера)   А дверь открыта, откуда к тебе монах пришёл?
 (Мабуу)  Закрыта. Захлопнул. 
 (Ольга)  Закрыта дверь теперь? 
 (Мабуу)  Да. А там ниче…
 (Гера)   Дырочка есть там?
 (Мабуу)  Где?
 (Гера)   В двери. Дырочка, какая-нибудь есть?
 (Лена)   Какое-нибудь отверстие круглое.
 (Ольга)  Свет, свет видишь ты из двери?
 (Лена)   Свет через что идёт - есть?
 (Мабуу)  Дырочка есть, а отверстия нету. 
 (Гера)   Посмотри у монаха в кармане, там должен быть… железочка такая длинненькая. 
 (Мабуу)  Где посмотреть?
 (Ольга) (шепотом) Чему ты учишь? Мабуу, Мабуу, ты в дверь постучи. Ты в дверь постучи кулаком. Умеешь стучать?
 (Мабуу) Умею.
 (Ольга) Ты постучи сильно-сильно. Монахи придут и тебя оттуда заберут.
 (Лена)  Позови. 
 (Гера)  Далеко от пещеры монахи?
 (Мабуу) Не заберут.
 (Ольга) Заберут, заберут. 
 (Мабуу) Нет.
 (Лена)  Они не знают, что ты там. 
 (Ольга) Это мы уже знаем, мы знаем…
 (Мабуу) Они знают, что я тут. Они меня заперли. 
 (Лена)  А-а. Значит, они тебя наказывают. 
 (Ольга) Ну, тебя наказывают. 
 (Мабуу) Наказывают…  Хитренькие. Я не хочу так. 
 (Ольга) Ну, а зачем тогда ты так делаешь - подглядываешь? Тебе не разрешают, а ты подглядываешь. 
 (Мабуу) Мне интересно. 
 (Ольга) Ну, конечно, интересно. Ну,  нельзя же так! Ты б у них хоть спросил…
 (Гера)  Мабуу,  а вот монах лежит рядом с тобой, да?  
 (Мабуу) Да.
 (Гера)  Вот, посмотри у него там, должны быть такие кармашки…
 (Шепчут)Ты чё? Совсем? У него лицо закрыто, ему же нельзя смотреть на монаха!
 (Гера)  Ты слышишь меня, да? Посмотри у него, где-то должна быть вот эта палочка, которая вставляется в отверстие, в дверь, в дырочку эту. 
 (Лена шепчет) Да его заберут!
 (Гера)  Когда ты там вставишь эту палочку в эту дырочку,  ты её покрути, попробуй там направо /налево … вокруг.  И она должна открыться. 
 (Мабуу) И чё?
 (Гера)  И ты выйдешь. Пойдешь  себе в пещеру.
 (Мабуу) Не получится. 
 (Гера)  Почему? 
 (Мабуу) Монахи никогда не врут. Если они сказали, что я стану такой же, значит буду. 
 (Ольга) Ну, это не сейчас. Это потом! Ты ещё старейшим будешь. Ты знаешь? У тебя семь жён будет. 
 (Гера)  У тебя уже семь, да?.
 (Мабуу) Чего? 
 (Ольга) Да-а! У тебя будут семь жён, ты будешь старейшиной, огонь всем будешь раздавать. Наказывать будешь своих соседей. 
 (Мабуу) Огонь я уже даю.  
 (Ольга) А-а.
 (Мабуу) Я взял у них камушек. 
 (Ольга) А, это нам рассказывал недавно.
 (Мабуу) Да. А они смеялись. 
 (Гера)  Как они смеялись? Ты ж их не видел! 
 (Мабуу) Я всем дал огонь, а они пришли и сказали: ``Покажи как?''
 (Ольга) Ага. Монахи, да?
 (Мабуу) Да. Я стал бить, по пальчику ударил. А они стали смеяться и сказали: ``Теперь ты будешь давать огонь''. А старшина обиделся и теперь он со мной не разговаривает.
 (Ольга)Не дружит. Да?
 (Гера) Ну, и Бог с ним.
 (Мабуу) Да… Еесли бы только я давал огонь, я бы ему не дал. А у него свой есть. Я теперь не знаю, как с ним обидеться. 
 (Ольга) Не надо обижаться. 
 (Гера)  В общем, если ты хочешь выйти из этой пещеры - поищи там, у монаха, железочку такую…длинненькую, палочку длинную. В общем, всё, что есть вытряхни там с одежды его.
 (Мабуу)  Не могу.
 (Гера)   Почему? Он же лежит, не трогает тебя.
 (Мабуу)  Нельзя монаха трогать. 
 (Ольга)  Правильно.
 (Гера)   Ты не монаха трогай. Ты трогай одежду его, ``шкуру''. 
 (Мабуу)  Дурак.
 (Лена)   Правильно.
 (Ольга)  Во-во! Правильно, правильно. Не слушай его. Мабуу… Мабу, придут монахи…вот тебя… ты немножко посидишь, подумаешь, прав ты был или не прав…что ты себя плохо вёл…
 (Мабуу) Монах!
 (Ольга) Во! Идёт?
 (Мабу)  Нет.
 (Гера)  Шевелится?  (Ольга) 
 (Ольга) Очнулся?  Встал?
 (Мабуу) Чего-то он говорит. 
 (Ольга) А-а, значит он живой, значит, ты его не сильно ударил. Сейчас он  встанет…. он тебе сейчас палкой даст.
 (Мабуу) Чего-то  он, как мой сыночек. 
 (Лена)  Бормочет?
 (Мабуу) Чего-то хохочет.
 (Ольга) А-а! Это он над тобой смеётся. 
 (Мабуу) Да-а.
 (Ольга) А ты-то чё делаешь сейчас? Ну-ка, расскажи, сейчас ты чё делаешь? Лежишь? 
 (Мабуу) Не-ет.
 (Ольга) А что?
 (Мабуу) Прячусь за камень и смотрю, как он хохочет.
 (Ольга) А как ты с нами разговариваешь?
 (Мабуу) Не знаю. 
 (Ольга) Ну, как ``не знаю''?
 (Мабуу) Просто - разговариваю!
 (Гера)  Ты не видишь нас?
 (Мабуу) Нет. 
 (Лена)  Вслух ты говоришь или ``просебя'' ты слышишь?
 (Ольга) Голос ты как слышишь, голос наш?
 (Гера)  Ты как говоришь нам? Мысленно или голосом?
 (Мабуу) Как это?
 (Гера)  Ну, вот как ты с монахом, с монахами  говоришь или как ``сам с собой”- вот,  как с нами?
 (Мабуу) Нет.  Как с собой. 
 (Ольга) Ну, ладно.
 (Мабуу) А он смеётся. 
 (Ольга) Над тобой?  
 (Мабуу) Да, показывает на меня…
 (Ольга) Пальцем? Чем?  Рукой показывает?
 (Мабуу) Не знаю. Не видно. И смеётся. 
 (Ольга) А-а. Ну, ты его рассмешил. 
 (Мабуу) Он говорит, что он теперь стал, как…
 (Ольга) Как кто? 
 (Мабуу) А можно я у него спрошу? 
 (Ольга) Спроси. 
 (Мабуу) …оче…очело… оче…
 (Ольга) О человек? 
 (Мабуу) Не-е.
 (Ольга) А как? 
 (Гера)  Чучело?
 (Ольга) Чучело? 
 (Мабуу) Нет.
 (Ольга) А как? 
 (Мабуу) Сейчас ещё спрошу.
 (Ольга) Ну-ка! Давай!
 (Мабуу) Наконец-то я ``очеловечился''!
 (Ольга) А-а! (Смех)
 (Мабуу) А это что такое?
 (Ольга) А это, как мы стал? Как мы!
 (Мабуу) Монаха бить тут? 
 (Ольга) Монаха бить нельзя. Зачем? Ну, ты стал… Короче,  ты стал лучше, чем был. Понял? 
 (Мабуу) Да. А если я его ещё раз стукну, я буду ещё лучше? 
 (Гера)  Да!
 (Ольга) Нет!
 (Гера)  Да!
 (Ольга) Нет, не будешь лучше.  Нельзя, нельзя бить монахов! Никого нельзя бить, ты что?
 (Мабуу) Чего? 
 (Ольга) Нельзя бить никого.
 (Мабуу) Если я  никого не буду бить,  меня жена слушаться не будет.
 (Ольга) Будет!
 (Мабу)  Попробуй с ними справится! Их пять штук ``оглоедов''!.
 (Смех)
 (Ольга)  А зачем тебе так много…справиться! Зачем ты их так много брал?  Жадность, да?
 (Лена)   Взял бы одну.
 (Мабуу)  Нет. 
 (Ольга)  А что? 
 (Мабуу)  Чем больше жён, тем, значит, я лучше. 
 (Ольга)  А-а! Во-от! А у нас наоборот. 
 (Лена)   Чем меньше жён. (смеется. прим.)
 (Ольга)  Ты понимаешь, чем меньше жён…Если одна жена, значит это хороший семьянин. А если много жён, значит -  плохой.  У нас наоборот! 
 (Мабуу)  Не хочу ``человечиться''. (не долго думал. Прим.)
 (Смех)
 (Мабуу)  Хочу много жён. 
 (Ольга)  Много жён…
 (Гера)   Мабуу, Мабуу,а ты спроси в монаха, это…  а вот ``очеловечится'' – это как вот? Это ты очеловечился, или он очеловечился? 
 (Мабуу) Он сказал, что ``Наконец-то очеловечился!''
 (Ольга) А ты ему скажи, что ты с нами разговариваешь.
 (Мабу)  Я сказал, но он тогда не отвечал.
 (Ольга) А-а. Ну, ты скажи, что мы передаём…желаем им здоровья мира и счастья. 
 (Гера)  Скажи, а он хочет поговорить с нами?
 (Ольга) Передай ему, что мы ему желаем здоровья мира и счастья. 
 (Гера)  С ним поговорить можем, с этим монахом? 
 (Лена)  Может он чё-нибудь скажет тебе?
 (Гера)  Через тебя. 
 (Мабуу) А что такое ``шишка''?
 (Гера)  ``Шишка'' – это на голове, когда ударишь камнем или палкой. 
 (Ольга) Это шишка, когда …вот выступ какой-то…выступ на камне есть выступ, знаешь? Выступ, шишка. Понял?
 (Мабуу)  Причём камень тут? 
 (Гера)   У тебя носик есть?
 (Мабуу)  Есть. 
 (Гера)   Вот это такой же носик, только на голове.
 (Мабу) (Смеется) Это у него носик ещё вырос? 
 (Ольга) (Смеется) У него шишка. Ты ему шишку набил, да? 
 (Мабуу)  Да… Не, он сказала что ``я теперь''…Он теперь с этой шишкой может разговаривать с самим Богом. 
 (Смех)
 (Ольга)  Короче,  ты его стукнул по голове, и у него на голове вскочила шишка, и ему больно, ты знаешь? Когда палками тебя бьют, у тебя шишки выскакивают? 
 (Мабуу) Не-е.
 (Ольга) Почему?  А-а … тебя не по голове бьют?
 (Мабуу) У меня только один носик. 
 (Ольга) А-а! А у него теперь два, представляешь? Ты ему второй ``носик'' сделал! 
 (Мабуу) А я его попросил показать, а он сказал: ``Нет!''
 (Ольга) Нельзя…
 (Мабуу) ``Ещё будешь просить – я тебя побью''
 (Ольга) А-а… Ну, монахов же нельзя видеть. 
 (Мабуу) Почему? 
 (Ольга) Не знаю. Это ты у них спроси почему.
 (Гера)  Кстати, спроси, почему нельзя видеть их?  Мы спрашиваем, скажи. 
 (Мабуу) ``Бог сказал''.
 (Ольга)  А, бог сказал… А почему?  Ты нас вот, видишь? Нет?  Не видишь?
 (Мабуу)  Я, когда подглядывал – видел. Но для того, что бы вас видеть, я должен лечь, и жена должна зажечь свечку.
 (Ольга)  О! Уже свечку даже!
 (Мабуу)  Да-а!
 (Ольга)  Ну, у вас был только огонь, а теперь уже свечка?
 (Мабуу)  Нет, это называется ``свечка''.
 (Ольга)  Эх ты-ы…А, мы теперь уже поняли, а то ты нам говорил – огонь… какая-то вода горит..
 (Мабуу)  Нет. Я учуся.
 (Ольга и Лена) Молодец!
  (Мабуу) Я теперь умею считать.
 (Ольга)  До скольки? 
 (Мабуу)  Э-э… Пять!
 (Ольга)  До пяти? Ну-ка, посчитай! Посчитай!
 (Мабуу) Э-э…Три и две.
 (Ольга) Ну, мы тебя учили. (было дело как-то раз. прим.)
 (Гера)  А три и три сколько будет? 
 (Мабуу) Так не бывает.
 (Гера)  Как? Дырка будет!
 (Мабуу) Что будет?
 (Гера)  Дырка. Если три и три. Протрёшь, и дырка будет. 
 (Мабуу) Не понял. 
 (Гера)  Это шутка. 
 (Мабуу) Если я возьму три камушка, возьму ещё три камушка… 
 (Гера)  И потри эти три камушка о три камушка.
 (Лена, шёпотом) Ну и шутки у тебя, Герка!
 (Смех)
 (Мабуу) А где будет дырка? 
 (Гера)  Вот и будет дырка, если тереть будешь. 
 (Мабуу) Я так попробую. 
 (Ольга) Мабуу, а ты не ходил к реке не смотрел на свое отражение в реке? Как в зеркало. Как в камень, в который ты смотришь, когда рисуешь? А? Не смотрел на себя? Мабуу.
 (Мабуу) Нету дырка?
 (Гера)  Это шутка, я же сказал.
 (Лена)  Шутка – это всё равно, что хохочет, смеётся…
 (Ольга) Ну, ты же шутишь? Бывает, что ты шутишь? 
 (Мабуу) Не знаю.
 (Ольга) Не знаешь? Ну, ладно. А ты у монаха спроси: почему они тебя не учат считать больше чем пять?
 (Мабуу) Нет. Они меня учат считать, сколько у меня жён!
 (Ольга) А-а! А сколько пальчиков?
 (Мабуу) Они говорят, что если я не буду знать, то я не замечу, как у меня уведут или подсунут. 
 (Ольга) А-а… ну да, другую. Ещё одну, да?
 (Мабуу) Да! Ну, я не против! Новую-ю!
 (Ольга) Ой, ты многоженец!
 (Мабуу) Чего? 
 (Ольга) Многоженец. Это, когда много жён.
 (Мабуу) Это ещё не много!
 (Ольга) А есть больше у кого-то, да? 
 (Мабуу) Да.
 (Лена)  Сколько ``больше''?
 (Мабуу) У старейшины. 
 (Ольга) Ну, давай, считай. 
 (Мабуу) Э-э… Три…(сбой)
 (Ольга) Семь.
 (Гера)  Семь жён у старейшины?
 (Ольга) Это у тебя столько будет, когда ты будешь старейшиной. 
 (Мабуу) Да?
 (Ольга) Да. 
 (Гера)  А тебе не страшно? 
 (Ольга) Мабуу…
 (Гера)  Мабуу, тебе не страшно?
 (Мабуу) Мабуу!
 (Гера)  Мабуу! Семь жён не страшно иметь? 
 (Мабуу) Не-е!
 (Гера)  Они ж тебя забьют тогда, как мамонта!
 (Мабуу) Чего?
 (Гера)  Они возьмут палки,  придут к тебе и набью тебе шишку! 
 (Смех) 
 (Ольга) Шишек будет у тебя много-много!
 (Мабуу) Я буду - как монах? 
 (Гера)  Нет, ты будешь, как ёжик!
 (Мабуу) Чего?
 (Гера)  Ёжики бегают у вас?
 (Лена)  Нет у них ёжиков!
 (Мабуу) Чего-о???
 (Гера)  Колючий такой, маленький зверёк.
 (Мабуу) Ты, какой-то странный, Петя, тебя жены бьют…
 (Смех)
 (Гера)  Бьют меня…
 (Мабуу) …ёжики бегают… У тебя сколько жён? 
 (Гера)  У меня одна. 
 (Мабуу) И-и-и!!! (разочаровано)
 (Гера)  А чё ты мне посоветуешь её побить, да, хорошенько? 
 (Мабуу) Нет, одна - это мало!
 (Гера)  Ну, мне хватает, знаешь. 
 (Мабуу) Это когда ж ты тогда будешь старейшиной? 
 (Гера)  Да я уже старейшина!
 (Лена)  Да! Он в своём доме старейшина.
 (Гера)  Да, мудрейшина. 
 (Мабуу) Не бывает старейшины с одной женой!
 (Гера)  А у нас бывает.
 (Лена)  А у нас все такие.
 (Ольга) А у нас, если самый, самый большой старейшина имеет больше одной жены - его убирают. Его не назначают старейшиной. Ему нельзя иметь много жён. Только одну.
 (Гера)  Иначе придут монахи и набьют шишку.  
 (Ольга) Да. 
 (Мабуу) А он спрашивает: …
 (Ольга) Монах, да? 
 (Мабуу)… ``Есть ли у вас солнце?''
 (Гера)  Да, есть. Жёлтое, скажи.
 (Мабуу) Чего?
 (Гера)  Ты скажи ``жёлтое'', он поймёт. 
 (Ольга) Жёлтого цвета. 
 (Мабуу) ``А есть ли у вас звёзды?''
 (Гера)  Да. 
 (Ольга) Да, есть. Очень много звёзд. 
 (Лена)  Так же, как у вас. 
 (Мабуу) Чего?
 (Все )  Есть, есть.
 (Гера)  У вас звёзды есть?
 (Ольга) На небе. 
 (Мабуу) Нет, звёзды, не видел. 
 (Ольга) (к Гере)  Ну, давай - с монахом… Сам хотел с монахом говорить…
 (Гера)  А что ещё монах хочет спросить? Ты скажи ему, что мы ждём его вопросов.
 (Лена)  Мы ответим. 
 (Гера)  Очень хотим с ним поговорить.
 (Мабуу) Видите ли вы меня? 
 (Гера)  Тебя? 
 (Мабуу) Да. 
 (Ольга) Ну, мы видим тебя… ну, другого, - не такого. 
 (Гера)  Нет, не видим, мы слышим тебя. 
 (Мабуу) А к вам…
 (Ольга) 1…3
 (Мабуу)  ``А  все ли у вас соблюдают заветы Руна?''
 (Гера)   Руна?
 (Ольга)  Да-да! Да у нас всякие бывают. У нас очень мало тех, кого соблюдают заветы. 
 (Мабуу) (шепотом) А что это такое?
 (Ольга)  Потом тебе скажем. Потом!
 (Мабуу) (шепотом) Ладно. 
 (Ольга)  Сказал? 
 (Мабуу) ``Что вы с ними делаете?''
 (Ольга)  С заветами? 
 (Мабуу) Нет. С нарушившими. 
 (Ольга) С нарушившими? Ну, пытаемся убедить, что они не так живут, как написано в рунах.  
 (Мабуу) Пытаетесь убить?
 (Все)   Нет, убедить!
 (Ольга) Убедить!
 (Гера)  То есть, рассказать, что они не правы. 
 (Ольга) Объяснить!
 (Мабуу) А-а! А я ему сказал, что пытаетесь ``много раз убить''. А он…
 (Ольга) Нет. Нет. Скажи ему, что не убить. 
 (Мабуу) Ну, у нас ``дить'' – это ``много''. 
 (Гера)  А-а-а… Тогда скажи - ``объяснить''.
 (Ольга) Мы им объясняем, что они не правильно живут.
 (Мабуу) ``Много ли у вас учеников?''
 (Ольга) У нас все ученики. Мы все учимся.
 (Лена)  Да.
 (Мабуу) ``Кто вы?''
 (Ольга)  Мы кто? Ну, мы - люди. Мы - люди. 
 (Лена)   Жители Земли.
 (Мабуу) ``Мы,  пришедшие, хотим научить'' … так и  не понял.
 (Ольга)  Что он спра… Как? Он,  это он хотел сказать? ``Мы пришедшие хоти научить'' кого?  
 (Мабуу)  Сейчас… ``Мы пришедшие хотим научить быть монахами и служителями Бога и соблюдать заповеди Руна''
 (Ольга)  А кого научить? Вас, да?  Мабуу.  Таких, как Мабуу?
 (Мабуу)  Меня!
 (Ольга)  Тебя и других, да?
 (Мабуу)  Да. ``А ещё, мы должны сохранять нить…''
 (Ольга)  Что нить? Нить времени…Нить… Ага!
 (Мабуу) ``Мы должны сохранять нити связывающие все времена и проранства…''
 (Ольга, Гера, Лена) Пространства! 
 (Олга)  Пространства, да?
 (Мабуу)  ``Мы ищем потерявшего среди…-  Меня? – Как, меня? А!  … потерявшего среди  нас. Он должен был взять здесь камень, принести его Богу, чтобы Бог дал нам новые советы''.
 (Гера)   Скажи, а сколько у них старых советов? Много? Нет?
 (Мабуу) ``Книга, вмещающая и прошлые и будущие времена, но один из них вырвал страницы, и теперь они должны найти или написать новые''. Во-о.
 (Гера)   Ясно. Спасибо.
 (Ольга)  А спроси монахи, они, где раньше жили? Спроси у него, где они раньше жили? На какой планете или в каком пространстве или времени? 
 (Гера)   На Земле они жили или нет?
 (Ольга)  Спроси у них.
 (Мабуу) ``Нет среди нас пришедших со звёзд''.
 (Гера)   А откуда они? Земные, да? 
 (Мабуу) ``Из поколения в поколение мы рождаемся, чтобы хранить книгу знаний и охранять  стол печали.
 (Гера)  Стол печали?
 (Ольга и Лена)  Престол,  престол. (шёпотом)
  (Мабуу)  Потому, что занявший его, овладеет миром и уничтожит его. Бог, создавший заповеди, сказал им, что листки эти надо хранить! Мы же, по небрежности нашей, .. А я-то…А тут причём? (к монаху)
 (0льга) Ну, ладно, ты переводи нам, и всё.
 (Мабуу) … забыли о ней и потеряли листки многие. Теперь миром правит хаос. И чтобы не распалась связывающая нить, мы должны собрать все времена в единое. И прошлое, встретивши будущее, станет сильнее,  как сжатый кувак''.
 (Гера)    Сжатый кулак.
 (Ольга)   Как сжатый кулак.
 (Мабуу) Да!
 (Мабуу)  Чего? Не-ет, я так не могу-у. (разговаривает с монахом.прим.)
 (Гера)   А что он тебя просит?
 (Лена)   По буквам скажи нам. По слогам.
 (Мабуу)  Ложится? 
 (Ольга)  Что? Поклониться?
 (Мабуу)  Ложу-усь. ``Нельзя увидеть будущее…
 (Гера)   Почему?
 (Мабуу)  ``…ибо сказано во второй заповеди, - ``Знающий будущее,  не исполнит его, ибо будет следовать лени своей. И  потому,  каждый должен создать своё, и только своё  будущее''. Множество нас, сохранит единство времени, если не бу…''
 (Лена)  Не будет… Что?
 (Ольга) Внимание, внимание! Герка!
 (Гера)  Угу.
 (Мабуу) ``Мы, скрывающие лица свои, чтобы не оставить следов в будущем, должны осторожны быть. Чтобы каждый из растущих, не помнил о нас, и, только тогда признает за своё и будет бороться за него. Потому мы, спрятавшие лица свои, должны будем исчезнуть, когда придёт время и когда будет каждый, как…'' - Я?(к монаху) `` …он  - будет подобен нам, но только не скрывающий лица свои. Придёт время, …'' – Я? Я не хочу-у.  Не я. Я понял. (к монаху) `` …придёт время, когда вы будете подобны богам, но забудете детство своё, нас, и отцов своих, и тогда хаос обретёт силу и движением вашим разрушит мир!  И тогда каждый из нас погибнет смертью, не имеющей возврата''.
 (Лена)  У-у-у…( разочаровано)
 (Ольга) Скажите, ну, а как вы себя называете, мы можем это понять?
 (Гера)  Спроси у него, как он называет?
 (Мабуу) ``Впередсмотрящие''
 (Ольга)  Первые, да? Первые? Впередсмотрящие.
 (Впередсмотрящий) Нет среди нас первых. Мы потомки их. Они же так давно были, что только в книгах остались. Но мы должны помнить их, чтобы не порвать нити. Он же… (Мабуу) Я? …является одним из узлов связующие времена наши, и потому должен быть пе… ``  - Я старейшиной буду-у!!! Молчу…  (Радуется) ``… и потому, будет первым среди своих, но только в этом клане.'' – Ну и ладно… `` Существует множество кланов, но, к сожалению, разрознены и разбросаны по земле. Мы же - должны собрать их и создать народ. И тогда будут учить нас, дадут нам… - У меня есть … `` … язык. И будут учить нас рисовать так, чтобы каждый мог понять. И дадим имя  А… амами…Амма … аммамит. 
 (Гера)  Аммамит.
 (Впередсмотрящий) И этот Амамит разобьётся, когда придёт время вашего желания стать подобно Богам. И тогда снова вы не будете понимать друг-друга, и будет много ламамитов. 
 (Ольга) А! Ну… ну, в Библии есть!
 (Впередсмотрящий)  И тогда будет время вражды. Ранее наученные убивать, получат право на…'' -  Шишки делать.
 (Гера)    Что делать? Чистки?
 (Мабуу)   Шишки!
 (Ольга)   А-а!  Бить? Убивать, да? 
 (Впередсмотрящий) И будет огонь жечь плоть вашу. И назовёте это ``священным''. И тогда мы должны будем придти в мир ваш и снова дать новые за-по-веди. И, нарушивший их, будет проклят среди вас. И тот же, кто будет исполнять только их и забудет про другие, будет также не принят''. - Во-о.
 (Ольга) Мабуу, спроси у него, как он разговаривает через тебя?
 (Мабуу) А, - я говорю.
 (Ольга) Ты его слышишь в себе? 
 (Мабуу) Не знаю,  у него  ротика-то  не видно.
 (Ольга)  Носика не видно?
 (Мабуу)  И носика не видно.
 (Ольга)  А-а, шишки не видно! 
 (Мабуу)  Не знаю. Вы меня запутали!
 (Ольга)  Ну, ладно. Ну, спроси у него…
 (Мабуу)  Во! И я его слышу, как вас, только плохо.
 (Ольга)  Мабуу, а ты нас любишь?
 (Мабуу)  Люблю!
 (Ольга)  И мы тебя любим. Ты знаешь, что мы тебя лю-юбим! Ты нам нравишься. Нам нравится с тобой разговаривать, знаешь? 
 (Мабуу)  А он говорит, что придёт…
 (сбой )
 
 Конец контакта.
Аоум. Алфавит 05 14 -01-1996г
Георгий Губин
\people{**    }
  
Аоум. Алфавит
Георгий Губин
 14-01-1996     
\soul{Что могут обозначать цифры?  Мы говорили, а если быть точнее – то мы видели что могут обозначать цифры, и какой силой обладают они. Нельзя говорить, что цифры имеют отдельные символы в отличии от оттенков цветов. Это всё едино. Столь едино, что, только играя вами, мы говорим о цифрах.}
  Цифра ``1'', вы должны помнить. Это цифра, обозначающая рождение. Это начало пути. И в начале пути всегда ждут вас опасности. В начале пути вы бессильны.  В начале пути вас можно уничтожить и сделать с вами что угодно. Вы пластилин, из которого можно вылепить что угодно. Вы так же боль, для рожавшей вас же  матери. Вы – боль для природы. Ибо ей пришлось отдать множество сил, чтобы родить вас. И она знает, что родила непокорного сына, который будет её же стараться уничтожать. И потому цифра эта имеет вид стрелы. И причём, не хватает одного звена. Почему? Потому, что это не обоюдная стрела. Эта стрела, которая убивает только вас. И больше никого. Потому - только одна сторона. 
  Цифра ``2. Здесь мы уже можем заметить, что она имеет достаточно мало острых граней, но они же всё-таки есть. И есть они у основания. Это говорит о чем? – что это фундамент, на котором зиждется жизнь – вы не знаете. Для вас он остр. И, пытаясь достичь его, вы расшибаете себе лбы. Потому у этой цифры есть острые углы, но у фундамента. Что дальше? Дальше вы научитесь играть. Вы вступили в театр жизни и вы учитесь играть. И, чем плавнее будет эта цифра, тем меньше вам будет плыть и жить. Почему? Потому что вы должны подстроиться под любого и каждого. Вот, к сожалению, в чём сущность этой цифры сейчас для вас. Поймите, когда вы станете иными, эти цифры будут обозначать уже другое. Хотя, могут сохранить тот же вид, как и цвета. Никогда не забывайте этого. Далее. Это не законченный круг. Вы согласны?
\people{Да.}
\soul{Почему? Потому, что у вас ещё есть время одуматься, есть время вернуться и вновь пойти своей дорогой, а не дорогой всех. Потому, у вас есть шанс вернуться. Какая следует цифра далее? }
  Цифра ``3''. Это когда вы уже живёте в двух мирах. Это в мире чувств и в мире разума. И там и там вы не видите окружности. Не закончены. Почему? Потому, что разум ваш незакончен. Почему? Потому, что чувства ваши еще молоды и горячи, поэтому они могут творить множество дел, как бы это делал и разум. Вот вам цифра ``3''.
  Цифра ``4''. Это когда вы уже учитесь взаимодействовать с миром. это когда идёт обоюдное обучение обоих сторон. Жизнь учит вас, вы учите жизнь. И эта грань  столь тонкая, что для многих она незаметна. И когда вы получаете удары от жизни, вы говорите о незаконном наказании Божьем. Вы не видите последствий. Потому что вы создаёте причину и не знаете, не можете соединить удары, нанесённые на вас вами же. Потому, что это не просто черта. Черта, имеющая бОльшую длину – это, конечно, мир. И маленькая – это вы. Их соединение  – это взаимосвязь соединения с внешним миром. 
  Давайте возьмём цифру ``5''. Вы видите, что это цифра 2 наоборот. Почти было бы так, почти, но только есть одно ``но'' – это только можно видеть при зеркале. Давайте возьмём цифру ``5''.  Здесь вы уже видите, что острый угол вверху. Это означает, что вы уже обладаете достаточным разумом и хотите поставить всех на место относительно себя. Иными словами, вы уже знаете себе цену. И хотите переменить весь мир. Но, он скользок, и вы не можете удержать в руках. Почему? Потому, что вы не можете ощутить, потому, что вы прошли цифру ``4''. И потому, вы, находясь на земном шаре, не ощущаете того. Вы даже не ощущаете себя полной, полной единицей всего мира, часть этого мира. Вы этого не можете сделать. 
  Цифра ``6''. Здесь  вы уже видите законченный круг. Вы согласны?
  
\people{Да.}
\soul{Но он находится внизу. Это ваш фундамент. Это значит, что мир вами теперь понят. }
  
\people{Это и цифра у человека - шесть? ( имелось ввиду ``число человеческое'')}
\soul{Мир вами понят. Именно вами. А принят он верно или нет - это уже не важно. Вы уже поставили все точки над ``и''. Вы уже достаточно взрослый человек, вы уже считаете, что всё познали. Потому, для вас будет замкнутый круг. Но, вверху нет соединений! Почему? Потому, что это разум ваш создал понятия, но, не вы сами. Вот вам, цифра ``шесть'' – цифра зверя. Ибо вы видите Мир искаженным через свой разум. А когда-то говорили вам, что разум ваш это и есть сила дьявола. Потому, что они не имеют основ божьих. Что с того божье?  Это чувства ваши. А вы их забываете. У вас - голый разум. Голый разум, который познал весь мир. Вот вам -  цифра ``шесть''.}
  Цифра ``7''.  Это когда приходится работать уже нам. Это сомнения. Это когда вы стараетесь разорвать прежние свои понятия, вы возьмите шестёрку и попробуйте её разорвать. Что вы получите? Вы получите цифру ``семь''. Потому цифра семь это и есть цифра Христа. Он пришёл, чтобы вас сбить с проторенного пути, чтобы остановить  вас и показать вам верный путь. Вот что значит цифра ``семь''. А что делаете вы? Да, вы начинаете рушить, рушить всё.  И вы можете вернуться опять назад, на прежние цифры, на те, что уже были произнесены. Но если вы не сделаете того, вы перейдёте на цифру ``8''. И здесь вы уже имеете два полных круга. Это значит, что разум ваш и Мир - соответствуют. Вы согласны? Вы могли бы заметить  разницу, что нижний круг больше, всё-таки, вашего. Почему? Потому, что Мир, всё-таки, больше вашего разума. Больше. Это душа не имеет размеров. Разум же - имеет. 
  Теперь цифра ``9''. Цифра ``9'', это когда уже остался один разум. Всмотритесь в эту цифру. Вы отказались от природы, вы отказались от сил природы. Почему? Потому, что вы возомнили себя богами. И вы уже хотите править природой. Вот вам цифра ``9''. А это значит, что это конец. Это конец вашей жизни. Ещё один из понятий цифры ``9'', это когда приходит время разуму вашему расстаться  с внешним миром. Посмотрите, разум ваш остаётся заключённым окружностью, внизу же вы видите разрыв. А мы ж говорили, что низ означает окружающий мир. Это значит, что приходит время, уйти из этого мира. Вы поняли? И, теперь, цифра ``ноль''. Это когда – всего лишь, только один круг. В нём нет ни разума, ни природы. В нём есть единство. Это тот мир, где вы растворились, где вас нет физически. Иными словами, или вы ещё не родились, или вы уже умерли. Т.е. вы просто соединились с миром, но мир этот, всё-таки, искажён вашим разумом, потому что он, всё-таки, имеет форму овала, но не круга. Вы согласны? Идеальным считается круг. Форма овала говорит о том, что у вас осталось воспоминание о прошлых жизнях. И у вас есть воспоминание о будущей жизни. И рождаясь, к сожалению, если вы не будете пытаться меняться, вы вернётесь на цифру ``11''. Что это? Это то же самое, что была ``единица'', но только в два раза хуже. 
\people{Скажите, а число рождения – влияет?  Именно – число, месяц, год.}
\people{Может это кодовые какие-то комбинации?}
\soul{Хорошо.  Буква ``А''. Дети называют её ``крышей''. В какой-то мере они правы. Потому что вы ещё отделены от мира. Вы спрятаны в утробе матери. И она защищает вас. Пуповина связывает две разные личности – это мать и вы. Разорвав эту пуповину, вы должны понять, что это будет буква ``Л''. Но, давайте по порядку.}
  Итак, буква ``А''. Вы видите опять острые концы. И мы можем назвать это крышей, дети  тут правы, потому что она действительно защищает. Крыша поката. А этим,  не оставляет никаких осадок. Это одна из функций вашего очищения. Чтобы вы родились достаточно чистыми. Давайте далее.
  Буква ``Б''. Если присмотреться и разобрать её по частям, то эта буква уже содержит слово ``Бог''. Здесь есть все три буквы. Попробуйте, разглядите. Что только вы подразумеваете под ``богом''. ``Бог'' – это не только покорность и служение ему, но это любовь. Это любовь, это защита. Вы слабы, но у вас есть защита. Так и ребёнок относится к матери. Для ребёнка мать – это бог, это земной бог, который даёт пищу, который может согреть и продолжить жизнь. Вы поняли?
\people{Да. }
\soul{Ну, давайте посмотрим на рукописные. Вернёмся к букве ``А''. Что вы видите? Это - ноль. Это ноль, но уже имеющий связь с внешним миром.  Вы согласны?}
\people{Да, это верно.}
  
\soul{Это мать, несущая в утробе ребёнка. Теперь буква ``Б''. В основании вы опять видите круг. Он замкнут. Над основанием – вы называете это ``серебряной нитью'' или ``связь''. Та же связь с богом. В данном случае – связь с матерью. Мать никогда не теряет детей, она их всегда чувствует, если она действительно мать. И она всегда связана с детьми этими нитями. ``Серебряная нить''. И когда вы теряете мать, нить не обрывается. Нить складывается в клубок, остаётся в вас, и это вы называете воспоминаниями о матери. Это вы называете, что мать, даже умерев, охраняет вас. Пойдёмте далее.}
  Буква ``В''. Когда-то мы пытались вам сравнить её с материнской грудью. Вы должны были бы помнить это.
\people{Да. Помним.}
\soul{Это значит, что вы, рождаясь в этот мир, получаете и должны уже получать физическую пищу; не переработанную матерью – свою. Но вы ещё слабы. Об этом говорит левая сторона буквы. Поэтому, вы должны ещё питаться соками матери. И беда в том, что матери теряют эти функции. Пришло время, когда вы пытаетесь создать искусственных матерей. Это не приносит блага. Ибо вы не можете повторить природу. Природа же сделает всё возможное, чтобы мать могла прокормить своё чадо. И природа не расчитывала и не ожидала, что вы в матерях будете уничтожать самих себя. Вот вам буква ``В''. Пойдёмте далее.}
  Буква ``Г''. Мы уже упоминали её в слове ``бог'', и о том, что эта буква содержится. Буква ``Г'' - как видите, здесь только один угол. Один угол. Это когда мать и ребёнок, только они могут понять друг друга. Даже отец не может понять ребёнка. Он не знает, что делать. Только мать знает, когда его уложить спать, когда его накормить. К сожалению, эта буква перерастает в прописную. Это когда мать уже имеет заботу не только о ребёнке, но и другие. Нарисуйте букву ``Г'' прописную и увидите, что она состоит из двух полуколец. Это значит, что мать уже начинает терять связь с ребёнком. Мы говорим не о ``серебряной нити'', а говорим о духовном, о духовном понятии. Когда мать, уже потеряв радость, что ребёнок  родился, начинает уже гневаться. И даже умудряется наносить проклятья. Не со зла, но слово, как вы говорите – не воробей, и отображается на ребёнке. Когда происходят в семье ссоры, он младенец, и не думайте, что он ничего не понимает. Да, слов он не понимает, но он их запомнит. А потом будет знать их назначение. Слов он не понимает, но он видит ваши излучения. Как вы говорите, ваши ``эманации''. Что делают они? – они ломают ребёнка, ломают его чистую решётку. Что можно представить в машинном варианте ребёнка? Это множество ячеек, и в каждой из ячеек заложен частички характера. Рождается ребёнок достаточно чистым, у него достаточно правильная решётка. Но любые корысти ломают её. Вы должны были заметить, что ребёнок всегда плачет, когда ссорятся родители или братья-сёстры. Даже когда в соседней квартире что-то происходит, ребёнок это чувствует и нервно спит. Вы же, не зная причин, начинаете волноваться и, тем более, вы увеличиваете его беспокойство. Вами сказано: ``прокисает молоко''. Это верно. Молоко, конечно, не киснет, но содержание адреналина и множества других веществ повреждает ребёнка,- почему вам и говорят: ``ребёнок уже может слушать музыку''. Мать, неся дитя ещё в утробе, должна посещать те (теряется).
  ``Д''. Дом. Да, действительно, он имеет понятия ``дом'' в первую очередь. Дом – это ваша Земля. Это действительно дом, кров, в котором вы родились. Но этот дом состоит из буквы ``А'', вы согласны?
\people{Да. }
\soul{Только вместо перекладины нет той пуповины. Вы уже родились. То-бишь, буква ``Л''. Она имеет основание, фундамент. Иными словами, дом – это то место, где вы можете чувствовать себя свободно, к которому вы привязаны. Фундамент. Ибо дом вы будете помнить всегда, всю жизнь. Вы, пройдёте, и всю жизнь будете искать старый дом, старый дом детства, где вы жили. Дом, где вы были ребёнком, и где родители ласкали вас. Дом, где у вас не было забот, не было печали. Дом, где, если вы плакали, то плакали чистыми слезами. Там, где если вы обижались, то обижались без злобы. Там, где если вы радовались, то радовались без всяких предвзятостей. Там, где вы давали и произносили слово ``на'' без всяких мыслей, что это вернётся. Дом, где вы были ребёнком,  и те ощущения ребёнка всю жизнь вы будете искать. Вы будете искать и вспоминать. Многие из вас скажут: ``Не было у меня дома; у меня был детдом, там было очень плохо-плохо-плохо''. Но пожалуйста, не путайте строение с домом, который носит это строение один и тот же слог. Дом - прежде всего в вас, а не то жилище. Если вы достаточно плохи, то и дом у вас будет достаточно плох. Именно тот дом, который вы имеете в понятии. Потому и есть детдомА. Это не значит, что вы были одни в прошлой жизни, не надо применять здесь слово ``карма''. Нет. Здесь оно не подходит. Почему? – потому, что вам надо бы помнить, что большинство плохих людей вырастает из хороших семей.}
  (обрыв плёнки) 
  (…)
\people{Да. }
\soul{Цифра восемь. Она похоже, конечно, на ``В''. Похожа. Но смысл, более близкий к ``З''. почему? Почему? Да потому, что вы забыли. Забыли родной дом. Вы забыли о своих родителях. У вас есть множество забот и вам некогда придти к ним. И если они не болеют, то даже радуетесь и говорите ``отмучались'', а по-настоящему вы радуетесь  себе. Буква ``З'' - буква печали. Буква боли. Боли (теряется)}
\people{Буква Е. Похоже, на интенсивное обучение чего-то.}
\soul{Ну, тогда скажем так: не похоже ли это на три единицы?}
\people{Похоже.}
\soul{Но только они параллельны. }
\people{Да.  И соединены.}
\soul{И соединены. И они лежат. }
\people{Да.}
  
\soul{Почему? Вы даже не помните, что обозначает единица. }
\people{Начало. }
\people{Это начало. }
  
\soul{Буква ``Е''. ``Есть''. Это когда вы уже начинаете осознавать (…). Вы уже начинаете осознавать своё ``Я''. Что вы – человек. Что вы такой же, как ваши родители. Вы такой же, как ваши сверстники. Конечно, вы ещё маленький, но вы уже знаете, что вы живёте. Вы уже начинаете понимать, что такое жизнь. Вы уже почувствовали, что такое боль. Вы уже почувствовали, что такое обида. Вы уже почувствовали множество-множество-множество. И всё это было для вас первое. Первое, а значит, самое запоминающееся. Вот что такое буква ``Е''. поняли?}
\people{Да.}
\soul{Есть и другие варианты, но давайте пойдём далее.}
  Далее, буква ``Ё''. Да, вы ещё не исключили её, хотя уже идёте к тому. Вы уже очень редко пишете её. А должны бы знать, что выражение ``расставить точки над и'' относится и к этой букве. Это значит, всё то же самое, что буква ``Е'', но только уже разобрано и разложено по полочкам. Давайте дальше.
\people{Е, Ё, Ж. Ж.}
  
\soul{``Ж''.}
\people{Да.}
  
\soul{Жук.}
\people{Да.}
\people{(…) была бы. }
\people{Буква Ж.}
\soul{Нет. Это две буквы ``С''. Но т.к. мы не знает ещё этих букв, то давай опустим это.}
  Первое,- это центральная. Центральная – это начало вашего жизненного пути. Начало вашего самостоятельного движения. Но, вас поддерживают. Вас поддерживают ещё мать и отец. Без них вы беспомощны, в отличие от более низших, в физическом плане. Возьмите животных и заметьте: чем выше и сложнее организм, тем больше и дольше ему нужны родители. (далее идёт сбой и вклинивание прерванного разговора с одной из реинкарнаций со времен Атлантиды )
\people{Да.}
  
\soul{Они есть, я их чувствую…}
\people{Кто? Гора?}
  
\soul{Гора, - тебя зовут. Он слышит вас.}
\people{А кто? }
\people{Это, опять Анур, да?}
\people{Сколько времени прошло, как мы встречались?}
  
\soul{Три дня.}
\people{А у нас, всего лишь, несколько часов. Около двух часов.}
  
\soul{Да.  }
  (конец разговора)
\soul{Мы говорили о поддержании родителями.}
\people{А, да-да.}
  
\soul{Что вы можете уже идти самостоятельно, но вам нужна поддержка. Вы ещё молоды и не знаете этот мир. Если в имени вы  встретите эту букву, то вы можете говорить, что вы ещё молоды.  Но, вы должны и помнить, на каком месте стоит эта буква. Если она стоит в начале имени, это значит, вы начинаете. Начинаете путь во всём. Если она стоит в конце, то вы прошли уже многое, как вы говорите – прошли уже множество реинкарнаций, хоть это всё-таки немножко не так. И вы уже находитесь на конце пути, но ещё, к сожалению, не всё знаете. И для вас этот конец ещё нов, и вы его боитесь. }
\people{А как же французское имя Жорж?}
  
\soul{А вы возьмите и совместите начало и конец. Это первое. И второе – будьте внимательны, возьмите ваше ``Ж'' и другое. }
\people{Да.}
  
\soul{Вы же видите, что они разные в написании и, значит, будет иной и  смысл. Вы же не можете сказать, что христианство и мусульманство, это одно и то же? Здесь есть различия. Хотя, речь об одном едином Боге. Также вы можете сказать и о разных писаниях, о разных языках. Да, основа звука одна, но графическое обозначение – разное. И вы не забывайте, что вы должны брать комплексно. Если вы хотите разобраться в букве, вы должны произнести её, вы должны нарисовать её во всех возможных вариантах, на всех языках,- если вы хотите полностью разобраться в ней. Вы должны написать её всеми почерками, которые могут быть. И, даже почерком дикаря, который случайно нарисовал эту букву, но даже не знает о том, что она обозначает. Только тогда вы сможете познать все значения. А пока – это один из вариантов. И не больше.}
\people{Скажите пожалуйста. А можно спросить вас? Я перебиваю. Я извиняюсь. А вот, допустим, вот лично я, в детстве, попросил близких называть меня другим именем. Это что значит? Или имя не соответствует, или просто я нафантазировал?(…) }
  
\soul{Нет, этого не может быть. Всё гораздо проще. Вы ищите созвучия, к которым вы ближе. Если бы вы были внимательными, то первое и второе имя – созвучны. Их нельзя назвать ``другими'', ``иными''. Они созвучны. Значит, основа та же. Вы не изменились. Изменилось только отношение к окружающему миру. Но, только не путайте. Изменилось отношение, и лишь только потом изменилось имя, но не наоборот. }
\people{Это что же , у меня в три года изменилось отношение к миру?}
  
\soul{А вы в три года считали себя достаточно глупым? А как же тогда ваше понятие о прошлом? }
\people{Всё понятно, спасибо. Продолжаем.}
\people{Буква ``З''.}
  
\soul{Про букву ``З'' мы вам только что говорили.}
\people{Извините.}
  
\soul{Мы можем добавить. Да, это действительно похоже на букву ``В''. Да, это действительно похоже на женские груди, но вам уже не нужна мать. Вы стараетесь избавиться от её забот, она мешает вам. Вам мешают родители. Вот вам проблема отцов и детей. Вы хотите жить самостоятельно и вам наплевать, что советуют они. Да, вы скажете: ``я могу послушать, но сделать по-своему''. Это - то же самое. Любой совет, которым побрезговали, именно побрезговали, по-другому сказать нельзя,  - это обозначение неуважения. И даже если человек - неправильно, в любом случае вы не должны отказываться. Вы должны воспринять, понять, потому что он делал это вам от чистого сердца. Есть и масса толковых предложений, за которые вы всю жизнь благодарны человеку, но он предлагал вам от злости. Так что же ценнее более?  Этот, а не то, что было дано добротой, но неверно? Вы, конечно, признаете то, что принесло вам пользу. Вот ваша беда. Давайте далее.}
\people{``И''.}
  
\soul{``И''. Вы можете найти аналог в цифрах?}
\people{Да, это верно. Извините.}
  
\soul{Вы мне можете сказать, что эта буква  значит? }
\people{Зеркало.}
  
\soul{А если быть точнее?}
\people{Я не знаю.}
  
\soul{Давайте, скажем по-другому: вы хотите соединить то, что заранее не соединяемо. У вас ещё нет понятий ``это можно'', ``это нельзя''. Вы учитесь. Вы учитесь, вы делаете первые шаги. Это же всё-таки ``единицы''. Вы стараетесь соединить. Вы не знаете, что можно, что нельзя. Именно в этот момент вы могли бы научиться летать. Ибо вы ещё не знаете, что это невозможно. Для вас ещё не существует законов физики. А именно - они действуют на вас, но вы не хотите подчиняться им. Почему? Потому, что эти законы вы познаёте впервые, впервые именно в этой жизни. И стараетесь соединить их со своими желаниями. Вот вам буква ``И''. Какая идёт далее буква?}
\people{``Й''.}
  
\soul{Опять же, - точка. Это - когда вы ставите окончательное решение. Когда вы уже говорите, что вы уже поняли  мир, что вы уже всё знаете. Происходит фундаментация вашего сознания, когда оно уже теряет гибкость. Вот, что это значит. Давайте, далее.}
\people{``К''.}
  
\soul{Прекрасно. Теперь посмотрите на букву ``Ж'' и на букву ``К''. }
\people{Половина.}
  
\soul{Вы можете добавить? }
\people{Да, теперь понятно, почему мужчины первыми умирают.}
  
\soul{Почему?}
\people{Потому что ``К''. Поддержка с одной стороны.}
  
\soul{Мы говорили об ``А'', потерявшей перекладинку.}
\people{Да.}
  
\soul{Это - когда вы хотите создать свой дом. Вы строите свой дом, в нём нет фундамента. И потому, этот дом очень зыбок. Вы часто меняете  дома. Иначе – вы часто меняете ваши понятия. Сегодня вы один, завтра – вы другой. Сегодня вы говорите об одном, завтра уже отрицаете это. У вас нет стабильности. У вас нету того дома, который был в детстве и держал мощный фундамент. }
\people{Понятно.}
  
\soul{Теперь представьте ``М'' объединённую … Попробуйте. }
\people{Друг другу навстречу.}
  
\soul{Единицы. Конфронтация. Т.е. битва с самим собой. И, опять, заметьте, односторонняя. Только ранит. Только  вас. Битва с самим собой, со своим отражением, со своей тенью. Как вы думаете, много от этого пользы – сражаться со своей тенью?}
\people{Нет.}
  
\soul{Нет. Почему? Потому, что вы уже заранее называете её тенью, заранее унижаете её, а значит – не готовы к борьбе, потому что противник вас недостоин, а значит – слаб. И этот противник приходит и побеждает вас. Ибо, вы не готовились! Вы считали его слабым и не рассчитывали на его силу. Давайте далее.}
\people{Буква Н.}
  
\soul{Вы произнесли её неправильно и поэтому очень трудно воспринять её. И только зная, что вы произнесли именно её, можно сказать о ней. Это ``эн''. И, заметьте, что по звучанию, не очень-то и большая разница. }
\people{Ну, да.}
  
\soul{А теперь давайте рассмотрим графически. Есть здесь разница?}
\people{Да, есть. Но …}
  
\soul{И большая?}
  
\people{Это соединение  двух единиц… соединены второй линией.}
  
\soul{Хорошо, тогда приведите аналог буквы четыре.}
\people{Цифры четыре.(поправка. прим.)}
   
\soul{Здесь мы говорили о маленькой черте, которой являетесь вы. Сдесь же, черты равномерны. Они имеют одинаковую длину. И существует та же самая маленькая перекладина – та же цифра ``4''. И в то же время, это та же буква ``М'', которая имеет крайние грани одни и те же. Лишь только в букве ``М'' более ломаные линии соединения. Что буква ``Н''? это когда вы уже твёрдо стоите на ногах. Это когда вы уже имеете полное понятие - не важно, верны они или нет,- для вас они достаточно полные: понятие об окружающем мире, понятия о себе и об отношении к вам. Верны они или  нет, это уже вас не интересует.}
\people{Скажите, а слово… Священное слово, я не знаю, можно его произнести? Аоум.}
  
\soul{Возьмите, разберите его. По тому же принципу. }
\people{Понятно.}
  
\soul{Давайте следующую букву. Мы говорили вам не о правильном звучании, когда вы произносите физически, а о духовном. Любое слово, правильно произнесённое будет обладать силой  любых  других слов, произнесённых просто так. Почему взято именно это? Потому что его тяжелее переврать, но всё-таки перевираете и это. Называйте следующую букву.}
\people{``О''.}
  
\soul{Буква ``О''. Аналог цифры ``0''. Букву ``О'' - возьмете и посчитайте, на каком месте стоит она в алфавите. Цифренный порядок. И попробуйте, разберите эти цифры, рассматривая их отдельно. Это делайте сами. Мы же, скажем вам другое: всегда  и везде  круг считался символом, символом-образцом. Круглая Земля, круглый мир. И даже когда Земля была плоской, небо, всё-таки, было круглым. Дальше. Вращение всех планет. Совпадает?  }
\people{Да.}
  
\soul{Совпадает. Прекрасно. Значит, мы уже можем сказать, что вы уже имеете понятие о космосе. Вы уже связываете себя с её органами. Вы ощущаете себя частицей Вселенной, а не просто Землёй, как вы говорите – ``чи'' . Вы уже считаете себя единицей. Пусть она будет полная, не полная, пусть она будет рассеяна в ней и будете бояться, но вы уже чувствуете влияние космоса. }
  
\people{Это мы можем об имени и о фамилии сказать?}
\soul{Конечно. Чем отличается имя от просто буквы? Тем, что вы уже имеет набор. Нельзя же сказать только просто, что вы просто ненавистны и всё? Ведь вы же можете любить, вы страдаете, вы живёте, у вас множество чувств, потому – и множество букв.  Но только не подумайте, что если короткое имя, значит и чувств меньше. Нет. Дело не в этом. Вспомните: ранее имена носили лишь только гласные. Почему? Потому, что раньше не было согласия. Раньше каждый из вас жиль дикарём отдельно. И хотя была принята община, это всего лишь было название ``Община''. Просто вы же: толпою легче забить мамонта, чем одному. Спрашивайте дальше.}
\people{Буква ``П''.}
  
\soul{Здесь уже нет аналогов в цифрах. Прямого аналога мы не имеем. Можно было б сказать ``ноль'',но тогда это было бы множество допущений. Законы, имеющие допущения  – только у вас. Буква ``П''. Это круг. Но только не имеет основания и имеет острые углы. Почему? Потому что вы хотите и работаете только фактами. Вы не хотите принимать ``что-то там такое'',- вам нужно подавать конкретное. Вы говорите о любви и тут же подсчитываете: что, какую пользу принесёт любовь. Вот ваше ``О'' . }
\people{``Р''.}
  
\soul{``Р''. Имеет ли аналог? }
\people{Буква ``Б'', но только без одной… }
\soul{В цифрах.}
\people{“Шесть”}
\people{“Девять''.}
\soul{Цифра ``6'', цифра ``9''. Значит, есть два варианта. И если быть ещё точнее, то четыре. Это ``6'', ``9'', ``69'' и ``96''. Эти два варианта мы можем опустить. Имеет ли различия ``Р'' прописная и заглавная? }
\people{Имеет.}
  
\soul{В чём? }
\people{``Р'' заглавная замкнутая, а ``р'' прописная – с хвостиком. }
  
\soul{Давайте  скажем, есть ли общий элемент?}
\people{Есть. Прямая. Единица.}
  
\soul{Прямая. И находится она в левой стороне. Как принято у вас ``левое-правое''? Правое – вы считаете это ``правда'', ``левое'' - ложь. }
\people{В принципе, да.}
  
\soul{Это значит, что пришло время, когда вы лжёте, когда вы можете лгать. Замкнутый круг – это вы лжёте в первую очередь самому себе. Вы делаете это всегда. Но замкнутый круг говорит о том, что ложь бьёт вас. Вы ещё не умеете достаточно хорошо лгать. Возьмём прописную букву. Здесь вы имеет уже волнистую линию. Договорились: черта – это ложь, ибо находится с левой стороны. Теперь попробуйте определить что такое волна. }
\people{Ложь, направленная во-вне?}
  
\soul{Мы вам подсказываем: волна. }
\people{А, волна…волны… - это распространение. }
  
\soul{Нет. Это значит, что вы ещё не имеете постоянного. Вы – как маятник. Вы волнуетесь о каждом пустяке. Волнует вас всё. У вас ещё живое восприятие мира. Чтобы не было разногласий, давайте договоримся: что мы говорим об алфавите,  относящий к возрасту человека. Если вы помните, мы начинали с нуля, начинали с младенца. Когда мы будем говорить и повторять глядя на алфавит, то это уже будет другая тема. Мы говорили вам о множестве значений. Сейчас же мы рассматриваем ваши возрастные изменения. Давайте далее.}
\people{Десять. }
  
\soul{Круг. Он замкнут. Вспомните, мы говорили вам о  цифрах. Вы помните, мы говорили вам о нуле? Мы говорили вам о девяти и о шести. И что обозначает замкнутый круг? Вспомните, что он, всё-таки, находится наверху. }
\people{А, это буква ``Р''.}
  
\soul{Верх, это сознание. Сознание, высшее до той степени, что уже преобладает над чувствами. Сознание заставляет вас ругать. Сознание бьёт вас. Сознание – враг ваш. Иногда вы это называете переходным возрастом. Когда вы не можете сознанием понять окружающие вещи. Мечетесь и чувствуете себя маленьким человечком. Ненужным. Тогда происходят проблемы с родителями, с родственниками; и жалуетесь на их невнимание, на недостаточность внимания. Вы согласны?}
\people{Да.}
  
\soul{И, тогда, вас уже смело можно обозначить  буквой прописной. Малой буквой. Давайте следующую.}
\people{Эс.}
  
\soul{Есть ли разница между прописной? }
\people{Никакой.}
  
\soul{Никакой. Только размер. Давайте возьмём заглавную букву. Давайте, всё-таки, присмотримся, как вы будете писать эти буквы. Напишите большую и напишите малую. И найдите различия между ними. }
\people{Большая - более округлая. }
  
\soul{Нет. Большая выделяется среди остальных. }
\people{Ну, естественно.}
  
\soul{``С'' прописная всегда будет малой, и всегда будет иметь тот же размер окружающих букв. Почему тогда, имя, носящее ``С'' или имя, носящее ``С'' в середине - будет различно? Почему?}
  
\people{(…). Если начинается с буквы ``С'', значит, выходит, что сам человек САМ. Даже ``самость'' с буквы ``С'' начинается, так? А если в середине, то здесь уже человек более, наверно,  прислушивается к другим; больше, чем к себе.}
  
\soul{Да, но вы забываете, что это не замкнутый круг. }
\people{Может, это стремление - замкнуть?}
  
\soul{Нет. Это стремления взять, взять как можно больше себе. Это стремление, когда вы хотите научиться жить. Это стремление приспособиться к окружающей среде. И потому – вы открыты, и потому – вы более  ранимы. Вот вам ``переходной возраст''. Но это уже не начало. Здесь вы уже не лжёте столь часто, как делали раньше. Здесь вы стараетесь поставить своё ``Я'' выше других. Вы уже хотите, чтобы с вашим именем считались. Вы хотите завоевать весь мир. Вы хотите, чтобы мир знал вас. В эти моменты вы представляете о своей смерти. И о той печали, что принесёт она не вам, а другим. Здесь вы проигрываете множество вариантов своего будущего. Здесь вы мечтаете. Здесь мечты ваши дают корни в жизни. Здесь вы выбираете профессию и здесь, а не когда в садике вы говорите, что будете космонавтом. Возраст выбора. Давайте следующую букву.}
\people{``Т''.}
  
\soul{Здесь  вы можете сказать о кресте. О кресте }
\people{Незаконченном.}
\soul{Незаконченный ли он? }
 
\people{Тау. Раньше это - ``тау'', называлась эта буква.}
  
\soul{Хорошо, давайте попробуем другие языки. Давайте, возьмём ``Т'' в латинском варианте. }
\people{Такая же. }
  
\soul{Прекрасно. Если взять другой вариант?}
\people{Английский если взять вариант, то же самое. Только немножко там хвостик.}
\people{В прописном.}
\people{Да.}
  
\soul{Давайте по-другому. Черта находится посередине. Её пересекает линия. Здесь, здесь время выбора быть вам земным или увидеть высшее. Та перекладина, та стена, тот потолок, что позволяет увидеть вам небо. Это время, когда вы разочаровываетесь в своих мечтах. Вы видите свои неудачи. Но вы уже не верите своим фантазиям. Вы теряете способность фантазировать. Ведь в мир вы могли придумать многое и безостановочно. Теперь же, став взрослым, вы уже не можете сочинять просто так. Вы обязательно будете делать поправки на жизненные ситуации. Вы обязательно будете вставлять технические термины, которые уже знаете. Согласны?}
\people{Да.}
  
\soul{Иначе: вы уже стали применять ``научный мат''. }
\people{``Научный мат''? }
\people{Мат – в смысле ``плохое слово''?}
  
\soul{Да, вы уже знаете слишком много слов, вы уже знаете слишком много терминов, вы уже знаете что можно и что – нет. И та перекладина является количеством ваших знаний, вашей границей возможностей. Вы уже знаете, что не умеете и не можете летать. Вы уже не мечтаете, о том, что когда-то научитесь этому. Вы уже знаете, что вы не можете плавать и увидеть подводный мир. Раньше же, в детстве, каждый из вас всегда интересовался, чем же дышат рыбки, и почему они молчат. Теперь же вы знаете, что рыбы не молчат. Они тоже разговаривают, но по-своему. Вы уже знаете термин ``ультразвук'', ``инфразвук'' и уже стараетесь применить в свою пользу. В свою. И слава богу, что вы не знаете, как этого сделать. Вот вам буква ``Т''. Середина же обозначает, что вы уже осваиваетесь в этом мире, вы уже знаете себе место и стараетесь закрепиться в нём. Давайте букву.}
\people{``Ч''.}
  
\soul{``Ч''. Давайте не будем говорить о ``чаше''. Иначе, это уже навевает на мысль, что ваша вся жизнь, это, вего лишь, чаша, которая наполняется, чем попало. Нет. Давайте представите немножко по-иному. Давайте мы перевернём эту букву. И вы уже скажете, что это можно назвать стулом. Стул. Что даёт вам стул? Опору. Наконец, вы можете сесть и отдохнуть. В ногах нет правды. Есть время подумать и поразмыслить, отдохнуть. Итак, буква ``Ч''. Вы – линия справа, она говорит о вас. Вы пытаетесь подвести черту, итог вашей жизни. Что вы сделали? И заметьте – за этой чертой нет ничего. Вы пока не гадаете о будущем, вы пока не строите планы о будущем. Вы хотите подвести черту о прошлом: что делали, где были ваши ошибки. Как часто можно услышать вашу фразу: ``Ох, какой я был глупый!'', ``ах, если бы, я бы многое сделал по-другому'' и многое-многое ``если'', потом вы останавливаете это резко и говорите: ``да, всё прошло, поросло травой'' и понимаете безысходность своего положения. }
  Итак, слово ``чело''. Что обозначает слово это? И то – лицо ваше. Иными словами, глядя на лицо, вы можете сказать, что это за человек. Даже если и ошибётесь, то не намного. И, глядя на вашу жизнь, можно сказать кто вы. Вы же прячете свою жизнь. У вас есть достаточно много нюансов, которые вы прячете всю жизнь и стараетесь забыть. И чтобы не дай бог об этом узнал кто-то, не дай бог ещё и ближний человек. Вы согласны, что у вас множество-множество того, что вы хотите спрятать? Черта. ``Ч''. Та же сама черта, черта. Граница, разделяющая ``было'' и ``есть''. И, заметьте, здесь уже нету ``будет''. Вы хотите разобраться, что же вы сейчас. Вам не до будущего. Вам нужны факты, факты, которые могли бы установить: ``так кто же вы в конце концов?'', ``что же вы творили?'' и ``зачем всё это вы делали?''. И, лишь только потом вы будете задумываться: ``а что же дальше?''.
  (разрыв записи)
  ``Хэ''? Что это? – это бумеранг. Это, в первую очередь, бумеранг, который возвращается к вам же. Слово ``хорошо''. Слово ``хорошо'', относящееся только к вам. Слово ``хорошо'' вы можете понять только тогда, когда вам хорошо. Когда к вам придут и скажут: ``мне хорошо'', вы не почувствуете того, пока вам не захочется сказать этого слова. Это бумеранг, который бьёт вас же, если вы промахнётесь. Вы же промахиваетесь довольно-таки часто. 
  Буква ``Х''.  Это значит, что вы поставили крест. Поломанный крест. Упавший крест. Крест на всё. Вы устали. Это конец алфавита. Это значит - уже идёт к концу ваша жизнь. Не важно, что вам будет всего, допустим, 10 лет. И в 12 и в 10 можно умирать. Никогда не давайте этих обозначений неодушевлённого и неживого. Никогда не соизмеряйте с возрастом. Ибо, возраст не в годах – запомните это. Давайте далее.
  Крест. Крест, обозначающий, что вы пришли к началу конца. Это значит, что вы уже давно несёте СВОЙ крест. Крест, который давит вас, если вы не знаете, как нести его. Вы ищете те кресты, которые помогают идти вам по жизни. Которые дают вам клинья. Вы носите те кресты, которые давят вас к земле, и вы не можете подняться, и вам тяжело, и вы всё жалуетесь, что у вас сложная жизнь, что  вас обидела судьба. Вы проклинаете всё и себя. Вы ищете виновных везде, но не в себе. Есть и тот крест, который позволяет вам взлететь, взлететь выше всего, взлететь выше всех дел и посмотреть на них сверху и увидеть истинное их положение: что вы сделали, что вы должны ещё сделать. Крест – это тот же пропеллер, который позволяет подняться вам в небеса. Но это, и тот же крест, который пригвоздит вас к сырой земле. 
\people{``Ц''.}
  
\soul{``Ц''. давайте вернёмся по кругу опять же. Наверно, это чаша. Чаша, но достаточно в грубой форме, столь угловатой и столь колючей и столь неустойчивой, что чаще приносит вам вреда, чем пользу. Вы становитесь грубыми. Грубеет ваше тело, иными словами – стареет. Вы уже чувствуете себя не той физикой. Вы уже чувствуете усталость.  Время уже чувствуется вами. Уже волосы ваши седеют. Уже мысли ваши стареют. Вы уже - перестаёте мечтать. Или что-то одно беспокоит вас: побольше прожить или умереть без мучений. Эти углы, и тем более остриё, колют вас. Вы говорите: ``боль. Что-то защемило в сердце''. Вы жалуетесь на весь организм. Вы жалуетесь на самого себя, вы начинаете шаркать ногами, вы уже становитесь брезгливыми ко всему. Вы уже начинаете брюзжать, вы уже становитесь нервными. Вас не устраивает ничего - вам нужен покой. Вы хотите покоя. Вы хотите разобраться, вы уже постарели одним только тем, что вы уже вспоминаете. Вы стали вспоминать. Встречаясь с друзьями,  вы говорите: ``а помнишь…'' - вот вам признак старости. И вы начинаете копошить старое прошлое, вы начинаете ковыряться в этой чаще, выискивать самые прекрасные моменты жизни, когда вам тяжело, чтобы успокоиться. А чаще всего, когда вам тяжело, вы вспоминаете ещё более худшее в прошлой жизни. Для чего? Да чтобы больше разжалобить себя. Так что вы так любите? Вы любите самих себя. Вы это делаете с таким успехом, что начинаете плакать, жалеть ещё больше. Вы начинаете обвинять всех: ``да вот я какой, да вот я какая, - а ты!…''. И так всегда. И маленький хвостик на этой чаще – это всего лишь след. След попытки перевернуть её, сбросить ту чашу, очистить её, чтоб было легче. Это всего лишь пятнышко, это всего лишь царапина на той полированной глади, на той поверхности, которую вы  называете ``жизнь''. Вы хотите исчерпать, исчерпать всё, что в себе. Вы  поднимаете память, которая помнит, и почему-то помнит больше - плохое. Вот, что буква ``Ц''. }
\people{``Ш'', шесть.}
  
\soul{Иначе – так же буква ``Е'', только всё наоборот. Мир перевернулся - в вашем представлении. Вы уже видите мир иными глазами. Чаще, к сожалению, глазами старца. Старца не по уму, а по возрасту. Старца, уставшего …}
(обрыв контакта)
 
\people{… Племя берков?}
\soul{Я шаман Ор, и прошу у вас помощи.}
\people{Ты шаман?}
\soul{Я слышу вас.}
\people{Слышишь? Чем мы помочь можем?}
\soul{Нас разбили берки, и мы должны уходить с нажитых мест.}
\people{Так.}
\soul{Если вы не можете нам помочь убрать их, то хотя бы  укажите место, где мы могли бы пропитать свои семьи.}
\people{Вы только ответьте на наши вопросы. Во-первых, кто такие орки?}
\soul{Берки. Это наши враги, они очень хотят занять наши земли.}
\people{У вас земля благодатная?}
\soul{Да.}
\people{А у них?}
\soul{Они же пришли с гор. В горах нельзя выращивать.}
\people{Понятно, а они умеют  выращивать, или вообще не приспособлены?}
\soul{Я знаю, что они умеют сражаться лучше нас.}
\people{Вы знаете о городе около моря?}
\soul{Да.}
\people{Вы знаете, что вы туда должны идти?}
\soul{Мы никто не знаем где он. Неужели вы нас пустите в мир богов?}
\people{Да.}
\soul{Мы все можем идти?}
\people{Все. Но вы и там будете бороться.}
\soul{Как берки?}
\people{Нет. Почему? Вы от них будете обороняться, они же будут нападать на вас, они от вас не отстанут.}
\soul{Но, вы же всесильны, укажите нам дорогу, но не говорите им.}
\people{Мы всесильны? Да нет, вы преувеличиваете. Скажите, а как вы считаете время своё? Чем? У вас есть понятие  ``год''?}
\soul{У нас есть понятие и отсчёт. Это моя обязанность.}
\people{Какой у вас сейчас год?}
\soul{Две тьмы  и сорок шесть.}
\people{Две тьмы и сорок шесть. Сорок шесть – это дней?}
\soul{Нет. Сорок шесть раз солнце сделало круг.}
\people{А тьма?}
\soul{Тьма…Это  когда все бусины, каждый круг…}
\people{А вы можете все вместе собраться? Весь ваш народ. Вас больше, чем их?}
\soul{Но не будут же дети воевать с нами.}
\people{Дети не будут. Я не о войне говорю. Я говорю о том, чтобы вы вместе молились. Вы знаете, что такое молитва?}
\soul{Это я  должен  разговаривать с богами!}
\people{Да, вы должны с богами разговаривать, но молиться должны все.}
\soul{Все не могут разговаривать, зачем тогда буду нужен я?}
\people{Чтобы руководить людьми, чтобы направлять их правильно.}
\soul{Для этого есть вождь.}
\people{Но в любом случае люди должны быть вместе. Понимаете, если вы вместе не соберётесь, не дадите отпор врагам, то ничего…}
\soul{Как понять ``вместе''?}
\people{Все, кто может держать оружие, чтобы не было разногласий. Вы должны слушать вождя.}
\soul{Вождь ждёт, что скажу я, а что должен сказать я?}
\people{Ты ждёшь, что скажем мы? Что мы можем сказать, если нас нет там, где находишься сейчас ты? Мы не знаем начала и не знаем конца.}
\soul{Тогда, кто вы?}
\people{Мы? Мы для вас уже не существуем.}
\soul{Это мир умерших?}
\people{Для вас - да. Наверное - да.}
\soul{Мне не нужен этот мир. Хотя я хотел бы найти тогда своего отца и прежнего вождя Гага, он знал, что делать. Вы можете мне его найти?}
\people{Нет. А вообще… может быть..,  Неизвестно.., Сейчас подумаю.}
\soul{Из какого племени вы?}
\people{У нас нет племен. Мы русичи.}
\soul{Русичи? Я не встречал такого.}
\people{Оно жило давно, когда тебя ещё не было, и не было твоего отца.}
\soul{Как это может быть? }
\people{Скажи, а что произошло? Это, вообще, Земля, или какая-то другая планета? Ты видишь ночью Луну?}
\soul{Мы не знаем, что это.}
\people{Луна – это, как Солнце, только ночью.}
\soul{Ночью не бывает Солнца.}
\people{Но это не Солнце, это Луна.}
\soul{У нас только одно Солнце.}
\people{А ночью на небе ничего большого нет, такого же, как Солнце?}
\soul{Нет. Ночью мы видим реку и огни, а Солнце уже спряталось от нас.}
\people{Скажи, а большой войны не было между богами?}
\soul{Боги не могут воевать меж собой.}
\people{Ага, а между людьми, когда жил твой отец, война была? Нет? Ты помнишь?}
\soul{Была.}
\people{Большая?}
\soul{Мы всегда воевали с берками. Они всегда хотели нашу землю.}
\people{А кто такие берки? Они такие же, как вы? Так же выглядят?}
\soul{Они жители гор.}
\people{Они темные, относительно вас?}
\soul{Они волосатые.}
\people{Все?}
\soul{Все.}
\people{У них волосы с ног до головы, всё тело покрыто волосами?}
\soul{Да, они похожи на наших собак, только ходят, как мы.}
\people{А они чем воюют?}
\soul{Тем же, чем и мы.}
\people{А чем вы воюете?}
\soul{Дубинками.}
\people{А они почему хотят вашу землю? Они её обрабатывают?}
\soul{Не знаю. Они хотят нашу землю, в горах нельзя сеять.}
\people{Вот из-за чего они хотят! Они умеют сеять, да?}
\soul{Мы не разговариваем с ними, и мы не можем понять, чего хотят они и говорят. Они лают, как наши собаки.}
\people{Может они  вас боятся? Их много?}
\soul{Много.}
\people{Они все вместе к вам приходят и убивают вас? Вы не пытались с ними договориться?}
\soul{Как с ними можно говорить, если собаки на двух ногах!}
\people{Тогда вы можете сделать тактический ход – уйти со своей земли. Потом когда они придут, посмотрят, что брать  нечего, сеять они не умеют, они с неё уйдут, а вы опять придёте.}
\soul{А куда нам идти? Вы можете сказать нам?}
\people{К морю. Там можно ловить рыбу, там есть пропитание. Море может прокормить. Рыба есть ещё, надеюсь?}
\soul{Мы не умеем плавать.}
\people{Возьмите дерево, вместе составьте плот. Много стволов деревьев. У вас есть чем рубить? У вас есть железо? Топоры есть?}
\soul{“Дерево богов'' есть.}
\people{Вот и возьмите это дерево и  сделайте плот. Идите к морю, и живите у моря, вас боги научат жить. Вас много?  Ваш народ большой?}
\soul{Мы уже жили у моря.}
\people{И что?}
\soul{Но боги гневаются и топят нас.}
\people{Как? Каким образом они топят вас? Тогда это не боги!}
\soul{Приходит вода и уносит нас.}
\people{Значит, вы очень близко к морю подходите. Близко нельзя, особенно, когда идут большие волны.}
\soul{А мы не можем  знать, когда бог проснётся, и будет волновать его.}
\people{А вы учитесь. Ведь есть некоторые приметы в природе: птицы замолкают, например, сильный ветер, тишина в природе, все животные как бы замирают. И после этого начинается буря.}
\soul{Когда замирают птицы? И что?}
\people{Тогда, значит, скоро будет буря и надо уходить. А вас боги не учили жить? Вы их видели?}
\soul{Нет, я только разговариваю, но не вижу их. }
\people{Разговариваешь, как с нами?}
\soul{Да.}
\people{А откуда ты знаешь, что есть город, где живут боги?}
\soul{Это я должен рассказывать о богах! Мне же говорил отец, а отцу говорил дед.}
\people{А откуда дед взял, что боги живут около моря?}
\soul{Когда-то был, который разговаривал с богами, и он дал ему других людей. Но он захотел большей власти и тогда бог дал ещё вождя. Потому я должен слушаться вождей. Это делал и мой дед, и отец. Это буду делать я и мои дети.}
\people{Скажи, а у вас только одно поселение ваших людей? Может, есть ещё другие?}
\soul{Есть.}
\people{Ну, вы тогда можете пойти, собраться вместе, объединиться и победить этих берков, или как их. Объединиться вы можете? Вы не воюете между собой?}
\soul{Если они не нападают на них, то воевать нам. А когда нападают на них, то зачем будем воевать мы?}
\people{Тогда, вы должны создать, вместе, войско, то есть, собрать людей, мужчин которые способны воевать. Они будут вас охранять и в любое время смогут защитить вас. Они ничего не должны будут делать, кроме того, как быть воинами. У вас есть воины?}
\soul{Которые только воюют? А кто будет сеять?}
\people{Ну, какая-то часть будет сеять, а какая-то воевать. Если на вас нападают очень часто берки, то вам необходимо создать войско. А у вас женщины не сеют?}
\soul{Нет.}
\people{А чем они занимаются?}
\soul{Они кормят наших детей и растут нас.}
\people{Растят вас? Вам необходимо защищаться, вас берки застают врасплох. Поэтому вам нужно организовываться.}
\soul{Что организовывать?}
\people{Вам нужно организовывать общее дело, защиту против берков. Чтобы на чеку были, разведчиков  высылать ``тех, кто хорошо бегает''. Есть такие?}
\soul{Да.}
\people{А у вас есть животные, на которых вы ездите?}
(контакт прервался)
…. ( выход на контакт Мабу)
\people{Ну…}
\soul{А они меня там закрыли…}
\people{Кто? Монахи?}
\soul{А там нас много, но они без шкур.}
\people{Кто? Монахи?}
\soul{Нет,  монахи говорили, что забирают нас к богам, а они здесь без шкур…}
\people{Кто, они?}
\soul{А ещё они сказали, что время придёт, и я тоже буду такой же, как они без шкур… Я сегодня пойду, я знаю, где у них пещера. А туда уводят всех, я даже, может быть найду себе дочку.}
\people{Кого ты найдёшь?}
\soul{Дочку.}
\people{Свою?}
\soul{Да, которую забрали. Они говорят, что они там, они ждут, когда их заберёт бог.}
\people{Знаешь, ты лучше туда не ходи!}
\soul{Почему?}
\people{А монахи побьют тебя. Ты же сам туда хочешь, а монахи тебя не пускают? Там могут там выколоть глаза, или что-нибудь ещё.}
\soul{Интересно. }
\people{Тебе интересно будет без глаз, когда ничего не будешь видеть?}
\soul{А я  же им ничего не скажу, я тихо…}
\people{Они же всё знают, ты не сможешь тихо!}
\soul{Они не всё знают, потому, что  когда я рассказываю,- а я не все рассказываю,- они не догадываются. Они не всё знают.}
\people{Они могут  догадаться, тогда ты тоже попадёшь туда, только другим способом. Ты, лучше, спроси у них, можно тебе или нельзя.}
\soul{Я знаю, что нельзя.}
\people{А зачем же тогда ты хочешь?}
\soul{Интересно. Я потом вам расскажу.}
\people{А-аа. М}
ы боимся, ты потом не расскажешь.
\soul{Почему?}
\people{Потому что тебя бог заберёт. А там не рассказывают. Там ничего нету, там ты ничего не будешь знать, и разговаривать с нами больше не будешь}
\soul{Почему?}
\people{Потому, что тебя не будет, бог тебя заберёт! Как же мы с тобой разговаривать будем? Ты хочешь с нами говорить?}
\soul{А как  меня могут боги забрать, если меня монахи не дадут?}
\people{Как раз они и дадут, за то, что ты пошёл в пещеру.}
\soul{Ладно, не пойду.}
\people{Молодец! Только нас не обманывай ладно, а то я больше с тобой никогда говорить не буду. У тебя сколько жён сейчас?}
\soul{Семь.}
\people{Семь? Ты старейшина?}
\soul{Да.}
\people{Вот, видишь, тебя монахи столько учили, учили, столько жён тебе дали, а ты хочешь их теперь обмануть? Ты смотри, а то они другого старейшину найдут!}
\soul{Ну, почему  ``обмануть''? Я только посмотрю и убегу. Я даже и вам говорить не буду. Вы согласны?}
\people{А мы узнаем. Мы сами тут, как боги.}
\soul{Не, вы не похожи.}
\people{Почему? А ты видел богов?}
\soul{Нет, богов не видел, но монахи не похожи на вас.}
\people{Почему не похожи? Чем?}
\soul{А я их не вижу, одни шкуры,  а вас видно.}
\people{И какие мы? Мы похожи на тебя? Ты же в зеркало себя видел?}
\soul{Нет.}
\people{Не похожи?}
\soul{Не похожи.}
\people{А чем мы не похожи?}
\soul{Я красивей.}
\people{А почему ты красивее?}
\soul{Не знаю, я больше нравлюсь.}
\people{И жёны у тебя красивее, чем мы, да?}
\soul{Красивее.}
\people{Скажи, а ты с жёнами разговариваешь?}
\soul{Разговариваю.}
\people{О чём?}
\soul{Чего?}
\people{О чём разговариваешь?  Что ты им говоришь?}
\soul{Ну, первую я бью, когда она много спит…}
\people{Это лень, да?}
1-2-3..
(Конец контакта)
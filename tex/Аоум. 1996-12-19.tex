Аоум. глава 38-я 19-12-1996г
Георгий Губин
\people{**}
 
 
Контакт от 19 декабря 1996 года  
\people{**}
(Белимов) …людей и судьбе Земли. Какой 97 год? Насколько судьбоносный? Ведь ожидается, что он каким-то будет особенным  годом. Вы можете это подсказать или опровергнуть?
(Подсознание) - Мы не собираемся быть пророками. 
(Ольга)* Мы сами это придумываем просто, - для себя. У нас каждый год…Заметьте, в последнее время каждый год для нас какой-то… 
\soul{Вы очень-очень  любите, чтобы за вас кто-то решал вашу судьбу. Вы приходите к гадалке игадаете. Вы хотите найти подтверждение своему будущему шагу. Если он не совпадает, то Вы скажите:'' гадалка дурна'', - и будете искать другую. И вы постоянно ищете поддержки со стороны, потому что не верите себе. Вы никогда не верите себе. И даже, когда вы будете кричать: `` Вот самоуверенный человек! Делает всё только по-своему!”– не-ет, он также неуверен, как и Вы. Он просто питается от вас этой уверенностью.Понимаете?  Вы никогда не были уверены в себе. Вы всегда ищите причину извне, но не в себе. Всегда виноват кто-то, но не Вы. И все беды приходят на вас извне. И потому, ранее вы придумали множество Богов, которые ссорятся между собой из-за вас, устраивают космические войны из-за вас. Почему?Почему так?  Почему такая противоположность? Вы не верите себе и,в то же время, вы возвышаете себя столь сильно, что Боги волнуются о вас!  Проходит время, и вы видите, что Боги не очень-то и заняты вами. Ну что же, это плохо конечно, но давайте найдём оправдание и этому. И вы тут же придумали астрологию.}
Обрыв контакта.
(Подсознание) …Существует взаимодействие. Ну, всмотритесь. Всмотритесь, как она действует на вас…Всмотритесь,- астрология, это, всего лишь, адвокат, который защищает ваш характер, каким бы он не был, скверным или хорошим. Понимаете?  Вы нашли виновника вашего характера, ну, а значит и судьбы. Иначе, если пожнёшь характер, значит, пожнёшь и судьбу. Звёзды виноваты. Звезды попутали. Ан нет, попутал вас чёрт. Всё-таки, у вас остались Боги. Но только из множества Богов осталось  два. Дьявол и Бог. Черный и светлый. Так проще, когда есть только два цвета. Когда-то их было множество.  В зарождении Земли, зарождении вашей цивилизации, было множество-множество Богов. И Бог камня, причём чуть ли не у каждого камня свой Бог, и  Бог ручья, Бог реки, Бог облака… Множество Богов. И представляете, как трудно потом из этого количества Богов выбрать того, который был более внимателен к вам? Представьте, каково ваше состояние, если вам нужна помощь Бога, а какого бога, Вы не знаете, того или другого? Тут подойдете к одному богу, к одному камню, помолитесь ему. Потом, ночью, вы проснётесь и  подумаете - ``Нет, этот Бог наверно был занят или ещё что-нибудь''. Вы найдете ему оправдание. Пойдёте, поищите другого Божка. Сложновато, не правда ли? А как быть бедному шаману, чтобы угодить вам? Ему надо, бедному, догадаться именно какому Богу Вы больше доверяете. Сложно, не правда ли?Тогда постепенновы стали убираться в этих Богах. Чтобы было поменьше путаницы. И их становилось всё меньше и меньше, и, наконец, их стало двенадцать. Вроде как полегче. Этим двенадцати вы дали иемена, совпадающие со временами года. Так?
(Ольга Васильева)* Угу.
\soul{Чтобы: ``Ага…Сегодня какое число? А какой сегодня месяц? А! Значит, вот этот божок…”А вы вспомните, что все ваши звёзды когда-то носили, да и носят сейчас, имена богов.}
(Ольга Васильева)* Ну…Ну, да-а…
\soul{Но на всякий случай вы оставили Млечный Путь. На всякий случай, - а вдруг, опять придётся вернуть множество? Но, двенадцать,  это тоже много. Надо поставить вождя и над ними. И как вы думаете, кого выбрали вождём?}
(Ольга Васильева)* Ну, вот… год там  -'' такого-то''. Год – ``секого-то''. Это так наверно?Это так, наверно?  Тоже двенадцать лет…
\soul{О, нет… }
(Ольга Васильева)* Нет?
\soul{Это тоже Бог. Но самый главный бог, в вашем понятии в звёздном плане, это…?  Полярная звезда.}
(Ольга Васильева) * А-а…
\soul{А как-то говорили вам о Полярной звезде, что это двойник вашего Солнца. А вот смотрите, если переместимся, скажем, на Марс, - будет та же самая Полярная звезда или нет?}
(Ольга Васильева)* Да нет, наверно..
. Нет. Ибо, относительно её  будет другая неподвижная звезда. Значит, у марсиан другая полярная звезда, а значит и другие Боги?  Не очень. Разница небольшая.
(Ольга) *Ну, мы же в одной системе солнечной. В принципе, да.
(Подсознание) –Тогда чему же вы удивляетесь, когда вы на Марсе находите пирамиды?
(Ольга) * Да.
(Подсознание) - Пойдемте дальше. Мы пришли к тому, что существует только два бога, осуществляющие ``чёрное'' или ``светлое''. А, теперь, мы пытаемся заговорить о Единстве. О Едином, содержащем и это ``черное'' и это ``белое''. Попытка чего?
(Белимов) * Примирения.
(Подсознание) - Примирения? А вы, сейчас, замечаете примирение? Может быть наоборот - сейчас век раздора?
\people{Раздор во всём…}
\soul{И, простите, одна религия противоречит другой, хотя говорит практически об одном и том же Боге. Почему?}
Обрыв контакта.
(Молодая душа) - Мы проходим, как бы, школу. Как бы, заранее обучаемся, и нас готовят к любым неожиданностям. И мы, хотя и принято почему-то, что дети знают будущее свое, - нет, они знают всего лишь ход, план будущих событий, но не сами события. И вот, встреча ссебе подобными, это просто обмен знаниями. И, бывает, где-то в уголочке перешептываются, потому что каждого учат по-своему. Относительно его будущей линии. А иногда, всё-таки, хочется знать больше.  Авось пригодится. И вот, с кем-то разговариваешь, можешь увидеть только здесь, потому что, может воплотиться через тысячу лет, а не сейчас. И с ним интересно вести  беседу и делиться секретами. И хотя, об этом высшие знают и не разрешают делать это, но и не мешают. Они делают вид, что не замечают. Понимаете?  И вот,  всегда было интересно,- почему? Почему это является запретом и в то же время, никогда за это не был наказан? Хотя это строжайший запрет! А всё строжайшее, всегда почему-то хочется попробовать. И когда приносят камень, которым станешь ты, и этот камень, пройдя туннель жизни и смерти, превратиться в эмбрион, вот этот камень, который содержит всю информацию о тебе, и ты вдруг видишь в нём и то, что взял вне закона. То, что можно сказать  украл у другого. Понимаете? И никто никогда не наказывал за это и даже не упоминал об этом. А мы говорим, как о высших силах, могущественных силах, которые очень плох…
Обрыв контакта.
(Молодая душа) … ты знаешь, что приходит время, и ты должен прийти, то собираются твои лучшие друзья, и совершается печальный ритуал. Он очень печален, очень похож на будущие похороны, которые будут когда-то ещё. И, хотя все знают, что это всего лишь  только временно и через какое-то время всё вернется, но, если твой друг родится только через тысячу лет, и ты никогда его не увидишь, - это очень печально. Когда ты приходишь и становишься тем камнем, то эта печаль остается в тебе. И когда ты приходишь в лоно, и когда ты видишь, что будущей матери, это прибавило волнения, а чаще сожаления. Это неожиданность и, порой, для многих - печаль. И когда ты чувствуешь эту печаль, и когда ты чувствуешь, что идёт речь о твоей судьбе и ты не можешь ничем помочь себе. И когда ты со страхом ждешь, что будешь разорван на части… Помните? Мир не рожденных назван Ракшей. Совместили просто Бога Шиву- разрушителя, и созвездие Рака. И получилось мир Ракши.
Обрыв контакта.(7-10)
(Молодая душа)…я ещё маленький, чтобы мне говорить на ``Вы''.
(Белимов) Хорошо. Мы это учтём. Глеб, скажи, ты помнишь, почему этот строжайший запрет, этот закон…
(Молодая душа) - Я ещё не Глеб.
(Белимов) *А кто? Как?
(Молодая душа) - Пока, я никто. Пока я не ношу имя. Имя будет дано только…
(Ольга) * Когда ты родишься?
(Молодая душа)- Нет. Оно дается раньше.  Оно дается матери. И мать, мечтая о ребенке, уже даёт ему имя. И, хотя отец и мать решают, что имя дали они, всё-таки, нет. Ведь что-то натолкнуло именно на это имя? А бывают имена двойственные. А бывает так, что имя, данное в детстве, потом меняется. А бывают очень противные имена. Родится ребенок, а у него имя очень-очень нехорошее. Когда он становится взрослым, он меняет это имя. Ну почему родители дали ему такое имя, такое плохое?
(Белимов)* Ну,  да. ИндУстрия какая-нибудь, или СталИна.
\soul{Понимаете, ведь имя - это звук, а мы очень чувствительны к звукам. Ведь к нам ближе относятся те слова в Библии. Помните?}
(Ольга) * ``Вначале было слово.  И слово было…''.
(Молодая душа) - И сейчас я не могу содержать никакого имени. Имя будет дано только тогда, когда уже будет начертана моя судьба. А сейчас же  я свободен, и проходит время моего обучения. Я ещё не знаю, где буду я и буду ли вообще. Я не знаю своего имени, не знаю своего будущего. Не знаю ни малейшего. Здесь многие подобны мне. Многие с сожалением узнают свое имя, другие, ждут его с нетерпением. И стараются обмануть и прийти раньше времени. Это бывает очень редко. Очень редко, когда открывается канал, и вот этот ``заяц'', приходит не в свое рождение. Вот тогда ему обеспечена жестокая казнь. Вы называете это абортом. Но чаще, чаще всего– выкидыш.  Потому что закрывается канал, и он, не имея поддержки, погибает. И вот, многие женщины, мечтая о ребенке, открывают эти каналы. Открывают, и обязательно кто-нибудь попадается в него. Но всё это кончается печально. И тогда женщина клянет судьбу,- она может иметь множество, множество, как вы говорите, выкидышей. А для нас это слово одно из самых страшных. У вас  страшное, это самоубийство, а у нас страшное - это. Ибо аборт - мы жертвы убийцы.
(Ольга)* А как же тогда? Почему женщина не должна мечтать о ребенке?
(Молодая душа)- Нет, мечтать-то должно. Но, поймите, иногда, вы хотите в мечтах больше чем можете.
(Ольга) * Ведь, практически, каждая женщина может иметь детей.
(Молодая душа) - А в прошлой жизни? Что она сделала в прошлой жизни? Она отказалась от детей, она убила их. И вы хотите, чтобы все повторялось снова? И что же? Будет повторение. Будет. И самым жестоким будет не для женщины, а для ребёнка, ибо он обманулся, но не женщина.
(Ольга)* А вот, всё-таки, почему так много абортов делают женщины?
(Молодая душа)- Почему? Потому что вы больше хотите телом и всё! А ваши тела умеют открывать. 
(Ольга Васильева)* Умеют?
\soul{Умеют. Для того и созданы вы, для того и создано у вас всё, чтобы открыть эти каналы, чтобы продолжить жизнь. Открылся канал и был пойман ребенок в ваши сети, но вы же не хотели его!Не хотели! Вы просто играли в любовь…}
Сбрыв контакта
(Молодая душа)-  Когда мы проходим обучение, то мы, конечно, проходим все жизни заново. Понимаете, мы обходимся без имён. Даже когда смотрим, как вы говорите, ``кино''  - имён нет. Если хотите, - немое кино. Но это нет, это совершенно не немое кино,  потому что, мы обучаемся чувствам, а слова не нужны. Потому что слова, чаще всего, несут ложь. И очень печально, что потом, когда-то и я, получив имя, буду говорить слова, и они тоже будут нести ложь. А пока, мы проходим обучение и учимся не лгать, или, хотя бы замечать ложь. И вот, когда мы научимся тому, мы родимся, но имя будет дано уже нами. 
(Ольга)*  А, вы сами выбираете, да?
(девушка)* Вы сами выбираете?
(Молодая душа) - Ну, это ещё наверно не скоро. А может быть, и да….
(Ольга Васильева)* А кто вас учит?
\soul{Мы… }
(девушка)* Вы… Ты сам себя учишь?
\soul{Мы созерцаем Божественное и находимся в этом мире, и потому, у нас нет учителей. Каждый момент времени, хотя у нас и нет времени, является учителем. Мы учимся, наблюдая за жизнью, и тогда каждый из вас - уже учитель. Но мы воспринимаем вас не так, как вы. Мы не слышим вас, не слышим вас словами, не видим вас. Но мы…}
(девушка) * Чувствуете?
(Молодая душа) - Мы живём в ваших чувствах. И вы являетесь нашими учителями. А когда нам кто-нибудь понравится, мы пытаемся подсказывать, но нас не всегда слышат. Иногда дают имя ``интуиция''. И если этот человек, - будущий человек,  трижды услышит о себе ``интуиция'', то, уже рождаясь, он будет обладать этой интуицией, потому что ему было дано имя вами, учителями.
(Ольга) * Значит…
(Молодая душа) - Нет, имя он будет носить земное, но он будет обладать повышенными способностями, которые вы называете… 
(Ольга Васильева)* Как называем?
\soul{Вы называете нас ``третьим глазом''.}
(Ольга) Ах, вон как… Но это не так?
(Молодая душа) - Нет, конечно же, не только нас, потому что это бывает всё-таки редко, когда приходишь к будущим родителям, задолго до своего рождения или даже не в том порядке. Бывает, что учителем является собственная мать, но рожать она будет ученика всего лишь только через, допустим, тысячу лет. 
(Ольга и девушка)* У-у… Ничего себе… 
\soul{За эту тысячу лет, мать, конечно же забудет  учителя, себя… и, естественно, забудет ученика. Ученик же, если не воплотится нигде, то он будет помнить…}
(Ольга и девушка)* Говори! Говори!
Срыв контакта
(Ольга) * Я, говорит, мальчиком тебя назвал…
(Молодая душа) …не знаю. Ну, наверно, теперь знаю.
(Ольга) * Вот, уже теперь знаешь.
(Молодая душа) - Я теперь не знаю, что будет. Вы говорите, что я буду мальчик по имени Глеб. Но тогда, получается, что я могу узнать свою судьбу. А это значит, что теперь я знаю своё имя и здесь я теперь уже не смогу остаться. Значит, занятия мои уже кончились, потому что Вы дали мне имя. У нас это не бывает.
(Ольга) * Ну, я не знаю, как это назвать, мы просто беседовали с мальчиком…
(Молодая душа) - Я не могу скрыть от всех, что я уже знаю свое имя. Но я вот не знаю, относительно вас, это в прошлое или в будущее? Если это в прошлом, то значит, я просто повторю то, что было. А если в будущем… Но в любом случае, вы дали мне имя. И даже сказали, кто я буду. У нас нельзя этого делать.
(Ольга) * Мы, наверно, не правы. Прости нас, пожалуйста. Мы как-то…
(Молодая душа) -  Я не держу на вас зла, потому что рано или поздно я получил бы имя. Просто необычно -  вы дали его. Обычно не вы это делаете…
(Ольга) * А кто?
(Молодая душа) - К нам просто приходят. Приносят камень, на камне написано уже всё. Лишь несколько строк должен написать я.
(девушка) * А каких строк? Желание загадать? Записать? Да? Желание?
(Молодая душа) - Да, желание. Но я не мог этого знать. Но я уже знаю.
(Ольга)* А камень…ты видел эти камни у других?
(Молодая душа) - Нет. Понимаете, я не мог знать про камень. Я не мог знать про желание. Это всё сказали сейчас Вы, а значит, я должен сейчас уже писать желание…
Срыв. Идет счет.
(девушка)*Это я виновата.
(Ольга) * Да нет!)
(Подсознание) - Что было, то было. И что? Что вы хотите далее?
(Ольга) * Мы даже не знаем, что мы наделали…
(Подсознание) Тогда,отдохните.
Конец контакта.
Аоум. глава 24-я 13-02-1995г
Георгий Губин
\people{**}
  
 Начало
\soul{1824й год. Ему в наследство - старинная книга. }
\people{Где? }
\soul{35й год. Ища пропитание, идёт на обман. Как бы переводит страницы той рукописи и называет её ``Книга Судеб''. 95й год. Книга завершена. Она неоднократно переделывалась,  дважды горела. Сын, хороня отца, дал клятву сохранить книгу. Три друга изучают книгу. Есть и женщина, желающая одного из них и презирающая их занятие.  }
\people{Скажите. Вы можете ответить на такой вопрос? Вот у меня при пробуждении, не так давно, я почувствовал на своей голове руку. Я её пощупал и потом, опять ушёл в сон, и сзади кто-то стоял. Мужского пола. Вы не можете сказать, это домовой был, или кто-то другой из какого-то мира?}
\soul{Мы не должны отвечать на подобные вопросы, которые касаются вас. Вас, лично. Если хотите, мы можем сказать так: это были вы.}
\people{Ага, спасибо, хоть за это. При нашей прошлой беседе переводчик говорил, что провели рукой. Но это сделал не я. Я так понял…}
\soul{Это, тоже был он.}
\people{Это, тоже был он. Хорошо. Я тут послушал кассету, наконец-таки. Последнюю. Насчёт невидимого. Я, действительно, в этой жизни думал насчёт этого, как это может происходить. Скажите, мои мысли насколько  верны?}
\soul{Вы, похоже, не внимательно слушали кассету. Вы так говорите, что вы не присутствовали на контакте. Мы говорили вам, о прошлом. Вы не внимательны. Далее, о каких мыслях говорите вы? О каких? Если вы только издалека глянули и пошли далее, не поняв и не заметив, только запомнив, что вы куда-то глядели. А вы говорите уже о мыслях. }
\people{Наверно. А вы не можете сказать, как переводчик соорудил, ну, не один он,  систему одну, где пропадают всякие энергии, и лучи выпрямляются, и модуляции пропадают?}
\soul{Вы создали пустоту. Вы должны её заполнить. Вы же, только создали. Далее. Как вы можете отличить переменный ток от постоянного?}
\people{Ну, только по времени.}
\soul{Хорошо. Скажите, постоянный ток, что это, в вашем понятии?}
\people{Движение электронов. Ну, шум-то есть какой-то, наверное, у переменной составляющей, присутствует.}
\soul{Хорошо, давайте говорить, о времени.}
\people{Постоянный ток, это… количество ампер. Это, сила тока. Разность потенциалов между двумя электродами.}
\soul{Тогда, простите, в чём же разница между переменным и постоянным током?}
\people{Переменный меняется по времени.}
\soul{Значит, по-вашему понятию, по времени не меняется постоянный ток?}
\people{Постоянный ток не меняется по времени. Может, меняется в небольших пределах, за счёт шума. }
\soul{А как же вы тогда существуете? Если, в вашем понятии, не меняется во времени?  Все науки ваши связаны со временем, а вы говорите ``не меняется''.  Вы можете назвать, что вы называете ``постоянным'' и ``переменным'' током?}
\people{Ну, постоянный ток не имеет собственной частоты, т.е. колебаний за единицу времени, а переменный, естественно, имеет.}
\soul{Давайте так, переменный ток – вы можете… }
 
\people{Амплитуда меняется…}
\soul{И всего лишь?}
\people{Ну, и сила тока.}
\soul{В чём разница, тогда, между переменным и пульсирующим?}
\people{Пульсирующий - одного знака идёт. }
\soul{Хорошо. А что вы называете переменным?}
\people{Меняются значения.}
\soul{Как вы говорите, ``знака''.}
\people{Ну, знака.}
\soul{Пойдёмте далее. Вы получили однополярность. И вы говорите, что обладает частотой. Тогда, скажите, имея в существующем вами, приборе во-времени, когда прибор покажет вам пульсирующее напряжение постоянным и, даже, переменным?}
\people{После диодного моста.}
\soul{Зачем же?  С какой частотой? Давайте тогда уж прямо - с какой частотой, c каким временем приборы должны переменное напряжение зафиксировать постоянным? }
\people{За период, гораздо меньший, чем период переменного тока, чтобы ``схватить'' значение.}
\soul{Когда ваши приборы уже не смогут зарегистрировать. Что будет им мешать?}
\people{Скорость, наверное.}
\soul{В вашем понятии, время.}
\people{Время. }
\soul{Так почему же тогда вы говорите, о постоянном? Мы говорим вам, о пустоте, и вы не можете совместить понятие ``пустоты'' и ``постоянный ток''. В вашем понятии, постоянный ток – это пустота частоты.}
\people{В нашем?  Да.}
\soul{По вашим приборам. Согласитесь, что даже ваш курс физики прекрасно знает, что нет ничего постоянного. И вы говорите ``электрон'', но это же - дискретность. Вы согласны?}
\people{Согласен, согласен.}
\soul{Так почему же вы тогда называете ``постоянным''?}
\people{Ну, электроны бегут со скоростью 1,9 см/сек. Вычеслено уже.}
\soul{Хорошо,}
\people{А ток двигается гораздо более, скажем так.}
\soul{Волновой.}
\people{Ну, да.}
\soul{Как вы можете объяснить движение электронов и, в то же время, сигнал, передаваемый ими, обладает скоростью света?}
\people{Ну, электроны близко расположены, ударяют друг об друга, за счёт передачи колебаний.}
\soul{Колебаний. Почему же вы тогда это называете постоянным током?}
\people{Угу. Вы правы, наверное.}
\soul{Нет…То, что делает сейчас ваша наука - мы не правы. Пока вы не переступите и не отбросите понятия. Вы слишком много разъединили. Вы говорите, о волновой среде, и тут же подразумеваете, о постоянном. Они не совместимы. Но вы умудряетесь совмещать, исключая или то, или это. Потому и не можете взять вместе. Вспомните теорию единого поля.}
\people{``Вспомните''! Если б я её знал!}
\soul{Вспомните. Мы не говорим вам, что вы её знали. Вы должны, примерно, догадываться, что это такое.}
\people{Ну, приблизительно.}
\soul{Примерно, приблизительно. И поверьте, ребёнок знает более вас, ибо он не отягощён  научными понятиями, не отягощён границами возможного и нет. Не отягощён терминами, которые тормозят вас, и очень сильно. И мы когда-то говорили вам об этом. Слова ваши  - губят вас. Ибо слова ваши имеют уже – имеют уже(ограничения в значении) и подразумевают. Потому,  и истина высказанная - есть  ложь.  Спрашивайте далее.}
\people{Скажите, насчёт детей уж завели разговор, да? Но ведь если ребёнок не отягощён ни понятиями, ни запретами никакими, т.е. он фактически может всё? Так?}
\soul{Нет. Вы забыли, о себе. Вы забыли, о взрослых, окружающих его, о среде, что держит его. Вы же говорите: ``аура, мысли  - вещественны''. Но, почему же вы тогда не можете понять, что это всё вещественное может задержать любого!}
\people{Ясно. Т.е. по-вашему, выходит, если отгородить ребёнка на необитаемом острове…}
\soul{И что ж? Вы можете найти здесь необитаемый остров? Что вы имеете в понятии ``биополе человека''?}
\people{Ну, более тонкая энергия, материя, тепло, плюс все излучения, которые он излучает.}
\soul{И каков радиус его действия?}
\people{Радиус, наверное, до бесконечности, только сила уменьшается, скажем так, что ли.}
\soul{Сила уменьшается?}
\people{Или тоньше становится, т.е. на более грубое, не воздействует. Ну, скажем, так.}
\soul{Назовите  радиус, в вашем понятии.}
\people{Вселенная.}
\soul{Вселенная.  Говорите  о  силах, что уменьшается с расстоянием. Значит, тогда  вы можете говорить, о радиусе, где вы ещё можете заметить биополе?}
\people{Понятно. }
\soul{Назовите. Это же одна из ваших мер вашего могущества!  Вы говорите:``у него столько и столько-то''!}
\people{Ну, считается, что 10-12 метров – уже много. Биополе сильное. }
\soul{И чем вы измеряете силу? Чем? Давайте так,-  вы возьмёте грубейший прибор, попытаетесь им что-то измерить. Кто-то иной подойдёт с более чувствительным прибором, и он скажет, что здесь - более, и вы будете спорить. Так кто будет из вас прав? Почему к вам приходят и говорят: ``у вас 10-12, а у него 8-7'', и вы верите тому? Но вы забываете, о чувствительности.}
\people{Да. Верно.}
\soul{Если вы имеете, в вашем понятии, только в вашем понятии, - пусть у вас будет множество метров, но если нет резонанса, то вы можете сказать ``у него и нет, этого поля''.  А для многих, биополе – это чушь. Они тоже правы. Они его не видят и не чувствуют. А другие доказывают, что видят и даже могут указать размер. Кто из них прав?}
\people{Ясно. Каждый относительно самого себя прав.}
\soul{О какой силе говорите вы? Что вы подразумеваете под силой, говоря о биополе? Вы можете ответить на этот вопрос?}
\people{Да, понятие относительно, получается.}
\soul{Относительно? А вы уже ставите ярлык.}
\people{Согласен. Ну и что же, выхода нет никакого, получается? То есть, мы своими мыслями губим в зародыше всё, так сказать, действительно истинное?}
\soul{Прекрасно вы рассуждаете: ``Ребёнок только родился и, если он не умеет ходить, можно сказать, он и не научится. Он ползает на четвереньках. Это губит его. Значит, он никогда не станет на 2 ноги.''}
\people{Но были случаи, когда в стае волков, девочку, её потом поставили на ноги, она ползала на четвереньках, выла по-собачьи и она долго не протянула в городе.}
\soul{О чём говорите? О чём? И что задаёт вопрос ваш?}
\people{Ну, вы сказали насчёт ребёнка, что он ползает. До 12 лет девочка ползала.}
\soul{Какой вопрос задавали вы?}
\people{Я задавал, о наших мыслях, о том, что они губят в зародыше, получается.}
\soul{Что ответили мы вам? И как вы приняли ответ?}
\people{Ну, приняли, возможно, неправильно.}
\soul{Мы, как-то говорили вам, что нельзя осуждать человека за то, что он, просто, другой расы. Вы помните?}
\people{Помним.}
\soul{Вы же, кроме того, умудрились разделить и на остальное. У вас есть и первые, и вторые, и третьи и далее, и далее. Вы помните тот разговор?}
\people{Угу. Это - насчёт развития чакр…способностей человека, да?}
\soul{Чакры? Хорошо, давайте скажем так: по вашим понятиям - дикий человек. Рассуждая  вами,- у него не развиты чакры и он ``низок''. }
\people{Ну, он дик, может, но не низок. }
\soul{В вашем понятии - низок. Это вы так говорите.}
\people{Хорошо.}
\soul{Далее, вы говорите, что он низок, ибо не развиты чакры. А вы уверены в том? Почему вы решили, что чем ниже, тем меньше развиты чакры? Как вы связываете первые, вторые, третьи и первые, вторые, третьи чакры? Да можно быть диким и иметь развитыми чакры все. А можно быть и шестым, но пустым. Вы можете это совместить? Нет. Потому, что вы придумали одну меру. Меру по чакрам. Тогда, объясните - дьявол, - сколько ж развито у него чакр? Он же не совершенен?}
\people{Ну, не знаю. Наверное, одна-две.}
\soul{Да? Как же тогда он борется с вами и довольно успешно, имея всего одну-две чакры? Как вы можете себе представить: придут люди ``шестые'' и будут сильны, и будет борьба со злом? Что, шесть чакр будут воевать против одной-двух?!}
\people{Переводчик говорил, что у него с детства сидели цифры в голове: 17, 23. Это что за число такое? Какую роль оно играет в его жизни? И вообще, подсказка это или что?}
\soul{Это решать ему.}
\people{Ясно.}
\soul{Это может зависеть от него, будет ли это подсказкой или пустым звуком, или что-то ещё. Ему искать. Потому и дано. А если мы придём и укажем путь вам - мы будем только врагами вам, и не более. Многие приходят к вам и помогают. И вы говорите: ``он добрый'' и далее. И не замечаете, что вас просто хотят отучить ``ходить''.}
\people{Вот, у меня сны такие бывают, пророческие что ли. Личного плана. Сбываются. Ну, сам смысл их, он сбывается с течением времени.}
\soul{Так, может быть вы создаёте эти сны? А может быть вы -любитель подбирать факты, удовлетворяющие вас? Может быть, это оправдание, выдуманное вами? }
\people{По крайней мере, не телесное ``я'' это придумало.}
\soul{А как разделяете вы ``телесное'' и ``не телесное'' и, при этом, будучи находясь сейчас в сознании, и отвечая и спрашивая сознанием? И вы говорите, о не телесном? Мы говорили вам, о том, что вы любите столь искусно лгать, что вы уже сами забываете, где правда, а где истина. И многое выдумано вами. Вашем сознанием, которое хочет обмануть вас, чтоб вы дюже уж не старались уйти от неё и не сделать вопреки ей. Поймите, чем тоньше ложь, тем она правдивей для вас.}
\people{И насчёт гаданий разных можно так же сказать? Да?}
\soul{Вы ,чаще, лжёте. Вряд ли вы когда говорите правду. Говорите одно, и считаете, что вы правдивы. Вы лжёте, вы прекрасно это осознаёте. Вы говорите: ``не нашёл слов'' и далее, и далее. }
\people{Скажите, ну, откуда вот, допустим, когда знаешь человека день-два-три, снится сон, что, вот так и так будет или приходит чувство, что будет так-то и так-то, и это сбывается, хотя, вроде всё делаешь, чтобы этого не было?}
\soul{Как вам объяснить?  Бывают такие сны, про истинные вещи. А есть и те, что создаются сознанием вашим. Оно создаёт игру, в которую вы играете, считая, что вы играете по вашим правилам. Или даже не замечая, что вы играете. И выполняете её прихоти, не свои. Далее, … (теряется)}
\people{Часто замечаю, что вы говорите что-то о нас, чего на самом деле такого не было, скажем так. По крайней мере в сознании. А то, что за сознанием находится, мы просто ещё не осознаём. Так что вы как-то корректируйте это дело и не обижайтесь, в случае чего, если мы не так поняли или что-то ещё.}
\soul{Чаще всего, вы не понимаете. И мы когда-то говорили вам, что вы принимаете на веру то, что вам больше нравится, а всё остальное – ложно.  Неужели вы думаете, что сжигающие на кострах неверующих, сомневаются в Боге? За веру в него и сжигали. И считали себя правыми. Прошло то время, теперь они обвиняются вами, и вы считаете, что вы правы. Придут другие, и обвинят вас, что вы были. И так будет всегда. И всегда вы будете считать, что вы правы. Ибо вы, чаще, подвластны ``времени''. (взглядам на правду в том времени. Прим.)}
\people{Скажите, как вот отличить сны… Кстати, кто их даёт, так сказать, ``вещие''. Откуда они идут?}
\soul{Здесь всё верно. Любой ответ будет верным, потому что они различны. Да, есть сны, что даёт ваше сознание. Есть сны, что называете мечтами. Как правило, мечта сознания и мечта истинного вашего ``Я'' совершенно различны, хотя бы тем, что сознание мечтает в связи с материальностью. Иногда, оно хитрит, хитрит и создаёт вам сны, в которых вы духовно развиты сильн, и вы уходите в другие миры, чтобы утешить вас, чтобы успокоить вас, обмануть, приостановить. Страшны эти сны тем, что они правдивы. Правдивы для вас. И трудно вам определить, ложны они или нет. Ибо, ложь никогда не бывает голой, она основывается на чём-то, и просто искажает правду. Есть сны, что созданы вашими мечтами. Как правило, вы их не помните. Не помните потому, что вы не можете мечту свою произнести словами. Истинно ваши мечты не произносятся словами, никогда не произносятся вслух, потому что выи не найдёте слов. Потому, что они рождены чувствами. Потому и снов тех помнить не будете. Но играют роль больше, чем сны явные.}
\people{Вы нам довольно-таки много сказали насчёт снов. Скажите, вы, вроде как, о духовном собирались говорить, что технического ничего не даёте, а сами вроде как нас подталкиваете… технически, техническое. Ну, это понятно, что для сознания даёте. По-другому мы не можем пока с вами говорить, к сожалению. Не понятно, то вы говорили - нельзя насчёт техники, а тут, вдруг, – целый час…}
\soul{Вы просите,…}
\people{А. Ясно.}
\soul{… и мы отвечаем. Мы могли бы отвечать на любые заданные вами вопросы. Мы могли бы это говорить прямо. }
 (плохое качество)
\people{… Скажем так, о ваших целях.}
\soul{У нас нет целей.}
\soul{Мы разговаривали с вами. В вашем понятии, у нас нет цели. И даже в нашем понятии - их нет. В том и счастье ваше и наше, что мы не можем и не хотим просчитать все шаги будущего, ибо это было бы слишком скучно, страшно, даже если б вы знали, что будет счастливый конец. Когда бы пришёл этот конец, вы бы уже не были счастливы. }
\people{Да.}
\soul{Вы бы знали, что вы, просто, прошли положенный отрезок пути. Потому и не говорим вам, о будущем. Потому и не хотим видеть сами.}
\people{Ну, да. В принципе, было бы скучно жить, если всё знать наперёд. Скажите ещё такое… Ну, вот вы говорите, что ``целей нету, и просто ведём диалог'', но, с другой стороны, вы, на прямо поставленные технические вопросы, говорите, что ``схему мы вам не дадим''. Я не к тому, что схему прошу, я к тому, что если у вас нет цели, в принципе, вы и сказали бы нам, потому что - какая вам разница… Получается, что цель, всё-таки, есть? Чтоб мы сами думали?}
\soul{Хорошо, какую цель вы ищете, разговаривая с другом? У вас есть цель? Придя к любимой и разговаривая с ней, у вас есть цель? Вы можете назвать её? Если вы сможете назвать цель, простите, или он вам не друг, или не любимая. Или слишком вы циничны, что, в принципе, одно и то же.  Далее, мы вам говорили, о рабах. Если мы будем делать все шаги за вас, принесём вам всё готовое,а что будете делать вы? Это один из способов погубить вас. Придут иные, в вашем понятии, ``друзья'', и будут делать за вас всё. Тогда, - кто вы будете? Кто? Растение? Или аквариумные рыбки, за которыми ухаживают?}
\people{Понятно. Ну, а вообще-то, вы могли бы просто сказать, допустим, для начала, дать толчок человеку, чтобы в зависимости от своего развития, он пошёл дальше - нарисовать кусок схемы, которую он видел во сне? Плату. }
\soul{Мы уже дали.}
\people{Да, дали, но он, сознанием, не помнит. Когда просыпается человек, мы мало что вспоминаем, а такое, как схемы, текст – помнится ассоциативно.}
\soul{Хорошо. Ну, давайте, мы за вас научимся ходить!  Мы за вас будем всё помнить, мы за вас будем жить. Всю жизнь проживём за вас. И все ваши беды и счастья возьмём на себя. Если вы не помните – это ваша вина. Значит, вы ещё не готовы. Вы же знаете множество снов, которые все помните. Помните как сейчас. И все до мелочей. Значит, вы были готовы к нему. Есть множество снов, о которых вы только имеете понятие, что они были, но даже не можете вспомнить, о чём. Но почему же мы должны за вас решать и за вас всё делать? А кто же тогда вы? }
\people{Ну, вобщем, смысл сводится к тому, что ``делайте всё сами''. Сами думайте, сами…}
\soul{Нет, вы применяете слишком прямо.}
\people{Прямо… А ``криво'' - как?}
\soul{Вы и так делаете всё сами. Любой совет, кто бы, и что бы вам ни сказали, вы всё равно делаете сами. Вы, просто, понимаете и слышите только тот совет, который хотите услышать.}
\people{Да.}
\soul{А другой, вы примете за ложь, или за шутку, или за что-то ещё. И, в любом случае, вы будете делать сами. Но всегда, у вас будет повод обвинить кого-то, давшего совет, которого вы послушали, хотя, вы и ждали его и готовились к нему. Но виновным будет он, а не вы. Потому, что очень тяжело обвинить самого себя. }
\people{Скажите, вы как-то говорили, что человечество, это один, и вообще, вся Земля, похоже, это одно… С растениями, с морями, со всеми нашими ощущаемыми чувствами вокруг. Так вы видите нас?}
\soul{Мы не видим вас, в вашем понятии. Ибо вы – физика.}
\people{Ну, вы поняли смысл - ощущаете.}
\soul{Но мы говорили вам, что мы никогда не покидали вас. Вы – физика. Но вы состоите, ваше истинное ``Я'', - ваше истинное, что называется ``человеком'',- не имеет физики. И в этом смысле, мы - это вы, и вы – это мы, потому что, мы состоим из единого одного истинного ``Я''. А всё остальное, всего лишь, шкуры и одежды, которые меняются, меняются столь быстро, в нашем понятии, да и в вашем истинном  тоже, что даже и забываются.}
\people{Скажите, вы говорили, что у вас есть ``высшие'', которые вы не пронизываете. А те-то, наверно, пронизывают и вас и ещё более…?}
\soul{Да, и мы думаем, что это бесконечно. Мы, тоже ищем, также, как и вы. В нашем понятии, у нас есть своя физика. Своя. Когда вы дойдёте до наших, вы скажете ``мы избавились от физики'', и поймете, что вы теперь получили другую. Что вы, просто перешли на другую ступень. И надо идти дальше и дальше. И мы тоже не знаем ни начала, ни конца. Начало мы не хотим знать.}
\people{А могли бы?}
\soul{И вы не хотите знать. И причина проста, и вы должны осознать её. Осознать, почему вы не хотите знать начало. Ибо вы любите тайны. Если быть точнее, только тайны держат вас, только тайны заставляют вас двигаться. И когда вы делаете последний шаг к раскрытию тайны и знаете, что она последняя и единственная - вы никогда не сделаете этот шаг. Вы уйдёте от него. Уйдёте дальше, чтоб потом, когда-то, вернуться. Но никогда не захотите узнать Первую Истину. И Конец. Даже если бы вы желали знать конец того пути, вы не узнаете его, ибо он бесконечен. Представьте, что будет за этим концом? Ничто? Этого не возможно.  Не возможно, чтобы остановилось развитие. Жизнь – это бег.}
\people{Просто, бег? В никуда?}
\soul{Разве? Подумайте. }
\people{Скажите, а вы проходили нашу стадию?}
\soul{А как вы думаете?}
\people{Не знаю.}
\soul{Вы знаете, что многие, в вашем понятии, что ``ниже'' вас, контактируют с вами. И вы даёте множество ответов. Они спрашивают вас, и они хотят достичь ваших вершин. Те же -  тоже кому-то отвечают. И далее -далее-далее. Мы отвечаем вам, но спрашиваем сами, у других. Тех, тоже спрашивают, и далее-далее-далее. Мы же говорили вам о бесконечности совершенства.}
\people{Совершенство идёт по пути ``утончения'' материи?}
\soul{Отбросьте понятие ``материи''. Отбросьте. Почему вы связываете материю и духовное?  Тогда, зачем было бы создавать легенду, что бог создал по подобию своему? Что, он тоже носил ваши одежды? Он тоже имел плоть, которая имеет свойство болеть? Подумайте, что говорите вы! Далее. Материя вам мешает до тех пор, пока вы зависите от неё. Мы пришли в мир ваш и разговариваем с вами. Но мы же не зависим от вашей физики? Нет. Как и вы, отвечающие на более ``низших'',  не зависите от их физики.}
\people{Переводчик видел на контакте, как кто-то открыл дверь. Не кто-то, а ``он второй'', подошёл, сел и вот, беседует с ним. Как это - он здесь, и он – там?  Ну, это же 2 разных получается? Там другая энергетика, скажем, другой разум, может, свои понятия, о жизни там, о чём-то ещё. А у него - свои. Как он видел себя? Я не пойму что-то.}
\soul{Когда-то мы говорили о множестве вас, и вы тогда говорили, что прекрасно всё поняли. Теперь вы не поняли. Послушайте далее, это не обвинение. Нет. Вы всегда поверхностны. Вы говорите ``понято'', потом оказывается, что вы не поняли ничего. Когда нет вопросов, значит ничего и не было понято. Вас – множество, вас самих – множество. Вы же сами себя разделили на множество тел. Но почему бы тогда вашему грубому физическому не спать, а старому не гулять, и виртуальному сделать ещё что-то, и так далее-далее. Неужели вы думаете, что они все вместе, и не могут разойтись?}
\people{Это что ж получается? ``Проходной двор'' что ли?}
\soul{Разве? Хорошо, простите, когда вы ложитесь спать,  вы - проходной двор? Что гуляет по мирам иным? Что? Ваша ``физика''?}
\people{Нет, не наша ``физика''.}
\soul{…имея ваше грубое тело? Одно из других тел гуляет. Одно из других тел может любить, может ненавидеть и многое, многое. Оно может жить. Жить так же, как и вы. Но, оно живёт - совершенно по-другому, ибо у него совершенно другое сознание. Сознание же ваше, которым вы увидите сон, просто плохой переводчик. Поймите, если ваше астральное тело ушло гулять на другую планету, и вы увидели это во сне, это не значит, что эта планета выглядит именно так. Просто сознание ваше перевело, как могло. Потому, что в каждом из тел ваших – своё сознание. Если быть точнее – свой уровень восприятия этого сознания. Всё же, говоря о единстве, мы должны сказать вам, что сознание ваше - тоже едино. Вы только разделяете себя на людей. Сознание же - одно. Вы, каждый из вас – просто приёмник, причём, очень плохо-плохо настроенный и обладающий столь узким восприятием, и….(сбой в контакте)}
 1-2
\people{Скажите, а как тогда быть единым, если каждый думает, каждая часть тебя думает по-разному? Может поэтому в голове хаос из мыслей и не всегда можно настроиться на что-то определённое? }
\soul{Если вы будете пытаться сделать это сознанием, только сознанием, то вы ничего не добьётесь. Поймите, через сознание вы можете для посторонних, не посвященного, совершать чудеса. Да, вы можете перемещаться, владеть временем и далее, далее. Но вы будете, всего-лишь, издеваться над самим собой, убивая самого себя. Рано или поздно вы поймёте, что вы это делали без чувств. Или чувства были ложными, созданы сознанием, чтобы успокоить истинное.  Далее. Мы говорили, о вас, как о приёмниках. Да, каждый из вас – приёмник. Но он же должен совершенствоваться? Ребёнок, только рождающийся, не может говорить. Он не понимает, что видит, но он, всё же, видит. А вы можете представить, чтоб, не зная слов, вы будете видеть и думать? Вы можете такое представить? Как вы будете думать без слов?}
\people{Я чего-то не задумывался над этим…}
\soul{А вы представьте, вы сможете это сделать? Нет. Хотя, вы это делаете всегда. Но, сознанием, вы думаете только словами. Если быть точнее – мысль рождается где-то в глубине и, сильно искажённая, приходит к вам в сознание. То самое, что вы называете сознанием, поверхностью. И это сознание уже даёт команду мышцам. Мышцам языка, гортани. И это, чаще всего, вы называете сознанием. Но разве это так? Вы не ищете корни, вы не задумываетесь о них. Не то, что шевелит язык ваш, а ЧТО заставило шевелить его?}
\people{Мы, получается, как марионетки. Кто-то дёргает за верёвочки, куклы думают что это они что-то делают.}
\soul{Вот и подумайте. Вы должны совершенствоваться. Вы должны это делать. Но, поймите, чисто сознанием, шевелением языка – это ничего не даст. И вы должны заглядывать глубже в корень. В то, что создаёт эти мысли. В то, что заставляет работать ваше сознание, шевелить тем же языком. И вы должны думать, и вы должны чувствовать и развивать те чувства.  Чаще всего, вам даётся, в вашем понятии, талант, что у нас называется ``даром'', и этот дар губится в вас, из-за лени вашей. Вы рассуждаете так: ``Бог дал мне способности – вот и прекрасно, что еще?''.  И вы будете эксплуатировать эти способности, но не развивать. Для многих ``непосвященных'' вы будете ``велики''. А что, в чём здесь величие? В том, что вы остановились? В том, что вы этот дар губите на что-то материальное и подобное? И вы называете: ``надо прожить'', ``а как же быть'' и далее, далее. Вы находите множество причин, чтоб обмануть себя. Уж это вы умеете!}
\people{Ну, всё-таки, жить-то надо. Вы сказали, что тело, тоже нельзя ``”казнить''. Оно ж тоже кушать хочет.}
\soul{Чаще всего, ваше тело хочет кушать более, чем ваш мозг. Это почему же? Почему вы не можете совместить? Почему вы берёте крайности? Если развиваться морально - в вашем понятии, ``духовно"- значит, будешь голодным. А чтобы быть сытым, значит, будешь духовно голодным, некогда будет говорить о душе. ``Если бы у меня был миллион, у меня было бы больше времени подумать о душе''. Но вы подумайте, что это говорит? Разум ваш? Лень и жадность. И не больше того!}
 1-2-3-4
 (Конец)
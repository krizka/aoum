Аоум. глава 28-06-96
Георгий Губин
28-06-1996 – 1
(начало контакта)
\soul{И лишь только тогда можете менять его.}
\people{(Гера) Угу. То есть, пока руки не поднялись, нельзя давать дальше, да? (о счёте при выводе на контакт.прим)}
\soul{Не только. Вы должны увидеть изменения подтверждающие, что счёт был принят вами.}
\people{(Гера) Угу. Спасибо.}
\people{(Белимов) А каким образом?}
\people{(Ольга) Ну, вот, хотя бы жест был.}
\people{(Б) Жестом? Хорошо бы всю инструкцию получить об этом. А то загадками, догадками. Ошибок много делаем. }
\people{(О) Только в практике приобретается. По-моему так, да? }
\soul{Вы спешите. В том ваша беда.}
\people{(Белимов) Но вы можете нам основные правила сейчас изложить? А то, действительно, мы…}
\soul{Мы уже говорили вам, вы просто не внимательны. Мы говорили вам, что нет строгих инструкций. Да они вам и не понравятся.}
\people{(Б) Угу. }
\people{(О) Но, просто надо быть внимательным, действительно, и… ну… как бы, слушать, наверно,  внутренний голос что ли…Иной раз так и бывает, что внутри что-то такое подсказывает, но мы это не слушаем.}
\people{(Г) Скажите, я вот сейчас не докончил счёт обратный – мне доканчивать надо постепенно?}
\soul{А вы запомните его, и, не выходя, начинайте.}
\people{(Б) Запомним ли…}
\people{(О) Ты обратный счёт давал? Обратный? С двенадцати?}
\people{(Б) Хорошо. Ребята, надо воспользоваться тем, что это вышли наши. Это…так сказать постоянные и задать те вопросы, которые мы ещё не-до задали. Вот тут у меня есть они. Сейчас.}
\people{(Г) Вы согласны ответить на несколько вопросов? }
\soul{Спрашивайте. }
\people{(Б) Скажите, вот, за последние дни, опять к нам пошли потоком сообщения, что Земле будет плохо, что максимум через три года жизнь  землян будет невыносимой. Это нас дёргают за нервы специально, может тёмные силы какие-то или же, действительно, объективные обстоятельства таковы?}
\soul{Сколько раз вам объявляли конец света? И когда же вы жили прекрасно? Всё относительно. }
\people{(Б) Да. Но недавно было сообщение, что японцы обнаружили в водах тихого океана бурные выделения углекислого газа, а это предвестник того, что могут начаться сильные разломы коры и т.д. Вы можете подтвердить правоту или нет этих наблюдений?}
\soul{Да. Мы не спорим в ваших науках.}
\people{(Б) Угу. Это научно всё же.}
\people{(О) Ну, в общем, жизнь идёт, она продолжается и, по всей видимости, она должна постоянно, так сказать…}
\people{(Г) В любых условиях продолжаться.}
\people{(О) Да. Изменения должны быть и в лучшую сторону. Тогда, собственно,  мы просто боимся, наверное, пугаемся всего. }
\soul{И это зависит от точки зрения. Давайте возьмём фаши. Цели были  благородны, ибо ``отделим от голода и нищеты путём регулирования рождаемости''. Вы помните?  Цели благородны… а как исполнялись они? А придёт время, и вы активно примите эту сторону и уже будете уничтожать миллионы, что бы потом, через пятнадцать-двадцать лет, посочувствовать тем, кто останется.(сроки подошли уже. прим)}
\people{(Б) Человечество,-  уже подтверждённые данные из разных источников,- что не менее двух процентов людей всей Земли подвергались экспериментам со стороны сил космоса. Так оно есть, или это выдумка всё? }
\soul{Хорошо. А где остальные девяносто восемь? }
\people{(Б) Над ними не делают…}
\soul{Они изолированы?}
\people{(Б) Они не делают такого прямого эксперимента. А два процента, когда их вводят в гипноз, они вспоминают, что – да, они забирались в какие-то лаборатории, там исследования были, яйцеклетки и прочее. И вот, вопрос…}
\soul{Так, будьте точнее.}
\people{(Б)Так.}
\soul{Вы спрашиваете о прямых исследованиях?}
\people{(Г) О физических. }
\people{(Б) А какие ещё могут быть? Да, о прямых, о заборе людей на лаборатории какие-то, физических людей, физических сущностей. Подтвердите вот это. Настолько споры у нас суровые идут, что правды нет. Никто не знает. }
\soul{Менее двух процентов. }
\people{(Б) Менее одного?}
\people{(Г) Менее двух.}
\people{(Б) Менее двух.}
\soul{А вот вы, ``ставите эксперимент'', как вы говорите – двенадцати  процентам. }
\people{(Б) Повторите…}
\people{(О) Это мы. Мы ставим двенадцати процентам. }
\people{(Г) Люди, да?}
\people{(О) Да.}
\soul{Да. Ибо ваша наука обучается на жертвах.}
\people{(Г) Да.}
\people{(О) Скажите, пожалуйста, вот, вообще…вот, о рождаемости той же… Вот мы, действительно, может быть, как-то… человечество, живём какой-то такой хаотичной половой жизнью, что у нас и случайные дети появляются, в принципе … и, так сказать…ну, нежеланные что ли, и… Ну, в общем-то, всё это не очень приятно, будем сказать так…Вообще, человечество, когда-нибудь… вот,  регулировалась рождаемость естественным путём, вот, как  животные, допустим, вот это вот делают?}
\soul{Да. Хотя бы вспомните войны…}
\people{(О) А-а… вот…}
\soul{… и что предшествовало им.}
\people{(О) А-а…Да.}
\people{(Г) Повышение рождаемости.}
\soul{И что интересно, если Земля захочет избавиться от вас, она будет не убивать вас, а повысит рождаемость.}
\people{(?) Повысит?}
\people{(Г) Да.}
\people{(О) Всё правильно.}
\people{(Б)  Ну, мы это пока не наблюдаем, особенно в России.}
\people{(О) Да? А разве нет? А Китай?}
\people{(Б) А-а, в Китае…}
\people{(О) А другие страны востока? }
\people{(Б) Скажите, есть ли силы, которые охраняют человека и помогают избежать опасности на протяжении…}
\soul{Вы уже спрашивали. Ангелы хранители. }
\people{(Б) А-а, ясно. }
\soul{Но, чаще, вы их не слышите и делаете всё наоборот. }
\people{(Г) Скажите, может быть, это те первые люди, которые жили на Земле не физически?}
\soul{А как же тогда теория реинкарнаций? А где же тогда были вы? }
\people{(Г) А-а,  ну, значит, мы с ними как-то разделились. }
\people{(О) Да, это мы…если можно называть так - солнечным человечеством. Да? Это наши Ангелы у человечества…}
\soul{Мы когда-то говорили вам, что вы убиваете, даже не зная о том. Так вот и вы, для кого-то, являетесь ангелом-хранителем. }
\people{(Б) Вот, на днях, мне, спрашивающему…}
\soul{Но не считайте, не ровняйте себя с Богами…}
\people{(Г) Понятное дело.}
\soul{… ибо вы не знаете этого, - первое. Второе,- вы ведёте себя, как боги. Заметьте, что бы вы ни делали, вы делаете только в своих целях.}
\people{(О) Да, точно.}
\soul{Куда бы вы ни пришли, вы устанавливаете свои законы. Верны они или нет - вас это не волнует. Почему? Почему?  }
\people{(О) Жажда,  опять же. Жажда власти. }
\people{(Г) Да нет, просто,  у каждого свои жизненные потребности.}
\people{(О) Да нет, мы просто считаем себя, что мы цари природы, что всё остальное - это подчинено нам должно быть, в наших собственных целях использовано. Собственно, это сейчас такая теория доминирует, это действительно так.}
\people{(Г) Сейчас учёные приходят к тому, что земля, так сказать, тоже  разумное существо.}
\soul{Учёные пришли к тому, что, действительно, большинство, как вы говорите, ``проценты'' – это серая масса, это всего лишь только ``биологическая масса''. Вы говорите:  ``У каждого свой мир''. Вами сказано ``Сколько людей, столько и взглядов''?}
\people{(Г) Да.}
 
\soul{А посмотрите, всмотритесь и увидите, что это не верно. Большинство имеет взгляды окружающих и, в итоге –  ``десять/двенадцать'' и не больше.(% - прим.) И потому, вам нужны вожди. И потому, даже богов вы называете…как? Вспомните. И как вы себя называете.? Овцами?}
\people{(О) А-а… ну, да. Пастухами и овцами. }
\soul{И, при этом, вы считаете себя Богом.}
\people{(О) Парадокс.}
\people{(Б) Ну, то - от нашей ``темноты''.}
\people{(О) Нет, ну, вообще-то, я… мне так кажется, всё-таки, человечество должно найти выход, потому, что мы сами себя губим,  в принципе. Это настолько уже явно, и уже, по-моему, это каждому ясно, что мы себя можем просто уничтожить. Ну, неужели у людей не проснётся хотя бы жажда жизни, уж не говоря о другом?}
\soul{Вот благодаря  этой ``жажде жизни'' вы погибните быстрее.}
\people{(О) Да?}
\soul{Ибо вы создаёте войны, чтобы выжить, ибо вы ищете мир, лязгая оружием. Вот вам ``жажда жизни'' – уничтожить всё, но остаться живым самому.}
\people{(О) Да нет, я не это имела в виду.}
\people{(Г) Не это имелось в виду.}
\soul{Не это - имелось, но это - делается. }
\people{(О) Да, это делается. Правда.}
\soul{Ибо избранных гораздо меньше - мы говорили вам о ``серой массе''.}
\people{(Б) А вот, нынешнее разоружение, по крайней мере, у нас, - это к благу, или наоборот, может превратить в порабощение нашей страны?}
\soul{Всё зависит - честны ли обе стороны.}
\people{(О) Ну, всё-таки, мы, вот, разделились на государства. Даже так, да, - люди . На государства, на нации и не только. Ну, это, может быть и должно так, потому, что все люди разные, так же как и нации, там, какие-то разные, и семьи какие-то разные, обычаи и так далее и так далее. Но, ведь  в принципе, человечество всё должно как-то объединиться, соединится, понять, что не столь важно -  эти страны границы, что ``мы лучше'', а ``вы хуже'' там, что ``мы должны жить лучше и грабить других''  за счёт, так сказать, этого. Но, должно же как-то…}
\soul{Есть  всего лишь только  два варианта: это быть всем – одной ``массой'', той  самой массой, которой нужен (пауза). Или все должны стать ``избранными''. Так что легче? }
\people{(О) Ну, легче стать серой массой, это точно.}
\soul{Что вы и делаете. Что успешно делает ваша политика. }
\people{(О) А вот, скажите, люди, которые стоят у власти -  они владеют силами, но не умеют ими управлять.}
\soul{Вы даёте эти силы. Если вы хотите диктатора - он и придёт. Вы выбираете. Ваше воплощение мечты. Если большинство желает палача, палач придёт.}
\people{(О) Да. Но как же?}
\people{(Б) Скажите, для тех, кто успешно сдаст экзамен и будет переведён в следующую эпоху - Сатья  Йогу,- как быстро будут меняться существующие сейчас социально- экономические, политические отношения на планете? Каковы свойства…}
\soul{Ничто не случится. Ничто не случится, потому, что вы это приняли, как ``экзамен''. Только экзамен. Всего лишь - один шаг. А как вы сдаёте экзамен? - Со шпаргалками. Или просто заучивание и не всегда понятием предмета. И пока вы будете так считать –  на  этот вопрос мы ответим – ничто не изменится. }
\people{(Б) Угу. А вот какова судьба низших слоёв тонкого мира и его обитателей, остающихся там после 2000 года. Ведь большая часть…}
\soul{А мы вам говорили, что-то о 2000 годе?}
\people{(Б) Ну, как-то ожидается, что…}
\soul{Мы говорили вам о конце света?}
\people{(Г) Вы говорили, что как представлять будем, так и он придёт.}
\people{(Б) Переход будет.}
\people{(О) А зачем вы представляете всё это, если это не верно? А вот, давайте, знаете…вот, ещё раз вернёмся к святым, то есть - к святым людям, которые умирают и оставляют тела, которые не разлагаются. Вы можете подробней вот объяснить, почему у них тела не разлагаются и вообще - нужно ли это? Вот, как-то меня этот вопрос затронул. Ведь в принципе…}
\soul{Что такое ``разложение''? Это всего лишь новая форма жизни…}
\people{(О) Ну, да.}
\soul{… которая присутствовала в вас, она есть в вас сейчас. Тот же ``червь победитель'' -  он есть в вас, но он не может, как вы говорите, активизироваться. А когда вы умираете, вы отдаёте ему все права, и он становится властелином. Только и всего. Но, если же вы знаете своё тело и управляете им, - не важно - осознано или нет, божественно или разумом,-  то вы не дадите ему этой власти, и тело останется нетленным. И тогда, вы уже можете брать древние легенды, когда душа приходит отдыхать в это тело. Именно тогда она обладает, как вы говорите, ``чудотворной'' силой. Именно тогда она  может лечить. Всё остальное время – это просто тело. И хотя она и имеет энергию, чаще,  эта энергия не принесёт вам пользы. Что делаете вы? Представьте, человек из серой массы, и хотя это грубо, но, к сожалению, это так, - человек серой массы -  он приходит и хочет (пауза). Для чего вы приходите к святым телам?}
\people{(О) Ну,  наверное,  больше всего - излечиться хотят.}
\soul{Излечится. А вера?}
\people{(О) Ну-у, кто верит, а кто нет, кто на ``авось'' приходит, - ``а вдруг поможет''?}
\soul{А если есть только зерно веры и не больше? Как вы думаете,  придя  и  увидев серую массу верующих, как вы думаете, встанет тот на ихней же уровень? Прорастёт это зерно? Прорастёт. Только с тем условием, если зерно было ниже уровня ``серой массы''. А что нужно сделать, чтобы наверняка эта масса была ``выше потенциалом''? Что? - Ваш  – реклама. И как бы вы не называли – это всего лишь реклама. Это всего лишь - никакой разницы, что вы рекламируете – напиток, или рекламируете мощи святых. Никакой. Почему? Потому, что рекламирующие это делают просто в целях наживы.}
\people{(О) Скажите, вот, в восточных странах, там часто бывают, во всяком случае, так…ну, может быть это тоже реклама… описывается, что ученики видят своих учителей, которых они там при жизни…которые умерли…в телах как бы они приходят. Это возможно? }
\soul{Да.  Вы же видите сны.}
\people{(О) да.}
\soul{Вы видите их в телах, а не в астральных тех.}
\people{(О) Ну, да. А, вот, про Христа…Вот, он, когда умер,  три дня, значит, он там лежал, в нише. Потом, когда Мария Магдалена пришла, увидела, что его нет… тела нет… Это действительно было так?}
\soul{Да.}
\people{(О) И.. ну, как - он остался? Его тело осталось с ним?  Вы не можете этого сказать? Или, вот, действительно, как нам говорили - он владел этими силами…}
\soul{Мимикрией. }
\people{(О) Мимикрия. А-а, ага…}
\soul{Но это было бы неверно. Просто - управление телом.}
\people{(О) Он умел, да, это делать?}
\soul{Это должны уметь делать и вы. Простите, когда вы плачете -  у вас одно лицо, по сравнению с тем, когда вы смеётесь? }
\people{(О) Да нет.}
\soul{На уровне подсознания  вы меняете своё лицо. А теперь, возьмите и войдите в роль. Вы - артист. Вам не хочется плакать, но, порой, вы должны плакать. Как, вы - измените?}
\people{(Г) Ну, если хороший артист, то изменит. }
\soul{Ну, почему вы тогда исключаете?}
\people{(О) Интересно.}
\soul{Вы ищете чудесных объяснений, чудесных явлений, и потому, пока вы будете искать чудеса, вы ничего не научитесь, потому, что всё лежит на поверхности, и всё очень просто. Вы же ``намудрите'' и сами заблудитесь в этих ``мудростях'', и, потом, плюнете на всё это и скажете: ``Ничего этого нету'' и будете обвинять других, но не себя, не вашу ``мудрость''. Разум – он на то и разум, чтобы разумно обмануть и себя  и вас.}
\people{(Г) Значит,  всё дело в том? Берётся ``ноу-хау'' - знать как? }
\soul{Нет. Не надо знать ``как''. Не нужны инструкции. Ну, возьмите, возьмите пример артиста, вы можете сделать инструкцию, как сделать гримасу недовольства, страха, смеха? Вы можете найти эти инструкции? }
\people{(О) Нет. }
\soul{И как вы думаете, если она будет создана, по этой инструкции вы создадите реальную гримасу? А что, тогда, ещё не хватает? }
\people{(О) Да, здесь только одно нужно, только внутреннее, внутренний…}
\soul{Тогда, простите, если вы должны создать гримасу плача, самое идеальное  - если вы будете плакать. Согласны?}
\people{(Г) Да.}
\soul{Так чем тогда управляется ваше тело? }
\people{(Г) Настроениям.}
\soul{Настроениями?}
\people{(О) Нет. Внутренним настроем. }
\soul{Для того, чтобы стать артистом вы должны менять настроения? }
\people{(Г) Для того, чтобы стать хорошим артистом – да. }
\soul{Нет. Вы должны владеть чувствами, но никак не настроением. }
\people{(Г) Но…}
\soul{Что такое ``настроение''? Что это? }
\people{(Г) Да, что это?}
\soul{Это, всего лишь, последовательность чувств. }
\people{(О) А-а… И нужно выбрать определенные чувства, да?}
\soul{Нет. Нужно сделать правильную  последовательность. }
\people{(Г) Ага.}
\people{(Б) Скажите,  я, в последнее время, как-то получилось, встречаясь с предсказателями, с элементами проскопии… Вот, допустим, первый вариант - предсказание на камешках, по системе древнего учёного. Насколько оно близко и, вообще-то, отражает реальность?}
\soul{Вы бросаете камни? }
\people{(Б) Да, я бросаю. }
\soul{Что вы думаете перед тем? }
\people{(Б) Думаю об этом вопросе и  как-то настраиваюсь.}
\soul{Хорошо. ``Как-то'' настраиваетесь, значит,  и ``как-то'' получили ответ. Кое-как.}
\people{(Б) ``Кое-как'' это может быть, да?}
\soul{Ну, вы же настраиваетесь  кое-как. Вы не можете управлять своими  мыслями, вам некогда, вы спешите или что-то ещё, и вы не можете толком  настроиться, - итого  - толком и не будет ваше гадание. }
\people{(Б) То есть, приблизительно - так получается. Ясно.}
\soul{Первое. }
\people{(Б) Так.}
\soul{И второе. Здесь тоже нет инструкций. Вам, хотя бы… хотя бы научитесь, научитесь просто увидеть эти камушки, и хотя бы, хотя бы научитесь чувствовать - нравится вам эта картинка или нет, - расположение камней. Хотя бы, пока, это. Если вы чувствуете, что вам не нравится, хотя бы вид камней, или хотя бы - как они расположены,- да, тогда вы можете говорить, что будет что-то плохо. Но, будьте осторожны, как вы говорите – ``настроение''. Если вы веселы, то, естественно, и камни будут ``весело'' разбросаны. Вы согласны?}
\people{(Б) Угу.}
 
\soul{В том и сложность. }
\people{(Г) Тут надо отрешиться от всего? Да?}
\soul{От всего. Полностью от всего и оставить лишь только вопрос. И чем точнее будет задан вопрос, тем точнее будет дан и ответ. }
\people{(Г) Скажите, пожалуйста, а, вот, по жизни бывают такие ситуации, когда, ну, очень уж надо было ответ, и,…как-то, я его спрашивал и у звезды… ну, короче, - у кого я только не спрашивал,- и у деревьев… и получал ответ. Это не значит, что лично они мне сказали? Это, может быть, я в себе его  увидел сам? }
\soul{Конечно. Но, если вы говорите о единстве, если вы говорите о всезнании, зачем вы ищите извне? Вам нужна подсказка. Вам нужен толчок. Вам нужен резонанс. И вы, задавая вопросы, уже знаете ответы…}
\people{(Г) Угу.}
\soul{… но, только, пока не можете сформировать. И тогда, вы даёте себе ``установку'', гипнотизируете самого себя - ``Если этот камешек, или если я выберу эту карту, то значит – да, если не эту, значит – нет''. И вот эта установка - сильнее заданного вопроса. И тогда, вы выбираете именно эту карту, именно это расположение камушков, которые говорят, допустим,- ``да'', потому, что вы хотели больше ``да'', но вами неправильно был задан вопрос. Вы получили ответ ``Да'', а вот на какой вопрос вы получили ``Да''? Пока вы просите о чём-то, множество мыслей пронеслось. Вы согласны? }
\people{(Г) Угу. }
\soul{А вот такая из них была сильнее, где корень, где начало – вы не знаете. И потому, это ``Да'' может быть относиться к другому. И когда вы убедитесь, что вместо ``да'', нужен был ответ ``нет'', вы скажете: ``Я не верю в эти камешки, всё это чушь''. И тогда, вы можете не садиться,  не гадать. Простите, вы идёте к рати святых. Вы хотите, вы веруете. И та же сила ведёт вас, когда вы хотите гадать. Но, только, к сожалению, чаще, этой силой управляете не вы, а ``предсказатели''.}
\people{(Г) Скажите, именно поэтому хорошо слышится ответ, когда, действительно,  очень уж волнует,  и, так сказать, все мысли  в этом направлении, посторонние мысли не лезут в голову, а не просто так - ``А! Будь, что будет''?}
\soul{Безусловно. Стоит ли об этом спрашивать?}
\people{(Г) Ну-у… лишний раз подтвердился.}
\people{(Б) А вот другой случай был, когда одна целительница, или ясновидящая, прямо обращалась к своему учителю и рукой, потом,  писала ответы. Вот, тут какая связь? Что за учитель? Она выходит на информационные слои?}
\soul{Давайте поговорим об учителях. }
\people{(О) Давайте.}
\soul{Начнём с ложных или с истинных?}
\people{(Г) Давайте с ложных.}
\people{(О) С истинных давайте начнём.}
\soul{Так решите же, с каких начнём?}
\people{(Б) В начале - ложные, мне кажется. А потом уже – на истинные.}
\people{(О) Ну, большинство… Давайте, как большинство.}
\soul{Большинство? }
\people{(Б) Ложные.}
\soul{Ложные… Это те самые учителя, которые не дают истинным учителям говорить. Это – те самые учителя, которые родили вы сами, ваша мечта. Простите, а почему - учитель? Так… давайте сделаем так - вы живёте на необитаемом острове, никогда у вас не было никаких учителей, ни физических, ни, тем более, других. Как вы думаете, по-другому будут называться они? Так и будут называться ``учителями'' или будут сразу уже ``богами''? }
\people{(О) Ну, где как, наверное.}
\people{(Б) Богами, наверное.}
\soul{А здесь - только учителя. Простите, когда шаман приходит к камню, камень, для него - учитель. Итак, вы, видя свою беспомощность и не веря себе, ищете помощь извне, и находите. Выдуманные учителя. Именно тот самый дух, о котором вы мечтали всю жизнь, и которого вы не могли найти. И, для вас, он является учителем. И очень страшно, если вы будете ему полностью подчинятся. Конечно, с первого взгляда, можно сказать, что вы подчиняетесь самому себе, своей мечте, и в этом нет ничего страшного. Нет, это страшно. Потому, что над-чувства, чувства, нарисовали этого учителя, а потом, разум нарисовал, дорисовал  да исправил. Понимаете? }
\people{(О) Угу.}
\soul{Давайте возьмём так. Возьмём человека, возьмём его мнимого учителя, и возьмём - когда этот человек  находится в печали. Как вы думаете, какой будет учитель? Будет успокаивать и будет жалеть. И будет жалеть именно так,- заметьте, - учитель жалеет лучше, чем другой какой либо ближний человек, потому, что вы-то знаете, как вас лучше пожалеть.}
\people{(О) Ну, да.}
\soul{И вот, он вас жалеет. Помогает или наоборот?}
\people{(О) Ну, если… как воспринимать…}
\people{(Г) Если вы начали с ложных, то, наверное…}
\people{(О) Вряд ли это помощь истинная. Скорее всего - ещё раз…}
\people{(Г) Здесь двояко, можно, конечно…}
\soul{Ну, вот, смотрите. Если вы настроены, чтобы вас жалели… и  вас жалеют, и вы больше и больше плачетесь в жилетку тому учителю. }
\people{(О) Ну, да. И что?}
\soul{И, в итоге, вы теряете силы, ибо они были потрачены на самолечение, как вы говорите -саможаление. И что? Когда к вам придёт кто-то, другой, ближний вам человек, придёт и будет жалеть вас, а вы не будете ему веровать, потому, что он же не знает, как вас надо лечить. Согласны? }
\people{(О) Согласны.}
\soul{И вы будете видеть его ошибки, ибо вы уже нарисовали картину, идеал. Тот самый учитель. И не видели этого идеала в ближнем. Вот, и здесь уже первый ``кирпич''. И так дальше, дальше и дальше. И в итоге - вас лучший друг становится вам врагом. }
\people{(О) То есть, происходит подчинение, да, вот этому… и человек теряет вообще  свое истинное ``я'', - то есть, как бы заблуждается?}
\soul{Нет. Сперва, вы занимаетесь чисто самолечением, самообучением, а потом - фантом, который уже может существовать уже и без вашей помощи, ибо он достаточно набрал энергии. И вы подпитываете, пока веруете в него.  И вот этот фантом, конечно же, хочет приобрести свободу. И тогда, он будет уже вам подсказывать совсем, совсем ``не то''. Для чего? Да чтобы вы отвергли его. Прогнали. Сказали: ``Нет, это плохой учитель''. Или создать любой принцип недопонимания. Или наоборот - сделать вас рабом.}
\people{(О) А когда его прогонят, он уйдёт, и что будет делать?}
\soul{Он не уйдёт. Просто он поменяет шкуру, только и всего. Он может самого себя убить. Представляете, - приходит другой учитель, происходит борьба, конечно же ``хорошие парни побеждают всегда'' – и у вас новый учитель. И вы считаете, - добро восторжествовало – и больше увязли. }
\people{(Б) Продолжите уже не по ложным, а по истинным учителям. Их поведение. }
\soul{Истинный учитель…}
,
\people{(О) Нет, ну, может ещё по ложным хотите сказать?}
\people{(Б) Может ещё по ложным есть ещё что сказать?}
\soul{Множество. Множество ложных.  Ложь – это именно то, что не существует. Возьмите словарь Даля. Найдите, как описывается ложь. А теперь попробуйте объяснить сами, что такое ложь. Найдите слова, найдите антонимы. Найдите синонимы. Попробуйте! }
\people{(О) Ну, ложь – это уход от истины. }
\soul{Уход от истины. Хорошо, а вы знаете истину? }
\people{(О) Да нет, конечно.}
\soul{А как же вы тогда узнаете, что вы ушли? }
\people{(О) Ну, всё равно, вот внутренний-то есть импульс…всегда. Просто может быть мы к нему не очень-то и прислушиваемся, просто, вот, насколько мне…вот, так я, иногда, думаю про себя, что…}
\soul{Итак, ложь – это относительно. Да? Ибо вы разумом не можете определить, насколько он ушёл, ибо вы не знаете истины. Конечно, проще знать, что у вас в кармане столько-то, столько-то, а вы говорите ``столько-столько''. Здесь вы можете, количественно,  определить насколько вы солгали. Но - как количество. И при этом, вы не сможете определить, намного ли вы солгали, - всё зависит от внешних обстоятельств,  для чего вы лгали. }
\people{(О) Ну, ложь – это иллюзия тоже, вот, что мы…}
\soul{Хорошо. Иллюзия. И мы снова вернемся: а вы видели истину? Вы видели истинную картину? }
\people{(О) Нет.}
\soul{Нет. А как вы тогда можете говорить, что это – иллюзия, а это – нет? Только из-за того, что это вам больше нравится, а это - нет?}
\people{(О) Да нет.}
\soul{А как?}
\people{(О) Ну, я не знаю как, вот, мы определяем…}
\people{(Г) А скорей - так и есть ``нравится– не нравится''.}
\people{(О) Ну, вообще-то зачастую, наверное, да. Если не понравится – значит, то - ложь, а если понравится, значит – это правда. Ну, это общепринято, в принципе. Но, это же - тоже ложь. }
\soul{Давайте дальше. Что ж такое ложь? Скажите ещё что-нибудь.}
\people{(О) Каждый человек, наверное, какую-то точку отсчёта выбирает от себя относительно, и для него ложь - всё то, что ему не подходит. Наверное - так.}
\people{(Г) Наверное, один из вариантов истины, скажем так. }
\soul{Прекрасно. Итак, вы утверждаете, что всё – ложь. Всё – ложь, истины нет. И то, что мы сейчас говорим - тоже ложь. }
\people{(Г) В принципе – да,- то, что вы говорите. }
\soul{Да.}
\people{(О) ``Мысль изречённая  - есть ложь''. Ну, тогда всё, что мы говорим…}
\soul{``Я лгу'' – как вы думаете, человек, сказавший это, лжец или нет? }
\people{(Г) Смотря  как считать…}
\people{(Б) По крайней мере, в этом слове…}
\soul{Ну, давайте подумаем. Итак,  ``Я лгу''. Если человек говорит правду, значит он лгун? Но раз он лгун, он же  сказал правду? Значит - он уже не лгун?}
\people{(Г) Да. Но это зависит от данного момента. }
\soul{От данного момента?}
\people{(Г) На данный момент - он может быть не лгун.}
\soul{Именно какого?  Но, он же - лжёт. Он же, сейчас же, вам сказал, на данный момент произнёс эту фразу. }
\people{(Г) О лже-лгун.,}
\soul{Но, он же уже вам сказал, что он лжёт. Так он говорит правду или нет? Вы можете объяснить? }
\people{(Г) Объяснить можно, в принципе.}
\soul{Тогда - пожалуйста. }
\people{(Г) Если он… он может солгать, что он сейчас лжёт. }
\soul{Значит - он лгун.}
\people{(Г) Да, значит - он лгун, получается. }
\soul{Но он сказал, значит, правду?}
\people{(О) Да-а.}
\soul{На данный момент, он сказал правду. }
\people{(О) Да.}
\soul{Так, значит, он не лжёт? }
\people{(О) Он сказал правду, что он, да, солгал. Вот так. }
\people{(?) Правдивый человек. }
\people{(Б) Парадокс.}
\soul{Попробуйте решите его. }
\people{(Г) Хорошо, мы подумаем. Интересно.}
\people{(О) А, вот, скажите, а вот такое, а вот - об истинных учителях…мы все так не…}
\soul{Истинный учитель… Его голос слишком тих. Он никогда не повышает голоса. Во всяком случае,  редко делает это. Это тот, который, чаще, бьёт вас, когда уже не может сделать другое. Когда вы уже идёте тропой гибели, именно он сбивает вас с этого пути. Вы воспринимаете его, естественно, врагом. У вас сорвалась сделка, вы не успели попасть туда-то, туда-то. ``Злые силы'' - говорите вы,- ``злой рок''. Вот, чаще, истинного - вы принимаете за ``злой рок''. Чаще, вы его стараетесь уничтожить, избавится от него, ибо он не лжёт вам. Ибо он не будет поддакивать вам. А вам не нравится это. }
\people{(О) Да, мы любим, когда нас ``гладят по головке''. }
\people{(Б) А есть способы, как получше, всё-таки, этого друга истинного определить, почувствовать? Какие лучшие способы? }
\soul{Верить в себя. Если вы будете веровать и любить себя, а не обобщать с ``общей массой'' и в то же время, не выделятся, не становится фараоном, тогда вы сможете услышать его. А пока, вы - средняя масса, или пока вы играете в Наполеона, вы не услышите его. И если вы будете унижать себя, и если вы будете говорить ``Господи'' и так далее, с чувством унижения, ``господи'' не услышит вас. Не услышит и учитель. }
\people{(Г) А кто услышит? Ложные учителя?}
\soul{Тот, кто унижает. И вы преподнесете ему прекрасный подарок. }
\people{(Г) Так это сколько ж народу верит в такого Господа?}
\soul{В какого?}
\people{(Г) Вот, в которого вы, сейчас, говорили,- в которого они верят.}
\soul{А вам и сказано - ``Князь мира сего''.}
\people{(О) Так мы и есть, человечество, – это и есть дьявол? Мы, вот, и есть этот ``сын утра'', который пал?}
\soul{О, нет. Неужели тогда Адам, и та же Ева соблазняли друг друга,- самого себя? Давайте скажем так: сила. Сила. И чем меньше вы будете питать силу эту, тем быстрей, как вы говорите, избавитесь от дьявола. Но, заметьте, даже у вас сказано: ``Уйдёт царство дьявола на тысячу лет''. А почему не навечно? }
\people{(Г) Да?}
\soul{А почему? Да потому, что нельзя жить только в добре, нельзя жить только в раю, иначе - вы будете похожи просто на животных, которых кормят. Вы перестанете думать, мыслить.  И кем вы станете?}
\people{(О) Ну, ясно, да. }
\soul{И какой….(срыв контакта, интонация изменилась. Прим.)}
\people{1…}
\people{(Б) Можно продолжить?}
\soul{Да, вы можете спрашивать. }
\people{(Б) У вас была долгая пауза. Мы не поняли, это – вы же? }
\soul{Я - только одна из сторон. }
\people{(Г) А мы с вами как-нибудь общались?}
\soul{Да, но они носили имена. Называли себя или  ``масками'', или ещё кем-то. А я, пока, никакой. }
\people{(Б) Это новый сеанс…А сейчас,  вы как чувствуете обстановку? В каких мы масках? Что мы? Как устроены? Вы ощущаете?  Вы должны ощущать?}
\soul{О, нет. Я ощущаю только то, что ощущает мой хозяин - мой ``дом''. }
\people{(О) Какой ваш ``дом''? Вам нравится ваш ``дом'' теперь? }
\soul{Мой ``дом''? Если быть честным, то - нет.}
\people{(О) А почему?}
\soul{Он не ухожен. }
\people{(Б) А что не хватает? Как это ``не ухожен''? То есть, у него беспокойство какое-то?}
\soul{Очень много хлама. Неубрано. Очень много лишнего. А иногда начинается генеральная уборка, но почему-то всегда - нужное выкидывается, а какая-то безделушка остаётся. }
\people{(Б) А ``генеральная уборка'' – это что подразумевается? Мытьё? Моет себя просто,  или мысли очищает? И что лучше…}
\soul{Вдохновение – это  генеральная уборка.}
\people{(О) А очень редко бывает, да?}
\people{(Б) А если молитва, если покаяние – это…}
\soul{Тоже - уборка.  Ну, вы знаете, как вы каетесь. Вы ж не начинаете это делать откровенно, получается - тяп-ляп, тяп–ляп. И весь этот мусор – под кровать, в уголок, под ковёрчик. Но оно ж осталось у вас, вы ж не выкинули. И потом, вы берёте какую-нибудь игрушку, как вы говорите -  мысль,-  рассматриваете её, крутите и не знаете, куда её применить. Вот вы с ней носитесь, носитесь туда-сюда, она никуда не подходит, вот вы взяли её и выкинули или отложили на запас. И вы даже не знаете, что эта мысль могла бы вам очень сильно помочь. }
\people{(Б) А действительно, что  истинное покаяние, глубокое покаяние… может существенно измениться настроение, состояние, судьба даже человека?}
\soul{Ну, конечно же. }
\people{(Б) Да? Но мы как-то не сталкивались с этим, толи не было повода так каяться. }
\soul{Но только знаете, вы же не знаете истинное это было или нет. Когда вы толпой начнёте все каяться, то вряд ли здесь будет пахнуть истиной. Просто - вы будете играть. И вот эта толпа жаждущих о прощении – она не получит его, хотя, будет чувствовать себя прекрасно и тоже изменится судьба, но у всех -  как был не ухожен дом, так и останется.}
\people{(Б) Церковь-то любит, именно,  как раз, большие, общественные, громки покаяния, чтобы все дружно вместе. А они не правы, получается?  }
\soul{Конечно же - нет.}
\people{(Б) А почему тысячелетиями это практикуется?}
\soul{Ну и что? А вы тысячелетиями живёте в нищете и практикуете это дальше. Хотя вы знаете, что вам это не нравится, что войны – плохо, а чё ж вы их практикуете-то? И вам больше нравится уйти уже по одной дорожке: ``Вот я уже здесь ходил, по ней и пойду. Зачем мне по- новой? А то заблужусь, не дай Бог ещё шишки набью''. Вот и приходите по ним. А она же вся дорожка-то, разбита, а  вы всё идёте и идёте, а другой не хотите. Вам новую дорожку построили, а вы не идёте, - вы же не знаете, что там будет. }
\people{(Б) Скажите, а вы посоветуете чаще делать личные покаяния, внутренние, человека перед собой?}
\soul{Ну, как это можно посоветовать?! Ну, если вам  нужно каяться – кайтесь. Но если вы скажете: ``Каждую пятницу я буду каяться''. Ну и как вы думаете, это будет искренне? Можно, конечно, сказать, дать обязательство, что ``я в день должен  обязательно написать хотя бы одну страницу'' и будете строго поддерживаться этого правила. Вы будете писать, вы будете хорошим писателем. Относительно. Потому,  что иногда вам будет не хотеться писать, но вы должны  обязательно написать эту страницу. Теперь представьте, вы в эту пятницу решили хорошо покаяться и, причём, не просто решили, а действительно - вам потребность покаяться. Вы покаялись, а потом: ``Я это буду делать каждую пятницу''. Вам же это понравилось, вы почувствовали облегчение. В следующую пятницу этого уже не будет. И тогда, вы уже будете стараться получить то прежнее удовольствие. А как? Обманывая  себя. И дойдёт, как вы говорите, до маразма: вы будете записывать свои грехи в блокнотик, чтобы потом не забыть, когда покаяться.}
\people{(Б) Ну, и получается, что ничего это делать не надо? }
\soul{Если вам это не нужно, пожалуйста, не делайте того. Но если вы почувствуете – скорее и не откладывайте, иначе - уйдёт, забудете и потеряете. }
\people{(Б) А это в любой обстановке можно делать, даже в трамвае и в очереди, или же всё-таки…}
\soul{А вы считаете, что каяться надо обязательно на коленях?}
\people{(Б) Ну, как-то, более торжественно.}
\soul{Ну, представьте, вы захотели покаяться в трамвае, вы бухнулись на коленки и на место водителя стали креститься. }
\people{(Б) Ну, не бухнулся, а мысленно так, в трамвае, каешься. Ну, значит, это не даст результата? Надо чтобы настрой был?  }
\soul{Ну, что вы говорите? Но ведь вы же-то уборку делаете в себе, а не снаружи! Ну, зачем вам нужна эта свечка? Эта свечка нужна только тем, кто не может без свечки. Вот вы хотите ключик. Вот, придумали себе ключик: именно свечка, именно икона, именно вот этот Бог, вот именно то-то и то-то. Вот вы верите в Иисуса Христа, значит, только он, верите в Иегову, значит, только он. Это - ваш ключик. Понимаете? Христианин не пойдёт и не будет покаяния просить у Будды. Правильно? Потому, что он сразу потеряет настрой. }
\people{(Г) То есть, это и есть кумир, получается, да? - Ключик.}
\soul{Кумир? Вот пока вы будете считать его кумиром – это будет всего лишь только  пустой звук. Этот тот же самый  Будда будет всего лишь только куклой, которую вы сами слепили, кое-как и всё. И тот же Христос, для вас, будет посмешищем. И вы будете к нему обращаться, каяться, просить, но где-то в душе вы будете знать, что он просто вами создан, нарисован и слеплен, потом будете сомневаться.  Но, что интересно, вы будете… вы потеряете-то, веру-то и в настоящего! А потом ещё и других за собой потянете: ``Всё это чепуха, всё это ерунда, послушайте совет старого человека''}
(Конец  1 стороны кассеты)
28-06-1996 – 1
\people{Мабу, ты что ли?}
\soul{Я!}
\people{Чё ты выдел? }
\soul{Всё видел. }
\people{Ну, всё…Ну расскажи, что ты ``всё''-то видел? Может, мы этого не видели, что ты видел.}
\soul{Тебя видел.}
\people{Где?}
\soul{Петю видел.}
\people{Где?}
\soul{Как где? Здесь.}
\people{В пещере?  }
\people{Во сне?}
\soul{А тебя не видел (Белимову).}
\people{А меня видел, Мабу? }
\soul{Не-а. Новая жена. }
\people{А ты…сколько у тебя жён теперь? До скольких ты считать умеешь? }
\soul{Хитренькие какие!  Вы сами отгадывайте. }
\people{Ой! Ну, как мы отгадаем, если ты ничего нам не говоришь?}
\people{Мы, вот,  думаем, что у тебя три сейчас. }
\soul{Ни-и!. В тот раз вы быстро отгадали. Вы спросили сколько у меня камушек  здесь лежит, я сказал. Вы сразу сказали: `` А-а! Столько-то жён''. }
\people{А-а! Теперь у тебя я знаю сколько камушков. }
\soul{А теперь, не догадалися? }
\people{Ты когда камушками научился считать… Когда у тебя уже было…}
\soul{Хитренькая! Не скажу. Не буду. }
\people{Я сейчас сама вспомню. Когда ты камешки научился считать, когда у тебя уже было пять жён.}
\soul{Не-а. Не угадала.}
\people{Не угадала?}
\soul{Не-а. }
\people{А до девяти ты считаешь?}
\soul{Чего?}
\people{До девяти умеешь считать?}
\soul{Не-а.}
\people{Ага, значит, ``пять''. Я не угадала, да?}
\soul{Ну, тогда-а-а… – четыре! }
\people{Хитрая!}
\soul{Почему? }
\people{Неправильно. }
\soul{Тоже неправильно?}
\people{Это ты нас обманываешь? Ты во сне нас видишь?}
\soul{Я не обманываю. Зачем мне обманывать?}
\people{Он не умеет обманывать. (Белимов) }
\people{Умеет! (Оля и Лена хором)}
\soul{Умею.}
\people{Ты во сне нас видишь, Мабу?	}
\soul{Я вчера, подошёл к старейшине, попросил у него огня. }
\people{Угу.}
\soul{А у меня уже был.}
\people{У тебя уже был? }
\soul{Был.}
\people{Значит, ты его обманул?}
\soul{Нет. Не обманывал, но мне нужны были два огня.}
\people{А-а! А зачем тебе два огня? Холодно, что ли? }
\soul{У меня сын болеет.}
\people{А-а! А мы ж тебе советовали вынести его на солнышко. Из пещеры вынести его. }
\soul{Ну и чё?}
\people{Ну, он выздоровеет, потому, что солнышко ему поможет вылечиться.}
\soul{'' Солнышко поможет”…Это же не зерно вам.. (недоверчиво. Прим.)}
\people{Спроси у монахов. Они тебе скажут. }
\soul{Это же не зерно вам. }
\people{Ну и что? Всё равно поможет солнышко. Спроси у монахов, они тебе подтвердят. Они скажут, что солнышко поможет вылечить.}
\people{Мабу, а что у тебя интересного сейчас было, в эти дни было? Счастливый ты был от чего? Может мяса поел или что?}
\people{Мяса он не ест. Он не знает, что это такое. Мабу, ты нас слышишь?}
\soul{Нарисовал голову. Мне сказали, что это ку..кух. }
\people{Как?}
\soul{Кух.}
\people{Кух?}
\soul{Да.}
\people{Кто это?}
\soul{Голову нарисовал, а  глазки не рисовал.}
\people{А! Круг. Круг, да?}
\soul{Вы и то знаете…}
\people{Мы знаем, а ты сказал нам ``кух''.}
\soul{Забыл.}
\people{Забыл. Ладно. }
\soul{Потом, они говорят, а теперь нарисуй камень, и чтоб этот камень лежал в…}
\people{В круге.}
\soul{Да.  А я им говорю: Дураки вы что ли?  Как это камень будет в голове?'' А они говорят: ``Это не голова, это – круг''. Показали. А мне понравилось. Я тогда взял и ещё круг нарисовал, уже в камне. А потом, в этом круге ещё нарисовал камень. И так много-много раз. }
\people{Молодец. Ты быстро понял. }
\soul{У меня… не знаю, что у меня получилось. Но им понравилось.}
\people{Им понравилось?}
\soul{Да. }
\people{Ну, фигуры геометрические…}
\people{Мабу, а у тебя ``ну как'' есть?}
\soul{Не надо дразниться.}
\people{Почему, помнишь, ты у меня просил ``ну как''?}
\soul{Ты жадина.   Он мне не нужен.}
\people{Не нужен тебе ``ну как''?}
\soul{Не нужен.}
\people{Ну, вот, а ты у меня так сильно просил. ``Дай мне, дай мне, жадина''. А теперь, оказывается, что тебе не нужен. }
\soul{Это тебе не нужен. }
\people{Всё ясно. А ты сейчас спишь, Мабу, или нет?}
\people{Или монахи положили?}
\soul{Я это давно без монахов делаю.}
\people{Один?  Жена, да? }
\soul{С женой…}
(Срыв контакта)
\people{1…}
\soul{Страшно, что мы, в принципе, марионетки, марионетки от погоды, от природы. Понимаете, чисто марионетки. Вот, зимой нам холодно, ладно, мы откладываем множество работ, когда потеплеет. Пришло тепло – нам теперь жарко. То голова болит, то ещё что-то. Понимаете? И мы опять отложили. А в итоге получается мы – просто марионетка. Марионетка, зависящая от погоды. Но мы всегда найдём, оправдаемся, скажем: ``Это не очень-то и нужно'', и в итоге,  действительно, благородная цель потом уходит в никуда, и она, действительно, уже как бы и не нужна. Сколько таких вот дел мы уже побросали? Просто из-за того, что нам было лень.}
\people{Ну, а ты, в этом состоянии понимаешь, что это наши упущения, мы ленивые люди, не целеустремлённые, так что ль?}
\soul{Так, если бы понимали, что мы ленивые, так мы же не говорим, что мы ленивы, мы говорим ``Некогда''. Найдём множество любых мелочей, с которыми будем возиться, возиться, лишь бы главным-то не заняться. И потом, конечно же, будем оправдываться: ``Я же то-то, то-то, то-то делал. Мне просто было некогда''. А это - просто лень. Была просто лень и ложь.}
\people{Скажи, нам нужно посчитать дальше? Мы вот не считали… другой счёт, который ты… или нет?}
\soul{Ну, вообще-то, я слышал, что вы считали, и считали до тринадцати.}
\people{Ну, мы продолжим тогда?}
\soul{Зачем?}
\people{Не надо, да? }
\soul{Ну, вы считали  же до тринадцати? }
\people{Да. Но ведь не до девятнадцати? Значит, мы что-то потеряем?}
\soul{Но - вышел же? (на уровень 19-ти. прим)}
\people{Ну, хорошо. Ладно. Скажи, пожалуйста, просто попробуем твою память…Какие предметы ты в школе особенно любил? }
\soul{Географию. И то, только первые два класса, потому что была очень хорошая учительница, и когда она ушла на пенсию, то этот предмет мне уже не стал нравиться. Что ещё? Физика. Физика мне нравилась только тем, что я ``доводил'' физика. }
\people{Значит, тебе не физика нравилась, а доводить физика нравилось? }
\soul{Получилось как? Конечно, новый предмет – очень интересно, но до этого я уже вроде как и занимался. И на самом первом уроке я ``поставил его в угол'', то есть, я заметил ошибки, которые он произвёл. Я его поправил, и он меня невзлюбил. И для того, чтобы теперь быть, как говорится, ``на высоте'', что бы он мне не ставил сплошные двойки, мне приходилось учить эту физику. А в итоге, потом, толи привычка, толи что, но мне она понравилась. Но что интересно, у нас где-то полтора года шла ``война''. Довольно крупная ``война''. Я отвечал, и если бы не было этой ``войны'', я получал бы оценку ``отлично'', но - ``три'', и не больше. Даже класс…ну то, что у нас класс был очень недружен, - даже класс возмущался, не мог понять. И вдруг, эта вражда очень резко превратилась в дружбу. Буквально – мгновенно. }
(Срыв контакта)
\soul{Человек, вообще-то, не может фантазировать. Он может увеличить какое-то действие, или просто уменьшить. Но никак…Он не сможет сочинить что-то новое. Вот вы говорите, как можно объяснить описание в Бхагавад-Гите. Ну, давайте представим так ситуацию, описывается оружие несущее огонь, да? И там есть такие слова, вот смотрите: ``…но  пользуйся им не против людей, а против богов, что пришли извне.'' Вот вам и, пожалуйста, инопланетяне. Правильно?}
\people{Ну, да.}
\soul{Ну, ладно, пока не будем объяснять, что к чему. ``И не пользуйся им без нужды, ибо он принесёт множество смерти, и будет неуправляем, и солнце уничтожит надолго.'' Помните же? А что дальше? ``И даже в чреве матери будет смерть''. Пока не будем говорить о чём? Что это была за часть. Давайте, просто возьмём так -  огонь. Ну, как можно пофантазировать про огонь?  Ну, хорошо, взял человек палку, да? На него идёт тигр. Вот этой горящей палкой он испугал этого тигра. А вот, смотрите, как он теперь будет дальше фантазировать. Тигр, - он его ``увеличит'', чтобы это было заметно, что это была очень сложная борьба, да? И этот тигр стал до самых небес.  Правильно? Человек остался таким маленьким, воин, да, а тигр был очень огромный. Если ему дать ещё такие-то мистические силы, там, что он всё-таки не колдун,  понимаете, экстрасенс-тигр, скажем  по-современному. Пожалуйста. Ну, и огонь. Для того, чтобы уничтожить такое могущество, нужен, значит и огонь могучий? И вот уже легенда, фантазия нам нарисует огромнейший огонь, который уничтожает всё, и который ярче солнца. Понимаете? Можно, конечно,  объяснить, что вот он убьёт солнце. Заметьте, это уже как было сказано. Там было сказано, что убьёт солнце. А много ж времени, правильно, не полностью он убьёт? А возьмите теперь, вот вы спрашиваете, не было ли  это ядерным оружием, да? Ибо всё совпадает. Да, конечно, вам ядерная зима, тут прекрасно всё описывается. И будет съедено солнце, сожжено. Но это можно ещё придумать, да, это можно пофантазировать, потому, что мы берём аналогию солнца и огня. И огонь столь сильный, что даже солнце пожрал. А вот как это объяснить, что ``чрево матери'' – это уже фантазией не может быть. Потому, что огонь как может влиять на чрево матери? Вы говорите радиоактивность,  да? Но здесь бы уже не смог человек пофантазировать это, потому, что ему нечего было увеличивать. }
\people{Угу. На тот период.}
\soul{Ну, да, конечно. Вот  и получается, что в принципе, ваш вопрос, да, положительный, да это было описание ядерного оружия. }
\people{Так оно и было?}
\people{Да нет, да?}
\soul{Ну, почему? Наоборот, это подтверждает, что - да, это было описание ядерного оружия. Очень много оружия. Возьмите… Хорошо, давайте возьмёт Архимеда. Вспомните его солнечные зеркала,  сжигающие паруса. Не корабли. Паруса именно сжигают. Здесь уже  нельзя уже говорить о том, потому, что это не было просто фантазией. Потому, что иначе бы - сжигался корабль, а не паруса. Потому, что парус-то легче зажечь. Но вы знаете, зеркала не могут этого сделать. }
\people{А вы знаете, что недавно показывали по телевизору личный эксперимент делала какой-то там…ну, показывали, я лично видел. Сожгли, действительно, корабль и корабль сгорел. Даже не зеркалами - начищенными пластинами железными! }
\soul{На каком расстоянии стоял корабль?}
\people{Достаточно далеко, я не помню в морских. (в терминах длины. прим)}
\soul{И каковы были зеркала?}
\people{И несколько человек стали, как линзы, так сказать, держа зеркала, в одну точку направляли все…}
\soul{Тогда, давайте вспомним так, - вот, смотри, возьмём увеличительное стекло, так? Вот смотри, изобрели увеличительное стекло и лишь только через двести лет изобрели микроскоп и телескоп. Почему? Только через двести лет. Представляешь, какое это было огромное запаздывание? А почему? Потому, что одна линза искажает, а если  взять и посадить вторую линзу - ещё больше будет искажение. Правильно? Представляешь, человек имеет сотню линз, он точит их, и ни разу в жизни он их вот так случайно хотя бы не поставил. Двести лет!}
\people{Наверное, в этот промежуток были какие-то открытия, просто не было развития дальше. }
\soul{Разве? }
\people{Наверное, да. }
\people{Нет, скорее всего, это не было необходимости.}
\soul{Ну, как же это не было необходимости? Ну, представьте, тот же Галилей, а ведь если бы, если бы не вот эта вот ошибка, тот же Галилей мог бы видеть уже через телескоп. Вы представляете, сколько вот множество таких запаздывающих открытий? Что именно кто-то специально вас держит. Двести лет ни один не попробовал их поставить хотя бы случайно и увидеть, что, наоборот, наоборот - стало ещё больше! Так нет, эти двести лет мы все считали, что одно искажает, а два ещё хуже будет, представляете? А сколько…}
(Переводчик вышел из транса и рассказывает о своих впечатлениях)
(Перепад)
\soul{…я вижу, как воздух выходит с лёгких, а мы его не используем полностью. Очень много кислорода выбрасываем, да. …надо задерживать. Получается, наоборот, углекислого газа не хватает в нас. Мы слишком быстро выбрасываем воздух, стараемся получить новый, и что интересно, получается, что после бега, чтобы отдышаться, – наоборот, надо замедлить дыхание. Тогда быстрее отдышишься…}
\people{Да, кстати. }
\people{Мы продолжаем сеанс 28 июня 96 года.}
\soul{…Если взять какой-то текст, записать его на магнитофон, и его нужно очень срочно выучить, оказывается, его надо записывать, модулируя низкой частотой, где-то порядка 16 Гц, чтобы, как говорится, амплитуда этого текста  при записи менялась с частотой 16 Гц. И, в итоге, эта запись уже будет запоминаться гораздо лучше, усваиваться организмом. Почему-то так.}
\people{А нужно это делать? Может лучше развивать память? }
\soul{Да, но английский язык-то учат сейчас так.}
\people{А последствия какие могут быть?}
\soul{Никаких, практически, потому, что это же не постоянно. }
\people{А, вот, смотри, были эксперименты несколько лет назад, тоже английский язык учили ночью на магнитофоне, и через несколько лет эти люди  начали умирать. Это правда, нет?}
\soul{Нет. Понимаете в чём дело, вот берём эти 16 Гц. Ну, хорошо, давай так сделаем, возьмём мозг, да? Лобная часть. Вот буйного, чтобы успокоить ему отрезают лобную часть, разрывают связь, и он становится тихим беспомощным. Лоботерапия, да? (название – лоботомия) Очень сильно распространена у нас. За рубежом она практически не применяется, но, все же есть. И вот, частоты различны, и вот, что ещё интересно, самый большой эффект и самый безвредный эффект будет, если мы сможем снять биотоки мозга. Снять, да, - как бы увеличить ``обратную связь''. Мы прослушиваем текст, допустим на магнитофоне, потому что легче манипулировать этим магнитофоном, чем речью человека, или через усилитель, если человек говорит в микрофон, да? Как говориться - внушение электронным путём. Страшная штука. Дело в том, что если мы сейчас снимем биотоки мозга, эти биотоки мозга будут, как бы, управлять этим сигналом, да, речью или чем, модулировать, то это будет довольно таки страшная вещь. Вот тебе, пожалуйста, способ ….внушить, при чём это сможет сделать любой, имея эту установку.}
\people{Слушай, биотоки мозга в широком диапазоне. Их магнитофон может просто не захватить. }
\soul{Нет, но мозг-то работает не в широком диапазоне. Вот, возьми, что отвечает…элементарно, даже не будем брать, как говорится, именно, конкретно, да, а вот, произносится речь, ты её слышишь. Так? Во-первых, тебе надо что? Ты же - слышишь, значит, надо пройти по слуховому нерву и найти точку вхождения его в мозг. Правильно? Вот этот участок мозга отвечающий за слух. Пожалуйста, отсюда сними. Так, теперь дальше пойдём. }
\people{Свет.}
\soul{Зачем свет? Мы говорим только о звуке. Про свет – взять, конечно, световой, естественно, нерв. И, потом, как говорится, область, отвечающая за память. И что интересно, если мы будем воздействовать на левую половину, возьмём с левой половины, да, то здесь уже, как говорится ``говорящий человек'' он уже будет ассоциироваться нам мистически. Потому, что уже мы берём не логическую часть, да, -  мистическую. То есть, мы на уровне чувств уже будем запоминать этот голос, понимаешь, эту установку. Вот тебе – пожалуйста. И, что самое страшное, это у нас применялась, установка, хотя бы на самоубийство высоких чинов. Когда даётся установка ему - определённая фраза, код, пароль. }
\people{Фильм был такой. Да? Фильм.}
\soul{Фильм про это? Ну, это надо, чтобы кто-то представил тогда его. }
\people{Ну, сюжет именно такой…}
\people{Звонят по телефону ему, и он выбрасывается в окно. А там - установка…}
\people{Обычная фраза…}
\people{Нет, не обычная. Там, значит, что народ, чё-то…}
\people{Ну, каждому - своё.}
\soul{Ну, ясно. Я понял. В принципе - да. Дело в том, что…Илона Давыдова – тот же принцип. Взять тот же принцип, но только взять, как модулируется. И что интересно, эта запись не копируется. Её нельзя перекопировать по той причине, что при копировании, ну-у…даже если мы отбросим АРУЗ, (авторегулировка записи)который выровняет нам амплитуду и полностью сбросит вот эту вот запись, то есть это вот кодирование, - существует такое понятие, как ``инерционность''. Понимаете? Инерционность. А почему, получается, мы можем записать вот именно оригинал, да, потому, что мы это действуем на входе. Понимаешь, на входе. Получается, сам сигнал на входе – далее всех различных емкостей, и, получается, он не срезает, не срезает эту кодировку. Когда же мы копируем, мы проходим полный тракт усилителя.}
\people{Но, он же в реальном времени работает.}
\people{1…}
\soul{(Другие)….проблематично создать резонатор для суб-низких частот. Это пока что останавливает, чтобы стало распространённым видом оружия. Но можно взять резонатор на более высокие частоты, габариты его будут уменьшены, как вы желаете, но - пожалуйста, тот же эффект. Модулируйте суб-низкими частотами. Мозг сам уже отсеет. Вот вам будет и воздействие, как вы говорите, психотронного…(срыв)}
\people{Психотронного.}
\soul{(Гена)…обезьяний царь. А помните его легеду, как говорится, всё…}
(срыв)
\soul{(Гена) Слышу.}
\people{Слыш?}
\soul{Слышу.}
\people{Слушай, а что ты там про обезьяньего царя? Про легенду…Не знаешь?}
\soul{Ну, задавай, не знаю.}
\people{Скажи, пожалуйста, вот, по памяти, ты может сейчас вот в этом состоянии сказать, как улучшить память. Ты можешь сказать, какие есть такие не слишком сложные пути улучшения памяти?}
\soul{Ну, чисто, как говориться, не используя никакой техники? Ничего?}
\people{Да.}
\soul{Ой, довольно-то очень просто. Это надо, чтобы это стало необходимым. Это - первое, чтобы было необходимым. Так? Потом, когда это станет необходимым, сразу отпадают такие вещи, как отвлечение от чего-то, да? Потом, надо использовать… ну, скажем так, надо запомнить какой-то набор цифр, да? Самое сложное цифры или набор букв, которые ничего не значат для тебя, ну, скажем так иностранные китайские иероглифы, да? Здесь просто надо ассоциации, нарисовать свою картинку этого знака. Понимаешь? Не именно каждой буквы, а всего слова. Потому, что, в принципе, надо запоминать слово именно не по знакам… Или тот же самый набор цифр, да? Не отдельно по цифрам, а сразу - общий. }
\people{Точкой.}
\soul{Ну, зачем точкой? Нет, сразу нарисовать картину, если вам нужно запомнить цифры 19, 62, 705, 15, 14 - да? Вот это число. Попробуйте сейчас повторите.}
\people{19, 62}
\people{15, 14.}
\soul{Уже не правильно. А вот взять…Да хорошо. 1962 – вам легче запомнить это. Правильно?}
\people{Да.}
\soul{Уже - часть. Уже - мы пришли к тому, что это не отдельная цифра, а вот, именно уже хотя бы часть числа, но всё-таки уже ближе. Уже можно привести аналогию. Первое. Дальше что. Понимаете, они кажутся простые советы, да? Но их очень трудно выполнить. А вот выполняя их, это действительно будет очень эффектно. Ну, давайте возьмём так: вы вспоминаете какой-то иностранный текст, но, во-первых, слова, произношение этих слов, уже вам что-то напоминает. Так же?}
\people{Бывает – да, бывает – нет.}
\soul{И, причём, есть прямые аналогии. Допустим, пожалуйста,- как по-английски сказать слово ``мне''?}
\people{Mи.}
\soul{“Mи''. А теперь представь, что ты не русский, скажем так - грузин, да? }
\people{Понятно. Я понял.}
\soul{Ну, как ты скажешь это ``мне''? Постараясь, грузин скажет это по-русски. Он исказит его и получится ``ми''. Так же? Теперь, возьми…}
\people{Ассоциации. }
\soul{Ассоциации. Давай  возьмем простейшую ассоциацию. Вам нужно запомнить, ну, пусть будет слово ``магазин'' – ``shop''. Как вы запомните? Вспомнить ту же самую рекламу. Вот, пожалуйста. Любой ключ к запоминанию. Любой ключ. И при этом, вы ещё …Вот я говорил, что не надо отвлекаться, да? И тут же вам сказал противоположное, что вы должны запомнить вот этот момент, как вы запоминали, понимаете…. И вот эта вот трудность ``не отвлекаться'', она тоже запомнится. И, в итоге, чтобы запомнить вот этот вот набор слова, да или набор цифр вам нужно будет просто вспомнить именно не сами числа, а вот состояние,  в котором ты был, когда ты его заучивал, понимаешь, что интересно… И вот, смотрите…}
(срыв)
\people{1…}
\soul{И можно сделать это и технически. Тоже, довольно-то будет просто, но это ``просто'' войдёт как в привычку что ли…Как привычка к лекарству, к допингу. Понимаешь? И поэтому, желательно бы этого не делать. Конечно, привычка тоже будет запоминание именно вот на таком уровне, как я рассказал, да? Тоже являться ``допинг-привычка''. Но, эта привычка безвредна, потому, что ты всегда ею воспользоваться, а технически это всегда надо иметь при себе, именно тот аппарат, который позволяет это сделать, да? И очень просто, аппарат очень простой, конечно. Вот самый простой вариант – это воздействие на лобную часть. Это просто два  электрода, пусть это будут просто монеты. Но только монеты надо взять, одну монету надо серебренную, скажем так, а другую – медь, и подавай полярность.  То есть, тот же самый принцип. Вот, вспомните – электросон. Но там, вот что наша беда, что мы для улучшения контакта применяли сперва, может представить страшно себе, свинец! У нас электроды были - свинец! Вы представляете, какой электролиз? И вот этот свинец попадает в лобную…}
\people{16, 17…Так что насчёт свинца в лобной части?}
\soul{Что `` на счёт свинца в лобной части''? }
\people{Понятно. ``Проехали.'' }
\people{Про сон ты говорил, как улучшить. Ну, ладно. }
\people{Ты не помнишь, что я тебе вот сейчас говорил? У тебя, интересно,  в сознании отложилось? Ты же всё должен помнить, вот, секунду назад? }
\soul{Ну, я уже объяснял, что моя память, ты знаешь, как вот визирчик, маленькое окошко, и, вот, я вожу этим ``окошком''. Но, ваши вопросы…}
\people{Про электросон. Про, эти, электроды из свинца, что они попадают в лобную часть…}
\people{Раньше были электроды - делали из свинца. }
\people{И чем это кончается, ты не сказал, что это опасно и почему. }
\people{Вот, свинец, который  воздействует  на организм, лобную часть. Мы знаем уже в общем-то. Объяснено какая…}
\people{Он тяжелый метал. Ну, и чем это грозит вообще?}
\soul{Чисто физиологически, чисто по научному?}
\people{Да.}
 
\soul{Свинец чем грозит? А он знаешь… как объяснить… Это инородный, во-первых, метал, и, причём, заметь, ты меняешь ионный обмен. Вот самое страшное у свинца. Понимаешь?}
\people{Масса – да. Правильно.}
\soul{А медь – она очень активна, но понимаешь нам не хватает меди, просто, и всё. И поэтому, как говорится, небольшой излишек извне, он, в принципе, не помешает.}
\people{Ну, и значит, электросон это…А память как улучшить с помощью этого?}
\people{Куда прилагать электроды?  }
\people{К лобной части.}
\people{На затылок и на лоб, получается?}
\soul{Нет. Нет.}
\people{Лобная часть – это что?}
\soul{Висковые доли. }
\people{К одному и второму виску, да?}
\people{Прям к виску? }
\soul{Да.}
\people{И какой ток пустить? }
\people{Вот это от монеток пойдет это.}
\soul{Да, уже, в принципе, будет достаточно взять от монеты. Но только тогда надо…Ну, здесь, как сказать… Дело в том, что…Самое лучшее было, если бы вы просто эти монеты, они бы на вас ещё и держались бы сами. И вот, когда они отпадут, но именно отпадут, не от того, что вы пошевелили мышцами,  понимаете, и отбросили их. Для этого надо принять позу какую-то, достаточно удобную, да, чтобы не было именно мышечных воздействий на эти монеты, иначе вы их отбросите. А они сами должны отброситься. Когда наберётся определённое количество заряда, произойдёт ионный обмен, и они сами отбросятся. Это самый идеальный, но это сложный, и поэтому, гораздо проще взять из нацепить, закрепить чем-то, тем же лейкопластырем. Но здесь уже не будет саморегулировки.}
\people{А их соединять надо между собой? }
\people{Нет. }
\people{Как же ``нет''?}
\people{Они уже…}
\people{А вот, левое/ правое – это не будет влиять, или это тоже надо почувствовать как это?}
\people{Ты слышишь, да?}
\soul{Слышу.}
\people{Слушай, ещё вопрос тебе такой, вот мы про электросон говорили, а с памятью это как связано это дело? }
\people{Ну, как? Что тут непонятного?}
\soul{Что ``с памятью''? }
\people{Ну, вот, электросон, вот эти монеты медные. }
\people{Ионный обмен.}
\people{Реакции памяти…Это реакции, правильно?}
\people{Электро-реакция.}
\soul{Я не знаю, о чём вы спрашиваете.}
\people{Ясно. Понятно. }
\people{Ладно. Скажи, пожалуйста, на твой взгляд, чем обусловленный твой выбор жены?}
\people{Это личное. Он туда не пустит. }
\soul{Ну, если честно, то, конечно…}
\people{Ну, мы поняли. Не надо.}
\soul{Дело не в том, что я не хочу говорить, я просто сам даже не хочу знать ответ, честно говоря. Чтобы…А то вдруг, не дай Бог, разочаруюсь там или просто… Просто – побаиваюсь что ли…}
\people{Слушай, вопрос…}
\people{Ну, а какие качества жены тебя дополняют, как личность? Отметь, хорошие качества жены.}
\people{Каких у тебя нет.}
\soul{Ну, это мне надо сейчас рисовать жену.}
\people{Сложно? Ну, не надо. }
\people{Скажи, Гена, а вот ты  на стройке повесил молоток на лоб, как надо смотрелся. Скажи, с помощью чего, каких сил и вообще – почему?  Это же не магнит в черепной коробке создал себе?}
\soul{Ну, тогда, пожалуйста, прямая аналогия, почему мы с тобой разбили плиту?}
\people{Ну, извини, мы применяли физическую силу. (было дело с бетонной плитой…в 1986м. прим.)}
\soul{Да, а потом мы применяли, но ничего не могли с ней сделать. В чём была разница?  В настрое, в направленности силы.}
\people{Ну, тогда мы забор кирпичный разбили, тоже с тобой. Помнишь?(полностью пролёт забора из белого кирпича. Прим.) }
\soul{Направленность силы. Понимаешь? Дело в том, что мы… как сказать…Ну, в общем, мы…через нас проходит сила, течёт через нас, да? Как бы её только не называли, но она течёт через нас. И вот, когда мы сможем…Она течёт, вот, просто, поток, вот, знаешь, без толку, да? Вот, просто протекает вхолостую…(срыв)}
\people{Ты слышишь, да? Всё нормально?}
\people{Говори, говори. Вот сейчас переводчик говорил о силе. Что она протекает сквозь нас и в холостую. Вот мы должны найти применение этой силе.}
\soul{(Другие) Во-первых, она не протекает вся, как вы говорите ``в холостую''. Это - первое. Второе, - да, вы должны направлено воздействовать на эту силу. И, причём, не грубо физически, материей, потому, что это бесполезно, потому, что сила создает вам эту материю, ту физику. Ваше желание, резонанс. Тот же самый резонанс. Если вы желаете то-то, вы мечтаете, о чём-то, и чем активней будете мечтать, - только  два варианта: или тем активней будете отдаляться, ибо вы будете мечтать не в ту сторону, или наоборот - будете приближаться. Возьмите примеры: картины, сюжет, та же таблица – всё это работа ваших сил.  И если вы скажете: `` А что здесь такого? Она приснилась бы мне, я бы тоже нарисовал эту таблицу''. Нет. Труд. Не просто желание - труд. ``Просто желанием'' вы не добьётесь ничего, потому, что надо работать. Просто желание, голое желание – это крик лени и не больше. Ибо вы хотите ``лежать на печи''. Давайте возьмём вашу любимую сказку.}
\people{Про Емелю.}
\soul{Как вы думаете, хороша сказка? }
\people{Ну, некоторые разрисовывают, типа того, он там медитировал, привлекал силы, и вот у него всё получалось.}
\people{Нет, мне эта сказка всегда не нравилась, действительно, тем, что он сказал там, вот, давай это - и это пошло.}
\people{Про богатыря.}
\soul{Даже любовь - он зарабатывает с помощью щуки. Видите как! Даже это! Не кажется ли вам, что народ просто посмеялся над своей ленью? А мы теперь – ``Сказка!''. И причём мы изменили её смысл. И приходят другие и говорят: ``А вы знаете, это был величайший экстрасенс…”}
\people{Скажите, тридцать три года спал, а потом встал и… так сказать …}
\soul{И заметьте, найдите аналогию, вспомните, где было упомянуто ещё тридцать три года.}
\people{Ну, понятно - Библия. }
\people{Ну, это символически, что в человеке до тридцати трёх лет, вот, он спал…}
\soul{А почему? Почему именно тридцать три?}
\people{Может  более реальные взгляды  на жизнь просыпаются?}
\people{“Три'' плюс ``три'' - в суме ``шесть'' даёт.}
\soul{А шесть – это сила дьявола. Как это объяснить?}
\people{Число человеческое.}
\soul{Число человеческое? }
\people{666.}
\people{Нет.}
\soul{А не число ли зверя?}
\people{Зверя. }
\people{Ну, и что? Почему ``тридцать три'' это важный фактор?}
\soul{Давайте посмотрим, сколь долгий был срок жизни тех времён, когда создавалось это.}
\people{Да. }
\soul{Тридцать три – это что? }
\people{Мудрость.}
\soul{Зрелость. Это сейчас вы уже говорите  ``восемнадцать лет''. }
\people{Ну, да, мы не правы. }
\people{Скажите, вот переводчик не сумел ответить, какие качества жены…}
(перепад)
\soul{(Гена)…слышу, что пять, шесть, во, а я…}
\people{Да. Вот интересно, в этот раз лишь получается как? Мы считаем с ``одиннадцати'' до ``девятнадцати'', выходят, вот, наши собеседники, да. Считаем, сейчас, вот так вот, вышел ты.}
\people{Всё перепуталось.}
\people{Нет, по-моему, они, знаешь что делают, чтобы они прошли тот счёт там, они может быть какие-то пути ищут. }
\people{Он часто отвлекался…}
\soul{Я не теряю вас. Теряю и даже…Понимаете получается как -  вы в разные стороны как бы разбегаетесь, а я не знаю за кем идти. }
\people{Ну, может у нас у всех…}
\people{Ну, я же говорю – ``базар, базар''!  Когда - ``ля-ля-ля''.}
\people{Слушай, это вот насчёт этой силы, значит,  только силой желания молоток на лоб вешал.}
То есть, ты просто, как это сказать, чисто собезьянничал и повесил, не задумываясь особенно? 
\soul{Да. В принципе да. }
\people{То есть, просто хотел. Без всяких побуждений… вообще без ничего? Без удивления….ну, чтобы удивить кого-то.}
\soul{Ну, просто был интерес. Интерес. Понимаешь?  И этот интерес не был ничем мотивирован, и получается вот… как сказать…чистота, вот эта чистота позволила мне это совершить. А вот, когда я уже начал, как говориться, показывать на публику, да, здесь уже появилось сомнение: ``А вдруг не получится? Вот посмеются''. И да, - я уже стал терять. И взять, хотя бы, фокус с линейкой, что я мог двигать подшипники, да? }
\people{Да, да. Чё это было? Почему отталкивалось? Мы с тобой тогда делали эксперимент.}
\soul{Дело в том что…}
(обрыв записи)
\soul{(…)искривление перегородки.}
\people{Какой? }
\soul{Искривление носовой перегородки. }
\people{У тебя?}
\soul{Да, вот это позволило изменить, вот эту электромагнитную составляющую так, что получается, ну, как объяснить…поток воздуха. Скажем так… }
\people{Ты нас слышишь, нет? }
\soul{Только в потоках воздуха, да? - когда я вдыхаю, получается,-  у меня разность течения воздуха. Потому, что один канал у меня узкий,- искривление перегородки,-  второй – нет. И в итоге,  получается, потенциал.}
\people{Погоди! Ты же не дышал! }
\soul{Как же я буду не дышать, если я эту линейку отодвигал достаточно, - и как же я не дышал? Да, я старался не дышать, чтобы, как говорится, видно было, что я линейку сдвигаю не воздухом. Но ты забываешь, что у нас ротовая полость соединена с носовой, носоглотка, понимаешь? И поэтому, вот когда ты задержишь дыхание, да, ты это должен быть заметить, что циркуляция воздуха происходит. Особенно, когда уже приходит время всё-таки вздохнуть… }
\people{Нет, ``извини пожалуйста'', циркуляция воздуха не происходит, потому что мы можем в горловой  полости заткнуть вот этим маленьким язычком носовую полость - полностью.}
\soul{Тогда, попробуй, попробуй сейчас, зажми нос и подержи,  и ты увидишь в конце, что у тебя нос, как говорится, надулся. А с чего бы он у тебя надулся, если ты говоришь, маленьким язычком прикрываешь? Попробуй. Это же можно попробовать прямо сейчас. Когда у тебя уже будет не хватать воздуха, у тебя будет циркулировать. Во-первых. Во-вторых, никогда ничто не стоит внутри тебя. Понимаешь? Тот же самый воздух. Ну, подумай сам, если у тебя… Подожди. У тебя работает сердце. Сердце, а значит, уже происходит, кровяное давление меняется. А лёгкие - они же насыщены всеми этими. Согласен?}
\people{Согласен. Но лёгким до линейки ``далеко'', извини.}
\soul{Ой, простите, пожалуйста, но…линейка далека, но связь с лёгкими всё-таки существует.}
\people{Хорошо, а как же тогда…}
\soul{Знаешь, что, давай скажем так…Возьми учебник анатомии, возьми учебник физики. Возьми учебник анатомии для того, чтобы поглядеть, как твои лёгкие связываются. Так? Теперь возьми учебник физики и посмотри о сообщающих сосудах.}
\people{Так, хорошо. Но, всё-таки…}
\soul{Подожди. Теперь возьми и посмотри, как создаётся…Возьми принцип гидравлики.}
\people{Гена…Тебя можно так называть? }
\people{Да, можно, уж сколько раз…}
\people{Нет, по-моему - тормознулся…}
\people{Спрашивай.}
\people{Слушай, а как же ты тогда двигал стрелку компаса, я не пойму? Там уже точно воздушных потоков-то не было. Стекло.}
\soul{Электрический заряд. }
\people{Чем создавала этот заряд? }
\soul{Мы все…Я полностью же ``проводник''. И если я сумею себе внушить, изменить, хотя бы просто, хотя бы давление, допустим, руки, да? Задержать кровь в руке, как говорится, да? Напряжение мышц, или чем  - это не важно, но альвеолы  перекрыты.}
\people{Да. }
\soul{Давление меняется, а значит и меняется электромагнитный поток. Это - очень элементарно. И вот, это вот, это знание позволяет, не то что, как говорится, двигать компасом, но и…}
    (Запись обрывается) (Конец контакта)
\chapter{Аоум. глава 1-я 25-12-1993г}
\begin{verbatim}
 ***
 Садится ночь на подоконник,
 Очки волшебные надев,
 И длинный вавилонский сонник,
 Как жрец, читает нараспев.

 Уходят вверх ее ступени,
 Но нет перил над пустотой,
 Где судят тени, как на сцене,
 Иноязычный разум твой.

 Ни смысла, ни числа, ни меры.
 А судьи кто? И в чем твой грех?
 Мы вышли из одной пещеры,
 И клинопись одна на всех.

 Явь от потопа до Эвклида
 Мы досмотреть обречены.
 Отдай - что взял, 
 что видел - выдай!
 Тебя зовут твои сыны.

 И ты на чьем-нибудь пороге
 Найдешь когда-нибудь приют,
 Пока быки бредут, как боги,
 Боками трутся на дороге
 И жвачку времени жуют.

\author{Арсений Тарковский}
\end{verbatim}

\date{26-12-1993г}

\soul{"<– Но будьте осторожны. У вас было сказано: - ``благими намерениями…'' Согласитесь, что любое добро может перерасти во зло. Всё зависит, что вы поймёте, и как вы поймёте. Мы б могли бы в вас прийти и объяснить всё так, чтоб вы поняли. Но простите, тогда вы ничто не будете знать более, и уже встретившись с новым, растеряетесь, и вам будут снова нужны контакты. Вы должны идти, идти сами. И мы, всего лишь, приходим к вам не советом, а только сомнениями. Ибо, сомневаясь, вы начинаете думать.">}

\soul{ - …любовь. И тогда вы увидите, и тогда сама любовь придет к вам и поможет. Если у вас будет только одно желание – у вас ничего не получится.}

\people{Так, а ещё, кроме веры и желания?}
\soul{У вас всех есть зерно любви, зерно веры, и надо его просто растить.}
\people{Так. Что является ``поливной водой'' для этого зерна?}
\soul{Ваши эмоции.}
\people{Ага.}
\soul{Если у вас будут преобладать отрицательные эмоции, то и ростки вырастут соответствующие. Я ответил вам?}
\people{Да. Вполне. Спасибо.}
\soul{Вы уходите?}
\people{Я не знаю, что ещё сказать.}
\soul{Вы просто говорите. У вас в вопросе больше эмоций.}
\people{Угу. А в диалоге? Всегда меньше эмоций?}
\soul{Нет.}
\people{Или вам важны только вопросительные эмоции?}
\soul{Нет.}
\people{Любые?}
\soul{Любые. Мы не питаемся эмоциями. Мы находимся в мире эмоций. И чтобы нам удержаться в этом мире, извините, у вас это называется «в волчьей стае по-волчьи выть».}
\people{Понятно. Каков он, мир эмоций? Вы опишите мне его. Как? С ваших понятий.}
\soul{Это очень красивый мир. Вы можете видеть его и физически. Но это очень опасный путь и мы не дадим вам его. Могу сказать одно: его видели многие. Это очень красивый мир! Это мир красок! Ярких красок! Вы не видели столько красок нигде!}
\people{Хорошо. Это состояние самадхи или нирваны?}
\soul{Нет. Разве в этих состояниях видят краски?}
\people{В этих состояниях блаженствуют!}
\soul{Блаженство, это одно из состояний эмоций.}
\people{Понятно. То есть, они появляются в этом мире, только в каком-то, допустим, определённом месте, да? Ну… или так, - в часть этого мира вклиниваются?}
\soul{Если хотите, то да.}
\people{Сравнение не точно?}
\soul{Нет. Это слепец. Он всё ощущает, его всё это греет, но он не видит.}
\people{А! Он не понимает, что это.}
\soul{Да. Он не может… Точнее, - его органы не могут воспринять это. Но было много случаев… простите, в отрицательных условиях, и поэтому мы не можем сказать вам, как это сделать. Хотя, это можно увидеть и органами.}
\people{Но есть люди, которые видят. Вот тут есть примеры, которые говорят, что они видят органы насквозь.}
\soul{Видите и вы! Но вы не воспринимаете этого. Простите, вы видите всё и вся. Я говорил вам об этом. Но вы не умеете этим управлять.}
\people{А-а, понял.}
\soul{Если хотите, - то человек – это луковица. И, причём, бесформенная. И чем больше вы воспринимаете, тем больше…}
\people{Ощущаем?}
\soul{Да. Вы становитесь красивее, ``грациозней'', в вашем понятии. Нам трудно объяснить что это. В теории ``грации'', я имею в виду и в духовном смысле.}
\people{Я понимаю.}
\soul{Но, извините, у вас бывают ``луковицы'' и пустые.}
\people{Бывают? То есть как? Без души?}
\soul{Да.}
\people{То есть, есть люди, ходящие по земле, которые…}
\soul{Множество!}
\people{Они убили в себе сами душу?!}
\soul{Мы не можем вам дать ответ.}
\people{Это секрет?}
\soul{Нет.}
\people{Повторение?}
\soul{Нет.}
\people{Что это?}
\soul{Это не секрет, но мы не можем.}
\people{Ясно. Это, так сказать, ``табу''?}
\soul{Нет.}
\people{Причина, по которой вы не можете сказать?}
\soul{Причина и есть - ответ.}
\people{Ну, что ж… Не можете, так не можете. Дать счёт?}
\comment{(обрыв записи)}
\people{Как переводчик выходит на контакт с вами? Это происходит чисто по его инициативе, или есть причины, при которых он впадает в это состояние?}
\soul{Нет, это его инициатива. Вы постоянно слышите нас, и вы хотите этого, и, поэтому, стараетесь.}
\people{Понятно.}
\soul{Мы всегда с вами.}
\people{Т.е. мы не умеем слышать?}
\soul{Да.}
\people{Вы можете оценить мое состояние? Могу я слышать его или нет?}
\soul{Вы слышите ``шёпот''.}
\people{Да. На нашем языке это называется стихи?}
\soul{Не только.}
\people{Мысли?}
\soul{Духовное состояние.}
\people{Понятно. Те, кто был 15000 лет назад до нас?}
\soul{Вы.}
\people{Тело было такое же?}
\soul{Нет.}
\people{Какое?}
\soul{Вы нас спросили про теорию Дарвина.}
\people{Понятно. Т.е. обретая человеческое, мы обрели человеческий облик?}
\soul{Нет.}
\people{Кто нас сделал такими из обезьян?}
\soul{Вы сами. У вас есть аналоги в фантастике.}
\people{Приведите пример.}
\soul{Когда человек с помощью машины времени узнаёт, что жизнь не могла зародиться на Земле, он возвращается в прошлое и создает её. Вы объясните, кто создал жизнь?}
\people{Получается, мы сами.}
\soul{Вы сами. Здесь нельзя рассуждать логически. Просто надо принимать на веру.}
\people{Я понял. Это предел логики.}
\soul{Как вы можете объяснить, что вы создали жизнь, и, при этом вас не было? Как вы появились, чтобы создать эту жизнь? Вы можете объяснить логически? Логика – это ваш предатель.}
\people{Она не поймёт кольцо времени?}
\soul{Здесь нет кольца. Здесь нет аналогии. Есть только попытки.}
\people{Вот поэтому и не поймёт, потому, что нет аналога.}
\soul{И, тогда, любое ваше предположение окажется правдой.}
\people{Ясно.}
\soul{Или ложью. Это безразлично в данный момент.}
\people{Ибо всё это вместе есть истина?}
\soul{Ибо всё это есть вами выдуманное.}
\people{То, что нами выдуманное не существует в мире?}
\soul{Вы не поняли. Все, любые ответы, на такой вопрос могут являться ложными и так же правдивыми. Ибо они далеки от истины. Истина – это уйти от логики. В простом варианте.}
\people{Понял.}
\soul{Надо понимать всё в целом. Нельзя делить прошлое и настоящее. Именно это деление вы не можете понять и создаёте кольцо времени. Оно реально, но реально в вашем мозгу. Это мыльный пузырь, готовый лопнуть. В любой момент. Но вы любите его и оберегаете.}
\people{Ясно. Т.е. это привычка?}
\soul{Нет. Это – нежелание измениться. Страх нового.}
\people{У нас страх всегда?}
\soul{Да. Это защитная реакция. Она может быть положительной, может быть отрицательной. Всё зависит от обстоятельств.}
\people{Понятно. От чего зависят обстоятельства?}
\soul{От поведения. Любое обстоятельство рассматривается по-разному. Разные индивидуумы – разные обстоятельства. Вы поняли?}
\people{Понял. Почему иногда хорошему человеку, по нашим меркам, выпадает плохая судьба, ему достаётся больше чем всем?}
\soul{Он больше воспринимает. Равнодушный человек не старается взять больше. Он проходит мимо и не замечает.}
\people{Наша цель – увеличить, так сказать, наши чувства?}
\soul{Нет. Понять единство мира.}
\people{Это цель всей жизни?}
\soul{Нет. Цель есть выше. Пока вам надо понять единство мира.}
\people{Ясно.}
\soul{Вам нужно научиться врага любить. Вы должны искать врага, но не создавать их. Ибо это ваш учитель. Вы должны идти по дорогое, называемой жизнью. Множество тропинок, вы можете идти по любой, и не заметив этого. Но дорога одна. И вы можете идти вперёд и назад. И, в отличие от Земли, вы не вернётесь в эту точку.}
\people{Ясно. Т.е. будет новое качество? Появится.}
\soul{Нет. Сперва идёт количество.}
\people{Есть ли прямая зависимость между количеством и качеством в мире?}
\soul{Да. Чтобы получить определённое качество, нужно набрать определённое количество.}
\people{Ясно.}
\soul{У вас это называется ``заряд''.}
\people{Ясно.}
\soul{Неважно, что является зарядом. Любовь, ненависть – это силы. Имея нужные силы, вы перейдёте на новое качество. Не имея их, вы не придёте.}
\people{Зависит ли от силы, допустим от любви, что я обрёл большую любовь или обрёл большую ненависть - получу я хорошее или плохое?}
\soul{Да, это прямая.}
\people{Связь прямая?}
\soul{Прямая. Чем сильнее любите, тем больше и бед на вас придёт. Cравните тяжёлоатлета, и сравните обыкновенного человека. Тяжёлоатлет старается взять больше. И вот ответ на ваш вопрос «почему добрый и кроткий получает больше». Он старается взять большее, он замечает большее, и ему достаётся за это.}
\people{Хорошо, если я обрёл великую любовь и ушёл в отшельники в пустыню, где нет никого?}
\soul{Нет! Вы предали себя, вы предали всех. Вы рождены в обществе, и вы должны мучиться в обществе, в вашем понятии. И быть отшельником – это уйти от мира, это ``уйти в себя''. Это - эгоцентризм. И он не принесёт пользы, какой бы огромной любовью вы ни обладали, во-первых. Во-вторых: вы никогда не сможете обладать огромной любовью, чтобы уйти в отшельники. Вы можете привести примеры, имеющих большую любовь и будучи отшельниками?}
\people{Нет. Это теоретический вопрос.}
\soul{Любовь себя? Пожалуйста. Она может быть великой. Но это лишь будет любовь себя. Нельзя быть отшельниками. Простите, мать, нарожавшая детей, уходит в отшельники - кому оставила детей?}
\people{Я понимаю. Вопрос такой:- Допустим, если я настолько люблю людей, что видя, что я им мешаю в этой жизни, - я уйду в отшельники. Это что будет, не любовь?}
\soul{Вы можете мешать? Как вы можете мешать людям?}
\people{На меня будут злиться, что я такой, какой есть.}
\soul{Это ваша ноша. И чем тяжелее вы её возьмете, тем больше ваша любовь. Здесь прямая аналогия.}
\people{Понятно.}
\soul{Маленький человек не возьмет большой груз – он его раздавит. Чтоб вас не раздавило, вы должны прилагать большую и большую силу, – или вы будете раздавлены. К сожалению, чаще всего, так и получается.}
\people{Есть ли предел любви в мире?}
\soul{Нет. Даже в вашем мире этого нет, в вашем понятии. Вы можете назвать предел своей любви?}
\people{Нет.}
\soul{Вы можете найти начало этой любви?}
\people{Начало любви? Восхищение.}
\soul{Нет начала любви. И нет её конца. Если вы видели конец любви – это не любовь.}
\people{Понятно. Значит, любовь может быть или вечной, или её не может быть вообще?}
\soul{Или её не может быть. Cегодня вы любите, завтра - вы не любите, – так не бывает.}
\people{Понятно.}
\soul{И вы не можете её растерять. Чем больше вы любите, чем больше вы её отдаёте, тем больше вы получите.}
\people{Бед?}
\soul{И любовь!}
\people{Ну, допустим, если я полюблю какую-то группу людей, то эта группа людей рано или поздно полюбит меня, так?}
\soul{Вы должны любить группу людей или всех?}
\people{Понятно.}
\soul{Вы должны любить всех. И при этом у вас будут враги, страшные враги. Чем сильнее вы, тем сильнее враги. Тем и сильнее будут ваши сторонники.}
\people{Любовь нельзя победить?}
\soul{Нет.}
\people{Враги не знают об этом?}
\soul{Враги - те же самые, что и вы. Они тоже ищут то же.}
\people{Понял.}
\soul{Вам было сказано: ``Любите врагов'', но не создавайте их – это ваши учителя. И даже более.}
\people{Вы не знаете, есть ли у меня враг?}
\soul{Я должен назвать?}
\people{Если знаете.}
\soul{Как вы думаете, будет ли это пользой?}
\people{Это будет информация. А в каком качестве я её использую, будет уже зависеть, пользой это будет это или вред.}
\soul{Это всё равно, что вы спросите: «Когда я умру?»}
\people{Я не умру?}
\soul{Вы не поняли.}
\people{Телесно.}
\soul{Нет. Телесно вы умрёте. Иначе, вы бы не росли. Всё созданное умирает.}
\people{Я знаю этот закон. Хорошо, значит главная цель на Земле - научиться любить. Хотя бы Землю, потом, весь мир.}
\soul{Эта одна из целей.}
\people{Понятно.}
\soul{Научившись любить, вы увидите другие цели. Вы будете бесконечно, бесконечно что-то искать, находить и бороться.}
\people{Это и есть ``дорога''?}
\soul{Это и есть ``дорога''. И когда вы преступите определенные, в вашем понятии качества, то у вас уже будут другие. Вы уже будете видеть всё с другого. (уровня осознания. прим). И, тогда эти вопросы будут уже не нужны. И у вас появятся многие новые.}
\people{На прошлом контакте, вы сказали, что помогая нам, вы нам делаете ``медвежью услугу''.}
\soul{Да.}
\people{Сейчас вы мне делаете ``медвежью услугу''?}
\soul{Я не даю вам лишнего.}
\people{Значит, в мыслях мы слышим тот же ``шёпот'' или не всегда?}
\soul{Это не в физическом смысле ``шёпот''. Простите, это чувство. Всё передаётся чувствами. И ваши стихи – это ваши стихи.}
\people{Понял.}
\soul{Если хотите, это перевод чувств. Это ваш мир эмоций.}
\people{С вашей точки зрения у меня правильные эмоции?}
\soul{Нельзя так рассуждать.}
\people{Ну, грубо говоря: хорошие или плохие?}
\soul{Простите, вы имеете все краски. И у вас есть моменты, когда они хорошие и когда они плохие. Всё течёт. Вы – вода. Вы непостоянны. И это - ваша беда и ваше счастье.}
\people{Если человек впал в слабость, ну, духом упал, кто ему помогает подняться?}
\soul{Все! Помогают все. Даже враг! Но, это надо увидеть, эту помощь.}
\people{Понял. Можно просто верить, ничего не делая? Сидеть и верить.}
\soul{Нет.}
\people{Можно ли вообще ничего не делать?}
\soul{Нет.}
\people{Получается, что даже уйдя…}
\soul{В любом случае вы находитесь в движении. Есть три вида движений. Это хаос, положительные и отрицательные.}
\people{Есть мысль, что хаос из хаоса – Великий порядок…}
\soul{Нет. Хаос – это неумение управлять, это неумение понять. Как такового, в прямом cмысле, хаоса не существует. Всё закономерно. Хаос – это показатель случайности. Но, вы же знаете – случайностей не бывает. Задавайте вопросы.}
\people{Если сейчас на кассете прозвучит какая-либо музыка, вы увидите, услышите, ощутите её?}
\soul{Мы находимся в мире эмоций. И если вы её будете чувствовать, то будем чувствовать и мы.}
\people{Если будет чувствовать переводчик?}
\soul{Мы находимся в мире эмоций, но мы не называем имён.}
\people{Понял. Вы видите сумму всех эмоций?}
\soul{Если хотите, то - да.}
\people{Всей Земли или всего мира?}
\soul{Нет. В данном случае, Земли.}
\people{Значит, есть мир эмоций Земли, есть мир эмоций Луны, Солнца?}
\soul{Нет. Все это сливается в одно. Но рассматривая картину, вы можете разглядывать одну деталь?}
\people{Понял. Как на общем плане выглядят эмоции нашей Земли? Плохо, хорошо?}
\soul{Если честно, то плохо.}
\people{Очень?}
\soul{Очень. Вы – зараза. Но вы любимый. Вы непослушный дитя, но, которого любят. Его надо наказать, но нужно наказать так, чтобы ему не было больно, - сильно больно. Вы понимаете, что такое ``жалость''?}
\people{Да. Когда будет это наказание? Вы, конечно, не скажете?}
\soul{Почему же? Вы постоянно наказываете друг друга. Вы сами себя наказываете. Никто не придет и не уничтожит вас. Вы сами сделаете всё. Вас нельзя уничтожить – мы говорили об этом. Но вы можете прекрасно уничтожить себя.}
\people{Т.е. разлагаясь морально, духовно, мы убиваем себя?}
\soul{Да.}
\people{Т.е. будет какой-то момент … или не будет … или вечно это?}
\soul{Всё в движении. Вы – ``вода''. И наступит момент, когда вы будете ``кипеть''. И будет момент, когда вы будете ``льдом''.}
\people{Хорошо. В прошлом контакте вы сказали, что Христос - это энергия. А может ли эта энергия переходить…}
\soul{Она и не покидала вас!}
\people{Она растворилась по всей Земле?}
\soul{Это и есть шёпот. Шёпот вашей души! Душа – это часть Христа! Это часть энергии.}
\people{Ясно.}
\soul{Вы же, персонифицируете Христа.}
\people{Но как персона он был?}
\soul{Как человек – да.}
\people{Как персона.}
\soul{Нет.}
\people{Это миф?}
\soul{Ему пришлось прийти сюда человеком. Христос – это всё. Он ваш поводырь.}
\people{Христос – это поводырь Земли?}
\soul{Что вы скажете о Боге? Что, в вашем понятии, Бог? Лично в вашем?}
\people{Лично в моём? Двоякое понятие. Значит, всё-таки, по нашим земным понятиям, это - некто, или, скажем - ``нечто''. Грубо говоря, вы были правы, ``энергия'' вполне подходит под это, даже логическое рассуждение.}
\soul{Вы увидите его как персону, ибо ваш мозг может видеть только персону. Поэтому вам придётся видеть его персоной. Но это не так, это больше чем персона. И вы можете смело называть Христа сыном Божьим. Это правда.}
\people{Сын, значит - часть?}
\soul{Нет. Это Бог.}
\people{Это тоже бог?}
\soul{Нет, не «тоже». Это Бог.}
\people{Ага, понял. Хорошо. Как понять смысл «Бога Отца и Бога Сына и святого духа»? Значит, Бог не есть святой дух? Или это всё понятие «бог» включает?}
\soul{Вы разделили даже себя. На физическое, духовное и душевное.}
\people{Понял.}
\soul{Но вы же едины. Вы согласны с этим?}
\people{Согласен. Когда убийца убивает кого-нибудь, ну, допустим, с корыстными целями…}
\soul{Нельзя убивать с любыми целями. Вы можете представить себе убийцу с прекрасными целями?}
\people{Ну, допустим, человек мучается…}
\soul{Нет! Нельзя убивать в любых случаях. Вы создали эту жизнь, чтоб убивать её?}
\people{Понятно. Я не собираюсь убивать, по крайней мере, осознанно, в физическом плане.}
\soul{В любом плане нельзя этого делать. Осознанно, неосознанно – это лишь только самообман и самооправдание.}
\people{Ясно.}
\soul{Простите, но Бога вы не можете обмануть.}
\people{Это я знаю. Если правильно понимать изречение «бог в нас и мы в боге», значит, получается - «мир в нас и мы в мире»?}
\soul{Малое - в большом, большое - в малом.}
\people{Ясно.}
\soul{Нельзя брать геометрию. Геометрически у вас ничто не получится. Но вы должны принять это на веру, тогда вам удастся понять это. Вы поймете это не физически, а в духовном плане. Всё надо понимать через духовное, через душу.}
\people{Чувства и есть душа? Чувства?}
\soul{Это реакция. Это реакция души.}
\people{Так скажем, - одна из ипостасей души?}
\soul{Нет. Это реакция организма. Если хотите, то мир эмоций - это мир физический. Мы говорили вам об этом.}
\people{Вы сказали, что каждый миг нашей жизни показывает, какие мы ``чёрные''.}
\soul{Я не говорил этого.}
\people{Ну, я имею в виду «отрицательные» или «положительные».}
\soul{Простите, вы меняетесь, вы - вода. Вы можете быть сейчас ``чёрным'', завтра будете ``белым''.}
\people{Допустим, я буду меняться так быстро, что события просто не успеют откликнуться на мои реакции.}
\soul{Так не бывает.}
\people{Допустим, я буду сидеть здесь, взаперти…}
\soul{Происходит всё мгновенно. Здесь нет понятия о времени и расстоянии.}
\people{Хорошо, я стёр эти понятия. Значит, на меня будут действовать внешние силы?}
\soul{На вас сейчас не действуют внешние силы?}
\people{Ага, ясно.}
\soul{Как вы понимаете - внешние и внутренние?}
\people{Внешние силы – это, допустим, физический уровень. Допустим, если я кого-то ударил, то кто-то ударит меня.}
\soul{В физическом плане?}
\people{Да, в физическом.}
\soul{Вас волнует больше - духовный или физический?}
\people{Меня всё волнует. Я везде живу.}
\soul{С какой целью вас хотят ударить, с какой целью ударите вы?}
\people{Значит, от цели может зависеть и, так сказать, возмездие?}
\soul{Нет. Деяние зависит от мысли.}
\people{Угу, т.е. фактически, благой мыслью оправдано всё? Любое деяние?}
\soul{Благой мыслью вы не сотворите зло.}
\people{Угу. Логично. Хорошо, допустим…}
\soul{Давайте так – вы смогли бы убить человека, зная, что из-за него погибнет весь мир? Вы бы смогли это сделать?}
\people{Не знаю.}
\soul{Что вас останавливает? Вы же ``спаситель всего мира'', если убьёте его. Что вас остановило?}
\people{Душа.}
\soul{Вы ответили на свой вопрос.}
\people{Я понял.}
\people{От чего зависит, вот если я не задам вопрос, вы говорите о потере контакта. Мы что, сливаемся с общим фоном?}
\soul{Это – наша энергия. Ваш вопрос - и мы питаемся им.}
\people{Иногда на вопрос, вы отвечаете вопросом. Это с единственной целью…}
\soul{А как вы ведёте диалог?}
\people{Понятно.}
\soul{Вы заметили, какая сейчас идёт реакция?}
\people{Да.}
\soul{Не теряйте её.}
\people{Что вы можете сказать о самом ``переводчике''? Допустим, почему он имеет много секретов?}
\soul{Простите, это его личное.}
\people{Понял. Значит, подразделяется на личное и общественное?}
\soul{Нет. Мы действуем по принципу ``не навреди''.}
\people{Угу, вы действуете с наших… законов?}
\soul{Нам приходится действовать в ваших законах, иначе, мы бы не могли с вами разговаривать.}
\people{Хорошо, если, допустим, я стёр в себе некоторые наши законы, вы можете обо мне рассказать?}
\soul{Вы стёрли эти законы?}
\people{Я стираю эти законы.}
\soul{Сотрите. Тогда у вас будут другие вопросы.}
\people{Понятно. Мир имеет формы… объёма?}
\soul{Нет.}
\people{Мир не имеет никакой формы?}
\soul{Нет. Может быть - бесконечность и форма совместимы реально? Вы можете представить бесконечную форму?}
\people{Н-да, не могу. Вы знаете, что сегодня произошло? До этого момента.}
\soul{Мы знаем всё, что знает ``переводчик''.}
\people{Понял. Кто ему сказал, что пробита стена и открыт склад?}
\soul{Если хотите, то здесь идёт «синтез».}
\people{Хорошо, в синтезе скажите - как?}
\soul{Мы говорили вам, что мы вас никогда не покидаем. Вы слышите нас, как шёпот. И в определённом состоянии вы слышите нас. И мы слышим шёпот, и мы тоже ведём так же. Мы также ведём себя так же, как и вы. Мы так же ищем контакта. Мы этим ничем не отличаемся от вас. Мы говорили вам, что мы едины, и что мы похожи.}
\people{Я понял.}
\soul{Мы тоже слышим ``шёпот''.}
\people{``Маятник'' - одно из условий нашего контакта? (махи руки переводчика во время сеанса контакта. прим.)}
\soul{Да.}
\people{Когда он услышал, как вы говорили ``синтез''…}
\soul{А вы вели диалог?}
\people{Я спрашивал у него, где пробита стена, в какой части, и он отвечал. По моему, так.}
\soul{Не спрашивая, он бы тоже ответил.}
\people{Ясно. Получается, что я предугадал ответ? Ну, скажем так…}
\soul{Я понял вас. Вы знаете всё. Нет человека, который не знал бы ничего. Вы знаете всё. КАЖДЫЙ ИЗ ВАС ЗНАЕТ ВСЁ И ВСЯ. НО ВЫ НЕ МОЖЕТЕ ЭТО ВЗЯТЬ. Даже у вас, в подсознании, нет прошлого, настоящего, будущего. И даже подсознанием вы едины. Иначе вы бы не искали этого. Не имея зерна, не вырастить ничто. Вы поняли?}
\people{Понял. Получается, что наш мозг - это тормоз?}
\soul{Нет, вы должны приучить его. Это МАШИНА, которой вы не умеете управлять. Но, согласитесь, что от этого машина не может быть хуже, от вашего неумения. Вы согласны?}
\people{Согласен. Каковы пути, чтобы научится…}
\soul{Только в духовном плане.}
\people{Только в духовном.}
\soul{Вас пронизывает энергия, и вы должны научиться ей управлять.}
\people{Вы тоже что-то должны научиться?}
\soul{Да.}
\people{Не секрет?}
\soul{Нет.}
\people{Что?}
\soul{В духовном плане.}
\people{Ага, значит, для вас есть тоже духовный план?}
\soul{Он пронизывает все миры. И, заметьте, энергетически, - мы не пронизываем ВСЁ. Для нас тоже есть - недоступное. Ибо душа – это не энергия. Это выше. Иначе, было бы даже, что мы пронизываем и душу. Значит, душа была бы низше нас. Вы согласны?}
\people{Логично. Символ пирамиды – это символ стремления вверх?}
\soul{Нет, вам придётся ещё расшифровать это.}
\people{Справедливо ли утверждение одного человека, что на более низком уровне, допустим, наша Земля, это по нашим понятиям ``низкий'',- время течёт медленней, чем на более высоком уровне?}
\soul{Нет.}
\people{Наоборот?}
\soul{Время НЕ ПОСТОЯННО. Есть и более низкий уровень в вашем понятии, где время течёт и быстрее и медленнее. Если хотите, то расстояние – это, всего лишь составляющая времени. И, если вы хотите переместиться, то вам надо ИЗМЕНИТЬ ВРЕМЯ. Надо НАСТРОИТЬСЯ на это время, и вы будете там.}
\people{Существует ли в мире, на нашем, физическом уровне, способ достижения сокращения времени?}
\soul{Да. Вы всегда это умели.}
\people{Я имел в виду технические способы.}
\soul{Есть и технические способы.}
\people{У других цивилизаций?}
\soul{Нет. Она была и у вас.}
\people{Была?}
\soul{Да. И была потеряна. В частности, с приходом религии.}
\people{Религия - это зло?}
\soul{Нет. Вы превратили в зло.}
\people{Понятно. Любой ли технократический путь тупиковый?}
\soul{Да.}
\people{Чисто технократический. А если синтез души и тела?}
\soul{Это временно.}
\people{Потом, всё равно переходит в духовное?}
\soul{Да.}
\people{Значит…}
\soul{Это протез.}
\people{Понял.}
\soul{Задавайте.}
\people{Зачем сделали Сфинкса?}
\soul{Выражение себя.}
\people{Как назвать состояние, в котором находится ``переводчик''?}
\soul{Неконтролируемость.}
\people{То есть, умение не думать гораздо важнее умения думать? С ваших понятий.}
\soul{Если бы он не мог бы сейчас думать, то нам было бы гораздо легче.}
\people{Гораздо?}
\soul{Да. Нам тяжело бороться с вами. Вы очень сильны, и ваш мозг заблокирован.}
\people{Кто нам даёт эту силу, о которой всё время вы говорите – очень сильны? Или вы, что бы нас…(не договорил – ``не хотите обидеть''. прим.)}
\soul{Мы ищем эту силу.}
\people{Вы не так же сильны, как мы?}
\soul{Здесь нельзя так рассматривать. Вы сильны в своём мире. Мы сильны в нашем мире. Задавайте.}
\people{Сумасшедшие, кто такие? С нашего, вернее, с вашего понятия.}
\soul{``Не от мира сего''. В вашем понятии.}
\people{Это я знаю, что ``не от мира сего''. Они что, лучше воспринимают какую-то энергию, или как? Или вообще разучились управлять?}
\soul{Нет. Сумасшедшие бывают разные.}
\people{Ага… Что ещё можно казать?}
\soul{О сумасшедших?}
\people{Да.}
\soul{Сумасшедшие видят мир по-другому, поэтому и планы эмоций другие. Они более подвержены, более открытые.}
\people{Вы входите с ними в контакт?}
\soul{С ними не надо входить в контакт, они открыты.}
\people{Есть ``закрытые'' сумасшедшие?}
\soul{Нет.}
\people{То есть, получается, отключив логику, мы открываемся?}
\soul{Нет. Здесь немножко сложнее.}
\people{Как?}
\soul{Нам трудно объяснить вам. Ибо мы сами не знаем о вас.}
\people{Вы энергетический мир?!}
\soul{Да, но мы не пронизываем ваш мозг. Он закрыт. Забывайте об этом.}
\people{Так… Получается, в мозгу нет энергии?}
\soul{Нет.}
\people{Нет? В каком понятии?}
\soul{Мы, просто не можем проникнуть.}
\people{В мозгу своя энергия?}
\soul{Как вам объяснить… Дело в том, что, в вашем понятии, мы - другая энергия, и потому мозг принимает нас, как инородное. Мы говорили об этом.}
\people{Так. Т.е. на Земле мы разделили своим понятием даже понятие ``энергия''? Ведь энергия, в принципе, общая.}
\soul{Вы похожи на тех отшельников.}
\people{На каких?}
\soul{Уходить от мира. Вы и есть отшельники, в какой-то мере.}
\people{Да, справедливо. Значит, мы своими понятиями отрезали себя от остального мира. И стали жить.}
\soul{Вы хотели всё знать. Вы хотели везде и во всём найти смысл.}
\people{До сих пор. Так… Смысл искать не следует? Смысл в самой жизни?}
\soul{Надо смысл искать не в том направлении. Вы хотели всё конкретизировать.}
\people{Угу, понял. Мы хотели бесконечное вычислить конечным числом? Или разложить по полочкам, как говориться, единое?}
\soul{Нет. Вы пришли к морю, выбрали одну песчинку и сказали, что это песчинка и перестали замечать другие. Вы сказали, что все остальные песчинки такие же, зачем нам они нужны?}
\people{Но, даже вы… судя по вашей, вернее, по нашей логике…}
\soul{Мы ваши дети, мы были вместе. Вы разделились.}
\people{Получается, мы не разумные родители?}
\soul{Нет.}
\people{С вашей точки… (зрения. Прим.)}
\soul{Мы говорили вам, что вы родили мир, и мир родил вас. Здесь нельзя рассуждать логически.}
\people{Ну, ладно. Надеюсь, понимаю правильно.}
\soul{Нет, вы не сможете этого никогда понять.}
\people{Никогда?}
\soul{Пока вы мыслите логически,- никогда.}
\people{Вера – это не логическая мысль?}
\soul{Нет. Вы верите в Бога? Вы можете доказать его логически? Или можете опровергнуть его?}
\people{Что такое факт? Вот, допустим, я, замыслив что-то, испросив у Бога, как говориться, помощи, допустим, нахожу нечто, что я искал. И в довольно короткие сроки. Это я, значит…}
\soul{Вы хотите привести доказательства существования Бога?}
\people{Я веду диалог, я доказывать ничего не хочу. Допустим, для меня, чисто с логической точки зрения…}
\soul{Факт, это проявление реальности. Осознание её.}
\people{Ясно. С вашей точки зрения, чудеса возможны в материальном мире?}
\soul{В вашем зрении?}
\people{Да.}
\soul{Чудеса – это не понятое, а где-то в мирах, это нормальное явление. Там было бы чудесным, если бы этого не происходило.}
\people{Да? Интересно.}
\soul{Вы подумайте, здесь можно решить это логически.}
\people{Ну, да. Логически подтверждается. Всё верно.}
\soul{В Африке вы можете увидеть снег? Для них это - чудо, а для вас, это нормальное явление. Вы согласны?}
\people{Согласен. Что бы вы хотели, чтобы я у вас ещё спросил?}
\soul{Можете спрашивать, что вы хотите.}
\people{Что я хочу?}
\soul{Можете спрашивать всё, что хотите. Но, у нас одно условие, - давайте всё делать быстро.}
\people{У вас лимит времени?}
\soul{Нет, у вас.}
\people{В чём это выражается или выразится?}
\soul{В ``переводчике''.}
\people{Ясно. У него пропадает желание?}
\soul{Нет.}
\people{В чём же дело?}
\soul{Плохой заряд.}
\people{Что надо делать, чтобы был лучше?}
\soul{Вы не сможете этого сделать.}
\people{Но, вы скажите.}
\soul{И этого не можем мы. Мы не знаем природу нашего заряда, как и вы не знаете природу вашего заряда.}
\people{Понятно. Так, тогда такой вопрос: вы сказали, что мы выбрали песчинку, сказали, что эти песчинки похожи на другие песчинки, и нечего искать смысл. Так?}
\soul{Да. Вы взяли одну песчинку и стали рассматривать её, забыв о других.}
\people{Хорошо… Сейчас, вы секунду назад, вы сказали: ``Где право, где лево в шаре?''}
\soul{Относительно точки отсчёта можно сказать.}
\people{Значит, нам надо выбрать эту точку отсчёта.}
\soul{Нет. У нас это нельзя сказать.}
\people{В вашем мире, или у нас с вами?}
\soul{Нет. У нас нельзя найти точку отсчёта. И мы не можем сказать – вы или ``выше'' или ``ниже''.}
\people{Ясно. То есть, аналогов нет?}
\soul{Нет. Это, просто, другой мир. У нас тоже есть, как и в вашем понятии,- враги, друзья, любовь. Мы так же ищем ответы на многие вопросы. И не находим их.}
\people{Хорошо. С вашей точки зрения…}
\soul{У нас другие точки зрения. У нас не будет тут контакта.}
\people{Ну, вы, как-то нас оцениваете по каким-то критериям своим?}
\soul{По своим?}
\people{Да.}
\soul{Количеством заряда. Но это такое общее число, что мы не можем точно сказать.}
\people{Ясно. Грубо говоря…}
\soul{Мы изучаем вас, вы изучаете нас.}
\people{Угу… Вы говорили, что имеете связь со многими мирами. Так?}
\soul{Мы пронизываем многие миры.}
\people{Пронизываете… Значит, вы считываете информацию или… Я не знаю, как… Имеете представление, грубо говоря.}
\soul{Нет. Вы не поняли. Это вы делите на разные миры. У нас они едины, эти миры. Мы просто живём.}
\people{Ага!}
\soul{Аналог, это у вас – животные, растения, люди.}
\people{Понятно.}
\soul{Вы согласны, что у вас множество миров даже здесь?}
\people{Согласен.}
\soul{Вы живёте в них. Во всех этих мирах, но вы их не ощущаете мир растений, хотя и живёте в нём. Вы согласны?}
\people{Согласен, согласен. То есть, можно представить вас в вашем мире, как, допустим, нас, людей, в нашем мире. То есть, те же цветы…}
\soul{Да.}
\people{Грубо говоря, мы – какой-то там ``цветок'', который цветёт не тем цветом?}
\soul{Я говорил о наших мирах. И приводил ваш мир в пример. Тут вы являетесь человеком, в нашем мире мы являемся тоже - человеком.}
\people{Ну, понятно.}
\soul{Относительно остальных.}
\people{Так. Хорошо. В вашем мире кем мы являемся, относительно нашего мира? Допустим; цветок, трава, там, песчинка?}
\soul{Вы на физическом плане. Вас нет в нашем мире! Есть только составляющая. Это - душа.}
\people{Так. Простите, я…}
\soul{Душа пронизывает все миры.}
\people{Угу…}
\soul{Она вездесуща. Мир она не делит никогда. И ей, аналогов мы уже не можем придумать.}
\people{Ладно. То есть, допустим, скажем ``мировая связь''.}
\soul{Нет. Душа - это всё. Её нельзя разделить.}
\people{Ну ладно. Значит, мы не производим на ваш мир никаких… Как сказать? Ощутимых…}
\soul{Практически.}
\people{Практически? Значит, теоретически, всё-таки, производим?}
\soul{Да.}
\people{Чисто по духовной линии?}
\soul{Да.}
\people{То есть, мы засоряем, грубо говоря – как в нашем понимании – эфир?}
\soul{Почему только засоряете? Не только.}
\people{Ну, не только! Но, в сумме, где-то, получается ``ноль''. С минусом.}
\soul{Нет. В сумме – вы засоряете.}
\people{Значит, пред ``нулём'' – ``единица''… Или, не одна, даже… Ну, ладно. Значит, мы засоряем не только ваш, а и остальные все миры ваши… Ну, что вы считаете миром.}
\soul{Мир – един!}
\people{Простите. Каюсь.}
\soul{Вы поняли? Вы засоряете везде. Или творите добро везде. Во всех мирах. Даже в физическом плане.}
\people{Хорошо. Вы никогда не задумывались, что мы такие, как говорится -``сорильщики''…}
\soul{Нет.}
\people{Нет, я не о том.}
\soul{Мы и завидуем вам, и ненавидим вас.}
\people{Ага… Ваша ненависть как-то откликается на нашем мире?}
\soul{Да.}
\people{И любовь?}
\soul{Да.}
\people{Спасибо вам за всё.}
\soul{Как вы относитесь к людям? Кого-то вы любите, кого-то ненавидите.}
\people{Согласен.}
\soul{А к кому-то вы равнодушны. Так же относимся к вам и мы.}
\people{Но когда я, допустим,…}
\soul{Мы чувствуем вас, как заряд – чуть больше, чуть меньше. Мы воспринимаем вас, как заряд. Мы не можем раскрыть вашу душу. Вы для нас очень интересны. Мы знаем, что мы – часть ваша, и что вы – наша часть, но нам не помогает это.}
\people{Да-а… Как сиамские близнецы – каждый хочет кушать. Логично?}
\soul{Нет.}
\people{Нет? Как же тогда? С точки зрения логики есть аналоги – ``мы – часть ваша, вы – часть наша''? Или вообще, не следует…}
\soul{Мы приводили примеры. Вы родили миры, и мир родил вас. Я приводил вам пример из фантастики, написанной вашими людьми.}
\people{Угу. И вашими - тоже.}
\soul{И логически рассуждать тут нельзя.}
\people{Понятно.}
\soul{Если хотите, это – логический парадокс.}
\people{Да. ``Парадокс времени'', я думал об этом.}
\soul{Будущее должно быть тогда, когда есть прошлое. Но прошлого – нет. Из будущего пришли и сделали прошлое. Вы можете объяснить, что здесь было первое, прошлое или будущее?}
\people{Ясно. То есть, существует какой-то… Ну, не ``существует'', - грубо говоря, в нашем логическом понятии – есть какой-то ``мост'', который соединяет и прошлое и будущее?}
\soul{Нет. Нам не на чем показать. Здесь нет аналогов, и логически здесь нельзя размышлять.}
\people{Хорошо. С помощью чувств вы мне можете описать? Есть ли у нас такие чувства, которыми можно описать бесконечность?}
\soul{Да.}
\people{Скажите их.}
\soul{Это, жажда искания. Жажда любви.}
\people{Угу… Понял. Так…}
\comment{(часть записи не сохранилась. )}
\people{Ну, не вторгаясь в тайны переводчика, что вы можете о нём сказать?}
\soul{Конкретно.}
\people{ Можно и конкретно, если у вас различия есть ``конкретности''.}
\soul{Что именно?}
\people{Допустим, в духовном плане.}
\soul{Мы можем сказать его краски.}
\people{Хорошо.}
\soul{Синий. Вы помните, что это означает?}
\people{Да.}
\soul{Красный?}
\people{Да.}
\soul{Розовый?}
\people{Да.}
\soul{Зелёный?}
\people{Да.}
\soul{У него есть все краски.Простите, он имеет весь спектр.}
\people{Понятно. Палитра. Вы говорили, что краска есть, но может иметь, ну, с нашей точки зрения,..}
\soul{Она имеет множество оттенков. Ваша задача – сделать их равномерно. Тогда вы будете излучать белый яркий цвет. Это когда не преобладает ничто.}
\people{Хорошо.}
\soul{Простите, если в вас будет преобладать огромная любовь, но остальное будет задушено, - что это будет?}
\people{Что? Эгоизм?}
\soul{Нет. Это будет просто ``голая'' любовь, которая не принесёт ничего. В вашем понятии ``любовь''.}
\people{Угу… Уже, кое-что, близко… Есть ли, допустим, красный цвет ``положительный'' и красный ``отрицательный''?}
\soul{Да, конечно. Все цвета являются положительными или отрицательными. Красный цвет – это огонь. Вы помните?}
\people{Да.}
\soul{Огонь может согреть и может обжечь.}
\people{Значит, справедливо утверждение, что ``не имеющий сил для добра…}''
\soul{Да!}
\people{``…не имеет его и для зла''. Так… Значит, если, допустим, лично я буду делать великое зло, допустим, очень великое для людей, естественно оно откликается на все миры, я буду там плохо думать о всех, плохо делать всем, вот…мне будет жизнь… То есть, зло будет воздавать злу, получается? Меня будут такие же злые бить по спине, потому, что добрый не ударит.}
\soul{Нет, вы можете творить зло, и для вас это будет только доставлять удовольствие и тогда нельзя будет сказать, что вы плохо живёте. Но вам это воздастся.}
\people{Когда?}
\soul{И как правило, заметьте, - если вы были в прошлой жизни, допустим, злым, то в следующий жизни, вы будете добрый, но то зло, которое вы принесли в прошлой жизни, оно придёт к вам, и вы уже будете ``не заслужено обиженные''.}
\people{Угу. Но самое, я считаю, не справедливое, это то, что мы не будем помнить, за что мы были обижены.}
\soul{Это справедливо.}
\people{Почему?}
\soul{Нам трудно вам это объяснить, но это справедливо. Скажем так, первое – химически, - память. Ваш мозг работает с химической памятью, помните об этом. Другой нет. Вы согласны?}
\people{Не согласен.}
\soul{Второе, вы можете помнить духовно. С чем вы не согласны?}
\people{С химической памятью.}
\soul{Ваш мозг умер…}
\people{Хорошо, мозг умер…}
\soul{…вы получаете новый.}
\people{Как объяснить, когда люди… дитя… ну, ребёнок…}
\soul{Пожалуйста, не перебивайте. Я сказал - первое.}
\people{Ладно.}
\soul{Второе. Нужно иметь очень много сил, чтобы помнить многое. Простите, существует такой вид наказания, когда вы помните всё.}
\people{Вы ответили?}
\soul{Вы хотите многое забыть в своей жизни?}
\people{Я?}
\soul{Да.}
\people{Я, практически, помню всё.}
\soul{У вас есть моменты, которые вы хотели бы забыть и были бы счастливей?}
\people{Допустим - да.}
\soul{Зачем тогда вы спрашиваете? Представьте теперь ``груз'' ваших времён…}
\people{Так… Ну, я, всё-таки, хотел вопрос задать о химической памяти. Вы слышите?}
1-2-3…
\people{Ответьте. Вы слышите?}
\soul{Сильные эмоции переводчика. Он плачет.}
\people{Вам мешают сильные эмоции переводчика?}
\soul{Почему вы плачете? Задавайте.}
\people{О химической памяти. Так, вы сказали, что химическая память теряется. Как объяснить случаи, когда…}
\soul{Я объяснил вам. Я вам сказал, есть и духовное.}
\people{Ага.}
\soul{Вот и ответ, почему переводчик плачет.}
\people{Он вспомнил свою жизнь. Так?}
\soul{Да.}
\people{Прошлую? Не секрет, кем он был?}
\soul{Это его…}
\people{Тайна. Он будет помнить, когда встанет, свою жизнь прошлую?}
\soul{Нет.}
\people{Если я задам вопрос, положу микрофон и выйду, чтобы не слышать, вы надиктуете на ленту, кем он был, для него? Он потом может стереть.}
\soul{Вы можете спросить.}
\people{Но вы ответите?}
\soul{Зависит от вопроса.}
\people{Хорошо. Расскажите ему…. Ну, когда он встанет, он услышит, - расскажите ему, не больше того, что он хотел бы слышать о своей прошлой жизни.}
\soul{Я понял вас. Мы сделаем проще. Он будет помнить. Но то, что мы хотим. Извините, то, что он может помнить, то, что ему можно, что разрешено.}
\people{Хорошо.}
\soul{И это уже его право, отвечать вам на подобное или нет.}
\people{Справедливо. У каждого в руках своя судьба.}
\soul{Вы спрашиваете, сколько жизней прожил он или вообще?}
\people{Ну, не вообще, а, допустим, когда он жил до этой жизни? В каком году? Это же не секрет?}
\soul{Нет, это не секрет.}
\people{В каком?}
\soul{Мы должны перевести. Объясните, что у вас является годом?}
\people{Год, это когда прошло четыре разных фазы погоды. Ну, грубо говоря, периода погоды.}
\soul{Нет. Нам трудно так считать.}
\people{Хорошо, с какой точки зрения вам лучше?}
\soul{Пожалуйста, в астрономических.}
\people{Астрономических. Угу… В астрономических… За единицу отсчета, что брать? Какую величину?}
\soul{Подумайте. Мы не можем сейчас войти в контакт с переводчиком.}
\people{Понятно. Сейчас подумаю….Допустим, последнюю комету Галлея, вы знаете?}
\soul{Нет. Давайте по-другому. Давайте по движению.}
\people{Хорошо комета проходит…}
\soul{Нет.}
\people{Нет?}
\soul{Вы бъясните мне по движению вашей планеты.}
\people{За год она описывает один оборот вокруг Солнца..}
\soul{1248-й год. Год смерти.}
\people{1248-й … А год рождения?}
\soul{Не знаем.}
\people{Вы не знаете?}
\soul{Нет.}
\people{Потому что переводчик не знает?}
\soul{Простите, ему сейчас тяжело. Он плачет. И нам трудно.}
\people{Но мне задавать вопросы или прекратить это делать?}
\soul{Задавайте.}
\people{Так, значит… Кем он был? Мужчиной, женщиной?}
\soul{Женщиной.}
\people{Это обязательный переход: мужчина–женщина, женщина–мужчина?}
\soul{Нет.}
\people{От чего зависит? От духовного?}
\soul{Нет.}
\people{От чего?}
\soul{Я говорил вам, что если вы были добрый, то станете злым. И - ``наоборот''. Всегда меняется полярность.}
\people{Хорошо. Доброй женщиной он был или нет?}
\soul{В вашем понятии?}
\people{Да. Ну, в земном понятии.}
\soul{Да.}
\people{За что ему такая жизнь? Если он был добрым в прошлом, за что ему зло здесь?}
\soul{Он получает зло? Он несчастлив?}
\people{Ну да, если зло - испытание любви, то он, можно сказать, сверх-счастлив.}
\soul{Хорошо, давайте так, гипотетический вопрос - как вы думаете, он будет плакать по этой жизни?}
\people{Если он будет плакать, то этого никто не увидит.}
\soul{Вы не правы.}
\people{Ну, он постарается сделать так, чтобы этого никто это не видел. Насколько я знаю его. Я не прав?}
\soul{Здесь, мы этого не можем сказать.}
\people{Понятно. Ладно, оставим эту спорную тему. Значит, он был женщиной. Доброй. Допустим, в какой стране он жил? Вы можете сказать?}
\soul{Нет.}
\people{У него нет понятия, где страны?}
\soul{Нет. Мы не можем войти.}
\people{Так, ясно. Фактически, мы сейчас с вами ведём диалог ``по телефону''?}
\soul{Нет.}
\people{Ну, теоретически, скажем так.}
\soul{Да.}
\people{А ``телефонист'', грубо говоря, сидит на станции и плачет. И не слушает нас.}
\soul{Это уже его… Поймите, нам сейчас трудно разговаривать.}
\people{Ясно.}
\soul{Поэтому мы не можем понять, что вы хотите сказать сейчас.}
\people{Понятно. Ладно, это все второстепенные вопросы. Хорошо. До 1248 года….?}
\soul{Давайте сделаем проще. Дайте счёт.}
\people{Обратный, прямой?}
\soul{Прямой.}
\people{1–2–3–4–5}
\soul{Она была крестьянкой.}
\people{Ясно. В России?}
\soul{Нет.}
\people{Значит, место проживание не зависит ни от чего. Это по Воле Божьей, скажем так?}
\soul{Не-ет, зависит.}
\people{От чего?}
\soul{От многих причин.}
\people{Понятно. Не будем перечислять. Хорошо. В том сеансе, где была Наталья… он знает, о чём я говорю? Знает. Так, вы говорили, что он видит биополя.}
\soul{Он видит только тогда, когда мы входим.}
\people{Угу. Вы сейчас вошли и видите биополя? Ну, он видит? Вы пользуетесь его эмоциями? Таков механизм?}
\soul{Мы пользуемся общим планом.}
\people{Понятно. Общим планом. Так, он дал определение здоровью по биополям. Верно ли его определение?}
\soul{Да, это наше определение.}
\people{Ага, то есть, вы свои понятия через его… Говорите своими понятиями через его понятия?}
\soul{Как вы думаете, мы ли ими говорим?}
\people{Его понятиями – свои выводы, скажем так.}
\soul{Нет.}
\people{Нет?}
\soul{Вы забываете о единстве.}
\people{Ах! Да.}
\soul{То, что говорит он, говорим и мы.}
\people{Ясно. Ну, ладно, представим, что это вы. Так будет проще обращаться. То есть, с вашей… с его, так сказать…}
\soul{Вы хотите знать достоверность этой информации?}
\people{Да. Подтверждение. Достоверно? Правильно ли он сказал?}
\soul{На данный момент, этого было так. Но вы – ``вода''. И поэтому вам нельзя гадать.}
\people{Понял.}
\soul{На данный момент правильно, но я не вижу сейчас. Хотя… за такой короткий срок вряд ли произошли такие большие изменения.}
\people{Вы не видите сейчас? Так, вы говорите, что то тело, в котором жила душа переводчика, было казнено за еретику. Значит, еретика тоже бывает и положительной и отрицательной? Ну, в смысле - доброта бывает и положительной и отрицательной?}
\soul{А вы подумайте, какой задали вопрос вы. Подумайте о 1248-й год. Вы подумайте, какие тогда были понятия о еретике, и какие сейчас?}
\people{То есть, он помогал людям…}
\soul{Она.}
\people{Она. Простите. Она помогала людям, скажем, с необычным… с необычных точек, своих знаний, и её сожгли на костре?}
\soul{Да. Она занималась философией.}
\people{Это что, участь всех философов?}
\soul{Нет.}
\people{Большинства?}
\soul{Нет.}
\people{Тех, кто идёт правильно?}
\soul{Вы не поняли. Вы считаете, что всех, кого казнили, были правильными или неправильными?}
\people{Я пока ничего не считаю. Я жду, что вы считаете, потом, соглашаться с этим или нет, я порассуждаю.}
\soul{Конкретно сказать? Простите, тогда было казнено очень много народу. И, очень много было казнено и правильно и не правильно, в вашем понятии. Вы хотите знать, была ли она казнена правильно?}
\people{Нет, я хочу узнать, была ли она казненная за данную, которую я сказал, причину?}
\soul{Простите, она была казнена, за то же самое, что она делает сейчас.}
\people{То есть, всё повторяется?}
\soul{Да.}
\people{И конец будет тот же? Ну, логически…}
\soul{Вас сейчас будут казнить?}
\people{Я не знаю, что будет чрез пять минут. Я могу только предположить, но я могу и ошибиться. Вы можете предположить? Раз вы вычисляете… Вы говорили: «Мы вычисляем»…}
\soul{Мы можем вычислить.}
\people{Так, хорошо. Вычислите.}
\soul{Но вы ``вода''. Вы течёте. Мы можем дать только вероятность.}
\people{Хорошо, дайте вероятность в процентном соотношении к ста. (100\% прим.)}
\soul{А казнь?}
\people{Казнь, допустим, самосуд.}
\soul{Поймите, если мы вам скажем, то так и случится. Мы не хотим быть вершителями судеб. Это ваше право. Это первое. Второе, - вы представляете казнь, как казнь?}
\people{Ну, казнь как уход из этой жизни телесной.}
\soul{Тяжелый уход тоже может быть казнью. Можно умереть от старости, и это тоже будет казнь. Можно погибнуть под автомобилем, и это тоже будет казнь. Можно погибнуть и от самосуда и от топора палача - это тоже будет казнь.}
\people{Понятно. Можно самому себя убить, и это тоже будет казнь.}
\soul{И можно умереть под топором палача, но это не будет казнь.}
\people{Ясно. Смотря, с какими мыслями. Так?}
\soul{Нет. В каком качестве будете на данный момент.}
\people{То есть, важно качество на данный момент, а не важно, в каком качестве ты был всё это время?}
\soul{До чего вы дошли.}
\people{Ясно. То есть, если, скажем, благое намерение двигало вами… вы шли…}
\soul{Простите…}
\people{Вы перебиваете.}
\soul{Да, мы перебиваем, но причина одна,- нам тяжело работать.}
\people{Ясно. Всё, вопросов нет… Э-э… Говорите.}
\soul{Он также был и мужчиной.}
\people{Год?}
\soul{Четырнадцатый.}
\people{От Христа?}
\soul{Нет. 1914-й… И он был казнён. Казнён, как провокатор.}
\people{Так. Это вы тогда …два года, по-моему, назад…}
\soul{Вы ошибаетесь. Три.}
\people{Три. Да, я могу и ошибаться. Три года назад. Это были не вы?}
\soul{Сущность одна.}
\people{Сущность одна. Мир един. Правильно, сущность одна. Всё даже логично связывается.}
\soul{Вы хотите поймать нас на нелогичности?}
\people{Нет, я хочу себя поймать, где эта логика не права. Так… Вопрос?}
\soul{Пока не было.}
\people{Пока не было чего?}
\soul{Дайте счёт.}
\people{1–2–3–4–5–6–7–8–9}
\soul{Простите, у вас такие эмоции.}
\people{Плохие?}
\soul{У вас… Вы очень интересны.}
\people{Лично я, или переводчик?}
\soul{Вы все.}
\people{А-а… Говоря ``лично вы'', вы, всё время имеете ввиду нас?}
\soul{Мы всё больше убеждаемся, что вам нельзя давать прошлое.}
\people{То есть, у вас есть какая-то своя логика? Раз вы в чём-то убеждаетесь, значит…}
\soul{Нет, мы видим ваши эмоции. Мы видим его эмоции.}
\people{Ну, извините, делая кому-то боль, мы тоже видим его реакцию и делаем из этого вывод. Это та же самая логика.}
\soul{Мы сделали вывод.}
\people{(Вздох)}
\soul{Вы не поняли меня?}
\people{Ага… Что мы похожи? Это вы имели ввиду?}
\soul{Нет.}
\people{А что?}
\soul{Я имел в виду прошлое.}
\people{Что, прошлое помнить нам, всё-таки, нельзя. Не доросли ещё?}
\soul{Судя по тому плану, который он создал сейчас – нет.}
\people{Мы создаём плохие планы?}
\soul{Нельзя так рассуждать.}
\people{Ну, скажем, для вас - отрицательные?}
\soul{Для него лично - да.}
\people{А для вас? Без разницы?}
\soul{Да, это физический план.}
\people{Когда вы говорите, ``без разницы'', это равнодушие?}
\soul{Нет.}
\people{Что это?}
\soul{Это физический план. И он плохо связан с нами.}
\people{Ну, можно представить это, допустим, наш план, который плохо связан с вами, в качестве… на нашем плане, ``легкой тучки на небе''?}
\soul{Нет. Давайте сделаем так. Физический план – это, к примеру, проводник. Мы – энергия, проходящая по этому проводнику…}
\people{А мы – сопротивление?}
\soul{Да, вы – сопротивление.}
\people{У нас высокое сопротивление!}
\soul{И вы мало влияете на проводник.}
\people{А-а! Ну, в общем понятии,- на проводимость мы влияем мало.}
\soul{Вы знаете, что высокочастотные токи в изоляторе являются проводниками?}
\people{Да.}
\soul{И вы согласны, что от качества диэлектрика влияет не сильно-то, в данном случае, зависит на проводимость проводника? Я объяснил вам?}
\people{Зависит частота и мощность.}
\soul{Хорошо. Давайте, – частота. Что мощность и частота, излучаемая нами одинакова, но вы будете менять проводники.}
\people{Ага, я понял. То есть, получается модуляция.}
\soul{Нет. Получается, что мы мало влияем на проводник.}
\people{Мы – мало влияем…}
\soul{Это и есть то, что вы называете ``равнодушие''. Это не равнодушие – это малость влияния.}
\people{``Малость'' бывает, от нулевой точки вправо или влево, то-есть, плюс-минус?}
\soul{Нельзя так говорить. В мире нет плохого, хорошего. Есть просто жизнь.}
\people{Ясно. Так. Вы перестали ему показывать его прошлые жизни?}
\soul{Мы ему не показывали.}
\people{Ну… делать так, чтоб он вспомнил.}
\soul{Мы этого не делали.}
\people{Хорошо. Согласен. Он перестал вспоминать?}
\soul{Да.}
\people{Ему спокойней?}
\soul{Нам тяжело.}
\people{Тяжело? От его эмоций?}
\soul{Простите, вы можете судить о надёжности контакта по правой и левой руке.}
\people{Угу.}
\soul{Вы помните, в каком состоянии была его левая рука?}
\people{Так… То есть, если она падает - контакт теряется. Правильно?}
\soul{Нет. Пальцы.}
\people{Пальцы я не видел.}
\soul{Зря. Пальцы были ``щепотью''.}
\people{Понятно. Тогда - почему и крестятся щепоткой. Так? Что это за символ?}
\soul{Это не символ.}
\people{Ну, с нашей точки, грубой.}
\soul{Это особый способ распределения энергии.}
\people{Оно благотворно влияет на…?}
\soul{Если вы креститесь…}
\people{Я крещусь.}
\soul{Если вы креститесь, как вы думаете, это благотворно влияет или нет?}
\people{Понятно. Так…}
\soul{Мы не поймали момент, как ему удалось уйти от нас, и вспомнить прошлое. И мы не можем сказать, как это произошло.}
\people{Хорошо.}
\soul{Если подобное будет повторяться, нам будет очень тяжело.}
\people{Ясно.}
\soul{Но будет тяжелее и ему.}
\people{Я не буду больше заводить разговор о прошлом.}
\soul{Нет, здесь не виноваты чисто вы. Мы не можем установить эту природу.}
\people{Хорошо, виноваты все. Раз мир един, то виноваты все. Правильно?}
\soul{Да, вы близки.}
\people{Близок, но не совсем? Скажем так, виновата какая-то…}
\soul{Простите, вы много задаете вопросов, в которые уже были заданы.}
\people{Простите, я это запомню.}
\soul{Но у вас уменьшается их количество.}
\people{Да, я вижу свою глупость. И умею делать выводы, по крайней мере, так, как я умею делать. Если вам это нравится, я очень рад.}
\soul{Я вас охлажу.}
\people{Хорошо.}
\soul{Ещё три вопроса, и вы будете на прежнем уровне.}
\people{``На прежнем'' – это как? На худшем?}
\soul{Простите, сколько было задано вопросов в прошлом и сколько сейчас? В количественном отношении - ещё три вопроса, и вы догоните. Вы сравняетесь с прошлым контактом. Вы поняли меня?}
\people{Понял. То есть, час мы выиграли. Понятно. Количественный показатель я вижу. И качественно, тоже, чувствуется… Счёт? Дать счёт? Я дам счёт? 1–2–3–4–5–6…8-9
Он вышел в норму? Вы слышите?}
\soul{Два вопроса!}
\people{Еще два вопроса?}
\soul{Пожалуйста, быстро!}
\people{Если найдёте свой ``второй уровень'', скажем так, дойдете до ``высшего'', передайте привет! До свидания.}
\comment{(10 секунд тишины/прим/)}
\soul{Всегда ищите! Всегда двигайтесь! Двигаясь – вы создаете энергию! Энергия эта наша, это ваша! Всё едино в этом мире! Пожалуйста, не останавливайтесь! Будьте всегда в пути! Ищите, ошибайтесь, но ищите! Вы помните…}
\people{Хорошо.}
\soul{Не перебивайте! Да, вы умеете строить воздушные замки, они вам нравятся, вы в них живёте, вы их бережёте, - они бесполезны, но вы их строите. Стройте всегда! Ничто не проходит бесследно! Нет понятия - ``плохое деяние'', ``хорошее деяние''. - всё зависит от того, для чего вы это делаете, - мысль! Помните! Ищите врагов, но не создавайте их! Иначе они погубят вас, и вы погубите их! Помните! Будьте всегда в Пути!}
\people{До свидания.}
\comment{9-8-7….1}

Конец контакта.

\section{Conclusion}
``I always thought something was fundamentally wrong with the universe'' \citep{adams1995hitchhiker}

Аоум. глава 21-я 19-05-1994г
Георгий Губин
\people{**}
 
 
\people{Сегодня 19–го мая. Мы после небольшого перерыва вновь собрались, что бы задать вопросы и поговорить с вами. Сегодня у нас присутствуют ещё новые члены. Мы знаем, что случайностей не бывает. То, что сегодня не состоялась съёмка – это по чьей вине? Это случайность? Или это может спровоцировано вами, или то, что мы у вас не спросили разрешения на съёмку? Есть ли ваше согласие  на такую съёмку?}
\soul{Спрашивайте}
\people{Переводчик себя плохо чувствовал? Вы не могли заставить работать его речевой аппарат, так что ли?}
\soul{Спрашивайте.}
\people{Ну, мы этот вопрос и задали. Как на вашем отношении вновь присутствующие люди? Как вы к ним относитесь?}
\soul{Мы когда-то уже отвечали вам.}
\people{Вам всё равно?}
\soul{Спрашивайте далее.}
\people{Так. Скажите, пожалуйста, мы не часто задаём такой вопрос: мы были когда-то в прошлых жизнях связаны с новыми членами сегодняшними? Соприкасались наши пути?}
\soul{Нет.}
\people{Ни коим образом?}
\soul{Нет. Неужели вы думаете, что новая жизнь ваша – повторение старого? Тогда у вас никогда не будет, в вашем понятии, нового мира. Подумайте.}
\people{Что-то… Ну, значит, мы когда-то могли встречаться,  если у нас есть личные симпатии. Может быть, мы соприкасались? Или это вовсе не обязательно?}
\soul{Поймите же, в вашем понятии, вы повторяете и повторяетесь, Неужели вы думаете, что со всеми, с кем встречались в прошлом, вы встретитесь и здесь? Вы встретитесь только в том случае, если вы поступили с ними не правильно. Или они с вами. И, поймите, чтобы расти, вы должны иметь новые и новые. Спрашивайте.}
\people{Хорошее замечание. Ну, получается так, что с другими членами, которые участвовали в сеансах, мы сталкивались. Хорошо. Мы не поняли  вашего ответа и вашей реакции на то, может ли состояться съёмка. Как вы к этому отнесётесь? По чьей вине она сорвалась?}
\soul{Вы рабы техники. Будьте ими.}
\people{Хорошо. Значит, следующий раз, мы готовимся к съёмке, и надеемся, что вы нам поможете.}
\soul{Спрашивайте.}
\people{Хорошо. Мы продолжим вопросы, которые не закончили в прошлый раз. Их осталось немного.  Чем отличается медитация, и отличается ли, от концентрации или созерцания?}
\soul{Отличий множество. Что в вашем понятии медитация?}
\people{Ну-у… Это, более углублённое…}
\soul{Тогда найдите корень этого слова.}
\people{Рассредоточение.}
\soul{Разве?}
\people{Да.}
\soul{А вы подумайте. Мы вам говорили вам :''корень''. И что вы делаете в  обратном? Сравните себя сейчас и в глубокой медитации.}
\people{Ну, вот наш переводчик, он ведь находится в состоянии медитации, фактически? Мы правильно понимаем этот процесс? Нет?}
\soul{Нет.}
\people{А как вы бы его определили его состояние?}
\soul{Желание.}
\people{Ну, он же выступает как медиум?}
\soul{Нет. Мы когда-то говорили вам, и вы не помните, что вы отвечаете на свой вопрос. Вы. Тогда, подумайте, кто он? Он переводчик ваших желаний, ваших вопросов. Вспомните, мы говорили: задавайте вопросы. Мы питаемся ими. Вы помните?}
\people{Помним.}
\soul{Вот и подумайте. Вы задаёте вопрос и получаете ответ. Вы задаёте вопросы и получаете свои ответы. Он выполняет, всего лишь, ваши желания.}
\people{Но, вы не сможете меня уверить, что я знаю ответы на те вопросы, которые вот я задаю. Я если и знаю, то знаю очень поверхностно. А вы, порой, формулируете достаточно чётко.}
\soul{А вы когда-нибудь задумываетесь  над тем, что вы знаете?}
\people{Ну, если учитывать знания моей души, накопленные  за время реинкарнаций, может быть, действительно, знаем много. Но мы не научились извлекать из душ эти знания и хотели бы получить от вас эту ниточку, как извлечь.}
\soul{Каково же у вас сознание? Вы ничего не знаете? Вы не читали? Вы не писали? Скажите проще,  и будет честнее,– вы забыли и не обратили внимания. Ваши слова: `` Не слышу ничего нового.''  Как вы можете тогда сказать, когда вы обманывали? Сейчас?  Или когда говорили вы? }
\people{Ах…Это мнение не только моё, допустим, а мнение наших других людей, которые слушали эти разговоры и плёнки. Они действительно говорят: там нет конкретики. Там нет ничего нового, если по большому счёту. Ну, вы нас и предупреждали, что вы и не дадите нового. Но конкретику какую то хотелось бы. Это обесценивает наши контакты.}
\soul{Конкретно? Вы можете, конкретно, перечислить все жизни? Возьмите друзей ваших и конкретно расскажите об их характере. Вы можете сделать это?}
\people{Вы правы.}
\soul{Подумайте далее. Что спрашивает в вас? Сознание ваше? Сознание ваше получает много ``пищи''. У вас достаточно для этого техники и далее. Зачем же нам быть ещё одним магнитофоном?}
\people{Ну, по большому счёту, что вы от нас ожидаете? Чтоб мы научились думать. И, в то же время, не даёте нам ключей, как извлечь нашу память, которая…}
\soul{Мы не даём ключей для вашего сознания. Ибо оно и так закрыто.}
\people{А к чему вы можете дать? К сердцу? К душе?}
\soul{Мы же говорили вам. Чувства, чувства… Важно не слово, - что несёт… Спрашивайте.}
\people{Скажите. Такой вопрос – связан каким либо образом алфавит, хотя бы наш русский алфавит, с числовым рядом? И играет ли это какое-то значение?}
\soul{Да.}
\people{Знают уже люди об этом? Об этом сочетании и важности этого закона?}
\soul{Знаете и вы. Но вы опять не хотите вспомнить. Не хотите. Не ``не можете'', а не хотите.}
\people{Главное – желание?}
\soul{Давайте сделаем небольшое изменение – возьмите, и думайте только об одном и об одном. В вашем понятии – медитация. Вы сможете это сделать? Нет. Хотя, вы знаете все правила, но не можете.}
\people{Да. Такое есть.}
\soul{Вот, и подумайте. А вам достаточно конкретно сказано, как это сделать! Достаточно много описания техники, конкретно, что вы делаете с этим? Спрашивайте далее.}
\people{Да… Наше несовершенство тут сказывается. Вот, если по алфавиту, по нумерологии,  можем ли мы получить от вас в будущих сеансах ответы? Мы не готовили, так вот вплотную, эти вопросы, но мы можем рассчитывать на ваши ответы?}
\soul{Да.}
\people{Это интересная тема для вас?}
\soul{Для вас.}
\people{Ну, и для нас. Понятно.}
\soul{Поймите. Это ваша память. Это ваши миры. У нас – иное.}
\people{Хорошо. Мы будем готовить вопросы. Следующий вопрос – Эволюция человека идёт медленно. Если какой либо человек, который называет себя Учителем, посредством некоторых имеющихся у него оккультных знаний, воздействует на психику другого человека, вызывая у него открытие так называемого ``третьего глаза'' и другие способности. Ускоряется ли эволюция данного человека? Или она вредит ему, - такое  искусственное внушение? Открытие.}
\soul{Вы можете назвать учителей?  Те, кто приходит и говорят:''Учитель'', те губят. И подумайте, может ли ложь вам помочь?}
\people{Я в своей жизни не встречал таких учителей. Но некоторые из наших друзей находят их в своих же товарищах, кто обещает им открыть ``третий глаз''.}
\soul{Подумайте. Подобное – подобным. Приходит ``Учитель'' и говорит: Открою глаз вам.  Простите, что же тогда Иисус не открывал вам?}
\people{А вообще-то, ``третий глаз'' это утерянная способность человечества или же,  будущая его способность?}
\soul{И то, и другое. Это - первое. Второе – это,  всего лишь, техника. В более тонких{телах}, но техника. Потому, Истинный Учитель, не будет открывать вам его. Поймите.}
\people{И стремиться к искусственному открытию третьего глаза, что, допустим, мне хотелось бы, так…теоретически, но я не предпринимал. (попыток. прим.)}
\soul{Вы можете искусственно – любить? Ну, попробуйте.}
\people{Нет. Вы правы.}
\soul{А вот сейчас приезжает, к нам в город махариши, который даёт нам знания, чтобы очиститься и впустить Высший разум. Это такой же бред, как и все остальные?}
\soul{Подумайте! То же самое можно было сказать и об Иисусе. О любом пророке. Для многих он был ``бредом'', для многих – нет. Подумайте! С какой верой придёте и что хотите взять от него, то и получите. Откроете `` двери'', ``войдёте''? Нет. Что же вы хотите? Далее. Мы говорили вам когда-то о чакрах. Да, вы можете искусственно увеличить любую. Но что будет? Подумайте.}
\people{Неравновесие. }
\soul{Вами сказано: ``хороший человек умирает быстрее''. Вы помните?}
\people{Помним.}
\soul{Вот и подумайте. Ибо сердце открыто, но не умеет…}
\people{Защититься?}
\soul{Спрашивайте далее.}
\people{А вот,  искусственно открывать энергетические центры эфирного тела человека это не правильный ход?}
\soul{Давайте скажем так: Вы будете открывать искусственно или  ваш товарищ, друг, экстрасенс?}
\people{Ну, кто-то из экстрасенсов.}
\soul{Ну, вы подумайте, вы открываете или он? Много вы доверяете, держа открытой, свою квартиру? Много ли? А вы хотите открыть более!}
\people{То есть, не стоит идти на такие попытки и не стоит давать над собой производить такие эксперименты? Так?  }
\soul{Поймите. Вы пытаетесь уйти лёгким путём. Мы когда-то вам говорили о лифте и о рае. И вы хотите повторить.}
\people{Так…  Но мы знаем, что ряд болезней, заболевания человека, как раз из-за того, что закрыты какие-то энергетические центры, чакры. И их очистка, прочистка, даёт человеку исцеление. Разве не стоит этим иногда заниматься?}
\soul{Исцеление?}
\people{Ну, улучшение.}
\soul{Чем лечите? Чем лечите? Приходите и говорите: ``лечить буду, но покажите квитанцию!''  Чем же вы лечите?}
\people{Вы и эти нюансы знаете, да?}
\soul{Спрашивайте.}
\people{Тогда сразу… Не  собирался задавать вопрос… Но, у нас такая сейчас ситуация… Сейчас, ко мне лично, обратились за помощью мать и её дочь, очень такие плохие у них происходят явления. Похоже, какая-то сущность вселилась в обоих и командует,  вплоть до того, что дочь может закончить жизнь самоубийством.  Можем ли мы найти способ как-то помочь этим людям? Они, возможно, невинны в этой болезни. Как вы считаете?}
\soul{Ну, тогда что вам Христос? Как мы ответим вам? Подобное – подобным. Подумайте.}
\people{Их вина в том, что они допустили это в себе? Так что ли?}
\soul{Подумайте.}
\soul{Ну, теперь я вот, не могу равнодушно наблюдать , что они погибнут. Что дочь может действительно покончить жизнь самоубийством. Неужели не стоит в это вмешиваться? Они сейчас готовы и в церковь пойти и…}
\soul{Да? Тогда спросите, готовы они были это сделать ранее?}
\people{Нет. До болезни они, конечно, об этом не думали. А теперь готовы отдать всё, чтобы …}
\soul{Вот вам и мудрость.}
\people{Ну, вы так спокойно смотрите? Что -''пусть  погибают''?}
\soul{Нет.}
\people{А как? Вы не дадите нам совета?}
\soul{Простите. Почему вы, когда больно, - в вашем понятии ``прижмёт'', - только тогда… Но, почему нельзя сделать это ранее? Почему, вы живёте одним днём, - приходит последний день, и вы пугаетесь? Что вы делаете? Вы обещаете в тот день множество,  и вы готовы отдать всего и всё, и исполнятся  ваши желания. И немного пройдёт, чтоб вы опять жили одним днём. Так зачем же…}
\soul{1-2-3-4… }
\people{Ну… В этом наше несовершенство. Наверно, мы так… плохо устроены, что только когда увидим реальную…}
\soul{Поймите. Если будете говорить так, вы никогда не улучшитесь.}
\people{Ну, я  сейчас лично, в такой ситуации, что людей этих бросить не могу.}
\soul{Разве мы говорили вам `` бросить''? }
\people{А чем же я могу помочь? Только чтоб мне в жилетку плакались?}
\soul{А вам - конкретно?}
\people{Вы можете сказать, действительно, чем можно помочь? Может в церковь их послать? Может им найти колдуна, женщину-колдуна, которая этих сущностей выгонит? Насколько эти сущности крепко…}
\soul{Мы отвечали вам. Отвечали за теорию. Вспомните. С какой верой придёте, то и получите. И если вы…( срыв записи)}
 ( Переводчик на некоторое время  вышел из состояния контакта, и описывает свои ощущения, полученные во время  общения.)
 Продолжение контакта
\people{Вы никак не можете прокомментировать ситуацию, которая произошла сегодня? ( насчёт не планируемого перерыва в контакте. Прим.)}
\soul{Нет.}
\people{Не стоит, да?}
\soul{Мы отвечали множество (раз) вам.}
\people{Такой вопрос, может быть, смутит кого-то, -  Христос и Иисус – это две разных великих сущности  или это один человек, одна сущность?}
\soul{Мы вам говорили о рождении Христа. Вы помните?}
\people{Да.}
\soul{Вот и подумайте. Иисус – человек. Далее, вы поняли?}
\people{В нем рождается Спаситель. Понятно. В каждом человеке это может быть.}
\soul{Подумайте. Подумайте, вы спрашивали о Суде. Вы помните? Вы спрашивали о Христе. Вы помните? }
\people{Да.}
\soul{А теперь подумайте. Вы сказали: Спаситель. Если вы приходите к экстрасенсу, и не верите ему. Спаситель он вам?}
\people{Нет.}
\soul{Какой бы силой он не обладал, - он не спаситель. Подумайте. В вашем понятии, Иисус Христос для многих не был Спасителем. Для многих он был ложью. Вы помните?}
\people{Да.}
\soul{Мог ли он помочь им? Вот вам и ``Врата''. А вы говорите – ``Конкретно''. Что вы хотите? Чтобы было сказано конкретно? Потому и говорят вам притчами. Потому и говорят вам: ``Откройте, ибо стучатся к вам''. Ибо тогда, вы поймёте более. Поймите. Для многих Слово – может быть пустым. Для другого – нет. Потому и не говорят конкретно. Ибо говорят не одному, а всем. И каждый найдет своё. Найдёт свои слова, которые смогут доказать ему – его же слова. Спрашивайте далее.}
\people{Мы надеемся, что именно эти наши разговоры с вами будут интересны и другим людям. Мы сейчас, как раз и затеяли попытку опубликовать часть наших разговоров  и ожидаем, какая будет реакция. Может быть, она будет негативная, какую мы не ожидаем.}
\soul{Вот и подумайте. Вы опубликуете…  А для себя? Для себя – что вы опубликуете?}
\people{Нет, вы правы были, что эти сеансы на нас подействуют, Лично на меня они действуют, Я как-то меняюсь. Но может не так быстро, как вам хотелось бы.}
\soul{А вы не пытайтесь, не пытайтесь меняться силой. И тогда, - будет вам.}
\people{Хорошо. Вот, как вы прокомментируете ещё  такую ситуацию? Недавно, один волжский экстрасенс сообщил мне, что он просканировал, проанализировал болезни человека по имени Иисус Христос, и болезни человека, чьё лицо отразилось на Туринской плащанице, и выяснилось, что это совершенно два разных человека. Более того, человек, который на Туринской плащанице был изображён ( диагноз он свой делал по фотографии) – это, скорее всего, простой смертный, наделенный многими болезнями и даже не очень высокой духовной силы. Как вы это можете прокомментировать?}
\soul{Хорошо. Вы ему поверили. Придёт к вам другой и скажет: ``Не было ни того и ни другого. Не было той плащаницы''. Вы будете нас спрашивать?}
\people{Ну, хотелось бы какую-то точку зрения и вашу узнать.}
\soul{Мы вам ответили только что. Только что. Был человек Иисус. Почему же ему не болеть всеми вашими болезнями, если он человек? Далее. Большинство святых ваших – болели. Разве нет? И как вы можете по болезни определить духовность?}
\people{Но есть поговорка –'' В здоровом теле – здоровый дух.''}
\soul{''В здоровом теле – здоровый дух''… Вот и подумайте. А было бы точнее, если б было наоборот.}
\people{Ну, да, верно. }
\soul{Вот и подумайте. Далее. Что вы хотите? Вы хотите быть вечно здоровым, вечно молодым. Иначе (говоря)– стоять на месте.  Многие, многие великие дела - делаются через боль. Подумайте, - если бы вы не болели, многое бы вы умели сейчас? Даже вашей техники не было бы, ибо не было бы в ней нужды, если бы вы не болели. Далее. Вы скажете: Зачем это нужно? Зачем нам, то тело, которое может болеть? Вы забываете о сознании. Поймите, мы никогда не говорили вам, что вы должны умереть сознанием и возродиться духовно. Нет. Дух хочет познать. Познать. И потому – сознание вам (дано). Подумайте. Далее. Если вы скажете – плоха машина, - из-за того, что не умеете водить, - вы будете правы?}
\people{Нет.}
\soul{Нет. Вот вам и сознание – машина ваша. Потому и говорим вам: Нет плохих. Просто, и очень просто, - неумение управлять. Вы обладаете всеми, всеми силами, что окружают вас, и более. Но, - неумение управлять. Если даже малое и придёт к вам с тех сил, вы пугаетесь.  И куда бежите? Вот и подумайте. }
\people{Скажите тогда, Туринская плащаница – что это такое? Есть факты, что она совсем не тех лет, чтобы Христос умер и был погребен, а лет 600 по радиоуглеродному анализу.}
\soul{Представьте, если мы скажем: Нет той плащаницы. Сколько убьём мы веры? Подумайте. А если мы скажем:  Да. Было и то и другое, - сколько мы возродим верующих? И получается, - мы виновны в вашей вере. Мы.  А вы говорите – зомби. Вот вам и пожалуйста. Вот вам. Вы спрашиваете нас, а мы говорим вам: - Ищите. Ищите сами. Сами. Вы знаете. Вы знаете ответы на все ваши вопросы. Поймите, если вы задали вопрос, значит, ответ вы уже знаете. Еслиб вы не знали ответа, не было бы и вопроса. Всё просто. И, в той простоте,  вы боитесь….}
\soul{1-2…}
\people{Мы благодарим вас за такой ответ. Хороший ответ, нам нравится. Ещё такой вопрос – можете ли вы когда- либо отвечать на конкретные вопросы? Например, по целительству.}
\soul{Нет. Поймите, для каждого – своё. Как вы можете мне сказать о том и о том, не называя имени? Вы можете это сделать? И как мы можем сделать это, не называя имени? А что имя ваше? Неужели вы думаете, что вам даны имена просто так? Называя ваше имя, мы уже воздействуем на вас. Потому и не говорим. Вы же, чаще, принадлежите к другим. Спрашивайте.}
\people{Хорошо. Можно узнать ваше понятие Земной любви и Космической? Является ли любовь мудростью? И может ли стать человек мудрым, без любви?}
\soul{Множество. Вы можете нам привести множество примеров мудрости без любви. Вы – можете. И мы должны спросить  вас: Почему это у вас так бывает? Неумение управлять, незнание? Подумайте. У вас есть злые и добрые гении. Энергия одна – и у тех и у других. Одни силы, одни. Нет других. Иначе – вы противоречили себе. Есть силы божеские, и есть силы дьявола. Что же получается? Бог не вездесущ? Сила одна, а вот качество…как вы будете использовать…}
 1-2-3-
\people{Ну, слишком разнятся понятия любви земной и космической? По-вашему…}
\soul{Это в вашем понятии. Что вы знаете о космической любви? Ничто. Поймите, всё ``космическое'' вы представляете через ``землю''. Вы можете объяснить любовь? Нет. Вы будете пытаться найти аналоги. Аналоги из слов. Чувства вы не передадите.}
\people{Пока не будет у нас телепатии, так сказать, на Земле…}
\soul{Телепатии? А что будет, если вы будете обладать той силой? Что? Вы, пока, физически – преступники. А что будет далее, если вы останетесь  на том же уровне, с теми же понятиями, но будете обладать более большим могуществом? Что, вы будете воины ада? Спрашивайте.}
\people{Вы можете попытаться нам объяснить, что же такое, всё-таки, Жизнь?}
\soul{Жизнь? Ищите. Если вы найдёте ответ, что же такое Жизнь,  вот тогда вы будете мудры. Тогда вы не будете спрашивать о ``космическом'' и ``земном''. Неужели вы думаете, что вы живёте одной жизнью, а космос – другой? Вы возвышаете, и унижаете. Жизнь – одна. Вы делите. Делите так успешно, что путаетесь, путаетесь  в своих же фантазиях.}
\people{В нашем обществе, ну, наверно, не совершенном, часто людей делят на черных и белых, серых. Правильно ли и справедливо ли это деление?}
\soul{Нет истинно ``черных'' и истинно ``белых''. Всегда есть краски. Одних больше или меньше.}
\people{Хорошо. Как вы развиваете своё ментальное тело?}
\soul{У нас нет, в вашем понятии, ментального тела. Мы когда-то говорили вам, что это физика, только что более тонкая материя.}
\people{Ну, у вас же есть, наверное, духовный мир, ментальный мир?}
\soul{Нет. Это вы нашли названия. Неужели вы думаете, что вы действительно ``матрёшка''? Подумайте! Подумайте, какая вы ``матрёшка''?! Неужели вы думаете, что у вас действительно есть чёткие границы? У вас есть твёрдая материя  и более тонкая – тонкая и тонкая. Снимайте! Cнимайте. Вы сказали: ``семь тел''. Неужели вы думаете, что их действительно семь?}
\people{Ну, это классикой, вроде как аксиома становиться, - семь тел.}
\soul{Может быть всё проще? Вы просто видите. Или просто делите, чтоб было вам удобно? Возьмите и вспомните – один, два. Разве нет у вас больше чисел? Цифр..цифр..цифр.(идет сбой )}
\people{Это как цвета радуги, - плавно переходят один в другой. На востоке считается, что цифра семь…}
\soul{Да. Вы правы. Далее. Вы спрашивали, и мы говорили вам: сравните; камень, растение, вас, воздух и далее. Вы видите здесь утоньшение?  Вот вам - аналогия.}
\people{Скажите, существует ли связь животных и растений с их параллельными мирами? Именно с их.}
\soul{Вы, так горды, что забываете, кто же, всё-таки, вы, и что делает тело ваше.}
\people{То есть, мы относимся к животному миру и что, то, что и человек, то и животное. Так что ли?}
\soul{Мы вам отвечали на подобное. Мир един, а вы делите. Неужели вы думаете, что вы выше животных, растения и того же камня? Вы выше только в том, что в вас больше умения управлять энергией. В том и беда ваша и счастье. Ибо вы можете раскачать маятник ваш столь высоко и столь низко! Что не может сделать камень, ибо обладает меньшей энергией. Значит, в вашем понятии, вы можете сделать греха больше, чем сделает камень. Потому, вы, не помня, что это значит, но помня только чувствами, нашли и сказали: Младенец – святой!  Вспомните хоровод и истинное его назначение. Вы помните? Нет. В вашем понятии – это сейчас ``игра''. Тогда, всмотритесь, в вашем понятии,  существуют племена, которые ещё помнят, что такое хоровод.}
\people{Это важная физическая или физиологическая особенность?}
\soul{А вы подумайте.  Мы когда-то говорили о биополях. И теперь представьте.}
\people{Угу. Понятно. }
\soul{И вспомните. Мы говорим вам – вспомните. Ибо хоровод исчез у вас. Но есть, в вашем понятии, племена, и, заметьте, - мы говорим `` в вашем понятии'', - что знают истинное значение. Ваша физика… ( сбой контакта. Прим.)}
\people{А вот, действительно, что сообщество людей или животных могут рождать коллективный разум? То есть, двенадцать человек  почему-то становятся более умны, чем, два, три или семь человек. Или сообщество саранчи, допустим, или грызунов, в большой конгломерации, дают очень разумные действия. Это действительно – разум может объединяться?}
\soul{Вот смотрите. Вы говорили конкретно. Мы можем дать вам конкретно на каждое – каждое – каждое? Нет.  Потому и говорим вам -  ИНОЕ. Вы же, - не помните. Или не слышите. Мы отвечали вам, отвечали о розовом, отвечали об ауре Земли, и объясняли, что это такое, ауре города, и объясняли, что это такое. Вы помните?}
\people{Да, это припоминаем.}
\soul{Вот, конкретно о городе. То же самое – животное. Но вы уже не можете, не можете, что могли бы ответить то же самое, но уже о животных. Вот и подумайте, где логика ваша. Вы горды, что логичны. Но, что ж не применяете?}
\people{Но двенадцать человек, объединённых вместе, они действительно рождают разум более совершенный?}
\soul{А вы вспомните. Вы даже это число нашли в святых. Даже в колесе Зодиака, у вас – двенадцать. Хотя их более. Даже сфер Небесных – двенадцать. Почему?}
\people{Планет двенадцать…  А вы не можете сказать, почему?}
\soul{Можем. Ибо вы когда-то были рождены под тем числом. Потому и чувства ваши подсказывают вам, но значения – не знаете. Подумайте. Возьмите цифры ваши и складите. Посмотрите на них, какой ряд создают они. Вспомните. И не забудьте включить туда числа, несущие несчастья. Они тоже должны стоять в том ряду.}
\people{Насколько мне известно, раньше, число это не считалось несчастливым, - тринадцать. А наоборот, считалось счастливым, довольно-таки, числом. Его в связи с чем начали считать несчастливым, - с тем, что тринадцатый был Иуда. Простите, что я имя произнёс. Вот с тех пор его начали считать несчастливым.}
\soul{Нет. Нет. Мы сказали вам, когда-то вы родились под числом двенадцать. Подумайте, что было дальше. Дальше, счёт ``тринадцать''. Вы согласны? }
\people{Согласен.}
\soul{Вы были рождены и были потеряны. Спрашивайте.}
\people{Хорошо. Существуют гипотезы, что Меркурий раньше был спутником Венеры. Так ли это?}
\soul{Нет.}
\people{Существует ли десятая планета в Солнечной системе и каковы её параметры?}
\soul{Вы же, только что говорили о двенадцати. И спрашиваете нас …}
\people{А где двенадцатая?}
\soul{Ищите. Когда-то, их было семь. И когда говорили `` восьмая'' – спрашивали  так же: ``Верно ли?''}
\people{Ну, человечество когда-то убедится, что их будет двенадцать? Что их двенадцать?}
\soul{Да.}
\people{Это скоро будет или…}
\soul{У вас уже сейчас идут споры.}
\people{Хорошо. Существуют ли искусственные торможения извне в создании установки по осуществлению гравиально-термоядерного синтеза? Мешает ли нам кто-нибудь?}
\soul{Поймите. Вы нашли способ, один из множества способов, выжить далее. Выжить, или умереть, - вот вопрос!}
\people{То есть, вы связываете это с таким понятием – выживет ли человечество? Стоит ли нам заниматься этим, да?}
\soul{Что делаете вы? Что? }
\people{Ну, сейчас разрабатывается установка Токомак уже в боевом назначении. То есть, огромных размеров, в отличие от лабораторной.  Это для человечества большая беда будет, да?}
\soul{Как вы думаете, это будет беда вам или счастье?}
\people{Мы ожидаем, конечно, что ничего хорошего из этого не выйдет.}
\soul{Поймите. Раньше, вы видели глаза противника. И потому, могли спросить – ``За что?'' и `` Почему?'' И потому,  благородства было более! А что сейчас? Что делаете вы? Вы можете нажать ``кнопку''. Согласитесь, убить множество, не зная об этом и не видя, как умирают и гибнут, легче, чем убить одного, но самому.}
\people{Но это строиться не для того, чтоб убивать. Для того, чтоб получать знания людям из самых лучших побуждений.}
\soul{Да? Простите, а вы убиваете для чего? Из лучших побуждений? Назовите мне хотя бы одно, чтоб было бы действительно, для ``лучших побуждений'' и не применялось в ином? Найдите. Найдите и вспомните, какие вы, всё-таки, по количеству? Вы помните? Вы говорили о расах. Спрашивали нас Бхаговат (Гиту). Вот и подумайте. Многие шли вашими путями. И что, где они?}
\people{И кому-то неугодно создание такой системы оборонного обеспечения?}
\soul{Нет. Кому-то неугодно, чтоб вы были самоубийцами. А вы – самоубийцы, и, чаще,  палачи, которые казнят себя. Медленно, но успешно. И вы это называете ``Жизнью''.}
\people{Скажите, а вот древние города, майя там, в Индии, - брошенные, - неизвестно, куда делись люди. Там ни останков, никого нет. Вы не подскажете, куда они ушли? В другие места, или что с ними было? Эпидемия там или что?}
\soul{Давайте, мы вам ответим на все, на все вопросы. И что будете делать? Многому вы научитесь? Вы будете сочинять новые вопросы. Только и всего.}
\people{Мы хоть не на ощупь будем идти. Вы хоть направление дайте нам.}
\soul{Наощупь? Сознание ваше идёт наощупь, но не чувства ваши. Вы же, слушаете, много ли, чувства?}
\people{Скажите. Вот в прошлый раз мы остановились и не знали, стоит ли предотвращать и устраивать борьбу против закачки сточных вод в подземные пласты в нашем регионе. Стоит ли бороться против этой закачки?}
\soul{Если вы не будете бороться, то вы, когда-то в будущем,  будете уничтожены. Уничтожены, своими же ``трудами.'' Вот вам и ответ. Вы, благими намерениями хотите сделать то? Да. Только отсрочите. А ваши дети   будут гибнуть от ваших деяний. Но вам легче, что это будет там, когда-то, когда вас уже не будет.}
\people{Но дети будут гибнуть, да? И мы сами тогда. Ведь вы говорили уже. А если мы воспользуемся этой контактной ситуацией,  сошлёмся на ваш авторитет, это вызовет какую реакцию у людей? Негативную или, всё-таки, положительную? Может, всё-таки, прислушаются?}
\soul{Давайте сделаем так, – возьмите, выйдите и спросите, каким авторитетом обладаем мы?}
\people{Пока – невысоким.}
\soul{Подумайте. Вы приходите и со знанием, сознанием доказываете, что то-то и то-то вредно. Или вы приходите, и сознанием и чувствами доказываете ему, что есть мы, которые хотят… и далее… и далее. Что? Поймут вас?}
\people{Скажите,  пожалуйста, могли бы вы сказать Единую Теорию Поля? Или дать формулу для единого описания всех взаимодействий? И существуют ли неизвестные человеку взаимодействия. Какие силы вращают биолокационную рамку?}
\soul{Вы. Вы. Мышцы ваши. А вот что управляет мышцами вашими? А мы говорили вам, и вы пытались доказать нам, что подсознание  знает более. Мы же говорили вам. Тело ваше видит более, чем сознание ваше. Потому и двигает. Неужели вы думаете, что придут какие-то необычные силы и будут двигать рамку вашу? Не проще ли это сделать на более другом уровне? Двигаете вы. Но кто будет подсказывать двигать её? Другое. Далее. Если мы говорим вам – Мир един! А вы нас спрашиваете о Единой Теории! }
\people{Но она есть, эта Единая Теория?}
\soul{А теперь, представьте, что будете обладать ею сейчас. Кем вы будете? Вы будете уметь всё! Для вас не будет существовать понятия ``экстрасенс'', ибо вы станете все  ими.  Для вас не будет существовать понятия ``электричество'' ,''ЭДС''…( срыв) }
\soul{Спрашивайте.}
\people{Переводчик себя плохо чувствует. Ему что-то мешает, да?}
\soul{Множество.}
\people{Откуда это множество приходит вообще? Что оно здесь делает?}
\soul{Вы живёте в том множестве.}
\people{Ну, про Теорию Единого Поля – ясно, что нам пока ещё рано.}
\people{Я так понял, что подход, в принципе, верен, но нам рано это узнать. Поверить.}
\soul{Для того, чтоб создать теорию ту, вам придется отказаться от многих законов, что созданы сейчас. В том и трудность ваша. Сумев уйти от того,  вы найдёте. И мы скажем вам далее, что будет время, и создаст, в вашем понятии, теорию, тот, кто будет мало знаком с наукой.}
\people{Скажите, а вот альтернативно существует такая теория мюонного взаимодействия, которая альтернативна теории полевого взаимодействия. Что вы можете сказать по этому поводу?}
\soul{Ничего. Это одно только из видений. И всё. У вас множество полей. Множество полей. А может быть у вас просто множество взглядов? Может то кусочки мозаики одной большой картины? Поймите, для того, чтобы увидеть всё – вы должны уйти. Уйти. Поймите. Подойдя близко к картине, вы увидите только фрагменты. Чтоб увидеть её, вы должны отойти. Отойти. И тогда будете её видеть целиком. Вы же, создали законы и не хотите уходить от них. Не мне вам доказывать, как вы держитесь за них.}
 
\people{Что представляют из себя квазары?}
\soul{Вы можете увидеть их, в вашем понятии, в одном из тонких …(срыв)}
\people{В одном из тонких тел?}
\soul{А вы подумайте. Куда вы будете утончаться?  Вы. Вы были камнем. Далее.}
\people{Так стоит понимать, что квазар это какое-то вместилище душ или какое-то  биологическое человеческое место?}
\soul{Нет. Только опять одно из тел. }
\people{Это астрономический объект?}
\soul{Поймите, есть понятие `` живое'' и ``не живое''. И есть оно у вас. Есть и там.}
\people{А есть ещё более удаленные объекты, чем квазары?}
\soul{Есть.}
\people{Какова их природа?}
\soul{А вы возьмите, и увидьте разницу. Возьмите далее. Сумейте. Сумейте продолжить линию.}
\people{А существуют ли, вообще,  границы Вселенной? Мы понимаем, что это условное, просто. Как … Если они существуют, как их следует понимать?}
\soul{Условно. Это то, что вы знаете, то, что вы видите, - то границы ваши. Далее. Да, граница физического мира есть.}
\people{Скажите, какую роль играют кометы?  Они, почему-то, мигрируют, возвращаются периодически. Даже есть такие, которые приходят в какую-то точку, разворачиваются и идут обратно. Какая роль комет?}
\soul{Силы. Силы. Многие из них приходят, чтобы уравновесить мир ваш. Другие приходят – наоборот.}
\people{Ну, это что, разумные их действия? Или около них, - действия?}
\soul{Тогда давайте спросим, разумны ли ваши машины?}
\people{В какой-то мере в них вложен, конечно,  интеллект человека. Создателя.}
\soul{Представьте, приходит посторонний и видит машины ваши. И что он скажет? ``Они приходят и уходят''.}
\people{То есть, это назначение - машина?}
\soul{Вы, знаете назначение комет?}
\people{Нет. }
\soul{Ну,  мы же говорим вам, - разумна ли  машина?}
\people{А кто же создатель таких машин? }
\soul{Вам подобные. Вы живёте в физическом мире и воздействуете на него. Вы создаёте машины. Вы уже научились создавать то, что может оторваться от земли. Продолжите линию. И тогда, вы сможете уже передвигать звёзды. Вы. Вы. И будет то время!  И, заметьте, можно передвигать их не зная о теории Единого Поля, а обладая только видами энергии. }
\people{Вот, есть предположение, что кометы, чаще всего, населены. Они, видимо, играют роль тех же автомашин, как у людей, так?}
\soul{Нет.}
\people{Наверно, они образовались в момент рождения Вселенной. Мы до сих пор можем видеть то вещество, которое было в первозданном виде при рождении?}
\people{Праматерия.}
\soul{Нет. Мы говорили вам о теории Большого Взрыва. Вы помните?}
\people{Нет.}
\soul{Далее. Вы спрашивали о границах. Да, в вашем физическом мире есть границы. А теперь представьте, если вы выйдите за те границы. Что будет? Та вселенная, из той, что вышли, для вас - будет рождена. Рождена вами, когда вы выйдите. Или, когда будете входить. Вот вам – теория Взрыва.}
\people{Так. А, скажите, пожалуйста, какова природа шаровой молнии? Почему в конце её существования наступает такой взрыв?}
\soul{Вы идёте – неумением управлять. Хаос. Как вы мне можете сказать – горение – и взрыв? Разница в чём? В скорости. Вот и подумайте. Вы же сами можете привести нам примеры, когда, всё-таки, не было, не было взрывов тех.}
\people{А почему шаровая молния, чаще всего появляется во время грозы?}
\soul{Ну, вы же сказали – молния.}
\people{Это условное понятие. Потому что во время грозы избыток электронов появляется, там и молния.}
\people{Не обязательно во время грозы.}
\soul{Поймите. Её природа, в вашем понятии, электрическая. Далее. Уже сейчас вы можете создавать их, но не умеете управлять.}
 
\people{Но природа (их) – электрическая?}
\soul{Да.}
\people{А они будут полезны для человека, если мы научимся управлять ими?}
\soul{Да.}
\people{В каком смысле? Они дадут энергию? Или какое-то оружие может быть?}
\soul{Что вы делаете сейчас, чтобы накопить энергию? Что?}
\people{Конденсируем…}
\soul{Простите. Энергии – гораздо меньше, чем тары. А теперь представьте, если вы, всё-таки, избавитесь от неё?}
\people{От кого? От энергии или от тары?}
\soul{От тары.}
\people{А вы как-то, по-моему, намекаете, что тара, это наше тело?}
\soul{Нет. Мы говорили о молниях. Мы говорили об электричестве. В данном случае тара ваша – электричество. А теперь представьте, если вы уйдёте от тары той? Подумайте. Самое идеальное, когда тарой является то вещество, которое должно храниться. Вы согласны?}
\people{Да.}
\soul{Вот вам – шаровая молния! Вот вам и электричество!}
\people{А нет ли какой-нибудь связи между шаровой молнией и управляемым термоядерным синтезом?}
\soul{Нет. В вашем понятии – нет.}
 
\people{А альтернативой может служить синтезу – шаровая молния?}
\soul{Нет.}
\people{Скажите, правда ли (это из других контактных ситуаций) что грозы, это проявление мыслительной деятельности Земли, как живого существа?}
\soul{Да. Здесь вы правы. Представляете, сколько гроз происходит в вас сейчас?}
\people{В нашем мозгу, да? Это близкая аналогия?}
\soul{Мы же говорили вам, Земля…. (срыв контакта)}
\people{Вы остановились, - Земля – что?}
\soul{Поймите, нет ничего мёртвого. Везде  есть  ЖИЗНЬ. И Земля ваша, если хотите, - Существо. Такое же, как и вы.}
\people{Что вы можете сказать о расширяющейся Вселенной? О точке сингулярности и пограничных состояниях?}
\soul{Мы говорили вам, возьмите надувной шарик и начните его сжимать. И представьте наблюдателя. Для одного – будет шарик расширяться, для другого – сжиматься. Относительно кого нам ответить вам?}
\people{Дело в том, что разбегание галактик, свидетельствует о том, что это всё равно друг от друга разбегаются.}
\soul{А может, вы  уменьшаетесь? И тогда тоже – ``удаление''.}
\people{Мы  уменьшаемся? }
\people{А может ли смениться расширение – сжатием?}
\soul{Да.}
\people{От чего это зависит?}
\soul{Всё – относительно. Относительно понятия вашего – `` сжатие''. Смените. ( понятия.  прим.)}
\people{В современном  представлении считается, что процесс заданный управляет гравитационным взаимодействием и определяться - расширение или сжатие - будет  независимо от того, какова общая масса вещества содержащегося во Вселенной. Так ли это?}
\soul{Давайте сделаем проще. От того, что будете расширяться или сжиматься, вы не погибнете. Не погибнет тот мир. Поймите. Вы говорите ``периодичность'' и далее. А мы же говорим, что нет расширения и сжатия. Мы говорим вам о наблюдателях, а вы ведь – наблюдатель. Поменяйте (точку отсчёта, положение ) и для вас – она будет сжиматься. Или расширяться. Относительно кого отвечать вам? Далее. Всмотритесь, подумайте, - Вселенная ваша – утончается! Подумайте. Возьмите камень, возьмите его плотность, и возьмите плотность воды и воздуха. Вот вам – расширение камня и расширение вашей Вселенной. Неужели вы думаете, что вы должны утончиться, уйти в более тонкие миры, а Вселенная – нет?}
\people{То есть, процессы похожи? Да?  Процесс, практически, от нас не зависит уже в коей-то мере? }
\people{Иначе, процесса расширения не было?}
\soul{Подумайте.}
\people{И катаклизмов космических тоже, практически,  не бывает?}
\soul{В вашем понятии, катаклизмы  были. Для других – нет. Что вы называете ``катаклизмами''? Когда вы испытываете ваше оружие, для вас это катаклизм? А для кого-то  – да. Относительно кого отвечать вам?}
\people{Значит, подход  к вопросу с точки зрения расширяющейся Вселенной  в принципе не верен?}
\soul{Вы подумайте…(срыв)}
\people{Мог бы я узнать о своих прошлых перевоплощениях?}
\soul{Нет. И вы будете постоянно знать. Поймите. Вы пришли – и ушли. Мы оставим вам след. Ну, зачем, если вы только пришли и уйдёте? Вы  проходите мимо. ( спрашивающий действительно вскорости исчез из поля нашей контактной ситуации)}
 
\people{Я могу ошибиться со своей прошлой жизнью.}
\soul{А если мы вам скажем? Будете жалеть? И что измениться? Что? Это будет одна из версий. Только и всего. И, подумайте, когда-то вы спрашивали о птичке.}
\people{А сталкивался ли я с близкими мне людьми, которые сейчас меня окружают?}
\soul{Да. Мы же говорили вам. И говорили в начале.}
\people{Я имею в виду, персонально себя, как конкретного индивидуума.}
\soul{Ну, подумайте, если  мы говорим о всех? Вы, что…?(срыв)}
 1-2-3-
\soul{Переводчик ваш  (неразборчиво)}
\people{А если мы спросим  по переводчику? Каковы его нынешние предназначения?}
\soul{И мы ответим вам? Стали бы Вы отвечать, если бы Вы знали?}
\people{Хорошо. Ну, вот он имеет изобретения, множество рационализаторских предложений, стоит ли ему в этом совершенствоваться и идти этим путём?}
\soul{Это мы должны спросить у него, каким путём пойдёт он? И мы должны спросить у Вас, каким путём пойдёте Вы? Не мы направляем вас. Мы, всего лишь, попутчики ваши. Куда пойдёте, туда пойдём и мы. К сожалению…(срыв)}
 1-2-
\people{Скажите, вы как-то сказали, что когда вы…}
\soul{Дайте счёт. Дайте счёт с 19-ти до 11-ти.}
 (счёт)
\people{Скажите. Вы как-то обмолвились, что когда - ``Вы будете нами…'' Мы с вами так и не встретимся, я так понял. Когда мы будем вами, вы будете уже кем-то другим?}
\soul{Вы не внимательны. Вы столь не внимательны, что можете фантазировать, - в вашем понятии. Возьмите, и услышьте снова. И подумайте. Далее… Если вы говорите, - ``подобны вам'', ``мы это вы, и вы – это мы'',  то это не значит, что мы имеем тела ваши. Это не значит, что идём дорогой с вами. Вас множество и вы подобны друг другу. Одна дорога ли у вас? Почему же мы должны идти вашими дорогами? Почему? Подумайте. Проще. Всё проще. Дороги наши пересекаются с вашими. Есть время, когда вы слышите ``шёпот''. Но не столь сильный, чтобы мог говорить переводчик. Есть время, когда переводчик, - и мы говорим не о нём,- о всех,  - становится нами. Ибо в вашем понятии – есть пересечение. Говорите ``логика''… Логика, - и тут же, нарушаете. Вы не хотите думать, хотите искать готовый ответ. Подумайте. Подумайте и всмотритесь, возьмите аналогию, возьмите во всём. В малом, есть проявление большого. И, тем более, большое проявляет на малое. Как живёте вы, так живёт и Вселенная. Как живёт Вселенная, так живёте и вы. Все процессы, что творятся в вас, – творятся и во Вселенной. И наоборот.  Если вы утончаетесь, - будет утончаться и материя. Будет утончаться, в вашем понятии, и космос. И если вы превращаетесь в камень, то и ваша Вселенная, что видите вы,  будет превращаться в камень. С какой точки зрения будете смотреть. Поймите. Вы говорите ``параллельные миры''. В вашем, в вашем мире, она, может быть, расширяется, - мир, находящийся здесь, в этой же Вселенной, но вы говорите ``параллель'', - идёт сжатие. И вы, - сегодня вы один, завтра вы другой. Вы – маятник. Маятник. В вашем понятии, от ``тонкого'' к ``грубому'' и наоборот. Ибо не можете удержаться на ``тонком''. И легче – уйти в ``грубое''. Но, приобретя ``тонкое'', если вы найдёте силы, вы можете уйти далее. Но, чаще, достигнув чего-то, достигнув какого-то уровня, вы бросаете, ибо говорите: ``Я пришёл. Вот он я, герой!'' И что? Вы  перестаёте. Перестаёте, потому и ``падаете''.}
\people{Мы не сильно мучаем переводчика?}
\soul{Спрашивайте.}
\people{Существует мнение, что каждую новую жизнь человек занимается приблизительно тем же, чем и предыдущую. Ну, конечно, с учётом новых исторических и социальных условий, и чтобы выйти на новый более высокий уровень, нужно услышать какой-то не объявленный нам посыл.  Может быть, граничащий даже с самопожертвованием. }
\soul{Да. Легче – продолжить то, что когда-то было. Первое.  Второе, - почему всё геройство должно быть обязательно самопожертвованием? Ну, почему вы так любите это? Почему вы так любите убивать себя? Спрашивайте.}
\people{Здесь же вопрос граничит о том, что есть подвиг? Вот, каков механизм его совершения? Зачастую, человек, который идёт на подвиг, не успевает принять обдуманное решение. Всё происходит спонтанно.}
\soul{Да. Чаще, вы делаете верно тогда, когда не думаете. Когда сознание ваше…}
\people{То есть, поэтому подвиг считается  высшим проявлением духовных сил человека, да?}
\soul{Да.}
\people{Он помогает перейти человеку на более высокий уровень в следующей жизни? }
\soul{Да. Вы это называете самопожертвованием.}
\people{Всегда помогает?}
\soul{Да. Вы это называете самопожертвованием. Вы только так понимаете самопожертвование. Вы только так понимаете. Неужели вы думаете, чтоб перейти на более высший уровень,  надо совершать подвиг? Мы когда-то говорили вам, что в вашем понятии всё искажено. Странник, ушедший от мира сего, - совершил подвиг. И он для вас святой. Человек, который мучается с вами – он не святой, ибо видите его каждый день, каждый час. Разве он может быть святым, если он живёт вместе с вами? Конечно, к сожалению, в вашем понятии, он не станет выше, ибо он не святой, ибо он не совершал подвиг.}
\people{Скажите, - человек бросился на амбразуру и погиб. Он поступил правильно, в общечеловеческом плане? Или можно было не делать жертву? Переждать. Другим способом.}
\soul{Ждите. Вы только и делаете, что ждёте. Вы ждёте, когда это сделает другой.}
\people{То есть, вы расцениваете такую смерть, такую гибель…}
\soul{Мы говорим вам и повторяем, - вы правдивы, к сожалению, всего лишь, два раза в жизни; в момент рождения и в момент смерти. Всё остальное – вы лжёте и носите маски.}
\people{Ну, вы заинтересованы в том, чтоб мы, всё-таки, изменились? *Совершенствовались. }
\people{Или вам вообще – без разницы?}
\soul{Тогда зачем приходим к вам?}
\people{Так вы, как наши братья, да? Пытаетесь, как детям малым, нам объяснять постоянно и отвечаете на одни и те же вопросы, которые, в принципе, вас должны утомлять. Да?}
\soul{Поймите, нельзя говорить о ``малых братьях''. Во многом, мы для вас - `` младшие''. Но в другом, - мы для вас ``старшие''. Всего лишь, - разные дороги. Разные дороги, но  ведущие к одной цели. Потому и взгляды иные. Потому и жизнь другая. Но, пересекаются пути наши, пересекаются. И тогда мы можем слышать. Слышать друг друга. Подумайте. Всмотритесь в историю вашу. Когда… всегда ли у вас были ``времена чудес''? Всегда ли к вам приходили святые? Вспомните.}
\people{Только тогда, когда трудно бывает.}
\soul{Вот и подумайте, кто же мы, иные. Поймите, для многих вы – дети, которых охраняют. Но не охраняют так, как это делаете  вы – ``не ходи туда'', ``туда'' и ``туда''. Ибо, настоящий родитель не даст упасть, но не будет предотвращать попытки упасть. Ибо, не познавши – не научитесь. Ибо, сидя на цепи – не узнаешь.}
\people{Неужели так весь мир устроен, как вы сейчас говорите?}
\soul{Нет. В вашем понятии, существует иерархия. Поймите, есть только рождённые, есть и старцы. В вашем понятии, да, мир один. ( срыв)}
 1-2….
\people{Скажите, вот такой вопрос. Вы сказали, что мы два раза в жизни правдивы. А является ли, допустим, первая  любовь третьим разом, когда мы правдивы? Или, и тут мы лжём?}
\soul{Это столь короткий миг, когда вы правдивы. А далее, вы надеваете маску.}
\people{Ну, смерть тоже  - не длинный путь. Момент смерти. }
\people{Мы сейчас касались того вопроса, что Земля – это живой организм. Скажите, Луна, тоже живой организм? Живая?}
\soul{А вы подумайте, в вашем понятии, - в понятии ученых,- даже там есть жизнь. Вы, отрицаете? Вы хотите сказать, что Земля мертва?}
 
\people{Нет. Мы приходим к выводу…}
\soul{Хорошо. Назовите мне, хотя б одного, кто мог сказать, что Земля мертва.}
\people{Ну, это по незнанию могут говорить, что мёртвая. На самом деле мы уже ощущаем.}
\soul{Ощущаем? Возьмите Землю, увидьте в ней, и подумайте, может ли мёртвое растить живое?}
\people{Но вот Луна? Она не случайно лишена атмосферы? Она выглядит, как тело мёртвое. Мы хотим узнать – это мёртвое?}
\soul{Тогда подумайте. Придёт рыба на сушу и скажет: Здесь не может быть жизни, ибо нет воды.}
\people{Ну, ладно. Иносказательно вы нам сказали, что Луна тоже живой организм. Хорошо, спасибо.}
\soul{Спрашивайте.}
\people{Скажите,  пожалуйста, я хотел знать своё жизненное предназначение. Почему я стал, кем стал. Например, учителем, а не тем, кем хотел. Это зависит от меня или меня направляли?}
\soul{А как вы думаете. Как Вы видите разницу между направлением и желанием?}
\people{Это когда мои собственные желания противоречат тем, что я получил в итоге.}
\soul{Нет, здесь нет противоречия. Противоречие только в том, что Вы не достигли, не достигли. Независимо, кем бы Вы были. Независима профессия Ваша. Не зависит. Вы можете быть, кем угодно. Или, в Вашем понятии, есть разделение? Вы видите…(срыв)  Вы вспомните, что заставило Вас? Вы, помните?}
\people{Задать такой вопрос?}
\soul{Мы спрашиваем. Мы спрашиваем, почему Вы стали учителем? Вы забыли?}
\people{Случайность.}
\soul{Случайность? Какую же дату Вы называете той случайностью?}
\people{А можно ли ещё предпринять усилия и переиначить судьбу? Это в силах спрашивающего?}
\soul{Переиначить? А может быть в судьбе уже где-то записано, что вы будете переиначивать? А может быть,  и  нет судьбы, а есть всего лишь жизнь?}
\people{Но обычно пользуются теорией, в которой люди сами формируют как бы свою судьбу. Нет ли здесь противоречия, в своем предназначении? И может быть, вообще никакого предназначения у человека  нет? }
\soul{Поймите. Вы можете идти, в вашем понятии, любой дорогой, которую вы называете ``профессией'', но выполнять своё предназначение. Почему Вы связываете именно с профессией? Почему? И вспомните дату. Дату, когда Вы, как говорите, изменили судьбу. Разве Вы не меняли? Вспомните. Или Вы всегда хотели стать? Вы же говорите о случайности. Спрашивайте.}
\people{В каждом человеке живёт ``второе я''. То, с кем он ведёт всегда какой-то внутренний диалог. Есть ли это `` второе я''  часть вас?}
\soul{Часть.}
\people{И поэтому, вы всё знаете о нас?}
\soul{Нет. Но мы вам говорили о памяти. Вспомните.}
\people{Что нам надо понять - вообще – память есть у всего.}
\soul{В том и ответ. Далее.  Поймите, знать всё – это трудно. Знать и видеть всё, - очень трудно. Представьте, если вы сейчас будете знать ваше будущее. Полностью. Полностью, каждый миг. Будет ли весело вам?}
\people{А правда ли, что далай-лама знает  даже дату своей смерти?}
\soul{Дату смерти? Да, можно узнать. Ибо, многие уже слышали эти даты, но забыли, или не верите. Потому и легче вам.}
\people{Правда ли, что ламы перерождаются в теле вновь? Что вновь становятся ламами? }
\people{Что это та же личность. }
\soul{Всё зависит от того, как вы прожили. Вы же спрашиваете… Рассудите логически, Вы говорили о подвиге. Помните? А теперь подумайте далее. Был ангел. Святой. Вы согласны? Обладал множеством, был при боге, в вашем понятии, и кто он сейчас? Падший ангел. Подумайте. Значит, всё-таки, можно уйти вниз, можно -  упасть?}
\people{В таком представлении перерождение далай-ламы – одна из основ буддизма. Значит, она не верна? }
\soul{Почему же? Почему? Тогда вспомните. Вспомните, много ли Вы говорили правды? Может быть, то, что не по нраву Вашему забываете и прячете? Ну, почему, вы приводите примеры только те, когда рождались ламой? А почему не вспомните о других? Вы не знаете о них? Ибо не говорят. В вашем понятии…(срыв)}
\people{Может переводчик несколько плохо себя чувствует? Мы правы? Нет?}
\soul{Подумайте. Вы считаете нас жестокими?  Вы считаете, что цель оправдывает средства?}
\people{Нет. }
\soul{Спрашивайте.}
\people{Если б я, не называя имени близкого мне человека, только думал бы об этом человеке, то вы могли бы знать, о ком я веду речь?}
\soul{Нет. }
\people{Только нужно сообщать словесно, что бы знать?}
\soul{Поймите. Мы будем отвечать вам только на то, что произнесено вами. Ибо вы – рабы. Рабы техники вашей. И потому, мы будем отвечать вам подобно. Мы играем в ваши игры, но не вы в наши. Если вы не можете обладать телепатией, значит, придя к вам, мы не будем делать того. Мы играем в ваши игры. Мы приходим, или же…( срыв)}
\soul{Мы не принесем вам доказательств никаких. Ибо вы должны найти сами. Сами. И вспомните, пришёл Спаситель. Творил ли чудеса для рекламы?}
\people{Однако, он по воде ходил…}
\soul{Реклама?}
\people{Не ради, может, рекламы.}
\soul{Далее. Вы, поняв всё же, кто мы, перейдёте на другой уровень. Потому и не будем говорить вам. Потому приходим и рождаем в вас сомнения. Потому мы играем в ваши игры. Поймите, если мы будем играть в другие, - поймёте ли вы нас? Будет ли вам интересно играть в те игры, что не знаете правил?}
\people{По аналогии. Если перевести, смотря на какую-то игру, можно догадаться и о правилах.}
\soul{Нет. Много ли вы догадываетесь, когда вам требуется ``конкретно''? Столь конкретно, что ``подобное'' – и вы уже не видите ответа.}
\people{Скажите, пожалуйста, а какое ваше мнение о спиритических сеансах? Спиритические сеансы, это сеансы с вами или при помощи вас?}
\soul{Отрицательно. Мы вам отвечали на подобное. И говорили вам, что тот мир, ближайший к вам. А значит, в вашем понятии, подобен вам. Не намного, но тот же. }
\people{Существует такая гипотеза, что мысль человека не то, благодаря чему человек становится умнее, накапливает знания, а наоборот, устройство, которое в той или иной степени, блокирует человека. Ну, если так можно высказаться, от какого-то ``высшего'' знания. }
\soul{Вы правы. }
\people{Недаром, гениальность граничит с сумашествием.}
\soul{Вы правы! Мозг ваш, всего лишь ключ, и не более.}
\people{Ну, а есть способы раскрепостить мозг? Обойти его запреты и физиологические особенности.}
\soul{В том и цель ваша, что бы вы смогли! Поймите, вы нас спрашиваете, о высшем и спрашиваете, ``как перейти'' - виноват мозг. Ищите! На то оно и высшее, чтоб учиться. Как вы думаете, сможете вы пройти ваш университет, не пройдя школу?}
\people{Так мозг это ``ключ'' или  ``замок?''}
\soul{Ключ.}
\people{К высшему.}
\people{А кто тогда сумашедшие? Каков механизм? Есть врождённые сумашедшие? Есть, которые стали в зрелом возрасте сумашедшими?}
\soul{Неумение управлять. Мы отвечали вам на подобное. Неумение управлять. Всмотритесь, что делает ваше сознание? В вашем понятии,  оно блокирует. Ваше слово. Блокирует чувства ваши. Блокирует эмоции ваши. Сознание создает маску, Вы согласны?}
\people{Да.}
\soul{В вашем понятии существует ``в волчьей стае по- волчьи выть''. Вы согласны, что это одна из масок? Ваше сознание – мешает вам. Ваше сознание влияет на мозг ваш. И теперь, представьте, - нет его, сознания вашего. Нет, что б могло остановить чувства, эмоции и далее. Нет сознания, что управляло бы энергиями. Кто они? В вашем понятии, сумашедшие. Спрашивайте.}
\people{Ну, если ребёнок от рождения лишён рассудка? Следствием чего это является? Просто дефектом биологическим? Или кармическим?}
\soul{Множество. Поймите. Поймите, что придя в этот мир, он обладает телом, живущим в этом мире. Душа выбирает тело. Да. И у неё свои законы выбора. Согласитесь, что если бы всё выбирало сознание, то не было бы среди вас нищих, калек и больных. Вы согласны? У души свои меры и потому она выбирает и потому знает, знает что будет. И, чаще, в наказание. Но есть такое понятие, как ``отдых''. Не мало…( срыв)  }
 Конец записи. 
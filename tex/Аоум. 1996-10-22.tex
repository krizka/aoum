Аоум. глава 22-10-1996 г
Георгий Губин
***
(запись без начала)

- … на табуретке, значит, стихи рассказывать.(…)из шоколада. А потом, мне сон снится, значит, что папа зарабатывает деньги - стоит на табуретке и начальнику рассказывает стихи, а тот ему деньги даёт.

*(Белимов) А почему папа… это…

- Ну, это самое… ну, как… Мне на день рождения - зарабатывать на подарки… Ну, а мне во сне, знать, снится, как папа-то  деньги зарабатывает? Наверно, так же - стихи рассказывает… А начальник ему деньги даёт. - Зарабатывает!  



*(Ольга) “На табуретке стоит”! Вот воображение! ( смех)
*(Белимов) Сегодня 22 октября, и сейчас примерно 17 часов вечера. Мы опять собрались, чтобы провести сеанс. 
Что вы сказали?

- У вас сегодня тихо.

*(Ольга) У нас сегодня тихо?
*(Белимов) Да, может быть. И мы продолжаем счёт. 11-12-13-14… (пауза)
*(Ольга) Дальше.
*(Белимов) 15-16-17-18-19. Мы бы хотели поговорить на этой стадии. Мы полагаем, что находимся на свободном сознании переводчика.

- Нет.

*(Белимов) Нет. А почему не вышли?

- Ошибка счёта.

*(Белимов) Ошибка счёта?

- Вы прервали.

*(Белимов) Понятно. Если мы, ну… вначале, тогда поговорим с вами, а потом, видимо, повторим, чтобы выйти на свободное сознание. Скажите, вы можете рассказать, почему переводчику было недомогание после прошлого сеанса? Что было неправильно сделано? В середине октября был сеанс…

- Практически всё вы сделали неправильно.

*(Белимов) Ну, мы же готовились. Мы делали памятки, которые…

- Вас не было.

*(Белимов) То есть, мы её не запомнили, да? Мы неправильно делали?

- Вас не было.

*(Белимов) А-а-а, меня не было. Но, а я…

- Почему говорите вы?

*(Белимов) Это что, повлияло сильно на сеанс что ли?

- Пусть каждый отвечает за себя. Почему говорите вы?

*(Ольга) Я же была.
*(Белимов) А-а-а.
*(Ольга) Ну, не знаю… ну, конечно  там кое-что напутали. Вы нам не объясните, может быть? Или мы сами должны догадаться? То есть, сами должны разбираться в этом?

- Пока вы догадаетесь…

*(Белимов) Да.

- И так, у вас было нарушение счёта – первое. Второе, - мы просили о спокойствии, а вы устраиваете междоусобицы. Вы уже говорили об этом.

*(Ольга) Ну, да.

- И вы сами поняли, и пришли к тому, но продолжаете. А мы же говорили вам и о чувствительности.

*(Белимов) Да. И дальше продолжаться в таком роде не может. То есть, мы должны, действительно, быть полностью, как один организм, полностью настроены только на сеанс. Так ведь?
*(Ольга) Ну, как? Уже сколько раз рассказывали.

- Нет, мы не говорили, чтобы вы были машинами, готовыми для сеанса, но будьте просто людьми и имейте уважение. Вы помните о вседозволенности?

*(Ольга) Да.
*(Белимов) Да, помним. К сожалению, это как-то не убирается из наших отношений.
*(Ольга) Вы не скажете, в общем-то, переводчик, он не очень хорошо себя чувствует, я не знаю, можно ли назвать, какой орган у него страдает?
*(Белимов) Что у него нарушилось? И чем можно помочь?
*(Ольга) Может быть…
*(Белимов) Или время?

- Вы дайте счёт от 31 до 35, от 42 до 45.

*(Белимов) И обратно?

- От 53 до 55, от 64 до 65, и наконец, 75. И потом, пожалуйста, не перепутайте, всё сделайте обратно.

*(Белимов) Всё сделать так же - обратно? Это можно делать прямо сейчас?

- Можете.

*(Белимов) 14-15-16…

- Уже слышу.

*(Белимов) …17-18-19. Мы не закончили эту серию.
*(Ольга) Ну, значит… Они сказали нам дать.(счёт. Прим.)
*(Белимов) Подожди. Ты нас слышишь?

- Слышу.

*(Ольга) Ты как себя чувствуешь?

- Как обычно.

*(Ольга) Нормально?

- Не знаю. Я… Мне сейчас тело не очень интересует.

*(Белимов) Скажи, ты сейчас, вот, в этом состоянии своём,- свободном сознании,- ты можешь предположить или знать твёрдо, что тебе помешало, что тебя вывело из равновесия? Что тебе сделало болевые ощущения или неприятность, дисгармонию в организме?  Не можешь сказать?

- Мне надо за что-то зацепиться.

*(Белимов) Что тебя расстраивало? Что чувствовалось? Почки плохо работали или что?

- Нет, понимаете, я могу вспомнить только в том случае, если я найду, вспомню какое-то одно мгновение, и тогда я смогу уже вспомнить остальное. Мне надо какая-то зацепка. Как… помните, в прошлый раз говорили о дороге?

*(Ольга) О дороге?
*(Белимов) Ну, а… мы это сейчас не вспомним, к сожалению, но в прошлый сеанс, что тебя вообще-то, по-человечески, так не понравилось? Что там было? Что настораживало или мешало тебе?

- Я не могу сказать, потому что я не нашёл…

*(Ольга) Тебе конкретно нужно…?

- …не нашёл приметы.

*(Ольга) Скажи, у тебя кошка лежит на тебе. Ты никак? Реакции никакой нет?

- Меня не интересует сейчас тело.

*(Белимов) Угу. Скажи, вот ты читал памятку нашу, с нашими знаками, о проведении сеансов…

- Я не читал. В этом сознании  - я не читал.

*(Ольга) А-а-а, в этом сознании ты не читал.
*(Белимов) А то я хотел спросить, будет ли дополнение, всё ли там верно? Значит, ты не читал. И сейчас не можешь никак прокомментировать.
*(Ольга) А ты сейчас…

- Понимаете, я же в том состоянии… В общем, в этом состоянии я не читал.

*(Ольга) Ясно.
*(Белимов) Понятно. Тогда вот что, - мы хотели бы на этом периоде счёта узнать, можешь ли ты вспомнить, что было с тобой до твоего рождения, до выхода на свет? И, например, можешь ли ты вспомнить, как происходило оплодотворение? Это была вспышка? Это был какой-то другой, волновой процесс? Как это происходило?

- Нет, это было падение.

*(Белимов) Откуда?
*(Ольга) Ощущение падения, да?
*(Белимов) Ощущение падения?

- Да. Это падение. Подобное ощущение – это, когда во снах падаешь в бездонный колодец. Вот это было падение - падением к свету. А падал я - от другого.

*(Белимов) От какого другого? Из другого мира?

- Нет, другой свет, живой.

*(Белимов) А-а-а, от другого света к другому. Ага.  А скажи, вот падение в детстве, которое мы переживаем – это что за процесс? Это связано с какой-то нашей прошлой жизнью? Почему все это переживают, но твёрдо мы не знаем причины? Говорят, что рост… (считается, что организм растет во время падения во сне. Прим.)

(часть контакта вырезана. Прим.)

- …организму отключиться от внешнего мира, чтобы решить свои проблемы, чисто физиологические.

*(Белимов) И они решаются в этом… в таком процессе? Решаются?

- Да, иначе это нельзя было бы повторять это множество раз.

*(Белимов) А как они.. А что разрешается? То есть, человек успокаивается…

- Давайте скажем так: когда вы спите, множество каналов информации перекрыты, но организму это мало, ибо вы можете всё-таки проснуться от громкого звука, или от толчка, поэтому, организм на какие-то мгновения,-  а это всего лишь мгновение,- отключаются полностью. Иначе - закрывается, и уже “проигрывает” себя в самом себе, не обращая внимания на наружность, ибо в эти мгновения нельзя вмешиваться.

*(Белимов) А почему вот я, например, падал… у меня какие-то этажи всё время были… падения через этажи, через населённые этажи. А другие падают - просто как в пропасть?

- Это мешает уже ваша логичность.

*(Белимов) Уже вмешивается логика, да?

- Да, ибо вы не можете вспомнить. А помните же, что человек не может фантазировать? Ему обязательно надо что-то дорисовывать, что-то, за что можно ухватиться. И проще, конечно, нарисовать этаж, потому что страх, страх вот этого падения, он живёт всегда. И поэтому, чтобы избавиться от него, мозг может нарисовать ему картинки.

*(Белимов) Угу. Ясно. А, вот, почему многих людей преследует во снах какое-то страшное чудовище, которое пытается раздавить,- человек выглядит маленьким, оно настигает его? У тебя это было во снах?

- Здесь множество причин. Есть внешние причины, есть и внутренние, вплоть до психологических травм. Обычно - это всегда бывает у детей, ибо ребёнок сам маленький, и все, кто выше его, уже больше его. И он знает, что не может ответить на обиду, создаёт себе…

*(Белимов) 11-12…-19

- Слышу.

*(Белимов) Он не может скрыть обиду и…? И поэтому, у него мерещится… что-то большое на него нападает?

- Просто один из способов самолечения - пережить заново, но в более усиленном варианте, как бы сравнить, что это всё-таки страшнее, чем было наяву. Вспомните, раньше, как вы говорили: «Хуже смерти не бывает» – и вам было легче пережить. Правильно же?

*(Ольга) Да.

- Тот же принцип.

*(Белимов) У тебя…

- Увеличить, увеличить обиду до такой степени, чтобы уже явная обида уже казалась более меньшей. Так, чаще, мы избавляемся от комплексов.

*(Ольга)Угу… Скажи, или скажите, как вот сейчас говорить, у меня такое ощущение, что ты не один. Я не права?

- Нет, права.

*(Ольга) Права? В общем, это новое, наверное, для нас? По-моему, мне так ощущается, то есть вот на этом счёте, если можно принимать во внимание счёт, то мы впервые… ты впервые, во всяком… в таком состоянии…

- Нет.

*(Ольга) Нет?

- Это впервые, всё-таки из-за того, что просто опять перепутали счета.

*(Ольга) А, вот так. Ну, в общем, мы путаемся, и из-за этого потом сами не знаем…
*(Белимов) Хорошо, я продолжу…

- Каждая цифра обозначает какую-то ступень, на которую я поднимаюсь или опускаюсь. Каждая цифра – это просто… нельзя сказать, что чистая нумерология, но она… каждая цифра – это как бы пароль в какую-то ячейку. Или, может быть, ячейка связана с мыслями или с какими-то действиями, или просто с датой.

*(Ольга) А скажите, можно сейчас исправить то, что мы вот нагородили?

- Уже нет.

*(Ольга) Уже нет, да?
*(Белимов) Ладно. Вот нам… нас вол… интересует вот в таком состоянии, свободном сознании, может ли ты, или другой человек, помнить то, что происходило в период зачатия, беременности и прочее… Тут у нас есть ряд вопросов.

- Но это помнит, вообще-то, каждый.

*(Белимов) Но мы не можем извлечь из памяти. Ну, вот, например, - значит, было падение,- оплодотворение воспринималось, как падение к какому-то другому свету. Так ты полагаешь?

- Не само оплодотворение, а только приход.

*(Ольга) Приход.
*(Белимов) Куда?
*(Ольга) Ну, там. А ещё оплодотворения не было - да ведь?

- Давайте будем говорить не чисто о физическом.

*(Ольга) Ну, да.
*(Белимов) Вначале - что? - Душой?
*(Ольга) Организм…

- Вначале ребёнок ищет своих родителей. И найдя их, он заставляет физический уровень подготовить себе “ложе”.

*(Белимов) То есть…
*(Ольга) Значит, это неправда, что есть, говорят, родителей… то есть, дети не выбирают своих родителей, это выходит, наоборот, да?

- Нет -  дети выбирают.

*(Ольга) Ну, вот видишь!
*(Белимов) Но, как же понять?  Что, тогда, может, детей “штамповаться” очень много.
*(Ольга) Почему  -  “штамповаться”?

- Ну, почему?

*(Белимов) И рождаться, рождаться с каждым половым актом…

- Ну, почему же? Они не рождаются, во-первых, “с каждым”. Во-вторых - тогда вспомните, у кришнаитов вообще запрещено.( п/акт без цели зачать ребенка прим.) Почему? Чтобы не создавать той среды - не “созревать” попусту.

*(Ольга) А это как?

- Ибо при воссоединении происходит выброс энергии, который привлекает не рождённых душ, а придя, они видят, что они здесь не нужны. И каково разочарование? И вот тогда вам может быть отомщено в любом варианте.

*(Белимов) Как?
*(Ольга) То есть, я в общем-то об этом знала, но как-то не была убеждена полностью, что действительно, просто половой акт ради удовольствия – это, действительно, приносит страдания многим.

- Но вы вспомните.

*(Белимов) В других мирах приносит страдания?

- Вам же было сказано, что нельзя… нельзя…
(сбой контакта.)

*(Белимов) 11-12-…-19
*(Ольга) Слышишь?

- Да.

*(Белимов) Что нельзя? Продолжи мысль. Что, нельзя часто…
*(Ольга) Не часто, а вообще, устраивать пустое.
*(Белимов) Пустое… половые акты?

- Не прелюбодействуйте.

*(Белимов) То есть, вот это напрасно, да?
*(Ольга) Скажи, а вот, ну… будем говорить так, вот мы сейчас допустим, ну, практически, все мы люди страдаем от сексуальной неудовлетворённости. То есть, много очень людей. И мы ищем себе там партнёра, муж с женой – у них свои, так сказать, проблемы в этом плане. Ну, то есть, не проблемы, вернее, ну, вот жизнь, будем говорить, да? А ведь, наверное, тогда нужно эту энергию куда-то в другое что-то, ну, в творчество или ещё во что-то перенаправлять, чтобы не было вот этого, постоянной мысли: “вот, мне нужна женщина”, или “вот мне нужен мужчина”?
*(Белимов) Как правильно поступать людям?

- Ну, вам не говорят, чтобы вы воздерживались постоянно. Пожалуйста! Но только делайте это любя.

*(Ольга) А-а! Тут, то есть, должна присутствовать действительно любовь, а не просто…

- Где у вас часы?

*(Белимов) Нет. Ты их чувствуешь?

- Они громко тикают.

*(Белимов) Но мы спрятали, в этой квартире нет. В другой комнате, значит, слышимость высокая. Сбивает. Мы убрали часы,  которые были в электронной… в записной книжке.
*(Ольга) Вот. Ты слышишь, да?

- Да.

*(Ольга) Скажи, вот, не знаю, ты или, может быть, вы ответите… Вот, действительно, а сейчас даже распространено такое мнение среди медиков, что онанизм полезен, то есть вот… то есть это, происходит выброс какой-то энергии лишней, и чтоб человек не зацикливался и так далее…

- Он полезен только в том случае… Куча вариантов. Если вы не можете сбросить энергию в куда более нужные дела, то лучше, конечно, онанизм, чем изнасилование.

*(Белимов) Ясно.

- Но это не лучший вариант.

*(Ольга) Ну, это ясно, естественно. Вот скажите, ещё такой уж, если на эту тему уж зашёл разговор, о монахах. Вот скажите, испокон веков всегда были монастыри женские и мужские отдельно, и это нормально? Или были, может, когда-то смешанные, действительно, когда люди… то есть, или не люди, не знаю… вот, монахи, пусть они будут обоих полов, и их не привлекало именно половое какое-то чувство, а просто действительно, ну, братство, что ли?

- Если б не привлекало, не было б смысла разъединять.

*(Ольга) Ну, неужели вот…

- У вас, всё-таки, есть животное.(чувство.прим.)

*(Ольга) Ну, да.

- И оно, всё-таки, требует своё, кем бы вы не были. И если бы вы могли не соблазняться, не нужно было бы деление.

*(Ольга) Ну, да-да. А вообще-то, это вот, даже школы вот у нас, ну, будем говорить, до некоторого времени в России школы были тоже разъединены.

- Это только привлекает внимание. И уже действует противоположно: чем больше запретов, тем слаще плод.

*(Ольга) Ну, да.
*(Белимов) Хорошо, я хочу… поговорить со свободным сознанием - что ты помнишь определить. Как ты полагаешь, в этом состоянии твоя душа сразу была введена в яйцеклетку или позже? Тогда на каком периоде позже?

- Только что была речь о ребёнке выбирающем.

*(Белимов) Который выбирает. То есть, сразу душа уже присутствует, так сказать.

- Душа, прежде, придёт.

*(Ольга) Вот, подожди. Можно…

- И потом, с помощью родителей, подготавливает, совместно…

*(Ольга) А вот, скажите, вот… ну, ребёнок выбирает родителей, а допустим, они уже женаты или они могут не знать даже друг друга?
*(Белимов) Сколько задолго выбирается семья…

- Даже в актах изнасилования.

*(Белимов) До чего?

- В актах изнасилования…

*(Ольга) Тоже есть.

- …женщина может иметь ребёнка.

*(Ольга) Да.

- …ибо ребёнок выбрал.

*(Ольга) То есть, ребёнок выбрал мать, да?

- Поймите, это вы сейчас, имея право выбора, будете выбирать что получше, а не то, что вам нужно. Там же, будучи свободны от тел и проблем этого мира, выбирают то, что нужно, что необходимо, а не - хороша ли семья или плоха, или - с честью или без чести был рождён.

*(Белимов) Так получается, и изнасилование – вовсе не случайный акт, а спровоцированный буквально…

- Нет, это вам никто не говорил. Просто - он пользуется.

*(Белимов) Пользуется.

- Поймите, при воссоединении, каким бы оно ни было, создаётся энергия, привлекающая. Да, конечно, есть множество спровоцированных, как вы говорите. Есть множество. И можно найти виновного и только в ваших, с вашего мира, не говорите о потусторонних. Если вы захотели, вы будете изнасилованы. А если вы создаёте рекламы купить эти духи, то вы будете изнасилованы на Авеню 5.  Но, простите за выражение, о чём идёт речь?

*(Ольга) Ну, ясно. Вот, скажи, тогда вот по этому поводу, тогда можно привлечь несколько детей. Могут несколько детей придти к одному, к одной матери? Ну, так… к родителям.

- Могут, но они не будут ссориться.

*(Ольга) Ну, да, а кто получает право тогда, ну, так если сказать? Не право, а, вернее, кто…

- Это уже решают они.

*(Ольга) Они между собой, да?

- А потом, к сожалению, будете решать уже вы.

*(Ольга) А вот, ну, когда двойнята, тройнята? Вот это…
*(Белимов) Близнецы - как они?

- Ну, что ж, по обоюдному согласию.

*(Белимов) Договорённость, что мы родимся у такой-то мамы, да? Так что ли?
*(Ольга) Нет, ну, есть возможность, поэтому, так получается. В общем-то, да, - получается?

- Давайте скажем так: ребёнок проходит “школу”, и родители тоже.

*(Ольга) Да. Скажите…

- …И если родители должны пройти школу воспитания детей, и ребёнку нужно пройти то же испытание, но только по отношению к родителям, -  ну, что ж, будет ребёнок. Если же вы в прошлой жизни,- давайте не будем даже говорить о прошлой жизни, давайте будем тогда говорить об этой, - если в этой жизни вы не заслужили ребёнка, и что бы вы ни делали, его не будет. А мы когда-то уже рассказывали.

*(Ольга) Да. Мы помним. А вот…м-мм Ну, говорите.(Белимову. Прим.)
*(Белимов) Скажи, вот ты вспоминал, что ты был мальчиком с голубым камнем, когда ты… когда вошёлм  этим мальчиком в эмбрион? Тоже - с самого начала, держа этот камень? Почему не девочкой? Можно что-то разъяснить?

- Понимаете, дело в том, что здесь нельзя сказать об эмбрионе. Я не входил в него. Я просто пришёл в дом.

*(Ольга) И ждал
*(Белимов) И ждал?

- Да.

*(Белимов) Момента зачатия?

- Я подготавливал вместе с родителями этот момент.

*(Ольга) То есть, всё делается вместе.

- Всё вместе, ничто отдельно. Отсюда и сомнения родителей “быть или не быть?” А что перевешивает…

*(Ольга) Ну, кто, наверное, кого лучше уговорит или как?

- Вы не на базаре.

*(Белимов) А что перевешивает, скажи?

- Ну, вот по ситуации - мать хочет и боится. Она хочет ребёнка и, в то же время, боится. Что всё-таки заставляет её отказаться, или стать матерью? С кем разговаривает она? - С будущим ребёнком. Это вам скажет любая мать, потому что она всё-таки советовалась с ребёнком, которого ещё нет. Она мечтала о нём, она представляла его, разговаривала с ним.

*(Ольга) Скажите, а вот, как тогда происходит… ну, ладно…
*(Белимов) Тогда так: а был ли ты одинок в утробе и кто с тобой общался? И общался ли вообще кто-либо? Не помнишь этого?

- Конечно, - мать, родители, окружающие. Всё, что видела мать, видел и я. Всё, что переживала мать, переживал и я. Единый организм.

*(Белимов) Ты слышал голоса, музыку, смех, гнев?

- Позже - да. В начале, это было - эмоции, эмоции.

*(Ольга) Вот, можно вам вопрос задать про… личный, по этому поводу?

- Пожалуйста.

*(Ольга) Вот я насколько знаю, но меня мать не хотела рожать, то есть… а в те времена были аборты запрещены, когда я, так сказать, когда меня зачали, и очень долго… я родилась, в принципе, ну, нежеланным ребёнком, - отец меня отстоял. Отец настоял на том, чтобы мать родила меня, - не мать… И вот, как насколько я себя помню в жизни, то есть, вот в детстве, потом мне всегда был ближе отец, и хотя вот его нет, он умер, мне всегда… он для меня есть, то есть - я его люблю так же, как и если бы он… Вот, может быть, это сыграло роль? Вот… отец…

- Безусловно.

*(Белимов) То есть, она запомнила с прошлой жизни, или с того зачатия… отец был положителен, да. То есть, отец был положительно настроен, а мать,- хотя мать тебя дальше “те-тешкала”, кормила с груди…
*(Ольга) Нет, это не было в прошлой жизни, просто отец, для меня, был всегда как-то ближе, я не знаю даже, может, я всегда думала, что… может, потому что, как-то и… как-то мужчины, что ли мне ближе всегда бывали, потому, что и отец как-то… C отцом - больше…
*(Белимов Ольге) Ты задавай вопросы…

- Давайте скажем так. Ребёнок не умеет разговаривать и не понимает слов. 

*(Ольга) Ну, да.

- Был бы он ребёнком уже родившимся или ещё нет, но он не знает ваших слов…

*(Ольга) Да.

-… но он чувствует вас, и ему не нужны слова, он чувствует вас. Вы говорите ” слышит”,- но у него еще не развито, не зрение, у него не развит этот органы, но - единый организм матери, если хотите, на химическом уровне – пожалуйста. Ибо когда мать пугается – повышается андреналин, когда хочет мать спать,- ребёнок тоже будет спать. И всё - на уровне чувств.

*(Белимов) А вот в период беременности ты боялся чего-то? Были у тебя опасения какие-то, страхи, на уровне хотя бы эмоций?

- Всего, чего боялась мать,- а мать всегда боится, неся ребёнка,- будучи в автобусе, на улице или где-то – она всегда боится за него, за его здоровье, и эти страхи передаются. И, заметьте, дети развитых стран - гораздо слабее.

*(Ольга) Да.

- Те же, кто не боится и выполняют, продолжают работать…

*(Белимов) То есть, ближе к природе которые, да? У тех?

- те дети здоровее. Ибо они, ещё не рождённые, ещё не знают страх. Вы же - ещё не родились, а уже испуганы.

*(Белимов) Весь испуган, весь перенервничал…
*(Ольга) Скажите,- кочевой народ, да? Я не знаю, можно называть национальность?
*(Белимов) Цыгане,-  можно же…
*(Ольга) Вот я б хотела давно уже спросить, про цыган тоже. Вот они, в принципе, как-то живут… Ну, то есть - все как-то осёдло, в принципе, они вот - кочевой,- во-первых. Во-вторых,- они, как бы паразитируют. Я не знаю, как раньше они жили, а сейчас, в принципе, они вот воруют, выпрашивают, просят, гадают там, и в общем-то ещё и угрожают… Вот это… ну, это нормально вот? Как они такой образ жизни ведут, нормально или тоже вот, как и мы, допустим, цивилизованные, которые тоже какие-то свои недостатки имеют?

- Всё, что на физическом уровне - придумали вы. Вы придумываете цыган. И убери их – и вы по ним начнёте скучать. Вы придумали войны, а также перемирия, для того, чтоб собраться к новым войнам . Всё придумано вами, вами…
(сбой контакта. Прим.)

*(Белимов) 11-12-…-19

- Сейчас я  был где-то между “семью” и “двенадцатью”.

*(Белимов) А мы считали…
*(Ольга) Примерно в этом интервале, да?

- Да.

*(Ольга) А сейчас?

- 16-17, так, по-моему, были последние ваши цифры?

*(Белимов) Но я до конца досчитал, до 19. Может, скороговоркой, может, так чтобы вы…
*(Ольга) Мы опять можем, и с тобой, и… с ними, да?

- Трудно найти границы.

*(Белимов) Но мы продолжим тогда. Вот, по теории одного американца следует, что в перетональный период беременности закладывается очень многие страхи, комплексы, фобии. Это так? Ну, я только что…

- Ребёнок рождается уже с характером, вы могли бы это заметить, и лишь только потом он “куётся” жизнью. Но, как правило, очень многие черты характера, начального характера, остаются.

*(Белимов) И то, что неправильно выносили плод, это может очень сказываться на характере, так?

- Не “может”, а обязательно скажется!

*(Белимов) Скажется, да. То есть, мы должны, в принципе, пропагандировать то, чтобы матери всегда были спокойны, чтобы жизнь была нормальная, чтобы страхи её не преследовали. Так?

- Не лучше было бы - создать эти условия, а не пропагандировать? 

*(Белимов) Ну, и создавать, конечно.

- «Ну, и конечно» – вот вы всё время так и делаете: «Ну, и конечно».

*(Белимов) А на твой взгляд, что было в тебя заложено тогда, в перетональном  периоде: смелость, трусость, осторожность, независимость, терпимость, упрямство – какие черты, ты полагаешь, были заложены основные?

- Терпение.

*(Белимов) Терпение было заложено? А твоя скромность? Почему вот такой ты скромный? Ты не выпячиваешься – это терпение, да? В общем, терпение – самая главная черта, хорошо. А вспомни свой первый страх. Можно вспомнить страх? 

- Да.

*(Белимов) А в какой период? Я говорю про беременность.

- Это было 18-е февраля 61-го года. Когда речь шла обо мне, быть мне или не быть. Вот тогда я почувствовал первый страх. И вот тогда я уже понял, что у меня будет смена отца.

*(Белимов) В каком смысле «смена»?
*(Ольга) Ну, у него же…
*(Белимов) Другой отец был?
*(Ольга) Да. Он родной… Он же жил…

- Ребёнок, он очень чувствительный, и он может предсказать будущее. Вероятностное будущее, и довольно там мало ошибок, к сожалению.

*(Ольга) Тебя это не пугало?

- Это?

*(Ольга) Да.

- Это должно пугать.

*(Белимов) То есть, тебе бы хотелось иметь родных родителей, не отчима-отца, а родного отца.

- В то время?

*(Белимов) Да.

- Это было, как потеря сейчас какого либо ближнего человека. Но ты уже вырос… Да, конечно, у меня уже отчим, и я уже как-то спокойно отношусь у этому, безразлично. Но тогда - нет.

*(Ольга) Скажи, а вот в детдоме дети есть брошенные, есть - у которых родители умерли, но это… но они вот тоже знали, когда… что у них вот так - родителей не будет? Всё это…

- Не всегда. Бывает столь неожиданно. Но, чаще, чаще родители уже склонны отдать. Они уже готовы, ибо, не будучи готовым, никогда не совершишь такой жестокий поступок, поступок, изменяющий всю жизнь, как и для родителей, так и для ребёнка. Значит, уже были изначально готовы. И, конечно, ребёнок это чувствовал, но он всё-таки ещё надеялся…

*(Ольга) Надежда…
*(Белимов) Я хочу…

- А вы заметьте, спросите у тех детей, ведь они были более ласковы у других, ибо они чувствовали, они хотели задобрить, или хотя бы, чтобы полюбили их больше.

*(Ольга) Я это заметила, вот детдомовские дети и… в общем-то и дети, и взрослые, которые вырастают, уже взрослыми становятся, они более человечны что ли… И дети тоже, действительно, они были ласковы, более открыты, и эгоизма у них меньше. Это действительно так.

- Не совсем.

*(Ольга) Ну, я, во всяком случае, встречала.

- Понимаете, они помнят отношение в детдоме к себе, а значит, будут примерно так же относиться и к своим детям, ибо они считают это нормой. Ваша же поговорка: «меня лупили – ничего, вырос умный – буду лупить и я».

*(Ольга) А, между прочим, насчёт «лупить», да? Вот, вообще, можно ли детей…?

- Это только ваше бессилие.

*(Ольга) Это от бессилия, да?

- Если вы не можете действовать головой, то, конечно же, руками легче.

*(Ольга) Но это… во все, не знаю, вот сейчас… да и я однажды такой поступок совершила, в принципе. Никак… детям очень с… Вот, ну, детям говорят, могут, ну, очень долго говорить, говорить, говорить – они никогда, в общем, не реагируют. И как только за ремень берёшься, сразу выполняют то, что им говорили.

- Но разве это хорошо?

*(Ольга) Это плохо.

- Это всего лишь ваша беспомощность, это значит, вы…

*(Ольга) …не сумели правильно, да, вот, с ребёнком?

- Где ваш авторитет? Только не насилие.

*(Белимов) Я хотел уточнить, когда вот первый страх,- это уже шла речь об аборте, ты уже был зачат или только ещё планировался, не хотели тебя делать?

- И да, и нет.

(Выход на Мабу. Прим.)

- Я вынес его на солнышко. Не помогает.

*(Ольга) Кого? Сына?

- Да.

*(Ольга) Не помогает?

- Нет.

*(Ольга) Но, понимаешь, нельзя же сразу один-два раза. Надо очень долго, чтобы он почувствовал, что ему лучше.

- Ему будет лучше.

*(Белимов) В пещере-то, наверное, темнее, холоднее, хуже…

- Нет, там холоднее. А я уже знаю, что такое «холодно».

*(Ольга) Знаешь?

- Да. Меня в пещеру не пустили и…

*(Ольга) Кто не пустил?

- …было холодно.

*(Ольга) Жёны не пустили тебя?

- Почему жёны? Попробовали бы они меня не пустить!

*(Ольга) А кто тебя не пустил?

- Ну, опоздал я.

*(Ольга) Кто? Опоздал? Куда?

- Вы же сами сказали: “Иди, посмотри, что там за речкой?”

*(Ольга) Ну…

- Я пошёл. Но опоздал.

*(Ольга) А потом?

- А потом, я обиделся на вас.

*(Ольга) Да? Но, а потом, ты пришёл к себе в пещеру, и тебя не пустили?

- Нет, дверь закрылась.

*(Ольга) Пещера закрылась?

- Да.

*(Ольга) А она закрывается?

- Конечно.

*(Белимов) Ты никогда об этом не рассказывал, как она закрывается.
*(Ольга) Да. А как она закрывается?

- Вы чё, не знали?

*(Белимов) Нет. Ты же никогда не говорил. Как она закрывается, расскажи?

- Как? -Камнем!

*(Ольга) Камнем? И кто её закрывает?

- Мы.

*(Ольга) А кто твою пещеру закрыл?

- Ну, кто в пещере живёт? – жёны живут.

*(Ольга) Жёны? Ну вот, я же тебе говорю, что жёны.
*(Белимов) От кого вы закрываетесь? От зверей, или от кого?

- Не знаю.

*(Ольга) Чтобы теплее было?

- Нам сказали, что мы должны закрываться, чтобы мы не видели их богов.

*(Ольга) А-а-а, богов.

- А я не видел их богов.

*(Ольга) Так и не видел, да?

- Нет.

*(Белимов) Ты в эту ночь промёрз сильно, да, когда был вне пещеры? Замёрз?

- Вчера я понял, что такое холодно.

*(Ольга) Понял, да?

- Потом, я полночи вас ругал.

*А-а-а…

- А потом, искал богов.

*(Ольга) Ну, и нашёл?

- Не нашёл.

*(Ольга) Вот видишь…
*(Белимов) Зато ты кое-что познал полезное.
*(Ольга) Они, наверное, в эту ночь не прилетели.

(Выход на сознание. Прим.)

*(Белимов) Ну, я хочу продолжить…
*(Ольга) Сейчас ты помнишь… помнишь сейчас, что было?

- Нет. Но счёт… Слышал счёт - и всё.

*(Белимов)Но ты переключился…
*(Ольга) А вы можете сказать?

- Кто “вы”?

*(Ольга) А-а-а! Ты… Всё ясно.
*(Белимов) Скажи, пугался ли ты в период беременности, когда родители ссорились? Ты это ощущал, что идёт зло друг на друга? Или они вообще не ссорились у тебя?

- Да ссорились они, наверно. Ссорились. Всегда ссорились.

*(Белимов) Да? Но ты…

- В период беременности, нет, не помню.

*(Белимов) Пугания нет, не помнил, да?

- Не помню.

*(Белимов) Скажи, ощущал ли ты вот половой акт во время беременности матери? Пугало ли это тебя?

- О! Я вообще эти времена не помню.

*(Белимов) Не помнишь, да? То есть, период беременности ты в утробе не очень-то помнишь, только страхи какие-то помнишь?

- Почему я страхи какие-то помню? Я что, когда-нибудь говорил, что я помню страхи?

*(Ольга) Нет.
*(Белимов) Только что говорил, что…
*(Ольга) Нет.
*(Белимов) Значит, на другой вышли… Скажи, объясни, мы сейчас на каком уровне разговариваем? Дело в том, что только что ты подробно рассказывал момент зачатия, как ты, как создавал это, и вдруг ты сейчас ничего не помнишь. Какой-то перескок произошёл. Мы должны понять, с кем мы сейчас тогда разговариваем, какая память у тебя сейчас…
*(Ольга) С 11-ти до 19-ти..

- Нет.

*(Ольга) Нет?

- Нет.

*(Ольга) А какую?

- Где-то на 11, а может быть, чуть побольше, но даже не до 12-ти.

*(Ольга) Ясно.

(Сбой.прим.)

*(Ольга) Слышишь? Объясни нам, пожалуйста, я вот так, может быть, правильно, может, неправильно поняла. Вот мы даём счёт, да? И, допустим, на какой-то цифре ты уже начинаешь разговаривать, а дальше уже, то есть, значит, ты… ну, то есть вот этот счёт – тоже какой-то уровень? Ну…

- Ну, в какой-то мере, да. Но, понимаете, вы счёт даёте неравномерно. Бывает, вы называете цифру просто как пустой звук. Естественно, я не могу выйти на этот уровень. И поэтому, вы говорите «17», да? Вы говорите до конца, но вот всё остальное было, как пустой звук, произнесено лишь бы только произнести, понимаете? Поэтому, я уже не воспринимаю этот счёт.

*(Ольга) В общем, я так поняла, что вот, если на какой-то цифре ты уже говоришь, лучше эту цифру запомнить, и потом можно считать опять до этой же цифры, к примеру, если там сбой был.

- Да-да.

*(Ольга) Вот так? И не надо считать до конца, так ведь?

- Да, но… есть, конечно, риск, что вы не произнесёте,- опять будет пустой. Вы можете уйти на более низкий. Меньше шансов просто. Потому, что вы можете произнести, допустим, до тех же 17, пусто, а на 19 произнесёте с чувством, и тогда эти 19 я выйду. Какой-то есть шанс. А когда вы будете только до 17-ти, они все могут оказаться пустыми, и я буду опять “висеть”.

*(Ольга) Ну, в общем… даже считать нужно…
*(Белимов) Боже, какие нюансы…

- Нет, всё очень просто. Просто надо, как бы вжиться, - не просто вот монотонно повторять эти цифры. А вот объяснить смысл этих цифр нельзя. Это не значит, что, допустим, на уровне 12 будет только этот и этот человек, или тот же уровень, нет, это не обязательно - всё зависит от эмоционального настроя вас, переводчика, собеседников. Понимаете? Та же самая цифра 12, допустим, она может иметь множество вариантов. Множество людей может на неё выйти. Вспомните, совсем недавно сколько людей прошло?

*(Ольга) Да. А вот, скажи, сейчас Мабу вышел опять… вот, опять для нас неожиданность
*(Белимов) Как он прорывается? Мы… мы не на том уровне были?

- Нет, дело в том, что он, в принципе, ребёнок.

*(Ольга) Ну, да.

- Он ещё более открытый. И, конечно, своей напористостью, своей энергией…

*(Ольга) Он хочет с нами… То есть, мы… он хочет… Ну, да… ну, скажем, так – разговаривать с нами…

- Он готов залезть в любую “щелку”…

*(Ольга) Ну, да. Ну, как все дети, да. А, может быть… я просто знаю, что, если так честно сказать, я так скучаю по Мабу. Мне так хочется иногда с ним поговорить. Может быть, моё желание с его… тоже перекликается с его желанием.

- Ну, видите ли, сперва, конечно, он выходил к вам не просто из-за желания, а вот чтобы “заработать”.

*(Ольга) Ну, да.

- Это уж потом он привык к вам, он уже просто хотел с вами поболтать. А вначале - он просто “зарабатывал”.

*(Ольга) Очки, да?
*(Белимов) Ну, ясно. Я продолжу по перетональному периоду. Тот период, когда ты был в утробе, тебе какая музыка нравилась? Нравилась ли? Ощущал ли её?

- Осталось повторить, что саму музыку, конечно же, я не слышу. Я слышу вибрации. Вот я уже определяю, нравится мне эта вибрация или нет. И что интересно, это может даже не совпадать с мнением самой родительницы. Если родительнице будет нравиться одна музыка, я могу отвергать её, - мне будет нравиться другая. Вот почему бывает несоответствие вкусов у матери, у детей.

*(Ольга) А-а, ну, да.

- Музыка? Что такое музыка? Это же просто вибрации, определённые…

*(Белимов) Есть приятные, есть неприятные, да?

- Да.

*(Белимов) …для ребёнка. Классическая более приятная или всё-таки рок-н-ролл?

- О, нет, конечно же, нет. Классика более ближе. Понимаете, есть ещё смысл ритмики. Старые племена в основном действуют на ритмике. Бой барабана. Те же самые часы, которые сбивают – это тот же самый принцип того же барабана, бубна шамана. Тот же самый принцип. И поэтому , надо уйти от этого, чтобы не был создан ритм извне. Дальше,- музыка больше вам будет нравиться та, которая соответствует ближе, подходит к вашим вибрациям. И если вы слишком вертляв, то, конечно же, вам будет рок-н-ролл больше нравиться или тяжёлый рок, что будет говорить о том, что вы вообще-то человек не очень-то и хороший, более жестокий. То есть - вибрации, близкие к вашим вибрациям.

*(Ольга) А это…

- И по этому можно определить характер человека.

*(Ольга) Ну, это да.

- А вы помните, как Бог относится к вам? - По звону колоколов! Помните? Как-то была статья, что если вы очень плохо относитесь к колоколам, значит, Бог не воспринимает вас, и вы очень далеки от Бога,-  и наоборот. Ну, конечно, это всё “натянуто”, но зерно истины здесь всё-таки  какое-то есть.

*(Ольга) А, ну, значит, в общем-то, и в цвете… то есть, если человек любит более тёмные тона или… ну…

- Ну, скажем давайте так: красный цвет – более импульсивный цвет, согласны? И спокойный человек его не будет любить.

*(Ольга) Ну, да.

- Энергичный,- не будем показывать, в какую сторону эта энергетика, в отрицательную или положительную, но более энергичную,- значит, более яркие цвета. Но и бывает и так, что человек всю жизнь любил спокойные цвета,- допустим, зелёный,- и вдруг, на какие-то мгновения ему требуется красный.

*(Ольга) Ну, да. Это нормально, наверное.

- Нормально.

*(Ольга) Слушай, вот…

- Ибо мы чувствуем себя хорошо только в том случае, если внешние вибрации совпадают с внутренними.

*(Ольга) Не знаю, вот… можно цвета называть, так… вообще, без “всяких” там, да? То есть, в цветах нет там каких-то ограничений? Любой цвет можно называть?

- Ну, вы можете сейчас назвать и имя.

*(Белимов) И имя сейчас можно?
*(Ольга) Да. Ну, вот я, например… мне всегда нравился белый цвет, и нравится, как говорится. Я его предпочитаю, почему-то. Ну, мне нравятся и другие, конечно, цвета, я ничего не хочу сказать… Ну, вот я предпочитаю белый цвет. А на востоке белый цвет – это цвет… ну, считается - саван.

- Цвет смерти.

*(Ольга) Да.

- Да, но им считается и красный цвет смерти, и чёрный. Это можно где-нибудь найти так, что и зелёный цвет - будет тоже цвет смерти. Всё зависит от окружающей обстановки. Представьте: непроходимые леса. Смерть будет ассоциироваться, естественно, с зелёным цветом.

*(Ольга) Ну, да, звери там всякие напасти…

- Почему чёрный? Потому, что ночь всегда мы боялись. А почему белый? Что - боялись дня?

(начало второй кассеты. )

-…Голубой – вода. А это – “ничего”.

*(Ольга)Вот вообще-то, наверно, вот когда люди предпочитают… Вот, когда человек предпочитает какой-то цвет,- в одежде, допустим,- и в роли…ну, обстановка, комната там… ещё где-то, какой-то определённый цвет превуалирует, вот это…
*(Белимов) О чём это говорит?
*(Ольга) Ну, это, кажется, о том же говорит, что… вот, недавно мы беседовали, да? Что, в общем-то, какой он сам. Об его характере.

- Да, но только не забывайте, что в последнее время, вы стараетесь обмануть – носите не соответствующие цвета. И предпочитаете, как бы обмануть других другим цветом. Вы не любите зелёный цвет, но вы будете усиленно носить его по нескольким причинам. Одна из причин – это обмануть вас, другая причина – быть независимым от цветов (от цвета. Прим.), то есть – почувствовать себя сильным, - как бы “уйти” от привычек. 

*(Ольга А…(перепад записи)

- Мода? 

*(Ольга) Да.

- Мода, это всего лишь “бег со всеми”. И не обязательно”в нужную сторону”.

*(Ольга) Ну, да.

- Это “зов толпы”.

*(Ольга) А вот, действительно, когда… ну, если вот, опять …это чисто на “внешность” наверно распространяется. Вот, человеку “идёт ” такой-то цвет – это вот о чём-то говорит, или действительно, просто чисто на внешность распространяется?

- О, нет.

*(Ольга) Нет? Тоже – имеет и более… 

- Всё взаимосвязано. Всё. Поймите, вся физика всего лишь только для того, чтобы…

*(Ольга) Отражать, да?

- Отражать – это “раз”. И чтобы научиться управлять этой физикой. – Предмет изучения. (о физическом мире. Прим.) Предмет обладания.

*(Ольга) Это очень интересно, кстати. Сейчас… Я раньше об этом особо не задумывалась как-то… Да, это интересная тема.

- Если вам… Если вы зависите сильно от моды… разве это говорит о вашей силе?

*(Ольга) Нет, конечно.

- Ну, конечно, вы говорите, что “надо быть красивым, не буду же я одеваться, как чучело?”

*(Ольга) Ну, да.

- И, в то же время, вы отрицаете - “встречают по одёжке…” – и тут же встречаете по той же одёжке.

*(Ольга)Ну, да. И это есть.

- Вот – “Двуликий Янус” (сравнение наших слов с их реальным выражением нами в жизни. Прим.)

*(Ольга Белимову) Продолжайте вы.
*(Белимов) Я хотел бы, у тебя, Геннадий, узнать всё-таки про перитональный период. Был ли момент, когда ты уже хотел вырваться из живота матери, и как это сказывалось на твоём самочувствии?

- О, нет! Такого никогда не было. Это был мой “дом” – зачем я должен был уйти от него? И, тем более, что я рассказывал же вам тогда, что роды, для меня это было, как представление “конца света”, и зачем же я буду тогда приближать этот конец света?

*(Ольга) Ну, выходит так, что человек, находясь в утробе матери, не хочет… То есть, это так, как мы сейчас вот живём, да, и нам не хочется умирать. Хочется продлить свою жизнь. Точно так же, как и там? Да?

- Да.

*(Ольга)И это – точно такая, как и там . И вообще – любая жизнь, да?

- Да, да, да.

*(Белимов) То есть, это не было для тебя “актом освобождения” из…

- Не-ет!

*(Белимов) …из “клетки”?

*(Гена) Пока я освоил, что это был “акт освобождения”, прошло уже достаточно много времени. А сами роды представлялись как “конец света”, “неминуемая смерть”.

*(Белимов) Угу… Ясно.
*(Ольга) Вот, мы всего боимся, оказывается, а родиться - в любом мире мы боимся, да?
*(Белимов) А ты не вспомнишь, кого хотели твои родители, когда тебя вынашивали?  Мальчика или девочку?

- По заказу.

*(Белимов) И какой заказ? Они, как раз, мальчика и хотели, да?

- Да.

*(Белимов) А скажи… Я просто не знаю, ты был тем…

- Надо было бы спросить - “а какого же это произошло числа?”

*(Белимов) Ну, и какого числа?

- Двадцать третьего февраля. (Праздничный день СА и ВМФ СССР. Прим.)

*(Ольга) Эх ты!
*(Белимов) Как раз - в праздник. Так сказать – мужской праздник и…

- И потому - и был. (заказан мальчик.прим.)

*(Белимов)… и тебя решили… А если другие так же решат – в день мужского праздника заиметь ребёнка – у них получится, как ты думаешь?

- Почему бы и нет?

*(Ольга) Да в любой день! Если…
*(Белимов) А если день 8 Марта и хотят девочку – то может получиться и девочка, да?

- Ну, нет, это не зависит. Дата не зависит. То, что говорят вам гороскопы – “родите такого-то и такого-то числа” или “ повернитесь на север ” – всё это чепуха!

*(Белимов) Угу… То есть – твои родители хотели мальчика… Я не понял…

- Существуют другие понятия – физиологические. Чисто физиологические. Есть такие периоды, когда женщина может быть беременной  - может быть – и мальчиком, а может быть – и девочкой. А там уж… как получиться, как “попадёте”.

*(Белимов) А ты был первым  в семье ребёнком или..?

-(Гена) Да.

*(Белимов) Первым? После этого - были ещё дети?

- Да.

*(Белимов) Мальчики? Девочки?

- Девочка, и ещё мальчик.

*(Белимов)Ты как понимал маленьким ребёнком – ты хотел братика или сестру – и так получилось, или совсем от тебя уже не зависело?


- Нет, я знал, что будет сперва девочка, а потом мальчик. Я даже знал их имена.

*(Белимов) До рождения знал?
*(Ольга) До их рождения.

- Нет. До их рождения.

*(Белимов)До их рождения? Ты знал имена?

- Да. Потому что я видел их!  Вы не забывайте, я был ребёнком! Ребёнок видит! Его нельзя обмануть тем, что вы “напились воды”.

*(Белимов) Ты видел ещё не родившегося ребёнка, его душу,- в семь е,- которая появилась?

- Когда родился младший брат, я его видел, но не мог говорить с ним. С сестрой же я разговаривал, ибо был младше. И разговаривал, как вы говорите ” до прихода”.

*(Белимов) На каком языке ты разговаривал? Ты ж ещё… Ну, говори. На чувствах? Да?

- А зачем нужен язык?

*(Белимов) То есть, телепатия уже работала?
*(Ольга) Да нет!

- Никакой телепатии. Телепатия… Что такое “телепатия”? – Это вы опять будете пытаться словами (передать.прим.) и не больше! Нет. Я не знаю, как это объяснить… Просто… Просто разговаривали.

*(Белимов)Угу… А как же… И ты знал заранее, что родится девочка, и имя её…знал? И как её…

- Конечно.

*(Белимов) И как её зовут сейчас?
*(Ольга) Не надо. Зачем?

- Звали,-Марина.	

*(Белимов) А что, она умерла что ли?

- Да.

*(Белимов) А от чего?

- Болезнь.

*(Белимов) Маленькой умерла?

- Нет.

*(Белимов) Взрослой?

- Это было всего лишь три года назад.

*(Белимов) Три года назад умерла? От болезни? Ясно.

-Человек умирает не именно тогда, когда умер, а загодя. За год, за два уже можно сказать, умер человек или нет. Ибо многие уже ходят мертвецами и всего лишь только ждут, когда закончатся наконец физиологические процессы, и он умрёт уже для окружающей среды.

*(Ольга) Вот я  иногда чувствую, вижу, что бывает такое, что человек…
*(Белимов) Скажи, не это ли тебе подсказало гибель мотоциклиста,- твоего родственника или знакомого там? Это может быть? Ты уже чувствовал что ли? Подсознательно? Ты не можешь в этом состоянии вспомнить, что тебя сподвигло сказать ему?

- Это чувствуют все! Это чувствуют все, но… Дело в том, что… Понимаете, это очень слабый шопот, его надо услышать.


*(Ольга) Ну, да.
*(белимов) А сестре ты… Ты что, заранее чувствовал?

- Да.

*(Белимов) Однажды ты сказал на сеансе имя акушерки, которая тебя принимала. Ты можешь повторить это имя?

- Это был не я.

*(Белимов) А кто?

- А кто был… восемь минут назад?

*(Белимов) А-а! Сейчас мы опять сменили это… уровень. На каком тогда мы находимся этапе и с кем можно разговаривать?

- В цифрах? (в переводе на уровень счёта. Прим.)

*(Белимов) Да.

- Цифры могут оставаться одними и теми же. Смысл не в этом. Смысл счёта… Понимаете, вы растёте, поэтому от вас больше требования. Раньше было проще – посчитали - вышли, поболтали - ушли. Ну, так же не может продолжаться бесконечно? Вы растёте, значит, с вас надо больше требовать. Теперь за каждую циферку вы будете отвечать.

*(Ольга) Скажите, а…. Ну, ладно.
*(Белимов) Нас интересует, какого рода вопросы можно задать? Мы сейчас пробовали проследить память переводчика, насколько он помнит даже предыдущие… ну, очень ранние периоды, которые дети обычно не запоминают. Сейчас мы какой уровень можем расспрашивать? Что он может помнить?
*(Ольга) Да то же самое, да? Можем?
*(Белимов) То же самое? Или меньше? Или больше?

- Ну… я не знаю, что помнит другой, и что он говорил вам… Но, знаете, что интересно – что воспоминания наши будут в какой-то мере всё-таки различны.

*(Ольга) Ну, да.

- Потому, что будут восприниматься по-разному. Например – есть таки…таки…

(сбой контакта.прим.)

*(Белимов)Мы сейчас разговаривали о перитональном периоде. Ты можешь это продолжить, или это уже опять “другие”?

- Вы запутались?

*(Ольга) Да.
*(Белимов) мы запутались в “персонажах”.
*(Ольга Белимову) Нет-нет! Ну…”Спрашивайте”. Вам ясно? (намекает ему, что разговор ведут “первые”.прим) Скажите, пожалуйста, вот в прошлый раз… если можно, конечно, на этот вопрос ответить…

- Мы должны вас “сбивать”.

*(Ольга) Да.

- Мы должны вас “сбивать”, и поэтому мы будем, простите, но мы будем сбивать вас с толку. И будем ” прыгать” и ” скакать”.

*(Ольга) Ну… да. В этом вы правы. Нам  нужно… Нас нужно…
*(Белимов) Учить. Ну, хорошо. Ладно.

- Нет, нет! Зачем учить? Разве мы унижали вас? Нет, мы хотим сбить вас. Вы взяли какое-то направление и идёте напролом – давайте остановимся. Как можно вас остановить? – Толчком.

*(Ольга) Скажите, вот в прошлый раз “гость” какой-то вот … Переводчик о “госте” говорил…

- Вот у него и спросите.

*(Ольга) У него, да? Угу.
*(Белимов)Ну, мы сейчас тему хотели, всё-таки, чтоб потом не возвращаться… о рождении, о перитональном периоде…

- Спрашивйте. Но только не будем говорить конкретно о нём .(о переводчике. Прим.)

*(Белимов) Ну, имя акушерки так и не может вспомнить переводчик?
*(Ольга Белимову) Не надо, не надо.

- А он его и не забывал.

*(Белимов) Ну, так пускай и назовёт.
*(Ольга Белимову) Зачем? У него и спросите.

- Мы говорим об “общих”? (намёк на тему разговора.прим.) 

*(Ольга Белимову) У него и спросите.  Задавайте.
*(Белимов) Слышал ли крики матери и пугали ли они тебя при рождении? (Б-в ещё не понял, что в теме сменился вектор . прим.)
*(Ольга Белимову) Нет, это “в общем” надо сказать.
*(Белимов Ольге) Оля, тогда я не знаю о чём спрашивать… Давай, мы тоже…

- Как вы говорите? Прежние? Вот, мы – прежние.

*(Ольга) Хм… Ну, тот же самый вопрос, только в общем… То есть, не конкретно к переводчику относящийся, а в общем. (…) другого мира, чтобы родиться здесь. То есть, он сначала развивается в утробе матери – для него это место особое. Он общается с другими…. Вот с родителями, с теми, кто общается… с кем родители общаются и видит все там…

- Пока он находится в утробе матери, у него есть возможность вернуться в тот мир, откуда он пришёл.

*(Ольга) Ага… Но…

- Но…чаще, вы делаете аборт. Прерываете естественный ход. Вот тогда ребёнок не может вернуться в тот мир, он останется здесь, ибо он – убиенный. А помните, говорили вам об убийцах? 

*(Ольга) Ну.

- У ребёнка есть множество способов уйти. 1) Уйти, не оставив нигде следа. Хотя, такого не бывает  никогда, ибо мать всегда будет помнить о нём. 2) Отомстить… в прямом смысле. Ну, значит, этот ребёнок, если б он родился, он не был бы “очень хорошим” – в вашем понятии. 3) Или, просто научить. 5) Или, просто пережить. Пережить убийство. Или, быть может… ваша версия – когда ему нужна была реинкорнация в столь маленький период, чтобы что-то понять и уйти дальше. Тогда тоже будет аборт. А если быть точнее – выкидыш. Ибо аборт – это всё-таки решение матери.

*(Ольга) Да.

- Но и тогда – мать борется за жизнь. А если быть точнее – организм. Мать хочет аборт, организм – нет, этого всё-таки хочет мать. А вот теперь и представьте, какая происходит битва внутри… И вся эта битва происходит именно где находится ребёнок… Ребёнку больше всего достаётся в этой битве. И, наконец, организм победил, аборт не состоялся и рождается ребёнок… И вы потом караете Бога, что он “ненормальный”( о ребёнке. прим) Вы будете говорить: “Нет, не пила ни курила, а ребёнок…” И вы тут же придумаете в оправдание наказание с прошлой жизни… хотя не было никаких наказаний – вы, вы сделали его инвалидом. Но вы никогда в этом не признаетесь. И тот же инвалид, он никогда не признается, что он просто ошибка родителя… Он тоже будет кричать о прошлых жизнях, ему легче признаться во всех винах. Виновен/не виновен – не важно. Ошибка природы или ошибка родителя – это никому не приятно слышать.



*(Ольга) Скажите, а вот эти … ну, заболевания что ли…

(обрыв записи. На связь вышел переводчик. прим)

*(Ольга) …Разве?
*(Белимов Ольге) Не довольны? Нет, в принципе, у меня есть раздражение. Вот как-то мы с тобой всё-таки… Ты слишком много знаешь, получается. Тогда не надо его спрашивать, а тебя спрашивать.
*(Ольга) Нет, не про то. Просто… нас учат, чтоб мы ситуацией владели…
*(Белимов Ольге)Ты всё время меня обрываешь. Меня это… Я всё-таки с ним хочу общаться. Хотя может быть с тобой тоже отдельно стоит…
*(Ольга ) Нет, я не про то, что…  (к переводчику) Ты нас слышишь?

- Слышу.

*(Белимов) Ну, это “товарищи спрашивающие”?
*(Ольга Белимову) Нет.

- Нет, это я.

*(Ольга Белимову) Спрашивайте.
*(Белимов) Ну, так мы не дали счёт…

- Почему? Вы давали счёт.

*(Белимов) Гена, это… мы в каком сейчас… Мы можем довершить вопросы о рождении, о… об этой ситуации? Ты можешь что-то ответить?

- Сам акт рождения? Нет, я не помню. 

*(Белимов) А помнишь имя акушерки, которое ты называл уже?

- Я? Нет, я не называл. Я могу…. где-то с порядка… с пяти, с семи лет. 

*(Белимов) Вспомнить, да? Угу, ясно. Но это что-то поздноватый период.
*(Ольга) Скажи, пять/семь лет – вот в этом возрасте, вот ты хотя бы , взять если – вот у тебя были воспоминания о другой жизни, или ты уже забыл о них? Или ещё какие-то были…
*(Белимов) Неясные сны, воспоминания…

- Они принимаются, как фантазии, и родители убеждают, что это фантазии. А потом - сам уже уверишь, что это фантазия и потом они из-за этого быстро забываются. Должна хоть поддержка какая-то быть.

*(Ольга) Ага.

- Понимаете, вот горит этот огонёк воспоминаний, а им никто не пользуется. Или ещё обвинят тебя во лжи – какой ты нехороший… и конечно же, стараешься вроде как “исправиться”… “всё это - глупости” – и всё, всё это уходит.

*(Белимов)А у тебя были? Обвинения были? То есть, ты что-то лишнее вспоминал видимо, да?

- Да это бывает у каждого ребёнка.

*(Белимов) А вот сейчас твой пятилетний сын тебе рассказывает – он что, что-то помнит? Ты его не правильно выспрашиваешь, не интересуешься?  Его ты тоже заглушаешь?

- Я пытаюсь не заглушить, конечно. Но так получается, потому что - ребёнок вспоминает почему? – потому, что он отвлечён. А вот когда я задаю конкретный вопрос, то уже как бы заставляю его идти в каком-то направлении, то есть – отвлекаю его. Конечно же он сбивается.

*(Белимов) Понятно. То есть, надо им чтоб какая-то непосредственность и желание. (было при общении с детьми. Прим.)

- Понимаете, можно заметить по детям – когда они говорят меж собой, они очень спокойно, раскованно разговаривают. И даже лучше разговаривают… или, может быть хуже. “Так нельзя” потому что он знает, что ему никто сейчас не сделает никакого замечания, что ты не так сказал это слово. А вот когда он кому-то рассказывает, конкретно кому-то, то он уже больше запинается, потому, что он старается вроде как сказать и боится, что его не поймут, или не дай бог ещё и прикрикнут, что, мол, “разговаривай поразборчивее!” Это очень сильно влияет на ребёнка, и поэтому, как правило, ребёнок, играя сам с собой, он как бы… Он играет просто сам с собой. Он знает свои слабые и  сильные места, и ему легче. К сожалению, это так.

*(Ольга) Скажи… Я уже спрашивала на прошлом… Помнишь прошлый контакт? Который был две недели назад.

- Нет. Я не могу его помнить.

*(Ольга) А?

- Я не могу его помнить. Я – тоже самое, что и когда не лежу, но только вот что - немножко с памятью получше и меньше скромности  – по тому же принципу, что я сейчас говорил и о ребёнке.

*(Ольга) Ну, ясно. Вот… вообще – о детях, вот, смотри – ну, ребёнок, да, фантазирует, потом его фантазии не поддерживают и потом он превращается, будем говорить, уже во взрослого практически. Да? То есть, он начинает приспосабливаться к мнению окружающих как бы. То есть …

- Да.

*(Ольга) Вот… в общем, чем мы все и страдаем, наверно. Да? В силу вот этого, как раз.

- Да. Да-да. И вот, как говорили в прошлых контактах, что если бы все вокруг не знали, что не умеют летать, то ребёнок бы прекрасно бы летал. А вы тогда задали вопрос – а если бы необитаемый остров какой-то и всю жизнь… всё детство ему доказывать, что он умеет летать – помните, какой был ответ? – Что нет, он не сможет летать, потому что он-то чувствует, что вы лжёте. Покажите. Первый же вопрос – “ А покажи!” – а ребёнок всегда старается увидеть. Ему интересней пощупать, увидеть, чем просто услышать. И первый же вопрос “Покажи!”-  и вы уже не будете знать, что сделать.

*(Ольга) Ну, а если сказать ему, например, что вот дети умеют летать, а взрослые не умеют? Он будет чувствовать ложь эту?

- Безусловно.

*(Белимов) Наверно, мы его не сможем убедить, да?

- Если он очень сильно захочет летать – остаётся только один вариант – не стать взрослым. Это тоже очень плохо. И заметьте, как интересно – в детстве мы мечтаем стать взрослыми, а когда мы взрослые, мы уже мечтаем о детстве.

*(Ольга) Да.

- Какое несоответствие – “Если б молодость знала, если б  старость – могла”…

*(Ольга) Да, это точно.
*(Белимов) А скажи, впервые когда ты почувствовал опасность огня… или мороза там? - Язык приморозил… Были? Ты помнишь эти моменты?

- Я могу спросить.

*(Белимов) Так. Что? Что помним мы?
*(Ольга) Ну, спроси.

- Нет, я могу спросить того, кто может это помнить, но только у самого себя, но я не знаю какого “себя”.

*(Белимов Ольге) Ну, мы, видишь, вышли на какой-то странный…(уровень. Прим.) Тогда так – вот в этом состоянии ты можешь ощущать, что вот на предыдущих сеансах мешало? Что наибольшую помеху вносило?

- Вы опять не поняли – я тот же, что и вне контакта, только что… Понимаете, по принципу ребёнка, я более раскован сейчас, и поэтому обладаю большей информацией. Именно из-за того, что я менее раскован. (в смысле – более раскован. Он ошибся . прим.) Вот и всё. Всё различие. Понимаете, дело в том, что вот мы стараемся что-то вспомнить, да… И чем больше мы будем стараться вспомнить, тем больше шансов, что мы как раз этого и не вспомним. И вот, когда мы уже на это “плюнули” – оно вспоминается само. Понимаете, принцип в чём? Потому, что мы выбрали какое-то вот определённое направление и давай! - идём, идём по нему! А посмотри в сторонку – вот она! А мы не можем. Для нас, самая короткая дорога – это прямая дорога. Понимаете? И получается, так же, как “лошадиная фамилия”. Всё – лошадиная фамилия – мы в это упрёмся и будем вспоминать… и лишь только в конце мы всё-таки догадаемся, что это был Овсов. Понимаете? Где-то как-то совсем не близко с лошадью, но главное, что всё-таки “лошадиная”.

*(Ольга) Вот, скажи, вот ты сейчас можешь, допустим, ответить на наши вопросы, которые по прошлым контактам мы можем там коe-какие задать? Вот я, допустим… Меня кое-что интересует из прошлых контактов. Можешь ли ты сейчас спросить у того, кто был тогда, на том контакте допустим? Или , ну…в памяти так - можешь?

- Хм… Я попробую.

*(Ольга) Ну, хорошо. Вот я тебе задам такой вопрос – вот ты… Я могу даже число назвать, не  знаю только нужно или нет… Нужно число назвать?

- Не знаю. Ещё не знаю.

*(Ольга) Вот скажи, двенадцатого февраля у нас был с тобой контакт, мыс тобой были вдвоём… Вспомнил?

- Ну-у, я-то, может быть… Я так-то помню.

*(Ольга) Мы так вопросы некоторые задавали такие…Ты был в таком же состоянии…  То есть счёт  был с 11-ти до 19-ти… и ты видел собаку,  вот которая  копошилась там в ящике – что у неё синяк – кто-то её ударил… Потом, значит, рыбок видел…ну, только  не как рыбки, а…

- Да, я помню.

*(Ольга) Помнишь, да?

- Да.

*(Ольга) А вот сейчас ты можешь быть таким же, как тогда? То есть, вот - именно  видеть что-то так? Вот, допустим, можно тебе вопрос задать? Что твориться, допустим, вот на улице  около подъезда?

- Не-ет, дело в том, что я помню сам факт, что был такой контакт. Понимаете?

*(Ольга) Ну, да.

- Но там-то я был не в том состоянии, в котором сейчас. Сейчас… Я же повторяю вам, что я тот же самый!

*(Ольга) Ну, ясно!  Но, всё равно… Ну… не сейчас, а в каком-то другом состоянии тоже… ну, тоже в этом диапазоне…

- Давайте скажем так, что – нет. Сейчас  я никаким необыкновенным я не умею пользоваться. Если у меня глаза закрыты – значит закрыты.

*(Ольга) Не… тогда у тебя тоже были закрыты.

- Ну, как вам объяснить… Ну…не такой я был – скажем так.

*(Ольга) Ясно. Вот я тебя и спрашиваю.

- Сейчас у меня, просто… Я сейчас могу свободно говорить просто – не стесняясь. Вот это – нету скованности, поэтому я могу вплоть до 5-ти летнего возраста что-то вспомнить, какие-то события понимаете… но, которые происходили… ну…как говориться – “внешние события” вспомнить. Чисто внешние.

*(Белимов)  А вот в этом состоянии ты можешь вспомнить, почему  твоего сына называли Сергеем Ивановым в детском садике? Это тоже для тебя предположительно?

- Я помню, что я ему хотел задать такой вопрос, но … я не задал ему.

*(Белимов) М-да…  (разочарованно. Прим.)

-Если… начнутся движения рукой, то, значит… как бы было событие, которое было очень неприятным и не хотелось бы его воспринимать. Тогда,  лучше, сразу “уйти”.


*(Ольга) Слушай… Вот ещё такой вопрос… Я сейчас если выйду – ничего страшного не случится?  Ну, как – “страшного”…  Просто о трёх всегда… Не меньше трёх. (на контакте просили быть в количестве 3-х…7-ми человек. прим)  Ничего “такого”…?

- Ну, вас уже сейчас пятеро.

*(Ольга) Пятеро?
*(Белимов) С собакой.
*(Ольга) А-а… ну, да… пятеро. Вот… я хочу задать вопрос, пока не забыла – вспомнила… Вот первый контакт ты проводил…вы были вдвоём, правильно? Прводили - было такое?

- Вы опять не можете понять, что вы выходите на разных.

*(Ольга)Это ясно, я не про то.

- Да, я, конечно, присутствовал при этих контактах, но поймите, что я не вёл диалог – я только присутствовал, потому что во мне сидит множество, и все они могут с вами говорить, и все хотят говорить. Понимаете, но если не вёлся диалог никогда – он не будет говорить об этом диалоге – просто закон этики – кто сказал, тот и отвечает. Понимаете? И я не могу сказать за кого-то – это только они могут. И то – они могут только дать в общем смысле. Понимаете? Не конкретизуя именно этого человека, или именно эту личность.

*(Ольга) Нет-нет, я не про то. Про то, что вы контакт начинали вдвоём. Ну, ты знаешь с кем – я не буду имя называть.
*(Белимов) С другом.
*(Ольга) Да. А потом… Довольно долгое время вы… А потом уже – стали приглашать кого-то.

- Нет, понимаете, что мы сейчас говорим вам : Вас пятеро. Но это для меня… Скажем так – для нас, для меня – вас пятеро. 

*(Ольга) Ну, понимаю.

- А уйди я на другой уровень – и вас будет всего только двое.

*(Ольга) А!

- Понимаете?

*(Ольга) Ага, понятно. Тогда, уже всё  я поняла. Спасибо.
*(Белимов) Геннадий, я могу расспрашивать о перитональном периоде? Или это совсем уже не помнится? 

- Почему же? Вы только что нас спросили, и мы дали вам ответ. 

*(Белимов) А-а…ясно.

- Но только помните об одном ограничении.

*(Белимов) Ясно, ясно… Была ли, всё-таки, какая-то обида на маму, когда ты был в утробе?

- Вы знаете, в принципе вот это вот понятие, что ребёнок не помнит обиды – это не верно. Запоминается всё. И вот эти вот обиды, или наоборот – счастливые моменты в этой жизни – они все будут складываться, для того, чтобы создать характер. И, конечно же, были  обиды, конечно же, они все не могли исчезнуть бесследно. Они оставили какую-то черту, заложили какую-то основу. Понимаете? И что здесь столь множество составляющих, что нельзя сказать, что если постоянно человека обижать – да, он может стать обозлённым на весь мир, понимаете? – а может и наоборот, потому что надо глядеть еще и на сопутствующие. Конечно же обиды были, это безусловно.

*(Белимов) А как ты запомнил – тебя “били” родители вот в период беременности, или всё-таки не очень… не ожидали как-то? Были какие-то сомнения – мальчик или девочка, типа?

- Ну, во-первых, речь шла – быть или не быть…

*(Белимов) Даже так? Да?

- Да. Речь шла об аборте. Но это было мимолетно, но всё-таки оставило… (след. Прим.)

*(Белимов) Угу…

- Что было далее? - Далее – ссоры. Ссоры родителей. Всё это обязательно сказывалось. Сказывалось даже отношение, особенно матери, к окружающим. Если она едет в автобусе и её кто-то толкает – она злится. Получается – начинаю злиться и я. Я начинаю волноваться, потому что во мне повышается те же самые химические составляющие. Это тоже оставляет след. А так как это был 61-й год , и хотя родители были молодые, по театрам они всё-таки не ходили и мне не посчастливилось слушать хорошей музыки. Чаще всего – я слушал дворовую -  игру гитары, ибо это было, всё-таки, в бараке. Понимаете? И вот это вот… любовь к гитаре…а если быть точнее - не любовь, а привычка, она осталась на всю жизнь. Она и сейчас сталась. И если вы могли пмнить, тогда… Ах, вы не можете помнить… Шестьдесят первый год – тогда… как бы вам сказать… В Волгограде был концерт органной музыки. И вот эти родители…Единственный концерт, на котором, как говорится, я был.  Он меня потряс…Понимаете? Нельзя сказать, что я слышал эту музыку, нет, я её не слышал, но те вибрации, которые создавал, мне были столь понравившиеся, что я мать замучал. Я просто мать замучал, и она дня три мучалась мной, от того, что я просто не мог успокоиться.

*(Белимов) То есть, ты барахтался там…

- Да. Восторг!

*(Белимов) Восторг? Был восторг?

- А что делает мать? Что делает организм? - Он старается успокоить ребёнка. Понимаете? А как он его успокаивает? – Естественно – отравляя его. Понимаете? Словно приглушая его.


*(Белимов) То есть, она седативное принимала что ли?

- Ну, зачем же? Сам организм вырабатывает это всё. Успокоить разбуянившееся тельце – как это сделать? –Просто “оглушить” его! “Оглушить” его - той же самой “химией”. Понимаете? Э-э… Как вам сказать… Перекрыть кислород – нельзя этого, конечно, сказать… но – недостача кислорода, недостача каких-то питательных веществ. Понимаете? Вот… То есть – защитная реакция организма, самого организма – организма матери. Мозг выдает команду… Как вам объяснить… Мать понимает, что она несёт в себе ребёнка… а организм сам – этого не понимает. Понимаете? В нём не ещё той мысли, что он несёт ещё одну жизнь, и соответственно, что для него я – как микроб какой-то или вирус, или человечек будущий – для него это безразлично. Понимаете? И реагирует, соответственно, он  так же. Но только я немножко-то всё-таки посильнее-то этого вируса, поэтому я выживаю и успокаиваюсь. Успокаиваюсь и выдаю ответное – как бы защищаюсь. Как бы называю пароль – что я свой. Понимаете? Вот в чём  проблема сейчас новой болезни… Понимаете… Чужеродная клетка – она не даёт, не называет своего пароля, что, мол, я – свой. Понимаете? И тогда получается – организм начинает бороться. Бороться - и он погибает, а значит, он отдаёт и своё генетическое. Понимаете? И получается – эта клетка как бы узнаёт код, и она притворяется “своим” – членом этого большого организма. Понимаете? И творить свои дела продолжает, которые не соответствуют этому организму – вот вам, пожалуйста, опухоль. А когда вам говорят, что это живое – помните, вам говорили, что рак – это просто вмешательство - какая-то чужая жизнь вселяется? – Да, это не ложь. Это всё так. Но только всё это…не так, что это “чужеродное” – это инопланетное. Понимаете? Это просто чужеродная клетка, это та же самая клетка, которая перешла на сторону…

*(Ольга) Ну, ясно. Ага.
*(Белимов) Скажи, а органную музыку с тех пор ты очень любишь? Уважаешь?

- Да. Восторг остался. Понимаете… любое событие – оно оставляет след и…формирует характер. Да, характер формируется  ещё в утробе матери. Рождается, потом он “куётся”, “прерковывается”, потому что человек может прожить 50 лет и резко потом измениться после какого-то случая. И, как выговорите “резко”… Ну, конечно… Даже нет – это та же самая “чаша”, которая переполнилась. Понимаете?  И естественно, если я в детстве был восторжен и привыкши к гитаре, и восторжен этим же оргАном, - то это, конечно же, осталось.

*(Белимов) Я хочу спросить, а не может ли быть восторг от органной музыки, которую ты слышал, может быть, не впервые слышал. То есть,  ты её знал по прошлым… Или в сферах четвёртого измерения ты музыка органа применялась, поэтому вызвала восторг? Может быть такое?

- Понимаете, я помню себя только, как говориться, ” мальчиком с камнем”, не больше.

*(Белимов) А-а! Ясно, ясно.

- Я …помню этот камень. И в то же время здесь трудно сказать … Да, вроде как держу в руках камень, но… вообще-то, тело-то  “не то”.И руки, ноги…  Ну, я не знаю, как сравнить. Например – я был одновременно и этим камнем, ибо я нёс информацию о самом себе, и в то же время я был извне этого камня, потому что мне нужна была эта информация, чтоб воспользоваться ею. Понимаете? И поэтому, нельзя здесь сказать, найти какие-то критерии. Грубо будет сказать, что я нёс отдельно этот камень, и грубо будет, если я скажу, что я был этим камнем. Всё это столь слилось… Понимаете, мне нужна была информация. А раз мне нужна была информация, то я – только инструмент, который обрабатывает эту информацию. Понимаете? И в то же время, я была и сама информация, чтобы , как говориться, “инструмент владеть инструментом”. Вот как вы думаете, инструмент владеет информацией, или информация инструментом?

*(Белимов) Наверно, всё-таки инструмент.

- Понимаете, обоюдно. Всё обоюдно.

*(Ольга) Да.

- Ибо инструмент не может делать что угодно и как угодно – он просто не будет знать , как это делать. Поэтому , ему будет нужна информация. Правильно? Значит, он пользуется информацией, он “выше” этой информации. Но, простите, та же самая информация , получается, владеет тем же самым инструментом. Правильно? Потому, что инструмент выполняет то, что задала она. Согласны? Так кто же из них главней? Да никто.

*(Ольга) Никто. Вот, смотри, я вообще-то родилась в такой семье, в рабочей, где ни мать, ни отец  особо…  тоже, как у тебя, классическими какими-то там не увлекались и вообще – в таком месте родилась, где и концертов не было никогда, наверно, в те времена, но, ты знаешь, я вот с детства… Ты слышишь?

- Слышу.

*(Ольга) Вот, с детства начала увлекаться и классической музыкой и классической литературой. Всегда поражались – ну, с  чего это вдруг я…. Ну, в принципе – выпадала я из их понимания, так сказать, - их жизни. Ну, это я не то, что хочу похвалиться или ещё… А просто…это может быть, так сказать, идёт из прошлой жизни, да?

- А вибрации? Вибрации, которые более подходят к вам.

*(Ольга) Ну, вот… А-а! Вот как. Всё тогда.

- Понимаете, конечно, можно жить где-то в глуши. Может вы в детстве в индейском племени родились. Конечно, там никакого оргАна не будет и никаких концертов он не услышит, кроме как бубна и барабана. Но, понимаете, придя на большую землю, он будет рад услышать тот же самый оргАн и отказаться от того же бубна. Но это не значит, что это было с прошлой жизни, это вибрации. И человек стремится к тому месту, где он может получить эти вибрации. Понимаете? Ну, зачем индеец приехал на большую землю? Понимаете? А если мы вам скажем, что он приехал, чтоб услышать этот оргАн…знаете, мы вам не соврём. Конечно, это не единственная причина, но одна из причин.

*(Белимов) Гена, а маленким, крошкой, когда ты больше спал – можешь ты вспомнить, посещал ли кто-то тебя помимо родителей, из другого мира?

- Безусловно. Конечно же да.

*(Белимов) А как…Как это выглядело?

- Вы знаете, дело в том, что если передавать на физический уровень, то бишь что мозг запомнил, что был приход, да… что я не один, то приходится приобрести форму, которую бы мозг не отвергал. Понимаете? Тот же самый дедушка, и те же двое – мужчина и женщина…Понимаете? –они не имеет этих фигур, этих форм они не имеют. Понимаете? Но для физики этого мира – они оденут одежды и всё причитающее к ним. Понимаете?  Ну, давайте скажем так… Вот, смотрите, умирает человек, он идёт к богу. Но никогда христианин не попадёт к богу-кришнаиту. Понимаете? Он просто испугается. Вы согласны же?

*(Ольга) Ну, да.

- Бог-то один, но ему придётся одеть форму христианского бога. Понимаете? – чтобы просто не испугать его!

*(Ольга)  Ну, да. Чтоб ближе был. Роднее.

- Да. Та же самая вибрация. Понимаете? Это всего лишь речь идёт о вибрации. Понимаете? Если вы всю жизнь были христианином, значит у вас соответствующая вибрация, и характер  совершенно другой, чем у того же кришнаита, мусульманина или ещё кого-то. Согласны?

*(Ольга) Да-да.

- И идя с этими вибрациями, вы в резонанс войдёте именно  с теми же подобными вибрациями. И если вы всю жизнь кричали, что бог - сволочь и негодяй, и всех наказывает… то вы и придёте к этому богу, и он вас накажет, потому что ваши вибрации соответствовали этому богу. Понимаете? Бог-то один, но… а вибрации каковы? И если будете говорить, что бог всемилостив и не прокурор и простит – то бог простит. Но, понимаете, если вы будете, как говориться, убивать, насиловать… грешить направо и налево, и при этом будете кричать, что бог не прокурор, а судья – всё равно вы придёте к карающему богу, потому что вы-то знаете, что вы делали плохое. И вибрации, истинные вибрации, всё-таки те вибрации - вызванные страхом – о каре. Понимаете?  Вот почему ребёнок попадает в рай, каким бы он не был. Почему? Да потому, что он не создал ещё этих ложных вибраций! Понимаете? В отличие от вас - взрослого. Вы верите в бога, который вам всё простит, или вообще ни в какого бога не верите. Понимаете? Но если вы верите во всепрощающего бога и тут же грешите – вы-то знаете, что вы грешите. Понимаете? В вас сидит, всё-таки, этот “червь сомнения”.Вы согласны?

*(Ольга) Да.

- И вот это вот и есть ваша основная вибрация. И соответственно, на основную не наложена, ибо ложную вы отбросите. Потому, что когда вы умерли, вашего сознания уже нет. А ложную вибрацию создало только лишь сознание.  И вот эта истинная вибрация приведёт вас к этому богу. А там уже  - можете возмущаться и кричать сколько хотите – “Почему же ты меня не прощаешь?!” – это будут уже ваши проблемы, но вы не поймёте вообще-то. Понимаете? И вот почему ребёнок, по той же причине…ребёнок всегда попадает в рай. Ибо он не делает специально. Если и сделал он какое-то зло – но он же не делал его специально! Правильно?

*(Ольга) Да.

- Потому, что это у него это – чисто эмоции и не больше. Нет разума в его преступлениях. Понимаете? – чистые эмоции. И поэтому, он… И причём, заметьте, чистые эмоции – потому, что эмоционально, взрослыми, мы можем кого-то убить сгоряча, - но это не то, не путайте! Понимаете? То есть, то, что вы заслужили, то и получите, что посеете, то и пожнёте. Понимаете? Не больше, не меньше. И, естественно, есть понятия о ложных мирах. Но вам говорили уже об этом.

*(Ольга) Да. Скажите, а по той же причине люди уходят из глухих деревень в столицы…ну, вот, становятся учёными большими? Именно эти вибрации заставляют… То есть, поиск этих вибраций соответствующих заставляет?

- Да, конечно. Конечно же, безусловно, только это и это. Понимаете…человек родился – он знает свои задачи. А если быть точнее – он приходит, чтобы родиться – он знает эти задачи. Но когда он пришёл и начинает приобретать плоть, будучи ещё в утробе, он уже постепенно начинает забывать. Понимаете? И те же самые…Перед родами, в цифрах…Пусть будет в цифрах, то процентов 50 он уже забыл. Ибо, потому, что его матрица была чиста, но он…Его матрица смешалась с матрицей матери. Понимаете? А матрица матери засорена множеством других матриц окружающей среды. Понимаете? И вот эта грязь, скажем так, оседает…Не обязательно грязь, нет конечно.

*(Ольга) Ну, ясно – много чего.

- Да. Но, понимаете, вот что получается… Ребёнок – заметьте, когда ребёнок родиться, на кого он в первую очередь похож? На маму…а через год – на папу… А потом – ни на кого. Понимаете? Это действительно так. Это закономерно. Это , практически, все. Почему? Потому, что в нём борется множество матриц. Грубо – это борьба. То есть, организм пытается очиститься от внешнего. Понимаете? Но это не всегда получается, и поэтому у него остаются какие-то привычки. А заметьте, привычки родителей тоже передаются, хотя они столь короткие! И если верить науке – то они никак не могли передаться детям, ибо эти привычки имели очень короткий промежуток. Согласны? А они – передаются! Почему? Потому, что была наложена матрица! Потому, что когда-то, ребёнок и мать - были единым организмом. Их нельзя было различить никак, только с помощью хирургии, и то = это называется убийство, как и матери, так и ребёнка. И, вот, смотрите, рождается ребёнок, при рождении – сколько мук преодолевает он? Для него это – смерть. Понимаете?

*(Ольга) Да.

- И лишь только потом, с первым криком, он понимает, что он живой! Но он живой уже в совершенно другом! (мире. прим) У него появилось тело! На которое, почему-то, кто-то всё время давит…Понимаете? Ему надо справиться с этим телом. Как вы говорите? “Судорожные движения ребёнка”? Нет, он просто хочет понять, что с ним происходит, почему его ручки и ножки сначала были свободны, а потом их кто-то постоянно давит, жмёт, ломает… Понимаете? И он хочет…он хочет с этим справиться. Первое его – это скинуть. Понимаете? Скинуть всё. Почему первые дни рождения и опасны. Понимаете? Первые 9 дней после рождения  чаще – ребёнок умирает. Почему? Потому, что он может не справиться с этой задачей. И причём, это может умереть даже совершенно здоровый, в физическом понятии человек. Он может умереть. И так он постарается, старается с этим телом бороться, но он не может побороть.  Раз тело ему дали – значит дали, что ж с ним сделаешь? И тогда, он должен уже учиться. Он уже учится овладеть им. Сперва он это делает силой – и опять ничего не получается! Понимаете? Он пытается достать до той же побрякушки, но у него это не получится, это будут хаотические движения… Иногда он будет доставать до этой игрушки. Понимаете? И он постепенно будет учиться запоминать, как он это сделал. И, смотрите, вот ребёнок, который лежит в определённой позе… Он может дотянуться до какой-то игрушки, если он захочет. А возьми и поменяй ему позу – и он это не сможет уже сделать, он уже промахивается. Почему? Потому, что не было ещё опыта этого. Вот почему ребёнок очень подвижен в малом возрасте – чтобы изучить своё тело “от” и ”до”. Понимаете? Вот вам, пожалуйста – вы приходите к невропатологу, он вам говорит: Закройте глаза и достаньте кончик носа… - Всё, то же самое, это делает ребёнок, только он сам себе, как говориться, невропатолог. Понимаете? Возьмите, хотя бы его глаза. Он не может их спрямить, он не может их сфокусировать. А вот, почему? Потому, что он хочет испытать все движения. Вы сейчас скажете – нет, просто у него ещё нет, как говориться….мозг ещё не обладает этим телом. Но мы говорим то же самое, просто другим языком – “испытать”. Испытать и проверить все возможные варианты, как бы заранее подготовиться к любому другому…. Понимаете?

*(Ольга)Да-да.
*(Белимов) А правда, что ребёнок видит в перевёрнутом виде?

- Да. Безусловно.

*(Белимов) А почему оно потом налаживается?

- А вы возьмите, возьмите сейчас очки, которые перевернули бы изображение…

*(Белимов) Так…

- И оденьте. И первые несколько дней вы будете ходить и видеть перевёрнутый мир. А потом, опять вы увидите его нормальным. И вот, когда вы снимете очки и будете уже без этих очков, у вас будет перевёрнутый мир. Потому, что мозг настроился на очки.

*(Белимов) Угу… Ясно.

- Понимаете, мозг обустраивается! Задача мозга – выжить! Выжить в окружающей среде. Душе проще. Ей наплевать, какой будет окружающий мир, ей надо выполнить свою задачу. Она может прекрасно воплотиться в таракана, в травку, в любое животное, соответствующее своей задаче. Понимаете? Ей всё равно. А вот мозг… Его задача, как раз-то, сохранить себя и соответственно – свою физическую оболочку. Он, к сожалению, не думает о душе. Понимаете?

*(Ольга) А его задачи?

- Он озабочен этим, дело только не в этом. Вот, вы говорите: “Зверь какой-то, а не человек! В нём нет души.” Почему? Потому, что мозг озабочен, как говориться, в сознании самое главное – это что? – Еда! А всё остальное потом! Мы едим для того, чтобы работать.

*(Ольга) А потом – для того, чтобы есть.

- Да. В итоге, он просто напросто….

(Конец кассеты и контакта)
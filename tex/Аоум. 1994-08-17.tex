Аоум. глава 22-я 17-08-1994г
Георгий Губин
 17-08-1994
(Белимов) …сеансы, эти возможности связи, контактной ситуации, и Геннадий, вот у него, как раз, получается выход на контакт. Ну, наверное, мы попробуем. Геннадий, сейчас попробуем это продемонстрировать. Не уверены, получится или нет, в той ли степени получится, как мы желали бы. Сейчас, всё это на ваших глазах произойдет. Геннадий, пробуй.
  
 (Гена) Здесь я ещё не спал.
 (Работник студии) Так, какая рука будет действовать? Правая рука у него будет работать, да? Вот эта?  Так, сам ты будешь лежать спокойно?
  
 (Гера)  [Прямой счёт семь раз] 1-2-3-3-4-5-6-7-8-9.
\soul{Спрашивайте.}
\people{Хорошо. Скажите, на ``переводчика'' сильно влияет необычная обстановка?}
\soul{Мы же говорили вам, мы не покидаем вас. Всё зависит от того, сможете ли вы услышать нас.}
\people{Мы стараемся, но сегодня необычная обстановка. И мне кажется, ``переводчик'' трудновато выходил на сеанс.}
\soul{А вы подумайте, для вас она обычна?}
\people{Да, и для нас, тоже, волнительна. Необычная. Мы волнуемся. Вы утверждаете, что существуете в эмоциональном плане землян, что вы бестелесны и представляете особый вид энергии. Сегодняшний сеанс в эмоциональном плане как-то отличается от прежних наших сеансов с вами?}
\soul{Вы не внимательны. Мы не говорили, что мы существуем, мы говорили, что приходим в ваш план. Вспомните. }
\people{Это существенная поправка. Да, мы забыли, что вы именно приходите. }
\soul{Что помните вы?}
\people{Ну, мы не совершенны, конечно, в нашей памяти. Ну, надо тогда находить наши общие точки соприкосновения, понимания. Каково на ваш взгляд сейчас состояние ``переводчика''? Он скован? Или вам сегодня, всё-таки, как-то  легче общаться через него?}
\soul{Нам всегда было трудно приходить к вам. Независимо, к нему или другим. Ибо вы полны другими заботами. }
\people{Так оно и есть. Скажите, а вот перерыв наших сеансов, мы, вот, на лето прекращали, он как-то сказался на вашем отношении к нам, как к исследователям? Есть ли  у вас элемент осуждения за прерванный контакт? Например, на месяц? Мы с вами не общались. }
\soul{Мы же говорили вам, и говорили сегодня – мы не покидаем вас. Вы, или желаете, или не желаете слышать нас, потому для нас нет перерывов. Вспомните, мы говорили вам и о времени. }
\people{Угу. Мы сегодня хотим продолжить…}
\people{1-2-3…}
\soul{Спрашивайте.}
\people{Мы, сегодня, хотим продолжить прерванную в прошлый раз тему, о жизни после смерти, о жизни души, состоянии души. Вот скажите, что вы можете пожелать тем, кто категорически не  согласен и не приемлет, что существует жизнь после смерти? Что вы им скажете?}
\soul{Ничего. }
\people{То есть, они сами должны дойти или…}
\soul{А вы подумайте, если вас не хотят слышать, вы будете кричать? Далее, если вы не верите – это ваши проблемы, и куда вы попадёте,- вспомните, вами сказано ``По вере вашей!'' - вы помните? Вот и подумайте. Если вы не верите –  вы вернетесь, вернетесь и будете возвращаться до тех пор, пока не поймете.}
\people{Ну, вот этим сеансом мы хотим лишний раз, может, показать нашим оппонентам, что существует какая-то связь с иным миром, в частности через  посредника, через словесный, и слуховой, и речевой аппарат переводчика. И я, надеюсь, у кого-то хоть поколеблятся их сомнения.}
\soul{Мы же говорили вам, каков наш ``вес''. Вспомните. }
\people{Ну, ваш авторитет пока не велик, действительно, потому, что это очень сомнительно для многих. }
\soul{Дело не в том. Если вы не желаете – вы не увидите, не найдете, даже если будете существовать там. Если же вы хотите, то вам не нужно будет… }
\people{1-2-3-4-5…}
\soul{Спрашивайте.}
\people{А вы не можете подсказать, что сейчас мешает переводчику?}
\soul{Если хотите, то - ваше здоровье. Не конкретно вас. }
\people{Конкретно меня?}
\soul{Нет.}
\people{А-а, ну, в общем. Может быть - переводчика. Тогда, я приступаю к вопросам…}
\soul{Нет. Среди вас есть, как вы говорите ``сердечник'', есть и ``болезнь лёгких''.}
\people{И это сказывается?}
\people{1-2-3-4-5…}
\soul{Спрашивайте.}
\people{Хорошо, приступим к вопросам.  Скажите, почему у человека существует инстинктивный страх перед небытием?}
\soul{Страх? }
\people{Да.}
\soul{Что вы называете страхом? Боязнь потерять то, что имеете сейчас – вот ваш страх.}
\people{Нет, ну может быть страх, что нас больше не будет на Земле. Может быть, вот этот?  Он -движущий?}
\soul{Боязнь. Боязнь, основанная на эгоизме. Боязни и жадности жизни. Жадности той, что имеете.}
\people{Вы так оцениваете?}
\soul{Подумайте! Вы боитесь потерять то, что уже имеете и не хотите приобрести новое. Далее. Если бы у вас не было страха, что бы вы делали? Что бы вы делали? Вам надоело жить –  ``Пойдёмте дальше''? }
\people{Ну, этот страх… Он движущей силой у человечества является?}
\soul{Мы же говорили вам, что нет чисто отрицательных или положительных качеств. Нет. Для кого-то, это движет, для кого-то - нет. }
\people{1-2-3-4-5-6…}
\soul{Спрашивайте.}
\people{Скажите, вот, как бы нам не внушали, что человек смертен, что он один раз только живёт, откуда, всё-таки, у человечества, у людей, заложено предчувствие бессмертия? Оно как-то заложено где-то в генах. Откуда?}
\soul{Мы ответим вам так же, как и в прошлый: жадность жизни. Вы согласны?}
\people{Ну, да, вообще-то.}
\soul{Далее. Вы же, вы же писали и говорили, о памяти. Подумайте. Мы будем говорить вам в общем. Мы никогда не скажем вам точно, ибо тогда, вы не сумеете найти сами. Вы будете рады готовому ответу, и всё. А мы приходим к вам не ``шпаргалками''. }
\people{Ну,  мы ценим вас за это. За то, что вы так себя ведёте с нами. Стимулируете нас думать и самим добиваться. }
\soul{Мы не будем говорить вам конкретно  ничто, и будем сеять в вас сомнения. Сомнения даже в том, что есть ли мы вообще или нет, ибо  это заставит вас думать.}
\people{Да, мы это ощущаем.  Хорошо, продолжим вопросы. Откуда, всё-таки, у людей  есть убежденность в конечности бытия? У тех, кто не верит в продолжение жизни.}
\soul{Материализм. Это ваша наука. Далее, - мы говорили о борьбе, вспомните,- в вашем понятии ``добра'' и ``зла''. Вот и подумайте. Приходит к вам новая вера, и она говорит, что жизнь одна. Для чего? Чтоб вы более уверовали в эту веру. Вот вам, одно из правил христианства. Вы столь беспечны, что даже Бога сделали беспечным.}
\people{Ну, в наших глазах…  Наверное, он не такой беспечный. Это нам кажется. Так?}
\soul{Нет. Вы творите. Каким представляете – таким и будет. Вспомните – ``по вере вашей''.}
\people{Скажите, почему христианство довольно упорно отрицает идею перевоплощений многократных?}
\soul{Мы же вам ответили только что. Чтобы укрепить веру, чтобы вы были рабами веры той. Хотя, тут же и оправдываетесь - ``Не создайте кумира''.}
\people{Это наше заблуждение человеческое, да?}
\soul{Это ваше. Только ваше.}
\people{Какие чувства преобладают у людей в миг смерти?}
\soul{Боль.}
\people{Ну, это разумеется.}
\soul{Нет. Вы опять говорите, о физической боли, а мы вам  говорим, о иной. Боль. Сперва, боль, о том, что не успели и теряете. Потом, боль, что вы потеряны, боль, что вы никому не нужны. Вспомните описания умерших. Вспомните. И лишь только когда вы увидите огонь, вы приобретаете  новую боль, -  боль страха, - или обжечься, или не успеть. Всё зависит от того, во что вы верили.}
\people{Скажите, а мы читали по Раймонду Моуди, что у некоторых, просто радость появляется в  миг смерти. Они поняли, что они здесь были в клетке и…}
\soul{Вы говорите: `` в миг смерти''. Тогда, назовите мне длину этого мига?}
\people{А-а. Угу. Понятно. Почему люди довольно примитивно представляют муки и радости после смерти?}
\soul{Если вы не помните, и  лишь только фантазии ваши. Фантазировать, вспомните, мы говорили, вы не умеете, - вот вам и примитив. Далее. Почему вы говорите, о всех? Знаете ли вы всех? }
\people{В чём заключается функция добрых духов?}
\soul{По вашим понятиям, есть и добрые духи и злые. А мы говорили вам, что это  - вы же. Есть же в вас и доброе и злое?  Подумайте. Подумайте и примените к себе. Вы - дух. Вас можно назвать добрым или злым?}
\people{Наверное, смешано всё вместе.}
\soul{И для чего вы живёте? Вы знаете свою цель? Нет. А хотите узнать иные. Вы не знаете себя, а хотите узнать иное. На кого вы похожи?}
\people{Скажите, насколько наивно представление людей, что в райской жизни духи только и делают, что поют дифирамбы богу.}
\soul{Нет. Мы как-то говорили вам, что в раю сложнее, намного сложнее, чем здесь.  И мы говорили вам, что для кого-то, ваш мир - рай. А есть миры, где ваш мир - ад. Вот и подумайте, вы же говорили о бесконечности миров. Вы говорите: ``Рай''. Неужели вы думаете, что вы будете отдыхать? Быть может, это только начало, начало битвы?}
\people{То есть, душа там трудится?}
\soul{Более. Поймите, вы же говорили ``Рай выше''. Ваши слова. Значит, и более трудиться будете, и более ответственны будете. Вспомните, маятник. Чем выше – тем более сил, и тем ниже вы можете опуститься. Вы помните? Сложите – и будет вам ответ.}
\people{Скажите, а если человек при жизни не хотел трудиться духовно, душевно, он и там, наверное, будет лениться также, или его там заставляет что-то?}
\soul{А вы пойдёте на один из уровней. Один из уровней, в вашем понятии, где все не хотят трудиться.}
\people{Угу. То есть, если душа стремится трудиться, то поступит, попадёт  на более высокий уровень, там, где будет воплощаться его идеи, его стремления трудиться? Да?}
\soul{Давайте скажем по-другому. Мы говорили вам о мирах, для которых вы являетесь адом или раем. Так вот, ваш мир – это всего лишь один из миров, из бесконечности миров, что вы называете адом или раем. И, относительно вас, всё, что ниже вас – ``ад''. Всё, что выше, вы называете ``рай''. }
\people{Хорошо. Почему, в целом, осуждается стремление людей к материальной независимости, к преуспеванию? Это идёт из религий.}
\soul{А вы подумайте, много ли вы успеете, если будете набивать сумы? И какой ценой вы будете делать то?}
\people{Ясно. }
\soul{Далее… Сказано было: нищета – это не значит, что вы будете в раю. Вспомните, было сказано. Многие же из вас: ``Если нищ – буду там''. Нет. Чаще, вы нищи от лени. И потому, вы всегда находите причины и виновных.}
\people{Скажите, а в чём заключается страдание низших духов?}
\people{1-2-3-3-4-5-6-7-8-9…}
\soul{Спрашивайте.}
\people{Скажите, в чём заключается страдание низших духов?}
\soul{Вы невнимательны. Мы говорили вам о вашем мире и бесконечности ``адов'' и ``раёв''. Вы помните? Чем вы страдаете? }
\people{Ну, разное бывает.}
\soul{Мы же говорили вам, души умерших  – тот же мир, что и вы. Он не намного лучше вас. И единственное, что вы можете, в вашем понятии, иметь контакт с соседними мирами выше и ниже. И ``выше'' – вы называете ``раем'', хотя, не намного и более…}
\people{1…}
\soul{Спрашивайте.}
\people{Хорошо. Скажите, смерть не всегда освобождает человека от искушений?}
\soul{Нет! Вся жизнь ваша – искушение, и смерть ваша – одно из искушений! Вспомните, ибо вами было рассказано, как вы падаете и поднимаетесь и выбираете миры, в которых жить будете. Вот вам и искушение.}
\people{То есть, человек и в загробный мир тянет свои искушения и не может от них избавится там?}
\soul{Что вы понимаете под искушением? Что это в вашем понятии? Желание в рай или в ад? Согласитесь, это искушение! Желание видеть Бога – это искушение! Вы согласны?}
\people{Да, согласны. Ну, я имел ввиду, вообще-то, тлетворные искушения, не хорошие.}
\soul{Нет. Смерть ваша, всего лишь потеря, потеря одного из тел. А далее зависит -  или вы никогда больше не получите того тела, и, значит, будете в более ``тонких'' мирах или получите то же, или ещё хуже, в вашем понятии - ``ад''.}
\people{То есть, с потерей энергетики человек приобретает более грубое материальное тело?}
\soul{Нет! Вы говорите ``с потерей энергетики''. Тогда,  вспомните ядерную физику. В вашем понятии, минералы -  имеют энергии более!}
\people{Тогда,  подскажите нам, каковы самые большие страдания, которые могут претерпевать злые духи?}
\soul{Почему вы говорите о злых духах?! Почему вы не спросите о себе? Почему вас интересует что-то? Страх узнать о себе или нежелание? Мы говорили вам, ВЫ будете наказывать себя. Вы! По вашей вере, по вашим поступкам. Если хотите - совесть ваша. Совесть, но открытая.}
\people{А если совести особо нет? }
\soul{Разве?}
\people{То есть, она ущемленная, или у всех одинаково развита? Совершенно бывают разные люди.}
\soul{Нет. Мы вам говорили  - души чисты. Есть только грязь, грязь поступков, созданных из мыслей. А мы вам говорим о совести, что будет чиста, открыта, где уже мозг не будет давать поправки. Вот - муки ваши.}
\people{Тогда, по большому счёту, что… как… нам… вроде  и жить не стоит, я так понял?}
\soul{Разве?}
\people{Нет, это не следует…}
\people{Или как? Устранять влияние мозга, что ли? Жить ``чисто по совести''?}
\soul{Вы спрашивали, о муках. Вы не спрашиваете, что вам  делать  сейчас, а спрашиваете, что будет за наказание вам. Разве в этой жизни вы не были наказаны? Разве никогда вам не мучила совесть? И вы всегда находили себе оправдание, мозг ваш всегда оправдывал вас. Желание ваше - искушение! И вы делали обманом. Всмотритесь в себя, всмотритесь, сколько вы задушили в себе, сколько?!}
\people{Но есть…Счёт!}
\people{1-2-3…}
\soul{Спрашивайте.}
\people{Но есть большая категория людей, у которых, как будто бы совесть отсутствует. Как они? На том свете или где-то воздастся ли им за неверие? }
\soul{Вы невнимательны. Мы вам говорим, о душе,  и что она чиста. Есть только поступки, созданные мыслями. Подумайте! Вы говорите: ``Нет совести''. А может быть просто, вы так залгались, что даже не видно правды вашей, совести вашей? Но придёт время, и даже при жизни, когда вы остановитесь, оглянитесь и - вот вам наказание! И представьте, когда вы умираете, вы теряете, в вашем понятии, мозг. Остаётся только ваше ``я'', и мозг вас уже не оправдает, и не создаст лживых истин. Вы поняли?}
\people{Угу. Хорошо. Но поскольку духи не могут скрыть друг от друга своих мыслей, и поскольку все поступки  их открываются и известны, то, из этого получается, что виновник находится в постоянном присутствии своей жертвы на том свете?}
\soul{А вы подумайте. Вы и здесь не уходите от жертв. Они рядом с вами, они всегда с вами. Только, вы стараетесь их забыть. Подумайте!}
\people{Вы правы. А там, на том свете, это ещё более очевидно? Там уже мысли…}
\soul{Давайте договоримся так: не ``на том свете'', а, если хотите, вашим языком – ``промежуточном''. Ибо ``на том свете'' –  это значит, что вы уйдёте в иной и приобретёте одно из тел, и будете жить снова.}
\people{Это так получается? Да? Мы, вообще-то, не слишком осведомлены, о ``том свете''.}
\soul{Один из миров, что находится в вас. Мы же говорили вам: ``Мы рядом''. Вы помните? Вспомните, вы ищете иные миры, а в вас? Сколько в вас этих миров? Сколько? Посмотрите в себя. Возьмите хотя бы день сегодняшний и гляньте, сколько раз вы изменялись, сколько раз вы были иными, сколько было вас сегодня? Сколько?! И  все они, в вашем понятии, на ``том свете'' соединятся, и будут помнить все деяния. Все, которые вы хотите забыть здесь. И вот тогда вы скажете: ``За что мне наказание - помнить всё?''}
\people{Скажите, а когда нас мучает совесть – это наказание за то, что мы совершаем в этой жизни?}
\soul{Чаще, то, что сделали вы здесь. У вас нет, в вашем понятии, памяти, о ``прошлой жизни''. Ибо память ваша, чем оперируете здесь, – химическая. В вас же не было ``прошлой жизни'', не было вашего мозга. Согласитесь, и значит, нет памяти, о прошлом, и нет памяти, о будущем. И есть только, в вашем понятии, ``ячейки'', куда вы можете войти, и, лишь только тогда, память химическая будет. Но она не будет мешать вашей совести, ибо, чаще, мозг ваш - хороший сторож.}
\people{Скажите, вот интересный вопрос такой, а преследуют ли жертвы на том свете своих убийц?}
\soul{Давайте договоримся, о каком свете говорите вы?}
\people{Ну, потустороннем.}
\soul{Промежуточном, в вашем понятии? Нет. А вы подумайте! Если  всех вас мучает совесть… Подумайте.  Далее. И в этой жизни вы можете преследовать, не зная, что вы являетесь жертвой. Подумайте.}
\people{Не очень ясно.}
\soul{Не ясно? Хорошо, вы это называете ``интуицией''. Вы видите человека, вы не знаете его, но он вам не нравится, как вы говорите ``не лежит душа''. Это и есть одно из понятий ``старая память''.}
\people{То есть, когда-то, у нас были отрицательные контакты, так что ли?}
\soul{Поймите, химически вы не помните ничто, но мозг ваш ключ, ключ ко множеству, множеству. И в этом множестве хранится всё, в этом множестве есть всё. И вы, невольно, увидя этого человека, настраиваетесь на него… Согласитесь,  первая  мысль - ``Что за человек?''. Вы согласны?  И уже этим вы хотите понять и раскрыть его, уже этим вы открываете ячейку. И, слава Богу, в вашем понятии, что вы не можете открыть её полностью! Иначе, вы повторили бы прошлое. Если вы были убийцей, то вы здесь будете им снова, ибо вспомните и  будете мстить или иное. Потому и не дано вам – память. Потому существуют ``реки забвения''. }
\people{Понятно. Скажите, а вот убийцы в промежуточном мире бывают наказаны своими жертвами?}
\soul{И да, и нет! Поймите, в промежуточном мире жертва не может наказать,  ибо она мучается сама, ибо для неё тоже есть множество жертв. Вы, каждый, имеете множество жертв. Каждый. И что же вы - всем мстите? Нет. Вы все, в вашем понятии, знаете всё и вся, о всех. И потому, вы не можете обвинить кого-то, зная, что в этом виноваты вы. Мы же говорили вам, о чистоте совести. Но есть другой мир, где вы получите уже тело, и там – да, если вы что-то… Спрашивайте.}
\people{Скажите, а омрачают ли ошибки ту душу, которая от ошибок очистилась?}
\soul{Почему вы повторяетесь? Мы вам ответили уже множество раз, и вы повторяетесь. Подумайте, - вы задаете одни и те же вопросы, но разными словами. И вы хотите, чтобы мы вам отвечали так же? }
\people{Хорошо. Какое состояние, для живущих на Земле, лучше;  боятся смерти, знать,  что ты бессмертен или полное равнодушие к смерти?}
\soul{Каждому своё. Для кого-то,  страх может вести, для кого-то,  он может остановить. Подумайте, мы же говорили вам: нет истинно плохого  и истинно хорошего. Сейчас, быть может, вам нужен страх, может быть завтра - вам будет нужна уверенность или иное. Подумайте! И мозг ваш выберет сам. Здесь он не ошибётся, ибо здесь он будет подчиняться инстинкту, всё остальное  - уже будут ваши мысли.  Самая первая мысль верна, ибо она истинна, всё остальное уже - ваше. }
\people{Я могу сказать о себе, знание  того, что мы живём вечно, полученное мною из  фактов, из ряда книг, которые мне попадались, как-то, сейчас облегчает мою жизнь. Оно делает её более оптимистичной, мне становится интересно уже побывать там, в ``том мире''.  }
\soul{А теперь подумайте. Иной скажет: ``Благодаря тому, что я знаю, мне стало страшно, ибо значит, со смертью не кончается, и я буду жить дальше и мучиться''. Согласитесь, те же самые  источники, но разные мысли и разные жизни. Спрашивайте.  }
\people{Хорошо. Вот существует мнение, что жизненные беды неурядицы, которые бывают у отдельных людей, всегда ли они - наказание за прежние и нынешние ошибки? }
\soul{Да. }
\people{Случайностей не бывает? Всегда?}
\soul{Да. Но это не значит, что вас наказывает Бог. Нет, здесь вы можете даже предсказать математикой. }
\people{Даже математикой можно?}
\soul{А вы подумайте. Следствие, следствие – и не более. Вы же, чтоб оправдать себя, говорите: ``кара Божья''.  ``Бог виноват или дьявол, но не я''. Подумайте, вы постоянно лжёте себе, оправдываете себя. }
\people{Вот, есть элемент несправедливости, если это наказание за прежние ошибки, в прежних жизнях, о которых человек  подчас  и не подозревает, то разумно ли такое наказание?}
\soul{Разве? Когда же вас наказывали за ``прошлую жизнь''? Вы можете мне подсказать? Вы мне приведёте примеры больных. Разве? }
\people{Есть ли кармические,  так называемые,  наказания?}
\soul{Кармические? А что - карма? Что значит - ``карма''? Следствие! Вы согласны? Так кого же вы обвиняете? Кого?}
\people{Но человек в этой жизни не помнит, что он натворил в прошлой, а его, жизнь - в этой жизни наказывает и сурово подчас. }
\soul{Не помнит? То беда ваша, что вы не помните. Чаще, вы не хотите помнить. Боитесь помнить.  Вспомните сны детства. Подумайте, почему дети говорят вам: ``Когда вы будете маленький…'' Ведь если вами рассуждать, вы же не можете стать маленьким? Почему дитё помнит, а вы нет? Почему? }
\people{Ну, это уходит, видимо.}
\soul{Уходит? Ваше воспитание, ваш страх, ваше нежелание знать. Вам легче жить в темноте. }
\people{Скажите, что происходит с человеком, который не делает зла, но и добра  тоже не несёт?}
\soul{Как это можно? }
\people{Ну,  есть  такие уникумы. }
\soul{Как? Вы объясните, как можно прожить,  ничего не сделав? Как? Даже если вы уйдёте, всё равно, это уже – поступок! Как вы можете не делать ни того и ни другого? Чаще, вы делаете зло – называете добром. И наоборот. Редко, когда вы всё называете своими именами. И никогда не было того, чтоб вы ничего не делали. Никогда не было и не будет. }
\people{Скажите, а этот контакт, как можно рассматривать, как зло или  как добро,  для контактирующих? }
\soul{Всё зависит от вас. Мы же говорили вам, что мы можем прийти и злом, можем прийти и добром. Ибо вы, как примете нас? Если вы примите нас за врагов, то и будете называть нас врагами, и, соответственно. Но если вы любого врага примете, как друга,  поднимет ли он на вас оружие?}
\people{Спасибо. }
\people{Скажите, а вот информация, которая поступаем к нам через сон, она поступает из другого мира или ещё откуда-то? }
\soul{И да, и нет. Чаще - работа ваша. Ибо многое вы знаете и много видите, но не можете обработать, ибо заняты другими делами. Когда спит ваш мозг, в вашем понятии, отключены все ваши чувства, отключено множество каналов, по которым вы получаете информацию и потому, вам легче работать, и вы это называете ``подсознанием''. Во сне вы можете творить чудеса - ясновидение и далее. Но это ваша работа, ваша заслуга. }
\people{Вот, у людей, когда появляются необыкновенные, необычные способности – это тоже работа, наверное, другого мира? }
\soul{Вы. Ваша заслуга! Поймите, к вам приходят подобные, подобные вам! Разве сможете вы разговаривать с высоким, если будете низки? Нет, ибо вы не услышите и не поймёте. И если вы получили что-то, в вашем понятии ``необыкновенное'', значит, вы заслужили того. Заслужили.}
\people{Скажите, порочный человек, не признающий своих пороков при жизни, всегда ли он признаёт  их после смерти?}
\soul{В вашем понятии - в промежуточном мире? Да, всегда. И мы вам уже отвечали это.}
\people{Я так понимаю, что это и есть ``чистилище''.  Да?}
\soul{В вашем понятии?  Пусть будет так. }
\people{А вот промежуточный мир, он длителен для человека? }
\soul{Нет. }
\people{Какими периодами он исчисляется? Годами? Неделями?}
\soul{Нет там вашего времени. Иное. Подумайте, если бы сохранялось ваше время..? Представьте, в прошлой жизни  вы жили многие тысячи лет назад. Не правда ли, что вам будет скучновато ждать эти тысячи там? }
\people{Да. Вопрос.}
\people{Скажите…}
\people{1-2-3-4-5…}
\soul{Спрашивайте.}
\people{Скажите, способности, которые получает экстрасенсы – это одарённость? Это тоже человек сам до этого доходит, или он получает энергию, как, вот, многие говорят - ``из космоса''? }
\soul{Вы сказали и то, и то.  Мы говорили вам, о заслугах ваших. То - заслуга. Далее. Вы все, будь вы одарённы или нет, вы связаны с космосом, ибо живёте в нём. И вы все получаете, в вашем понятии, одинаковые ``порции''. Ибо, экстрасенс и вы - живёте в одном мире. Только он может взять более, а вы менее. Он может управлять большими энергиями, вы управляете меньшими. И вся разница. Только в том разница. Кто дает умение вам? Вы! В вашем понятии ``прошлые жизни''. Или искушение, или дар. А если искушение, значит, в вашем понятии - ``чёрная сила''. Если дар – ``светлая сила''. Но бойтесь, бойтесь очернить её, ибо это так легко сделать!  Спрашивайте.}
\people{Это вы говорили, что божий дар, так сказать, очернить можно стяжательством денег сверх меры, скажем так.}
\soul{Не только. Что вы понимаете под ``искушением''? Что?}
\people{Ну…  может быть, власть над людьми, тщеславие, честолюбие. Это - тоже искушением является порочным? Да?}
\soul{Да! Но и чрезмерная доброта - тоже может. Иначе, мы были бы не правы, если бы говорили вам, что нет истинно доброго и истинно злого. Даже любовь к людям может быть добра, но может принести и зло. Вспомните Христа. }
\people{Ясно. Скажите, а может ли молитва воздействовать на порочного человека? }
\soul{Нет. А если быть точнее, зависит, верит ли человек в молитву вашу.}
\people{А вот искреннее прижизненное раскаяние  достаточно ли для исправления ошибок? }
\soul{Достаточно ли? Вы говорите, как покупаете.}
\people{Ну, например, церковь. Она, при исповеди, отпускает грехи. Мне кажется, где-то есть элемент здесь лицемерия. }
\soul{Простите, когда-то вы могли и купить. Вы верили в это? (об индульгенции.прим.)}
\people{Ну, наверное, как люди не верили в душе, а шли и, так, поступали ``на всякий случай'', может быть.}
\soul{Поймите, служащий церкви – человек. И он видит, что вы хотите услышать, и потому, чаще, говорит вам - вам приятное, но ненужное, думая, что этим делает вам добро и открывает вам ``ворота''. ``Любой ценой, но только верьте, верьте в Бога''. Чаще, так вы делаете. }
\people{1-2-3-4-5-6-7-8-9. 1-2-3-4-5-6-7-8-9. 1-2-3-4-5-6…}
\soul{Спрашивайте.}
\people{Хорошо. Скажите, когда совершается искупление? При новом воплощении человека или же на том свете? Допустим, в чистилище, в аду?}
\soul{Тогда вспомните, вспомните, искупление. Что же тогда сделал разбойник на кресте? Ведь, в вашем понятии, он не успел умереть. Он не успел попасть на тот свет. Почему вы спрашиваете о времени? Вы можете это сделать в любой момент, вы можете это сделать сейчас или позже. И нет в том большой разницы. Нет!  Ибо, вы сделали то. И когда-то вы сделаете, сделаете все. Но, может быть и поздно. Вспомните. Вспомните, о Конце Света, что говорили вам. И, хоть  неверно говорили, но есть в том истина, что может быть и поздно, когда вы увидите и скажете: ``Да, я был не прав!'' – но нет обратной дороги. }
\people{1-2-3-4-5-6-7.}
\soul{Спрашивайте.}
\people{Скажите, вы говорили об искуплении – это значит, надо осознать, где ошибался человек? Ну, относительно себя, в чём он был не прав. То есть, найти эти ошибки, тогда только и раскаяться в этом? }
\soul{Кто бы вам ни говорил, что такое ``искупление'', вы не поймете, пока не придёт время то,  пока вы  сможете сделать то. Если вам будут объяснять, что такое ``любовь'', но вы никогда не видели и не чувствовали того, поймёте ли? Вы только будете фантазировать и всё искажать. И, потому, будете отодвигать время, когда вы могли бы сделать это. И, потому, вы никогда не поймете; что такое ``верить'' или ``не верить'', что такое ``любовь'', что такое ``ненависть'', и многое, многое, пока не переживёте. И потому, нет смысла говорить вам. Придёт время – и вы поймёте. Придёт час, когда вы скажете…}
\people{1-2-3-4-5…}
\soul{Спрашивайте.}
\people{Скажите, вот, у нас новый человек на сеансе - Станислав. Он хочет задать вопрос. Какой?}
\people{Я хотел бы задать вопрос. Вот, нам интересно общаться, конечно же, с вами. А вот именно у вас, есть интерес общения с людьми? }
\soul{А мы никогда не покидали вас. Даже вначале мы говорили, что мы всегда были в вас. Вы только можете  слышать  или не слышать.  Чаще, вы не желаете слышать. Не желаете. Чаще, вы принимаете нас, то за совесть, то за интуицию, то ещё за что-то, за многое.  И потому - не верите, не верите нам, опускаете нас. Мы же, всегда говорим с вами. Всегда.}
\people{Ну, у каждого человека внутренний разговор, так сказать, происходит, да?}
\soul{Давайте скажем так…- одно из ваших ``я''. Вспомните, ваших должно быть четыре. В других может быть иное. Вы помните? }
\people{Да.}
\people{Скажите… Продолжим тему, о загробном мире. Какова продолжительность страдания виновного, подчиняется ли она каким-то законам на том свете?}
\soul{Вы уже спрашивали, и мы вам отвечали, о времени, что нет ``времени'' вашего.}
  
\people{Ну, наверное, это и вечно не могут быть страдания, как вот нас уверяют религиозные деятели, что вечное страдание. Такого, наверное, тоже нет? }
\soul{Мы же говорили вам, что вы Бога… Бога - очеловечили. И все ваши и дурные и хорошие качества перенесли на Бога. Конечно, испугать сложнее, если говорить, о вечности. Какой смысл мучить вас вечно? Каков? Далее, кто будет мучить вас? Кто? Вы сами. Сами вы!}
\people{Ну, скажите, существует ли некое ограниченное место, определенное для мук или для радостей, в зависимости от заслуг человека?}
\soul{Хм, интересно вы рассуждаете! А что тогда ваша Земля? Она ограничена. Вы согласны?  }
\people{Согласны, но мы здесь живём.}
\soul{Вы здесь живёте – вот именно! И на ``том свете'', в вашем понятии, тоже – живут! B мы говорили вам, о том!}
\people{Кому кажется реальней жизнь? Нам на Земле, или на ``том свете”- нам же? }
\soul{Кто-то  разговаривает  с вами и считает вас ``потусторонним'' миром, как и вы, разговаривая с ними. Кто реальней? Кто? Они или вы?}
\people{Ну, трудно сказать. Нам кажется, что мы - более реальны. }
\soul{А им кажется - наоборот. Они не верят в вас, а вы не верите в них. Хорошо, если соседний мир и мир,  в вашем понятии, что ``прошедший'', тогда конечно - они помнят ещё, что когда-то жили в этом мире и приходят к вам,  приходят, чтобы предостеречь от своих ошибок. Или наоборот. Мы говорили вам когда-то. Вы помните? Помните?}
\people{1-2…}
\soul{Спрашивайте.}
\people{Скажите, а в том, другом мире… там так же есть и природа прекрасная и небо, и реки, горы?}
\soul{А где вы хотите жить? }
\people{Ну,  нам на Земле нравится. Мир прекрасен земной. А на том свете неизвестно. Может там сплошной туман.}
\soul{Вы будете жить в сплошном тумане, в пустоте или где-то ещё? }
\people{Не хотелось бы.}
\soul{Везде жизнь, везде жизнь. Поймите. Представьте, вы вышли на контакт с рыбой. Она будет удивлена, если окажется, что вы живёте без воды, и как там можно жить. И она посчитает мир ваш адом, ибо когда-то попробовала и чуть не задохнулась. А вы? Как вы считаете? Можно ли жить в воде?  Ведь там можно утонуть! Спрашивайте.}
\people{Скажите, а есть ли духи, не ведающие раскаяния?}
\soul{Мы отвечали вам, не повторяйтесь.}
\people{Из чего следует, что рай и ад не существует в том виде, в каком их себе представляет человек?}
\soul{Было бы всё просто. Чтобы изменилось, если бы вы поменяли просто место жительства? Подумайте, много бы изменилось? И было бы для вас это наказанием или что-то ещё?}
\people{Ну, почему, некоторым людям, нашим знакомым, показывают ад именно со сковородками, чертями? Некоторые видели его в снах или, уверяют, бывали.}
\soul{Простите, уже с детства вам  говорят об ``аде со сковородками''. Иного не можете придумать, потому и снится вам. А когда снится вам не понятое, вы просыпаетесь и не можете понять, что же это было,  вы не называете, почему-то, это раем и адом. Ибо, в вашем понятии, ад – сковородки, рай – облака. А мы говорим вам, что, для многих, мир ваш может быть и раем и адом. Есть ниже вас, что считают вас раем и мечтают прийти, и, умирая, приходят к вам. Но придя сюда, они почему-то уже не говорят, что попали в рай и уже ищут новый. Чаще, они и являются ``одарёнными''. Они говорят вам: ``Мы видели ад, мы видели рай'', ибо что-то помнят, помнят. Но, как вам объяснить, об энергетике? Как можно было объяснить две тысячи лет назад об атомах, о иных мирах, иных временах? Согласитесь, проще было показать котел и масло, кипящее  в нём.}
\people{1-2-3-4-5-6-7-8-9. 1-2-3…}
\soul{Спрашивайте.}
\people{Скажите, у переводчика часто бывают такие остановки. Что мешает переводчику?}
\soul{Мы не можем заставлять его силой, не можем.  Это было ваше желание. }
\people{У него есть элемент противления, да?}
\soul{Оно есть у каждого из вас, у каждого. }
\people{Ясно. А скажите, вот переводчик, вначале, сказал, что у кого-то болит сердце, это ему мешает, и лёгкие. У кого из присутствующих? Можно назвать? }
\soul{Они не могут сказать сами?}
\people{Ну, если про лёгкие, то может быть я, ну не знаю, а сердце…}
\soul{Если бы мы говорили об вас, мы бы это сказали и в прошлых контактах, в вашем понятии. }
\people{Это рядом со мной справа сидящий человек, да?}
\soul{Почему он не скажет сам? Он не знает?}
\people{Он не уверен. }
\soul{Вы не знаете даже себя?}
\people{(Сидящий слева) Так, ну, у меня, допустим, сердце. Про лёгкие я не знаю. 1-2-3-4…}
   
\soul{Спрашивайте.}
\people{В Библии есть такое понятие ``страждущая душа''. Вы можете нам объяснить, что это такое?}
\soul{Что вы понимаете под понятием ``стражд''?}
\people{Страждущая ….Ну, ищущая, не находящая покоя, желающая что-то.}
\soul{А вот теперь подумайте, если, будучи в другом мире, вы искали рай, пришли в этот мир и сказали: ``Нет, это не рай'' и хотите далее и далее. И будете искать, по вашим понятиям, так, как хотите вы, и вряд ли найдёте, ибо вы будете искать мозгом. Далее, есть души в вашем понятии ``наказанные'', которые не верили в существование иного. И что же? Они заблудились. Ибо каждому - по вере его. А представьте, если вы не верите ни во что, куда вы попадёте? ``Ни во что'' и попадёте. И тогда, этот мир будет отражаться ему тенью, и в этом мире он будет тенью. Вы это называете приведениями, полтергейстами и многими, многими. А это - одна из душ, что не верила ни во что и не может попасть ни в один из миров. И потому, ищет, ищет и создаёт везде тени. }
\people{То есть, вот…}
\people{1-2-3…}
\people{Скажите, а вот у меня есть многие знакомые, ряд знакомых, которые… ну, не верят совершенно,  в разные миры, в многомерность, в НЛО и прочее. Их что… после смерти их ждёт довольно печальная участь? То есть, они попадут в совсем не интересный мир, полтергейстный какой-то?}
\soul{Нет. Неужели вы думаете, что все, кто не верят, попадут в те миры? Нет.  Неверие – тоже вера. Далее. Нет среди вас тех, кого можно было бы назвать атеистами. Назовите мне хотя бы того, кто ни разу не сказал: ``Боже, за что?'' или что-то в этом роде. Назовите мне того, кто в беде не призывал Бога. А вы говорите - ``Не верю''. Очень редко, когда человек не верит, истинно не верит. Всегда есть сомнения. Да. }
\people{Сейчас - вопрос.}
\people{Скажите, душа после смерти человека ограничена или нет, в выборе другого мира?}
\soul{Нет, она может прийти в любой мир. Но, так, как совесть её чиста и не обманывается, она выберет мир, в котором может и должна жить. В лучший она не уйдёт, ибо будет обман. В этом же ,``промежуточном мире'', в вашем понятии, только в вашем, - она не может обмануть. И только этим она ограничена, только этим. }
\people{Скажите, вы сказали, что полтергейст – это тень. В этой тени, однако, на наш мир, так сказать, мир не теней…  воспроизводит своей силой довольно-таки разрушительные действия. }
\soul{Ну и подумайте, что вы говорите! А в вашем мире нет теней, что могут принести разрушительную силу? Тогда представьте, если солнце ваше затмить, и надолго. И вы будете удивлены, что Тень смогла уничтожить Землю вашу! Далее, почему, в вашем понятии, материя и энергия совместимы?  Вы что хотите сказать этим вопросом, что если нет  материи, значит, нет энергии? Тень – нет материи. Вы согласны? Или нет? Подумайте.}
\people{Тень – это…}
   
\soul{Хорошо, подумайте и скажите, тень имеет материю или нет?}
\people{Да. Так, как она на материи отражается. Значит, она…}
\soul{Нет. }
\people{Материя – носитель тени, получается. }
\soul{Вы ответили правильно, но неправильно сказали почему. }
\people{Возможно.}
\people{Можете объяснить?}
\soul{Нет. Это должны знать вы, и должны знать  уже со школы. Вы забыли то, что учили. И придёт время, когда вы забудете то, что говорите сейчас и что слышали. А вы слышали гораздо менее, чем…}
\people{1-2-3-4-5-5-6-7-8-9…}
\soul{Спрашивайте.}
\people{Скажите, переводчику мешает свет? Он открывал глаза.}
\soul{Нет.}
\people{Это другие эмоции мешают, да?}
\soul{Спрашивайте далее.}
\people{Хорошо. В каком смысле, вот интересный вопрос,  нужно понимать слова Христа - ``Моё царство – не в этом мире''?}
\soul{Понимайте впрямую.}
\people{Значит, туда, куда он ушёл… Ну, как сказать…}
\soul{В вашем понятии, ``небеса''. Мир, что выше вас.}
\people{Да, но раньше небом, вы сказали, называлась наша душа. Значит…}
\soul{А вы подумайте! Вы подумайте и вспомните, что сказал Христос?  Чтобы вы открыли душу, ведь душа ваша – дом! Почему тогда нельзя сказать ``Небеса - его дом'', если душа ваша называлась ``небесами''?}
\people{Вот именно в этом слове… Надо слово ``небо'' читать в Библии, в этом смысле?}
\soul{Подумайте, сказано было вам: ``В каждом, из вас,  Бог''! Не в том мире, в котором вы живёте - в каждом из вас. В каждом из вас! Вот его мир – душа ваша! }
\people{1…}
\soul{Спрашивайте. }
\people{Может ли на Земле, когда-нибудь наступить царство ``добра''?}
\soul{Когда-нибудь? Нет, в вашем понятии, `` истинного добра'' и ``истинного зла''  не будет. Будет время, когда будет добро, когда будет зло, более. Более, чем сейчас.}
\people{Сейчас, что преобладает на Земле, в частности в России, зло  или добро?}
\soul{Каждому своё! Вы подумайте, - для кого-то - больше добра, для кого-то - больше зла. Относительно кого вы будете мерить? }
\people{Ну, вы как-то говорили, что аура над Землей в районе России…}
\soul{В районе России или Земли? Неужели вы думаете, что только вас? Вас? Вы избраны?}
\people{Вы как-то сказали, что Россия будет духовным центром. Это ваши понятия.}
\soul{А это есть ``избранность''? Почему вы понимаете ``избранна'' в ином? Почему ``избран'' – это значит,  какие-то привилегии? Почему?}
\people{Нет, ну, в России, скорее всего, судьба избрана для бед разных. Нас преследуют беды издавна. И те же семьдесят лет коммунистического строя, тоже, скорее всего, была испытанием  серьёзнейшим, для народа. }
\soul{Прекрасно! Теперь ``кто-то'' виноват, но не вы! Не народ виноват – ``кто-то'' его испытывает, ``кто-то'' его наказывает! Да..  вы научились искать виновных.}
\people{Ну, в то же время,  для кого-то, это было райское время, так сказать, ``райские'' семьдесят лет? Они ни в чём не знали нужды, допустим, некоторый контингент людей.}
\soul{Ну, и как вы назовёте? Как вы ответите на собственный вопрос? Относительно кого вы будете мерить?}
\people{Ну, вот, они устроили себе ``рай''. Вот,  у тех, у кого хватило ума, скажем так, или хитрости,  я не знаю, что назвать, вот… ну, а остальные потихонечку приспосабливались, как могли. А кто-то не приспособился. }
\soul{Что в вашем понятии ``рай''? }
\people{Рай – это спокойствие души.}
\soul{Спокойствие души? А вы знаете тех, что устроили,- были ли спокойны их души? }
\people{Они всё равно жили беспокойно в борьбе за власть и прочее? Ну, хорошо, мы не будем касаться этих вопросов. Мы продолжим, о душах и душе. Чувствительные ли духи к воспоминанием тех, кого они любили на Земле? }
\soul{Да.}
\people{Им помогают эти воспоминания, оставшихся на Земле? }
\soul{Бывает, что и мешают.}
\people{Слишком большие страдания мешают, да? }
\soul{Да. Ибо она не может уйти. Она боится покинуть вас. Боится. Вы страдаете и не отпускаете её. Вы всё время представляете: ``Вот, если бы был бы, живой, то было бы так или так''. А мысль - материальна!  И ему приходится оставаться здесь. Он не может стать живым. Не может, ибо, вы не можете материализовать до той степени.  Но, вы уже мешаете. Потому и сказано вам: ``Оставьте умерших - хоронить умершим''. }
\people{Это эгоизм, чаще всего, - слишком переживать?}
\soul{А вы подумайте,- умер близкий вам человек – вы переживаете за него или за себя? Может быть, вы плачете за себя - ``Какого человека мы потеряли!'' Подумайте.}
\people{1-2-3-4-5-6-7…}
\soul{Спрашивайте.}
\people{Скажите, а вот интересный вопрос,-  присутствуют ли духи на церемониях открытия им памятников, издания книг, исполнения песен, если они умерли ? Это им нравится?}
\soul{Да, если они не ушли, то они придут. Далее. Вас - множество. Умирая,  вы – минимум, семеро. В вашем понятии, это и астральные поля и сама душа. И подумайте,- основа  – то единое, без тел -  уходит дальше, уходит в другой мир, чтоб получить иные тела. Все остальные остаются жить здесь, и у них разное время. Одно из тел живёт всего лишь семь дней, другое - живёт год, четвертое - живёт до тех пор, пока хотя бы один помнит о нём. Пока хотя бы одна книга есть, о нём. В вашем понятии - жить вечно. Даже вы, материалисты, знали это и говорили: ``Вечно живой - в памяти''.  Вспомните.}
\people{Это справедливо?}
\soul{Да. Вот это тело всегда придёт к вам, если вы вспомните его. И, тем более, если вы создадите его изображение, или даже сумеете дать форму.}
\people{Интересно. А вот чем объяснить желания некоторых людей быть погребенными в том или ином месте после смерти?}
\soul{То - ваше желание.}
\people{Оно играет какую-то роль большую для человека?}
\soul{Да, оно играет. Но играет - для погребенного. Ибо умирать он будет более спокойно, чем ему скажут ``Нет, ты будешь в ином!'' Потому и всегда старались исполнить последние желания, чтобы человек умер спокойно, чтобы не создавал отрицательные энергии, иначе  – подобное к подобному, придут и скажут – ибо твой мир отрицательный. Потому и стараются выполнить последние ваши желания, потому…}
  (Часть вырезана)
\soul{…только во снах, вспомните себя, вспомните, когда вы, вдруг, чувствуете себя кем-то. Мгновения – но всё-таки, вы были! Вы можете вспомнить это, но вы не придаёте тому значения. Вы называете это ``фантазией'' или что-то ещё. }
\people{Но, реальные те случаи, когда, вот, мы сталкивались, наша группа,-  когда в человеке живёт чья-то чужая душа и не желает уходить. Она начинает командовать, слова какие-то, команды дает, ругается. Это действительно - реальные случаи? Это не фантазии самого человека?}
\soul{Мы вам ответили. Конечно. Конечно, это возможно. И возможно только в том случае, если ``гость'' ваш сильнее. Вы же, в теле своём, чувствуете себя гостем, не чувствуете хозяином. Вы не знаете ваше тело, не знаете, что  в нем болит, пока вам не скажут, или пока оно само не закричит вам. Вы не знаете вашу душу. Вы даже не знаете свой характер и ищете его в гороскопах. Каков же вы ``хозяин''? Каков? И когда к вам приходит ``гость''  – в вашем понятии ``другая душа'', - и если она сильнее вас, вы принимаете его за хозяина и даете ему свой дом? и подчиняетесь ему, ``хозяину''. И, тогда, вы уже - кто?}
\people{Раб. А скажите, можно изгнать такие души, чужие?}
\soul{Будьте сильнее! Почувствуйте себя хозяином! Вы - хозяин.  Вы, и никто иной! Вы же предпочитаете быть слугой. Вы говорите ``встречали''. К вам приходят и спрашивают: ``Что делать мне?''. Вы даёте совет, а выполняют ли советы ваши? Если скажут вам: ``Вы больны'',  представьте, что вы здоровы! Вы же знаете, что мысль материальна. Вы  знаете это, но почему же тогда вы не стали здоровы? Вы говорите, что ``я верю, я верю''. А верите ли вы в это или просто знаете? Если бы вы веровали, и не было сомнений ваших, вы б сказали: ``Да я здоров! Да я хозяин!'' - и вы были бы хозяином. А вы, только знаете и не более. }
\people{Простите, но я встречал человека, по складу своему руководителя, он руководит многими людьми, сильная внешне личность, но в нём, вот, такая вот душа - командует, насмехается…}
\soul{А здесь, всё гораздо проще. Нет в нём другой души. Нет. Он, просто, устал быть командующим. Ему тоже хочется, чтобы им командовали. И все из вас хотят быть и хозяевами, и сильными, и слабыми. Нет в нём другой души. Нет. То - его мечта. Если хотите – подсознание, где он хочет быть слабым. Ибо, слабым быть - легче. Меньше ответственности.}
\people{1-2-3-4-5-6-7-8-9 [Прямой счет три раза]  9-8-7-6-5-4-3-2-1. [Обратный счет два раза]}
  (Белимов) Геннадий, Геннадий, расскажи, что ты помнишь, что ты ощущал из сеанса?
  (Гена) Отрывки сеанса самого помню.
  (Белимов) Что? Отрывки?
  (Гена) Ну, да.
  (Белимов) У тебя какие-то картинки были или ты диалоговые ситуации запоминал? О чём мы говорили? О чём шла сейчас речь?
  (Гена) О полтергейсте чё-то там было…
  (Белимов) Так. О потустороннем мире было?
  (Гена) Потом, о каких-то путешественниках…
  (Белимов) Путешественники? 
  (Гена) Да. Что-то ещё…
  (Белимов)  У тебя были картинки в голове или же…
  (Гена) Да нет…
  (Белимов) Как ты думаешь, мы много работали, сеанс долго длился на твой взгляд?
  (Гера) Ну, это я не знаю.  
  (Белимов) У тебя ощущение какое сейчас? Усталости? Или ``ничё''? Хорошее? Как-будто ты выспался? 
  (Гера) Или, как обычно?
  (Гена) Как обычно.
  (Белимов) То есть, ты много потерял энергии?
  (Гена) Да нет, я вроде не устал ничё.
  (Белимов) Что ещё ты можешь сказать? Необычная обстановка. Ты как-то не совсем… вёл, часто останавливался. Никак не подействовало на тебя?
  (Гена) Ну, когда лёг, конечно, это самое…чё-то…вроде как…
  (Белимов)  Свет мешал?
  (Гена) Нет, свет не мешал. Просто как-то, это самое…телевизоры тут всякие! Ну, неудобно просто, как-то вот так вот.
  (Белимов) Ну, ты же их не видел?
  (Гена) Нет, я видел тогда, когда только лёг. 
  (Белимов) Ты рассматривал руки – это что? Ты опять видел насквозь? Косточки видел, как однажды рассказывал?
  (Гена) Нет, это я не помню. 
  (Белимов) Не помнишь, да? Ну, всё ребята, сеанс закончен.
Аоум. глава 27-11-94 г
Георгий Губин
\people{**}
\people{Воскресенье. Сегодня состоялся показ по программе всероссийского телевидения одного из наших контактов. Скажите, пожалуйста, вам это известно, что сегодня состоялся такой показ по телевидению?}
\soul{Мы говорили вам, что всё, что известно вам… }
(срыв)
\people{Мы поняли,  что вам это известно. Скажите, как вы оцениваете состоявшийся показ?}
\soul{Должны оценивать вы, не мы. Мы говорили вам, что нас не интересует это.}
\people{Но мы ещё не знаем, как это откликнется на наших контактах, поможет ли это нашим контактам.}
\soul{Всё зависит от вас. Если вы боитесь – будет вам страх. Если нет - что может случиться с вами?}
\people{Нет, мы не боимся. Мы ждём помощи от наших коллег - может быть  повысим уровень наших вопросов и получим более интересные ответы даже.}
\soul{Да. И борьбу тоже.}
\people{А-а. То есть, вы ожидаете? И какие-то препятствия в этом появятся? Да?}
\soul{Вы не видите их?}
\people{Ну, в принципе…}
\soul{Вся ваша жизнь прошла гладко, в вашем понятии? }
\people{Нет, конечно. }
\soul{И вы будете бороться с коллегами, и будете бороться с собой. Спрашивайте далее. }
\people{Хорошо. Вообще-то, нас очень интересует… Мы, так  и не добились понимания, кто же должен, среди нас, быть четвертым. Мы всё ожидали, что мы сумеем это вычислить, и отсюда, получить более высокий уровень вопросов. Пока это не получилось. Вы тоже видите, что мы бессильны? }
\soul{Нет, вы не бессильны. Но бессилен ваш разум. Мы говорили вам, что вы хотите найти сознанием. И говорили вам, вначале, не найдете им. }
\people{А мы, всё-таки, пытаемся разумом. А надо чем? Душой? Сердцем?}
\soul{Ищите.}
\people{Хорошо.}
\soul{И далее. Разве так важно вам, найти четвертого? А почему вы не можете найти в себе? Почему вы всегда ищете извне? Почему вы всегда задаёте вопросы и ищете готовые ответы? Почему вы не можете найти в себе? }
\people{Ну, как-то неожиданно - в себе находить ещё одного человека. }
\soul{Одного ли? Если вас – множество. И если вы всмотритесь, вы увидите. Вы увидите и ужаснетесь. Где же ваше истинное я, если каждое мгновение вы - другой, в любом положении вы меняетесь? Вы согласны…  }
\people{Да, с этим можно согласиться.}
\soul{…что вы разные; дома, на работе и далее, далее? А где же ваше истинное? Истинное ``я''? }
\people{Да, это трудно обозначить и определить. Но поиск четвёртого человека, ну, не подводил нас логикой к тому, что четвёртый может в ком-то из нас заключаться. }
\soul{Логикой. Сознанием. И потому, мы говорим –  не найдете. }
\people{Ну, хорошо, это остается больной для нас проблемой. Боюсь, что мы так и не узнаем. Какое-то осенение должно быть, чтобы мы это получили. }
\soul{Давайте спросим вас, что вы подразумеваете, говоря ``осенение'', ``озарение'', ``вдохновение'', ``талант'' и далее? Что вы называете этими словами? Сознание ваше?}
\people{Прорыв, прорыв сознания. }
\soul{Сознания ли? }
\people{А чего ещё? Мы…}
\soul{А вы вспомните, когда бывает у вас ясная голова, и вы в прекрасном сознании  - есть ли в эти мгновения порывы? Есть ли в эти мгновения вдохновение? Есть ли?}
\people{Да,  пожалуй, нет. Не всегда. }
\soul{Почему же, то, что приходит извне, и вы это называете вдохновением, дуновением и чем угодно, не вспоминая Бога? Ну ладно, это ваши проблемы. Почему вы всё списываете на сознание? Сознание, не против - получить. Оно не отказывается. Оно примет любой подарок.}
\people{Ага…Но, мы это ещё не понимаем. Не знаем, почему всё это происходит. Мы хотим вот что…Интерес к нашим контактам есть, уже появляется у других групп людей и так далее, ибо выяснилось, что мы с вами почти год общаемся, но мало, всё-таки, знаем о вас. И, вот, цикл вопросов таких, просим. Первое, - имеет ли ваша цивилизация своё собственное название, и можете ли вы его сообщить нам? }
\soul{Хорошо, давайте представим: вы приезжаете  в чужую страну и не знаете языка, и вам назовут название, естественно, на своем языке. Вы поймёте его?  }
\people{Нет, но, бывает, мы так страну и обозначаем -  по их лингвистическим…}
\soul{Хорошо, тогда вспомните, если мы не имеем вашей физики, как мы можем произнести слова? Ваши слова.  И как мы можем дать перевод? }
\people{Но, вообще…}
\people{В данный момент, вы же пользуетесь переводчиком? }
\soul{Пользуемся переводчиком. Но переводчик не был у нас, в отличие от нас. И когда он придёт к нам,  увидит нас, может быть, тогда он подберёт. Это будет грубо, но сможет вам передать. Вы же, привыкли смотреть название. Одно название даёт вам уже - направление. Читая книгу, вы читаете название и прежде книгу. Не нравится название - и отброшена книга. Вы согласны?  }
\people{Да, так случается.}
\soul{В одном слове, вы заключаете весь смысл книги. Как мы можем дать вам ``наше'', если вы не поймёте, и будет искаженно? И тогда все разговоры наши будут искажёны только из-за одного названия, ибо, ваша - психология. Ваша. Наше - иное. Мы приходим к переводчику и говорим его словами. Его. Ляжете вы – мы будем говорить вашими словами. Мы будем пользоваться вашим запасом. Вашим. И, хотя переводчик имеет гораздо больше, больше слов, мы будем пользоваться только тем, что легче добыть, ибо легче будет говорить переводчику. Потому и не пользуемся терминами. Переводчик их знает. Знает их прекрасно. Но термины те – названия. Вы поняли?}
\people{Ну, не очень, но…}
\soul{Названия. И если вы хотите услышать от нас слово, допустим, ``экстрасенс'' – это уже будет настраивать переводчика. Мы говорили вам, что не может владеть, и не имеем права владеть им полностью. Если же мы возьмём то слово, то, значит, мы должны войти глубже. Там, где он спрятал его. Там, где окутал его множеством понятий. Мы же берём те слова, что пользуется часто  и, потому, не нагружены лишними смыслами. Вы поняли? }
\people{Да, поняли. И причём, была мысль, что прежде, чем с вами разговаривать на серьезные такие темы, стоит оговорить названия, термины, в которых мы будем говорить. Это справедливая мысль? Мы-то, раньше, с вами не договаривались о терминологии.}
\soul{Поймите, вы имеете, - каждый из вас, - свой словарный запас, более или менее. Но, многие слова, для вас, являются ``ключом''. Мы приводили вам пример - ``экстрасенс''. Под ним вы подразумеваете множество, ибо вы не часто пользуетесь словами теми, и потому, является ``ключом''. ``Ключом''. Но мы не хотим войти силой, и потому, мы будем вынуждены говорить теми словами, которые употребляете часто. Часто, и потому, вы не придаёте большого значения. }
\people{Угу. Ну, хорошо. Скажите, а какое место вы занимаете среди цивилизаций Земли, если занимаете? }
\soul{Нет. Нет места.  Нет места для нас, как и  вам, для нас. Мы не занимаем. Мы не можем занимать. Если бы мы занимали какое-то место, мы бы тогда говорили бы о физике. Мы же - физики не имеем. И если рассматривать, вы не можете увидеть нас ни чем. Ни любым прибором, ничем вы не можете увидеть нас. Вы не можете увидеть нас никак. Если экстрасенс приходит и говорит вам: ``Я вижу нити, потоки''  и более – не верьте ему. Он не может видеть того. Он может видеть только реакцию. Реакцию, которую производит мозг на нас, но не самих нас.}
\people{Ну, это интересное ваше замечание. Скажите, а вы представляете полевую, от слова поле, или плазменную форму разумной жизни?}
\soul{Мы говорили вам, что все поля ваши  материальны. Физика. Мы ответили вам?}
\people{Ну, понял. Так вы, значит, вообще совершенно необычный вид энергии?}
\soul{В вашем понятии – нет, нет названий ваших, и потому,  мы не можем сказать, как называется ``мы'', какие мы носим имена и носим ли мы их. Вы привыкли к именам. Ибо имя ваше – название, и по имени вы судите. Мы же, не носим их. Не носим, по той причине, что мы не имеем физики. И, спрашивая о полях, вы невнимательны. Мы не можем иметь полей, в вашем понятии. Но, у нас есть своя физика. Своя. Не ваша. И мы, тоже имеем множество тел и тоже хотим, и тоже идём той же дорогой. То же, что и вы. Мы, тоже ищем. Мы не ``высшее'', но и не ``низшие'', только потому, что мы не меряем теми мерами, что меряете вы. }
\people{Вы сказали, о телах. То есть, вы имеете какую-то свободную форму воплощений. И насколько вы в этом безграничны? }
\soul{В вашем понятии, мы не имеем тел,  а можно сказать и наоборот –  мы имеем множество.  Потому, что мы можем присутствовать везде  и, в вашем понятии, одновременно. Потому что, если нет вашей физики, значит, нет вашего времени. Значит, нет ваших ограничений. Значит, мы можем везде существовать или нигде. Всё зависит от того, с какой точки вы будете смотреть. Если вы будете наблюдать нас приборами и не увидите нас, вы скажите, что нас не существует.}
\people{Да.}
\soul{А с другой стороны, вспомните, много ли вы видели, хотя бы недавно, не имея сегодняшних приборов? И вы отрицали то.}
\people{Вы справедливы.}
\soul{И если вы скажите, что ``мы создадим когда-то приборы, чтобы видеть'' нас – ошибетесь. У вас есть множество приборов, чтобы увидеть реакцию. Реакцию воздействия на нас - вас самих. Ваших материалов, веществ, природы, ваших тел, вашего мозга. Только реакции, но не нас.}
\people{А скажите, действительно ли, что шар,  это наиболее частая и энергетически предпочтительная форма существования энергетических форм жизни?}
\soul{У вас – да.}
\people{Угу. А у вас не обязательно?}
\soul{А вы подумайте. В вашем понятии, - Бог совершенен. Шар?}
\people{Ну, мы не знаем, как Бог выглядит.}
\soul{Подумайте, рассудите логически, примените ваше сознание. Если вы говорите ``Шар совершенен''. Бог – совершенен, значит, Бог – шар? А как же тогда ``подобие''? Поймите, в одной среде хорошо быть шаром, в другой среде - можно быть и другим. Всё зависит от того, где вы живёте. Поймите, вы – человек, неужели вам было бы удобней быть сейчас шаром? Подумайте.}
\people{Да нет, нам это неудобно.}
\soul{Вы совершенны, - даже в вашем понятии, вы совершенны, - для данной среды. Для иной – нет. И потому, вы меняете тела, меняя миры. Здесь  вы - человек. Где-то, вы будете шаром, а где-то, камнем. Где-то, будете бесформенны, а где-то, и будете не иметь границ. И, где-то, вы не будете иметь, в вашем понятии, тела. Ибо будете,-  в вашем, только в вашем понятии,- воздухом, который занимает любое пространство. Всё зависит от того, в каком мире вы живёте и для какой цели.  Поймите, если душа главенствует телом, почему тогда тело должно иметь строгую форму, если душа может менять обитель? Спрашивайте далее.}
\people{Скажите, а есть ли и каковы средние размеры индивидуума вашей цивилизации? Диаметр, допустим.}
\soul{Чем мы измерим вам диаметр? Чем?}
\people{А-а… То есть, у вас нет понятий наших метрических?}
\soul{Мы только что отвечали вам, что мы можем сказать и назвать вам любой размер, и мы будем правы. Любые. Мы можем сказать, что мы имеем микроны. Мы можем сказать, что мы имеем тысячи, тысячи метров, километров, как угодно. И мы будем правы и там, и там. Ибо вы меряете своими мерами. А ваших мер нет у нас. У нас нет физики. Есть только реакции. Реакции ваши. И если что-то в вас изменится, от нас - только на микроны,- можете сказать, что это наш размер. Если же вы измените многое – тот размер. Всё зависит от ваших реакций. От ваших. Измеряйте вашими. Измеряйте себя, тогда вы поймёте, каких мы можем быть размеров. }
\people{Скажите, а вот более легкий вопрос, - а чем вы питаетесь, энергетически? }
\soul{Чем?}
\people{Да.}
\soul{Но только не физикой вашей. Когда-то, мы говорили вам, вы поняли, что мы питаемся эмоциями.}
\people{Да, эмоциями.}
\soul{Мы не стали отрицать, ибо было легче с вами разговаривать. Но, подумайте, если мы не имеем вашей физики, как мы можем питаться вашей физикой? А мы говорили вам, эмоции ваши  – тоже физика. Вы согласны? }
\people{Угу. То есть, мы не знаем…}
\soul{Если сказать грубо, очень грубо, то – вашими реакциями. Вашими реакциями. И потому, говорим вам: ``Будьте осторожны'', если вы примите нас злом, поймите, то мы сделаем зло. Ибо вы сделали это. Не мы - вы отреагировали злом. Если вы примете нас добром – вы отреагируете добром. Вы согласны? Это ваши реакции.}
\people{Да, да, да.   }
\soul{Мы, всего лишь, только возбуждаем вас. Всё остальное зависит от вас, что в ``сосудах'' ваших. }
\people{Допустим, такой вопрос,- если бы не было контактов между вами и нами, то,  значит, у вас бы не было пищи? Можно сделать и такой вывод.}
\soul{Нет. }
\people{Но, если вы питаетесь нашими эмоциями…}
\soul{Мы говорили вам, что это грубая аналогия. Грубая. И наше питание, если хотите, в вашем понятии, это - знание. Знание, о вас. }
\people{Так, значит, мы вам, всё-таки, необходимы? }
\soul{Нет.}
\people{А с какой целью вы тогда идёте на контакт?}
\soul{Целью? Может быть, вы приходите, - не мы? Вы зовёте, - не мы. Мы были и будем всегда с вами. Мы никогда вас не покидали. Ваше же сознание, - а вы ведете контакты только сознанием, ибо, если вы будете задавать вопросы духовно,  мы  ответим вам духовно. Мы говорили вам, что придёт время контактам, когда не будет переводчика, когда не будет этих движений, когда - только ``мысль''. Вы поняли? }
\people{Да. Мы хотели бы дождаться этих времён и выйти на подсознание…}
\soul{Вы же, - сознание ваше, - питается нами, если можно б было сказать точнее. Вы жаждете ответов, значит, вы питаетесь. Не мы. Мы приносим вам пищу. И вы просите нас  - не мы. }
\people{Скажите, есть цивилизации…}
1-2-3-4…
\soul{Спрашивайте.}
\people{Есть цивилизации, которые питаются якобы разницей температур, их земляне называют термофагами. Вы не относите себя к такому виду цивилизаций?}
\soul{Вы повторяетесь.}
\people{То есть, не относите?}
\soul{Ну, вы подумайте! Вы говорите, опять, о физике.}
\people{Ну, мы  другого просто не знаем.}
\soul{Спрашивайте.  }
\people{Хорошо. Мы выходим на контакт физически, но вы же - отрицаете физику. Как вы можете…}
\soul{А вы забыли о реакции? Где ваша логика? Вы забыли о реакции. Мы приходим – и вы реагируете. Вспомните. }
\people{Ну,  мы, прежде чем приходим, мы же - вызываем. Допустим, переводчик испытывает какие-то физические свойства…}
\soul{Реакции. }
\people{…прежде, чем прийти на контакт. }
\soul{Вы забываете, о реакциях. Когда вы попадаете под гипноз, вы испытываете физическое? На вас воздействуют физически? Подумайте. Когда вы слушаете музыку, на вас воздействуют физически? }
\people{Да.}
\soul{Разве?}
\people{Да, нет. Эмоционально, скорее всего…}
\soul{А где ваша душа? Где ваши эмоции? Физика? Да, это физика. Но более тонкая, более тонкая физика. И потому, нельзя сравнивать. Вы же говорите о ``тяжелых'', и о ``тонких'', одинаково. }
\people{Хорошо. Разговор о музыке. Музыка же, всё равно - является физикой. Происходит колебания воздуха, мы их воспринимаем. Это, тот же  физический процесс.}
\soul{Тогда почему же вам не послушать гул паровоза? Те же колебания!  Почему же вам они не нравятся? }
\people{Так, здесь, два взаимодействия - физическое и эмоциональное. }
\soul{Эмоциональное?}
\people{Совокупность. }
\soul{А что рождает ваши эмоции? Что рождает ваши эмоции? Физика? Или, может быть порывы души? Что рождает ваши эмоции? Они рождаются сами? Колебаниями воздуха? Совокупностью физики? Тогда, почему же одну и ту же музыку могут слушать с разными? Почему тела ваши разные, а музыка одна, и воспринимаете одинаково?}
\people{В каком регионе Земли  преимущественно вы обитаете? }
\soul{Мы отвечали вам.}
\people{Вы - повсюду, да?  В солнечной системе, так?}
\soul{Нет. Нет, мы не можем занимать только солнечную систему. Мы есть везде. И мы говорили вам, и говорили не раз, - нет мест, где нас нет.}
\people{Ну, а на высотах? Тут тоже вам нельзя говорить, если вы везде, получается?}
\soul{Мы находимся везде, но не везде мы можем ощущать себя. И вы находитесь во множестве миров, но не можете ощущать себя во всех их. Подумайте и рассудите логически. Вы живёте везде…}
\people{Раз.}
\people{1…}
\soul{Вы живёте в мире физики, и говорите, о тонки телах, хотя,  имеете грубые. Значит, тонкие тела живут среди вас. Значит, вы живёте в тех мирах, но не ощущаете того. Вы носите в себе даже те миры, но не ощущаете того. Так же и мы - существуем везде. Мы ничем не отличаемся от вас, но не можем ощутить себя…}
\people{1-2-3-4-5-6-7-8-9. 1-2-3-4-5-6-7-8-9.}
\soul{Как вы слабы.}
\people{…6.}
\soul{Как вы слабы!}
\people{В каком смысле?}
\soul{Любое, что мешает вам, воспринимаете с агрессивностью. Вот и переводчик ваш }
агрессивен. 
\people{То есть, он слышит наши разговоры и реагирует на них?}
\soul{Нет. Агрессивность его не связана сейчас с вами. Спрашивайте далее. }
\people{Он сопротивляется вам?}
\soul{Нет. Вы говорили ``эмоции''. Вы порождаете эмоции, и в вас порождаются эмоции не физикой, а более тонким, более тонким, поймите вы это. И это более тонкое ранимее. И если вы, ещё, сознанием… Это хуже. Это гораздо хуже. Когда сознание пытается ответить за душу. Когда сознание берёт роль управителя тела, несмотря на душу. }
\people{Скажите, каким способом, образом, вы общаетесь между собой? }
\soul{У нас нет способов общения, в вашем понятии, никаких. Мы же говорили вам, что мы едины. Как мы можем общаться с самим собой, если мы едины? }
\people{То есть, у вас нет, разве, индивидуальностей? Мы так поняли, что у вас есть индивидуумы.}
\soul{Есть. Но вам трудно понять - быть индивидуальностью, и быть единым. Если говорить вашим языком, чтобы было понятнее вам, - пусть это будет ``мысль'', столь быстрая, что не имеет измерений. }
\people{Угу. Хорошо. Тут есть группа интересных вопросов, жизненно важных. Попробуем их осветить. Скажите, существуют ли у вас родственные связи, дружба, любовь? }
\soul{Да.}
\people{Скажите, тогда, что главное в вашей жизни и в деятельности? }
\soul{Почему вы не спросите, если мы едины, какая может быть любовь и дружба? Почему вы не спросите того? }
\people{Ну, так давайте, считайте, что мы так спросили.}
\soul{Любовь и дружба, значит, у вас не существуют?}
\people{Существуют. Но вы называете ``любовью'' что?}
\soul{Ну, это, скорее всего,  взаимодействие полов. }
\people{Почему же? Не только. Вы можете любить и свой пол. }
\soul{Да, правильно. }
\people{То есть, более высокая грань человеческих чувств? }
\soul{Не будем говорить о телесной любви – это грубое, и вы не имели это. Есть любовь, и вы не можете найти определение её. Вы не можете понять, что тянет вас, что зовёт вас, почему и что вы хотите. Вы называете это ``любовью''. Чем же мы отличаемся от вас, если мы имеем то же? Да, мы – едины. Но мы имеем индивидуальности. Мы говорили вам это только что. И мы - любим. Но, в отличие от вас, мы любим себя. В том понятии, что мы - едины. И, в то же время, мы любим другие индивидуумы. Это трудно понять, трудно, потому, что вы хоть и едины тоже, но вы слишком индивидуальны. Вы не можете понять своё единство. Вы не можете понять, что ``человек'' – это ничто. ``Человечество'' – да. И потому, вы, чаще,  любите себя, или кого-то другого, но не любите всё человечество. И когда вы говорите: ``Я люблю всех людей'',  вы лжёте, потому что вам не хватает сил любить всех. Мы же, говорили вам, не имеем противников, мы не имеем врагов.}
\people{Угу. А что главное в вашей жизни или в вашей деятельности?}
\soul{Что? Вы можете ответить относительно себя? Что главное в вас? В вашей жизни, что главное? Только не говорите мне ``Достичь вершин духовных''!}
\people{Нет, ну, наверное, творческие какие-то попытки. Выразиться как-то. Скорее всего, вот это. Вот, лично в моей жизни.}
\soul{Дайте, пожалуйста, чёткий ответ. Вы можете это сделать?}
\people{Да нет, пожалуй.}
\soul{Может быть, вся ваша жизнь в том, чтобы понять, найти ответ на этот вопрос?}
\people{Скорее всего, так оно и есть.  }
\people{А если, допустим, вопрос так поставить…  Заставить себя отдать максимальный вклад, который каждый из нас сможет сделать на развитие всего человечества?}
\soul{Заставить? Если вы говорите ``заставить'' – вы никогда этого не сделаете.}
\people{Хорошо. Не заставить, а сделать что-либо.}
\soul{Что-либо? Что – ``либо''? Вам привести множество примеров, когда делалось множество, чтобы запомнится, чтобы для всего человечества?}
\people{Нет. Вы немного неправильно поняли.}
\soul{Вы хотите ``заставить'' или ``хотите''?  Если у вас есть мысль ``для всего человечества'', значит, вы уже лжёте себе. Лжёте. Вы  возвышаете себя.}
\people{Хорошо,  не для человечества - для общества!}
\soul{Общества? Человечество и общество – вы разделили? Есть человечество, и есть, отдельно, общество? }
\people{Обществ много.}
\soul{Много обществ? Вы говорите ``Хочу для всего человечества'', мы говорим вам, что человечество – это одно, единое, а вы ещё говорите и ``общество''!.  Вы разделили себя на ``человеков'', разделили на ``общество'','' касты'' и так далее. Как же вы хотите… Где ваше, ваше доброе намерение? Если вы хотите сделать для ``общества'', а  другого ``общества'' нет! Потом… Так?}
\people{Ну, у вас эта цель существует, в вашем мире?}
\soul{Мы же  говорили вам, что мы сознаём, мы сознаём своё единство. Вы же – нет. Вы поделили себя на ``индивидуумов'', и сами себя - тоже. Посмотрите в себя! Вы себя поделили на множество ``себя''. Потом, идут ``другие''. Вы - первые. Как бы вы не говорили, что ``я последний, я всех люблю'',-  вы всегда первый. Вы, всегда первый. Просто, вы обманываетесь. Обманываетесь.}
\people{Вы говорите, что вы едины, но в то же время, отвечаете на вопрос, что у вас тоже есть индивидуумы. Как тут…? Логики, по-моему, никакой нету.}
\soul{Логики нет. А мы говорили вам, что вы задаёте вопросы сознанием. Вот вам и ``логика'' ваша. И мы говорим вам, что нет ``человека''. Нет! Человек – ничто! Но, есть ``Человечество!'' ``Человечество'' – да, есть, но нет ``человека'', и вам это не понять логикой, не понять того. }
\people{Скажите, есть ли преимущественное направления в вашей деятельности, которым вы, в любом случае, отдаете предпочтения?}
\soul{Наша задача -  не навредить. Если мы видим вред, мы уходим, в вашем понятии.}
\people{Скажите, а подвержены ли  вы болезням?}
\soul{Да. }
\people{А если болеете, то, как вы излечиваетесь?}
\soul{Если болеет один, то болеют все. Но и силы всех идут на лечение. В отличие от вас. Болеете вы -  вы лечитесь только сами, своими силами. В том и беда ваша. Хотя, если бы вы могли понять единство человечества и ничтожество одного, вы бы болели, от этого не уйти, но вы бы также легко и излечивались. }
\people{Скажите, а есть ли у вас неизлечимые болезни?}
\soul{Да. Вы это называете ``болью''. Болью сердца. Болью, которой не найдете названия. Если хотите - ``любовь'', если хотите – ``ненависть''. И многое, многое то, что вы называете ``чувствами''. }
\people{Ну, это разве это болезнь?}
\soul{Да, это болезнь, если они ``голые''. }
\people{А если у вас специальные места, где у вас лечат ваших больных?}
\soul{Если мы находимся везде - можно найти отдельное место? }
\people{То есть,  специальных стационаров или больниц у вас нет?}
\soul{Лечите вы. И вы приносите нам болезни - вашими реакциями. Мы приходим к вам ``добром'' и ваши реакции ``зло'' – мы болеем. И наоборот. От вас зависит, быть нам здоровыми или больными. Вспомните. В одном из контактов, мы сказали, что будем наказаны. Вы помните?}
\people{Да. И это получилось?}
\soul{И вы, конечно, не вспомните, какие были чувства у вас.}
\people{Может быть, сочувствие, но мы, наверное, слишком глубоко это не восприняли. Но вы имели наказания? }
\soul{В вашем понятии?}
\people{Да хоть в вашем.}
\soul{В вашем понятии – нет. В вашем понятии, мы бессмертны, значит, и не можем болеть. Болезни наши - только в нас.}
\people{Скажите, а, вот, у землян есть неизлечимые болезни?}
\soul{Есть.}
\people{И вы можете их назвать?}
\soul{Давайте скажем так: Земля больна вами. Вы - главная болезнь. Но это не значит, что вы плохи, и вы должны исчезнуть. Нет. Ваша же болезнь – ваша жадность, ваша ненасытность. Вам не хватает всегда и всего! И, даже затрагивая духовное,  вы хотите получить всё и вся! И если  стать кем-то, то обязательно Богом! Быть абсолютом! Даже здесь вы пришли жадностью. Вот болезнь ваша. Болезнь – вы сами. }
\people{Скажите, а как появляются болезни у землян? Только ли  от нервов, от злых чувств?}
\soul{Только.}
\people{Только?}
\soul{Только. Поймите. Нет наказания. Никто не наказывает вас, - ни ``белые'', ни ``чёрные''. Никто. Вы, сами несёте ответ за каждое ваше мгновение. Вы выбираете. Никто. И если говорят: ``Дьявол искусил'' - разве это не ваша вина? Разве не вы позвали его? }
\people{Скажите, а болезни появляются без причины, или существуют обстоятельства, вызывающие определенные болезни?}
\soul{Вы. Мы вам ответили только что. Вы. }
\people{То есть, человек сам повинен в своих болезнях?}
\soul{Не чаще.}
\people{А всегда?}
\soul{А всегда. }
\people{Но, он же, не хочет заболеть, допустим, смертельной болезнью, типа рака или чего-то?}
\soul{Да, он не хочет заболеть, но он всё делает для этой болезни. Возьмите, хотя бы, пример - зачем вы курите? Вы знаете, что это вредно, но вы курите. А потом вы будете говорить: ``Я болел раком, но я этого не хотел того''. Зачем вы набиваете в себя то, что вам не нужно?! Хотя, вы знаете, что это вредно. И придёт время, когда вы будете страдать, и будете говорить: ``За что?'', хотя, вы сами сделали это.}
\people{Понятно. А почему существует такое велико разнообразие болезней у человека?}
\soul{Вас же - множество. Должны же вы как-то отличаться? У вас множество поступков, множество всего. Должны же вы как-то отличаться? Или вы хотите – ``всем одинаково''?}
\people{А что необходимо делать людям, чтобы не болеть?}
\soul{Что? }
\people{Да.}
\soul{Мы вам ответили только что. Что вы должны сделать? }
\people{Чистота мыслей, добрых поступков. Так?}
\soul{Так.}
\people{Но, бывает и очень добрые люди, а почему-то умирают, как бы раньше других.}
\soul{Очень добрые люди… В вашем понятии, как, как вы измерили их доброту? Вы видели, что в них? Вы видите только наружность и часто не видите то, что скрывается за ней. Далее. Да вы можете привести множество примеров святых, что страдали и умирали. Вы говорите ``Испытывает Бог''. Бог виноват в вашей болезни? Его вините? Даже святые хотят обвинить Бога в болезнях, хотя приносят благодарность за них! Испытание. Да, конечно, так  легче пережить болезнь, считая, что это придёт ``свыше''. Бог ``прислал'' вам. А может быть - вы? Почему вы так любите оправдывать себя? Любыми путями - оправдать себя! И даже, когда вы говорите о себе, вспоминая все плохие эпитеты - вы это делаете с удовольствием  для себя,- вот, мол, ``я какой – честный''!}
\people{Скажите, а на стадии серьезного заболевания, что нужно предпринять человеку, чтобы, всё-таки,  излечиться?}
\soul{Мы ответили вам и более не скажем. Подумайте, приходили к вам, приходили сильнейшие, и говорили, что сделать. Вы, услышали их? Вы, исполняете их?    (заповеди. Прим.) }
\people{Ну, может быть обращение к Богу, если мы тут неверующие были, поможет таким…?}
\soul{Если вы, больны и приходите к Богу, можете расшибить все колена и лбы, и не получить.  Ибо вы будете жаждать избавиться от боли. Не более того. И когда придёт избавление, вы тут же забудете о Боге. Тут же! Ибо вы не поверили в него, а, просто, хотели купить его. Он же - не покупается. И он видит вас. Видит в вас всё, в отличие от вас, любящих ``наружность''. }
\people{Скажите, а есть ли для вас понятие ``смерти''? }
\soul{Нет.}
\people{Вы не умираете?}
\soul{Давайте скажем так: вы не умираете тоже. Смерть, и уже сегодня говорилось не раз, – это рождение нового. И мы говорили вам, и вы должны были помнить, и это даже снято вами, о смерти, о  её мгновениях и  длительностях и далее. Вы помните? }
\people{Да, конечно.}
\soul{Что же тогда вы называете смертью? }
\people{Ну, мы думали, что….}
\soul{Переход, дверь в иной мир вы называете смертью?}
\people{Да, вот, именно - момент перехода.}
\soul{Далее, - умер - и вы, говорите: ``горе, несчастье'', и вы страдаете. Вы страдаете, об умершем? Или, о себе, что потеряли? }
\people{Да, скорее всего, о себе, конечно. Скажите, а может человек, изменив свою судьбу, отдалить день смерти, и каким образом это он может?}
\soul{Да, он может это сделать. И каким образом – мы только что говорили вам  об этих ``образах''. Подумайте, мы говорили вам о болезнях. Болезнь и, в вашем понятии, смерть – это одно. Далее. Вы умираете по многим причинам. Одна из них – грехи ваши. Другая – когда пришло время, и было понято, что вы уже не исправитесь здесь. Есть другие причины, когда вам нужно просто отдохнуть ``там''. Есть причины, когда вам нужно обдумать, что делать далее. Там.  Есть причины, когда вам надо перейти на ``новый уровень''. Там. Вы поняли, сколько имеет множество понятий ``смерти''? Лучше было бы спросить, какой смертью лучше умереть, но это не дало бы вам ничего, потому что вы не любите принимать советы. И лишь только делаете вид, что вам нравиться то и это, и делаете всегда по-своему, ибо, это легче. }
\people{Нет, но мы слышали вообще странную вещь, что человек умирает от рака – это самая страшная для человека болезнь, - то это, как бы, благо. Как бы ``божественная благодать'' для него. Это правда?}
\soul{Конечно, легче обмануться, легче переносить страдания, если есть оправдание. }
\people{Нет, ну, значит, неправда?}
\soul{Поймите, всё относительно. Ваш мир  нереален! Всё, что вы видите,  реально только для вас. И если, для вас, счастье  – ``умереть от рака'', – да, для вас, это будет счастьем. Для кого-то, это ``страшно'' - то будет для него ``страхом''.  Поймите, нельзя говорить, о всех, можно сказать только о вас, спрашивающих. Ваш мир  нереален, поймите это.  Реаль – это то, что вы не видите. То, что находится внутри вас, то - реаль. То, что находится вдали от вас, невидимое, – это реаль. Вы же, то, что видите  – нереально. Посмотрите, посмотрите вокруг, я вам докажу, что это нереально. Спросите себя, что вы видите на этой картине, и спросите другого. И он  увидит другое. Спросите любое, и вы не найдете одинаковых мнений. Разве это реаль?}
\people{Скажите, а есть ли у вас друзья среди других мыслящих существ? Или есть ли друзья среди людей, человекообразных? }
\soul{Если мы говорили вам, что мы находимся везде – разве это не ответ?  И если мы говорили вам, что мы не имеем врагов – разве это не ответ?}
\people{Ну, да, в какой-то мере, ответ. Тогда, не могли бы вы помочь наладить связь с более цивилизованными человекоподобными существами?}
\soul{В каком понятии вы меряете ``цивилизованность''? Каким? }
\people{Ну, более высокоразвитых, чем мы. Которые уже болезни научились побеждать, те, против которых мы бессильны.}
\soul{Нет тех миров, где нет болезней. Нет таких миров. Вы должны были понять из прошлых }
ответов. Болезни ваши – это отражение ваше. Отражение вашей души. 
\people{Скажите, а знаете ли вы, что все люди разные, по своим характерам, развитию, по чувствам? }
\soul{Вы знаете это? }
\people{Ну, мы-то, знаем. А  вы-то, догадываетесь или ощущаете это как-то?}
\soul{Если мы находимся везде – мы слепы?}
\people{Ну, ладно. Спасибо. А какие качества людей вам больше всего нравятся? Цените ли вы честность, смелость, умение пожертвовать собой, у людей?}
\soul{Если мы вам скажем, что - да, мы ценим честность, то мы соврём вам, ибо вы по-разному понимаете понятие ``честность''. По-разному понимаете,  что такое ``любовь''. Каждый из вас имеет свои понятия на жизнь. Свои взгляды на понятие честности, любви,  благородности, и далее.  И поэтому, мы не можем любить то, что называете вы. То, что мы любим в вас – это то, чему вы не имеете названия. Это то, что вы называете ``неожиданностью'', `` порывом'' и далее, далее. Это любим в вас. То, что вы называете, всего лишь…(запись обрывается) }
\people{Вы говорили, что у вас нет врагов. Но, если наши эмоции приносят вам боль, болезнь, то почему вы не считаете нас врагами?}
\soul{Вы здесь увидели противоречие?}
\people{Да.}
\soul{В чём? А вот теперь, смотрите, что вы сказали сейчас! Если вы видите в том противоречие, вы уже говорите, о том, что вы имеете врагов. Тогда почему же, когда болен ваш друг, - вы не называете его врагом? Он же приносит вам несчастье, он приносит вам множество неудобств. Он - враг? }
\people{Я не говорил об этом. }
\soul{Далее. Какой вопрос вы задали? Поймите, вы ведёте записи. Вы ведёте технический способ, и, поэтому, мы будем отвечать вам на то, что задали вы, сознанием вашим, языком вашим, и то, что может воспроизвести ваша техника. Если же мы будем отвечать вам на то, что вы подумали – многие ли поймут?}
\people{Я вам задал вопрос: то, что если мы вам приносим боль, почему вы нас не относите к отрицательным своим воздействиям на вас? }
\soul{А вы подумайте, какой страшный вопрос задали вы. Кто тогда вы? Всё отрицательное, что приносит вам боль – вы считаете врагом? Ответьте мне на этот вопрос. Вы считаете врагом всё, что приносит вам ``отрицательное''?}
\people{Нет, не всё. }
\soul{А почему же тогда вы нас спрашиваете?}
\people{Ну, вы не ответили  всё-таки.}
\soul{Мы не ответили вам? Вы подумайте. Далее. Пришёл к вам Христос… Сколько боли причинили вы ему! Что, он считает вас ``врагами''? Что вы задали? Посмотрите, на себя, и что заставило вас задать. И дай Бог, в вашем понятии, если это задало всего лишь ваше сознание.}
\people{Хорошо, будем так считать. Ну, тогда повторите, все-таки, какие качества личности наиболее ценятся и предпочтительны у вас?}
\soul{Мы ответили вам.}
\people{Ну, неожиданные порывы…}
\soul{Вдохновение. }
\people{Вдохновение. Угу.}
\soul{Если хотите талант, все, что вы говорите от Бога.}
\people{Можете ли вы, иногда, поступать вопреки разуму, поддаваясь чувствам?}
\soul{В вашем понятии - нет. }
\people{То есть, вы только руководствуетесь рассудком?}
\soul{Нет. В вашем понятии, у нас нет разума, у нас нет чувств, ибо они соединены столь крепко, что не имеют разделения, и что мы хотим сделать от вас. }
\people{Скажите, вот есть такая версия, что болезни у людей - живые существа. Она верна?}
\soul{Отчасти - да.}
\people{А можно ли с ними, с этими болезнями, войти в контакт? Можно ли их изгнать, попросить уйти? }
\soul{А вы подумайте, кто вы? Если вы хотите, вы - колония множества живых существ; клеток, микробов и далее, далее. Вы согласны? }
\people{Так. Согласны. }
\soul{Если душа ваша –  хозяин над телом вашим, хозяин может избавиться от неугодного ``слуги'' и может нанять другого. Вы поняли ответ? }
\people{Ну, поняли.  То есть, мы можем болезнь уговорить уйти. Так?}
\soul{Уговорить?}
\people{Ну, да. Или, там,  приказать. Мы уж не знаем, как себя с ней вести.}
\soul{Можете. Но можете это сделать духовно. Да, вы это можете сделать, но, к сожалению, вы ищете избавления болезни сознанием. То есть, тело хочет вылечить себя, не призывая на помощь душу. И, когда душа приносит вам совет, столь необычный для тела, то вы говорите: ``Умопомрачение'' и боитесь исполнить то. И потому и болеете, что не восприняли то, что было нужно вам. И когда к вам приходят и говорят: ``Есть способ излечить так и так'', - вы говорите - ``Да, я знаю его уже давно'' и, потому, ему не доверяете. }
\people{То есть, надо слышать душу стараться, когда слишком больно. Да? }
\soul{Если мы говорили вам, и даже вы говорили нам, что - духовные болезни – так лечите, тогда, душой. }
\people{Ладно. А скажите, вам известна болезнь такая, как рак? }
\soul{Да.}
\people{Чем она вызывается? }
\soul{Чем?}
\people{Да. }
\soul{Само название ваше уже говорит об этом.}
\people{Нет, она не похожа на существо рак, ракообразное.}
\soul{А мы и не говорили, о раке. Вы измените слово. И, когда-то, она звучала…}
\people{А-а… Рок, да?}
\soul{Да. Это уже вы - перевели.}
\people{Понятно. А это вызвано неправедным образом жизни или самой судьбой человека? }
\soul{Если мы говорили вам, что вы - хозяин, и что всё, чтобы не случалось с вами – виноваты вы, - причём здесь ``судьба''? Судьба – это, всего лишь, ваша ``подруга''. Любые её удары направлены вами. Если вы любите судьбу свою – она ответит вам любовью, если вы ненавидите е` – получите ненависть. Всё, что било вас, нанесено вами же, вашими же руками, вашими мыслями.}
\people{Вот, наверное, поэтому есть народности, где вообще не болеют раком. То есть, роком. Есть такие?}
\soul{Называйте ``раком'', вам это будет привычней.}
\people{Угу. То есть, такие народности есть? Они как-то сумели душой жить, сохраняя здоровье?}
\soul{Поймите. Вам трудно разделить ``духовное'' и ``физическое''. Вам трудно, ибо вы не видите здесь границ. Поймите, многие болезни ваши  можно назвать, что они биологические. Если вы влили себе яд,  это же не значит, в вашем понятии, что вы ``духовно'' заболели? Нет. Это сильно отравило вас. Но только вы не можете оценить одно ``но'',- что заставило вас влить яд? Что?}
\people{Да, мы не сможем это. Скажите, одна исследовательница аномальных явлений заявила, что она мечтала бы умереть от рака. В чём она права или не права? }
\soul{Мы вам отвечали. Мы вам отвечали, если вы живёте в нереальном мире, то все ваши мечты – это только ваши. Ваше видение – только ваше. И если ваше счастье в том, чтобы умереть от рака, и когда вы умрёте от него – и вы будете счастливы, и вы будете правы в этом, ибо это ваше счастье. Но, для другого – нет. Если он, конечно, не хочет того же. }
\people{Скажите, вот очень важный вопрос, для одного человека: можете ли вы помочь излечить больного и при каких условиях: 1) если это будет единичный случай или для демонстрации ваших возможностей, 2) если это будет сугубо секретом между нами? Можете так помочь излечить?}
\soul{Зачем нам демонстрировать себя? Мы, наоборот, приходим и будем создавать вам много, много сомнений - есть ли мы вообще.  Ибо, это заставит вас думать. Поймите, если вы будете уверены, что мы существуем и будете знать ``от и до'' кто мы, то вы воспримите все наши ответы или за правду, или за ложь, не будете уже думать о них. Даже если это - истина, вы будете говорить: ``Я знаю то''. Не более. }
\people{Одним словом, вы не…}
\soul{И тогда, мы потеряем смысл. Смысл.  И вы, всё, что приобрели, потеряете. И потому, мы, когда-нибудь, придём и посеем вам столь сильные сомнения, что вы откажетесь от нас или нет. Всё будет зависеть от вас. Придёт время, когда вы будете осмеяны, и тогда  вы должны будете пройти испытание: остаться ``при своём'', или пойти, куда ``идут'' все. Пойти со всеми, лишь бы не отстать и не быть осмеянным. Придёт время, когда - ``ваше”-, в вашем понятии, -будет осмеяно, ваши взгляды будут оплёваны, и вы должны будете найти силы или отказаться от них, или  поднять их. }
\people{А кем - оплёваны? Нашим обществом? Человечеством? Нашим окружением?}
\soul{Да. Но не понимайте это буквально. Почему вы принимаете всё буквально? }
\people{Нет, не буквально. Мы понимаем – метафора.}
\soul{Разве их нет? Разве сейчас не ``плюют'' в вас?  Метафора!}
\people{Не так уж сильно. В принципе, мы привыкли к скептицизму тех,  кто…чем мы занимаемся. Но, мне кажется, люди сейчас более привыкают к аномальным явлениям и к необычным таким проявлениям и вряд ли будут гонения в ближайшие годы.}
\soul{Мы говорили вам, о вашей реакции, о добре и зле. Вы должны помнить. Придёт время, когда вы примете нас за ``зло'', ибо мы будем вынуждены дать вам испытание. Испытание вашей твёрдости. И придёт время, когда вы примете нас за зло, или за добро. Это должны будете решить вы. Когда мы придём, в вашем понятии ``жестокостью'' и поможем осмеять и ``оплевать'' вас, и вы должны будете разобраться – ``добро'' ли то было или ``зло''. }
\people{А кому вы поможете оплевать и осмеять?}
\people{Ну, нашим оппонентам. }
\people{Вот, я и говорю - кому.}
\people{Оппонентам  всё-таки?  Да? То есть, вы нас попробуете так экзаменовать? }
\soul{Вся ваша жизнь – экзамен. Поймите это. И многое зло, вы принимаете за добро, и, так же, добро принимаете за зло. А можно сказать проще, было бы точнее для вас - нет зла, есть только добро, и добро вы принимаете за зло, или, всё-таки, за добро. Приходит время, когда бьют вас, и вы говорите: ``Это зло''. Да, это будет зло, если вы будете говорить, что то - ``зло'' и не увидите в том урока. Но, если будете биты, но воспримите добром, новым уроком, испытанием, то это останется вам добром и не приведёт зла. Вот вам одна из заповедей, о щеках.}
\people{Скажите, но вы, всё-таки, не беретесь излечить одного серьезно  больного человека? }
\soul{Мы же говорили вам. Мы говорили вам, о вашей физике, о нашей незаинтересованности, и о том, что вы подразумеваете под добром и злом. Что? Болезнь – это зло или добро? Как вы воспримите это?}
\people{Зло, конечно, для родственников этого человека. }
\soul{Да? Были множество, что были прикованы к постелям болезнями, ими были написаны множество произведений, и не считали болезнь злом, и даже были рады болезни той. Возможности временной. Было множество тех, кто прозябали в нищете, и создавали великое! И сколько множество использовало нищету ту, обогащаясь!}
\people{Скажите, а известно вам искусство врачевания филиппинских хилеров?}
  1–2–3–4–5–6–7.
\people{Известно вам искусство врачевания филиппинских хилеров?  }
\soul{И да, и нет. Ибо мы не поддерживаем.}
\people{Но многие специалисты относятся к хилерам, как к шарлатанам или фокусникам. А на самом деле?}
\soul{Они не шарлатаны. Но мы не поддерживаем.}
\people{Так, понятно. А действительно, что они могут вторгаться в человеческое тело, не расторгая кожные покровы?}
\soul{В этом нет ничего необычного. Мы говорили, о вас, как о множестве колоний. Вы – колония множества живых существ. И вы сами говорили: ``Можем ли мы уговорить болезнь отойти?'' Если вы можете уговорить болезнь отойти - почему не уговорить…}
\people{Тело раскрыться. Да?}
\soul{Спрашивайте.}
\people{А каким образом они это делают, вы не можете это сказать?}
\soul{Если мы вам скажем, вы начнёте делать то? Нет. Они не говорят того, почему должны будем сделать мы? Почему мы должны быть предателями? Почему мы должны быть раскрывателями чужих тайн? }
\people{Хорошо. Вы справедливы. Тогда, на самом деле они способны удалять раковые опухли, как они говорят? }
\soul{Да. Могут.}
\people{А может, это лишь…}
\soul{Они могут избавить вас лишь только от болезни. А от самих корней - нет.}
\people{То есть, это временное избавление?}
\soul{Ну, почему же? Всё зависит от времени. Вы можете, просто, не вырастить новую болезнь, не успеть, ибо умрёте. И тогда,  можно будет смело сказать, что врачеватель добился и вылечил вас полностью от той болезни. Лишь благодаря тому, что не хватило времени вырастить новую. Но, вы забываете  и другое. Если к вам приходят излечивать одну болезнь, вы получите другую, не лучше и не хуже той.}
\people{Угу.  А скажите, перспективны ли исследования и методики лечения хилеров, для лечения людей? }
\soul{Да. Вы должны помогать друг другу. И, хотя вы это делаете неправильно, всё-таки, иногда, вы это делаете с чистой душой. И когда приходит к вам врач и лечит вас, он должен делать это. Он должен. Даже если это он делает чисто физически. Он должен всё равно делать то. Ибо, это тоже - урок для него и для вас. И каждое ваше мгновение – это урок. Урок. Школа. И пока вы не поймёте того, вы будете совершать множество ошибок. Вы будете, повторять множество уроков. Множество. Пока не поймёте и не исправите. Но нам трудно объяснить вам, и вы будете говорить, что мы не ответили вам. Как мы можем ответить то, что вы не можете понять? Что не может понять ваше сознание, но прекрасно видит ваша душа. Но ваше сознание столь могуче, столь сильное, что закрыло душу. И душа та воспримет лишь только тогда, когда, в вашем понятии, придёт смерть. Когда ваше сознание умрёт. Ваше сознание умирает. Ибо ваше сознание видит химию. Ваша память – химия. Все ваши чувства – это реакции души. Реакции души. И если от страха в вас – адреналин, – это не значит, что адреналин порождает страх. Нет. Это реакции души рождает адреналин, и тело отзывается страхом. Далее. Далее…(18:09)}
\people{1–2–3–4–5–6–7–8–9. 1–2–3.}
\soul{Спрашивайте.}
\people{Скажите, а могут ли другие люди, не филиппинцы, овладеть искусством хилеров?}
\soul{Это можете сделать даже вы. }
\people{А что мешает?}
\soul{Что? }
\people{Да.}
\soul{Ваше сознание. }
\people{То есть, неверие.}
\soul{Неверие. Да, это- первое. Если вы уверуете, что это вы можете сделать, то вы найдёте другие причины, но они уже будут более преодолимы. Но, если вы будете говорить: ``Я верю и я верю'' – это не значит, что вы стали верить. Просто, сознание пытается обмануть. Ибо у вас есть, и вы называете это ``инстинктом'', хотя вы не правы, истинный инстинкты, что даны природой, не вредят вам. В том теория ваша не верна. Вы же называете инстинктами то, что приобрели сами. И  в вас есть страх, страх неизведанного, и  как бы вы ни говорили, что ``Я не боюсь'' -  страх остаётся. Остаётся в вас. Он может перейти в сомнения. Чаще, этот страх, есть просто сомнения. Вы называете другим именем. Мы когда-то говорили вам, что самое страшное в вас, самое сильное в вас – страх. От страха рождены все ваши чувства, от страха ``иные''. От страха. И, даже любовь, замешана страхом. Любовь чиста. Да. Но, чаще, к сожалению, любовь ваша – есть составляющая страха. Страх порождает эгоизм. Любя, вы рождаете ревность. Вы поняли ``цепь''?}
\people{Угу. Понятно. Скажите …}
\soul{Рождая ревность, рождаете ненависть. Вы начинаете ненавидеть. И даже любящее существо, которого вы ревнуете – вы начинаете ненавидеть за тот страх, который оно принесло вам. Ибо вы обвиняете любимого, в том, что вы ревнивы. Так, легче. Легче оправдать себя и обвинить другого. Это столь тонко, что вы не можете найти звенья той цепи. Спрашивайте.}
\people{Скажите,- вернёмся к хилерам, - правы ли те, кто обвиняет хилеров в колдовстве и в пагубных для людей методах лечения?}
\soul{Простите, вы все разные. Есть и доброе, и злое. Но, если ведите только злое, что ж, и здесь вы правы, ибо вы увидели только злое и не видели добра. Если кто-то придёт и скажет, что видел только добро и не видел зла – он тоже прав. Ибо он видел добро, но не видел зла. Как вам объяснить то? Вы не можете понять это сознанием. Это - противоречие. Лишь только душа знает ответ. }
\people{Скажите, а действительно ли излечиваются больные после операции хилеров? Может они, вскоре, умирают, а мы, просто, не знаем об этом? Ведь в фильмах об этом не показывалось. }
\soul{Нет. Вы можете излечивать и себя и других. Это может сделать любой. И каждый из вас уже лечил. Вспомните, вспомните себя. И не воспринимайте буквально. Приходило время, когда вы были в ссорах. Вспомните. А разве хороший совет - не лечение? Вы лечили малым. Лечили малым, в вашем понятии, ибо вы не могли проникнуть внутрь и избавить от болезни. А мы скажем вам, что вы лечили больше. Гораздо больше, чем умение быть ``хирургом без ножа''! }
\people{Скажите, у хилеров часто бывают неудачи? }
\soul{Они есть у всех. Ошибки – это ваше несчастье, и счастье. И здесь- противоречие. }
\people{Вот, ни разу мы не задавали такой вопрос, -  ощущаете ли вы вкус, запах?}
\soul{Когда мы приходим, в вашем понятии, к ``переводчику'', всё, что ощущает переводчик, ощущаем и мы. Ибо мы видим его реакции. И потому, мы можем сказать, что мы чувствуем вкус, чувствуем боль и далее, далее. И мы можем помочь ему избавится от того. Вспомните. Вы тоже лечите гипнозом. Вы согласны?}
\people{Ну, лично я…}
\soul{В вашем понятии, - что такое ``лечение гипнозом''? Самое простое объяснение, это поверить, что вы не больны. Сознанием вы не можете это сделать, ибо у вас страх и неверие. Сомнения. И потому, не можете себе доказать, что вы здоровы. Если ж вас ввести в бессознательное,- заметьте, бессознательное  - ``не имея сознания'', - вам легко внушить, что вы здоровы. Ибо душа не будет закрыта сознанием, хотя, ещё закрыта плотью. И чем глубже, в вашем понятии, вы сумели ``войти''  в тело, тем ближе вы к душе, и тем больше эффект от лечения, тем больше вы будете здоровы. И если вы, выйдя из этого состояния, скажете сознанием, что вы, всё же, больны –  вы тут же заболеете… }
\people{То есть….}
\soul{…что чаще и приходит. Вспомните сны ваши, когда вы больны, - во снах вы - здоровы. Вы здоровы - во снах. И там вы излечились. Если вы проснулись бы и сказали: ``Я здоров'', сознанием, - вы бы уже не болели. Но вы встаёте и вспоминаете, что вы больны. Что? Это гипноз. Это тот же гипноз. И он имеет один же корень, и то же название. Вы согласны?}
\people{Да.}
\soul{Вот и подумайте. Лечение гипнозом, и лечение ``себя во сне''. Вы когда-нибудь видели себя, во сне, больным? Вы скажете: ``Да, видели. Было и будет''. Но когда вы встаёте, вы же не чувствуете той болезни? Вы скажете: ``Противоречие!'' - тому, что мы сказали. Да? Нет. Ибо мы говорили вам, что многие болезни – это уроки. Если вы видите во сне, что вы больны, значит, вы проходите ещё одну ``ступень''.  И, просыпаясь, успешно пройдя её, вы никогда не будете болеть той болезнью, что болели во сне. Вы поняли столь длинный монолог?}
\people{Ну…В общих чертах. И при прослушивании, больше обратим внимание. Но вот смертельно больному человеку посоветуете, как раз так – самовнушением,  побороться за свою жизнь? }
\soul{Если мы вам скажем: ``Обратитесь к Богу'', то это такая битая фраза, что вряд ли вы поверите в неё. }
\people{Ну,  может быть так и посоветовать, действительно обратится к Богу, с большими молитвами, с большим желанием вылечиться? Может быть, поможет это?}
\soul{Прежде, этот человек должен хотеть жить. Жить. И не сознанием, ибо сознание здесь бессильно что-либо сделать. Сознанием вы скажете: ``Да, я хочу жить'' Конечно, никто не хочет из вас умирать и мучиться. Он должен хотеть жить духовно, в этой жизни. В этой. Ну почему вы так цепляетесь? Почему вы боитесь потерять? Почему вы боитесь уйти и страшитесь смерти? Если вы же говорите: ``Смерть – это параллельный мир, это шаг в новое, это новая ступень'' и многого много. И, в то же время, вы боитесь. }
\people{Для многих, это, всё-таки, небытие. }
\soul{Но и не подумайте, что мы говорим вам: ``Бросьте и идите умирать''. Нет. Если вы это сделаете, вы сделаете сознанием, не душой. И потому, вы говорите о самоубийстве, как о плохом. И даже хороните отдельно. Ибо убивало сознание, а не душа. И когда вам говорят ``Его душа… его душа…''.}
\people{1-2…}
\soul{У этой куклы нет души. А я её любил! А зачем?}
\people{Какую куклу? Переводчик, что-то вспоминает о своей прошлой жизни? }
\soul{Ключ.}
\people{О каком ключе вы говорите?}
\soul{Мы говорили вам, в начале, о ключе, о названиях, о словах. Вот и ``ключ''. }
\people{Хорошо. Мы продолжаем вопросы. Каким образом вы перемещаетесь в пространстве?}
\soul{Зачем нам делать то, если мы находимся везде? }
\people{Но если вам надо на другой край Вселенной попасть?}
\soul{Вы не можете понять – ``единый и существующий везде'', и, в то же время `` индивидуум''. Каждый из нас - индивидуум. Вы должны были бы заметить, что иногда мы говорим ``я'', иногда говорим ``мы''. Вы помните?}
\people{Помним.}
\soul{Для нас это не имеет значения. Это - только для вас. Ибо, один, или множество нас – это едино. И нам трудно объяснить, что такое индивидуум, который находится везде или нигде, и что ему не нужно перемещаться, потому что он находится везде или нигде. Противоречие. Ему не нужна память, ибо он видит. Зачем ему помнить, если он может увидеть  любое мгновение? Ибо мы не знаем понятия, о времени. Мы говорили вам, память нужна лишь только тем, кто не властвует временем. И для них существуют понятия ``прошлое'' и ``будущие''. Мы не обладаем этим. Не обладаем. И, когда-то, этим будете не обладать и вы. Для вас это будет, сперва, большая потеря, когда вы потеряете чувство времени. А бывает, что вы теряете, даже при вашей жизни бывает, когда вы теряете мгновения и даже более. Время – и вы пугаетесь этого. }
\people{Понятно.}
\soul{И придёт время, когда вы потеряете многое. Если вам потерять это сейчас, неожиданно, неподготовленному сознанию, – вы испугаетесь, и, в вашем понятии, это будет ``смерть''. Cмерть в ``никуда''.  Ибо вы не будете знать пространства, ибо вы не будете существовать, в понятии сознания. Вы поняли, почему вы должны ``утончаться'', -  в вашем понятии? Хотя, это очень грубая аналогия - ``утончение'', но вам легче принимать так. Пусть будет ``утончение''. Спрашивайте, далее.}
\people{Так. А как вам представляется информационное поле Земли? Вы пользуетесь им? }
\soul{Мы живём в нём, в вашем понятии. Далее. Что вы подразумеваете ``информационным полем''? Знаете ли вы, что даже атом, и более менее, есть частицей того поля? Вы говорите, как об отдельном. Вы – частица того поля. Частица. И, глядя на вас, я могу прочитать всё поле. Всё и вся.  Глядя на любую точку,-  вы говорите ``лептон'', - вы можете прочитать всё и вся. И, тогда, уже нет понятия, - уже там нет понятия о ``прошлом'' и ``будущем''. }
\people{Скажите, а могут ли пользоваться этим полем земляне в будущем, и пользуются ли они должным образом? }
\soul{Вы все пользуетесь. Вы пользуетесь всем. Всем, что здесь есть. Но не ведаете тем. Не ведаете. Вы пользуетесь информационным полем, но, не подозревая того. Вы подумайте, логически. Вы поймёте, что пользуетесь. Что даже чувства ваши основаны, во многом, на этих полях. Подумайте. Вы увидели, впервые, человека, и вы уже чувствуете к нему приязнь, или наоборот… Согласны?}
\people{Да. Это знакомо нам.}
\soul{И при этом, вы удивляетесь - ``Ну, почему, если наружно он красив…''  Или, наоборот. И вы не можете понять причину. А причина проста - вы сумели прочитать. Прочитать не сознанием. Ибо сознание не пользуется тем полем, в вашем понятии. Хотя, сознание прекрасно владеет и им. Но лишь малым. Малым. }
\people{1–2–3.}
\soul{Спрашивайте.}
\people{Скажите, а использование информационного поля Земли повысит способности и общий потенциал человеческой цивилизации или нет? }
\soul{Да. Есть два варианта. Или поднимет, или опустит. Мы говорили вам, о преступниках. Вы помните?}
\people{А-а. Да-да-да.}
\soul{Мы сегодня говорили вам, что всегда будут существовать болезни. И нет миров, где их нет, и, вы помните?- значит, есть и преступники. И разница лишь только в том, в могуществе их. Когда-то было достаточно ножа, чтобы убить. Сейчас, вы можете убить преступника и убить себя, или кого-то иного, даже не видя его. Раньше, вы могли убить только одного, сейчас, вы можете убить множество. Тысячи. Придёт время, когда вы будете обладать огромными энергиями и убивать уже более. Почему вы говорите - ``обладая большей энергией'', и вы говорите, что это - больше ``святости''? Тогда почему же вы не вспомните, о падшем ангеле?}
Он обладает большей эне…
\people{1.2..}
\people{Скажите, видите ли вы ауру Земли, и какого она цвета, если видите?}
\soul{Цвета?}
\people{Да.}
\soul{Если мы вам скажем, что аура ваша темна, что даст это вам? }
\people{Ну, аура Земли темна? Да?}
\soul{Вы считаете идеальный цвет – белый цвет, ибо он содержит всё. Здесь вы правы. Земля ваша, когда-то, была бела, пока не пришёл на неё человек. Придя, человек принёс тучи. Эти тучи растут более и более. Страшно было бы, если бы ваша аура, вашем понятии, была черна. Но этого не будет. Не будет. Ибо есть силы, извне, и в вас, которые не дадут, не дадут уничтожить вам. Это - один из инстинктов природы. Истинных инстинктов, а не тех, что называете вы ``инстинкт''.  Когда вы, упав уже столь глубоко, вы уже не упадёте глубже,  ибо испугаетесь сами, и будете подниматься. Ибо сам дьявол придёт и скажет: ``Свергните меня!''.}
\people{Скажите, как вы взаимодействуете с астральным планом Земли? }
\soul{Это физика ваша. И мы, так же ведём разговоры, в вашем понятии `` контакты'', и с ним. }
\people{Угу. Для вас, есть в этом мире интересные цивилизации или, просто, партнёры? }
\soul{Мы говорили вам, что у нас нет врагов. }
\people{Угу.}
\soul{А, значит, всё является ``нашими партнёрами''. Если мы находимся везде и вся, какие могут быть речи…речи…речи…}
\people{1–2.}
\soul{Скажите, поддерживаете ли вы контакты с Шамбалой? Это астральный мир или энергетический? }
\soul{Его назвать нельзя ни тем, ни другим. Это - иной мир. И вы, для упрощения, называете его своими именами. Это - другой мир. Мир, в который вы должны придти. Мир, который вы называете ``мечтой''. Но, чаще, вы удаляетесь от того мира. Даже тем, что вы им находите названия где-то и далее. И, когда вы придёте в мир тот, вы будете удивлены. Удивлены тем, что он не соответствует мерам вашим. Ибо вы уже представляете, какова она, и будете разочарованы, и, для каждого, это разочарование принесёт или горе, или счастье. Смотря  на что он рассчитывал. }
\people{Какова структура вашей цивилизации?}
\soul{В вашем понятии - у нас нет цивилизации. Ибо цивилизация  в вашем понятии, это: ``структура”;  ``общество”;  `` расы”; ``страны”; `` народы''.}
 
\people{А вы  кому-нибудь подчиняетесь? }
\soul{Нет. В вашем понятии – да. Ибо вы, тоже подчинены, но вы не знаете кому. Мы же говорим вам – душа. Вы говорите – Богу. Хотя, вы стараетесь не слушаться его. И вы имеете двух глав – Бог и анти. Мы же, говоря вашим языком, похожи на вас. Мы, тоже  ищем. Мы не ``абсолют''. Поймите. Далее. Если придут к вам и скажут ``Мы есть абсолют и будем говорить с вами'', не верьте тому. Не верьте. Ибо вы, сознанием, не можете услышать его. Не можете. Чем могучей, чем сильней сознание, тем больше сознание боится, тем больше преград. И потому, вам легче вести контакты с ``низшими''. Ибо они могут быть ниже сознанием, или намного выше его. Лишь немногие могут, в вашем понятии, вести контакт ``душой''. И тогда, будут приходить ``высшие''. Действительно ``высшие силы''. Но они подчинены тоже. Всё подчинено, всё имеет. (иерархию. Прим.)}
\people{1–2.}
\people{Подскажите. Значит, у вас нет отчётности перед ``высшими инстанциями'' за качество, допустим, наших контактов с вами?}
\soul{Почему же? Разве мы говорили того? Мы же говорили вам, что мы приносим наказания, и уже получали.  Вспомните. }
\people{Да-да.}
\soul{Вы наказываете нас. Вы. Вы – наши повелители. Ибо мы приходим к вам, когда вы просите, и не можем придти ранее. Потому, что вы просите, всё-таки, чаще, сознанием. И, в то же время, мы скажем вам - да, вы спрашиваете сознанием, вы  ищете ответы сознанием, и всё же - душа, душа зовёт. Душа жаждет. И потому, мы приходим и говорим вам. И, хотя мы говорим сознанию вашему, - душа слышит нас. Ибо душа звала, всё же. Душа. Лишь сознание только приобрело зов - в физическое. Спрашивайте.}
\people{Скажите, а есть ли научный интерес у вас, к нашим, вот этим, контактам? }
\soul{Поймите, мы говорили вам, что вы и учителя наши, и ученики наши. Мы учимся вести и разговаривать с вами. Вспомните первые, и вспомните сейчас.(контакты. Прим.) Далее. Вы – ученики. Ибо спрашиваете, и мы отвечаем. Вы – экзаменаторы. Ибо по вашим реакциям смотрим, что сделали мы. И мы экзам…}
\people{Скажите, а как у вас осуществляется связь с глобальным высшим разумом? }
\soul{Вы опять говорите о ``высшем'' и о ``низшем''. Опять! Вот вам - и наша неудача, и ваша непонятость. Мы говорили вам, что мы не делим на ``высшее'' и ``низшие'', и что в вас  есть высшее. Ну, подумайте сами, даже логикой, сознанием вашим, - подобие Богу… Что же, Бог ``низок''? Значит, есть в вас ``высшее''!}
\people{Мы не верим.}
\soul{Значит, есть в вас ``высшее''! То ``высшее'', что ``выше'' уже нет. Но у вас есть и ``низшие''. И ``низшему'' легче, гораздо легче, чем ``высшему''. Ибо ``высшее'' не может применить насилие. И потому, Бог не придёт к вам незваным. Ибо, вы испугаетесь, и тогда, получается - он принёс вам насилие. Он не будет принуждать вас верить себе. Вспомните, разве Христос требовал с вас веры в него? Нет. Он только просил! Ибо, тогда, он бы ничем не отличался от вас, жаждущих славы и почёта. Он пришёл и учил вас. И мы хотим быть подобны ему. И вы хотите быть подобны ему. В том наше ``высшее''. В том, что мы хотим быть истинно подобными Богу. }
\people{Ну, это плохо, что у нас, у Землян, нет авторитетов, даже в лице Бога, высшего разума? Мы, как-то, отрицаем…}
\soul{Разве? Что говорите вы? Что говорите вы? Если в вашем понятии  ``высшие”…  Выше – это Бог, и он, всё-таки, занимается вами! ``Абсолютное зло'' – и тоже, занимается вами! Как же вы тогда говорите о ``невнимании'', если ``высшие'' занимаются вами!?}
\people{Скажите, мы готовы сейчас прекратить контакт. Как вы считаете, достаточно мы поговорили?}
\soul{На то, что вы спрашиваете душой, мы и отвечаем душой. Мы не смотрим на ваши желания и  почёты. Мы смотрим на то, что имеете вы внутри. Мы не смотрим на вашу внешнюю красоту. Мы смотрим то, что вы имеете внутри и чем отличаетесь. }
\people{Хорошо. А почему именно переводчика выбрали… У вас были какие-то основания выбрать именно этого переводчика?}
\soul{Мы выбирали? А почему не выбрали вы себя? Ложитесь и сумейте. }
\people{Но мы пробовали, и у нас не получалось. }
\soul{В том беда наша? Мы можем применить к вам насильно, а что это даст вам? Что? Представьте, что сделает ваше сознание, и где вы тогда будете? А многие делают так. Идёт серия контактов, и где контактёр? Что происходит  с ним? Ибо - идёт насилие.}
\people{Скажите,  после показа по центральному телевидению нашего контакта, не могут ли многие желающие повторить их, и не будет ли это вредом для человечества? }
\soul{Мы говорили вам, всё зависит от того, с чем вы пришли. Мы говорили ``решето''. Мы говорили вам, что мы - всего лишь сила, в вашем понятии ``сила'', а разум - ваш, ваше сознание. И мы говорили вам, что ваши вопросы – и ваши же ответы. Вспомните и сложите всё. И если вы хотите зла – будет зло. Хотите добра – будет вам добро. Вы можете задать любой вопрос. Если рассуждать логически, мы должны ответить на любой.  Если вы спросите об оружии, мы должны ответить об оружии. И если вы спросите, как спасти человека, мы должны ответить. Если вы спросите, как убить его, мы должны ответить. Рассуждая вашим сознанием.}
\people{Вы не всегда отвечаете. }
\soul{Но, мы же говорили вам, что мы, всё-таки, слушаем душа вашу. }
\people{И поэтому даете такие абстрактные ответы? }
\soul{Абстрактность – это ваше счастье. Если бы вы получили всё готовое, что бы делали вы? Что бы вы тогда делали? Вы бы умерли. И если бы вы не боролись, что было бы с вами? И что в вашем понятии ``абстрактность''? Абстрактность – это то, что не может понять логика. Тогда любое искусство абстрактно, ибо логика не может понять. И говорит, вашими словами ``колебания''. Музыка – это все лишь колебания. Живопись – это всего лишь набор красок, цветов, если хотите волн. Сознание принимает это так. Разве сознанием вы рассматриваете? }
Любое, любое то, что вы называете абстрактностью – это искусство. Искусство можно воспринимать только духовно. Мы же не хотим делать из вас роботов. Есть контакты, когда приходят и делают из вас ``машины''. Или, в вашем понятии, умалешнными. Это, когда  приходят, и отвечают, и дают вам те ответы, которые вы хотите услышать, ваше сознание. В тех контактах вас восхваляют, если вы того хотите. Вас называют Богом, провидцем и кем угодно, лишь бы угодить вам. И тогда вы размягчаетесь, тогда вы довольны и говорите ``Вот контакт. Вот я – Бог! Вот я - пророк!'' И вы называете это ``контактом''? А когда приходят к вам, вы задаёте, и вам дают ответы, что не нравятся вам,  вы говорите: ``Это не контакт''. Контакт – это только то, что вам нравится, что вы можете понять? Тогда, извините, зачем вы пошли в школу? Вы пришли в первый класс, много ли вы могли понять? Много ли? Но вы продолжали учиться. И то, что для вас является сейчас абстрактностью, счастье, что вы не можете сознанием это понять. Ибо ваша душа будет давать пищу сознанию, и сознание будет думать. Поймите, вы когда-то решили, что сознание не нужно. По той простой причине, что есть душа. А сознание, всего лишь, мешает. Но мы говорили вам, что есть эфир. Эфир – это всего лишь библиотека имеющая множество знаний. Знаний, но не разума. Сознание дано, чтобы природа могла осознать себя. Чтобы могла думать, а не иметь те, просто знания, мёртвым грузом. И пока сознание ваше слабо. Когда-то и вы не умели ходить, но научились. Также сознание ваше растёт и растёт. И оно всё меньше будет видеть в абстрактности -абстрактность. Вы поняли меня? 
\people{Да.}
\soul{Если мы будем говорить вам конкретно, это значит, что мы будем вам врать. Лгать в угоду вам. Вашему сознанию. }
\people{Действительно, уже такие контакты были. А как вы оцениваете сегодняшний наш разговор?}
\soul{Вы должны оценить его. Вы. Не мы. }
\people{Я нахожу в этом разговоре довольно-таки много противоречий. В том, что вы сами сказали.}
\soul{Прекрасно. Значит, мы добиваемся своей цели.}
\people{Хорошо, это говорит о том, что мы  мыслим.}
\soul{Мы неоднократно говорили вам, и мы даже делали это специально, когда мы говорили вам одно, и тут же говорили вам иное, и тут же говорили вам: ``Противоречие''. А теперь вдумайтесь. Вы не можете сложить, ибо вы видите противоречия. А знаете ли вы, что всё великое состоит из противоречий соединённых вместе?  Сознание ваше не может понять,-  ``Как? Это должно существовать и должно?''. Сознание этого не может понять. Но приходит мгновение, когда вы создаёте это  и -  ``существует и нет''. И мы можем вам даже привести примеры, на вашей физике, на ваших науках. Но пусть это сделает сам переводчик, ибо он знает ответ на этот вопрос. И вы знаете, когда-то он говорил вам, восторженно и радостно объясняя, что такое ``АРИЗ''. }
\people{АРИЗ? АРИЗ? Мы, что-то, такого слова не знаем. (алгоритмы решения изобретательских задач. Прим.)}
\soul{Спрашивайте далее. }
\people{Скажите, вот сейчас у переводчика серьёзно больна жена, можно ей чем-то помочь, нетрадиционными способами?}
\soul{Да.}
\people{Чем и как ей помочь?}
\soul{Что делает ваш переводчик? }
\people{Уколы делает. Ну, может быть химия не очень здесь поможет, но…}
\soul{Уколы? Мы говорили о не уколах. Поймите, мы говорили вам, о вас, и в этих контактах, что вы делаете более, чем ``хирург без ножа'', вы помните? }
\people{Да.}
\soul{Если вы придёте с душой и добротой - гораздо легче будет переноситься любая боль. И потому, не имея извне, вы создаёте, порою, сами. Если к вам не приходят, не жалеют вас, вы создаёте мифы, что от рака умереть – это счастье. И вам действительно будет легче перенести эту болезнь, ибо вы обманули себя. Но, этот обман не будет караться никем, ибо вы обманули себя, но не обманули никого. Вы поняли?  Далее. Вы всю жизнь занимаетесь самообманом. Всю жизнь. И даже сейчас, споря с нами, вы занимаетесь самообманом. Ибо спорит сознание ваше. Сознание ваше хочет конкретных ответов. Вы задаёте конкретный вопрос и получаете неконкретно. А теперь, представьте, пришёл бы к вам Христос и стал бы говорить конкретно ``от'' и ``до''. Представьте  Библию вашу, которая была бы инструкцией  вашей жизни от и до. Читали бы вы эту инструкцию? И много бы вы бы поняли из неё? И много вы могли бы сделать выводов? Теперь представьте, на примере Библии. Она абстрактна. Она абстрактна, и, поэтому, даёт множество направлений думать. И каждый может думать по-своему. И каждый, тогда, чувствует себя индивидуумом. И каждый имеет свой характер, и каждый имеет свой взгляд. У вас есть сейчас инструкция, которая требует только одно. Инструкция, созданная для ``манекенов''. Спрашивайте. (предположительно имелась в виду конституция. Прим.)}
\people{Хорошо. Вы сказали, что мы обманываем себя, но рассуждая по логике вами же сказанного, не мы обманываем себя, а наше сознание обманывает нас?}
\soul{А вы, как сможете понять и разделить себя от сознания? Попробуйте себя отделить от сознания. Кто вы будете? Вы сможете себя найти без сознания? Почему вы молчите? }
\people{Да, это неотделимо от нас. Так скажите, а если переводчик более ласково, с добротой, с любовью подойдет к жене, она что, будет быстрее выздоравливать?}
\soul{Идёт процесс выздоровления. Идёт. И тормозит всего лишь только страхом. Страхом, что не пройдёт болезнь. Злостью, что болеет. Множество, множество причин, которые тормозят. Но, и есть причины, которые дают её вылечить. Здесь, столь тонко, что лучше не будем говорить о том, пока вы в сознании.}
\people{Хорошо.  Меня очень интересуют творческие задатки переводчика.  У него хорошо получаются литературные вещи. Что мешает реализоваться, писать ему, то есть делать этот труд, книгу, вторую книгу? Что мешает?}
\soul{Страх. Страх начать. А начав, появится новый страх, что это будет никому не нужно. Забыв о том, что это, прежде, нужно для себя. Любой писатель пишет сперва, для себя, и, лишь только потом, для читателя. Любой поэт сочиняет только для себя, и потом - для читателя. Если он  сочиняет только для читателя, то это будет ложно и никогда не станет великим. Ибо будет писать не от сердца. }
\people{Вот такой вы совет даёте? А может быть тут его личная лень, несобранность мешает?}
\soul{Мы говорили вам о цепи: страхе и порождающем всё остальное. Да, одним из звеньев является лень. Начало – страх. Страх рождает множество – сомнение  и далее, далее. И в итоге – лень. Что такое лень? Это оправдание. Оправдание себя. А лень всегда любит оправдаться. И вы всегда это делаете успешно: ``А нужно ли это?'', ``А какая цель? А зачем и почему?'', ``В следующий раз'' и далее. И вы называете это ``ленью''. }
\people{Скажите, а вы можете придать уверенность переводчику, или нам в этом разговоре, что у него получится литературное произведение?}
\soul{Поймите, ми говорили вам, о ваших ``контактах'',  и далее. Мы не имеем, мы не имеем никакого права воздействовать на вас. Мы можем только говорить вам. Говорить, причём, говорить вашим языком, вашими словами. Говорить вашему сознанию. Ибо мы не хотим владеть вашими душами. Мы не можем их купить или продать. Но мы не хотим их и испачкать. И поэтому, мы будем говорить с вами сознанием. Душа, если захочет, будет говорить с нами. И она говорит. Но сознание не слышит того. Неужели вы думаете, что сейчас произошёл всего лишь один только контакт? Сейчас.  Неужели вы думаете? Их произошло, минимум - три. Минимум - трое разговаривали с нами. Вы поняли?}
\people{Да нет, не очень.}
\soul{Давайте тогда скажем так: с нами сейчас говорило шесть человек. Вас трое и ``другого''. }
\people{Но, нас двое сейчас.}
\soul{Вот, где же ваша логика? Если мы говорили, о троих, и говорили сразу, о шестерых. Вы не можете поделить? }
\people{Ну, всё правильно, правильно. }
\soul{У вас есть сознание, у вас есть подсознание. Подумайте. Вы произносите вопросы сознанием. А что толкнуло вас задать этот вопрос?}
\people{Понятно, душа.}
\soul{Подсознание. Ибо, подсознательно,  вы  хотели задать вопрос. Сознательно? Сознательно, это, когда вы уже ``вертите'' на языке, если хотите. Вы его формируете в слова и произносите. }
\people{Раз…Две личности мы видим. А третья личность? }
\soul{А где ваша душа?}
\people{А-а… Ясно. Сознание, подсознание и душа. Всё понятно. Хорошо. Спасибо. Мы это, наверно, запомним. Мы хотели бы закончить контакт. Даём обратный счёт. Есть возражения? }
(Конец записи)
Аоум. глава 20-01-96
Георгий Губин
\people{**}
1996-01-20_01
\people{Алфавит, вернее, начали… (Ольга)}
\people{Это не с вами? (Гера)}
\soul{Нет.}
\people{Нет? (Ольга)}
…
\soul{Спрашивайте.}
\people{Мы запутались. (Ольга)}
\people{Скажите, вот идёт смена миров там, это… это с вашей помощью происходило или самопроизвольно? (Гера)}
\soul{Мы же говорили это.}
\people{Я не помню. (Гера)}
\soul{Даже это говорили вам и мы.}
\people{Скажите, вот, вообще-то, когда мы разговариваем, будем говорить так, с разными персонажами как бы, если можно так назвать, происходит смена, вот, вас или как Мабу, его можно назвать? (Ольга)}
\people{Имени не говори. (Гера)}
\people{Сначала, когда мы с ним в первый раз встретились, нам показалось, что уже он ну, где-то… по времени более ближе к нам, с нами разговаривал человек, а потом, он перешёл почти на детский голос. Вот как это объяснить? То есть, он стал с нами разговаривать таким голосом, как будто ребёнок. (Ольга)}
\soul{Неужели вы думаете, что произносятся слова, что слышите вы? - Образы, образы. И как подаёт вам переводчик – это его.}
\people{Это его восприятие, да? Всё ясно. Нет, просто, как-то… (Ольга)}
\soul{Поймите, детьми, вы более правдивы. Детьми, вы меньше лжёте. И даже если говорите неправду, то это всего лишь фантазии, но не больше. Можете вы ребёнка обвинить во лжи? Первая ложь – тогда вы становитесь взрослыми, тогда вы теряете детство. Первая ложь.}
\people{Сознательная ложь, да? (Гера)}
\people{Сознательная ложь? (Ольга)}
\soul{Да, ложью называется только то, что…}
\people{1-2-3..}
\soul{Теперь поиграем в слова. Попробуйте, назовите мне реку, носящую имя.}
\people{А кому это ``вам''? (Ольга)}
\people{Кстати, кто это? (Гера)}
\people{Лена.(название реки в Сибири.прим.)}
\soul{Хорошо, -  Лена. Вы можете эту реку назвать по-другому?}
\people{Можно хоть как назвать. (Гера)}
\soul{Разве? Вы можете назвать её Еленой?}
\people{Нет.}
\soul{Нет. Почему? Почему вы имя, носящее человека, можете назвать, а реку же - нет?}
\people{Я понял вас. То есть, мы трактуем по-своему имя человека, да? (Гера)}
\soul{Представьте - всё, что говорили вам, перенесём на неодушевленное. }
\people{Понятно. (Гера)}
\soul{Тогда, получится, слово ``стул'' будет иметь характер. Стул будет обладать человеческим характером. Возможно ли это?}
\people{Город обладает человеческим характером? Нет? (Гера)}
\soul{Мы говорим о стуле.}
\people{А-а-а, о предмете? ``Стульчик'' можно сказать. (Гера)}
\soul{Прекрасно. Давайте по буквам разберём, относительно человека, характер этого слова. И что получится? Стул будет обладать вашим характером. Вы можете представить эту ситуацию?}
\people{Да нет.}
\soul{Потому, буквы и составляющие их слова должны быть относительно у одушевлённого и неодушевлённого предмета. Никогда нельзя неодушевлённый предмет расшифровывать именем человека. Согласитесь, река Лена, и имя Лена – это не значит одно и то же. В данном случае мы говорим об одушевлённом и нет. Хотя, если быть точнее, река тоже имеет душу, если быть точнее, составляющая душ. Теперь, представьте…}
\people{(счёт)}
\soul{Спрашивайте.}
\people{Так, а на чём мы остановились-то? (Гера)}
\people{Река Лена имеет душу. (Ольга)}
\people{Нет, нет…это не те… (Гера)}
\people{Мы путаемся, ритм один и…(Ольга)}
\people{Вы там что-то хотели сказать и запнулись, мы это не поняли. (Гера)}
\soul{Спрашивайте. Ваши вопросы – это наша пища, и не задавая их, мы не будем отвечать вам. Вы же используя машины, должны произнести вопросы вслух, иначе не будет понято, а вы уже стали…}
\people{(счёт)}
\soul{Мы говорили вам о подготовке. Мы говорили вам, что вы должны быть свободны, свободны от всего. Вы же…}
\people{(счёт)}
\soul{Вася, Вась…}
\people{Не Вася. (Ольга)}
\soul{Вась.}
\people{Кто Вася-то? (Ольга)}
\soul{Вась.}
\people{Ну, чего? Ты меня слышишь? (Ольга)}
\people{(счёт)}
\soul{Несчастный из вас…}
\people{Несчастный? (Ольга)}
\people{Несчастней? (Гера)}
\people{Кто несчастней? Или несчастный? (Ольга)}
\soul{Несчастный из вас.}
\people{Я бы не сказала про себя такого.}
\people{Сказать, произнести что ли? (Ольга)}
\soul{Почему же? Почему мы видим темноту? Почему мы видим печаль?}
\people{Темноту? Печаль? А кто вы? (Ольга)}
\people{Вы - краски? (Гера)}
\people{Вы кто? (Ольга)}
\people{Назовите себя.}
\soul{Представьте ситуацию, что мы назовёмся чужим именем, допустим, назовёмся одной из красок. Что сделаете вы? Поймёте ли вы это?}
\people{А вы уже нас… краски… с нами, так сказать, контактировали. (Ольга)}
\soul{Раньше, вам было легче.}
\people{Чем? (Ольга)}
\soul{Раньше, вы могли определить по жестам.}
\people{Да. (Ольга)}
\soul{Теперь, вы не можете этого сделать. Это одно из условий, условий спектакля, в который вы играете.}
\people{Ах, вон как. (Ольга)}
\soul{Вы должны решить множество задач, и вы должны научиться определять каждого. Рано или поздно придёт чужой, и он будет играть с вами, и вы должны будете найти его, вы должны будете назвать его пятым, хотя вы ищите и не можете найти четвёртого и даже третьего. Он же придёт и назовётся пятым, и вы должны будете ему поверить. Уж он будет говорить то, что знаете только вы, и тогда вы будете уверены более в других…}
\people{Чем себе? (Ольга)}
\soul{Поймите, вы не видите правды, вы слышите только то, что хотите. Вам легче добраться, если сказать вам с откровением, тогда вы будете верить. Когда придут к вам друзья и будут называть вас и ругать вас, вы скажите ``враг'' и ``чёрный'', ибо вы любите, когда вас ласкают, когда с вами играют, когда вам подают. Когда же говорят вам правду, вы боитесь, вы храните её в себе, и даже себе лжёте. Почему? Потому что вы артист, вы актёр, вы хотите играть и не хотите жить…}
\people{(счёт)}
\soul{… чтобы различать, одушевлённые и нет. Можете вы назвать одушевлённый предмет?}
\people{Всё, что двигается, наверное, одушевлённый. (Гера)}
\soul{Разве?}
\people{Человек? Если человека можно одушевлённым предметом… ой,- предметом… (Ольга)}
\soul{Давайте вернёмся к русскому языку. Можете ли вы сказать, дерево – одушевлённое или нет?}
\people{Да. (Ольга)}
\people{Оно растёт. (Гера)}
\soul{По вашим правилам, как вы напишете ``дерево'', с большой буквы или с малой?}
\people{С маленькой.}
\soul{С маленькой. По вашим правилам – неодушевлённый. Прекрасно. Давайте сыграем опять в слова.}
\people{(счёт)}
____
\soul{Подглядывал. (Детским голосом)}
\people{Куда, Мабу, подглядывал? (Гера)}
\soul{Не скажу.}
\people{Скажи.}
\people{Ты, наверное, в пещеру подглядывал, да? (Гера)}
\soul{Нет, что вы делали.}
\people{Мы? Что мы делали? А что мы делали? (Ольга)}
\soul{А вы… Я хотел сперва обидеться, думал, вы не только со мной говорите, а потом, стало интересно, только почему-то больно.}
\people{Почему? Где больно? (Ольга)}
\soul{Не знаю.}
\people{Ну, просто больно где, внутри? Где больно? Обидно? (Ольга)}
\soul{Нет.}
\people{А как? (Ольга)}
\soul{Не знаю.}
\people{А куда ты подглядывал? (Гера)}
\soul{Что вы делали.}
\people{Когда ты подглядывал? (Ольга)}
\soul{Сейчас.}
\people{А-а, и чё мы делали? Ну, давай, рассказывай. (Ольга)}
\soul{Вы с кем-то говорили.}
\people{С кем, не видел? (Гера)}
\soul{Нет, не видел.}
\people{А у нас есть кто-то новый, среди нас, кого ты ещё не видел? (Ольга)}
\soul{Я вижу только животики и головочки, а больше не умею рисовать.}
\people{А ты считать-то умеешь? (Ольга)}
\soul{Чего?}
\people{Считать. (Ольга)}
\soul{Умею.}
\people{До скольки? Сколько у тебя жён, ну-ка посчитай. (Ольга)}
\soul{У меня?}
\people{Да. (Ольга)}
\soul{Три.}
\people{Три? (Ольга)}
\soul{Три и одна.}
\people{Ааа. Три, три и одна. Ты знаешь как: три… –  потом как? Не знаешь дальше, как считать? (Ольга)}
\soul{Три.}
\people{Потом, четыре, потом, пять, потом, шесть, потом, семь. У тебя семь жён, да? (Ольга)}
\soul{Не-е.}
\people{Почему ``не-е''. (Ольга)}
\soul{Три, три и одна.}
\people{Ну, ладно, три, три и одна. А нас сколько? Ну-ка! Сколько нас, посчитай, сколько у нас головок? (Ольга)}
\soul{Три.}
\people{Три и…?(Ольга)}
\soul{…и две.}
\people{Три и две? А где две-то? Ну-ка, сколько жён, сколько здесь жён? (Ольга)}
\soul{Нет жён.}
\people{Да не твоих жён, Петиных жён, Петиных. Сколько Петиных жён? (Ольга)}
\soul{Две.}
\people{А сколько…  мужчин, ну, как… мужчин? Кроме Пети ещё кто есть? (Ольга)}
\soul{Две его жены.}
\people{И всё? Больше никого нет? (Ольга)}
\soul{Не-е.}
\people{Ну, ты, ты неправильно посчитал. (Ольга)}
\soul{Почему?}
\people{Ну, головок… Посмотри, сколько у нас головок? Ну-ка, посчитай головки. Сколько? Ещё, ещё раз посчитай внимательно. (Ольга)}
\soul{Три и две.}
\people{А кого ты ещё видишь? Ты кого-то видишь? Лежит кто-нибудь? Лежит.  Кто лежит? (Ольга)}
\soul{Нет.}
\people{Не видишь? Все сидим? (Ольга)}
\soul{Да.}
\people{Ну, ты рассказывай, куда ты ходил, где тебе… А! Ну, ты, значит, за нами подглядывал, а ещё за кем ты подглядываешь? Ну-ка, рассказывай. (Ольга)}
\soul{Ещё я ничего не видел.}
\people{Нет, ну, еще ты за кем-то подглядываешь тоже? (Ольга)}
\soul{Не-е-е, это я один раз попробовал.}
\people{Только за нами, да? (Ольга)}
\soul{Неее, один раз.}
\people{За нами подглядывал? (Ольга)}
\soul{А я теперь всегда буду подглядывать.}
(Смех.)
\people{Подглядывай, ладно, мы разрешаем тебе, если тебе интересно. А огонь у нас есть здесь? Огонь есть в пещере в нашей? Ты видишь огонь? (Ольга)}
\soul{Не-а.}
\people{Не видишь? Темно у нас в пещере, а? (Ольга)}
\soul{Не-а.}
\people{Темно? А где же огонь у нас? (Ольга)}
\soul{Не знаю.}
\people{Мабу, а ты же долгое эхо видел в священной пещере? (Гера)}
\soul{Кого?}
\people{Эхо, эхо. (Ольга)}
\people{Ходил в священную пещеру?}
\people{Водили тебя монахи? (Гера)}
\soul{Водили.}
\people{Ну, и долгое-долгое эхо слышал там? (Гера)}
\people{Он уже старейшина. Ты давно старейшина? (Ольга)}
\soul{Чего?}
\people{Ну, старейшиной когда тебя сделали? Старейшиной. (Ольга)}
\soul{Не-е-е.}
\people{Да у тебя уже семь жён.}
\soul{Ну и чё?}
\people{Ну, а как? ``Старейшина'' ты ж сказал? (Ольга)}
\soul{Ещё, сказали, рано.}
\people{Рано ещё? Но, а с монахами ты часто встречаешься? (Ольга)}
\soul{Часто.}
\people{Ну, и чему ты ещё научился у них? (Ольга)}
\soul{Они меня учили глазки рисовать.}
\people{Глазки? У тебя получается? (Ольга)}
\soul{Получится – будет старейшина!}
\people{А-а, всё ясно. Значит, когда если глазки умеешь рисовать, тогда, значит, будешь старейшина. Если нет, не научишься, значит, не будешь, да? (Ольга)}
\soul{Не буду.}
\people{Вася, слышишь меня? (Гера)}
\people{Какой Вася? Мабу.}
\people{Мабу. (Ольга)}
\soul{Мабу!}
\people{Ну, Мабу. Ты же был Васей. (Гера)}
\people{А теперь он - Мабу. (Ольга)}
\people{А я Петя. Ты меня помнишь? (Гера)}
\soul{Помню.}
\people{Я тебя просил монахам передать, чтобы они поговорили с нами через тебя. Они могут это? (Гера)}
\soul{Чего?}
\people{Чтобы монахи нам вопросы какие-нибудь задали или что-нибудь сказали, передали нам привет. (Ольга)}
\soul{Вы ничего не просили.}
\people{Просили. Ты забыл, наверное. (Гера)}
\soul{Нет, я помню.}
\people{Ну, вот мы тебя сейчас просим, чтобы монахи как-нибудь это… что-нибудь сказали. (Гера)}
___
\people{(счёт)}
\soul{Назовите букву.}
\people{``Р'', мы остановились. (Гера)}
\soul{Имеет ли различие ``Р'' прописная и заглавная?}
\people{Конечно, имеет. (Гера)}
\soul{В чем?}
\people{``Р'' заглавная замкнутая, а ``р'' прописная – с хвостиком. (Ольга)}
\soul{Давайте, скажем, есть ли общий элемент?}
\people{Есть. Прямая. (Гера)}
\soul{Прямая.}
\people{Еденица. (Ольга)}
\soul{И находится на левой стороне.}
\people{Да. (Гера)}
\soul{Как принято у вас: лево и право? Право –  вы считаете, это правда, и левое – ложь?}
\people{В принципе, да.}
\soul{Это значит, пришло время, когда вы лжёте, когда вы можете лгать. Замкнутый круг – это когда вы лжёте, в первую очередь, самому себе. Вы делаете это всегда, но замкнутый круг говорит о том, что ложь бьёт вас, вы ещё не умеете достаточно хорошо лгать. Возьмём прописную букву. Здесь вы видите уже волнистую линию. Договорились, черта – это ложь, ибо она находится с левой стороны. Теперь попробуйте, определите, что такое волна?}
\people{Это ложь, направленная вовне? (Гера)}
\soul{Мы вам подсказываем: волна.}
\people{Волна… волны… Это - распространение? (Ольга)}
\soul{Нет. Это значит, что вы ещё не имеете постоянного, вы – маятник. Вы волнуетесь о каждом пустяке, волнует вас всё, у вас ещё живое восприятие мира. Чтобы не было разногласий, давайте договоримся, что мы говорим об алфавите, относящегося… к возрасту человека. Если вы помните, мы начинали с нуля, начинали с младенца. Когда мы будем говорить, и подставлять заново алфавит, то это уже будет другая тема. Мы говорили вам о множестве значений. Сейчас же мы рассматриваем ваши возрастные изменения. Давайте далее. Круг, он замкнут. Вспомните, мы говорили вам о цифрах. Вы помните, мы говорили вам о ``нуле''?}
\people{Да.}
\soul{Мы говорили вам о ``девяти''.}
\people{Да.}
\soul{И о ``шести''. И что обозначает замкнутый круг? И вспомните, что он, всё-таки, находится наверху.}
\people{Это буква ``Р''? (Ольга)}
\soul{Верх – это сознание. Сознание, выросшее до той степени, что уже преобладает над чувствами. Сознание заставляет вас лгать, сознание бьёт вас, сознание – враг ваш. Иногда, вы это называете ``переходным возрастом'', когда вы не можете сознанием понять окружающее и мечетесь, мечетесь и чувствуете себя ``маленьким человечком'', ненужным. Тогда происходят проблемы с родителями, с родственниками, вы жалуетесь на их невнимание, на недостаточность внимания. Вы согласны?}
\people{Да.}
\soul{И тогда, вас уже смело можно обозначить буквой прописной, малой буквой. Давайте следующую.}
\people{Скажите, можно вопрос задать? Может, поэтому дети очень в основном ``Р'' поздно так начинают говорить, из-за этого, нет? (Ольга)}
\soul{Да, в этом тоже есть элемент. Тогда вы должны были бы вспомнить и другие буквы.}
\people{Ну, мы ещё до них не дошли, в основном это шипящие. Дети редко выговаривают их . ``Л'' тоже. Дальше - ``С''. (Ольга)}
\soul{Есть ли разница между прописной?}
\people{Никакой. (Гера)}
\soul{Никакой, только размер. Давайте возьмём заглавную букву. И давайте, всё-таки, присмотримся, как вы будете писать эти буквы. Напишите большую и напишите малую, и найдите различия в них.}
\people{Большая более округлая, наверное. (Гера)}
\soul{Нет. Большая выделяется среди остальных. }
\people{Естественно. (Гера)}
\soul{``С'' прописная всегда будет малой, и всегда будет иметь тот же размер, окружающий букву. Почему тогда имя, носящее ``С'', или имя, носящее ``С'' в середине, будет различно? Почему?}
\people{Преобладание каких-то качеств, наверное. (Гера)}
\people{От окружающих тоже зависит. (Ольга)}
\people{По большей или меньшей степени. (Гера)}
\people{Если начинается с буквы ``С'', значит, выходит, что сам человек, даже сам начинает… ``Самость”- с буквы ``С'' начитается, так? А если в середине, то здесь уже более, наверное, человек прислушивается к другим, больше, чем к себе. (Ольга)}
\soul{Да, но вы забываете, что это не замкнутый круг.}
\people{Может, стремление замкнуть? (Гера)}
\soul{Нет, это стремление взять, взять как можно больше себе. Это стремление, когда вы хотите научиться жить, это стремление приспособиться к окружающей среде, и потому - вы открыты, потому - вы более ранимы – вот вам ``переходной возраст''. Но это уже не начало. Здесь вы уже не лжёте столь часто, как делали раньше, здесь вы стараетесь поставить свое ``Я'' выше других, вы уже хотите, чтобы с вашим мнением считались, вы хотите завоевать весь мир, вы хотите, чтоб мир знал вас. В эти моменты вы представляете о своей смерти, и о той печали, что принесёт она не вам, а другим. Здесь вы проживёте множество вариантов своего будущего, здесь вы мечтаете, здесь мечты ваши дают корни в жизнь, здесь выбираете профессию, здесь, а не когда в садике говорите, что вы будете космонавтом. Возраст выбора. Возраст, когда…}
\people{(счёт)}
\people{Что? (Ольга)}
\soul{Имеет зелёный цвет, но ауру, чаще всего, красную, ауру жёстокости, ауру защиты. Почему? Потому, что растения беззащитны, она не может защититься, она издаёт лучи, она кричит от боли, но она не может уйти. Животное может дать сдачу, животное может убежать. Человек может запомнить и, рано или поздно, отомстить – растение должно терпеть. Растение имеет красный цвет. Почему? Потому, что ей больно, больно не только за себя, но и за всех. Она издает крик, и растения, что услышат их - лес – преобладает багровый цвет, цвет агрессии. И когда вы топчете траву или ломаете деревья, и тогда вы чувствуете, чувствуете усталость, чувствуете боль, вы не можете понять и объясняете это причиной усталости: усталость от того, что вы рвали или ломали. А теперь представьте: вы идёте в лес отдыхать, и вы отдыхаете, нет той усталости, даже на той же опушке, где вы уже ломали. И представьте, когда вы идёте работать. В вашем понятии - уничтожать – это работать. Вы приходите и рубите, рубите, не разбираясь. Вы берёте и рвёте цветы любимым и дарите им отрезанные ножки и ручки, вы дарите им уродцев. Роза имеет колючки, чтобы защититься, они бессильны, они бессильны. Роза – самый активный цветок, который может более или менее отомстить вам, потому у вас есть приметы, приметы цвета роз. Вы же, собрав букет, приходите и приносите в радость в дом? Нет… Вы приносите множество боли и множество смертей. Ничто не умирает сразу. Если человек уходит, он уходит долго, он уходит первые три дня, он уходит первые девять дней, он уходит первые сорок, потом - год. И уже, если не вспомнить о нём, он теряется. Он уходит туда, где уж нет возврата, он уходит, откуда нельзя вернуться.  (…) рождается здесь, и тогда вы удивляетесь, откуда у ребёнка такой характер. Вы даже не можете понять, что, когда-то, вы его забыли проводить, и когда-то, на могилу вы принесли ему срезанные цветы – символ смерти. Именно тогда, когда он хотел от вас помощи и поддержки. Теперь вы делаете хуже, вы приносите металл и пластик. Живое, пусть уже убитое, вы заменили металлом, заведомо не имеющим никогда жизнь. Этим металлом вы оковываете его, не даёте выйти ему и уйти с земли. Металл тяжёл, пластик ложен. Что делаете вы? Вы стараетесь, стараетесь обмануть себя и окружающих, вы переживаете: умер – и в себе ищете и давите жалость, вы выдавливаете слёзы, чтобы…}
\people{Один…(счёт)}
\soul{…``Я верую, Господи!'' – Верую или желание верить? Видел ли кто из вас Господа Бога? Многие могут сказать: ``Да, даже разговаривал с ним!''. Не здесь ли вы осудили его, не здесь ли вы поставили выше других? Вы приходите в церковь и молитесь, вы ставите свечку, убирая другую, потому что нет больше места. И хотя вы знаете, что свеча – это чья-то жизнь, вы её убираете. Что свеча – это поп…}
\people{Раз, два…(счёт)}
\soul{Я слышу их.}
\people{Анур? Мы давно не виделись. Сколько времени прошло? (Ольга)}
\soul{Я не знаю. Ну, года два.}
\people{А они - рядом с тобой? (Гера)}
\soul{Нет.}
\people{Ты уже один? (Гера)}
\soul{Один.}
\people{А куда он делся - Анис? (Гера)}
\soul{Он на службе.}
\people{Скажи, мы так не поняли, он приближенный или сам жрец? (Гера)}
\soul{Нет, он просто один из жрецов, и он приходил и лечил меня, он научил меня лепить горшки, он вывел меня в люди, он…}
\people{Один, два…(счёт)}
\soul{Время, когда я верил этому шарику, я всегда держал его, и мне казалось, что он даёт мне тепло. Но однажды он был потерян, и мне пришлось идти за новым. Но что-то я не ощутил большой разницы, есть он на мне или нет. Я не вижу на вас шаров. Вы их не признаёте?}
\people{Нет.}
\soul{Или вы тоже потеряли их?}
\people{Нет, у нас это не принято. (Ольга)}
\soul{Не принято.}
\people{Мы все считаемся людьми, у нас нет такого различия. Мы все люди. (Ольга)}
\soul{Люди?}
\people{Да. (Ольга)}
\soul{А кто мы?}
\people{Вы - тоже люди. (Гера)}
\soul{Мы - тоже люди?}
\people{Да. (Ольга)}
\soul{Мы теряем это имя, когда мы носим…}
\people{Один…(счёт)}
\soul{… пробовал со мной заниматься, но уже почему-то не получалось. Он говорил, что это связано с болезнью, а точнее, что я просто вылечился. Так зачем мне тогда нужен этот шар?}
\people{Какой шар? (Гера)}
\soul{…прекрасно обойтись и без него.}
\people{Можешь. (Ольга)}
\soul{Единственный, кого я признаю, это сына, сына…}
\people{Один…(счёт)}
\soul{Сынок, не делай этого, не делай.}
\people{Восемь, девять…(счёт)}
\soul{Гора хочет говорить с вами, он не верит мне. Он говорит, что я вру.}
\people{Вру. (Ольга)}
\soul{Шаманы не могут лгать. И он хочет, чтобы вы говорили с ним.}
\people{Хорошо. (Гера)}
\people{Гор? (Ольга)}
\people{Это - Гор? (Гера)}
\soul{Гора хочет слышать вас. Что я должен сделать, чтобы вы слышали его?}
\people{Гора – это вождь? (Ольга)}
\soul{Да.}
\people{А твое имя? (Ольга)}
\soul{Это значит ``большой''.}
\people{ГорА? (Ольга)}
\soul{ГОра!}
\people{Ну, да. (Ольга)}
\people{Что сделать, чтобы он слышал нас, или мы его? (Гера)}
\soul{Чтобы ему говорить с вами.}
\people{Пусть задаёт вопросы, мы будем отвечать. (Ольга)}
\soul{Берки приходили к нам…}
\people{Да. (Ольга)}
\soul{Но соседнее племя не хочет, не хочет с нами…}
\people{Помогать вам? }
\people{Общаться? (Ольга)}
\soul{И берки снова одерживают победу, и мы снова ушли. Мы же не успели посеять, но мы уже растеряли зерно, пока уходили, и скоро мы будем умирать с голода. И потому, ГОра не хочет, не хочет верить мне, я не могу привести ему какие-то доказательства. Сора же, постоянно науськивает его.}
\people{А кто такой Сора? (Ольга)}
\soul{Это его приближённый, и он постоянно говорит против меня, и ГОра больше верит Соре, чем мне.}
\people{Ну, значит, что ты… объясни так ему. Вот руку покажи, пять пальцев, да? Вот, потом сожми в кулак, и пусть он попробует его разжать. У тебя руки крепкие? (Гера)}
\soul{Достаточно, чтобы держать оружие.}
\people{Пальцы, по одному  - слабее, чем вместе. (Ольга)}
\people{Да, пальцы ломают по одному. (Гера)}
\people{Это, во-первых, - кулак сильнее, чем когда пальцы по отдельности. Ты это ему объясни, нужно объединяться. Только объединившись, люди могут выйти из беды, могут себя спасти. (Ольга)}
\soul{Но они ждут знак. Какой знак я им дам?}
\people{Ну, скажи, чтобы он… Он сейчас рядом с тобой - Гора? (Ольга)}
\soul{Да. И Сора тоже.}
\people{И Сора? Скажи, пусть он задаст нам вопрос. (Ольга)}
\people{Ты слышишь его сейчас? (Гера)}
\soul{Он хочет, чтобы сам слышал. Сделайте это.}
\people{Не можем мы так. (Гера)}
\people{Он, у него нет ушей, как у тебя, он не может слышать? (Ольга)}
\soul{Мы одинаковы с ним.}
\people{Да, внешне. (Гера)}
\people{Только внешне. А у него нет способности слышать нас, как у тебя. Этим обладают очень редкие люди, объясни ему это, что не каждый может слышать. (Ольга)}
\people{Мы тоже сидим так же … (Гера)}
\soul{Сора не поверит, и не даст это сделать ГОре. Дайте, дайте какой-нибудь знак.}
\people{Один…}
(счёт)
\soul{Говорите следующую букву.}
\people{``Т''}
\soul{Часть вы можете сказать о кресте. }
\people{Не законченный.}
\soul{О кресте – не законченный ли он?}
\people{Раньше эта ``тау'' называлась буква. (Ольга)}
\soul{Хорошо, давайте попробуем другие языки. Давайте возьмём ``Т'' в латинском варианте.}
\people{Такая же. (Ольга)}
\people{Всё правильно. (Гера)}
\soul{Прекрасно. Давайте возьмём любой другой вариант.}
\people{А какой у нас? (Гера)}
\people{Английский, если взять вариант? ``Т'' – время, то же самое, только немножко, там, хвостики…(Ольга)}
\soul{В прописном.}
\people{Да. (Ольга)}
\soul{Давайте по-другому. Черта находится посередине, её пересекает линия. Здесь есть время выбора, быть вам земным или увидеть выше. Та перекладина и та стена, тот потолок, что не позволяет увидеть вам небо, это время, когда вы разочаровываетесь в своих мечтах, вы видите свои неудачи и вы уже не верите своим фантазиям, вы теряете способность фантазировать. Детьми вы могли придумывать многое и безостановочно, теперь же, став взрослым, вы уже не можете сочинять ``просто так''. Вы обязательно будете делать поправки на жизненные ситуации, вы обязательно будете вставлять технические термины, которые уже знаете. Согласны?}
\people{Да.}
\soul{Иначе - вы уже стали применять ``научный мат''.}
\people{Научный  мат? (Ольга)}
\people{Мат, в смысле, плохое? (Гера)}
\soul{Да. Вы уже знаете слишком много слов, вы уже знаете слишком много терминов, вы уже знаете, что можно и что нет. И та перекладина является количеством ваших знаний и вашей же границы, возможности. Вы уже знаете, что вы не умеете и не можете летать, и вы уже не мечтаете о том, что когда-то научитесь этому. Вы уже знаете, что вы не можете плавать и увидеть подводный мир. Раньше же, в детстве, каждый из вас всегда интересовался, чем же дышат рыбки и почему они молчат. Теперь же вы знаете, что рыбы не молчат, они тоже разговаривают, но по-своему, вы уже знаете термин ``ультразвук'' и ``инфразвук'', и вы это уже стараетесь применить в свою пользу, в свою, и слава богу, что вы не знаете, как это сделать. Вот вам буква ``Т''. Середина же обозначает, что вы уже осваиваетесь в этом мире, вы уже знаете себе место и стараетесь закрепиться в нём. Вы уже начинаете…}
\people{Один, два, три…(счёт)}
\people{Каждый из нас? (Гера)}
\soul{…окружает множество, множество ``черни'' и ``светлого''. Чернь приходит извне и рождается из вас, это когда вы желаете что-то кому-то, когда вы хотите отомстить, когда память ваша становится злобной, это тогда, когда вы согласны заключить договор с кем угодно, лишь бы получилось, это тогда, когда вы забываете о совести и о любви. Светлое? Светлое – это в первую очередь ваша совесть, это вспоминание о нём, это любовь, она пока что затаена в вас, пока не проснулась в вас, но ждёт своего часа. Чёрное же - ставит запоры. Запоры очень просты, у вас нет времени или нет желания, вам скучно или неинтересно, вам страшно или слишком всё понятно. Схватив только верхушечку, вы уже говорите: ``Я знаю! Зачем мне дальше?'' – и бросаете. И тогда вы уже знаете множество, множество вещей, вы уже знаете весь мир, и плаваете только по верхушкам, вы даже не собираетесь входить в глубину. Приходит время, когда эти верхушки столь многочисленны, что вы уже можете назвать себя избранным, и тогда вы уже начинаете играть в ``спасителя мира'', вы тогда уже начинаете применять свои верхушки для лечения других, не замечая, что вы сами больны. Тогда, вы называете себя ``экстра'', даже не зная о том, что ``экстра'' – это просто ``лучше'', но не ``больше''. Каждый из вас обладает всем этим, каждый из вас - всего лишь проводник космических линий. Всего лишь. И чем больше ваше сопротивление, тем хуже вы проводите. Но если чуть больше вы стали проводить, чем остальные, и заметив эту разницу, вы тут же возбуждаете в себе гордость, а гордость слепа и глупа. Она не дает увидеть вам окружающее, она не дает увидеть истинное отношение к вам, и вы создаёте кривое зеркало, в котором отражаетесь только вы и больше никто, всё остальное для вас неверно: одни лишь вы только красивы, одни лишь вы только сильны. Вы ищите себе друга жизни. Как вы ищете его? Как? -Подобно себе. И если находите, то разочаровываетесь – вам не нравится, что он похож на вас, а если быть точнее, вы замечаете в нем множество, множество грехов, именно тех, которые есть у вас. И если вы жадны, то и друг ваш окажется жадным. Он повторяет вашу жадность, но вы не заметите, что вы жадны, вы заметите, что он жаден, в первую очередь. И всё плохое есть у вас, всё, что замечаете плохое - есть у вас, потому, потому так и резко вы видите это. Удивительно, что если есть в вас что-то хорошее, то оно почему-то не увеличивается, а уменьшается в друге вашем. Заметьте, ``хорошими делами прославиться нельзя''. Почему? Потому, что доброе вы быстро забываете: всю жизнь человек вам будет делать доброе, но если он один раз скажет ``нет'', вы это запомните гораздо больше, чем остальную доброту. И для вас этот человек уже будет плохим, вы уже не вспомните, что он вас когда-то спасал, когда-то вам помогал, вы это уже не вспомните – он уже отказал вам, - он уже предал вас. Каждый судит в меру своих способностей. Вы знаете это, но только не приравниваете к себе. Почему? Эгоизм? Нет, не только. Это жадность, жадность. Жадность в этом мире – есть всё: это жадность ``во славе''. Вы хотите быть замечаемым более, чем другие. Прекрасно! Вы остаетесь в этим другом и находите в себе противоположное – и уживаетесь. Удивительно, с противоположным вы уживаетесь более, чем с подобным себе. Почему? Уж если вы жадны, противоположность - что?  - Доброта. И вам нравится доброта, вы с жадностью хватаете всё, что он вам даёт, говорите: ``Каков прекрасный человек!'' – и про себя: ``Какой дурак! Вот, идиот!'' - Это же ваше, ваше! Вы же не будете делать так, и вам этот ``идиотизм'' нравится, потому что он выгоден вам. Почему, почему? Да потому, что вы не хотите, не хотите видеть истинное, вам больше нравится кривое зеркало, вам больше нравится автопортрет больше, больше. И когда на вас рисуют карикатуры, вы обижены. Вы любите шутить, вы любите шутить над другими, но когда та же самая шутка будет относительно к вам, вы тут же обидитесь и будете обвинять в глупости. Почему? Уж вы не хотите заметить глупость свою. Можно подумать, что мы ненавидим вас. Хорошо, давайте будем хвалить, чтобы как-то оправдаться. Да, вы плохи, но удивительные качества в том, что иногда ваше плохое является благом, благом для многих. Благо – когда вы со злостью говорите кому-то правду, он уже знает, что это правда. Уж когда вы это говорили от доброты, это могло бы быть и ложью. Почему вы верите больше человеку, говорящему во гневе, чем когда он обращается с вами ``по душам''. Почему? Потому, что вы меряете и меряете самим собою. Мы хотели вас похвалить и не получается. Давайте попробуем ещё раз. Да, у вас есть души. Нельзя говорить об их размерах, но можно сказать о их объмах, можно сказать о том, что вмещают они. Да, это сосуд, огромный сосуд, и в этом сосуде есть много хлама, но есть и хорошее,  с жадностью вы хватаете это всё. И опять мы вас не хвалим. Мы не можем хвалить вас, не из-за того, что мы ненавидим вас, а из-за того, что хотим остановить вас. Друг – тот, кто может сказать вам правду, даже если она плоха. Мы же не хотим быть подхалимами вам. Очень легко похвалить вас, очень легко. Уж и действительно есть в вас много прекрасного. Да, вы ошибаетесь, но это прекрасно, что вы делаете ошибки, и ещё прекрасней, когда вы замечаете их и исправляете. Это прекрасно, когда вы можете сказать: ``Да, я ошибся!'' – и признаёте ошибки те не только в себе, но и другим. Почему вам нужно покаяние? Почему? Чтобы открыть силы, силы признаться другим, что вы…}
Конец 1-й записи 
1996.01.20-02 
(2757 год)
\soul{Жёлтое солнце так пекло, что Ануру уже стало тяжело держать и терпеть его. И тогда он решил найти тень, но больная нога не давала покоя. Ему нужно было быть у ручья, чтобы опустить ногу, успокоить ноющую боль. Но у ручья не было деревьев, ему приходилось выбирать: или терпеть, или уйти. Бон, он где-то был вдалеке. Анур старался позвать его, но Бон был так занят сбором каких-то трав, что было не до него. Молодой лекарь ушёл в деревню за лекарствами, вот уже вторые сутки его не было здесь,  Анур же волновался, не сбежал ли снова он, скотина. Бон успокаивал: ``Куда ему уйти от нас! У него же даже нет  жило…(сбой)}
(время неизвестно.)
\people{(Гера) О ком? }
\soul{Вспомните. Об этом я не мог узнать в газетах.}
\people{(Гера) О ком!? }
\soul{Те слова, что слышали вы. Вы помните их? ``Придите и возьмите его, но оставьте тело'' – вы помните эти слова? Вы никогда и нигде не произносили ни журналистам, и ни даже своей жене. Об этом знают только вы и я. Подумайте, о ком я знаю…(сбой)}
\people{(Ольга) Один… }
(времена  шамана Ора)
\soul{Да, берки, действительно, были похожи на стадо голодных волков: их лохматые тела пугали, и они с криками и воплями бросались на первую попавшую деревню и уничтожали всё, что попадалось. Им не нужно было зерно, и они разметали его…(сбой)}
\people{(Ольга) Один…}
(неизвестное время из прошлого)
\soul{Я видел вашу звезду, я видел, как она упала. Я даже могу показать это место, и может быть, найдём и попробуем поднять её, господин? Не это ли является свидет…(сбой)}
\people{(Ольга) Шесть… }
\soul{Приходится с ними идти…}
\people{(Ольга) Один… }
\soul{Встало. Солнце…(сбой)}
\people{(Ольга) Один… }
(1248-й год)
\soul{Больно… Больно за него, потому, что он смог, смог нанести мне этот удар. Я пыталась, я пыталась крикнуть ему, но он уже не слышал меня. Я уже ушла из этого мира, но ещё не пришла туда. И я ещё могла видеть его и кричала ему, но он не слышал меня. Я видела его слёзы, но и видела страх.  Он боялся за себя. Он боялся за себя, не за меня. Он ненавидел меня за то, что я родила его, что я кормила его. За что? Я могла бы это стерпеть.  Я могла бы стерпеть удар ножа, но я не могу терпеть, когда ненавидят. Ненавидит сын, только за то, что я родила его, только за то, что я кормила его, только за то, что я не знаю его отца… В чём вина моя? В чём? Почему мне никогда не везло? Почему, только родилась – и уже была продана, как какая-то вещь? Почему, почему названная мать учила меня и любила меня больше, чем родная? Почему же родная мать всегда обходила меня стороной? И почему же она… она прокляла меня? Почему? (Обрыв записи)}
….Плот собирался всей деревней. Кто-то приносил брёвна, кто-то вязал верёвки. Виновный всего же этого был привязан к дереву, и уже тело его было в рубцах, в зубцах жертвенного ножа: того самого ножа, что убил его мать. И плот уже был спущен на воду, и уже был изготовлен костёр. И каждый, каждый оспаривал право зажечь его первым. Вся деревня была против. Вся деревня хотела увидеть, как горит человеческая плоть. Каждый хотел увидеть мучение и каждый проклинал его за болезнь, каждый считал, что только он и его мать виноваты. Деревня, почти уже вымершая, почти уже пустая, всё-таки гудела. Гудела ненавистью, гудела проклятьями к молодому парню. Ему же уже было всё равно. Он не мог вернуть мать, он не мог вернуть себя, он жаждал только, лишь бы быстрее… Он молил богов, чтобы огонь съел его сразу, и не мучил его…
\people{(Ольга) Один… }
(2757-й год)
\soul{…Сердце зубатого Ануру показалось странным. Сердце должно быть горячим, но оно было холодным. И хотя оно истекало кровью, сердце всё-таки было холодным. Что-то было не то. Анур не мог понять, что это, но он знал, что это что-то неживое, и оно не могло дать жизнь этой птице. И Ануру стало страшно. Зачем, зачем жильцам понадобилось оно, зачем он должен был отнести его и за это получить ``жило''… Лишь только за это? Когда многие должны были пройти множество обрядов, чтобы вернуть, вернуть этот шарик. А здесь - всего лишь только сердце. Может быть это, может быть что-то другое, но оно пугало Анура, пугало, и он стоял в растерянности и не знал, что делать с ним. Да, он добыл его, и ему было тяжело. Да, где-то там, на площади, лежал его сын, и этот сын отдал свою жизнь за этот кусочек непонятно чего. Сердце – не сердце, камень – не камень… не поймешь. И эти муравьи, что они говорили там, что? Анур пытался вспомнить, но не мог сделать этого, он понял только одно, что если он доберётся до пустоши, сын вернётся. Он понимал, что это колдовство, и что жильцы должны будут наказать его. Но вы же никому не скажете этого? Они не будут знать об этом. Зато у них будет сын, сын уже никогда не расскажет того. Ему лишь бы был вернут сын, лишь бы сын был живой. Сара… Она кричала. Она кричала, и этот крик испугал его. Он никогда не слышал, никогда не слышал подобных криков. Он видел, как умирают охотники в лапах зверя, он видел, как на кострах очищения сгорали женщины, когда жрецы плясали вокруг, а она кричала. Но той боли, того крика он не слышал. }
(1912-й Сергей Иванов в пустыне)
…Он умирал среди песков, и он уже стал слышать голоса. Он испугался: - Это, наверное, и есть начало смерти. Но почему я их вижу  множество, почему такие разные? Или просто я разговариваю с самим собой? Но я же не могу говорить  женщиной… И причём не одна… Почему? Почему мне говорят о столь далеком будущем, где меня уже не будет? Как это может быть? Как? А где буду тогда я? Я сейчас умру, а они говорят ``нет''. Может быть, тогда правда, может быть, всё-таки, кто-то найдёт меня сейчас, и я буду живой? Но эти голоса…Я боюсь их и хочу услышать… Почему, почему человек умирает так медленно? Почему, почему не умереть сразу? Почему мне надо обязательно мучиться? Почему, почему я молюсь  и не помогает? Почему? Будь проклято это солнце! Оно печёт, но не может убить меня. Почему? (Обрыв записи)
(прибл. 1913-г)
 Было написано множество книг, но если приглядеться, вчитаться в них, то мы можем увидеть упаднический характер. Человек, который потерял  веру в жизнь, и потому, он пишет, пишет подобное и страшное. Его можно, конечно, сравнить с Кингом, но ужасы, связанные с мистикой и современностью, ужасы, связанные где-то в его бредовом мозгу. Я пытался прочитать, и меня попросили дать резолюцию, меня попросили написать введение сборника очередных его рассказов. Вы знаете, я не нашёл в них ничего замечательного, ничего замечательного…(сбой)
\people{(Ольга) Один, два… }
\soul{На какой букве остановились мы?}
\people{(Гера) На ``Т''.}
\people{(Ольга) На ``Т''. ``Т'' не до конца мы разобрали. А можно вам вопрос задать? }
\soul{Спрашивайте.}
\people{(Ольга) Подска… ой, Господи, вопрос-то забыла уже… Извините… }
\people{(Лена)Давай я задам. Можно у вас спросить? Скажите, пожалуйста, вот мне недавно сказали, что я с другой планеты. Вы можете сказать относительно нашего мира, это так?}
\soul{Думайте и решайте. Думайте! Давайте начнём с того, что вы называете ``планетой''? Планета –  именно этот дом, где вы живёте сейчас?}
\people{Нет, я  имела в виду планету… Земля, в смысле, там…}
\soul{Прекрасно! Земля! Но, вы знаете ли, сколько есть этих ``Земель''? Вы  говорите о параллельных мирах? Физических? - Физически есть и они. Когда-то вам ``они'' говорили об изменениях, о непохожести миров. И уже трудно сказать, с этой вы планеты или с другой. Проще сказать, на том ли вы уровне. Важна не планета, не то, где вы будете в будущем, или где вы были в прошлом. Это не важно, на какой планете будете жить вы физически. Физически - не должно интересовать вас, где жили вы, вы должны интересоваться, как жили. Прежде всего - как. Вас же интересует, на какой планете были вы, кем вы были в прошлой жизни, но не интересует, чем и как занимался этот человек.}
\people{Почему? Интересует.}
\soul{Интересует ли?}
\people{Интересует.}
\soul{Если вам скажут, что вы были когда-то путаной, вы не поверите тому.}
\people{Поверю. Почему?  Я сон видела.}
\soul{Вы будете искать множество оправданий. Почему? Почему? – Потому что вы хотите видеть себя лучше. И скажем же наоборот: всё, что ни сказали бы вам, если это сказано в необычной обстановке, вы обязательно в это поверите, вы обязательно найдёте множество доказательств. И если мы скажем, что вы были путаной, вы скажите: ``Да, я видела сон''. }
\people{Да, я видела.}
\soul{Если же мы скажем, что вы никогда ей не были, вы тут же найдёте другой сон, который будет подтверждать, что вы, действительно, никогда не были. Вы всегда найдёте множество, множество фактов, которые могут подтвердить то, что вы желаете. Давайте попробуем, сделаем так… (сбой)}
\people{(Гера) Один… }
(сбой)
\soul{Зелёный цвет. Иногда нам приходится брать другие краски. Это редко, но бывает.}
\people{(Ольга) Расскажите нам, что зелёный означает и какие бывают зелёные…  Какие оттенки зелёного бывают? }
\soul{Знаете, я рождаюсь гораздо раньше, чем носитель этих цветов. Ещё за год, лоно матери уже приобретает зелёный цвет, зелёный цвет жизни, и этот цвет уже манит, манит и приглашает детей. Дети, увидев его, приходят, и тогда – ``тайна рождения'', и в этот момент я переливаюсь всеми радугами, я уже забываю, какой цвет мой - я меняю их все. Приходит время, и я снова приобретаю свой основной цвет зелёный. И когда ребёнок начинает шевелиться и делает первые вздохи, я приобретаю столь яркий цвет, что даже матери могут увидеть его. И когда они разговаривают с дитём, они разговаривают и со мной, и когда я слышу, как один из них говорит: ``А нужен ли нам ребёнок этот?'' – я чувствую страх ребёнка, и тогда из зелёного я становлюсь грязным, и тогда я уже боюсь самого себя. Ребенок начинает выделять яды, и мать начинает плохо чувствовать себя. Ему бы, бедненькому, затихнуть, но он боится. А если отец говорит: ``Может быть, не нужен он?'' – а мать чувствует токсикоз – она уже соглашается. Ребёнок тогда боится ещё больше, и тогда мать уже решает полностью… Тогда приходят другие цвета и стараются восстановить баланс. Проходит порыв у матери, отец успокаивается – успокаивается и ребёнок. Когда мать едет в автобусе, и кто-то нечаянно толкнёт, то ребенку становится больно, и эта боль передаётся матери. Мать, испугавшись ``не навредила ли чего'', начинает ругаться с толкнувшим. И тогда, уже мать выделяет яды, и тогда, уже ребёнок начинает задыхаться. И тогда , опять я теряю чистый цвет, и опять я становлюсь грязным. Когда мать заходит, и ей не уступают место, про себя она злится, и тогда злится ребёнок, и тогда мать чувствует его беспокойство… И если это повторяется часто, и если мать старается успокоить себя химией, ребенок постепенно начинает умирать, и тогда, я теряю полностью свой цвет. Я зелёный, а значит, я даю жизнь, значит, я признак жизни. Представьте, представьте, когда мать приходит и ложится на кресло, и когда в лоно, в дом ребёнка, входит чужеродное, и старается вырвать, вытащить его оттуда… Представьте, как кричит ребёнок, представьте, как он мечется и хочет уйти от смерти! Вы только подумайте - он ещё не родился, а его уже убивают… И тогда, я становлюсь чёрным, чёрным. Я уже теряю, теряю всё. И вместе со смертью ребёнка умираю я, потому что я – признак жизни, а жизнь - уходит. И тогда, в следующей жизни, я прихожу к матери, прихожу уже чёрным цветом и не даю матери родить. И сколько бы потом ни мучилась, сколько бы она ни пыталась сделать это, никакая химия, ничто не может помочь, потому что я – признак смерти, и я не даю, и я не пускаю сюда детей. Я пугаю их, и они уже не приходят в этом дом, в это лоно. (Обрыв записи)}
 
Цвет радуги – это если всё хорошо, но бывает, что ему соседствовать с голубым, а бывает и чёрным, всё зависит от того, что происходит извне и внутри. Я ношу цвет, но я не даю ничего, я – всего лишь призрак ваших желаний, я – всего лишь призрак, столь неясный и столь необычный, только в ваших представлениях, я всего лишь ваши реакции, только всего лишь, я – всего лишь призрак того, что вы хотите. И если вы хотите умереть, если вы потеряли интерес к жизни, со мной будут соседствовать чёрные краски. Чёрные… Или ``ничто'', не имеющее никакого цвета, – оно самое страшное. Чёрный – это значит, ещё можно вернуться, можно передумать, можно остановиться. Ничто, если здесь есть ``ничто'', оно не имеет никаких красок, а это значит ``истинная смерть'' – та смерть, после которой уже нет рождения. Это то, что вы называете действительной смертью, а всё остальное, всё остальное – это лишь только ‘’переход’’, это всего лишь ‘’лифт’’, ‘’метро’’ или какой-нибудь другой транспорт, который перемещает вас из одного тела в другое. (Обрыв записи)
(неизвестное время)
\soul{…эмир. Зачем, зачем ему было упоминать об этой звезде? Властелины не любят, когда говорят о их смерти, да ещё если рядом слуги… Если про них кто-то говорит о падшей звезде, этот человек уже не может жить. Хан не был исключением и дал строжайший указ: ``Всего мальчишку схватить немедленно, привязать к хвосту лошади и пустить. За ним же пусть будут идти трое, чтобы не ушёл, собака. И если что-то останется от него, разметать по всему полю, чтобы не осталось и следа!''. Говоря это, он так ехидно улыбался и с такой злостью представлял ту картину, что уже было желание самому, самому с этими тремя проскакать, но не царское это дело. Мальчишка же, испугавшись и не поняв, что случилось, стал кричать: ``Господин, я же только хотел…'' –  что хотел он, хан уже не слышал. Никто не хотел знать, что продолжит мальчишка, и для чего он пришёл сюда. И обещанный злотый должен был быть проглочен им. Хан не был скупым, он достал свой кулёчек и сказал: ``Пусть всё съест эта собака!'' – и мальчишка, захлёбываясь, должен был глотать эти монеты. Печальная сказка… и более печально её продолжение… - Кони были снаряжены быстро, нашлось три охотника, которые сопровождали бы. И каждый их них уже представлял, уже рисовал картинки будущего развлечения. Мальчишка уже не плакал, ему уже нечем было плакать. Его никто не бил, его нельзя было ударить – таково было указание хана: ``Относитесь к нему вежливо. Он – царь! Он принес весть о моей смерти. Так пусть будет он почитаем, и умрёт почитаемо!'' (Обрыв записи)  }
(наше время)
\soul{Что может быть горче детских слёз? Мальчишка плакал. Весь мир собрался в маленький кулачёк, в котором умещались все слёзы, те слёзы мира. Он стоял и смотрел на мать, на отца, он не знал, что делать. Отец же, что-то крикнув обидное, повернулся и пошёл. Мальчишка побежал за отцом, он кричал: ``Папа!'' – но отец не слышал, отец был слишком злой, чтобы что-то слышать. Он ускорял шаги – мальчишка бежал, бежал, споткнулся о камень,  поднялся. Мать же, мать, увидев упавшего мальчишку,  вскрикнула, побежала за ним. Бежа, она кричала вдогонку проклятья отцу: ``Остановись! Остановись! Услышь!'' – отец повернулся, что-то сказал в ответ грубое и пошёл дальше. Мальчишка не хотел отставать, ему уже было всё равно, лежать, стоять, он смотрел, как уходил отец, повернулся к матери и сказал: ``Мама, папа нас больше не любит.''}
\people{(Ольга) Девять… }
(
эта жизнь)
\soul{Вы знаете… Видите, Михалыч, моему сыну снятся странные сны, он мне рассказывает, а я боюсь. Ему снятся какие-то инопланетяне, к нему кто-то приходит, он даже завёл себе друзей. Представьте, он просыпается, рассказывает мне, что он играл в игрушки, которые ему подарил какой-то дядя в белом. И какая-то женщина, тоже в белом, рассказывает ему сказки. Мне кажется, здесь надо что-то думать. Ему всего 6 лет, а он уже рассказывает такие чудеса! Вы знаете, у него какое-то больное воображение. Я бы попросила вас посмотреть моего ребёнка. (Обрыв записи) }
(неизвестное время)
… приёмник принял его холодно. Мальчишки, больше из любопытства, выбежали наружу и рассматривали новенького. Новенький же, испугавшись и растерявшись, не знал что делать. К нему подошла воспитательница и сказала: ``Пошли!''. Она привела его в палату, показала ему тумбочку, кровать и сказала: ``Теперь ты будешь жить здесь'', – ``А мама?'', – ``Мама? Маму ты увидишь только в субботу'', – ``Почему?'', – ``Надо! Надо! Потерпи, сынок! Потерпи''…(сбой)
\people{Один…}
\soul{Всего лишь взрослые не умеют… (сбой)}
\soul{Давайте продолжим. Какую следующую букву будем рассматривать сейчас?}
\people{<<У>>}
\soul{``У''. Сразу вспомните слова, относящие к человеку, начинающего на ``У''.}
\people{(Ольга)У?}
\people{(Гера) Умный. }
\people{(Ольга)Удивление.}
\soul{Удивление. Хорошо. Любимый крик победы?}
\people{Ура? (вместе)}
\soul{Ура! Хорошо, назовите мне слово, содержащее всего лишь только две буквы.}
\people{(Ольга)Уа.}
\people{(Лена)Уа. Ребёнок…}
\soul{А теперь произнесите наоборот.}
\people{Ау.}
\soul{В чём разница?}
\people{(Ольга) ``Уа'' - когда кричит ребёнок, рождается, а ``Ау'' - когда потерялся.}
\people{(Лена)Когда в лесу потерялся…}
\soul{Так, теперь давайте подумаем. На что похожа эта буква?}
\people{(Ольга)<<У>>?}
\people{(Лена)Недоконченная ``Х''?}
\soul{Нет, давайте говорить о предметах.}
\people{(Ольга) Рогатка. }
\soul{Рогатка? Почему же не ``сосуд''?}
\people{(Ольга) Сосуд? }
\people{(Гера) Рюмка , да.}
\people{(Ольга) Да, рюмка, наклоненная, наклонённая. }
\soul{Тогда, давайте, попробуем, напишем прописную.}
\people{(Ольга) А, да! }
\soul{Есть ли общее?}
\people{(Ольга) Общее? }
\soul{Общее – чаша, - и там, и там увидите чашу. Чашу, которая пуста. Вы здесь не видите, в отличие от иных букв, каких-нибудь чёрточек. Представьте, что вы – чаша, вы – сосуд, и вы стараетесь всю жизнь наполнить его, и вы наполняете, и довольно успешно. Слишком успешно, что порой чаша эта переполняется, и тогда происходит срыв. Тогда, вы стараетесь избавиться от груза, избавиться от содержимого этой чаши. Теперь вспомните возраст восемнадцатилетнего ребёнка. Каким он является, какова основная черта его характера?}
\people{(Ольга) Стремление, наверное? }
\people{(Гера) Любознательность? }
\soul{Нет. Нетерпимость.}
\people{(Гера) А-а-а…}
\soul{Нетерпимость. Он построил уже свой мир, и об этом говорили прошлые буквы, он построил его, и он не хочет, чтобы он менялся, - для него уже всё понятно, для него уже есть чаша, которую он заполняет, и которой он является. И в виду своей жадности, он не собирается расставаться с этим грузом - он ещё молод, и он ещё не знает, что такое боль. Вы скажете, что есть дети, которые потеряли многое, и которые уже знали больше, чем взрослые. Давайте не будем говорить о крайностях, давайте не будем говорить о той границе, которая, дай бог, будет далека от вас. Теперь, представьте эту чашу перевёрнутой. Или, давайте, ещё лучше - представим эту чашу в латыни. Здесь вы уже не видите никаких хвостиков, здесь уже чисто только чаша и не более. Что говорит об этом? Попробуйте ответить вы.}
\people{(Ольга) Перевёрнутая чаша? }
\soul{Давайте будем говорить о латыни.}
\people{(Ольга)О латыни?}
\soul{Чистая чаша, не имеющая ничего.}
\people{(Гера) Без опоры, наверное. }
\soul{Опора?}
\people{(Ольга) Это не опора, это устремление вниз. Что-то такое. И чаша, наверное, принимает сверху. }
\soul{Согласитесь, что в чашу можно положить только сверху, она не перевёрнута.}
\people{(Ольга) Но,  если есть отросточек, то она может вылиться вниз как бы, вот… }
\people{(Лена)Вот так вот - закруглённая ``У''… Вот так вот, если перевернуть…}
\people{(Ольга) Нет, я про русскую ``У'' говорю.  }
\people{(Гера) Там - про латынь. }
\soul{Хорошо. Здесь вы поставили чашу, и она, действительно, не имеет более ничего, для неё нет ни ручек, ни рукояток, но она достаточно устойчива, и если человек пишет её с наклоном в какую-либо сторону, то это говорит о его характере, это говорит о том, умеет ли он сохранять то, что уже получено. Представьте букву, наклоненную вперёд. Это что значит? Это значит, что вы торопитесь, у вас нет терпения, вы стараетесь достичь, вы бросаете всё, лишь бы сделать что-то одно – и что в итоге? – В итоге вы бросаете и это дело, потому что появилось новое. И так -  это будет всегда незакончено. Теперь, представьте букву, которая наклонена назад. Это значит, что вы достаточно консервативны, вы не желаете менять своих взглядов, вы достаточно стабильны и упорны, и что бы новое ни происходило, для вас это очень туго доходит, вы это или не признаёте, или обвиняете ложью, или, что ещё хуже, ставите в вину другим. Получается - что? – Получается, вы должны написать её прямо – это тоже неверно. Буква, с задней стеной, чаша, должна быть прямой, передняя же должна быть наклонена – это значит, что вы уже познали старое и не хотите возвращаться к нему, вы идёте вперёд, но, не растеряв старого. Теперь, представьте эту букву достаточно пузатой. Вы можете написать её так? – Можете. Это говорит не о жадности, хотя можно было бы и сказать, что - да, похожи. Скорее всего, это говорит о том, что вы хотите как можно больше узнать  не просто от жадности, а просто больше, больше и больше… Вы не можете объяснить причины, что заставляет вас искать. Вы, просто собираете, вы просто собиратель, вы – библиотека, достаточно могучая библиотека, но бестолковая. Почему? – Потому что вы просто собиратель, вы коллекционируете ``марки'', и вам всё равно, зачем эта ``марка'' была издана, и что она обозначает. И если уж разговор идёт о ней, то вы просто говорите, как она была трудна, как её было трудно достать и какова её ценность. Вы можете нарисовать, что там есть, но это будет всего лишь карикатура, это будет всего лишь попытка создать, что уже есть у вас. Теперь, представьте, вы написали её довольно-то узко, чуть ли не слитно. О чём это говорит? – Это говорит о том, что вы слишком целеустремленны, до такой степени, что это переходит в растерянность, совершенно крайние вещи, но это так. Вы столь устремлёны, что всё теряете, вы видите только одну цель, только её и больше ничего. Про таких говорят: ``Больше своего носа он не видит''.}
\people{(Ольга) Скажите, а вот наклон почерка, как раз - вообще можно сказать о том, что это вот… ээ… как раз это… вы говорили сейчас.  Любой почерк,  то есть ну…вообще…}
\soul{Да, это характер. Всмотритесь, если почерк неаккуратен, обрывист и быстр, это что? – Это значит, что человек всё в жизни старается сделать быстро, необязательно качественно, лишь бы быстро, ему бы быстрее отделаться. Теперь, представьте, почерк очень аккуратный – педантичность, переходящая в такую крайность, что с вами просто уже тяжело иметь что-либо, вы будете занудой. Самой натуральной занудой. Каждую чёрточку, каждый завиточек вы будете выводить, не важно, что вы будете писать долго, а это значит, и делать что-то будете долго. Вы будете стараться делать качественно, да, но при нехватке времени вы можете забросить это. Так лучше уж сделайте плохо, но сделайте, чтобы это всё-таки было.}
``Ф'' – мужичёк-руки в бочёк. Похоже, не правда ли? Теперь давайте представим эту же букву, но в латыни.
\people{(Ольга) У нас сейчас с латынью плохо, между прочим.}
\soul{Хорошо, тогда давайте напишем ``Ф'' прописной. Есть ли различия?}
\people{(Ольга) Нет.}
\people{(Лена) Практически – нет.}
\people{(Гера) Размер. Нет…}
\soul{О размерах мы говорили с вами уже. Тогда, давайте что-нибудь вспомним из слов. ``Фанатик'' – на букву ``Ф''.}
\people{(Ольга) ``Фантаст''.}
\soul{``Фантаст'', на букву ``Ф''. ``Фома'' – на букву ``Ф''. Что объединяет их?}
\people{(Ольга) Фома неверующий, фантастам не верит, что это вроде как фантастика, и ``фанатик'' – это человек, который переходит границу, то есть…}
\people{Слепо.}
\people{(Ольга) Слепо.}
\soul{Нет.}
\people{(Ольга) Нет?}
\soul{Нет. Может быть фантаст фанатиком…}
\people{(Ольга) Нет.}
\soul{…до такой степени, что он просто не верит в реальность и верит только в созданные собой картинки?}
\people{(Ольга) Да нет, не может.}
\soul{Нет может? Тогда, почему же когда-то вы спрашивали о воображаемых странах, когда вы спрашивали об эльфах?}
\people{(Ольга) А-а-а, фэнтези../}
\soul{Разве нет? Фанат, фантаст и Фома… Фанат – это значит, ``исключающий всё'', - только своё''. А это, значит, и <<Фома>> – уже в другое он не верит. Другое, для него, не существует. Вы согласны? Фантаст – потому, что он живёт в мире фантастики, потому, что для него всего лишь только одна цель, он видит всего лишь только одну точку в мире и больше ничего, всё остальное ложно – вот вам и Фома. И чтобы покрасивее назвать эту точку, и чтобы не обмануться в ней, чтобы прибавить себе сил дойти до неё, ему приходится фантазировать. И когда он защищает эту точку, он тоже будет фантазировать, выдумывать множество фактов, чтобы доказать это. Факт – на букву ``Ф''.}
\people{Странно…}
Конец 2-й записи
1996.01.20-03
\soul{``Х''?}
\people{Да.}
\soul{Что это? Это бумеранг, это в первую очередь, бумеранг, который возвращается к вам же. Слово ``хорошо''. Слово ``хорошо'', относящее только к вам. Слово ``хорошо'' вы можете только понять, когда только вам хорошо. Когда к вам придут и скажут: ``Мне хорошо!'' – вы не почувствуете того, пока вам не захочется сказать этого слова. Это бумеранг, который бьёт вас же, если вы промахнётесь. Вы же промахиваетесь довольно-то часто. Буква ``Х'' – это значит, что вы поставили крест, поломанный крест, упавший крест, крест на всё, вы устали, это конец алфавита, это значит, уже идёт к концу вашей жизни. Не важно, что вам будет всего, допустим, десять лет. И в двенадцать, и в десять можно умирать. Никогда не давайте тех обозначений неосуществлённого и неживого, никогда не соизмеряйтесь с возрастом, ибо возраст – не в годах. Запомните это! Давайте далее. Крест, крест, обозначающий, что вы пришли к началу конца, это значит, что вы уже давно несёте свой крест, крест, который давит вас, если вы не знаете, как нести его. Но есть и те кресты, которые помогают вам идти по жизни, которые дают вам крылья. Но есть и те кресты, которые давят вас к земле, и вы не можете подняться, и вам тяжело, и вы всем жалуетесь, что у вас сложная жизнь, что вас обидела судьба, вы проклинаете всё и себя, вы ищите виновных везде, но не в себе. Но есть и тот крест, который позволяет вам взлететь, взлететь выше всего, взлететь выше всех дел, и посмотреть на них сверху и увидеть истинное их положение, что вы сделали, что вы должны ещё сделать. Крест – это тот же пропеллер, который позволяет подняться вам в небеса, но это и тот же крест, который пригвоздит вас к сырой земле. Спрашивайте.}
\people{(Ольга) Скажите, а вот цифра 10 римская, это что, тоже связано с этим?}
\soul{Хорошо, давайте вернемся к цифрам. Цифра 1 в римской - что вы видите здесь? Давайте договоримся сразу: все цифры ограничены сверху и снизу. Вы согласны?}
\people{Да.}
\soul{Ограничение. Вы знаете, что Рим погиб.}
\people{Да.}
\soul{Вы знаете это. И теперь, сопоставьте. Вы знаете, что это язык мёртвых, вы знаете, что этот язык не очень-то и применяется, вы знаете, что он достаточно сложен. Возьмем единицу. Что эта единица? – Это начало, вы должны были помнить. Возьмите II. Это два начала. И возьмите любые цифры и любые числа, они состоят не более, чем из трёх. А вы говорили о триединстве. Давайте посмотрим цифру V. Что это? – Это опять та же чаша. Чаша, которую вы осматриваете и изучаете. Вспомните цифру 5 в русском языке, в кириллице. Вспомните, что это? Мы говорили вам о цифре 5 – это оценка, вы помните? Смысл тот же и здесь. Это чаша, чаша, которую вы рассматриваете и ставите себе оценку, и ставите себе, и находите место в жизни, но, опять же, она ограничена. Что цифра IV? Цифра IV – это начало оценки, это начало оценить себя. Заметьте, в цифре IV сперва идет I, и потом… Возьмём цифру VI. Вспомните, что мы говорили о шести? – Мы говорили вам, как о цифре дьяволе, когда вы возомнили себе, что вы уже всё знаете. Помните? А теперь представьте, как вы оценили себя, цифра 5, и поставили черту, иными словами, вы поставили ``точки над i'', вы закончили изучать себя, вы уже остановились. Вот вам цифра VI…. Цифра VII. Здесь добавлена ещё одна черта. Это что? – Это новое начало, но чего, вы ещё не знаете. Вы чувствуете, что вам мало того, что уже есть, и вы хотите получить более, чего, вы ещё не знаете. Вы говорите: ``Я что-то хочу, но что, не пойму'' – это бывает у вас достаточно часто. И наконец, цифра VII.}
\people{1-2-3}
\soul{Итак, цифра VIII. Вы уже имеете три черты, и опять же - ограничены. Попробуйте теперь сами дать трактовку этому. Хорошо, давайте не будем мучить вас и скажем так… Вы знаете, что вам нужно что-то найти, вы становитесь на новой ступени, и вы начинаете искать эту ступень – вот вам ещё одно начало. Начало поиска, начало, где вы уже находите ниточку и пытаетесь схватить её, и идёте к ней. Теперь давайте возьмём цифру IX. Что?}
\people{Начало.}
\soul{Начало…}
\people{Начало конца.}
\soul{Начало конца поиска. Только начало… Вам ещё идти и идти, вам  идти очень долго и много. Итак, следующая цифра?}
\people{Х}
\soul{Цифра Х. Мы же говорили, что это ``крест'', это ``крест''. Да, его пытались и всегда пытались ограничить. Никто не хочет умирать. Было придумано множество рецептов бессмертия, и все пытались снять эти рамки, но постоянно писали их. Почему? – Потому, что была безысходность, вы поставили крест, вы поставили крест на самом себе, вы пригвоздили самого себя ко времени, вы остановили в себе то, что даёт вам жизнь. Итак, цифра XI. Или давайте сделаем проще, перейдём к другим цифрам. Итак, цифра пятьдесят. Вы помните, как пишется она?}
\people{(Гера) Это L  латинская, по-моему, потом… и чё-то там ещё…}
\soul{Прекрасно. Давайте остановимся на этом. Итак, пятьдесят. Это буква L.}
\people{1-2-3}
\soul{…моя чаша, но уже разбитая. Посмотрите, нет одной стороны. Итак, цифра XL (сто). Это цель. Та же самая чаша, она разбита там же, на том же месте, но только здесь вы уже видите более плавные границы. Иными словами, за множеством, за таким большим количеством, вы уже не видите, что вы потеряли, уже сглаживаются границы восприятия мира…(сбой)}
\people{1-2-3}
(1912 детдом)
\soul{Кто-то накинул одеяло, и толпа кричащих мальчишек стала избивать его. Он сперва кричал, потом, поняв бесполезность, он замолчал. Это ещё больше разозлило их, и они стали его бить уже ногами. Он сжался в комок и пытался занять как можно меньше места на кровати, он хотел спрятаться, он хотел спрятаться, но куда? Куда уйдёшь от одеяла, которое было придавлено со всех сторон, которое держало множество пацанов, которые пытались избить его, и которые пытались заставить его кричать? Но он молчал. Его разбитые губы…}
\people{1-2-3}
\soul{Была и другая ночь. В ту ночь никто не трогал его. Мальчишки были удивлены тем, что он никому не пожаловался. Его всегда считали отщепенцем, но их удивило, что он не стал говорить никому ничего. Он рассказал сказку, что он упал с лестницы. И хотя никто не поверил, не стали добиваться правды. Они знали, что мальчишка упорный, и всё равно не скажет, потому они оставили в покое и ожидали, что будет дальше. Ему же снится сон, сон детства, когда его друзья пришли к нему, ласкали его и залечили его раны. И женщина говорила: ``Я буду пока твоей мамой, можно?'' – он хотел сказать: ``Можно!'' – но разбитая его губа болела, и что-то мешало назвать её матерью. И хотя она была красива и ласкова, и он готов был растаять в её объятиях, что-то останавливало его. И мать, та мать, которая привела его сюда, виднелась где-то вдалеке. Он видел её, он видел её платок, он видел отца, ушедшего, он видел…(сбой)}
\people{1-2-3}
\soul{…Снова наступил вечер. Люди, пришедшие с работы, уставшие, уже наевшиеся, включили телевизор и стали ждать очередной проповеди. Диктор говорил по телевизору, что проповедь будет, но простите за технические…(сбой)}
\people{1-2-3}
(2757 год)
\soul{…встал и должен принести им. Я пытался, пытался разглядеть его. Стекляшка, она и есть стекляшка, я таких видел множество на пустоши, но, правда, они были уродливыми, по сравнению с ним. Красивая игрушка, но что делать с ним? Что? Бон сказал, чтобы я ничего не трогал, достаточно того, что я чего-то включил, что-то уже работает и с кем-то болтает. Мальчишка сновал туда-сюда, туда-сюда, он находил какие-то книжки и кричал: ``Какой рецепт, какой рецепт! Я его обязательно запишу, я стану великим лекарем! И тогда я приду к этому князю…'' – он до сих пор помнил князя, и до сих пор жаждал отомстить ему.  Анур же…(сбой)}
\people{1-2-3}
(1248 год)
\soul{Я видел Анура в деревне, он разговаривал с какой-то девушкой и дал ей какую-то траву. Когда я спросил у него, кто она, он почему-то заулыбался и ничего не ответил. Тогда я стал спрашивать, что ты дал её, он говорит: ``Траву''. Я спросила: ``Что это за трава?'' – он молчит. Тогда мне пришлось заглянуть в его мешочек, когда он спал, и я испугалась. Я поняла, что у него должен быть ребёнок, и он хочет убить его. Он дал ей траву, чтобы не было детей. Он мне ничего не сказал. Почему? Почему? Что делать теперь мне? Эта трава убьёт её, а он спит и не знает об этом. Что буду делать я, если я сейчас спущусь? Как я смогу объяснить ей, если я не умею говорить…(сбой)}
(1912 год Сергей Иванов)
\soul{Что я должен сделать? Что я должен сделать, чтобы услышать снова эти голоса, что? Почему? Почему я хочу эти голоса? Я их боюсь, это мой бред, а, может быть и нет. Они говорят, что я не умру. Я уже чувствую, я уже чувствую, как солнце всё становится жёлтым, как оно проникает в меня, как сжигает меня, как во рту уже всё сохнет и мне кажется, что с меня уже облазит шкура, а они говорят, что я буду жить. Я хочу им верить, но боюсь обмана. Сколько раз уже лгали мне, сколько раз лгал я… Я помню отца, помню смутно, я хотел вспомнить, как его зовут. Как обидно, когда не знаешь имени отца… Когда составляли на меня метрику, и спросили: ``Как зовут твоего отца?'' – я сразу же сказал: ``Иван''. Я почему-то испугался. Я испугался, что надо мной будут смеяться, что я не помню отца. Меня спросили: ``Кто твоя мать?'' – я стал придумывать имя и не смог придумать. Я запнулся, а они: ``Ладно, поставим прочерк''. И этот прочерк мне потом долго снился. Мне снился отец,, а рядом с ней всегда стоял какой-то прочерк, и этот прочерк был моей матерью. Это было очень страшно. Я просыпался в поту и хотел крикнуть: ``Отец! Батя! Назови, назови имя!'' – но всё время стоял прочерк. Меня спросили: ``Сколько тебе?'' – я сказал: ``Не знаю''. Тогда они засмеялись, сказали: ``Посмотри в зеркало'' – я глянул в зеркало, но ничего не понял. Они тогда спросили: ``Какой был у тебя волос?'' – я опять не понял. Они говорят: ``Он у тебя слишком седой, слишком седой для твоих годов. Что было с тобой, ты помнишь?''. Я говорю: ``Нет, я помню только, как я шёл, пришёл в какую-то деревню и помню, что меня хотели сжечь'' – ``Сжечь? Как это - тебя хотели сжечь?''. Я стал им объяснять про какую-то болезнь, они стали выяснять, что это была за болезнь. А одна, довольно  такая интеллигентная, вся в перстнях, даже отодвинулась от меня подальше. ``Что за болезнь? Ты можешь вспомнить?'' – я не мог. Тогда они стали меня спрашивать: ``Чем лечили тебя?'' – я не мог вспомнить и этого. Тогда они стали спрашивать: ``Тогда ли ты потерял волос?'' – я говорю: ``Я их не терял!'', – ``Какой ты глупый, мы говорим о цвете! Когда ты поседел?'', – ``Откуда же я знаю, когда я был седой?''. Тогда они стали выяснять, чем болели мои родители, я говорю: ``Я не знаю, я не знаю своих родителей'', – ``Не помнишь ничего? Но ведь тебе уже десять, ты должен что-то помнить!'' – я не помнил ничего, я помнил только последние четыре года. Четыре года в детдоме при монастыре, когда батюшка постоянно бил меня за то, что я плохо читаю молитвы, за то, что я забываю утром помолиться, за то, что я перед едой не крещусь, и когда молюсь, забываю сказать ``Аминь!''. Почему? Почему мне так всегда не везло? Почему я стремлюсь услышать голоса? Может, потому, что они не являются прочерком? Потому, что это не прочерк, а действительно живые голоса. Но, почему же я тогда брежу? Почему…(сбой)}
\soul{Как же так? Получается, всё напрасно?}
\people{Что?}
\soul{Это всё напрасно? И вы стали теперь как все? Что, опять пришёл царь?}
\people{(Ольга) Нет. Царь не пришёл. Президент у нас и Верховный совет Российской Федерации.}
\people{(Гера) Вы нас слышите?}
\soul{Странно…}
\people{(Ольга) Почему?}
\soul{Странно и страшно. Это получается, что всё напрасно.}
\people{Да нет!}
\soul{Я напрасно хожу, учу труды. Я напрасно хожу и прохожу ликбез. И всё это напрасно.}
\people{(Ольга) Да нет, учиться разве… учиться всегда нужно при всяком строе. Разве можно…}
\soul{Хорошо, лучше сказать, чем буду я?}
\people{(Ольга) Когда?}
\soul{Вы говорите, что я не умру. Сколько я проживу?}
\people{(Гера) Всегда.}
\people{(Ольга) Человек вечен, он не умирает.}
\people{(Лена)Все умирают.}
\soul{Вы говорите по книге, вы говорите о каких-то будущих или о прошлых жизнях. Мне об этом говорили, но я не очень-то в это верю. Я хочу знать, сколько здесь я проживу. Умру ли я в этой пустыне или нет?}
\people{(Все) Нет.}
\people{(Гера) Однозначно.}
\soul{Что буду делать я? Кем буду я? Если вы – моё будущее, значит, вы должны знать об этом. Тогда же я – ваше прошлое. Вы можете мне сказать, кто я буду?}
\people{(Ольга) Да. Сергей, да? Сергей Иванов?}
\soul{Сергей.}
\people{(Ольга) Вот, вы рядом с нами, вот здесь вот, и мы с вами разговариваем, а говорит нам голосом тот, который вы…}
\people{(Гера) …которым вы будете.}
\people{(Ольга) Тот, которым вы будете. Вот он живой, невредимый лежит…}
\people{(Гера) …в будущем, так что вы можете быть спокойны…}
\soul{А сколько ему?}
\people{(Ольга) Сейчас?}
\people{(Гера) 34.}
\people{(Ольга) 34 года.}
\soul{34…}
\people{(Ольга) Да.}
\soul{Это не так уж и мало.}
\people{(Гера) В 62-м году родился, 1962.}
\people{(Ольга) Но ведь ты живой.}
\soul{В 62-м?}
\people{(Гера) Да. Шестого числа шестого месяца.}
\soul{Значит, я уже всё-таки умер.}
\people{(Все) Нет.}
\soul{А в 62-м родился снова?}
\people{(Ольга) Да.}
\people{(Гера) Да, переселение душ существует, оказывается.}
\people{(Ольга) Не переселение…}
\people{(Гера) Ну, в  другое тело, почему не переселение?}
\soul{Как жалко! Я хотел ещё.}
\people{(Гера) Все мы тут тоже хотим…}
\soul{Я думал, что я буду жить, оказывается, вот как… Значит, я всё-таки умру.}
\people{(Все) Нет.}
\people{1-2-3}
(2757 год)
\soul{…не плачет, тем более охотник. Поэтому, Ануру пришлось уйти подальше от посторонних взглядов, спрятаться, и скупую слезу спрятать, стереть кулаком, да побыстрее. Он знал, что в нём сидит кто-то, что в нём сидит хищник, зверь, и этот зверь в любой момент может проснуться, но Анур уже не боялся его, он уже знал, что зверь не победит. Сперва Анур страшился, страшился его пробуждения. Иногда этот зверь спасал его, но почему-то за счёт других. Когда просыпался зверь, Анур не помнил ничего, он бросал всех и крушил всё, лишь бы остаться живым. Он даже забывал о сыне, о долге перед ним. Но, то ли зверь становился слабее, то ли Анур привыкал к нему, то ли становился сам сильнее. Зверь уже не давал ему той силы, это, конечно, было плохо, но уже зверь не затмевал, и это успокаивало Анура. Но, с другой стороны, этот зверь всё-таки сидел в нём, и Анур иногда чувствовал его, чувствовал столь хорошо, что даже в груди было больно, и эта боль заглушала боль ноги. И тогда Анур чувствовал, что сейчас что-то будет. Да, действительно,- или выбегал какой-то зверь, и приходилось бороться, или уходить от кого-то, но зверь всегда, всегда не ошибался, зверь всегда предсказывал, что пришло время борьбы, но этот зверь не стал таким сильным. Может быть, это хорошо, а, может быть, плохо. Анур не знал, он пробовал говорить об этом с Боном, но Бон улыбался, пожимал плечами, стал рассказывать какие-то сказки о прошлом, но это ничего не давало Ануру. Он хотел знать, что именно сейчас. Анур же говорил о каких-то колдунах, что когда-то у них был обряд, когда изгоняли одержимых. Что такое ``одержимый'', Анур не знал, но ему почему-то это слово приелось, и он понял, что это, наверное, о нём. И Анур хотел бы, чтобы изгнать его, и в то же время - нет, потому что этот зверь всё-таки помогал бороться. Анур вспомнил рассказ отца. Как-то отец рассказывал ему, когда на деревню напали соседи, отец потерял счёт времени и не мог вспомнить, что было во время битвы. И когда он очнулся, вокруг было множество убитых и покалеченных, сам же он стоял в крови, в чужой крови, и на нём не было ни одной царапины. Множество стрел было вокруг него, но ни одна не задела. Множество копий было воткнуто вокруг, и были порваны все его одежды, но ни одно не ранило его. И Анур понял – это осталось от отца наследство. Старик что-то говорил о звере, старик говорил, что если этот зверь станет властелином Анура, то тогда он будет похож на муравья ещё хуже. Интересно, что может быть хуже, хуже муравья? Хм, я – зверь… Анур не мог этого понять. Анур подшучивал над самим собой, подшучивал над Боном, и когда ему было слишком тяжело, он начинал шутить, и, кстати и некстати.  Это была его защита, это была его реакция, реакция, которая могла успокоить его и остановить движение того зверя.(сбой)}
\people{1-2-3}
(времена атлантиды)
\soul{Закончив последний горшок, остановив гончарный круг, Анур пошёл к Анису. Анис жил невдалеке, в стенах города. Когда он пришёл, увидев закрытую дверь, то был разочарован – опять на службе? Но на службу ему нельзя было попасть по той причине, что он…}
\people{1-2-3}
\soul{Слава Гею! Слава богам! Слава всем, кто дал мне соки этой жизни! Я хочу спросить у вас.}
\people{(Ольга) Да.}
\soul{Как вы относитесь к богам? Почитаете ли вы их, или вы сами боги?}
\people{(Ольга) Да нет, мы не боги.}
\people{(Лена)Бог у нас один, вообще-то…}
\soul{Я спрашивал Аниса, не попал ли я в мир?  Они говорят `` нет''.}
\people{1-2-3}
\soul{Вы говорите, нет. Вы говорите, что вы не умершие. Хорошо, вы проходили ли вы врата..?(сбой)}
\people{1-2-3}
(о  буквах)
\soul{Назовите следующую букву.}
\people{``Ц''.}
\soul{``Ц''?}
\people{Да.}
\soul{Давайте вернёмся и попробуем, опять же, назовём это чашей, чашей, но достаточно грубой формы: столь угловатой и столь колючей, и столь неустойчивой, что, чаще, приносит вам вреда, чем пользы. Вы становитесь грубыми, грубеет ваше тело, иными словами, стареет, вы уже чувствуете себя не той физикой, вы уже чувствуете усталость. Время уже чувствуется вами, уже волосы ваши седеют, уже мысли ваши стареют, вы уже перестаёте мечтать, и лишь только одно беспокоит вас: побольше прожить или умереть без мучений. Эти углы и, тем более, остриё, колят вас. Вы говорите: ``Боль. Что-то защемило в сердце'' – вы жалуетесь на весь организм, вы жалуетесь на самого себя, вы уже начинаете шаркать ногами, вы уже становитесь брезгливыми ко всему, вы уже начинаете брюзжать, вы уже становитесь нервными, вас не устраивает ничего, вам нужен покой. Вы хотите покоя, вы хотите разобраться. Вы уже постарели, одним только тем, что вы уже вспоминаете, вы стали вспоминать. Встречаясь с друзьями, вы говорите: ``А помнишь?'' – вот вам признак старости. И вы начинаете копошить старое прошлое, вы начинаете ковыряться в этой чаше, выискивать самые прекрасные моменты жизни, когда вам тяжело, чтобы успокоиться. А чаще всего, когда вам тяжело, вы вспоминаете ещё более худшее в прошлой жизни. Для чего? – Для того, чтобы разжалобить себя. Почему вы так любите, вы любите жалеть себя? Вы это делаете с таким успехом, что начинаете плакать и жалеть ещё больше. Вы начинаете обвинять всех: ``Да вот я какой! Да вот я какая! А ты!'' – и так всегда. И маленький хвостик на этой чаше – это всего лишь след, след попытки перевернуть её, сбросить эту чашу, очистить её, чтобы было легче. Это всего лишь пятнышко, это всего лишь царапина на той полированной глади, на той поверхности, которая называется ``жизнь''. И вы хотите исчерпать, исчерпать всё, что в себе. Вы проклинаете память, которая помнит, и почему-то помнит больше плохое. Вот, что буква ``Ц''. Теперь давайте…(сбой)}
\people{1-2-3}
(неизвестное время)
\soul{Кони не были спокойными, они как бы чувствовали, что должны были сделать. Мальчишка не вырывался, он ждал своей участи и смирился с ней. На смотрины пришёл хан. Он сам лично проверил, крепко ли привязан он. Лошадь же, испуская пену, не стояла на месте и…(сбой)}
(неизвестно)
\soul{Как стар мир! Как всё старо! Что(…) все надоели!}
\people{1-2-3}
(на связи Мабу)
\soul{(Далее детским голосом) Петя… Петь!}
\people{(Гера) Да, да, да, здесь…}
\people{(Ольга) Мы здесь. Петя, Петя. Погромче.}
\people{(Гера) Мабу, это ты?}
\soul{Мабу!}
\people{(Гера) Мабу.}
\soul{Почему вы никогда не говорите правильно?}
\people{А как это?}
\soul{Мабу!}
\people{(Ольга) А как говорим мы? Мы и говорим Мабу!}
\soul{Мабу! Но я не обижаюсь.}
\people{(Ольга) Ладно, не будем. А Пете тоже другое имя дали.}
\people{(Гера) Да, я теперь не Петя.}
\soul{Я знаю, у него две жены. А имя?}
\people{(Гера) Как две жены? У меня две жены?}
\people{(Ольга) Имя какое? Которое ему дали?}
\soul{Да.}
\people{(Ольга) Новое? Какое у тебя имя?}
\people{(Гера) Гера.}
\people{(Лена)Георгий.}
\people{(Гера) Гера, так попроще. Ты слышишь?}
\soul{Интересно, а почему?}
\people{(Гера) А почему тебя Мабу назвали, а был Вася?}
\soul{Это  ``Вася'' сказали, чтобы я назвался!}
\people{(Гера) Ну, мне тоже сказали…}
\soul{У меня же не было этого имени!}
\people{(Ольга) У тебя не было… Тебе за что имя дали? Расскажи, за что?}
\soul{За Петю.}
Смех.
\people{(Гера) Ну, вот видишь, а мне за ``Васю'' дали.}
\people{(Ольга) Второе имя ему дали, другое.}
\soul{Не хочу.}
\people{(Ольга) Почему? Другое не нравится имя?}
\soul{Не нравится.}
\people{(Гера) Ну, ладно, я буду Петей.}
\people{(Ольга) Назови Петей, значит, по-старому. Как у тебя дела, расскажи?}
\soul{Хоросо.}
\people{(Ольга) Жёны чем занимаются?}
Конец 3-й записи
1996.01.20-04
\soul{Они не могут со мной ругаться.}
\people{(Ольга) Почему?}
\soul{Нельзя.}
\people{(Ольга) А-а! А ещё  что тебе не нравится?}
\soul{Что мне не нравится? Когда… гром, а дождя нету.}
\people{(Ольга) А-а, это, наверное, у нас тут такие есть животные, которые очень сильно грохочут. Вот, наверное, ты их слышишь. Они людей возят.}
\soul{Нет.}
\people{(Ольга) Нет?}
\soul{Они…}
\people{(Ольга) Что они?}
\soul{Гром гремит, а люди…э-э… ложатся спать и потом не просыпаются.}
\people{(Ольга) А-а, убивают, да?}
\soul{Чего?}
\people{(Ольга) Убивают людей?}
\soul{Не знаю.}
\people{(Гера) А впереди у них есть такой длинный-длинный?}
\people{(Ольга) Палка. Длинная палка есть, из которой стреляют? А-а, гром, гром…}
\people{(Гера) Из которой гром идёт?}
\soul{Палка?}
\people{(Гера) Палка, да! Впереди у них там…}
\people{(Ольга) Палка большая.}
\soul{Я разные видел.}
\people{(Ольга) Разные?}
\soul{И палки видел, и не палки.}
\people{(Гера) Хмм, ясно, ну, понятно…}
\people{(Ольга) Скажи, а какая у людей, какие они? Опиши нам, в чём они одеты?}
\soul{В шкурах. И только почему-то те шкуры…}
\people{(Гера) Все одинаковые?}
\soul{…без травы.}
\people{(Ольга) Без травы? Ага.}
\soul{А ещё… а ещё я видел монахов. А они, а они… я видел ихнюю голову.}
\people{(Ольга) Голову? Лицо?}
\soul{Даже глаза.}
\people{И какие глаза?}
\people{(Гера) На тебя похожи?}
\soul{Разные.}
\people{(Ольга)На тебя похожи?}
\soul{Они не прячутся.}
\people{(Ольга) И что они делают?}
\soul{Учат.}
\people{(Ольга) Кого?}
\soul{Всех.}
\people{(Ольга) Всех?}
\soul{У них есть большие книги. Они держат в руках огонь, чего-то поют, потом все становятся на коленки… а дальше я не видел.}
\people{(Ольга) А-а, вот какие монахи, да? А вы не поёте?}
\soul{Поём.}
\people{(Гера) А что вы поёте? Спой песню какую-нибудь.}
\people{(Ольга) Спой какую-нибудь песню. Молитву.}
\soul{Молитву?}
\people{(Гера) Ну, что вы поёте? Что?}
\people{(Ольга) Что вы поёте?}
\soul{О монахах поём.}
\people{(Гера) Что поёте? Спой.}
\soul{О богах поём.}
\people{(Гера) Ну, ты спой хоть что-нибудь.}
\soul{О жёнах поём.}
\people{(Гера) О женах, что  там поёте? У нас тоже поют о жёнах.}
\soul{О жёнах? Что придёт солнышко, поднимутся жёны, мы им дадим зерна, а они пойдут…эээ…}
\people{(Ольга) Сеять зерно, да?}
\soul{Ну, да!}
\people{(Ольга) Работать пойдут.}
\soul{Да. А потом, будут приносить еду. О детях поём. И детям поём.}
\people{А что о детях?}
\soul{О детях? – Что они станут такими, как мы, потом придут монахи и заберут их богам, и у них там будет жить хорошо. А ещё, а ещё…}
\people{1-2-3}
\soul{Мы знаем о множестве миров. Мы можем высчитать по звёздам время. Мы можем раскрыть книгу и прочитать о каждом. Мы можем предсказывать судьбы, мы можем и учимся, как приносить смерть, достойную…}
\people{(Ольга) Кто вы?}
\soul{Мы можем каждого из вас положить и отправить в иные мира. Мы можем вместе с ним перейти реку смерти и гулять в тех мирах, где живут наши прадеды, где живут наши воспоминания. И мы можем вернуться и рассказать вам. Мы можем увидеть будущее. Мы берём чашу, наливаем её водою святою, окропляем её и видим будущее. Один из учеников взбалтывает воду – и тогда, вода рисует картины. Бывают столь необычные, что не знаем объяснения им. И тогда,  мы составляем книги, и прячем их от других. И книги эти передаются из поколения в поколение, но остаются в храме и никуда более. И в книгах тех все наши труды, все наши желания и всё это – для Бога. Мы, служители его, хотим познать его, он же дарует эти знания нам. И мы берём его, и ложим, крестим руки его, связываем ноги его, при главе его ставим огонь священный, читаем молитвы над ним – и тогда глаза его закрываются, и тогда говорит он нам, что видит он. Он уходит в дали, в те дали, куда не можем мы попасть. Он рассказывает нам то, что не можем видеть мы. И тогда мы записываем всё, и передаём из поколения в поколение. Мы шифруем записи эти, чтобы не попалось врагу нашему, чтобы силы тёмные не могли применить знания те. Мы, подглядывая его, можем творить чудеса. В одном из видений рассказал нам, как можно создать оружие. И мы, взяв тростник и наполнив его ядом, можем теперь защищать храм свой. И благодаря вещим снам его, благодаря Господу Богу, когда просыпается же он, не помнит ничего. И во благо его, иначе пришлось бы, чтобы не было не сохранности тайн, уничтожить его. Счастье его, что не помнит ничего.}
\people{1-2-3}
\soul{Когда стемнело, они сели вокруг стола, положили на него руки и стали смотреть на свечу. Свеча стояла посредине стола. Пламя её билось от ветра. Подтёки свечи давали знать, что скоро начнётся. Каждый с терпением ждал движения стола, и каждый хотел услышать первым, и первым задать вопрос. Это уже было, как в спорте: кто будет первый. И малейшее дрожание воспринималось как начало - задающий задавал вопрос, но в ответ была тишина. Раздавался хохот, остальные смеялись, потом кто-то говорил: ``Спокойно! Давайте настроимся!'' – и снова воцарялась тишина. И снова горел огонь, все смотрели на него и ждали, когда начнётся. Каждый думал о своём. И, хотя было сказано, что ``надо настроиться, и давайте будем думать только о Петре I'', каждый думал о разном. Кто-то думал, что скажет жена, узнав, чем занимается он. Кто-то не мог проблему решить с деньгами, ему завтра нужно было отдать долг. В гимназии другому грозили набить морду, и тогда он стал думать: ``А не пойти ли мне завтра, а не заболеть ли?'' И о Петре I никто, к сожалению, не думал. Может быть, из-за этого Пётр I обиделся и не явился на сеанс, может быть, из-за этого. Но стол не собирался двигаться, свеча уже догорала, сеанс ещё не начинался. Уже пропели петухи, уже проходила ночь, но все терпеливо сидели и ждали. Всё было бы прекрасно, но кто-то уже начинал посапывать, другой толкал его и говорил: ``Стёпка, ты что?! Давай, давай, думай! Я уже, знаешь, сколько думаю?! Мне уже Пётр I чуть ли…'' – и тут же замолкал, потому что он боялся оговориться ``приснился''. И, наконец, уже рассвело. Кому-то первому надоело, он встал, стряхнул руки со стола и сказал: ``Всё это чепуха! Пора, братцы, разбегаться по домам. Мне в два часа ещё свидание''. Другой сказал: ``А я болею'', – ``Ты что, не пойдешь на занятия?'', – ``Не пойду'', – ``Почему?'', – ``Надоело!'' – кому было охота признаваться в разбитой морде? Прекрасно! Всё было бы прекрасно, если бы только не одно ``но'': Сергей был разочарован, он ожидал лучшего, он ожидал чего-то, хотя бы пусть не Пётр, кто-нибудь другой, но хоть кто-нибудь, да будет. Петра не было, и это уже была третья ночь, и, в конце концов, ему уже надоело, и он перестал верить, но согласился на следующий сеанс. Следующий сеанс должен был состояться через неделю. Первые дни шли очень и очень долго, потом Сергей забыл об этих сеансах. Неделя прошла быстро и, наконец, настал новый вечер, пришло время снова кричать Петра. И снова все собрались в кружочек, зажгли свечу, и при этом приготовили запасные – а вдруг разговор затянется. Прекрасно! Все сели, разложили руки и стали ждать. Ждать пришлось недолго. Кто-то сказал: ``Я слышу! Пётр, это ты?'' – в ответ тишина. Опять кто-то захихикал, был слышен окрик: ``Да замолчи ты!''. Сергей подумал: ``Началось! Опять всё повторилось!''. И вдруг , он решил: ``А что, если…'' – и тут же говорит: ``Я слушаю вас, подданные мои!''. Целый гвалт поднялся над столом: ``Пётр, Пётр, это ты?'' – Сергей сказал: ``Да!''. Опять гвалт, опять куча вопросов, и ничего не разобрать, и Пётр, тобишь Сергей, решил навести порядок: ``Бояре, успокойтесь! Задавайте кто-нибудь по одному!'' – что тут началось! Все начали по одному, снова очередной гвалт, снова крик и шум. Пётр разгневался и, стукнув кулаком по столу, сказал: ``Боярин, остригу!'' – всё, это был конец! Тут же поднялся хохот, тут же поднялся визг, одна молодуха закричала: ``Не надо!'' – её успокаивал сосед: ``Да у тебя бороды-то нет, чего ты, старая, испугалась-то?''. Сергей понял, что он, наверное, что-то переиграл. Он тут же попробовал постучать по столу – бесполезно. Стоял большой шум, свеча упала, и это всех остановило. Тут же начали все впопыхах поднимать свечу – не дай Бог, огонь затушится, тогда нельзя, нельзя заново зажигать, да ещё и скатерть накапала, что скажет хозяйка? Итак, всё успокоилось.}
\people{1-2-3}
\soul{``Папа больше не любит нас?'', – ``Не знаю, отстань!''. Мальчишка испугался: ``И мама не любит. А кто же тогда меня любит? Папа не любит, мама не любит''. Он хотел заплакать, но слёз уже не было, и мальчишка тогда испугался ещё больше: ``И слёзы кончились! Чем я плакать буду?''. Это так испугало его, что, наверное, слёзы испугались за него и полились в три ручья. Мать наклонилась, встала на четвереньки и сказала: ``Да не реви ты! Вернётся, куда он, собака, денется?!'' – взяла его на руки, мальчишка обнял её. Ему стало как-то легче, и его сопливый носик натирал ей плечо, но она не обращала на это, лишь только бурчала: ``Я ж потом не отстираю, дурачёк!''}
\people{1-2-3}
\soul{Он был до того гениален, до того прозорлив, что он даже стал разговаривать о будущем, о далёком будущем. Он стал говорить, что будут такие вещи, что бояре смогут перемещаться, куда хотят, и заграница будет в наших руках. ``Пётр'' разошёлся не на шутку.}
\people{1-2-3}
\soul{Именем трёх. Вы будете расстреляны.}
\people{1-2-3}
(1932 смерть Сергея Иванова)
\soul{…был мимо. Крошки от стены больно впились ему в спину, и он был очень удивлён: ``Столько много не попали?''. Шальная мысль пришла: ``Может быть, это просто пугают'' – но нет, комиссар дал новую команду. Снова были возведены ружья, и он видел, как стволы направили на него. И эти чёрные дыры показались ему глубокими, глубокими тоннелями, глубокими ямами, куда он сейчас провалится. Он вдруг вспомнил солнце, что пекло его, и обрадовался, и обрадовался: если бы оно сейчас пришло к нему, если бы он опять оказался бы в этой пустыне, но нет, рядом была стена, и перед ним стояли ружья. И комиссар вытирал платком револьвер и говорил: ``Из-за какого-то гада мне его придётся снова чистить. Легче было заколоть скотину''}
\people{1-2-3}
\soul{Продолжим. Называйте букву.}
\people{(Ольга) ``Ч''.}
\soul{``Ч''. Давайте не будем уже говорить о чаше, иначе это уже навевает на мысль, что ваша вся жизнь – это только всего лишь чаша, которая наполняется чем попало. Нет, давайте представим это немножко по-иному, давайте перевернём эту букву.}
\people{(Ольга) ``У''.}
\soul{И вы уже скажете, что это можно назвать стулом. Стул. Что даёт вам стул? – Опору. Наконец-то вы можете сесть и отдохнуть, в ногах нет правды. Есть время подумать, поразмыслить, отдохнуть. Итак, буква ``Ч''. Вы, линия справа, – говорит о вас. Вы пытаетесь подвести черту, итог вашей жизни. Что вы сделали? И заметьте, за этой чертой нет ничего. Вы пока не гадаете о будущем, вы пока не строите планы о будущем, вы хотите подвести черту о прошлом: что сделали, где были ваши ошибки. Как часто можно услышать вашу фразу: ``Ох, какой я был глупый!'', ``Ах, если бы, да я сделал бы по-другому!'' – и многое-многое ``если'', потом вы останавливаете это резко и говорите: ``Да, всё прошло, поросло травой'' – и понимаете безысходность своего положения. Итак, слово ``чело''. Что обозначает слово это?}
\people{(Гера) Голова.}
\soul{Голова или лицо ваше? Итак… итак, лицо ваше. Иными словами, глядя на лицо, вы можете сказать, что это за человек. Даже если ошибётесь, то не намного. И глядя на вашу жизнь, можно сказать, кто вы. Вы же прячете свою жизнь. У вас есть достаточно много нюансов, которые вы прячете всю жизнь и стараетесь забыть, и чтобы, ``не дай Бог, об этом узнал кто-то, не дай Бог, ещё и ближний человек''. Вы согласны, что у вас множество, множество того, чего хотите спрятать? Черта, та же самая черта. Черта – граница, разделяющая ``было'' и ``есть'', и за этим ``есть'' уже нету ``будет''. Вы хотите разобраться, что именно сейчас, вам не до будущего, вам нужны факты, факты, которые могли бы установить, да кто же вы, в конце концов, что же вы натворили, зачем всё это вы делали. Лишь только потом вы будете задумываться, а что же дальше?}
\people{1-2-3}
(2757 год)
\soul{Дальше он уже не видел ничего. Дальше – для него было понятие такое растяжимое, что он уже не собирался даже думать об этом, ему бы выжить сейчас, ему бы выбраться из этого колодца, ему бы выбраться наружу, и вот тогда, тогда будет ``дальше''. Анур знал: ещё немного, и он уже не сможет отсюда уйти, ещё немного, и в этом колодце он найдет свой конец. Зверь, да будь ты проклят, когда же ты проснёшься – не мог, не мог проснуться, и не мог поднять его. Впервые в жизни Анур жаждал, жаждал, когда он вернётся и поможет ему. Глубок был колодец, и руки, уже искорябанные, не могли поднять его. Сперва Анур кричал, но это было бесполезно, бесполезно, голос его уже охрип, он уже вместо крика что-то хрипел. И только тогда, вдруг он понял: это его смерть, это его смерть. Шарик… Шарик мальчишки, вдруг соскочил с привязки и упал на пол. Анур поднял его и, вдруг он понял, что зверь не придёт. Зверь больше не придёт к нему, и зверь ушёл от него, теперь он… он стал человеком. Он стал иметь силу зверя, но нет того зверя. Анур понял, что в нём проснулся отец, и он понял, что…}
\people{1-2-3}
\soul{…время, весёлые нравы. Сергею было весело смотреть на причуды. Ему было весело, когда учителю в ухо попадала резинка, когда он чесался, озирался и не знал, начать шуметь или остановиться. Иногда учителю надоедало, и он тогда хватал первого попавшего, ставил его в угол и кричал: ``Как жаль, что нету розг!'' – и уже тут шла отборная брань. Сергею доставляло это удовольствие, он стоял в этом углу и был рад, когда учитель бесился, выходил из себя, и думал: ``А ведь я сильнее его''. И оплеухи, что доставались ему, он принимал за благодарность. Это было для него всё равно, что получить крест. И…}
\people{1-2-3}
\soul{Всё бы хорошо, но у этой куклы не хватает красок, она слишком облезла. Ты бы разукрасил её, чтобы она была более похожа на что-то. Что за чучело?! Оно умеет разговаривать? Ну и что? Любая пружинка – и она начинает у тебя уже заикаться, как старый заезженный патефон. Так ты хоть покрась, чтобы она улыбалась, матрёшкой что ли была. Что за чу….}
\people{1-2-3}
\soul{…не бывает. Чаще всего…}
\people{1-2-3}
\soul{…столь интересно, что про ``Петра'' уже и забыли. Сперва он назвал себя монахом. Потом, на следующем сеансе, он заявил, что он, именно он, вывел евреев из пустыни. Это уже был перебор. Уже начались сомнения. Слава Богу, что к Сергею это не относилось, все считали, что этот монах, что этот экземпляр и брешет. Сергей же научился так ловко подделываться, жаль только что он ещё не умел менять голоса, это было бы вообще замечательно. Что ``Пётр'', что ``батюшка'', что какая-то умершая ``боярышня'' – все говорили одним и тем же голосом, и, причём матерились почти одинаково. Что самое интересное в этом было – что все они могли предсказывать будущее и очень хорошо знали всех окружающих, но очень плохо знали тех, кто вновь прибыл. И когда какая-то новенькая спросила что-то о своём муже, то бедному монаху пришлось попотеть и выговорить целую кучу слов, чтобы просто взять да ``около'', да ``вокруг''. Но боярышня была очень довольна и сказала: ``Ты смотри-ка, ведь всё знает, паразит! Вы извините, что я вас обозвала''. ``Паразит'' не обиделся, ``паразиту'' даже было очень интересно, и он сказал: ``Ничего! Я уже привык''. И так, было…}
\people{1-2-3}
\soul{…повторить. Ах, если бы это всё заново прожить! Так это, наверное, было бы не интересно.}
\people{1-2-3}
(1248 год)
\soul{Море словно почувствовало, что ей дают дань. Она подхватила плот, отогнала его от берега, и пены волн разбивались о подножия креста, но не тушили огонь. Деревенщина была слишком хитра, и волны не доставали огня. Огонь был слишком слаб, чтобы сжечь Анура сразу, и это было больно. Ему было больно, потому, что языки пламени жгли его ноги, но не могли сжечь его. Но он не кричал, он не мог кричать, он слышал мать! Он слышал мать, и она кричала ему: ``Сынок, дай руку, сынок!'' Он растерялся и не мог дать её. Он стал объяснять ей, что руки связаны, она кричала ему: ``Да не те, не те! Оставь!'' Он не мог её понять. И тогда она пыталась затушить огонь, но что может призрак сделать с настоящим огнём? Ничего… Огонь только больше разжигался, всё больше и больше пламени языки жгли его, жгли. Если бы была на нём одежда, было бы, наверное, легче. Он был гол, он был гол, и почему-то вдруг ему стало стыдно, стыдно, что он без одежды. Ему почему-то стало стыдно, что он придёт к Богу голым, обнажённым, и Бог скажет: ``Где одежды твои?''. Что скажет он? – Что они не были одеты? Что скажет он? – Что он был изгоем? Что он убил свою мать? Примет ли его Бог?}
\people{1-2-3}
(1932 смерть Сергея Иванова)
\soul{…не ждал, не ждал ответа. Он одновременно мог верить в него, и нет. Он тут же говорил: ``Нет Бога!'' – и тут же говорил: ``Есть!'' `` Боже, если ты есть, убери, убери эти пушки!'' То ли Бог не слышал его, то ли его не было, то ли комсомол говорил в его груди и не давал услышать Бога, но эти пушки были направлены в сердце…}
\people{1-2-3}
(2757 год)
\soul{… сердце зубастого должно было отдаться. Анур, спрятав его в котомку, набросив за плечё, и стал думать, как ему выйти, как ему выйти обратно. Обратно уйти с балкона было нельзя – слишком много было шума на них, пойти по лестницам – там слишком много тварей, уйти по шахте – интересный вариант, но он не знал, как это сделать. Да, с ним были верёвки, но он не знал, хватит ли их длины.}
\people{1-2-3}
\soul{…”Сумма длинны катетов гипотенузы…”- ему снилась страна квадратов. Мир был плоский, и лишь только он один объёмный шагал по этому миру и кричал: ``Ну, поглядите! Ну, что ж вы не видите меня!'' – но все видели только ступни. И тогда он говорил: ``Но, это же геометрия! Это же просто геометрия!''  И какой-то толстый жирный квадрат сказал: ``Ты ещё сопляк - учить нас!''. На этих словах он проснулся. Проснулся, встал с постели, подошёл к окну и увидел луну. Вся детвора спала. Даже нянька, упёршись о тумбочку, сопела, что-то бурчала под нос. Стояла тишина. Луна, она была столь большая, что испугала мальчишку. И он, испугавшись, задёрнул занавеску. Гардины загремели, нянька подняла голову: ``Что ты делаешь? Почему ты не в постели?'', – ``Я, я, я… хочу в туалет'', – ``Хорошо, пошли''.}
Конец 4-й записи
1996.01.20-05
\soul{Я буду читать учебник. Наступило время очередного сеанса. Всё было, как обычно: обычная свеча, обычные люди, были все те же самые, на сей раз, новых не было, чтобы не вносить неразбериху. Итак, на сеанс вышел какой-то китаец, столь древний, что уж сам запутался в датах, но ничего. Китаец довольно-таки славно, довольно-то культурно, что удивительно, без ругани, стал отвечать на вопросы. Сергей пытался вмешаться, он умудрился даже несколько фраз сказать из учебника, но только и всего, только и всего. И причём, что самое удивительное, то ли действительно так, то ли нет, но те фразы совпали с ответом. И тогда Сергей решил: ``Нет, хватит! Крыша поехала''. После сеанса он сказал: ``Знаете, я чувствую, что пора. Давайте бросать это дело!'', – ``Да ты чё! Да ты чё, это же только начало! Ты представляешь, сколько много мы можем узнать?! Да ты что, Серёга!'' – Андрей был в ударе, он стал ему доказывать, он стал приводить обрывки каких-то монологов, принесённых когда-то Сергеем. ``Андрюша, может быть, хватит?'', – ``Да ты что, да мы можем… Ты не представляешь, что это будет! Да мы напишем книгу!'', – ``Книгу?'' – эта фраза остановила Сергея, он стал думать: ``Книгу. Где-то это я уже слышал''. ``Ребята, а я не про какие книги не говорил в сеансах?'', – ``Да какие книги! Да брось ты эти книги! Давай лучше договоримся, когда ещё будет!'' – Сергей развернулся и пошёл. В дорогу кричали: ``Дурак, вернись, дурак! А, куда ты денешься?!''. Да, действительно, Сергей никуда не делся, при первом же зове, при первом же крике он прибежал, он сам настраивал свечу, он сам садился за стол и стал отвечать. Но разница была в том, что он уже не мог помнить, о чём говорил он. И тогда он понял: ``Это уже всё! Теперь - только больница''. Друзья успокаивали, особенно очень сильно старался Андрей: ``Серёга, да брось! Да ты знаешь, да мы же станем, мы организуем!'' – множество, множество слов Андрей произносил в защиту сеансов. Сергей слушал их, слушал. Одна какая-то девчонка, Сергей не помнил её имени, кричала: ``Я ведь всё записываю!'' – и действительно, она всё записывала. Она записывала вплоть до чиханий и плевков! Это был удивительный конспект, в нём было всё расписано по времени, по датам, и каждый чих.}
\people{1-2-3}
\soul{Мабу!}
\people{(Ольга) Мабу!}
\people{(Гера) Мабу!}
\people{(Ольга) Мы уже по тебе соскучились!}
\people{(Гера) Как давно ты нас не видел!}
\people{(Ольга) Да!}
\people{(Гера) Сколько дней?}
\soul{Много. Две тьмы!}
\people{(Ольга) Ооо… Вот это да!}
\people{А сколько у тебя жён?}
\soul{Осталась одна. }
\people{(Ольга) Опять одна? Нет, ну, ты нам тогда говорил, что…}
\people{(Гера) Семь было, вроде.}
\soul{Ой, какая у вас память! Ну, я же говорил вам, что их забрали!}
\people{(Ольга) Да, да, всё правильно! Мы просто пошутили.}
\soul{Я и то помню, что у вас (…) новое имя. А жён у меня, всё-таки, было больше.}
\people{(Ольга) Ого! А значит, ты нам врал, да?}
\people{Ты нас обманул!}
\people{(Ольга) Обманывал нас, да?}
\soul{У меня их было…}
\people{(Ольга) Считай, считай!}
\soul{Вы говорили, и я забыл.}
\people{(Ольга) Три!}
\soul{Семь!}
\people{(Ольга) О-о-о, молодец! Я помню!}
\soul{А у вас только две! У меня больше! Вы, наверное, жадные, и вам не дают жён.}
\people{(Ольга) Мабу, а ты сказал монахам, что ты до семи умеешь считать? Похвастался им?}
\soul{Нет.}
\people{(Ольга) А почему?}
\soul{Не знаю.}
\people{(Гера) А ты скажи, скажи! А то у них рот откроется и не закроется больше.}
\people{(Ольга) Ты глазки рисуешь?}
\soul{Рисую.}
\people{(Ольга) И какие у тебя глазки получаются? Круглые?}
\soul{Они в головку не помещаются!}
\people{(Ольга) Вот это глаза ты рисуешь! Как же не помещаются? У тебя…}
\soul{Не знаю.}
\people{(Ольга) У тебя же глаза в голове помещаются!}
\soul{Да, там помещаются, а здесь нет.}
\people{(Ольга) Значит, внимательней надо быть.}
\soul{Я хитренький, я сперва глазки рисую, а потом головку. Тогда всё помещается!}
\people{(Ольга) Молодец! Ты глянь, какой ты умный!}
\soul{Да, животик.}
\people{(Ольга) Животик! А ручки-ножки научился рисовать?}
\soul{….тогда животик не помещается.}
\people{(Ольга) Вот, теперь животик. Вот, видишь, как у тебя не получается. А зеркало у тебя есть?}
\soul{Чего?}
\people{(Ольга) Ну, зеркало, в которое ты смотришь…}
\people{Камень. Себя видишь…}
\soul{Да.}
\people{(Ольга) Нет, нельзя уже, всё. Монахи не дают?}
\soul{Смотреть можно, а иметь нельзя.}
\people{(Ольга) А ты пробовал в реку пойти посмотреть? Знаешь, когда в реку смотришься, ты себя видишь. Не пробовал?}
\soul{Нет.}
\people{(Ольга) Вот подойди к реке, чуть-чуть наклонись к воде, и увидишь там отражение – это будешь ты!}
\soul{Дураки вы что ли?! Меня ж побьют!}
\people{(Ольга) За что тебя побьют?}
\people{(Гера) Когда поведут, посмотри.}
\people{(Ольга) Когда поведут умываться, посмотри!}
\soul{Да?}
\people{(Ольга) Да! Посмотри – ты там себя увидишь, какой ты большой.}
\soul{Врёте вы всё!}
\people{(Ольга) Да нет, мы не всём! А ты ещё расскажи нам, где ты ещё был.}
\people{(Гера) В священной пещере был уже?}
\soul{Был.}
\people{(Гера) Слушай, а расскажи, что это за длинное эхо там? Как оно выглядит?}
\soul{Не знаю.}
\people{(Гера) Ну как квадратное? Круглое? Как голова? Или как?}
\soul{Не знаю. Я там говорю, а она…}
\people{(Гера) Повторяет?}
\soul{Да, даже больше, чем моих жён.}
\people{(Ольга) О-о, даже больше, чем твоих жён у тебя?}
\soul{Да.}
\people{(Ольга) Сколько больше раз? Мабу, скажи…}
\soul{Мабу!}
\people{(Ольга) Мабу! А мы говорим ``Мабу'', а как? Мабуу! Так?}
\soul{Не правильно.}
\people{(Ольга) А как? Мабу.}
\people{(Гера) Моба.}
\soul{Сам ты Моба! Мабу!}
\people{(Ольга) Мабу! Скажи, а когда ты в пещере был, где… где там, ты испугался, помнишь ты это? Или так не ходил в пещеру эту?}
\soul{Какую?}
\people{(Ольга) Не ходил? – Молодец, мы тебе сказали, чтобы ты не ходил…}
\people{(Гера) Не ходи никуда, не суй нос свой куда не следует.}
\people{(Ольга) Ну, ты где ещё бываешь, расскажи. Чё ты видишь? Мы тоже…}
\soul{Я?}
\people{(Ольга) Да.}
\soul{Я видел солнце.}
\people{(Ольга) Солнце?}
\soul{Да.}
\people{(Ольга) Где?}
\soul{В небе.}
\people{(Ольга) В небе?}
\people{(Гера) Раньше не видел?}
\soul{Не-ет.}
\people{(Ольга) Какое?}
\soul{Два солнца!}
\people{(Ольга) Ааа, два солнца?}
\people{Одно – когда темно, другое – когда светло.}
\people{(Гера) Да?}
\people{Одно похоже на огонь.}
\people{(Ольга) Нет? Два сразу?}
\soul{Нет.}
\people{(Ольга) А как?}
\people{(Гера) По очереди?}
\soul{Одно колючее.}
\people{(Ольга) Одно колючее?}
\soul{Да. А холодное.}
\people{(Гера) Белое, да?}
\soul{Чего?}
\people{(Гера) Белое, как облака, да?}
\people{(Ольга) Светлое? Светит оно? Светит? Как огонь?}
\soul{Нет.}
\people{(Ольга) Нет? А как?}
\soul{Большое, оно колючее, на него смотреть нельзя, а другое – красивое.}
\people{(Ольга) На него можно смотреть?}
\soul{Только его всё время кто-то ворует.}
\people{(Ольга) Это ночью, да?}
\people{1-2-3}
\soul{Следующая цифра.}
\people{(Ольга) Цифра? Буква!}
\people{(Гера) Буква!}
\people{(Ольга) Мы закончили на букву ``Ч''. Следующая буква ``Ш''.}
\soul{Иначе, та же буква ``Е'', только всё наоборот. Мир перевернулся, в вашем представлении. Вы уже видите мир иными глазами. Чаще, к сожалению, глазами старца… Старца не по уму, а по возрасту. Старца уставшего, опротивевшему всё и всему, и вы хотите…}
\people{1-2-3}
\soul{Буква ``Ш''. Шар. Что в вашем понятии шар?}
\people{(Гера) Идеальная форма.}
\soul{Идеальная форма. Иначе, для вас - есть идеал. Есть идеал, которому вы хотели бы, если не покланяться, то хотя бы быть похожим на него. Вы создаёте идеал в себе, и возмущаетесь, когда этот идеал не совпадает с чужим. Но не так уж и сильно вы спорите. Вы уже знаете, что в споре рождается не истина, а вражда. После хорошего спора вы расстаётесь врагами, так и не поняв друг друга. И этот идеал…}
\people{1-2-3}
Конец контакта.
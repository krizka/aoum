Аоум. глава 11. 06. 1996 г
Георгий Губин
VG – 1996.06.11_-_01                                          
\people{(Белимов)…сеанс.}
[Щелчок - вкл/выкл записи]
\soul{… Не быть в головах.}
\people{(Белимов) Я об этом тоже подумал.}
\people{(Ольга) В ногах или в головах?}
\people{(Белимов) В головах…в головах.  Мы можем продолжить?}
\soul{Да.}
\people{(Белимов) Вы ждёте от нас вопросы, или сами можете задать тему?}
\soul{Нет, задавайте вы.}
\people{(Белимов) Это наши обычные партнёры, так мы понимаем?}
\soul{Да. }
\people{(Белимов) Тут, любопытный есть цикл вопросов от наших коллег из Приморья. Вы знаете…?}
\soul{И мы уже отвечали вам.}
\people{(Белимов) Отвечали на них? }
\soul{Да. Вы повторяетесь?}
\people{(Ольга Белимову) Имена не называйте. (шёпотом)}
\people{(Белимов) Ну, такие вопросы нам… Например, их интересовало – из чего складывается образ Христа? Это можно ответить?}
\soul{Мы будем за вас отвечать, о вашем духовном. Хорошо. А что тогда останется вам?}
\people{(Белимов) Ну, мы сравниваем ваши познания о земной цивилизации со своими. Со своим видением. То есть, с двух сторон рассматриваем вопрос. Это тоже же нам поможет.}
\soul{Да нет, вы его не рассматривали ещё ни с какой стороны. Будьте откровенны.}
\people{(Белимов) Ну, наверно.}
\soul{И когда-то, мы уже множество раз отвечали вам о Христе.}
\people{(Белимов) Угу.}
\soul{И вы спрашивали о датах, и мы отвечали вам. Вы не помните?}
\people{(Белимов) Ну, помню, что даты мы никак не можем окончательно вычислить. Они имеют большое значение.}
\soul{А мы когда-то вам говорили, что был просто обычный человек, и в нём родился Христос. Что должно случиться с каждым из вас. А вы не помните.}
\people{(Белимов) Угу. Скажите, какова роль архангелов? По-моему, это ещё не задавали. }
\soul{Разве? А зачем вы имеете друзей?}
\people{(Белимов) Чё?}
\people{(Татьяна Белимову) Друзей, зачем вы имеете?}
\people{(Белимов) То есть, их роль такова - как у земного человека друзья, да?}
\soul{Они в первую очередь – друзья. И лишь только потом будет - всё остальное. А вы же - привыкли к ``командным аппаратам''.}
\people{(Белимов) Угу.}
\people{(Ольга) Вот, извините, я могу задать вопрос? Нам ``переводчик'' в прошлый раз, когда мы с ним разговаривали, он нам так сказал, что мы, вроде как… Или вы, я уже не помню… Что мы называем вас ``друзьями'', а вроде как мы и не друзья что ли, или как вот? Может я неправильно поняла?}
\people{(Белимов) Вы дистанцируетесь, нас это немножко удивило.}
\soul{Удивило? А мы говорили о вседозволенности.}
\people{(Ольга) Ну, это… То есть, если люди - друзья, это не значит, что можно вообще всё с этим человеком, так сказать, всё что захочешь. Но ведь, это в первую очередь уважение, и вообще, так сказать… }
\soul{А какое имеете уважение вы?}
\people{(Ольга) К вам?}
\soul{К нам.}
\people{(Белимов) Ну, мы как раз к вам относимся, как к более просвещённой и старшей цивилизации, и поэтому , конечно, у нас уважительное отношение.}
\soul{Тогда ведёте как дети, забываете что вам говорят,…}
\people{(Белимов) Ну, мы не можем…}
\soul{…Делаете, что хотите и как хотите. И сколько раз уже предупреждали вас, и вы повторяете. И поэтому мы будем держать дистанцию. Вы привыкли к ``командным аппаратам''? - Привыкли. А будет для вас начальник - другом, вы уже больше себе позволите?}
\people{(Ольга) Ну, да.}
\soul{Ибо у вас понятие дружбы…}
\people{(Ольга) Это понятие вседозволенности, да?}
[Даёт счёт: 1-2]
\soul{Говорите.}
\people{(Белимов) А скажите, вмешивались ли ``чёрные силы'' в процесс создания Библии?}
\people{(Ольга Белимову) Уже нам сказали… [шёпотом]}
\soul{А что вы называете ``чёрными'' и ``светлыми'' силами? Вы, хотя бы различаете их?}
\people{(Белимов) Ну, пытаемся различать.}
\soul{Пытаетесь?}
\people{(Белимов) Ну, ``тёмные силы'' несут, в основном, зло…}
\soul{Да?}
\people{(Белимов) …подрывление энергетики, а светлые силы…}
\soul{По-вашему, добро будет ``сюсюкаться'' с вами? А удары судьбы, как вы говорите - ``учат вас'', от добра или от зла?}
\people{(Белимов) Да, скорее всего - от добра.}
\soul{А когда вам дают подачки, это добро или зло?}
\people{(Ольга) Зло.}
\people{(Татьяна) Зло, конечно.}
\soul{А вы умеете различить подарок - от подачек, заслуженное - от взятого?}
\people{(Ольга) Не всегда. Нам, может быть нравится принимать подарки. И наверное, это нашe… вот в этом…}
\soul{Когда-то мы говорили вам о картине. Помните?}
\people{(Ольга) Да.}
\soul{И когда придут, отберут у вас силу… Представьте, у вас отобрали – ``чёрные силы''. Вы не зная того, почувствуете себя бессильными. И что скажете? -Что были - ``чёрные''.}
\people{(Ольга) Угу.}
\soul{Когда-то говорили вам о силе. Как примените, так и будет вам.}
\people{(Ольга) Да.}
\soul{Сила – она нейтральна. Как и любое оружие, любой предмет - вы можете превратить и в добро и во зло.}
\people{(Белимов) Но, всё-таки, главная задача человека - противодействовать, и может быть, активно силам зла и бороться с ними?}
\soul{Этого пока ещё не заметно. Пока что, вы на стороне зла и довольно активно. }
\people{(Белимов) Угу.}
\soul{Хотя бы, взять все ваши науки - на уничтожение вас самих.}
\people{(Белимов) Ну, это справедливо. А вы проглядываете такие моменты, когда мы всё-таки перейдём на более гуманное отношение и к цивилизации, и к другим мирам?}
\people{(Ольга) К жизни.}
\soul{Если успеете.}
\people{(Белимов) То есть, вообще-то… перед вами… вы видите, что Библия заканчивает срок действия в 2000-м году, и вы не исключаете…?}
\soul{Мы не говорили о сроках. Мы говорим о вас. Вы знаете, что многое плохо. Вы знаете, одну из заповедей – ``беречь себя''. А вы бережёте? Вы знаете, что это вредно, но с удовольствием делаете. Что такое жизнь? Вы понимаете, что такое жизнь?  – Нет. И потому, вы уничтожаете её с момента рождения. Вы только родились, и уже совершаете ошибки. Потом, эти ошибки переходят в привычку. А потом, эти привычки, уже руководят вами. И вы, уже не можете без них. Из них, вы уже получаете удовольствие, хотя знаете и о вреде.}
\people{(Ольга) Скажите, вот о дьяволе. Изначально, вот, - что такое дьявол? У нас очень много, всяких таких интерпретации что ли. Дьявол – это вот, зло там, - то… Церковь это объясняет по-своему там…}
\soul{Да? И ``сын утренней звезды''.}
\people{(Ольга) Ну, да. Вы ответили, да?}
\soul{Мы вам говорили - падший ангел. Будьте внимательны. Иначе - вы, проклиная его, вступите на его дорогу и будете его помощником.}
\people{(Ольга) А, то есть, даже вот такие…ну…}
\soul{Если вам говорят ``Не убий!'' – Это относится ко всем. Если вам говорят ``Возлюби!'' – Это относится ко всем. А если вы  -“Этого люблю, того — нет'', тогда вспомните - ``Не суди!”}
\people{(Белимов) Ну, ведь роль Дьявола можно оценить как позитивную. Он даёт различие зла от добра, то есть…}
\soul{Простите, сила - нейтральна. И овладевая ею, и смотря в какую сторону, вы будете сами называться, или ``дьявол'', или кем-то другим. А вы же, найдёте множество причин, что не вы - виновны. – Дьявол! - Дьявол вас попутал. А забыли, что сами подали ему руку и повод!}
\people{(Ольга) Так, дьявол овладел силами, которые были, да? }
\soul{Вы - овладели силами. И вы, отдаёте эту силу, как вы говорите – ``тёмному''.}
\people{(Ольга) Угу.}
\people{(Белимов) Сейчас в России, скорее всего, руководства больше дьявола… дьявол имеет влияние, да?}
\soul{А что в вас лично, как вы думаете?}
\people{(Белимов) Да, хотелось бы думать, что более светлых сил, но… тоже есть неуверенность.}
\soul{А что заставляет вас сомневаться?}
\people{(Белимов) Ну, церковники прямо говорят, что те, кто занимается подобными контактами, это от бесовских сил. Может им виднее? Мы тут не арбитры.}
\soul{Что заставляет вас сомневаться?}
\people{(Белимов) Просто, исследовательский характер такой.}
\soul{Всего лишь? Вы, больше подвластны добру или злу?}
\people{(Белимов) Я думаю, что – добру.}
\soul{Но почему нет уверенности?}
\people{(Белимов) Ну, могу и уверенней сказать: Да! Добру!}
\soul{И вы солжёте. И то - начало, которое можно было отнести к добру - ваше сомнение,-  вы тут же губите, соглашаясь на утверждение извне, но не в себе. Вот это и есть - ложь. Вот это и есть - попытка сознания пристроиться.}
\people{(Белимов) Ну, вообще-то, сомнения тоже не относят к положительным чертам исследователей или вообще, человеческим чертам.}
\soul{Если бы вы были слишком тверды, никаких бы не было бы исследований.}
\people{(Ольга) Это точно. Только сомнения заставляют человека двигать…?}
\people{(Белимов) Ну, вот, я например, и руководствуюсь в своих исследованиях, но, а вдруг кому-то виднее? Той же церкви виднее, что мы занимаемся не хорошим делом. Тогда надо прекращать.}
\soul{Да! И для себя, вы не соглашаетесь. Но чтобы не прервать разговор и не испортить настроение друг другу, вы дадите согласие. Что это? Это есть  обман. Вот вам - уловка дьявола.}
\people{(Белимов) А как надо поступать вообще? С положительных сторон вот. Быть последовательным? Всегда последовательным, без сомнения?}
\soul{Меньше лгать и себе, и другим. И тогда вам  будут - менее.}
\people{(Белимов) Ну, тут трудно критерий выбрать, или лжёшь, или стараешься…}
\people{(Ольга) Скажите, вот, ну так, условно у нас разделяется на семь сил. И существует Семь Кумаров, ну, такое название что ли, ну…вообще. И вот один из этих семи сил и есть дьявол?}
\soul{Вы почти ответили сами. Семь стадий сна - вас устроит такой ответ?}
\people{(Ольга) Да. Наверное - да. }
\soul{И вам нужно, избавиться от него. А иначе – проснуться.}
\people{(Ольга) Угу. Это у нас картины пишутся…вот…некоторые художники…вот, ``Спящая царевна'' и сказки такие есть. Вот, наверное, как раз где-то в этом что-то есть…}
\soul{Шёпот. Проявление шёпота.}
\people{(Белимов) Тут у меня комплекс вопросов ещё, по книге Юрина. Каков механизм передачи чувств на большие расстояния, например, ощущение гибели родного человека?}
\soul{Любовь. Вы никогда не задумывались об этом, что это одно из сильнейших чувств?}
\people{(Белимов) Да. Представляем, конечно. А он, материально получается, что любовь может: ощущаться, передаваться, - то есть это всё от биополей, так что ли?}
\soul{Простите, вы умеете управлять ощущениями?}
\people{(Белимов) Да, в основном.}
\soul{Ну, представьте, гипнотизёр избавляет вас от боли, или заставляет вас увидеть сквозь стены. И представьте теперь, более сильнейший гипнотизёр – любовь,- заставляет сделать вас, то же самое. И что в этом удивительного? Удивительность в том, что вы не умеете всё это делать, хотя обладаете. Удивительно, что вы это всё попрятали по ``шкафчикам'' и не достаёте. И множество причин, чтобы не достать. Страх – первое, и второе – та же любовь.}
\people{(Белимов) Мы ещё не знаем механизма воздействия вот этого – на расстоянии. И хотим его определить. Может быть он всегда…}
\soul{А мы вас предупреждали, придёт время, когда вы захотите - создать самих себя. Вы помните?}
\people{(Ольга) Да-да-да. Было-было-было.}
\soul{Вот вам - попытка, просто физически эмулировать и не больше.}
\people{(Белимов) Скажите, почему…?}
\soul{Постепенно, вы превратитесь в роботов. И почему вы пока не они? Да потому, что у вас ещё слабо пока с интерфейсами. Скоро они будут встроены в вас, и вы будете просто ``машинами'', когда тут же…(теряется)}
\people{(Ольга) [Даёт счёт: 1-2]}
[Щелчок - вкл/выкл записи]
(Диалог с Мабу)
\people{(Ольга) Здравствуйте!}
\people{(Белимов) Мабу?}
\soul{(Мабу) Мабууу.}
\people{(Ольга) Мабууу. (смех)}
\people{(Белимов Ольге) Ну, давай.}
\people{(Ольга) Мабу? А Мабу?}
\soul{(Мабу) Чего?}
\people{(Ольга) (Cмех) Мы по тебе соскучились.}
\soul{(Мабу) Чего это вы ``соскучились''? Вчера разговаривали, уже соскучились.}
\people{(Ольга) Ну-у…}
\people{(Белимов) У нас проходят большие времена.}
\people{(Ольга) Да… У вас - вчера. А у нас знаешь, сколько времени уже прошло?}
\people{(Белимов) Вчера, это сколько у тебя жён было?}
\soul{(Мабу) Так не бывает.}
\people{(Ольга) Бывает.}
\people{(Белимов) Ты, пока в пещере или с монахами?}
\soul{(Мабу) Чего?!}
\people{(Татьяна) В пещере…}
\people{(Ольга) Жён то, у тебя сколько, вчера было?}
\soul{(Мабу) Три.}
\people{(Ольга) Вчера, сколько было?}
\people{(Белимов) Ну, три. Всё ясно.}
\soul{(Мабу) Ой, какая у вас память!…(удивляется плохой памяти участников)}
\people{(Ольга) Память плохая? (смех) – Да-а!…Мабу, представляешь? Почему вот,… не знаю, почему у нас такая память плохая. Вот у тебя лучше, чем у нас. Ты лучше помнишь, да, чем мы?}
\soul{(Мабу) Я так быстро не забываю!}
\people{(Ольга) Ну, у вас была вчера. А у нас было уже… Сколько времени прошло? Знаешь сколько?}
\people{(Белимов) Несколько месяцев.}
\soul{(Мабу) Так не бывает!}
\people{(Ольга) Не бывает?}
\soul{(Мабу) Не бывает.}
\people{(Белимов) А ты сегодня один лёг самостоятельно или с монахом?}
\soul{(Мабу) Сплю.}
\people{(Белимов) Спишь? (Удивлённо)}
\people{(Ольга) Значит, во сне и должен видеть. Ты значит, нас видишь, да? А?!}
\soul{(Мабу) Сплю! Вижу!}
\people{(Ольга) Видишь! А сколько нас? Ну-ка, давай! Теперь говори, сколько нас?}
\soul{(Мабу) Три.}
\people{(Ольга) Три?! (Удивлённо)}
\soul{(Мабу) Да.}
\people{(Белимов) А он больше не умеет считать, наверное.}
\people{(Ольга) Да-а… Это у тебя три жены, поэтому ты до 3-х считаешь, да?}
\soul{(Мабу) Почему? (обиженно)}
\people{(Ольга) Как почему? Ты же сказал, ты умеешь считать столько, сколько у тебя жён.}
\soul{(Мабу) Ну, сказал.}
\people{(Ольга) Ну-у! }
\soul{(Мабу) Ну-у и чё?}
\people{(Ольга) А чё, больше не умеешь считать?}
\people{(Белимов) Ты ошибся. Нас не трое, нас больше.}
\soul{(Мабу) Три! Я умею считать.}
\people{(Ольга) ``Три''?}
\people{(Белимов) Кого-то не считает - значит. (Смех)}
\people{(Ольга) А кто у нас? Сколько у нас жён?}
\people{(Белимов) Здесь.}
\people{(Ольга) Жён сколько, Петиных? (“Сколько лиц женского пола?'' прим.)}
\soul{(Мабу) Три!}
\people{(Ольга) ``Три!”}
\people{(Белимов) Всё правильно.}
\people{(Ольга) А ещё? Кто ещё есть? Петя? }
\soul{(Мабу) Нету Пети.}
\people{(Ольга) А кто есть?}
\soul{(Мабу) Жёны есть.}
\people{(Ольга) И всё?}
\people{(Белимов) Ну, ладно, бог с ним.  Мабу! (обращается к Мабу прим.)}
\soul{(Мабу Ольге) Чё, сама не знаешь сколько вас?! (возмущённо)}
\people{(Ольга) Нас – четыре. Вот, мы, поэтому и говорим - почему ты нас ``три'' видишь. ``Жён'' видишь - трёх, а одного - не Петю, другого, ты его имя не знаешь,…(не видишь прим.)}
\soul{(Мабу) Значит, плохо снится.}
\people{(Ольга) Плохо снится? Всё ясно! (Смех)}
\people{(Белимов) М-м… Мабу, а что тебе во снах снится? Ты можешь нам рассказать какой-то сон последний, который тебе запомнился хорошо?}
\soul{(Мабу) Ну, я же вчера вас видел!}
\people{(Белимов) А ты во сне только нас видишь? Может быть, другие какие-то ситуации?}
\soul{(Мабу) Жён вижу.}
\people{(Белимов) Жён?}
\people{(Ольга) Своих жён?}
\soul{(Мабу) Да.}
\people{(Ольга) Во сне?}
\soul{(Мабу) Да.}
\people{(Ольга) А каких? Какие они, как…?}
\soul{(Мабу) Старейшину видел. }
\people{(Белимов) Старейшину?}
\soul{(Мабу) Я его побил.}
\people{(Ольга) У-у!… Во сне? Побил!?}
\soul{(Мабу) Попробовал бы я не во сне его бить.}
\people{(Ольга) (Смеётся)}
\people{(Белимов) Как же ты посмел?}
\soul{(Мабу) А во сне всё можно!}
\people{(Ольга) А-а…}
\people{(Белимов) Во сне можно… Ты не рассказываешь о том, что ты во сне побил старейшину? Нет?}
\soul{(Мабу) (Запинается)}
\people{(Белимов) Нет?}
\soul{(Мабу) Я рассказал… }
\people{(Ольга) Кому?}
\people{(Белимов) Жене?}
\soul{(Мабу) А во сне ``не считается''.}
\people{(Белимов) М-м…}
\people{(Ольга) (Смех) Мабу, а ты расскажи, а чё ты во сне вчера видел? Как ты нас видел во сне? Ну-ка расскажи, напомни нам. У нас же плохая память.}
\soul{(Мабу) Чего?… Вопросы задавали.}
\people{(Ольга) На какие? }
\soul{(Мабу) Глупые.}
\people{(Ольга) О-о!… Правда?}
\people{(Белимов) А какие ``умные'' надо задать, на твой взгляд? Что надо тебя спросить, подскажи нам?}
\soul{(Мабу) Не знаю, что надо спросить, но дразнится - тоже нельзя.}
\people{(Ольга) А как мы дразнились?}
\soul{(Мабу) Что я монахом буду…}
\people{(Хором) А-а!}
\people{(Белимов) Это, мы тебе предсказываем. Сны-то, они имеют предсказательное значение.}
\soul{(Мабу) Что, я старейшину буду бить?}
\people{(Белимов) Нет, ты не будешь его, но…}
\soul{(Мабу) Я бы его побил, конечно!}
\people{(Белимов) Но, со временем ты станешь сам – старейшина. Будешь делить зёрна и воду.}
\soul{(Мабу) Ну, вот! Не буду с вами. Всё!  (Обиженно)}
\people{(Белимов) Мы тебе предсказываем это.}
[Счёт: 1…]
[Обрыв записи]
[Меняется интонация переводчика]
[Время нашествия Берков]
\soul{…Сора у вас. Он всё рассказал Горе, но Гора не поверил…}
\people{(Ольга) [даёт счёт: 1-2…]}
[Щелчок - вкл/выкл записи]
[Меняется интонация переводчика]
\people{(Белимов) Хорошо, мы продолжим. Почему, нередко, бывали случаи - именно дети хорошо чувствуют гибель близких?}
\soul{Потому, что у них меньше сознания.}
\people{(Белимов) И они лучше ощущают, биополя?}
\soul{Давайте скажем так:  чем взрослее вы становитесь, тем больше ваше сознание ``сковывается'' внешними обстоятельствами.}
\people{(Белимов) Угу.}
\soul{Будучи в свободном состоянии  – в свободном сознании, что было в прошлый раз,- вы, будете чувствовать гораздо больше. Хотя, меньше конечно, чем подсознание. Но всё же.}
\people{(Ольга) Вот, как раз, нам нужно от этого избавиться, от этих ``оков'', да? От этой ``скованности''.}
\soul{Конечно.}
\people{(Белимов) Но, это возможно?}
\soul{Эта скованность проявляется, прежде всего – характером.}
\people{(Белимов) Но, это возможно… какими-то упражнениями, да? Медитацией или чем?}
\people{(Ольга Белимову) Нет… (Улыбается)}
\people{(Белимов Ольге) Нет?}
\soul{Тогда представьте,- вы, не имеющий характера. Вы можете это представить?}
\people{(Белимов) Угу. Тоже плохо.}
\soul{Пока – да.}
\people{(Белимов) Мы полагали, что подсознание должно помнить и предыдущие жизни. Оказывается, вот тут, в последнем контакте это не подтвердилось. Оно только помнит нынешнюю жизнь.}
\soul{Вы были невнимательны.}
\people{(Белимов) Так. В чём?}
\soul{Если оно проиграется, то будет вспомнено. Если будет ``поставлен вопрос'', если хотите. Множество, подсознание не знает, пока не задан вопрос. Пока не были найдены ``ключики'', если хотите. И когда, с чьей либо помощью, или вы сами вспоминаете о прошлой жизни - подобный ключик - одно из ячеек, как вы говорите - ``ноосферы''. И тогда, уже будет и химически.}
\people{(Белимов) Угу. То есть, это можно…}
\soul{И мы вам уже отвечали на это.}
\people{(Ольга) Вот, ноосфера – это вот, хроники Акаши, только называют по-другому. То есть там все записи, да? И прошлого и будущего и всего, да? Или только прошлого?}
\soul{Нет там ни прошлого и ни будущего. Есть единое.}
\people{(Ольга) А-а… ну, это вот - ноосфера. А допустим, вот, ну… Нам так иногда говорят, что когда человек до какой-то стадии там или как это… дойдёт до какого-то уровня развития, он вспоминает все свои прошлые жизни. А про будущие нигде не сказано, что он вспомнит может и свои будущие. То есть - всё вспомнит сразу, то есть -он станет… Ну как? Для него, всё будет настоящее.}
\soul{И пока вы будете говорить в количественных формах - ничего не добьётесь.}
\people{(Белимов) Ну, к слову,-  мы можем выйти на подсознание переводчика и воспользоваться…?}
\soul{На подсознание ``переводчика'' вы не выйдите. Хотя бы по той причине, что ни мы, ни он не дадут этой возможности. Свободное сознание – да.}
\people{(Белимов) Угу. Ясно. А скажите, есть ли двойники у животного, например - у собак? Были случаи, когда на фото отображались образы умершей собаки. Получается и животные имеют тоже фантомы?}
\soul{Простите, мы говорили об энергетике, и вы тут же спрашиваете. А вы что хотели увидеть  Души? Только у вас - душа, да? У животных – нет. Так, по-вашему?}
\people{(Белимов) Ну, может быть и есть такой элемент сомнения.}
\soul{А мы вам говорили: ``Чем вы отличаетесь от животных?“}
\people{(Белимов) Ну, разумом.}
\soul{Монадой. Вы помните?}
\people{(Ольга) Да.}
\soul{Той ``божественной искрой''. Ну, простите, не искра светит в вас в фотографиях. И если вы сфотографируете стул и сожжёте его, никакой разницы между фотографией человека и этим стулом вы не найдёте.}
\people{(Белимов) Угу. А скажите, насколько достоверен эпизод, когда умершая собака, с ``того света'' - спасла хозяина, предупредив  автокатастрофу.}
\soul{И что в этом удивительного?}
\people{(Белимов) Ну, для нас - это удивительно. Многие  в это не верят, в этот эпизод, думают, что он придуман. Хотя вот, мы – уфологи, всё-таки склоняемся к тому, что…}
\soul{А если мы вам предложим другой пример, когда ребёнок в утробе спасает мать.}
\people{(Белимов) А каким образом? Подсказкой какой-то?}
\soul{А вы же только что слышали о детях. Простите, если дитё более чувствительно, то значит и ей легче будет до вас докричаться.}
\people{(Белимов) Угу. Бывают такие эпизоды, да?}
\people{(Ольга) Да часто, наверное.}
\people{(Белимов) Просто мы об этом незнаем, да?}
\people{(Ольга) Скажите… Я, конечно, может быть бестактна очень, простите. О страхе, вот…  Как мы этот страх получили? То есть ,и животные тоже имеют страх? Может быть это… }
\people{(Белимов) Может быть это защитная какая-то…? Полезное же качество человечества?}
\people{(Ольга) На какой-то стадии может быть…}
\soul{Безусловно.}
\people{(Ольга) Ну, а как вот от него избавится, от страха? Ну, если не совсем, то есть, вот как…}
\soul{Да от него нельзя избавляться. }
\people{(Ольга) Нельзя? А что нужно?}
\soul{Это одно из самозащиты. Просто, не надо отдаваться ему полностью.}
\people{(Ольга) Нужно контролировать его скорее, да?}
\soul{Ну, пусть будет так. Он должен, всего лишь только - подсказывать, но не руководить вами.}
\people{(Ольга) Ясно.}
\people{(Белимов) Скажите, какова природа - призраков? Почему они…порой, десятилетиями и столетиями появляются в одних и тех же местах?}
\soul{А потому, что вы их поддерживаете.}
\people{(Белимов) Энергетически? Страхом?}
\soul{Забудьте о них, и не будет их. А пока существует хотя бы одна книга, ждущая читателя, или хотя бы одна легенда - они будут.}
\people{(Ольга Белимову) А можно спросить?}
\people{(Белимов Ольге) Да-да. [Даёт согласие прим.]}
\people{(Ольга) А можно спросить? О ``хранителях времён'' можно спросить? Вы что-нибудь о них знаете?}
\soul{Что?}
\people{(Ольга) Ну, как-то вы с ними…(контактируете прим.)}
\soul{Что вы хотите узнать?}
\people{(Белимов) Это действительные персонажи или всё-таки какое-то выдуманное…?}
\soul{Мы когда-то вам говорили, что вы постоянно лжёте. И когда-то, вам тут же сказали, что лжи не существует. И что бы, вы не выдумали, это есть. Порой, и чаще всего, вы, как говорите: ``сочиняете'', ``фантазируете''. Да, вы считаете, что вы это придумали, сочинили и даже не догадываетесь, что вы просто где-то что-то ``прочитали''.}
\people{(Белимов Ольге) Ну, ты ещё расспроси про…}
\people{(Белимов) Но, нам показалось, что это очень могущественная цивилизация или, по крайней мере - организм.}
\soul{Это - начало. Они сами говорили вам об этом.}
\people{(Белимов) Да.}
\people{(Ольга) А вот они говорили, что у них такое время, вот, ноль целых одна тысячная (0.001)…}
\people{(Белимов) От чего?}
\people{(Ольга) А у вас ``времени нет'', вы так сказали, да?}
\soul{Вы путаете. }
\people{(Ольга) Путаем?}
\soul{Вы путайте разные вещи. По вашим понятиям, одна из теории – теория большого взрыва. Вот и представьте, вашу одну тысячную секунды. Не обладающая сознанием, но могущественна разумом. Ибо, сознание и разум - это совершенно разные вещи.}
\people{(Ольга) Ну, да.}
\soul{Сознание - это всего лишь только инстинкт. Инстинкт на данную обстановку, адаптация - если хотите.}
Мы же говорили вам о том. И разум - это умение владеть, умение видеть, умение видеть, умение слышать, умение применять все ваши чувства. Вот – разум! А сознание - это всего лишь…
\people{(Ольга) [даёт счёт: 1…]}
[Щелчок]
\soul{Спрашивайте. Но не забудьте при следующем перерыве дать обратный счёт, ибо вы уже его произнесли.}
\people{(Белимов) Обратный?}
\soul{с 19-ти до 11-ти.}
\people{(Ольга) Ага.}
\soul{Вы забываете?}
\people{(Белимов) Ну, мы тогда продолжим… Мы имеем дело не с подсознанием видимо, а с вами - с прежними.}
Известна ли вам природа фантомов? Ведь она неизвестна людям в полном понимании.
\soul{Возьмите учебники физики. Там есть всё. И не надо, искать более высшее. Вы уже знаете достаточно много наук, чтобы пойти вперёд или упасть, чтобы развить себя или уничтожить. И мы не хотели бы быть ещё одной ``палочкой-выручалочкой'' или ещё хуже - быть убийцей.}
\people{(Ольга) Можно ещё такой вопрос? Почему счёт от 1-ого до 9-ти, с 11-ти до 19-ти, а дальше вообще не имеет значение счёт? }
\people{(Белимов) Есть что-то за этим пределами?}
\soul{Да. Почему вам не начать с 21-ого?}
\people{(Татьяна) А это что-то изменит?}
\soul{Безусловно.}
\people{(Белимов) И так… до каких? До ``ста'', что ли?}
\people{(Ольга) Бесконечно?}
\soul{О, нет. Есть множество вариантов.}
\people{(Белимов) Ну, мы боимся экспериментировать, чтобы не повредить ``переводчику'', но если однажды попробуем, это не будет для него тяжёлой какой-то нагрузкой непредсказуемой?}
\soul{А как вы думаете? Начало счёта – вы попадаете к нам. Продолжение – вы попадаете на свободное сознание.}
\people{(Белимов) Угу.}
\soul{Тогда продолжите дальше. Попробуйте.}
\people{(Белимов) Тогда - на подсознание или на прошлое, получается, так?}
\soul{Нет. Это будет то же самое свободное сознание, но уже обладающее всем банком памяти.}
\people{(Белимов) М-м… Это интересно. Хорошо, что мы через три года общения с вами пришли к этому. А если с 31-ого?}
\soul{Тогда уже можно говорить о подсознании.}
\people{(Ольга) Ну, это опасно?}
\people{(Белимов) Это опасно? Есть элементы техники безопасности?}
\soul{Смотря, как вы будете вести себя, и что будет задано. И простите, если вы будете пугаться - будет пугаться и он.}
\people{(Белимов) Угу.}
\soul{Только, его чувства более усилены и потому вам всегда говорили…}
\people{(Ольга) Что настрой нужен.}
\people{(Белимов) О чём говорили? Что мыться надо, да?}
\people{(Ольга) О настрое, прежде всего.}
\people{(Белимов) О настрое, да?}
[Обрыв]
[Щелчок]
\people{(Белимов) Вы сейчас интересные сведения для нас сообщили. Мы действительно хотели бы, в дальнейшем продолжить эти эксперименты.}
\soul{Но, будьте готовы!}
\people{(Белимов) К чему?}
\people{(Ольга Белимову) К неожиданностям.}
\soul{Подумайте сами, вы теряетесь сейчас. А что будет дальше? А теперь представьте, вы дали счёт с 51-ого. И вам переводчик заявит, что он видит вас полностью, знает всё о вас. И что вы будете делать? Вы тут же включите ``спасительную реакцию''.}
\people{(смех)}
\people{(Белимов) Ну, таковы мы. Хотя любопытно о себе конечно…}
\soul{Инстикт…}
\people{(Белимов ) Да.}
\soul{…не чувствовать себя обнажённым.}
\people{(Белимов) Угу. [кашлянул]}
\people{(Ольга) Вот, скажите, вы говорили не раз, собственно, и мы об этом знаем, что… ну, в общем-то, мы живём в обществе очень тесно и чувствуем друг друга. Ну, каждый по-разному. Некоторые, может, больше чувствуют, некоторые – меньше. Вот,… а нужно ли, допустим, развивать вот такую чувствительность? Не развивать вернее, а вот такая чувствительность - это же очень неприятно - чувствовать, особенно вот…}
\people{(Татьяна) Болезни…}
\people{(Ольга) …болезни, отрицательные…}
\soul{Мы же вам говорили и повторим – страх не должен владеть вами. Да, вы должны обладать большими силами, чтобы видеть всё. И бОльшими силами - чтобы не пользоваться этим.}
\people{(Ольга) Угу.}
\soul{А вы же, в начале пути -начинаете бояться. Вы же, стараетесь сопережить, повторить увиденное.}
[Обрыв]
\soul{…не подготовлены -  будет страдать тем же. Вот вам - энергетически перенос болезни.}
\people{(Ольга) А вот, это вот, экстрасенсы как раз этим и бывают, то есть – бояться какую-то болезнь на себя как говорят – навлечь. Вот этим самым…}
\soul{Да!… И потому, вы придумали – ``Сжечь!''.}
\people{(Ольга) А-га!}
\soul{Вы можете это представить? – ``Беру и сжигаю, беру и сжигаю''. А что вы сжигаете?! И чем вы сжигаете-то?! Вы даже не знаете, что вы толком взяли, а сжигаете. }
\people{(Ольга) Ну, нас уже ``предохранители'' упрекнули в этом, конечно как-то мы…}
\people{(Белимов) Ну, сила мысли вроде…}
\soul{“Сила мысли”…}
\people{(Белимов) Мы мысленно…}
\soul{“Мы мысленно”… }
\people{(Белимов) …Хотя мы и не верим…}
\soul{Так лучше бы вы физически пытались это сделать, чем мысленно.}
\people{(Ольга) Ну, да.}
\people{(Белимов) Скажите, у нас есть товарищ из Волгограда, который много о себе задал вопросы, но я не уверен, что вы сможете отвечать. Вы же не были в той ситуации, вы не проследили её?}
\soul{И мы опять должны будем вам повторить, что мы не хотели бы, ваши сомнения умножать или рассеивать. Ваши проблемы. Вы и решайте. Почему приходите и спрашиваете: ``А вот, там-то и там-то - действительно ли это так?'' - А на что тогда нужны вы?}
\people{(Ольга) Да, всё верно.}
\people{(Белимов) Ну, наверное… Вот он, например, выходил в Млечный Путь…}
\soul{Давайте скажем: здесь больше желаний.}
\people{(Ольга) Да, я тоже об этом…}
\people{(Белимов) Желания?}
\people{(Ольга) Да, ЖеланиЯ. Больше желаний. Да-Да. Всё верно. Да! Мы чего желаем, то и получаем, а потом думаем, что, в общем-то, что-то с нами происходит, и мы были там-то и там-то, на какой-то планете в какой-то вселенной…}
\soul{Фантазия, основанная на реальности, на искривлённой реальности. - Очень просто. И вы должны были помнить пословицу ``незная где зло,…'' Вы помните?}
\people{(Белимов) Угу.}
\people{(Ольга) Ну, да. ``Не зная броду, не лезь в воду'' - грубо.}
\soul{Зачем же?}
\people{(Ольга) А-а! Нет!}
\soul{Вы, где-то услышали, но не дослышали. Взяли – переврали. Вот вам – ``испорченный телефон''. И теперь, этот телефон действует достаточно реально на вас. Сперва, вы сочинили, что были там-то и там-то, ибо вы желали. И причём ``сочинили'', заметьте - не солгали, - а желание заставило вас. Дальше, - идёт игра, дорисовка мелких деталей. И когда все эти мелкие детали будут дорисованы, вы уже убедитесь в реальности столь сильно, что довольно-то трудно будет доказать самим же, что вы сами себе-то всё это и придумали. }
\people{(Ольга) А, вот скажите, вы говорили, что человек идёт по жизни, да? Со всеми своими…}
[Даёт счёт: 1-2] 
[Щелчок]
\people{(Белимов) Так, в чём?}
\soul{Вы, продолжили счёт? Мы вас предупредили, при первом же перерыве - дать обратный. А теперь, вы должны ещё дать теперь обратный счёт с девяти до единицы.}
[Дают счёт: 9-8-7-6-5-4-3-2] 
[Щелчок]	
\soul{…забывая, что вся жизнь состоит из мелочей. И выкинь любую из них, изменится и жизнь. И вряд ли к лучшему.}
\people{(Ольга) Вот, я хотела и спросить о…ну… о жизни, о ветви, о дереве вот этом. ``Древо жизни'', так называемое. Вот человек идёт, попадает в какие-то ``веточки'' может, возвращается обратно к ``стволу'', потом идёт обратно, может быть обратно, даже вниз что ли, если можно так назвать. А вот, возможен ли путь, прям будем так говорить – ``по стволу''? То есть, не заходя в какие-то…}
\people{(Белимов) В тупики…}
\people{(Ольга) Тупики.}
\people{(Белимов) …В ответвления.}
\soul{Возможен, но не нужен.}
\people{(Ольга) Не нужен, да?}
\soul{Ибо, вы не будете знать всё дерево.}
\people{(Ольга) А-а!}
\people{(Татьяна) Как лестница.}
\people{(Белимов) То есть, жизненный опыт должен быть?}
\people{(Ольга) Должен быть обязательно, да?}
\soul{Ну, представьте,- вы перепрыгнули через ступени.}
\people{(Ольга) А вот скажите, если человек, допустим, умирает физически, потом… - ну, готовясь к следующему воплощению, допустим, да, может попасть опять в прошлое? То есть, по отношению к тому, как он последнее воплощение был? То есть, допустим не в 20-ый век, а 17-ый ?}
\soul{Даже по времени.}
\people{(Ольга) Может, да?}
\soul{Простите, здесь не имеет значение время.}
\people{(Ольга) Ну, да…}
\soul{И поэтому, нельзя говорить, что вы будете только в будущем. Множество раз, вы можете перепрыгивать и в каменный век, и столь множественное количество, вы можете попасть и в будущее. Это не имеет значения.}
\people{(Ольга) А что имеет значение?}
\soul{Груз, который вы несёте в себе.}
\people{(Ольга) И путём этих воплощений, мы должны избавиться от этого груза?}
\soul{Ну, поймите же вы… ВЫ должны идти в ``небеса'', но не ваша плоть. И какая разница - плоть у вас будет пещерного человека или будущего?}
\people{(Ольга) Ну, да.}
\soul{Есть ли в этом смысл? Есть ли в этом значение?}
\people{(Ольга) Да нет.}
\soul{Нет! Зачем же тогда спрашиваете?}
\people{(Татьяна) Интересно.}
\people{(Ольга) Мы же не совсем понимаем, может быть.  Действительно, как-то мы привыкли, вот, отмерять что…}
\soul{Вам нужна, всего лишь обстановка - та, которая могла бы дать вам больше пользы, больше знаний. И если для этого нужен 17-ый век, - пожалуйста, вы будете в 17-ом. Там, где вы,как вы говорите - ``готовитесь к следующей жизни'' - времени нет.}
\people{(Ольга) Ну, да.}
\people{(Белимов) Угу.}
\soul{И потому, вы можете, выбирать по другим критериям.}
\people{(Ольга) А-а! Ну, там выбирается, вот как…}
\soul{Да, вы выбираете. Но не свободный выбор.}
\people{(Татьяна) Не свободный?}
\people{(Ольга) О свободном выборе, тогда как?}
\soul{А вы должны его заслужить.}
\people{(Ольга) Ах, вон как! А вот, ``камень'', о котором всё время говорят, голубой и красный – девочки и мальчики,…}
\people{(Белимов) Кто его приносит?}
\people{(Ольга) …Там тоже не имеют цвета, будем говорить. Там же нет…?}
\soul{Это всё чисто символически для Земли. }
\people{(Ольга) Ну, это вот ``запись'', -  как нам ``переводчик'' сказал, что ``там всё записано'', - это значит запись всего?}
\soul{“Камень'' вы можете выбирать  сами.}
\people{(Татяна) Философский… Так, это свободный выбор ``камня''?}
\soul{Но, выбрав ``камень'' вы впишите в него судьбу – сценарий вашей будущей жизни. Вы сами сказали, что вы готовитесь к будущей жизни. Значит, вы уже должны построить план?}
\people{(Ольга) Да.}
\soul{Вот, вы их построили и записали ``на камне''.}
\people{(Ольга) Угу.}
\soul{И когда вы рождены, вы должны соблюдать вами же выдуманное.}
\people{(Ольга) Скажите, вот когда человек живёт, ну, в теле, - в физическом теле, - у него ведь множество меняется… Ну, вот мы боимся умереть, а ведь наши клетки там, или как, умирают бесконечно и рождаются новые. Вот, и они же, то есть…?}
\people{(Белимов) То есть – человек обменяется, новое становится или как?}
\people{(Ольга) Отражение ведь более…}
\soul{Только плоть меняется.}
\people{(Ольга) Только плоть? А суть остаётся, да? То есть, вот сама…}
\people{(Белимов) Сознание, да? Душа…}
\people{(Ольга) Нет…}
\soul{Меняется только плоть. И эта плоть - её измена - уже говорит вам о бессмертности.}
\people{(Белимов) Скажите, а если каждый выбирает свою судьбу, и записывает на камне…?}
\soul{Простите, каждый составляет свой сценарий кем быть, но выбор - не свободен. Ибо, как вы можете сейчас стать великим математиком, не зная ничего о математике? Вы можете стать?}
\people{(Белимов) Нет.}
\soul{Значит, вы не будете им.}
\people{(Белимов) Но вряд ли человек записывает раннюю гибель себе сознательно. Почему…?}
\soul{Разве? А если его не интересует эта плоть. Там другие меры, другие критерии. Мы только что говорили вам об этом. }
\people{(Ольга) А зачем тогда сюда приходит?}
\soul{А простите, почему тогда умирают дети?}
\people{(Белимов) Да, вот, непонятно.}
\soul{Хотя бы, один из примеров, чтобы как вы говорите - проучить. }
\people{(Ольга) Кого? Себя?}
\soul{Родителей.}
\people{(Ольга) А-а…родителей.}
\soul{Ваше слово - ``проучить''.}
\people{(Татьяна) Это жестоко так.}
\people{(Белимов) Ну, как ``проучить''? Это ведь…}
\people{(Ольга) Ну, извините, конечно, за такой вопрос может - об абортах тогда. Много очень таких толков об этом идёт. Церковь, то запрещает, то, наоборот -  разрешает…}
\soul{Убийство!}
\people{(Ольга) Это такое же…?}
\people{(Белимов) Это греховное? И человек попадает потом в адское…? В ад?}
\soul{О грехе? Стоит ли говорить о грехе, если всё это для вас, всего лишь только – слово, и не больше. Когда то, одно можно было…}
\people{(Ольга) [Даёт счёт: 1]}
[Щелчок]
(Диалог с тем, кто отвечает за будущее) 
\soul{…отгадать моё имя.}
\people{(Ольга) Настоящее?}
\soul{Я та часть, которая отвечает за будущее, которое предсказывает это будущее.}
\people{(Ольга) А зачем отгадывать мне?}
\soul{А вы так любите это.}
\people{(Ольга) Ну, да. Ну, мы уже поняли…}
\soul{Поэтому, мне приходится жить - и очень много работы. Все ваши планы и даже на завтра, мне приходится совершать и помогать вам  реализовывать. Разве это я, придумал квадрат Джуны? (нумерология: квадрат Джуны Давиташвили/квадрат Пифагора прим.)}
\people{(Ольга) Кого-кого? Квадрат? Джоун?}
\soul{Джуны.}
\people{(Ольга) Джуны? А что это такое? Мы не знаем.}
\soul{“Незнаете''. Как же вы не знаете?}
\people{(Ольга) Может ``переводчик'' знает, а мы - нет.}
\soul{Разве? Ну, хорошо. Давайте назовём по-другому: вы берёте дату рождения…}
\people{(Татьяна) Нумерология.}
\soul{…Берёте сумму, всех чисел и, наконец, сумму всех цифр. И рассматривая повтор цифр, вы можете уже говорить о характере человека, а значит и о будущем. - Не знали?}
\people{(Белимов) Слышали, но мы не предаём слишком большого значения этому. Хотя, может быть, и ошибаемся, а вы…}
\soul{И вот, смотрите, в нём содержится 9 квадратиков.}
\people{(Ольга) Это что говорит? О чём?}
\soul{О чём? И вот - повтор. Допустим, в вашей дате появляется цифра ``6''. Она повторяется четыре раза. Ага! Цифра ``6'' отвечает за космическую связь, значит четыре раза вы ``сильнее'' обычного человека…Здорово? – Здорово! И получается, чтобы собрать самые лучшие качества, - быть тем же Иисусом Христом,- мне надо как можно больше повторялось ``буковок'', чтобы в каждой клеточке подряд было написано как можно больше. И вот представьте, я хочу квадратик заполнить полностью, чтобы все 9-ть квадратиков были заполнены полностью. Представляете, какая мне нужна дата рождения, чтобы каждая цифра повторялась не менее 9-ти раз. А бывают такие даты?}
\people{(Татьяна) Нет, не бывают.}
\soul{Не бывают! И получается, родился человек в январе…Ага! - Характер один. Родился в том же январе, но на день позже - характер другой. О-о, как интересно! Главное, чтобы всё это не по-вто-ря-лось.}
\people{(Белимов) Как мы и догадывались.}
\soul{Но, что самое интересное - вся эта ерунда действует на вас.}
\people{(Татьяна) Наверное, мы сами придумываем и сами воплощаем?}
\soul{Пока вы не пришли к цыганке, и пока она вам не сказала, что вы болеете - вы и не болели.}
\people{(Белимов) Угу.}
\soul{Вы жили себе спокойно, пришли, она сказала что: ``Вот, у вашего мужа - черноглазая!'' И все черноглазые - вам враги.}
\people{(Белимов) А как иначе будешь реагировать? Мы верим в чудеса…}
\soul{Да-а! Конечно! Вы цыганке верите, своей жене не верите. А цыганку увидели в первый раз, ``А! Это ерунда!''. Ну, почему так? Цыганке вы верите больше чем своей жене, и тем более, больше чем себе. А почему? Да потому, что вы сами придумали измену жены. И с радостью приняли подтверждение от этой цыганочки.}
\people{(Белимов) Хорошо,…}
\soul{А когда она вам скажет другое, что у вас нету, вы тут же плюнете и ещё обругаете, что она ворует у вас деньги и пойдёте дальше, потому что она с вами не согласилась - она не подтвердила то, что вы хотите. Ну, давайте, возьмем…Что мы возьмём? Ну, давайте возьмём ``друидов''. (Гороскоп друидов [деревья] прим.)}
\people{(Ольга) Угу. Друидов, да.}
\soul{Хорошо, но только вы одно ``но'' забыли… Но пусть будет ``каштан'', а простите а рядом что с ним растёт? А где он растёт?}
\people{(Ольга) На юге.}
\soul{Как вы думаете, окружающая среда влияет на него?}
\people{(Ольга) Да, а как же. Конечно, влияет.}
\soul{На него окружающие деревья - влияют? }
\people{(Ольга) Да.}
\soul{Ага! Тогда, значит, мы берём…Прекрасно! У нас что, каштан? Хорошо, каштан. Какой месяц?}
\people{(Ольга) Вот, плохо помню. Мы плохо помним. ``Каштан'' - какой месяц,-  кто знает? Ну, уж если начали, продолжайте. У нас память плохая.}
\soul{Ну, хорошо. Пусть будет – ноябрь. Итак, ребёнок родился в ноябре. По друидам – каштан, по годам - мы ещё можем определить ``обезьяна'' это или ещё кто… а почему-то не берём родителей. }
\people{(Ольга) Ну да!}
\soul{А почему, когда мы составляем тот же квадрат Джуны, и берём нумерологический ряд вашего рождения, почему же вы в этот ряд не вставляете даты рождения ваших родителей? В конце концов, вы же всё-таки от них, вы должны быть на них похожи. Правильно?}
\people{(Ольга) А тогда и дедушку и бабушку…}
\soul{А почему, вы тогда не берёте…? Дедушку и бабушку?! Если они живут с вами, обязательно возьмите и их. Потому что, простите, на ваш характер дедушка с бабушкой - влияют тоже. }
\people{(Ольга) Короче, кто живёт, да?}
\people{(Татьяна) И соседи рядом.}
\people{(Ольга) Нет…}
\people{(Белимов) Так, ну, дальше продолжите. Продолжайте.}
\people{(Татьяна) Не получится тогда - все квадратики заполненные всеми цифрами.}
\people{(Белимов) Ну, мы когда… определим характер человека.}
\soul{Ну, хорошо. Вы говорите: ``Все квадратики''. Да, конечно, можно заполнить и будут все ``девятки''. Как здорово! Как красиво! А тогда, получается, ещё немножко не то. Понимаете, если я сейчас заполню так, - а математика такая штука интересная что, что угодно и куда угодно можно приписать и подтвердить,- вот, так вот мы, чисто математически выясняем, что у вас такой-то и такой-то характер. А вот дату смерти вы не можете указать, дату переломных моментов в жизни вы не можете указать. Не можете? – Нет. А вот что интересно, если бы вы взяли нумерологический ряд трёх поколений, значит, даже и дедушки с бабушкой уже мало, и берёте себя. И произведя двойные преобразования, вы можете узнать возможную дату смерти. Почему возможную?…}
\people{(Ольга) Ну…да. Наверное, это ещё зависит не только от…}
\people{(Белимов) Мы можем ошибиться, наверно, в расчётах.}
\people{(Ольга Белимову) Да нет.}
\soul{Нет, зачем же? Мы не будем брать ошибки.}
\people{(Ольга) От воли человека, наверное, тоже что-то зависит?}
\soul{Во-о! Наконец-то! А то уж  мне, бедненькому, потеть и потеть! Получается, что только я руковожу вами и не больше. Вот, я нарисую так квадратик – будет так. И зачем тогда нужен этот квадратик?}
\people{(Ольга) А вот генеалогическое дерево всё-таки…Люди должны знать своих предков, наверное?}
\soul{А вы знаете?}
\people{(Ольга) Да, мы-то не знаем.}
\soul{А почему-то раньше знали!}
\people{(Ольга) Да! А вот сейчас мы плохо знаем это.}
\people{(Белимов) А это желательно знать, да?}
\soul{Ну-у, как вы думаете? Хотя бы чисто для себя. Разве не интересно?}
\people{(Ольга) Интересно.}
\people{(Белимов) Да.}
\soul{А уважение? Неинтересно? Да если бы вы знали, как вы говорите  `` генеалогические деревья'' свои, у вас бы история-то была более честнее.}
\people{(Ольга) Да. (соглашаясь)}
\people{(Белимов) Скажите. Мы в первый раз столкнулись, по-моему, с вами впервые - с будущим. У нас сейчас в стране очень важный период. Имеется ли у нас будущее?}
\soul{Мы не будущее. }
\people{(Все Белимову) Это не будущее!}
\soul{Я всего лишь - та часть, которая сидит в каждом из вас…}
\people{(Белимов) О! (удивление)}
\soul{…и сочиняет будущее.}
\people{(Ольга Белимову) Сочиняет, именно.}
\people{(Белимов) Вот, какое будущее у нашей страны будет после 16-ого июня…?}
\soul{А это пусть ваша часть вам и сочиняет.}
\people{(Ольга) (Смех)}
\people{(Белимов) Ну, ничего, ни подсказку не дадите нам?}
\soul{А зачем я буду отбирать ``хлеб''? У вас Оно тоже сидит.}
\people{(улыбаются)}
\people{(Белимов) Ну, а что подсказывает у ``переводчика''? }
\soul{А это пусть он только и знает.}
\people{(Белимов) Ну, хитрые… Никак не выяснится. Четыре дня осталось, тут ничего же не изменится. Ну, хотя бы, любопытно - смена власти будет или нет?}
\people{(Ольга Белимову) Будет.}
\people{(Белимов) Да…тут…}
\people{(Ольга) А вот, вы подчиняетесь, значит… Помогаете сочинять будущее, так, да?}
\soul{Ну, не-ет! Вы же желаете знать? И даёте мне силу. Вы меня рождаете. А когда вы меня родили впервые? }
\people{(Ольга)У-у! Давно, наверное.}
\people{(Татьяна) Когда, наверное, появился интерес к будущему.}
\soul{Да-а, когда вы начали мечтать. Только тогда я ещё честненьким был и не обманывал.}
\people{(Ольга) А сейчас обманываешь, да?}
\soul{Ну, конечно. Вы же, не хотите, чтобы у вас было плохое будущее? И вы тут же найдёте кучу всяких поправочек: ``А может быть…'', ``А если так…'', ``Авось…'' Будет у вас прекрасное и хорошее будущее. А я бедненький пыхчу, целую кучу карточек где-то составлю, а потом сижу и думаю, по какой вы всё-таки жить-то будете?}
\people{(Белимов) А вы что, каждого из нас можете проанализировать или только ``переводчика''?}
\people{(Ольга) Ну, это в нас. В нас каждом своя, вот эта часть…}
\people{(Белимов) Ну, вот мне сейчас интересно - менять ли работу или нет?}
\people{(смех)}
\people{(Белимов) Вы можете подсказать, будет с этого толк или лучше сразу забросить…?}
\soul{Ну, это я должен быть в вас и должен все прошлые ``карточки'' поглядеть, что ж там до меня ``нарисовали'' то. Ну, вряд ли ваш ``предсказатель'' мне разрешит ``всунуться'' к нему.}
\people{(Ольга) Да.}
\people{(Белимов) А у ``переводчика''? Если вы в ``переводчике'' сидите. Будет у него издана книга когда-то собственная?}
\soul{Вот, я сейчас, произнеся ответ, создам минимум три ``карточки''. }
\people{(Ольга) Вот так!}
\soul{И по-какой вы будете жить? Откуда я знаю, что вы дальше-то придумаете? А вы дальше захотите ещё что-то и мне новую ``карточку'' надо будет нарисовать.}
\people{(Белимов) Тогда, надо вообще жить одним днём и не интересоваться ничем, будущим. Так что ли лучше?}
\soul{Ой, вы здорово придумали!}
\people{(Белимов) Вам, тогда там работы не будет.}
\soul{Ээ-й! Ещё больше!}
\people{“Ещё больше!'' (Смех)}
\soul{Ну, представьте, я на каждый день составляю карточки. А в день знаете, сколько я их составляю? }
\people{(Белимов) Мы не знаем.}
\soul{Ой! Вы едите в автобусе, и у вас вдруг возникает мысль: ``А не сойти ли мне на этой остановке?'' – Я уже ``карточку'' рисую.}
\people{(Ольга) Да-а!}
\soul{А вы подошли к двери, Ах! Не успели вы,-  а я теперь другую ``карточку'' рисую. Вы идёте домой и думаете: ``Жена будет ругаться или не будет ругаться?'' - я уже две карточки нарисовал. Вы пришли, а жены вообще дома нету. Ругаться некому - уже третью рисую? И знаете, сколько я их в день рисую? А теперь, представьте, как вы говорите - ``Жить одним днём''. Ну и чё?}
\people{(Татьяна) Наверное, надо не думать? }
\soul{Тогда вы ночью будете кучу сочинять, потому что вы так привыкли, что вы.. Днём-то вам лень их будет сочинять - ``Одним днём живут. Так в Библии сказано!'' - А кто там такого говорил-то? }
\people{(Белимов) (вздыхая) Да-а…Так, вы нас…}
\soul{И вы будете ночью сочинять…сочинять…сочинять. Вам хорошо – вы спите. А мне – карточки рисуй.}
\people{(Белимов) Вы - наш ближайший помощник, получается, в жизни, да? Вы же всё время нас сопровождаете.}
\soul{Э-э, хотелось бы увидеть, где у вас нет помощника.}
\people{(Ольга) Одни сплошные помощники, да?}
\soul{Одни сплошные помощники. Даже, если вы идёте вешаться, у вас всё равно будет кучу помощников чтобы повеситься.}
\people{(смех)}
\soul{Они всегда везде есть.}
\people{(Белимов) Но ведь, при этом и вы погибнете. Те, которые помогали в его жизни.}
\soul{Ой, это всего лишь одна из ``карточек'', и не больше.}
\people{(Ольга) И сколько же этих карточек, если можно в цифрах, так назвать?}
\soul{Ой! Лучше не говорите.}
\people{(Ольга) Вы их не считаете?}
\soul{Да если нам тут считать, когда же у нас тогда карточки сочинять будут? }
\people{(Ольга) Ага! Всё ясно.(улыбается)}
\people{(Белимов) Так и всё таки, какое предсказание вы дадите, хотя бы переводчику на ближайшие, ну…годы?}
\people{(Ольга Белимову) Да он не предсказывает.}
\soul{Ну, ``нарисовать'' можно что угодно, а толку-то?}
\people{(Белимов) Ну, дык, это бесполезно тогда.}
\soul{А вы знаете… Мне говорят: ``Хочешь поболтать?'' Я сразу нарисовал три ``карточки''.}
\people{(Ольга) Угу. ``Хочу, не хочу и…не знаю''.}
\soul{Ну, вот. Нарисовал, показал. А мне говорят: ``Ты, поболтай с ними, но чтобы так… понятно было, что дурью занимаются, братцы-то.”}
\people{(Ольга) Ну, да. Всё верно.}
\soul{Вот, я сейчас с вами разговариваю дурацким голосом на дурацкие темы… }
\people{(Ольга) Ну, почему…}
\soul{А потом нарисую ещё кучу карточек: помогло/не помогло, туда/не туда.}
[Щелчок - вкл/выкл записи]
[Диалог с первыми]
\people{(Ольга) Ну, мы…Скажите…}
\people{(Белимов) Вы были свидетелями этих превращений ``переводчика'', да?}
\soul{А нам показывали эти ``карточки''.}
\people{(Ольга) А вообще кто вот сказал, что ``Поболтайте''? Это ``переводчик'' сказал ему? }
\people{(Белимов) Или вы посоветовали?}
\soul{И мы, и другие. Мы хотим вас показать со всех сторон. }
\people{(Ольга) Ну, да.}
\soul{Это сложно сделать. Вы столь многогранны, что, как говорить его языком - множество карточек и все они разные. }
\people{(Белимов) И ничего конкретного. Вот мы тут больше привыкли конкретно…}
\soul{А какая конкретность нужна вам? Какая? Ну, как мы можем вам объяснить конкретно то, что вы не имеете даже представления? Конечно, вам нужно рассказать какая погода в Париже, какие улицы, какие люди. Но никогда вы не сможете понять близкого человека, своего же сына, - что он хочет и почему.}
\soul{Ну, почему вас подводит язык ваш?}
\people{(Ольга) Да, наверное, меньше говорить надо. Больше прислушиваться к себе и к другим. }
\soul{А вам некогда. Вы всегда в делах. Вы всегда спешите. Вы всегда считаете, что правы только вы и никто другой. Вы всегда оправдываете себя, а потом уже прежде и других. И та спасительная нить, которая зовётся совестью, чаще - столь слаба, что вы не замечаете её и перешагиваете, и рвёте её. А потом обижаетесь, почему и зачем вас карает Бог. Видите, даже здесь, вы уже обвинили Бога. ``Бог вас карает.'' А о бумеранге, который возвращается к вам же - забыли. А в тех же, небрежно брошенных слов - ``Да, ладно…''. Они потом придут к вам и будут предъявлены в той же форме. И вы уже будете обижены о невнимательности, хотя сами кого-то одарили этим же.}
\people{(Ольга) Скажите, и вот мы строим своё будущее, ну, то есть… Ну как? Наверное, не будущее, а как-то…}
\soul{Вы планируете своё будущее, но не строите никак.}
\people{(Ольга) Да. Вот, и те люди, которых мы любим, с которыми мы всегда, допустим, идём по жизни вместе, ближе они чем другие, тоже планируем с ними встречу в каких-то…Или это не совпадает? Должно совпадение какое-то быть? Если должны встретиться в каком-то там времени другом, если так можно назвать.}
\soul{Чаще – нет. Чаще - нет совпадений. Их и не может быть потому, что вы - разные. А как вы можете договориться? ``Давай, в следующей жизни встретимся там-то…там-то…там-то.'' }
\people{(Ольга) Да нет! Не так! А именно, вот, как-то желаниями что ли…}
\soul{А вот, если бы вы могли бы это делать, тогда бы вы уже были бы выше, намного выше. А сейчас – вы одни. Когда, как вы говорите - вы умрёте, вы будете совершенно другие, у вас будут совершенно другие меры. А значит - совершенно другие планы. И те планы, которые вы построили здесь, будут смешны для вас.}
\people{(Ольга) А так… как же…}
\soul{Ибо, вы строите планы относительно данного дня. А следующий день вы не можете учесть, ибо вы не видите будущее. Вы согласны, что вы составляете планы на будущее смотря на прошлое?}
\people{(Ольга) Угу.}
\soul{И этот промежуток, между датой в будущее и с настоящим вы не можете рассмотреть никак. Вы только предполагаете. У вас есть только одна цель – получить то-то  то-то, или сделать то-то  то-то. Ну, давайте скажем так: закончить институт. Конечно, легко планировать, зная,  что институт можно закончить через 5 лет. Уже,-  вы запланировали или вы подстраивались под этот план?}
\people{(Ольга) Подстраивались.}
\soul{Подстраивались. Ибо, вы видите, срок был дан не вами. И вы, довольно таки туманно представляете своё будущее. Вы согласны? Эти 5 лет для вас – пусты, ничего не значащие. Ибо, вы не знаете - что будет и как. И лишь только одна цель – прожить эти 5 лет и получить диплом. И представьте, вы проучились 2 года и уходите в декрет, а это было не запланировано. Кто план составил? Вы?}
\people{(Ольга) Нет. (улыбаясь)}
\soul{Кто?}
\people{(Ольга) Ну, наш внутренний человек. }
\soul{Ребёнок…}
\people{(Ольга) Ребёнок? (Удивлённо)}
\soul{Ребёнок - изменил ваш план. Теперь, вместо пяти лет, вы будете учиться восемь.}
\people{(Ольга) Скажите, родители говорят - не выбирают. А вообще-то получается наоборот - родители выбирают.}
\soul{Ро-ди-те-лей.}
\people{(Ольга) Ну, да. Ребёнок выбирает родителей. Выходит так? Родители, просто должны хотеть ребёнка. Так? Или они могут и не хотеть, а вот, ребёнок их выбрал.}
\soul{А зачем тогда нужны аборты? }
\people{(Татьяна) А они нужны?}
\soul{Ну, а вы подумайте сами. Вы ответьте сами себе. Если родители не хотели ребёнка…Ребёнок появился. К чему вы прибегаете?  }
\people{(Ольга) Ну, да.}
\soul{К аборту.}
\people{(Ольга) А вот, это может быть, конечно, грубо так можно сказать - незваный гость.}
\soul{Грубо! - ``Незваный гость''. Но ребёнку видней. И вы, соответственно были подготовлены. Где-то внутренне, вы уже - знаете. Знаете, что когда-то вы сделаете тот же аборт. Грубо – убьёте. И уже к этому времени подготовитесь и морально, и физически.}
\people{(Ольга) (Вздыхает)}
\people{(Белимов) Скажите, - вы в мире эмоции землян живёте, - вот сейчас настроение у россиян…?}
\soul{Мы не живём там. Мы приходим, чтобы разговаривать с вами.}
\people{(Белимов) Да, приходите. Какое сейчас настроение у россиян в связи со сменой власти? Она желательна - смена власти или нежелательна?}
\soul{Хаос.}
\people{(Белимов) И будет и придёт хаос?!}
\soul{А вы выйдите на улицу и спросите: ``За кого вы голосуете?'' И подойдите к другому. И все будут утверждать только одного?}
\people{(Ольга) Да…}
\soul{А что вы тогда спрашиваете - нужна ли смена или нет. Да многие жаждут этой смены, другие боятся этой смены…}
\people{(Ольга) Вот, скажите… Вот опять к абортам я вернусь, извините, конечно…}
\people{(Белимов) А…скачки какие… И как мы узнаем…? (улыбается)}
\people{(Ольга) Дело не в этом. Будущее мы сами…будем. Если ребёнок знает, что женщина сделает аборт, это вот не казнь ли, или вот как, карма что ли…?}
(конец первой части)
VG - 1996.06.11_-_02
\people{(Белимов) …Эта картина…}
[Перепад]
\people{(Ольга) …Вот тело, что умеет… это вот человек научил это тело… Уметеет управлять им.}
\soul{(Гена) Ну, скажем так, вот эта - душа, да?}
\people{(Ольга) Ага.}
\soul{(Гена) М-м… Как она? Она… Понятно, что она живёт в единстве. Всё это понятно. Но ей ещё  хочется всё ``пощупать''.}
\people{(Ольга) Угу.}
\soul{(Гена) Пощупать, поглядеть, именно попробовать на материальном вот, уровне. Вот, что она сумеет освоить.}
\people{(Белимов) Гена, как ты ощущаешь, Земля умирающий организм или всё ещё нормально с ним?}
\soul{(Гена) Вообще-то, да.}
\people{(Белимов, Ольга) Умирающий? }
\people{(Ольга) То есть, она уже старая?}
\people{(Белимов) То есть, ты ощущаешь, что может какой-то быть ``переход''? Сейчас много говорят о переходе в шестую расу и так далее.}
\soul{(Гена) Ну, говорят-то, вообще-то… Сейчас  – две версии, как говорится, и, причём, эти две версии в принципе не затрагивают чисто духовное, а затрагивают внешнее воздействие. Или метеорит нам на голову свалится, или Земля расколется. А не рассматривают вообще даже варианта, что именно - духовный перелом. Обязательно нам нужна вот, как говорится, теория катастрофы.}
\people{(Белимов) Угу.}
\people{(Ольга) А вот скажи - Земля старая? Вот как, ну… человек, если тело вот Земли… То есть, вот как, будем говорить так, как человек вот – ``старый - молодой'', как Земля? }
\people{(Татьяна) Юность – зрелость…}
\people{(Ольга) По-своему вот, как она? – Старая – молодая? Или какого возраста она?}
\people{(Белимов) Ты слышишь нас?}
(“Переводчик'' не отвечал)
\people{(Ольга) Ты, где был? }
\people{(Белимов) Что было с тобой?}
\people{(Ольга) Спрашивал кого-нибудь?}
\soul{(Гена) А что со мной было?}
\people{(Ольга) Спрашивал? Кого спрашивал?}
\people{(Белимов) Ты как-будто ``уходил''.}
\soul{(Гена) О, нет. Я не заметил этого.}
\people{(Ольга) Слушай, ну, ты всё-таки нам скажи, Земля - как вот, старая - молодая?}
\soul{(Гена) Да, она уже стареет.}
\people{(Ольга) Стареет.}
\soul{(Гена) Но, возраст - если взять относительно человека?}
\people{(Ольга) Ну, да. Да-да.}
\soul{(Гена) Ну, где-то, наверное, будет порядка… 28-ми, 30-ти. }
\people{(Ольга) У-у! Ну, это ещё молодая.}
\people{(Белимов) Да, это ещё молодость!}
\soul{(Гена) Да, но, сколько ещё жить-то неизвестно.}
\people{(Ольга) А-а! Сколько она… – Да! - По какое. У неё тоже кармические какие-то могут быть, ещё какие-то другие, да? - Как и у человека?}
\soul{(Гена) Закон один. Что для человека, что для Земли, что для Вселенной – закон один. Если для человека существует карма, то, соответственно, существует карма и на Земле. Потом,-  что такое Земля? Это всего лишь…  Что такое - человек? Это именно, как говорится, совокупность всех молекул. }
\people{(Ольга) Ага. }
\soul{(Гена) Правильно?}
\people{(Ольга) Ну, да. Атомов.}
\soul{(Гена) Ну, и как может…? Молекула имеет какую-то карму? Ну, имеет, конечно. Она должна, как говорится, родиться. Правильно?}
\people{(Ольга) Ну, да. Взаимодействовать.}
\soul{(Гена) И помереть, как говорится. Ну, скажем – клетка. Не молекула, а клетка, да?}
\people{(Ольга) Ну, да.}
\soul{(Гена) Вот, она должна, как говорится, размножиться и помереть. Так? Вот её ``карма'', как говорится. Получается, человек – совокупность всех вот этих молекул. Возьмём даже, вот этот простой, утрированный вот этот вот принцип, да? - ``Родиться –  размножиться – умереть''. }
\people{(Ольга) Угу.}
\soul{(Гена) У каждой молекулы. А теперь, вот это сборище молекул… А функция - одна и та же получилась, правильно? Только теперь надо в совокупности - родиться, в совокупности всем этим обществом, как говорится…}
\people{(Ольга) Исполнить своё дело!}
\soul{(Гена) Ну, как ``исполнить дело''? Мы говорим пока утрированно об этом, что молекуле нужно, а что это…}
\people{(Ольга) Угу.}
\soul{(Гена) Это размножиться, как говорится, и умереть. Правильно?}
\people{(Ольга) Угу.}
\soul{(Гена) Получается, не изменилось ничем. Единственное только, что здесь ещё теперь появилась, как бы другая ещё цель. Для молекулы цель - это войти в совокупность с другими, как говорится, быть в обществе, да? - Вот, для клетки. Для человека, то же самое осталось – быть в обществе. Всё, то же самое. Но, здесь он уже, как бы осознаёт. Он должен осознать  – зачем он в этом обществе. Свою задачу. Здесь моле… клетка чисто на интуитивном инстинкте работает, а здесь человеку уже дан инстинкт, потому что, иначе он мог бы и не хотеть размножаться. Правильно? И вот, причём инстинкт такой хитрый. Чтобы исключить вот это нежелание размножаться, вот и придумано как раз-то удовольствие самого акта размножения. }
\people{(Ольга) Ну, да. Ага. Всё понятно.}
\soul{(Гена) И, что делает, кстати, и моле… Клетка. Тот же самый эффект. То есть здесь действительно, чисто физический уровень, вот этот вот всё, да? Но ещё существует, как говорится, свобода воли что ли. Само вот это вот, именно - разум. }
\people{(Ольга) Угу.}
\soul{(Гена) То бишь, совокупность вот этих вот всех маленьких отдельных клеточек создало именно…}
\people{(Ольга) Коллективный разум.}
\soul{(Гена) Коллективный разум. В человеке в данном случае сейчас существует - коллективный разум. И этот коллективный разум ещё и создал сознание. Именно сознание вот, к которому мы привыкли, что вот это вот… вот это есть… А-а, то бишь - сознание… По-моему, я уже наверное в который раз повторюсь, да? - Это вот, адаптация к окружающей среде. }
\people{(Ольга) Угу. Слушай, ну, так про Землю. В общем-то, она умирает. Ты так сказал, да? - Где-то.}
\people{(Белимов) Но почему?}
\people{(Ольга) Не умирает? }
\soul{(Гена) Я не сказал, что она умирает!}
\people{(Ольга) Ну, старая, да?}
\soul{(Гена) Она - стареет!}
\people{(Ольга) Стареет, вот как! Она - стареет. А наша система солнечная?}
\soul{(Гена) А одинаковый возраст.}
\people{(Ольга) Одинаковы, да?}
\soul{(Гена) Да.}
\people{(Ольга) А вот, вообще вот, все планеты солнечной системы были созданы примерно одинаково? В смысле, в одно время?}
\people{(Белимов) Марс старее?}
\soul{(Гена) Нет! Нет-нет-нет! Луна - это новое. Хотя я не знаю, почему кругом говорят что луна - прародительница. Потому что это-то, в нашей солнечной системе это совершенно новое. Сама по возрасту? По возрасту, да! – Старая! Это осколок. Это осколок, причём чужеродный осколок, скажем так.}
\people{(Ольга) Угу.}
\soul{(Гена) Не нашей, именно вот…}
\people{(Белимов) Солнечной системы?}
\soul{(Гена) Да. … Марс…}
\people{(Белимов) Марс старее Земли?}
\soul{(Гена) Да, он старее Земли. А вот, Венера молодая.}
\people{(Белимов) Молодая, да?}
\soul{(Гена) Она моложе даже Земли.}
\people{(Белимов) А вот интересно, как твои ощущения, мы на Марсе будем когда-то? И сможем ли там: жить, действовать, работать?}
\soul{(Гена) Ну, это мне уже придётся говорить о будущем, да? О будущем мне не хотелось бы говорить вообще.}
\people{(Ольга) Не надо - не надо.}
\people{(Белимов) Хорошо. На луне, вот, были космонавты - астронавты. Всё-таки там есть какая-то жизнь, или это скрывается тщательно? Инопланетная, ``иномирянская'' жизнь есть? }
\soul{(Гена) Ну, вообще-то, чисто как в нашем понятии - нет. }
\people{(Ольга) Ну, да. Как физически - нет.}
\soul{(Гена) Нет, почему? Там физически - есть. }
\people{(Ольга) Нет, ну, я имею в виду…}
\soul{(Гена) Ну, скажем, в нашем физическом понятии? – Да, Нету!}
\people{(Ольга) Ну, да. Я имею в виду, как мы, это… Ген! (обращается к ``переводчику'')}
\people{(Белимов) Но там находят приборы, там полёты зарегистрированы… кораблей…}
\soul{(Гена) Знаете, для того чтобы мне, допустим, на ту же попасть Луну, или тот же Марс… Но, вообще-то, мне не хотелось. Почему-то я боюсь.}
\people{(Ольга) Не надо.}
\soul{(Гена) Это мне просто нужно дать счёт от 70-ти.}
\people{(Ольга) Не надо.}
\people{(Татьяна) Не надо.}
\people{(Ольга) Это потом как-нибудь может, когда мы научимся.}
\people{(Белимов) Хорошо. Когда в будущем, конечно…}
\people{(Ольга) Когда мы научимся, а то мы сейчас сам понимаешь: если ребёнку дать оружие…}
\people{(Белимов участникам) То есть это… астральные полёты он тоже может ощущать?}
\people{(Ольга) Семьдесят один, да? От 71-ого?}
\soul{(Гена) Это не астральный полёт, зачем же? Понимаете, мы путаем. Дело в том, что, в принципе, плоть, вот наша плоть физическая, да?}
\people{(Ольга) …Глубина сознания…}
\people{(Татьяна) …Сознание открытое.}
\soul{(Гена) …Оно нам не нужно, в принципе-то. И, как говорится, если я возьму сейчас, допустим, и, перемещаясь куда-то на какую-то другую планету, скажем так, да? - Ну, зачем мне нужна эта плоть?}
\people{(Ольга) Ну, да.}
\soul{(Гена) Зачем я буду таскать её с собой?}
\people{(Ольга) Таскать, да!}
\soul{(Гена) Ведь, вы же не таскаете с собой, допустим, тот же самый письменный стол?}
\people{(Белимов) Угу.}
\people{(Ольга) Да-да.}
\soul{(Гена) Только из-за того что он помогает писать. Правильно же?  }
\people{(Белимов, Ольга) Да.}
\soul{(Гена) А здесь, получается, что мы вот хотим этот стол взять с собой, прихватить.}
\people{(Ольга) (Смеётся)}
\soul{(Гена) То же самое получается.}
\people{(Ольга) Это как в поход свой дом брать, наверное? }
\soul{(Гена) Ну, да. А здесь… Вот, ``астральный'' - это не правильное, вообще-то, выражение. Ну, как…“полёт''. Дело в том, что, как… Вы же хотите, чтобы я увидел ту же самую луну сейчас данную, не в астральном мире же, правильно?}
\people{(Ольга, Татьяна) Угу.}
\people{(Белимов) Да.}
\soul{А именно в том, как говорится…}
\people{(Белимов) В трёхмерном.}
\soul{(Гена) В нашем, да!}
\people{(Белимов) Угу.}
\soul{(Гена) То, как это можно говорить об астральном полёте? Чисто сознанием, получается.}
\people{(Белимов) А ты способен сознанием, значит, на другие планеты перескочить?}
\soul{(Гена) Ну, а разум для чего дан? Разум - он создал сознание, как говорится, для Земли. Так?}
\people{(Ольга) Угу.}
\soul{(Гена) Он тут же по желанию - если я сумею запустить сейчас его механизм - он тут же создаст сознание: для Марса, для Луны, для любой другой планеты, понимаете? А это автоматически я уже, получается, нахожусь там.}
\people{(Ольга) Ну, да.}
\soul{(Гена) Причём это довольно…. Ну, нельзя говорить о времени перемещения, допустим. }
\people{(Ольга) Да-да-да.}
\soul{(Гена) Это уже, не нужные понятия.}
[Щелчок]
\soul{(Гена) Сл-сл… – Слышу.}
\people{(Белимов) Но получается, что и полёты на кораблях не нужны?}
\soul{(Гена) Ну, почему? Если нам необходим этот письменный стол, то конечно нужно. }
\people{(Ольга) (Улыбается)}
\soul{(Гена) Нам придётся создавать корабли, и всё, и так далее. И ещё почему? Потому, что мы же сейчас просто не умеем этого делать. Мы, как? Ну, живём же в этом теле и без этого тела себя не ощущаем. Мы же не принимаем сны всерьёз, правильно?}
\people{(Ольга) Да.}
\people{(Белимов) Угу.}
\soul{(Гена) Ну, и, поэтому нам нужно – хочешь не хочешь – мы будем сочинять эти корабли. И это будет и в 23-м веке, как говорится, это будет всегда, если мы доживём до этих времён. Нам придётся перемещать свою физику-то до тех пор, пока она у нас есть.}
\people{(Белимов) Гена, а если сознанием переселяться…?}
\soul{(Гена) Понимаете, для того, чтобы избавиться полностью от физики, мы должны… научиться в совершенстве владеть этой физикой, в таком совершенстве… Ну, получается как? ``Мы избавимся от физики!'' – Знаете, вот, сколько я слышал, да? – А, сейчас только понимаю, что избавиться от физики, это как раз-то, наоборот, понятие имеется в виду, что мы должны так…}
\people{(Ольга) Её усовершенствовать… Не усовершенствовать,…}
\soul{(Гена) …Понять.}
\people{(Ольга) …Овладеть ею.}
\soul{(Гена) Владеть ею до такой степени, что мы можем спокойно эту физику создать в любом месте.}
\people{(Ольга) Угу.}
\soul{(Гена) То есть, грубо говоря, для того чтобы нам, как говорится – ``Я хочу полететь на Луну!'' - Ну, зачем мне нужно два Харитоновых? Правильно?…}
\people{(Ольга) Ну, да.}
\soul{(Гена) …Который один здесь, другой на Луне, так? Я просто беру просто-напросто, как бы…}
\people{(Татьяна) На молекулу.}
\soul{(Гена) …Зная, владея вот этой физикой, я просто… Даже не на молекулу, потому что если я расставил, разбросал на молекулы – это не всё знание.}
\people{(Татьяна) Ну, да, вообще-то.}
\soul{(Гена) Потому, что кто-то придёт извне и может собрать снова этого человека, как говорится. Потому что, молекулы - вот они всё-таки никуда не исчезли. А именно, преобразовать её в совершенно что-то иное. Был Харитонов, стал, допустим там, скажем так - просто воздухом, именно воздухом. Именно, теми же самыми молекулами воздуха, а не просто, что ``распылённые Харитошкины''. Понимаете?}
\people{(Ольга) Ну, да.}
\soul{(Гена) Вот, и там я создаю себя точно таким же с той же плотью. Всё! Ну, всё до мелочей! Вот это и называется…}
\people{(Татьяна) Владение?}
\soul{(Гена) …Владение, полное овладение вот этой вот… }
\people{(Ольга) Силами?}
\soul{(Гена) …Материей, физической материей. А в итоге, получается, вроде бы как и нам и избавиться. Понимаете? Просто вот, как, ну, неправильно подобрано слово, что ли? Мы должны избавиться, как говорится, от физических тел, ``Мы должны утоньчаться!'', - Понимаете?}
\people{(Ольга) Угу. }
\soul{(Гена) Вот.}
\people{(Ольга) И мы должны, не избавляться что ли?}
\soul{(Гена) А мы понимаем смысл такой, что мы должны, просто вот, всё тоньше-тоньше-тоньше. }
\people{(Ольга) Ну, да.}
\soul{(Гена) В Освенцим нас посади,- и мы почти уже и на ``седьмом небе''! (утрировано сказано. Прим.)}
\people{(Ольга) Ага… Слушай, так, вот это вот Кришна, как раз вот этим вот… Ты же читал Гиту? (Бхогават Гита. прим.)}
\soul{(Гена) Ну, вообще-то, я всю не читал. Ну, я могу сейчас только посмотреть.}
\people{(Ольга) Ну, там, когда: ``И множество меня там…трам - папам… И вот, каждая там, где-то там, и на том, и на сём, везде на всех планетах и так далее…”}
\soul{(Гена) Да, ``Множество меня''. И причём, понимаете, в чём смысл? Здесь уже прямой смысл ``множества Я''. Потому что, если именно вот… Если мы владеем, то сколько много ``подводных камней''? ``Я овладел уничтожением этого тела!'' - Разбросать на молекулы? И всё и не больше. Это ещё - не овладел полностью.}
\people{(Ольга) Ну, да.}
\soul{(Гена) Когда я умудрился переместиться на Марс, но на Луне меня не осталось, как говорится, да? }
\people{(Ольга) Угу.}
\soul{(Гена) Это тоже - не полное владение. А вот, когда я… Полностью овладел физикой, именно тогда, когда я, получается, присутствовал…}
\people{(Ольга) Везде.}
\soul{(Гена) …абсолютно везде. Где нужно, там я и материю создам. И получится, что меня, - если мне необходимо, - меня будет сотни ``Харитошек'', которые будут лежать перед разными людьми и махать руками. Понимаете? И все всё по-разному рассуждать.}
\people{(Ольга) Ну, да-да-да!}
\soul{(Гена) Ну, вот. Полное владение.}
\people{(Ольга) Вот, в Индии так бывает. Ну, такие случаи описывают, да? – Что там материализации  происходят. Там, учитель перед учеником появляется в теле, да? – Вот, есть такие вот на Земле? Были люди? Или есть хоть, вот… Ты не слышишь?}
[Щелчок]
\people{(Ольга) Слышишь?}
\soul{(Гена) Слышу.}
\people{(Ольга) Вот, скажи, – давай, это выясним, а то мы так вот спрашиваем тебя, мучаем-мучаем – тебе это не утомительно? И, вообще, как вот, ты к этому относишься?}
\soul{(Гена) Ну, как - ``утомительно''? Не знаю. Чисто тело… Я знаю, что оно у меня есть, но оно мне пока, как бы ``до лампочки'' что ли. }
\people{(Ольга) Тебе самому-то интересно, или как вот, хоть с нами беседовать? Или, как? Ну, ведь знаешь, бывает  в конце концов вот, приходит друг, да? С ним говоришь-говоришь, а потом приходит момент, вроде бы как… }
\soul{(Гена) Да она знаете… Она (беседа прим.) интересна, в смысле реакции. Понимаете?}
\people{(Ольга) Наши?}
\soul{(Гена) Да, потому что, в принципе, я сейчас в таком состоянии, что, в принципе, вот все эти вопросы кажутся наивными.}
 
\people{(Ольга) Ага!}
\soul{(Гена) Что, мол, и так всё понятно, ясно.}
\people{(Ольга) Да-да-да.}
\soul{(Гена) А вот сама реакция очень интересна. Но я в смысле, что, конечно, если я сейчас проснусь, и, получается-то, все эти мои ``супер – знания'' конечно-то и опять где-то в глубине останутся.}
\people{(Ольга) А вот, как бы ты наружу их бы вытащил бы? Вот интересно, как сейчас.}
\people{(Белимов) Есть способ, действительно, подключаться…?}
\soul{(Гена) Есть множество способов. Но знаете, это очень опасно. Дело в том, что есть соблазн, как говорится, на ``чёрную сторону'' перейти.}
\people{(Ольга) Угу. Да.}
\soul{(Гена) То есть…}
\people{(Ольга) Ну, показать себя?}
\soul{(Гена) Я буду более желательным, желательной добычей, понимаете? - Если я буду.}
\people{(Ольга) Ну, да-да-да.}
\people{(Белимов) Угу.}
\soul{(Гена) Так-то - я вроде в себе всё скрыл, вот эта вот, как говорится, реакция… }
 
\people{(Белимов) Защиты?}
\soul{(Гена) …Самого… Да! - Защиты вот, сознания. Тот же самый - разум. Он, понимаете? - Разум хочет, чтобы мы знали больше, - ну, как и родители хотят, чтобы ребёнок знал больше, - но всё равно это делается постепенно, не сразу. }
\people{(Ольга) Ну, да.}
\soul{(Гена) Чтобы то же самое сознание не надорвалось. Вот. И, чтобы, как говорится, не быть приманкой, потому,  что  то же самое… В принципе, вот эти вот космические войны, как говорится… - Да! Они существуют. Почему? Потому, что эта вся, та же самая физика. Просто здесь, я - человек, да? Вот, муравьи с муравьями сражаются, так?}
\people{(Ольга) Угу.}
\soul{(Гена) Животные с животными сражаются? Человек с человеком сражается. Соответственно, и даже планеты с планетами воюют, понимаете? Только вот, не в таких вот планах, как вот, пушками там, допустим. А вот, допустим, гравитационными полями, да? }
\people{(Ольга) Угу.}
\soul{(Гена) То же самое, планета - страдает вампиризмом - она хочет приобрести себе, как говорится, лишний кусок плоти. Она тот же астероид себе притягивает, понимаете? }
\people{(Ольга) Ага.}
\soul{(Гена) Та же самая звезда - она борется за возможно большое количество планет. А бывает - звёзды, чаще, конечно, аскеты что ли, да? }
\people{(Ольга) Угу.}
\soul{(Гена) Они не признают рядом никого, и для того чтобы вот, быть одинокой, она просто-напросто притягивает и сжирает в себе. Понимаете? А бывают те, которые, как вот, знаете, любят вот… Ну… любит человек животных, да?  }
\people{(Ольга) Угу.}
 
\soul{(Гена) Он не любит людей, но всё равно какая-то живность нужна, да? - Он любит животных. Вот, та же самая, допустим, звезда - она не хочет признать планету, да ещё тем более, чтобы она была населена эта планета, понимаете? Ей достаточно будет завести себе парочку астероидов…}
\people{(Ольга) Угу.}
\soul{(Гена) …Комет.}
\people{(Ольга) Ну, да. }
\soul{(Гена) Понимаете? Вот, чтобы они, как говорится, не давали скучать. То есть здесь, чисто вот эта психология - она на более просто… Ну, это я грубо, конечно, сказал.}
\people{(Ольга) Угу. Ну, да-да-да.}
\soul{(Гена) Там совершенно по-другому, но примерно принцип тот же.}
\people{(Ольга) Слушай, так это вот - физика, всё в физическом, да? - Плане. А в более дух… Ну… Ближе к духовному, там меньше что ли вот, происходят таких столкновений?}
\soul{(Гена) Ну, если чисто полностью духовное…}
\people{(Ольга) То там нет, наверное?}
\soul{(Гена) Да, нет! И чисто даже духовное, это, как говорится… Ну, как объяснить? Есть миры, не Вселенные именно, знаете, а вот я не знаю, как даже сказать… Ну, как…Ну, вот, как у нас, да? - Город, да? - Один город, другой, да?}
\people{(Ольга) Угу.}
\soul{(Гена) И, скажем так… Или вот: одна звезда, там другая звезда.- Вот, также существуют не вселенные, а я даже не знаю что. Совершенно-совершенно что-то иное, совершенно, понимаете? Вот это множество вселенных, галактик там, миров… И всё это находится в одном. Скажем так, на Земном шарике, как говорится, да? Множество городов этих, да? - И посёлков. Вот, и так же вот эти. А есть ещё что-то извне, какая-то совершенно другая, скажем так, планета. Ну, я не знаю, как это даже объяснить.}
\people{(Ольга) Ну, да. Ну-ну!}
\soul{(Гена) Совершенно что-то иное. Вот, между ними борьба. И вот, где-то вот на границе, вот этих вот,… я не знаю - ``супер миров'', здесь нельзя подобрать ни одного слова никакого. Ну, примерно может быть поняли. Вот, на границе вот этих миров-то вот, и находятся вот те, которые говорят с нами. Понимаете?}
\people{(Ольга) Ага! Тонкие плёнки эти самые.}
\soul{(Гена) Не-ет! }
\people{(Ольга) Ну, как, они говорят…?}
\soul{(Гена) Нет-нет-нет! }
\people{(Ольга) Нет?}
\soul{(Гена) Нет! Ни в коем случае!}
\people{(Белимов) ОНИ не говорят с нами, да? }
\soul{(Гена) В том то и дело, что, вот это вот - множество пузырей, да?}
\people{(Ольга) Ага!}
\soul{(Гена) Множество пузырей, вот это множество пленок, соединяющее пузыри, и вот это как раз-то и есть, вот это вот – множество вселенных.}
\people{(Ольга) Ну!}
\soul{(Гена) А есть ещё что-то совершенно другое. }
\people{(Ольга) Угу!}
\soul{(Гена) Совершенно противоположное.}
\people{(Ольга) Может быть это то, что оно, как бы ``амортизатор'' что ли? Ну, как? Грубо, да?}
\soul{(Гена) Ну, я… – Не-ет! Я не могу сказать. Здесь нельзя подобрать слов, ни ``антимир'', ничего это не подходит. Совершенно – совершенно иное всё. Понимаете? И, вот, между ними… Ну, я не знаю как… Давайте скажем так, вот, вот эти вот - множество вот этой пены, да? - Вот этих пузырьков, плёнок – это вот наша Вселенная, да?}
\people{(Ольга) Угу.}
\soul{(Гена) А вот теперь можете представить, что вот это вот, как раз-то вот это множество вселенных - это всего лишь для каких-то других, ``супер'', скажем так, миров, всего лишь только маленький пузырёчек?}
\people{(Ольга) Ну, да. Да-да-да-да.}
\soul{(Гена) Или молекула даже, понимаете?}
\people{(Татьяна) Макрокосмос - микрокосмос.}
\people{(Ольга) То есть это всё-всё, и, как у нас в человеке всё это?}
\soul{(Гена) Ну, здесь нельзя говорить макро/микрокосмос по той причине, что это уже не космос, а совершенное другое.}
\people{(Ольга) Ну, это по отношению к чему? Допустим… Относительно нас, допустим, да? И относительно тех тоже, и относительно ещё других тоже, да ведь, так?}
\people{(Татьяна) Ну, вот это всё - наше полностью, все эти ``пузырьки'', для них они, как один ``пузырёчек'' для нас вот это всё, да?}
\people{(Ольга) Ну, да. Вот как наше тело для нас вот - целое, да? - Вот, со стороны. Так оно… Ты слышишь нас?}
[Щелчок]
\soul{(Гена) Да, слышу.}
\people{(Белимов) Ну, хорошо. А вот, скажи, вот в этом состоянии ты ощущаешь, что Солнце, это… термоядерная реакция там идёт или… говорят, что там совершенно другие энергии?}
\soul{(Гена) Ну, это мне надо было бы учебник физики надо взять, чтобы хоть знать что такое ``термоядерная реакция''.}
\people{(Белимов) Ага. Ладно, Ген. Тогда…}
\soul{(Гена) Нет, просто, именно представление. Я… Ну, как… Я чувствую, что сейчас такое Солнце, а вот имя, как говорится, что происходит там –  я не знаю, потому что это надо взять, действительно… Что такое ``термоядерная реакция'' для меня? Ну, если то, что по моим знаниям ``термоядерная реакция'', то – нет.}
\people{(Белимов) Угу. Ну ладно. Всё!}
\people{(Ольга) Ген, кто с нами разговаривает, вот, ты к ним сам приходишь, да? - Как бы выходит так. Почему ты именно туда приходишь, а не куда-то? Но, хотя ОНИ говорят, что ``Вы – это вот – Боги, а мы в вас''. Это что-то вот тоже значит?}
\soul{(Гена) Не-е… Ну, видите как… В принципе, вроде бы, вот как, была вот у нас, да? – Версия, что это, мол, будущий я, разговариваю, да?}
\people{(Ольга) Угу-угу.}
\soul{(Гена) Вообще-то, она не верна.}
\people{(Ольга) Это вообще даже – нет. Да? }
\soul{(Гена) Здесь…}
\people{(Ольга) А в чём…? Если ОНИ человеческую стадию проходили тоже.}
\soul{(Гена) Ну, как объяснить?  Дело в том, что они вот находятся как раз-то, понимаете? И ни в том мире, и ни в том мире. Понимаете? Они сейчас - нигде. Ни там, ни там. То бишь, в данном случае можно говорить, что они свободны. Они свободны от физики потому, что там действительно – нет вообще никакой физики! Понимаете? Ни физики вот этой метагалактики, допустим, ни физики не той метагалактики. Вот они сейчас абсолютно без всякой физики находятся. И, значит… Как объяснить? Любое движение молекулы, - вот это физически, - они очень хорошо видят.}
\people{(Ольга,Татьяна) Угу.}
\soul{(Гена) Потому что… Ну, как объяснить? Чистая вода, как говорится, да? И любая песчинка – она видна.}
\people{(Ольга) Ну, да-да-да.}
\soul{(Гена) И вот, по этому принципу – они видят. Но дело в том, что для того, чтобы… Ну, как сказать… Ну, не ``испачкаться'', а вот именно чтобы, вот эту… Ну, как бы – их не затянуло, вот в эту физику, понимаете? Им приходится очень, вот, окольными путями. Там, знаете, сколько вот этих ``переводчиков''…}
\people{(Ольга) Ну, да-да-да.}
\soul{(Гена) Чтобы сейчас сказать что-то, допустим, да?}
\people{(Ольга) Ага.}
\soul{(Гена) ``Переводчиков'' там тысячами. Понимаете? Чтобы вот это, окольными путями, чтобы самим, как говорится, не дотронуться до этой физики нисколько вот. Они вот… ``Граница'', да? Они начинают воздействовать на границу. То есть производят какие-то колебания вот этой границы. Вот эти границы, скажем так… Ну, не буду вдаваться. Ладно. }
\people{(Ольга) Ну, да.}
\soul{(Гена) Смысл в том, что получается очень множество ``переводчиков''. Все они: трактуют, переводят. Понимаете? Вот, как говорится…}
\people{(Ольга) Угу. И доходит уже…}
\soul{(Гена) Доходят, как говорится, до моего разума. Так? Разум переобрабатывает, заставляет сознание вести сразу параллельно несколько вот, работ. Одна работа сознания – это вот, именно занять чем-то тело. Понимаете? }
\people{(Ольга) Угу.}
\soul{(Гена) Другая часть сознания занимается тем, чтобы вот это вот занятие тела не могло мешать. Понимаете, как всё устроено? Потому что, чтобы, допустим, держать руки, те же, да? - Надо чтобы они, скажем так… Ну, не чувствовалось онемения, скажем так, потому что в таком положении обязательно, как говорится, нарушается кровообращение, так? Вот, - воздействие побочной реакции. Чтобы вот это вроде бы как бы вредное, да? – даже могло бы принести ещё и пользу –  это ещё одна часть сознания. Вот почему, проходит зубная боль и так далее, допустим. И вот это множество-множество, только даже на физическом уровне, чтобы вот моим телом заняться, чтобы хотя бы просто его очистить, да? - Вот, от шлака или от этих вот воздействий побочных вот, что сейчас во мне происходят. И потом уже только ещё, вот это всё происходит. И причём это должно идти, как говорится…, вот этот разум - не должен чувствовать давление извне. Вроде бы как он сам этим всё занимается, понимаете? Он, конечно, понимает, что это внешние обстоятельства заставили, –  те же колебания, оболочки, скажем так, – заставило его это совершать, но он должен делать самостоятельно. Вот почему надо быть осторожным, вот, при следующих счётах.}
\people{(Ольга) Ага.}
\soul{(Гена) Потому что, чем глубже, тем получается, больше я буду уходить вот, на уровень…}
\people{(Ольга) Ну, да-да-да. Ага. К Ним…}
\soul{(Гена) …От физики, скажем так. }
\people{(Ольга) Да-да-да.}
\soul{(Гена) Нет, к ним-то я не приду. Ещё не скоро.}
\people{(Ольга) (Смех)}
\soul{(Гена) А вот, именно уходить от физики. То бишь вот этот разум… Ну, как говорится, рычаги управления, вот эти нити вот. В принципе, что тело? - Это же марионетка. Вот эти нити, я всё большинство нитей буду бросать на произвол. И вот малейшее ``что-то'', конечно, может и, допустим, на счёте ``99-ть'' я могу много что творить, вплоть до того что вы будете сами всё это ощущать, да? - На 90-м счёте. Я смогу вам передать картины, которые я вижу сам. }
\people{(Ольга) Угу.}
\soul{(Гена) Понимаете? Грубо говоря, вы будете, не махая руками и не задавая вопросы, вы будете всё это проводить в уме через меня. Но, я тогда получу все ваши, и ``за'', и ``против''. Ну, и плохое, и хорошое. То есть, во мне станут сразу шестеро, понимаете? А вот, смогу я этих всех шестерых удержать…}
\people{(Ольга) Выдержать. Ага!}
\soul{(Гена) Вот это я не знаю… И поэтому, здесь… Ну, примерно такая технология.}
\people{(Белимов) Ну, мы не будем торопиться, да.}
\people{(Ольга) Нет-нет. Нам надо уже научиться, конечно. Зачем нам это?}
\people{(Белимов) Видимо ``школа'' будет учёбы. Гена! (Обращается к Гене)}
\soul{(Гена) А смысл счёта? В принципе, вот, допустим, кто-то возьмёт и захочет подражать сейчас, да? - И будет также вести счёт.}
\people{(Белимов) Угу.}
\soul{(Гена) Это ничего не даст. Здесь смысл чисто такой символический, понимаете? Это просто, как бы типа договоренности, что, мол, ``давай так и так''. }
\people{(Ольга) Угу.}
\soul{(Гена) Вот давай, допустим, возьмём такой: счёт-счёт. Всё договорено. Договорено вплоть до каждого мелоча. Даже вот эти ваши ошибки, да? Вот: недосчёт, пересчёт – он должен компенсироваться, потому что очень много раз вы просто недосчитываете до конца – ``А-а! Заговорил? Прекраcно! Хорошо!'' Всё это отражается. Даже это договорено - определённое количество ошибок. И когда вот совершается…набралось это количество определённое, да? - Происходит срыв.}
\people{(Ольга) Угу.}
\soul{(Гена) Происходит срыв, чтобы, как говорится, как бы сбросить вот эту вот, очистить эту чашу, скажем так. Сбросить этот счётчик, да? - Чтобы начать его заново. Но существует ещё один счётчик, который считает именно, сколько было ``сбросов''.}
\people{(Белимов) Угу.}
\soul{(Гена) То бишь, понимаете… Ну, вот, скажем… Этот счётчик считает сколько было ошибок, так? Потом - переполнился… }
\people{(Татьяна) Сбрасывает.}
\soul{(Гена) Сбрасывает. Так? Теперь ещё стоит счётчик, который считает, сколько было сбросов. И когда определённое количество настигнет, тут вообще - контакт прерывается. Так? Существует ещё… Видите какая цепь?}
\people{(Белимов) Угу.}
\soul{(Гена) Получается, существует ещё счетчик, который считает, сколько было этих ``контактов''. Понимаете? И, как говорится, когда этот счётчик не сможет перепол… Здесь уже, наоборот, - когда он уже не сможет переполниться, контакты будут прекращены и оставлены до следующего раза. Ну, в смысле, получается, до следующей жизни. Потому что это уже, как говорится, это у меня это уже стало… }
\people{(Белимов) М-м. Ну, понятно…}
\soul{(Гена) А здесь вот, получается наоборот. Понимаете?- Если он не успел переполниться, а переполниться имеется в виду не количеством контактов, да? - а вот, обработанной информацией. И вот, если вот эта информация идёт сейчас впустую, допустим, да? - ну, не совершается никакой работы над этой информацией, не увеличивается от этого заряд, да? - не изменяется. То есть, получается как бы бестолковость разговора.}
\people{(Ольга) Ну, да.}
\soul{(Гена) Вот здесь вот, к сожалению, хоть и грубо, но здесь вот как раз и подходит эта фраза – ``Не метать бисер''.}
\people{(Ольга) Угу.}
\soul{(Гена) И тогда, контакты прерываются полностью до следующей, как говорится, жизни, да? Или, не обязательно до следующей жизни, а, допустим, через десяток лет, через двадцать лет. Но это, к сожалению, это будет уже другой круг людей.}
\people{(Ольга) Угу.}
\soul{(Гена) Понимаете? Чтобы, как говорится, не вернуться. Или, может остаться тот же круг, но только с тем условием, если люди изменились тоже. То есть эта информация, как говорится, была всё-таки переработана.}
\people{(Ольга) Слушай, ну, а мы как, хоть меняемся, хоть немножко-то? }
\soul{(Гена) Не-е! Ну, об этом давайте не будем говорить.}
\people{(Ольга) (Смех)}
\soul{(Гена) Потом, что ещё, в этих контактах. А! Ну, опять же это, кто они, да? Получается, вот, вроде бы вот эта версия, что, я - это они, это не подтверждается. И хотя, что самое интересное, в принципе, я там есть. Понимаете?}
\people{(Ольга) Ну, наверное, есть. Если… }
\soul{(Гена) Я там есть, но там, именно вот… Получается… Ну, как объяснить? Я там не один.}
\people{(Ольга) Ну, да. Они же говорят: ``Мы осознаём своё единство'', - вот так. Так значит ты…?}
\people{(Белимов) ``Не один'' это значит со своим учителем, что ли?}
\people{(Ольга Белимову) Нет. Он, как Всё. Он, как Все. Да?}
\people{(Татьяна) Вместе с ними.}
\soul{(Гена) Ну, не знаю. Может быть. Я не могу это описать.}
\people{(Ольга) Так это единство?}
\soul{(Гена) Понимаете, беда в том, что вот это множество ``переводчиков'', да? Они всё так сильно искажают, что, в принципе, вот  что я сейчас говорю… Но, нельзя сказать, что вот 90% там лжи или правды.}
\people{(Ольга) Ну, да.}
\soul{(Гена) В принципе-то, я всё говорю правду, да? Но, просто вот, не те… Некачественный перевод, скажем так, да? }
\people{(Ольга) Угу.}
\soul{(Гена) Я вижу картину одну,… Ну, мы-то между собой никак не можем договориться же, правильно? }
\people{(Ольга) Да-да-да.}
\soul{(Гена) А здесь вот куча их сидит. Вот, некачественный перевод. Вот, на счёте ``99-ть'', конечно, тут уже, получается, будет…}
\people{(Ольга) Более чистый?}
\soul{(Гена) Ну, как ``более чистый''? В принципе, только я, как говорится. Одно звено только отпадает, в принципе, понимаете? – Я. То бишь, я уже не буду для вас ``переводчиком'', как говорится, просто буду силой поддерживающая вас. И всё. И вы уже сами будете, как я говорю,… они говорят через меня, да? Также они будут с вами говорить, только что вот без слов…}
\people{(Ольга) Ну, ясно.}
\soul{(Гена) Вот, как со мной сейчас ведут вот, диалог без слов, да? Вот, а потом, я уж просто вам перевожу. Вот, то же будет с вами. Но учитывая, что вы будете проходить через меня, вот эти вот все силы, понимаете? Тем более…}
 
\people{(Ольга) Можно не выдержать?}
\soul{(Гена) …Противоречивые. Ну, в общем - это будет сложно, конечно.}
\people{(Ольга) Слушай, так, а вот каждый, вот из нас…Ну, не только о присутствующих говорим, а, вообще, просто из людей ведь…Каждый из нас - как они говорили - может вот как ты, быть ``переводчиком'', грубо говоря? }
\soul{(Гена) Да, конечно. Каждый это может быть.}
\people{(Ольга) А у тебя…}
\soul{(Гена) Просто что? - Страх.}
\people{(Ольга) Страх? }
\soul{(Гена) Страх, что не получится. Сомнения, что не получится. Вот он ложится: ``Хочу хочу хочу!'' Знаете? И счёт будет давать, но всё равно будет какая-то неуверенность, что вот, не получится там, понимаете?}
\people{(Ольга) Ну, да.}
\soul{(Гена) А сколько много вот, взять… Ну, сколько вот много, вот этих вот, в жизни мы упустили из-за этой неуверенности или из-за ложного чувства стыдливости, понимаете? Вот идёт человек, да? – Смотрит. Вот увидел,- ветка поломана, скажем так, да? - дерева. У него мысль промелькнула поправить её, да? А он этого не сделает, потому что вроде как люди кругом, неудобно,… кто-то увидит (теряется)}
[Щелчок]
[Меняется интонация ``переводчика”]
\people{(Ольга) Ты слышишь?}
\people{(Татьяна) Нет, это не он.}
[Щелчок]
\soul{(Первые) … Дать обратный счёт. И вы опять про него забыли. }
\people{(Белимов) Но мы ещё не хотели выходить из контакта, но я чувствую, что ``переводчик'' устал, и мы уже может быть не так…}
\soul{(Первые) Устал ``переводчик''?}
\people{(Ольга) Да нет. Переводчик - он…}
\soul{(Первые) Как он может устать, если ему сейчас не до тела?}
\people{(Белимов) Ну, мы не знаем. Может быть ему физиологически надо окончить? Ну, а что вы посоветуете?}
\soul{(Первые) Устали вы! }
\people{(Белимов) Да.}
\soul{(Первые) В первую очередь - устали вы!}
\people{(Белимов) Да. Точно! Так что нам сейчас тогда предпринять? Обратный счёт дать?}
\soul{(Первые) Да.}
\people{(Белимов) Ну, давайте.}
\people{(Ольга, Татьяна) [Дают счёт: 19-18-17-16-15]}
[Щелчок]
\people{(Белимов Гене) Какие ощущения? Ну-ка, расскажи хоть вкратце, пока здесь магнитофон ещё работает.}
(Обрыв)
\soul{(Первые) …Множество. Представьте, мать убила в утробе ребёнка. В следующей жизни эта мать будет убита. }
\people{(Ольга) Угу.}
\soul{(Первые) Всё, что она сделала, вернётся ей.}
\people{(Ольга) Ну, да.}
\soul{(Первые) И это уже будет тот не рождённый ребёнок - один вариант, и самый частый вариант. }
\people{(Ольга) Угу.}
\soul{(Первые) Второй, – когда…}
\people{(Ольга) [Даёт счёт: 1]}
[Щелчок]
\people{(Белимов) Чё, убрать их надо, да? }
\soul{(Гена) Я слышу часы.}
\people{(Белимов) Всё, их не будет.}
\people{(Ольга) Это ты?}
\people{(Белимов) Кто это?}
\people{(Ольга) Это ты, да?}
[Щелчок]
\people{(Ольга) Ну, продолжить можно вот?}
\people{(Татьяна) Второй вариант.}
\people{(Ольга) Да, второй вариант.}
\soul{(Первые) Второй вариант - печальный вариант - Месть! Там тоже существуют эти понятия, ибо, когда человек одержим, идеей отомстить - вот ваша одержимость - она остаётся. Ибо это был источник его жизни. И умирая, источник не гаснет. И тогда - просто месть. Месть того же ребёнка, обиженного родителями.}
\people{(Ольга) Вот, песня такая…}
\soul{(Первые) Есть вариант, когда дитё виновато, виноват ребёнок, - да, в общем-то, и не важно, - перед родителями. И опять, можно завести речь об аборте.}
\people{(Ольга) ``Когда б мы жили без затей, Я нарожала бы детей От всех, кого любила, - Всех видов и мастей'' (из песни Вероники Долины прим.) Всё ясно. (грустно) Геннадий Степанович! (обращается к Белимову, чтобы задавал вопросы прим.)}
\people{(Белимов) Меня интересует, действительно, что некоторые сущности иных миров через энергетические дыры могут проникать в наш мир и становиться видимыми? Это порождает сказки о полулюдях-полузверях, оборотнях, леших, вурдалаках. В этом есть действительно, смысл и правда?}
\soul{(Первые) Здесь нет ничего удивительного, это просто физика и не больше. Было бы удивительно, если ваш мир был бы одинок. Вот это было бы удивительным.}
\people{(Белимов) Но пока многие так и думают. Мы с трудом находим аргументы, чтобы доказать иное. Пока не получается.}
\soul{(Первые) А вы поймите, всегда идёт борьба. Даже истина, вот она - лежит, бери, смотри, разглядывай - многие подойдут и не уверуют. И это нужно! Для чего? Да чтобы вы искали, боролись, не стояли на месте. Чтобы вы не гнили, как застоявшее болото.}
\people{(Ольга) Вот, вы сказали, что наш мир уникален тем, что он содержит всё.}
\soul{(Первые) А мы говорили об уникальности?}
\people{(Ольга) Да-а.}
\soul{(Первые) А если мы говорим: - весь мир един… Уникально то, что вы живёте в этом мире - и не знаете об этом. Уникально то, что вы всё это знаете - и не владеете этим. Уникально то, что вы владеете всем этим - и не знаете о том.}
\people{(Ольга) А вот скажите, некоторые утверждают, что в некоторых астральных мирах нет животных. Но есть растения.}
\soul{(Первые) Ну, а есть множество миров. Почему бы нет? А мы знаем множество семей, где не держат животных.}
\people{(Ольга) Ну, да.}
\people{(Белимов) Понятно. А вот, скажите, - вампиры, это загадка для нас,- неужели есть такие люди, которым необходима человеческая кровь? И почему она необходима? Что их вызывает?}
\soul{(Первые) А вам необходима белковая пища?}
\people{(Белимов) Так, необходима.}
\soul{(Первые) Ну, а что же тогда спрашиваете?}
\people{(Белимов) Ну, а что есть какие-то повороты ДНК или какие-то… сознания? Почему именно кровь?}
\soul{(Первые) А это уже просто голая медицина. Возьмите справочники, и вы найдёте. Да, действительно существуют такие болезни. Извне только что немножко приукрашено, что человеку надо обязательно превратиться в волка, допустим. А все остальные симптомы - боязни света, боли - всё это есть.}
\people{(Белимов) Угу. А природа вампиризма какова?}
\people{(Ольга Белимову) Ну, нам только что объяснили.}
\soul{(Первые) Та же самая, какая и у вас сила заставляет есть и применять белки. Та же самая. Вы занимаетесь тем же самым. Какая разница, вы питаетесь кровью человека или кровью животных? Есть ли в том большая разница?}
\people{(Ольга) Нет, большой нет. Тут уже этика как бы.}
\soul{(Первые) А что такое этика? Этика –  всего лишь, выдуманное вами на данный момент жизни. Простите, 10-ть веков назад этика была совершенно другая. Пещерный человек вряд ли обладал вашей этикой.}
\people{(Ольга) Да. Это верно.}
\soul{(Первые) И культура была совершенно иной. И всегда считалось нормой… женщина… заметьте, женщина, что сейчас являет…(теряется)}
\people{(Ольга) [Даёт счёт: 1]}
[Щелчок]
[Меняется интонация ``переводчика”]
[Выход на свободное сознание ``переводчика”]
\people{(Белимов) Ну, наконец-то, мы вышли на тебя! Мы хотим продолжить вопросы, которые мы на прошлом сеансе говорили. Мы убедились в том, что повышенные результаты относительно памяти, сознания твоего… с  целью проверки, уточнения вот этого признака памяти, мы продолжаем свои вопросы. Ты готов с нами поработать?}
\soul{(Гена) Готов.}
\people{(Ольга) У тебя есть желание?}
\people{(Белимов) Есть желание?}
\soul{(Гена) Как обычно.}
\people{(Ольга) (Смеётся)}
\people{(Белимов) Ну, вот помнишь ли ты: дату, день, возраст, когда ты впервые пошёл ногами?}
\soul{(Гена) Помню.}
\people{(Белимов) Дату назови.}
\soul{(Гена) Дату?}
\people{(Белимов) Да.}
\soul{(Гена) М-м… У нас не было календаря.}
\people{(Белимов) Хорошо. Возраст, cколько тебе было лет? }
(Пауза: 5 cек)
\people{(Белимов) Не помнишь? Ты ещё не мог считать тогда, да?}
\soul{(Гена) Да вот, как раз-то и считаю.}
\people{(Белимов) А-а! По дням, что ли? Тебе не больше годика наверно было.}
\soul{(Гена) Нет, я считаю ``кормёжками''.}
\people{(Ольга) Кормёжками?}
\people{(Белимов) О-ой! Неужели ты все кормёжки помнишь?}
\soul{(Гена) Да, конечно.}
\people{(Ольга) Аппетитом хорошим ты обладал?}
\soul{(Гена) Вообще-то, нет.}
\people{(Ольга) Нет?}
\people{(Белимов) Получается, что в цифрах можешь помнить, даже кормёжки свои.}
\soul{(Гена) Почему ``в цифрах''? }
\people{(Белимов) А как?}
\soul{(Гена) Я только помню кормёжки и всё. Я помню самую первую кормёжку, помню самую последнюю.}
\people{(Ольга) Всё верно. Тогда…}
\people{(Белимов) Последнюю, ты как…? Тебя отучили, горчицей мазали грудь или как? Каким образом помнишь?}
\soul{(Гена) Нет. Просто после этого не давали и всё. }
\people{(Белимов) Не давали. И ты плакал, ты переживал?}
\soul{(Гена) Нет.}
\people{(Белимов) Нет? Ты быстро привык, да?}
\soul{(Гена) Нет. Просто получилось так, что меня стали кормить, и так, и так. А мне-то что? Лишь бы сыт был.}
\people{(Белимов) Понятно. И спокойно перешёл. Ну, в общем, ты впервые пошёл ногами, где-то, наверно, в годик, да?}
\people{(Ольга) На какой кормёшке? Ну-ка посчитай! (смеётся)}
\soul{(Гена) Счёт? }
(считает: 5 сек) 
\people{(Белимов) Наверное, трудно?}
\soul{(Гена) Да, нет.}
\people{(Ольга) Тогда же он не умел считать. Как ему? Конечно.}
\people{(Татьяна Ольге) Как это не умел?}
\people{(Белимов) Неужели ты сейчас можешь выдать нам цифру?}
\soul{(Гена) Пожалуй, да.}
\people{(Ольга) Ну, хорошо. Говори!}
\people{(Белимов) Так, говори!}
\soul{(Гена) 1297-мь или 96-ть.}
\people{(Белимов) О-о! Молодец! А когда ты в первый раз заговорил? Ну, не в кормёшках, а хотя бы…}
\people{(Ольга) (Смеётся)}
\people{(Белимов) Ты был молчуном или всё-таки довольно рано заговорил?}
\soul{(Гена) Вообще-то, родителей я не беспокоил - спал неплохо. Заговорил?  – Нет, я заговорил поздно.}
\people{(Белимов) Угу.}
\soul{(Гена) Я помню, дядя Вася советовал повести меня к врачу и обрезать уздечку, чтобы я быстрее заговорил.}
\people{(Белимов) То есть у тебя что-то в гортани было не то, да?}
\people{(Татьяна Белимову) Под язычком.}
\people{(Белимов) Под язычком. И так они и сделали, родители?}
\soul{(Гена) Нет.}
\people{(Ольга) Правильно сделали.}
\people{(Татьяна) Испугался и заговорил.}
\people{(Ольга) Испугался ты, да?}
\people{(Белимов) Ты испугался и после этого? }
\soul{(Гена) Нет. Почему? Просто мне тяжело было. Я довольно-то долго учился говорить правильно. В принципе,  до сих пор толком-то и не научился. }
\people{(Ольга) Скажи, а думать, ты всегда думал? - В мыслях.}
\soul{(Гена) Всегда.}
\people{(Ольга) И как, в словах или вот, по-другому как-то?}
\people{(Белимов) В образах или…?}
\soul{(Гена) Нет, конечно, как-то по-другому. Но… Ну, как сказать? Вот, если сейчас вспомнить, допустим, когда я впервые пошёл?}
\people{(Белимов) Так.}
\soul{(Гена) Естественно я словами не думал. А вот, чувства… Чувства, - да.}
\people{(Белимов) Ну, ты осуждал иногда родителей, когда был маленький за какие-то их действия? Ты понимал, что они неправильно поступают, и другие люди обманывают, допустим?}
\soul{(Гена) Конечно, как и обычно дети наверно все - всегда обижался, не понимал за что, почему.}
\people{(Белимов) Но в результате они же оказывались, порой, правы? Ты лез туда, куда не надо было.}
\soul{(Гена) Да не всегда! Почему же? Не всегда!}
\people{(Белимов) Ты мысленно соглашался с ними, что они в чём-то правы, допустим?}
\soul{(Гена) Нет. Ребёнок просто забывает. Во всяком случае, я - просто забывал.}
\people{(Белимов) Угу.}
\soul{(Гена) Я, конечно, обижался, что наказан, вроде как несправедливо, за что я не мог понять, почему. Ну, естественно сразу мысль, что меня не любят…}
\people{(Белимов) Угу. Что ты – чужой ребёнок.}
\soul{(Гена) Потом начинаю ``назло''. А потом? Потом просто забываю.}
\people{(Белимов) Какую-нибудь обиду, до того как ты мог говорить, вспомни. Назови, за что ты обиделся на родителей?}
[Щелчок]
\people{(Белимов) Не помнишь?}
\soul{(Гена) Помню.}
\people{(Белимов) Скажи!}
\soul{(Гена) Они меня мучали горшком.}
\people{(Ольга) Горшком?}
\people{(Белимов) В каком смысле? Слишком часто сажали, да?}
\soul{(Гена) Нет, я не любил на нём сидеть. Я всё время его выкидывал, и матери приходилось всё время его искать.}
\people{(Белимов) Но ты же был не прав. Ты что хотел в штанишки…?}
\soul{(Гена) Да, но мне это трудно было объяснить, если мне было неудобно там сидеть.}
\people{(Белимов) Да-а, но ты был не прав.}
\soul{(Гена) Но это сейчас – не прав. А относительно дитя –  прав.}
\people{(Ольга) Ну, да-да.}
\people{(Белимов) Угу. А скажи, первое твоё слово. Ты помнишь его?}
\soul{(Гена) Помню.}
\people{(Белимов) Какое?}
\soul{(Гена) ``Дай!”}
\people{(Белимов) ``Дай?''.}
\people{(Ольга) (Смеётся)}
\soul{(Гена) Да, моё первое слово было – Дай!}
\people{(Ольга) Не, многие дети говорят ``дай''.}
\people{(Белимов) Ну, ясно. Всё правильно. А скажи, вот, когда ты был маленьким, - совсем годовалый, трёхгодичный,- были у тебя необычные сны? Ты можешь о каких-то снах рассказать необычных? - Не то, что кормление там или страх какой-то.}
\soul{(Гена) Да пожалуй, самые необычные-то сны и были в этих годах-то.}
\people{(Белимов) Да? Много? Если мы однажды посвятим этому сеанс, ты сможешь вспомнить эти необычные сны?}
\soul{(Гена) Ну, наверное, да. Но… Наверное, да!}
\people{(Белимов) Ну, вот, какой-нибудь расскажи вот, хотя бы один. Мы не будем сейчас на это обращать внимания. Хотя бы один. Что тебе снилось?}
\soul{(Гена) Я в поликлинике… А-а, ну, это я как бы во сне, я вспоминаю, что было днём.- Я лежу в поликлинике на весах. Этот коридор, такой узкий, длинный в полуподвальном помещении… И вот там, вот эти вот, весы, дескать, полукруглые, вот я лежу на них, и я чувствую себя неуютно, потому что одна ножка свесилась… Ой, не свесилась, а как…}
\people{(Белимов) Болтается.}
\soul{(Гена) Нет, лежит чисто на весах и холодно ей.}
\people{(Белимов) Угу.}
\soul{(Гена) Вот, это было. И это мне ``один к одному'' снится во сне. }
\people{(Белимов) Угу.}
\soul{(Гена) Ко мне подходит старушка, ``Усю-сю!”- улюлюкает. Что-то там пальчиками делает,.. ``козу строит''.}
\people{(Ольга) (Смеётся)}
\soul{(Гена) А я почему-то вижу её молодой. Я знаю, что это она, но она молодая. Рядом с ней, как я понял, её муж… и я гляжу на неё и так чётко… Ну, как представляю… Вижу что ли всё это. И я вдруг у неё как бы спрашиваю: ``А где дядя?”}
\people{(Белимов) Угу.}
\soul{(Гена) И вот, когда я спрашиваю, получается, я спрашиваю у этой бабушки с ``рожками''. И она, строя рожки,- конечно, я же не спрашиваю это вслух, а вот как-то получилось так, что спрашиваю, - она, вдруг, услышала, что ли мой вопрос…}
\people{(Ольга) Ну!}
\soul{(Гена) И отходит, перекрещивается - ``Ой, не сглазить! Ой, не сглазить!”}
\people{(Белимов) Угу. Ты смотри, это интересно!}
\soul{(Гена) Нет, это не сон. Это я всё, получ… Нет! Получается, сон! Но всё это было. Я просто вспомнил, то есть во сне я прожил, как бы заново вот этот момент.}
\people{(Белимов) Это, на мысленный образ женщины вышел? Она себя вспомнила вдруг молодой, как когда у неё был такой же ребёнок, скорее всего, и тоже рядом стоял… То есть она, скорее…}
\people{(Ольга) Вот скажи, а твоим глазам многие удивлялись?}
\soul{(Гена) На счёт глаз мне всегда везло и очень много помогало, потому что, сколько у меня было таких случаев, когда мне просто что-то прощалось за ``красивые глазки''.}
\people{(Ольга) (смеётся)}
\soul{(Гена) Или, что-то допускалось, какая-то вольность за те же ``красивые глазки''.}
\people{(Ольга) У тебя в детстве тоже они такие яркие были? Наверное, ещё ярче, чем сейчас, конечно.}
\soul{(Гена) Ну, конечно, поярче. Большие ресницы.}
\people{(Ольга) Ну, да.}
\people{(Белимов) А когда ты очки одел? Чем испортил?}
\soul{(Гена) Это был третий класс.}
\people{(Белимов) Третий класс?}
\soul{(Гена) Где-то уже в конце первого класса я не понимая, что со мной происходит, стал смутно видеть. Но я считал, что это так и все наверно видят. Потому, что это было всё не заметно же.}
\people{(Белимов) Угу.}
\soul{(Гена) И я думал, что это так и должно быть. И… какое-нибудь пятнышко на стекле… А когда я приглядывался, оно расплывалось, получается, как бы вот… Ну, как,… показывают кадры молекул,- что-то где-то что-то двигается и так далее… И вот, я эти ``молекулы'', как говорится, как бы из этой капли видел эти ``молекулы''. Ну, я думал, что это вот у меня зрение такое, что я даже молекулы разглядываю. Радовался-радовался. Оказывается,- нет. Просто, ну, как бы сказать,… вот это моя близорукость позволяла - к сожалению, я забыл об этом, вовремя не понял, никто не подсказал - позволяла увидеть собственную радужку. Не просто увидеть…}
\people{(Белимов) Угу.}
\soul{(Гена) Действительно, я разглядывал не пятнышко, а просто вот, разглядывая это пятнышко, я умудрялся увидеть свою радужку.  }
\people{(Белимов) Угу.}
\people{(Ольга) Cкажи…}
\soul{(Гена) То есть, свой собственный глаз, как говорится. И, если бы вовремя попался бы кто-то, смог бы мне помочь, подсказать, то я бы мог бы видеть и большее. Например, мне никогда не составляло труда вытащить занозу, потому что я просто видел. Даже где-то, скажем так, сзади где-то заноза, - это надо, чтобы кто-то, допустим, вытаскивал или как, - я это спокойно видел. Я мог спокойно читать текст, - а мне приходилось это делать часто, потому что я был любитель почитать,- а родители, конечно же, заставляют учить уроки. Ну, вот, представьте, я сижу, зубрю литературу. Так? А под этой литературой у меня лежит книжечка с каким-то интересным рассказом там или повестью. И вот, когда родителей нет, я, конечно, умудряюсь отодвинуть книжечку литературы и читаю. Когда они появляются, я накладываю литературу на эту книжку, но продолжаю читать.}
\people{(Белимов) Как бы сквозь корочки даже?}
\soul{(Гена) Ну, по тем понятиям это получалось, - да, так, что я вроде бы читаю, как, сквозь книгу. Я и думал, что я вижу сквозь книгу. Это уж потом я уже понял, что просто у меня хорошая зрительная память была.}
\people{(Белимов) Схватывал всё.}
\soul{(Гена) И просто я эту страницу, как говорится, дорисовывал уже в воображении и вроде бы продолжал читать. Потому что, следующую страницу,  допустим, я не мог прочитать, мне всё равно надо было её перевернуть. Вот, когда я вот это вот, логически, как говорится, дошёл до этого… Ну, это прошло столько времени и было столько много самообмана…}
\people{(Ольга) Скажи, вот можно нам в таком состоянии произносить имена? Можно общаться вот как всегда?}
\soul{(Гена) Да, наверное.}
\people{(Ольга) Ты точно знаешь или…? А то мы говорим, имя твоё произносим. А может, вдруг нельзя?}
\people{(Белимов) Мне кажется надо уже…(заканчивать прим.)}
\soul{(Гена) Дело в том, что я вообще-то считал, что - подсознание. А вот как я послушал,- получается, я, как понять, - свободное сознание?}
\people{(Татьяна) Ну, да, свободное сознание.}
\people{(Белимов) Угу. Свободное. Потом ещё будет, и ещё. Надо бы, когда-то попробовать. Нам надо подготовиться. Сны - преследования. Были у тебя в детстве сны - преследования, сны – кошмары? }
\soul{(Гена) Да, конечно, было.}
\people{(Белимов) И как ты думаешь, чем они вызваны?}
\soul{(Гена) Ну, большинство снов вообще-то было вызвано именно чисто физиологически. Это… Да, легко конечно в этом состоянии,- как говорится, свободном,- рассуждать. А так-то…. Ну, вот снится мне сон, что вот, допустим,  я борюсь с драконом…}
\people{(Белимов) Угу.}
\soul{(Гена) Страшный дракон. Конечно, если он сейчас мне присниться и вот я буду в обычном, а не в этом счету,- то я начну сейчас выдумать - ``Что бы это значило, для чего это?''. Там, ну, выдумать можно. Оказывается, всё очень просто. Просто рядом стояла печь, которая уже остывала. Вот, с детства сон – борьба  с драконом. Просто стояла печь. И я, повернувшись,.. и кусок одеяла, как говорится, отошёл и прислонился к этой печи, нагрелся. Потом мать меня укрыла одеялом, а этот кусочек-то - он же тёплый, он меня греет. Ну, и получается…}
\people{(Белимов) Разжигает дыхание, да!}
\soul{(Гена) И получается вот – борьба с драконом. Но это знаете ведь, это я сейчас так могу сказать. А вот, проснись я, и какой-нибудь сон, и я могу, конечно, наврать что-нибудь, сочинить. потому что… }
\people{(Белимов) Ну, мы порой через такие сны - кошмары считаем, что выходим на прошлые жизни, на нашу гибель допустим, в воде, от чего-то там… Это не так?}
\people{(Ольга Белимову) Да нет, это не всегда так. Это по-разному}
\soul{(Гена) Нет-нет. Нет, это бывает, конечно. Ну, бывает.}
\people{(Ольга) Ну, да-да.}
\soul{(Гена) Но, чаще всего – ``мусор''. Мусор вот, именно окружающей среды. Понимаете, что интересно? Ведь… Вот, человек, да? Вот, пока он падает с кровати, вот у меня…}
\people{(Белимов) Угу.}
\soul{(Гена) Вот, я падал с кровати. Пока я падал… А, ну, что тут падать-то? Всего чуть-чуть. Я умудрился в этом сне пролететь, наверное, полкосмоса, побывать на куче планет…}
\people{(Ольга) Ну, да-да.}
\soul{(Гена) Понимаете? И ещё, везде побороться там, что-то повоевать. Ну, а дитё всегда… Войны мальчишкам снится. Понимаете, пока вот, как говорится, не упал.}
\people{(Белимов) Ну, ты говоришь, что скорость мысли значительно больше скорости света, наверно так?}
\soul{(Гена) Ну, наверное.}
\people{(Ольга Белимову) Вот, Геннадий Степанович, можно я вот несколько…? (Вопросов задам прим.) Ген, скажи, ты нашу собаку так вот, сейчас как её воспринимаешь? Можешь что-нибудь о ней сказать?}
\soul{(Гена) Вы знаете, я вас, честно говоря-то, и не вижу так вот. Просто, как, закрытые глаза…}
\people{(Ольга) Нет, воспринимаешь. Воспринимаешь именно. Не видишь, а именно воспринимаешь.}
\soul{(Гена) Ну… что-то… Скажем так - эллипс и вот зад эллипса очень такой тёмный.}
\people{(Ольга) (Смех) А нас воспринимаешь как одно целое, да? - Как бы. Нет?}
\soul{(Гена) Ну, когда я слышу голос, как-то я ещё могу различить вот…}
\people{(Ольга) Кто – говорит.}
\soul{(Гена) Ну, кто говорит - это понятно. Я имею в виду, что именно вот, как говорится…}
\people{(Ольга) Ну, выделить нас.}
\soul{(Гена) Да, как-то ещё могу выделить. А так, вообще-то, да.}
\people{(Ольга) И вот сегодня, какое у нас? Как ты ощущаешь? В прошлый раз ты видел нас как серебристо-серое.}
\soul{(Гена) Ну, вот, если так вот сравнивать, то вообще-то здесь… Как вот если сравнить, допустим, с морем, да?}
\people{(Ольга) Угу.}
\soul{(Гена) То - спокойное море. Ну, вот, именно, что вот женский вот, чисто такой мягкий характер. }
\people{(Ольга) Ага.}
\soul{(Гена) Вот. Хотя чувствуются острые пики.}
\people{(Ольга) Ну, это эмоции наверно.}
\soul{(Гена) Ну, в принципе, при желании можно было бы заглянуть, что это за пики, но, честно говоря, у меня нет такого желания.}
\people{(Ольга) Не надо.}
\soul{(Гена)  Это знаешь как?}
\people{(Ольга) Ну, да.}
\soul{(Гена) Типа ``ворошиться в чужом белье''.}
\people{(Ольга) Ещё один. (вопрос.прим..).  }
\people{(Ольга Белимову) Геннадий Степанович, можно такое (…) ?}
\people{(Белимов) (Даёт согласие)}
\people{(Ольга) Гена, вот смотри, в этом состоянии ты поймешь, что я сейчас прочитаю? Смотри! ``Лха, вращающий Четвертое, Слуга Лха Семи, тех, кто вращаются, устремляя свои Колесницы вокруг Владыки своего, Единого Ока нашего Мира. Дыхание Его дало Жизнь Семи. Оно дало Жизнь Первому.'' (Из ``Тайной доктрины'' - [Блаватская Е.П] прим.)}
\soul{(Гена) Это я уже где-то слышал в контактах… ``Семь по семь'', - что-то такое было. }
\people{(Ольга) Угу.}
(пауза - 4 сек.)
\people{(Белимов Ольге) Ну, ты хочешь объяснения этому отрывку?}
\people{(Ольга Белимову) Ну, да.}
\people{(Ольга) Ты не можешь объяснить? Нет, да?}
\soul{(Гена) Ну, как? Ну, колесница – это, вообще-то… это время.}
\people{(Ольга) Угу.}
\soul{(Гена) Это движение, именно вот - время.}
\people{(Ольга) Угу.}
\soul{(Гена) Да, действительно, можно наш мир разделить на… как бы на семь… семь основных что ли … }
\people{(Ольга) Ну, да. Как радуга.}
\soul{(Гена) Ну, да. И, вот каждое, вот это вот, из этих семи, можно разделить ещё на семь.}
\people{(Ольга) И ещё и так далее, наверно очень много можно, да?}
\soul{(Гена) Очень много - это будет уже очень мелко, и поэтому оно не принесёт никакого такого значения.}
\people{(Ольга) Угу.}
\soul{(Гена) В смысле - она всё совокуп… Каждая мелочь отдельно, вообще-то, она просто мелочь, действительно. А вот в совокупности, вот это уже - всё. И вот, в принципе, больше вот этих, получается, как бы 49-ти, в принципе, делить уже не стоит. }
\people{(Ольга) Угу.}
\soul{(Гена) И, как объяснить? Получается, как бы, действительно, каждый человек - это центр вселенной. Это действительно так.}
\people{(Ольга) А, все вместе - человечество?}
\soul{(Гена) Тоже! Дело в том, что… В данном случае как? Человек. Ну вот – я. Так? }
\people{(Ольга) Угу.}
\soul{(Гена) Я же относительно себя рассматриваю Вселенную, значит, получается, я – центр вселенной. }
\people{(Ольга) Да.}
\soul{(Гена) Правильно же? Как говорится, куда мои руки могут достать, значит это всё вроде как ``моё''.}
\people{(Ольга) А вот, скажи, а ты знаешь, что, вот… Что такое человек? Чем отличается от других существ живых? Вот, чем? Вот мы чё-то много-много говорим на эту тему, ну никак собственно вот… Ну, чем-то отличается? С одной стороны он такой же, с другой стороны что-то в нём есть другое. Не другое вернее, а что-то такое, что его отличает. Ну, вот так.}
\soul{(Гена) Бессмертие.}
\people{(Ольга) А животные - смертны?}
\soul{(Гена) Да. Животные, получается, смертны. Это… физическое… Ну, как сказать? }
\people{(Ольга) Ну, вот - он умирает…}
\soul{(Гена) У животных что-то…Ну, в принципе… Животное – это, как бы временная что ли оболочка. Временная оболочка, как одежда что ли. Ну, в принципе - и человек тоже ``одежда''.}
\people{(Ольга) Ну, да.}
\soul{(Гена) Но в нём, в этом человеке что-то живёт такое вот, чувствуется бессмертие какое-то, глубина что ли, я не знаю. И вот, из-за этого вот бессмертия, у нас вот всё время-то и происходит вот эта вот - борьба. }
\people{(Ольга) Ещё один вопрос. А, вот: животные, растения,- ну, то что мы знаем, - минералы, собственно то, что мы видим, знаем и всё, они ведь тоже, наверное, в какой-то… Ну, то есть будут когда-то людьми? А?}
\soul{(Гена) Не-ет.}
\people{(Ольга) Станут?}
\soul{(Гена) Нет!}
\people{(Ольга) Никогда не станут?}
\soul{(Гена) Нет-нет, это ``форма''. Тот же камень, это просто чисто форма. Вот, то, что в камне находится, это да, это может быть человеком.}
\people{(Ольга) Ну, да-да-да. Именно душа.}
\soul{(Гена) А так, - нет. Это камень, просто камень. Это просто материя и этот камень может поменять… ну, как… за свою физическую жизнь, - именно этот камень,- может поменять очень много душ, скажем так.}
\people{(Ольга) А-а…}
\soul{(Гена) Та же самая гора…Ну, вот, допустим…}
 (Ольга) Ну, да-да-да. Любая.
\soul{(Гена) ..Царица горы, да? Ну, она сейчас царица горы одна, а через 1000 лет может быть другая. То есть, просто, как…Ну, вот человек, что человек? Вот, он, как говорится, в него вошла душа, ушла и он умер. Всё! Этой плотью уже никто не пользуется.}
\people{(Ольга) Угу.}
\soul{(Гена) Я не говорю о двойниках, конечно.}
\people{(Ольга) Ну, да.}
\soul{(Гена) А камень, допустим, он может менять хозяев, как говорится.}
\people{(Ольга) И животные?}
\soul{(Гена) Ну, животное - практически нет. Просто здесь чисто-чисто по времени. Камень живёт очень долго, поэтому может просто менять. Вот, цикл, как говорится, если человек… Вот у человека существует цикл, да вот? - Его плоти. Триста шестьдесят восемь. 368 лет может прожить это тело.  }
\people{(Ольга) Ого! А это…}
\soul{(Гена) Это, как говорится, максимум. Четыреста, допустим уже просто незачем. Этому телу уже просто незачем жить те же самые 400 лет. То есть, получается, как бы идёт… приостанавливается рост что ли души, скажем так.}
\people{(Ольга) А-а! Ну, уже да. Поняли.}
\soul{(Гена) И поэтому приходится менять тело. Ну, вот, возьми теперь вот эти вот… отрезок переложи на камень. Камень живёт, допустим 1000 лет, да? Ну, значит, три души он может поменять.}
\people{(Белимов) Скажи, в этом состоянии ты можешь подтвердить или знать твёрдо, что Земля это живой организм?}
\soul{(Гена) Ну, это ``знать''… Я ещё не родился, уже знал об этом. }
\people{(Белимов) Да? И у Земли есть душа, она мыслит?}
\soul{(Гена) Дело в том, что я - часть Земли. Я часть Земли. Я не могу сказать, жива она или мёртво. Если я скажу что оно мёртво, значит и я мёртв. Я - чисто часть Земли. То есть, вот эта вся моя плоть, всё, что я ношу, всё, что дало вот силы, вот это… ребёнок, всё это – Земля. Это кусочек Земли, всего лишь…}
\people{(Белимов) Но Земля мыслит, да?}
\soul{(Гена) …даже не младенец.}
\people{(Ольга) Скажи вот, на Земле всё… }
\soul{(Гена Белимову) Мыслит.}
\people{(Ольга) …Всё, что есть живое, ну, что живёт вот физически, да? Ведь у неё, у Земли есть душа тоже. Это душа как, она…}
\soul{(Гена) Душа Земли - это… ну, как сказать…}
\people{(Татьяна) Совокупность всех душ, что ли?}
\soul{(Гена) Ну… }
\people{(Ольга) Нет?}
\soul{(Гена) Не так…}
\people{(Белимов) Коллективный логос?}
\people{(Татьяна) Ещё есть, как бы ещё одна общая душа, помимо этих, да?}
\soul{(Гена) Ну, понимаете вот, как вот, как сфера, что ли. Скажем так, шар, а в нём множество вот этих шариков. }
 Меньше-меньше-меньше-меньше… И вот, до самого, вот этого вот…Ну, как её? Ну, вот - ``единство'', вот это вот, который уже шарик меньше… ну, внутри уже больше не содержит никаких шариков. Вот этот вот, допустим, вот…
\people{(Ольга) Единство.}
\soul{(Гена) …Скажем так, душа. – Нет! Это вот, допустим, душа человека, да? }
\people{(Ольга) Угу.}
\soul{(Гена) Вот, а душа города – совокупность, получается, живущих людей в этом городе. То есть вот он, ещё один ``шарик''. Сверху, как бы ещё одна сфера, да?}
\people{(Ольга) Ага. И потом… Ну, там, пла… – России, страна.}
\soul{(Гена) А потом: страны, материка.}
\people{(Ольга) Угу.}
\people{(Татьяна) Ну, и так далее.}
\people{(Ольга) Ну, и так далее. И Земли потом, да?}
\soul{(Гена) Получается, семь континентов.}
\people{(Ольга) Да!}
\people{(Белимов) А на деле пять, да?}
\people{(Ольга) А, ``пять'' - это тоже символическое число.}
\soul{(Гена) Ну, как же? - Шесть.}
\people{(Ольга) Шесть!}
\people{(Белимов) Ну, две Южных Америки.}
\people{(Татьяна) А почему семь континентов ты сказал?}
\people{(Белимов) А кто седьмой, как ты думаешь?}
\people{(Ольга) Всё вместе, да?}
\soul{(Гена) Нет! Действительно есть седьмой континент.}
\people{(Татьяна) А где же он?}
\people{(Белимов) Ну, вспоминай! Может быть ты нам чё-то…открытие сделаешь. Не Гиперборея?}
\soul{(Гена) Ну, я понимаю, что его нету на карте, на глобусе, но я ощущаю, что он есть сам по себе. Ну, я просто ощущаю.}
\people{(Белимов) А может это Шамбала?}
\people{(Ольга) А может это будущее? Рождающееся что-то? Ну, не будущее, вернее что-то рождающееся.}
\people{(Белимов) Ну, пускай он думает! Думай-думай! В этом состоянии ты можешь, что-то нам сказать….}
\soul{(Гена) Не знаю, но я вижу их семь.}
\people{(Белимов) Хм…}
\soul{(Гена) Ну, ладно.}
\people{(Ольга) Угу. }
\soul{(Гена) Чё я хотел сказать-то? }
\people{(Ольга) Да-да.}
\soul{(Гена) Вот, получается, вот эти вот шары, но, в принципе, получается, вот эти те же самые семь. }
\people{(Ольга) Пузырьки. Ага.}
\soul{(Гена) Колесница – время. Те же самые вот эти всё - семь тел, то бишь - человек. От человека и до…}
\people{(Татьяна) До Земли.}
\soul{(Гена) …До Земли. Ну, а теперь взять Землю, это, значит, относительно опять же человека. Всё так же берётся. Земля - вот она единственная, как говорится, это неделимая, да? Уже взять…}
\people{(Татьяна) Угу.}
\soul{(Гена) …Снова. Теперь, поверх неё будет что уже? - Наша галактика там, да, Вселенная. Вот и, получается, – ``семь по семь''. А-аа… И вот она родила… Рождено Одно. Это вот, вся… Ну, как? Скажем так, Господь Бог. Именно Господь Бог - это именно всё. }
\people{(Ольга) И во всём?}
\soul{(Гена) Именно всё. Вот, действительно, я ``частица Божья'', как говорится, да? Там 49-ая, скажем так, по счёту, да? А, допустим, галактика там, допустим, 7-ая по счёту. Ну, я просто, утрирую. Ну, вот, в принципе - вот получается, вот как…}
\people{(Ольга) А когда-нибудь человек может стать в будущем? Ну, грубо говоря, да? То есть, он стремится?}
\soul{(Гена) Ну, да. Конечно, он, как бы приобретает, ну, грубо сказать – заряд, конечно. То есть, он обладает той духовной силой, которая… Ну, как? Ну, скажем так, ``любая кухарка может править страной'', да?}
\people{(Ольга) Угу.}
\soul{(Гена) Вот! Получается, просто кухарка - она страной править не может. Правильно? Этой кухарке надо учиться-учиться, чтобы потом уже, как говорится, править страной правильно, а не просто так, как на кухне. Получается, так же. Человек – он должен духовно вырасти так, чтобы он мог управлять уже, как говорится, управлять вселенной, допустим.}
\people{(Ольга) Ну, ясно.}
\soul{(Гена) А вот, в данном случае вот, он сейчас дорос до человека, да? То есть, управляет, вот этой вот… ну, вот именно вот этой плотью человеческой управляет…}
\people{(Ольга) Угу.}
(Конец контакта)
\chapter{07-04-1996г}
 (без начала.)
\soul{…и чем меньше вы будете соблюдать, тем меньше ошибётесь.}
\people{Угу.  То есть, опять же, чем меньше будем соблюдать, да?  Ну, ясно. }
\soul{Но, на словах, вы не знаете, что делаете…}
\people{Ну, сейчас немножко такая ситуация, что мы вроде как подумали, что не будет ничего. Мы тут кинулись начало прослушивать заново вот. Но, оказывается, всё не так плохо, как мы думали. }
\soul{Вы ошибаетесь, и поэтому мы просим вас о начале. }
\people{Так мы,значит, ничего не поняли, да?}
\people{В каком смысле слово ``начало'' вы трактуете? ``Начало'' - вообще все контакты? Но мы уже два года с вами беседуем. }
\people{Три.}
 (сбой)
 1-2-3
\people{В каком смысле ``начало''?}
\soul{Мы просим о начале. А значит, задавать вопросы.}
\people{Ага, ладно. Мы далеко не ушли никуда. Скажите, а вот, ну, всё-таки, в нас изменилось же? Не может же быть так, чтоб, как ``горохом об стенку''? Ну, пусть, хоть в небольших количествах.}
\soul{Мы можем сказать о вас  только в красках. Синий цвет окружённый зелёным. }
\people{Разумная жизнь… Т.е. сознание и разум. Да?}
\soul{Синий цвет, да - это действительно ваш разум, окружён зелёным. А вы должны были бы помнить, что такое ``зелёный''.}
\people{Цвет жизни. Жизнь.}
\people{Хотя, раньше считался, вроде, я слышал, где-то это цвет смерти, наоборот был.  Не так это?}
\soul{Подумайте.}
\people{Не, ну всё зелёное. }
\soul{Что мешает вашему разуму?}
\people{А-а-а… Жизнь мешает, да?}
\soul{Вы не внимательны. Мы вам говорим: зелёный цвет окружает вас.}
\people{Разум.  Ой… синий.}
\soul{И, причём, если бы можно было нарисовать  красками, то ваш разум, это пятно, не имеющее чётких границ. }
\people{Ну, так залазает он во все, так сказать, эти проёмы.}
\soul{Ну, нет. Просто он не имеет границ. И не имеет формы, а значит, несовершенен. }
\people{Ясно. Т.е. совершенному  разуму необходимо иметь границы.Т.е. какой то…}
\soul{Он должен знать, где можно, а где нет, а не слепо тыкаться, куда ему хочется.}
\people{Так, ну, а законы - куда можно, куда нельзя?}
\soul{Законы? Вы выдумали законы. Вот вы им и служите.}
\people{Высший закон - полнейшее беззаконье, что ли? Так сказать, не иметь   никаких законов, действовать по обстановке?}
\soul{Да, вы любите крайности.}
\people{Ну, а как глубже понять, если мы по плоскости ходим? }
\soul{Плоскости?}
\people{По кругу.}
\soul{Вы видите, по чему вы ходите? Вообще, вы видите, что вы ходите? }
\people{Мы стоим? }
\soul{Вам надо ещё научиться стоять.}
\people{Мы даже не стоим. Сидим на месте. Лежим, наверно.}
\soul{“Сидим'', это уже достаточно взрослые вы.}
\people{А мы ещё наверно лежим, да? В пелёнках.}
\soul{Нет у вас  пелёнок. Иначе, это за вами ухаживают и выращивают вас.}
\people{А что вы от нас ожидали вообще-то? Вот, в начале контакта, если это были вы же.}
\soul{Мы ничего не ожидали. Ожидаете вы. Заметьте,- вы нас зовёте. Не мы.}
\people{Да, мы зовём.}
\soul{Можно конечно подглядеть, что вы хотите. Но, зачем?}
\people{Скажите, вот в вашем мире вы нас как-то видите? Вы сказали, что всё синие окружено зелёным.}
\soul{Всё синие.}
\people{Теперь, уже всё синее?}
\soul{Теперь, уже всё синее.}
\people{А раньше?}
\soul{И теперь, только зелёными пятнышками.}
\people{Ну, о чём это говорит? }
\people{Ну, то, что у нас один разум.}
\soul{Разум.}
\people{Т.е. и эмоций даже не существует, да?}
\soul{Почему же? У вас есть эмоции, но сейчас вы стараетесь их подавить.}
\people{Ну, так оно, наверно, и есть.}
\soul{А вы вслушайтесь в себя. Вы боитесь. Вы стараетесь не ошибиться в технике безопасности.}
\people{Да.}
\soul{И, чем больше вы будете стараться соблюдать, тем больше вы сделаете ошибок. Вы же не знаете её. Так и будьте попроще. Будьте естественны, а не играйте. Иначе, будут  очередные маски.}
\people{Скажите, вот для нас так и осталось загадкой: кто такие четверо и что может произойти, если эти четверо объединятся? Вы эту помните ситуацию? Вы можете хоть как-то больше сказать о ней?}
\soul{Мы вам говорим о начале, а иначе, скажем: мы вам ещё не говорили о четверых, ибо мы с вами разговариваем в первый раз.}
\people{А-а-а. Спасибо теперь мы поняли, что вы в первый раз. Тогда, пожалуйста, коль вы в первый раз… Вы какой, энергетический мир, другой мир, или наше подсознание? Как вы себя идентифицируете? Как вы можете объяснить себя?}
\soul{Мы?}
\people{Да.}
\soul{Всего лишь - энергия.}
\people{Энергия… Мыслящая видимо? Да?}
\soul{Вы встречали что-нибудь не мыслящее?}
\people{Ну, мы ещё…}
\soul{Что в вашем понятии ``мысль''?}
\people{Разумное.}
\soul{Давайте скажем так, чтобы вам легче: мы - мысль.}
\people{Даже так, именно мысль?}
\soul{Простите. Что такое энергия? Что в вашем понятии ``энергия''? Хорошо, давайте поближе к реальности. Назовите любой  вид энергии, который вы знаете.}
\people{Механическая.}
\soul{Механическая.}
\people{Кинетическая энергия там.}
\soul{Сила.}
\people{Сила.}
\soul{В первую очередь, это можно назвать силой. Много ли в ней ума?}
\people{Скорей всего, мало.}
\soul{Хорошо. Вы применили силу. Вы так же безумны, как эта сила? }
\people{Ну, смотря как. Если её с умом применить, эту силу, то может быть результат… }
\soul{Ну, хорошо. Вы совершаете, какую-то работу. Что умнее, инструмент, или вы, владеющий инструментом? }
\people{Тот, кто владеет инструментом. Ведь инструмент без рук, будем так говорить…}
\soul{Тогда, давайте представим эволюцию инструмента.}
\people{Ага. Тогда…}
\soul{Тогда вспомните, какой у вас был первый молоток и первый гвоздь.}
\people{Да, деревянные. }
\soul{И что сейчас?}
\people{Да.}
\soul{Вы стали умнее?}
\people{Нет.}
\soul{Вы, забивающий гвоздь, вы стали умнее? }
\people{Вы хотите сказать, что инструмент совершенствуется, а мы остаёмся на месте, что ли?}
\soul{К сожалению, это так. Действительно, это так. Тот, кто совершенствовал инструмент, это не вы. Простите, далеко ли вы ушли от каменного топора, если имели железный? Техника -  та же самая. Человек - тот же самый, разве только, что немножко полысел.}
(Обрыв связи. )
\soul{Эволюция? Да, человек эволюционирует,  раз он создаёт новые инструменты и совершенствует их. Но, пользуются ими, чаще всего, другие.}
\people{Скажите, а вот, ну, получается, в принципе, человек сам по себе инструмент, раз уж он что-то делает, он тоже проявляет силу…}
\soul{Чей-то? Да.}
\people{Чей-то. Да. Пусть чей-то, а, значит, надо что? Надо собственные, так сказать, инструменты…}
\soul{Ну, давайте поговорим вашими словами.  Бог создал животных и создал вас. Что животные, что вы, здесь живёте одинаково и творите одинаково. Но, только вы - немножко совершеннее, чем животные. Вы согласны? И то и другое - инструмент.}
\people{Мы творим, так сказать, на тех планах, которые не видим, да? Так? Т.е. что-то мы тут делаем, а там всё это…}
\soul{А только ли вы это творите?}
\people{Ну, я и говорю: все. Да. Животные, человек, там, растения…Что угодно!  Ну, будем говорить так, (…) правильно вы сказали, что человек, просто, более совершенный инструмент. Т.е. тело человека, да? Само тело? Или его личность - тоже можно сказать?}
\soul{Хорошо, давайте скажем по-другому: не будем говорить о боге, будем говорить конкретно о вас. Вы согласны, что вы были б совершенны, если у вас все части тела работали нормально?}
\people{Конечно.}
\soul{Тогда давайте у вас отнимем, что-нибудь. Вы  уже инвалид?}
\people{Да.}
\soul{Вы уже можете меньше совершить работы?}
\people{Да.}
\soul{Да?}
\people{Хотя…}
\soul{Физически.}
\people{Физически - да.}
\soul{Вы же привели пример, о физическом?}
\people{Верно да. Да, меньше.}
\soul{Меньше.}
\people{Если не научусь пользоваться, тем, что есть в совершенстве.}
\soul{А что у вас есть?}
\people{Ну, допустим, чего какого-то органа не будет, а мы можем без него обходиться. Даже, если половина мозга не бывает, половина, другая, берёт на себя обязанности.}
\soul{Прекрасно. Костыли - подходят?}
\people{Костыли? Подходят.}
\soul{Ну и как, вы будете совершенны с ними?}
\people{Да, нет.}
\soul{Мы говорим о физической работе, не об умственной. Но, чтобы научиться пользоваться костылями, вам придётся применить умственную работу. Вы согласны?}
\people{Согласны.}
\soul{Кем же вы являетесь в данный момент? И чем является ваше тело?}
\people{Костылём для разума, скажем так.}
\soul{Инструмент.}
\people{Инструмент.}
\soul{Тело ваше – инструмент. Оно потеряло своё свойство, и вы хотите его восполнить, и создали  ``костыли''. Здесь вы уже совершили умственную работу, а потом применяете физическую, чтобы научить тело, инструмент, жить с этими костылями.}
\people{А вы не скажете,  зачем?}
\soul{Потому, что вы привыкли совершать физическую работу, и вам трудно уйти от  привычки. Привычка – это вторая натура.}
\people{А человек, в каком направлении будет эволюционировать? Уходить от тела?}
\soul{Вы считаете,что вы живёте только тогда,когда у вас всё есть. Физическое, к сожалению. И если что-то не хватает, вы считаете это уродцем, инвалидом и много других слов. И уже говорите о несовершенстве этого человека. В первую очередь вы говорите `` какой он несовершенный, какой урод'', а уж только потом, ближе познакомившись, и привыкши к уродству, вы уже скажите о его уме. Не так?}
\people{Ну, да. У нас такая пословица есть: ``встречаем по одёжке''.}
\soul{Так что тогда человеком считаете вы? Тело (инструмент), или то, что внутри вас?}
\people{То, что внутри.}
\soul{Разве? Тогда почему же на первого попавшего  инвалида, вы скажете: ``несовершенный'', и пожалеете его, не подумав, о том, что есть у него внутри?}
\people{Комплекс наверно того и другого. Иного, совершенного  человека, мы не можем считать за человека, поскольку, ну, души нет, мысли.  Он просто…}
\soul{Ничто не случайно. Почему же тогда вы выбрали именно физические силы? Почему вы задали вопрос именно об этом? Почему вы не могли бы сказать о другом? Более простое. Вам всё надо попроще.}
\people{Нет, это просто, чтобы поняли многие, потому, что физические силы - они всегда и везде. А допустим, какие-нибудь нематериальные там физики, может кто-то и не понимать. А о том, что удар молотка  производит силу, это знают все. Дети даже. Так что, я начал, действительно, с простого, скажем так. Об энергии любви мы ещё не поговорили…}
\soul{Хорошо. Возьмём первое знакомство ребёнка с огнём. Как вы думаете, что думает ребёнок об огне, впервые увидев его?}
\people{Ну, чё-то привлекательное, думаю. Ну, наверно, даже, может быть, и не думает, а чувствует его. Наверное, скорее…}
\soul{Пусть будет так. Простите, мысли ваши рождаются и от чувств.}
\people{Ну, да. Ну, потом и заинтересуется, что это такое, - наверное,- горит? Подносит руку естественно.}
\soul{И что?}
\people{Обжигается.}
\soul{А зачем он подносит руку? Что толкнуло его поднести туда руку? Просто любопытство?}
\people{Ну, элемент познания. Разум, наверно. Нет, разума тут нет. Нет, ну если разум родился заново в этом мире…}
\soul{Как вы инерционны. И вы уже забыли себя детьми. Первое. Что - огонь? Тепло! А что - мать? Тепло! Так к чему стремиться, ребёнок? К любопытству или к матери?}
\people{К теплу.}
\soul{К теплу. Ассоциация – мать. А теперь, представьте, - ребёнок обжёгся. Что, что будет думать он сейчас?}
\people{Что, он ошибся и надо быть осторожным с огнём? }
\soul{О, нет.}
\people{Для познания.}
\soul{Он будет обижен. У него будет обида, что он почувствовал тепло и ласку и, вместо этого, получил боль. Обида!  А всё остальное -  это уже от взросления.}
\people{Так выходит, я так поняла, что вы намекнули, да? На то, что если мы боимся мы соприкоснуться к своему огню, потому что он может нас обжечь.  И…}
\soul{Простите, сперва, в ребенке больше инстинктов. И не надо говорить о прошлых жизнях, о его знаниях. Прошлой жизни пока у ребёнка нет.}
\people{Нет, мы не об этом. Я именно…}
\soul{Дальше.}
 1-2
\people{Говори. }
\people{Мы хотели бы узнать о вас, поскольку вы впервые к нам вышли, это, всё-таки, мысль? Это - не цивилизация? У вас как создана жизнь? Вы…}
\soul{Простите, вы живёте только благодаря тому, что есть мысль, и существует мысль. И мы же вам говорили, что все ваши тела всего лишь инструмент мысли, и не больше. Если хотите, то вы, сперва  рождаете идею об этом молотке. Рисуете его у себя, лишь только потом, физически, создаёте его. Вот вам -  материализация мысли.}
\people{А как эта мысль существует, в виде информационного поля вокруг земли или в космосе или как? Как вот вы, как вы отдельно без человека существуете?}
\soul{Это вы привыкли, что информация должна иметь своего носителя.}
\people{Так. А это не так? }
\soul{А почему вам не сказать, что информация ищет носителя?}
\people{Да.}
\soul{Почему бы не так? Почему вы решили, что нужно, сперва  носитель, а потом лишь только  информацию, а не наоборот?}
\people{Нет, ну, вообще-то подозревали, что существуют информационные поля.}
\soul{Простите, вы это и делаете, оказывается, вы лишь только подозреваете. Вы создали компьютер. Вы уже знаете, что такое `` информатика'', и, простите, разве не вы создаёте носители, мысль ваша? Или, может быть, носители создали вас? Может быть, носители дают информацию? Или, всё-таки, вы информацию наносите на них? Вы живёте, действуете по этим законам, и, зачем-то, выдумали другие!}
\people{Т.е. мысль, это всё. Тогда, выходит, везде… Она есть везде (…) И она передаётся всегда. Так же, как энергия любви наверно, да?}
 
\soul{Ну, возьмите, приведите аналог, прямой аналог - себя. Зачем вы любите уходить куда-то в дебри, и чем больше высоких слов, тем вам эти дебри кажутся мудрее? Да будьте ближе! Ближе к себе. Приведите аналоги относительно себя. Поживите, побудьте этим молотком, и вот тогда будете знать, как молоток должен ещё больше усовершенствоваться. Представьте, что вы- молоток, вы должны забить гвоздь. Этот молоток не очень доволен этим гвоздём, и молоток в первую очередь думает не о том, как изменить, а как изменить гвоздь, чтоб он полегче вошёл  и поменьше работы. А тот же гвоздь думает о молотке то же самое, только всё наоборот. К сожалению, молоток и этот гвоздь не могут соединиться. Почему?  Потому что есть владелец всего этого, и он, хаосом своих мыслей, просто забивает гвоздь, не задумываясь ни о чём. Давайте скажем ещё так: вам, за рабочий день, надо вбить тысячу гвоздей, и больше ничего.  Просто, тысячу гвоздей. Вы сидите и забиваете их в одну дощечку. Ну, первый день хорошо, такая работа, вы даже похвастаетесь, что ``я почти не устал''. А через неделю вам это надоест. Вы согласны?}
\people{Да.}
\soul{И вы будете уже уставать. Так какая это усталость, физическая или моральная?}
\people{Моральная.}
\soul{Моральная. А теперь представьте, к вам пришла идея  - а что если гвозди забивать в разные места? И будете ходить по квартире вбивать, а вот давай я, здесь вобью гвоздик, давай здесь вобью гвоздик. У вас опять появится интерес. Те же тысячу гвоздей, а работы вы совершили больше, вы уже и ходили, вы согласны?}
\people{Угу.}
\soul{А вот усталости уже не будет. Вам будет уже интересно, правда,не надолго, но всё-таки. Потом, можно придумать ещё что-то с теми же тысячами гвоздями, чтобы снова родился интерес. И вы будете совершать ещё больше работу, но уставать не будете, потому что у вас есть интерес. Так что такое усталость? Это привычка. Привычка и не больше. Простите, если вы поднимаете спичечный коробок тысячу раз, вы устанете? Устанете, устанете. Но, если вы будете делать, разговаривая с кем-то, то это вам войдёт в привычку, и вы будете тысячами движений и делать, не заметив, как и делает ваш переводчик, махая руками. Почему? Потому что он просто не задумывается, не смотрит на то, что он махает ими. Вы согласны?}
\people{Да.}
\soul{Так что тогда устало?  Мысль или мышцы, качающие эту руку? }
\people{Да ничё не устало.}
\soul{Ничего?}
\people{Мысль устала на одном месте держать, так сказать, внимание.}
\soul{Вот именно. Теряется внимание, теряется интерес. И эта мысль устаёт.  Устаёт, потому что не видит поддержки, не видит подпитки. Она теряет энергию. Любое другое изменение отвлечение превратит  эту работу в чисто механическую,  не требующую мысли. Тогда, чисто механически вы уже устаёте гораздо меньше, потому что мысль ведёт другую форму, а именно: ``А зачем я это делаю? Ах, я уже пятьсот раз сделал это!''  Cмотрите, сколько много!''  Уже понятие цифры,  и что это ``много'', и всё - даёт усталость. Рука ваша будет уставать. Простите, когда вы месите тесто, и просто месите и не о чём не думаете, или именно только думаете об этом тесте, как быстро надоедает, не правда ли? }
\people{Угу.}
\soul{А если вы будете месить за разговором, то поставь вам другую кастрюльку - и вы и не заметете.}
\people{Верно, да. Отвлечение внимания. Скажите, а как вы сейчас вошли в переводчика, как вы умеете пользоваться его словесным аппаратам, слуховым?}
\soul{Очень просто.}
\people{Так.}
\soul{Очень просто. Мы его отвлекаем. Как говорили только что.}
\people{Угу.}
\soul{Мы его отвлекаем, и значит мысль, которая управляет его языком  и тормозит его, отвлечена другим. Как ребёнка стараются отвлечь от чего-либо, давая другую игрушку.}
\people{Угу.}
\soul{Пожалуйста.}
\people{Скажите, а цель, какая у вас контакта с нами, вот?}
\soul{Цель? Это ``какая у вас''.}
\people{Ну, мы хотим познать и себя, и вас, и окружающий мир. И мир не всегда продвигался правильно. Мы считаем, что мир многомерен, многонаселён, много цивилизаций разных. Мы ошибаемся? А сегодня был спор, что, скорее всего…}
\soul{Очень. Даже очень вы ошибаетесь. Как вы говорите в ``технологии изучения мира''. Вы хотите изучить нас, вы хотите изучить ещё чего-то…  А себя? Вы хотите себя изучить через других? Простите, вы не можете себя ощутить без зеркала?}
\people{Ну, слабо ощущаем.  Почему? Ощутить,-  как раз в зеркале ощущаем.}
\soul{И что ж вы ощущаете?}
\people{Да  всё.}
\soul{Ну, что всё, что всё ощущаете? Если бы вы всё ощущали, то б не стали нас звать…}
\people{Мы, наверно, вам надоели.}
\soul{…и вы вообще не стали бы ничего делать, а только сидели бы и изучали себя. Впав в ещё одну крайность, так это делают ваши йоги.}
\people{Угу.}
\soul{Простите, они сидят и усиленно создают себе огонь на пупке. И в чём смысл?}
\people{Ну, да, ни в чём. А в чём вообще тогда смысл? Ни в чём.}
\soul{Потому, что вы делаете это чисто механически, и не больше, как это тело, которое сейчас говорит, всё это делает чисто механически, и не больше.}
\people{Ну, вам всё  известно вообще о мироздании, о том, о строении мира, вам всё известно?}
\soul{А вам - нет?}
\people{Ну, расскажите нам. Мы в потёмках. Мы,  действительно в тумане, мы не знаем многого о себе и тем более не знаем, а есть ли многомерность. Может мы не в трёхмерном, или, может, действительно,  люди - это венец природы и мы, единственные  во всём космосе, во всей Вселенной мыслящие существа. Ну, вы должны…}
\soul{Поверьте…}
\people{Так…}
\soul{Вы не какой не ``венец'' и не в ``глупости'' и не в ``умности''.  Таких, как вы, множество.}
\people{Посредственность, короче. Ну, дык, продолжите мысль. Есть другие цивилизации?}
\soul{Почему бы и нет?}
\people{Вы их видели?}
\soul{Что вам  от этих знаний толку?}
\people{Ну, как это -“толку”…}
\soul{Вы говорите о многомерности, но не ощущаете этого. Вы говорите о любви, но не ощущаете этого. Вы можете написать прекрасные статьи о чём-то, даже не видя это. И даже не веря в это. Это что пишет в вас? }
\people{Ну, у меня лично - стремление к познанию, стремление к истине.}
\soul{Познанию?}
\people{Да.}
\soul{Ну, хорошо. Вы накопите эти познания, а что дальше? Просто, ``большая библиотека'' и не больше.}
\people{Ну, почему? Когда-то…}
\soul{Энциклопедический словарь.}
\people{Когда-то думали, что земля плоская, перешли к круглой. Много всё  двинулось вперёд. Теперь…}
\soul{И вы ощущаете её округлость? Скажите, лично вы ощущаете округлость? А если сейчас с детства вас начнём учить, что земля плоская. И, если мы сейчас сможем аргументировано доказать вам, что земля плоская, вы сможете поверить. А  вы ощутите её плоскость? Вы ощутите, что она треугольная, квадратная, круглая? Вы сможете это ощутить?}
\people{Мы… Непосредственно - нельзя, но с помощью приборов.}
\soul{А зачем тогда нужно?}
\people{А что, не надо познавать мир что ли?}
\soul{Не надо быть библиотекой. Надо всё переживать. Всё.}
\people{Ну, вы-то, можете нам как-то раскрыть глаза, если вы такие продуманные? }
\soul{Мы будем за вас переживать?}
\people{Скажите…}
\soul{Тогда, вы будете одержимы.}
\people{Ну, как сотрудничество…}
\soul{Сотрудничество?}
\people{Как более продвинутая цивилизация, или, по крайней мере, мысль.}
\soul{Простите, как я могу сотрудничать с ``книгой''?}
\people{Скажите, пожалуйста, вот, читал я, простите, опять ту же книгу, писанную нашими людьми, может быть фантастика, может что. Ну, там описана ситуация, когда, вот именно, какой-то, ну, скажем так, по-нашему, дух там или мысль, входит в человека. Он преображается, становится совершенно другим, и вот он через него говорит, что есть другие виды жизни, так сказать, ну, пытается что-то людям сказать, там такое. Наверное, это всё реально могло иметь место, но вот для чего это бы делалось бы, если, ну, действительно было?}
\soul{Давайте скажем по-другому.}
\people{Ну, давайте.}
\soul{Что делает артист? Что делает артист? Почему вы говорите этот артист хороший, а этот плохой, почему?}
\people{Ну, вживается в роль, наверно.}
\soul{Вживается в роль.}
\people{В образ.}
\soul{В образ.}
\people{Угу.}
\soul{Он становится именно тем человеком, которого играет.}
\people{Ну, да.}
\soul{Мы же, оскорбляя вас, называя вас ``книгами'', простите за выражение… Хотите просто? Хотим вбить вам в голову, что не будьте просто ходячими ``энциклопедиями''! Вы знаете всё, о чём ни спроси. Вы знаете всё, а толку от этих знаний никаких. Потому, что вы топчитесь на месте. Вы говорите красивые слова, и множество их, пишите множество всего, сами не веруя в это, в то, что написали. Иными словами, любые ваши статьи можно назвать фантастическими потому, что вы ни одну из них не пережили. }
\people{Почему?}
\soul{Почему? Вот мы должны были спросить вас: ``почему же это так?''. Почему вы пишите о высоких чувствах и не верите в них? Почему?}
\people{А кто вам сказал, что я, допустим, что я не верю или все не верят? Или, неужели, нет никого, кто бы… }
\soul{Спрашиваете вы? Мы говорим с вами?}
\people{Да.}
\soul{Придут другие, будем говорить другим. Хорошо, если другое. А может быть, и то же самое. Вы спрашиваете, мы вам отвечаем.}
\people{Хорошо. Т.е. конкретно я никаких высоких чувств никогда не переживал и писал отсебятину, так сказать?}
\soul{И опять вы любите крайность.}
\people{Нет, ну, вашими словами…}
\soul{А как много было написано вами, которыми вы уже сомневаетесь? Как, что это навеяло на вас, что вы написали?  Вы сами, даже, порой, не узнаёте свои строки и не можете вспомнить, как вы могли это написать!}
\people{Это было.}
\soul{Не было?}
\people{Было.}
\soul{Было. А почему вы забыли, как вы это писали? А вы забыли! А раз забыли, то, значит, перестали быть этим. Если вы скажите, что ``Я любил''. Простите, это мы будем  приводить пример  - ``Я любил, теперь не люблю'',-  то это не значит, что вы остались в том состоянии. Это значит, что вы забыли, это значит, что вы потеряли. Это значит, что вы просто где-то украли, или с ошибкой получили, или, просто не уберегли чей-то дар, и растоптали его.  И после этого хотите, чтобы с вами… И, чтобы вас любили? А сколь часто бывает так? Сколь часто к вам приходит вдохновение, и что делаете вы? Написали, посмотрели, поглядели, похвастались, а потом  проходит время, и вы говорите: ``Как же я это написал, как обидно, что я не могу больше это повторить!'' Почему? Забыл? Да не забыл! Растоптали и выкинули.}
\people{А что бы вы посоветовали? Как нам себя вести? Вот допустим, мы идём… }
\people{Вы слышите нас?}
\soul{Говорите: ``зря''. Настоящий мудрец будет раздавать вам мудрость, и будет раздавать её щедро, а уж если вы не возьмёте эту мудрость… Простите, почему тогда он виноват?}
\people{Вы знаете, я тоже слышал такую, про учителей, про наших, школьных, такие статьи были, что говорят,  нету плохих учеников, есть учителя, которые не умеют учить.}
\soul{И то, и другое.}
\people{У нас ощущение такое, что вы тоже с этим сталкиваетесь.}
\soul{И то, и другое.}
\people{Вы уж не примите, конечно, к себе, потому что, действительно… }
\soul{Мы не можем  это принять к себе.}
\people{Мы слишком, наверно…}
\soul{Вы слишком не внимательны. Вы могли б, хотя бы, посмотреть по тону.(манера речи изменилась. Прим. )}
\people{Это - другой. Это другие. Да? }
\people{Да.}
\soul{Вот вам и мудрость ваша.}
\people{А… Это те же, только в другом качестве что ль? Нет, нет. Это была энергия мысли, она более, будем говорить, такая… ну, переводчик её воспринимает, может более…  Ну, тогда , простите, что мы …  А кто вы сейчас? С кем мы имеем дело?}
\soul{Вот, как вы привыкли! Вам нужно конкретно знать имена, с кем говорите. Иначе  очень трудно. А сами часто любите скрывать их.}
\people{Да нет, особо не скрываем. Было в контактах сказано - нельзя называть.}
\soul{Почему же? Вы часто используете это в жизни. Всем ли вы раздаёте адреса?}
\people{Ну, понятно.}
\soul{Очень вы любите гостей?}
\people{Не всегда. Всё правильно. Бывают какие-то…}
\soul{Ну, зачем вы так говорите? Вы говорили, только что, о мудрости, и тут же  теряете её.}
\people{Нет, ну, извините.  Вот, допустим…}
\soul{Ну, зачем вы всё обобщаете сразу, о всех? Ну, есть же те люди, которым вы не хотите дать те адреса? Есть же люди, которым вы никогда не скажете имя? Даже имя, не то, что адрес. }
\people{Угу. Конечно.}
\soul{Есть?}
\people{Есть.}
\soul{Есть. Зачем сразу кидаетесь в крайности и говорите, что…  Зачем? Где же ваша мудрость?}
\people{Скажите…}
\soul{Дальше… А  дальше, всё проще. Вы, уже стали судьями, и вы судите по себе. То, что вам нравится, - пожалуйста, и адрес и телефон, и всё что угодно! Если вам человек не понравился, вы ему ничего не дадите. А если и дадите, то, потом, будет жалеть об этом, и будете скрываться. В этом мудрость? Но, было бы мудрее, гораздо, если бы вы не стали бы делить на людей интересных и нет. На мудрых и нет, на плохих и хороших. Вот была бы где мудрость! А какой учитель может вас этому научить? Какой? Да никакой! Сам бог придёт и не сможет вам этого сделать!  Потому что, чтобы это сделать, он должен войти в вас и стать вами. Нет. Вы должны будете стать им, чтобы стать таким же мудрым, как он. Вы должны войти в учителя, чтоб стать таким же мудрым, как учитель. Понимаете?}
\people{Угу.}
\soul{А что делаете вы? К вам приходит учитель, вы хотите стать мудрым, как он, и, учитель входит в вас. И что? Вы осиливаете  его и говорите, что учитель плохой. Как вам можно дать, как вас можно накормить хлебом не имея его? И как можно вас накормить, тем же хлебом, если вы не возьмёте в рот его, и не проглотите его? Так кто  - хлеб виноват, что вы голодны? Или вы, что не взяли эту пищу?}
\people{Скажите, как вы себя, всё-таки, нам можете объяснить? Вы цивилизация, мысль? Ну, не мысль, видимо, эмоции какие-то, краски, что вы из себя  представляете? И скажите, как вы… Объясните нам, как вы вышли на нас? Как мы выглядим, вот сейчас? Как вы нас видите?}
\people{Да. Каждый видит по-своему, как вы это видите?}
\soul{А мы не видим вас.}
\people{Только слышите?}
\soul{Мы не видим, и не слышим вас. Просто - говорящий  нам пересказывает.}
\people{Проводник наш, пересказывает вам что-то? Переводчик?}
\soul{Да. Вы называете его ``переводчик''. Он просто пересказывает, сами-то мы это не видим.}
\people{Ну, а как вы в земном мире находитесь? Какие ваши функции?}
\soul{А мы там не находимся.}
\people{Так.}
\soul{Мы даже не знаем, где вы находитесь. А зачем нам это надо?}
\people{А вы где находитесь?}
\soul{Мы разговариваем с переводчиком.}
\people{Нет, ну,  у вас… В космосе, во Вселенной, в галактике, другие…}
\soul{О, нет. Мы в переводчике.}
\people{То есть…}
\soul{А где он находится, мы не знаем.}
\people{А можно сказать, что это подсознание переводчика? Или, как вы можете идентифицировать?}
\soul{Мы не знаем. Мы сами не знаем того.}
\people{Живёте там, где не знаете?}
\soul{А мы не знаем и себя. Мы только родились.}
\people{Только – это, как? В данную секунду, или 30 лет назад?}
\soul{Нет. Мы только родились, вы не внимательны.}
\people{А вы, по какому поводу, не скажете, - вы родились?}
\soul{Ну, наверно, переводчик…}
\people{Угу.}
\soul{…создал нас. Ну, а вы, безусловно, помогали, раз он переводит вас.}
\people{Тогда, может быть  дадите оценку переводчику, если вы можете дать, или нам, кому-то, как вы слышите, как…}
\soul{Ну, мы не дадим отдельно. Переводчик переводит. Переводчик даст оценку. Но, это даст переводчик, не мы. А мы не видим, как вы…}
\people{Нам интересно разговаривать с теми, кто может что-то дать нам новое. Дадите? Даёте ли вы нам новое? Мы пока сейчас не видим, не замечаем этого.}
\soul{Ну, мы ещё сами не знаем, что можем дать.}
\people{Ну, в общем, я поняла, если можно так сказать. Т.е. вы, наверно, это все мы, то есть как человечество, которое только родились, и ничего не видят, и ничего не слышат, и ничего не знает и только кто-то им что-то говорит, да?}
\soul{Наверно.}
\people{Вот так. Это, наверно, нам хотели показать нас, да? Какие мы есть, пока ещё только что родившиеся, и можем воспринимать только то, что нам сказали.}
\soul{Мы не будем спорить с вами. Чтобы вы не сказали, для нас это может быть истиной и ложью.}
\people{Вот, ещё это…}
\soul{Мы же только начинающие. Мы только родились. Мы даже не знаем, сколько мы будем жить. Мы даже не знаем, что мы будем делать. Мы только родились. Мы даже не знаем, с какой целью это сделали. Мы только знаем, что мы родились именно сейчас, и что мы слушаем переводчика. Он нам рассказывает - мы слушаем.}
\people{А как вы слушаете, с помощью какого информационного аппарата? Слухом, зрительным, на свете, мысли переводчика? Как вы слушаете? Не можете это объяснить хотя бы это?}
\soul{Мы, просто…}
 (конец).
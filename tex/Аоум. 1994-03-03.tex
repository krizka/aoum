Аоум. глава 11-я 03-03-1994г
Георгий Губин
  
Аоум. глава 11-я
Георгий Губин
 ``Действие даже самого крохотного существа - приводят к изменениям во всей Вселенной''.
                                                                              Никола Тесла
   03.03.94г
\people{Сегодня третье марта. У нас пачка вопросов. Их можно задавать?}
\soul{Найдите ответ, какими чувствами плачет младенец?}
\people{Искренними.}
\soul{Почему?}
\people{Потому, что он других не знает.}
\soul{Тогда, он знает искренние чувства?}
\people{Он знает, наверно, первозданные чувства… Он не знает лжи, он не знает..}
\soul{Искренность  первозданных  чувств? Тогда, вы отбрасываете прошлые жизни.}
\people{А…}
\soul{Хорошо. Пусть будет чувствами. Тогда, что рождает чувства?}
\people{Что рождает чувства?}
\soul{Младенец не обладает разумом. Вы согласны?}
\people{Согласен.}
\soul{Тогда, что рождает чувства?}
\people{Окружающий мир. }
\people{Может быть, душа. Может быть, воспоминания прежней жизни.}
\soul{Нет. С первым криком, с первым плачем, он забывает всё. Он помнит только в утробе.}
\people{Помнит в утробе?}
\people{Это какую вы тему  задаёте сегодняшнего разговора? Мы, как раз, только узнали, что чувствует младенец с первых дней своей жизни.}
\soul{Мы просим вас ответить.}
\people{Что?}
\soul{Что рождает крик?}
\people{Ну, чем? Боль… Это боль… Может быть, от страха..}
\soul{Ребенок не обладает разумом. Разумом, который приобретает в жизни.  Сделайте шаг, подумайте, какой разум сейчас  у младенца?   Инстинкт! }
\people{Вас разве удивляет “условные”  инстинкты называть? Для нас это простые понятия.}
\soul{Оно плохо для вас. В вашем понятии, инстинкт, – это плохо. Чаще, в инстинктах, вы откровенны.}
\people{Согласен. }
\people{Человечество надо привести, как говориться, к первозданным инстинктам?}
\soul{Теперь, подумайте, чем отличается младенец от животного? Оба обладают инстинктом.}
\people{Да.  }
\people{Конечно, они  отличаются.}
\soul{Чем? }
\people{Ну, видимо, душой всё-таки. Наличием души.}
\soul{Мы говорили вам, что нет смерти, и всё - живое. Или вы можете представить живое без души?}
\people{Наверно, менее развитой. В человеческих воплощениях.}
\soul{Мы когда-то вам говорили, что есть души развитые, более или менее?}
\people{Мне показалось, что душа-то, как раз, имеет развитие…}
\soul{У вас есть зерно разума. И вы можете его выращивать.}
\people{А животные?}
\soul{Животные –  имеют. Но, вы умудряетесь победить его. И оно (зерно разума.Прим.) становится  инстинктом,  уже полученным.  Полученным  вами.}
\people{Так… Ну, и что ж мы… (должны делать? Прим.)}
\people{Что вы советуете нам  предпринимать, человеку, людям, в этой связи?}
\soul{Будьте  искренними.}
\people{Во всём и всегда?}
\soul{Мы вам говорили когда-нибудь о Лжи? Мы вам говорили когда-нибудь о Правде? Поймите, нам приходиться многое умалкивать. И если мы вам скажем: Добро должно быть с кулаками, - вы не правильно поймёте. Если мы вам скажем: - Вы должны быть ближе к инстинктам, - вы поймёте опять не правильно, ибо понятие “инстинкт”  у вас искажённое. Далее. Если  мы скажем: - “Ложь” и “Правда”… И скажем, что этого не существует, вы опять не поймёте. “Лжи”  и “Правды” нет. Согласитесь, если вы спрашиваете: - Верна ли теория такая-то и такая-то…  Для изобретателя теории -  она верна. Но мы не сможем сказать вам, верна ли она Вам. Ибо, вы говорите о её основании.  Если вы хотите спросить, спросите , ЧТО вы не поняли, и, КАК поняли. И мы вам скажем.}
\people{Вот, мы наверно, по такому пути и хотим сейчас вести разговор. У нас сейчас споры идут, стоит ли продолжать контакты, что они  малоинформативны.  Мало дали, по крайней мере, нам, а, скорее всего, и вам. Но, другой концепции нет.  Хочется расспрашивать, то, что нас, и, собственно, вас интересует. И мне не хочется продолжить дебаты. Какую точку зрения вы поддерживаете?}
\soul{Мы скажем так… Результат мы почуем  больше, чем вы. Мы говорили вам, ваши вопросы – это ваши направления. Далее.  В ваших вопросах – эмоции. И мы говорили вам, что мы находимся в мире эмоций.}
\people{То есть, нам следует продолжать, даже если мы идём наощупь, не совсем понимая конечный результат наших ответов и вопросов в этой дискуссии? Следует продолжать?}
\soul{Вы  должны идти всегда. Далее. Если мы будем считать ненужным,  мы не будем говорить  с вами.}
\people{Вот это, если важно, особенно нашему Гере…}
\soul{Вы идёте, и, поймите правильно, - вы перестанете слышать нас. Мы же говорили вам,- мы вас не покидаем. И мы для вас – вечный шёпот. Подскажем, - в понятии, что вы перестанете нас  слышать. И в это  будет ваша вина.}
\people{Да. А скажите, кроме вашего шёпота, может,  чей-то ещё шёпот слышен был?}
\people{Который нас уговаривает: - Не надо контакты вести.}
\soul{Их множество. Вы забываете о борьбе.}
\people{Так… Минуточку! Такой вопрос…  Мы сейчас подошли, по-моему, к теме кривого единства… Скажите, раз идёт борьба… А Мир – един…  Интересно… Что ж… Мир сам с собой борется?}
\soul{Вот видите, как вы принимаете?}
\people{Ну, примитивно, но всё-таки…}
\soul{А теперь,  давайте приведём пример…  Ну, да,  вы люди. Согласитесь, что вам не скучно. Все вы – ЛЮДИ.}
\people{Да, верно.}
\soul{Почему ж вы боретесь с собой?}
\people{Разница взглядов… }
\soul{Разница взглядов? Много ли у вас было в разнице взглядов? }
\people{Ну, это жажда власти. Меркантильные интересы, конечно.}
\soul{Почему же тогда вы говорите о единстве миров, и свой мир отделяете? Если вы воюете, у вас есть, в вашем понятии, меркантильность и далее, почему этого не должно быть в других мирах? Вы говорите о единстве и, тут же, отделяете.  Спрашивайте.}
\people{Ну, наверно, для вас интересно наблюдать нашу  дискуссию по поводу полезности и бесполезности наших контактов. Вы уже наверно давно наслышаны об этом , да? И видите наши мучения.}
\soul{Нет. Спрашивайте.}
\people{Так… Тогда можно продолжать вопросы?}
\soul{Спрашивайте.}
\people{Слушайте. Является ли наш мозг с его организацией, клеточным строением, его химическим составом и другими  там взаимодействиями, в виде клеток,  нейронов между собой, вместилищем  разумной  информационно-полевой субстанции?  Если да, то в какой из жизней развития зародыша происходит проникновение этой разумной  полевой  структуры из единого поля  природы  Вселенной?}
\soul{Давайте скажем так… Мозг, всего лишь, физическое. А вы говорите о нём, как о духовном. В этом, уже, есть ваша ошибка. Далее. Мы говорили, жизнь есть во всех проявлениях. И даже в камне. Видите ли вы в камне мозг? }
\people{Нет.}
\soul{Тогда, почему спрашиваете?}
\people{А действительно ли, что человек  без мозга  может существовать  и,  даже выполнять довольно сложные действия?}
\soul{Может ли существовать камень?}
\people{Ну, он существует. Но слишком не равнозначно сравнение – камень и человек. Я думаю, что куда сложнее – человеческое.}
\soul{Вы думаете? Какими мерами вы меряете сложности? Вы меряете это физически: камень проще, - вы, сложнее. Придёт время,  и вы поймёте, что,  даже физически,  вы равны. }
\people{Ну, можете ли вы нам предоставить иную формулу  процессов, реакций, правил, на уровне атомов, молекул, белково-нуклеиновых соединений, ДНК, электромагнитных и химических взаимодействий, - ” религии мозга”?}
\people{Вам сложно понять?}
\soul{Нет. Сложно понять ВАМ.}
\people{Да уж…}
\soul{Далее. Вдумайтесь. Вдумайтесь в вопрос, что вам дали. И уже там вы сможете найти ответ. Далее. Вы говорили о ДНК… Как вы понимаете это? Что для вас ДНК? }
\people{Ну, ДНК это зашифрованная информация для развития, допустим, того же человека или животного.}
\soul{Зашифрованная информация?}
\people{Да. То есть…}
\people{Разные молекулы ДНК рождают разные организмы. }
\people{Да.}
\soul{Вы опять говорите физически. В вашем понятии,  информация – физический носитель.}
\people{Вы говорили о “ключе”.  То есть, это “ключ” который позволяет  открывать…  Если, вот, в вашем понятии, и в нашем понятии, виртуалы. То есть, создавая определённый “ключ”, или ключ животного, или ключ человека, мы, тем самым… То есть, разные виртуалы могут вселяться в человека.  Есть ли, вот,  разница между ключём животного и ключём человека?}
\soul{Тогда объясните мне, какой смысл этих контактов? Если в вас “ключ” – то-то и то-то. Тогда у вас нет права выбора. }
\people{Ну, ключи… Я так понял, что… Вот, кто…  Когда виртуалы переселяются, допустим, в …}
\soul{Нет! Физическое – воздействует только на физическое!  Подумайте. Если физически – эмоциональная команда – та-то и та –то. Более – вы тогда не сделаете, в вашем понятии. Неужели вы так думаете, что мир так не разумен,что подчиняется физике? Физическое – рождает физическое. В вашем понятии, ДНК создан только со стороны тела. Давайте не будем говорить о виртуалах в этом случае.}
\people{Скажите… Такой вопрос… Если мы едины, ну, камень, трава, - всё это Единство…}
\soul{Нет. }
\people{Нет?}
\soul{Вы опять воспринимаете Единство … И говорите;  “камень”, “трава”, “человек”. И тут же, разделили.}
\people{Верно.}
\soul{Далее. Мы не говорили вам, что Вы – камень.}
\people{У меня вопрос другого порядка. Скажите, а зачем надо это многообразие “форм материальной жизни”, так сказать? Физической жизни…}
\soul{Ну, хорошо. Если не будет разнообразия, и будет одна дорога – будете ли вы идти по ней?  И что вы приобретёте на ней?  Если вам дать одну букву алфавита – много ли вы прочтёте?}
\people{Можете ли вы предоставить понимание о процессах управления нашим организмом и эмоциями, для успешного исправления ошибок природы, хромосомного, скажем, характера? Если вы владеете этими знаниями, то не хотели бы нам помочь в деле исправления этих ошибок?}
\soul{Мы не владеем знаниями об “ошибках” хромосом. В вашем понятии, есть “ошибки”.Вы себя ставите выше законов. Законов природы!  Вы “не ошибаетесь”,  ошиблась природа!  Почему же?}
\people{То есть, на самом деле это не так?}
\soul{Это ваши ошибки! Ваши ошибки, и природа повторяет их.  Мы говорили вам, вы в единстве. Но, если ошиблись вы, - природе придётся  “ошибиться ” и с вами.}
\people{Наверно, вы ещё и не нуждаетесь в знании хромосом ,саморегулярных  так как вы их не имеете…}
\soul{“Ещё” или ”Уже”?}
\people{А? }
\people{Ещё или Уже?}
\people{Уже, скорее всего. Да. Поскольку  вы не имеете тела, вам и не нужны эти знания. Так?}
\soul{Тогда, зачем бы мы говорили с вами? Мы имели тело, и имеем его, но не в вашем  понятии. Поймите, вы, физические, и слишком много уделяете этому времени. Мы – имеем тело, но… Вы говорите,- “виртуалы”… Виртуалы, это тоже – физика. Есть гораздо выше!}
\people{Скажите, вот если вы находитесь на стадии изучения нас…}
\soul{Нет! Никогда не говорите, что мы изучаем вас, вы изучаете нас!}
\people{Так… У меня вопрос такой… В одном из контактов вы буквально повторили это – “Мы изучаем вас, вы изучаете нас”. Это для нашего уровня… Ну, так сказать…}
\soul{А как мы сможем вам объяснить Единство? Что, вы – есть и мы, и мы – есть и вы, }
\people{Ну, верно, верно…}
\soul{Вы начинаете говорить о чём-то физическом, вы говорите о каких-то “полях”. И забываете,  “поля” – это тоже “физика”. Мы же вам всегда говорили - духовное! Далее. Понятие “физики” – у нас не существует.}
\people{Понятно. Другие критерии.  Да?  Допустим, информационные сейчас открыли поля, которые физическому воздействию не поддаются, но они влияют на физическое.}
\soul{Поймите! Любая информация должна иметь носитель! Носитель – всегда будет физическим, какой бы ни была тонкой материя. Мы говорили вам о памяти. И говорили ,  идеально – не иметь её совсем. О какой “информации” тогда можно говорить, если в вашем  понятии “информация” это переработка памяти?}
\people{Тогда получается “Абсолют”? Самопознающая система? Т.е. самопознавшаяся.}
\soul{Если б мы были уже этой системой, мы бы не говорили с вами.}
\people{Угу…}
\soul{Далее. Мы говорили вам отличия – “вы – не помните. Мы – помним всё, что было”.}
\people{Да.}
\soul{Вы помните? (о том разговоре.Прим.)}
\people{Да. Ещё сказали, что счастья в этом нет.}
\soul{Нет в этом счастья… (с печалью)  И вы – одно из наших воспоминаний… вы любите вспоминать?}
\people{Приятное – любим.}
\soul{Спрашивайте.}
\people{Кстати, о памяти. Вы говорили, что у вас нет прошлого и будущего, вы живёте в настоящем. Не этим ли объясняются  беседы, что вы беседуете с нами во многих мирах… То есть, не имеется ли ввиду наше воплощение в разные периоды времени, когда разные… тот же Белимов, мог быть и монахом и учёным, и писать труды и не писать , и так далее?  Так у вас воспринимается или мы не правильно поняли? }
\soul{Начало вопросов… Нет.  Вы говорите:  Верно ли, что живёте в настоящем и знаете и прошлое и будущее? И здесь вы сумели разделить. Далее. Мы говорили вам о множестве “ветвей” и говорили о “стволе”. Вы создаёте множество “ ветвей”. Сейчас вы сидите здесь благодаря тому, что прошлое мгновение привело к этому. Но прошлое мгновение могло привести и к другому. Вы могли бы встать и уйти. Вы согласны?}
\people{Да.}
\soul{Это и есть “ветвь”. Каждое мгновение вы создаёте “ ветвь”. Каждое мгновение, вы “другой”. Каждое мгновение – вас множество. Лично Вас – множество. В Ваших понятиях, и в других понятиях. Каждое мгновение вы меняетесь. Поймите это. Мы говорим о вас как о множестве, ибо вас – множество. Каждая клетка отдельно – можно назвать вашей фамилией. Далее. Есть множество миров,  где есть вы, но немного измененный. Достаточно изменить  одну молекулу и вы уже другой. И это будет совершенно другой мир! Достаточно на вашей Земле, даже в вашей Вселенной, изменить направление движения одной молекулы, и она будет другой. И это тоже – ветвь. Мы же говорим вам о “ стволе”. }
\people{Под понятием “ствола” , что вы подразумеваете?}
\soul{Когда вы не будете рождать множество не нужного. Когда вы будете видеть цель и идти только к ней. Сейчас же вы шарахаетесь из стороны в сторону. Даже ваше физическое не знает покоя.}
\people{Скажите, на ваш взгляд, какие для человека необходимые условия существования для того, что бы он был идеальным, с точки зрения  соответствия планов; кармического, функционального, психического образа, образцом или эталоном, который был задуман  Природой или Творцом?}
\soul{Всего лишь только одно… Всего лишь только есть одно условие.}
\people{Да…}
\soul{Верить себе! Прежде всего – себе! Верить только себе! Своему истинному “я”. И, тогда, мы не будем нужны вам. }
\people{Нам далеко до этой веры?}
\soul{Мы говорили вам о начале “Пути”. Но, не унижайте себя. Согласитесь, что если человек впервые попал в рай, - а это гораздо выше, чем вы находитесь сейчас,- он тоже в начале” Пути”. Но “Пути рая”.}
\people{Верно…}
\people{Скажите. Есть предположение, что человек, всё-таки, не абориген Земли. }
\soul{Нет.}
\people{Поскольку у него устроен так вот скелет, что … Есть ощущение, что он…}
\soul{Нет. Мы вам ответили. Будьте внимательны.}
\people{Не абориген. Хорошо. Ладно. Поскольку человек гибок…}
\soul{Вы говорите о человеке или о сущности? Вы говорите физически? Физически – нет. Вас достаточно легко умертвить. И вы это делаете с  успехом. Ваша сущность – бессмертна. И , согласитесь, если она бессмертна, разве можно спрашивать “гибка ли она”?}
\people{Ну, мы ведём себя, подчас, как слон в посудной лавке.}
\soul{Нет. Вы больше похожи на детей.}
\people{А… Ну, да. Скажите, где предел, за которым стоит разрушение человека, как личности,  осознающей понимание  своего места в жизни, удовлетворяющей своё любопытство в познании мира, природы,  ко благу цивилизации?}
\soul{Что вы подразумеваете “ко благу”? Ваше “благо” -  это ваша гибель. Чем больше вы создаёте для себя “благ”, тем вы становитесь слабее.  Уберите сейчас  множество “блага”, созданного вами, и что будет с вами? Мы говорили вам, вы – рабы техники.}
\people{Но вы и говорили, что мы должны научиться управлять ею, а не стать рабами… Быть хозяевами, так сказать.}
\soul{Да, мы говорили “научиться”, но не быть рабами. Чаще, вы, всего лишь рабы.  Мы приведём вам доказательство, и приведём его сейчас.}
\people{Не надо.}
\soul{Сумеете ли вы выключить магнитофон и разговаривать? И всё запомнить.}
\people{Нет.}
\soul{Так кто же вы тогда? И в чём ваша информация тогда? Вы говорите – информация, это что-то духовное. И сами создали физический носитель. И вы, уже слушаете не себя, а слушаете технику.  В данном случае, вы будете слушать магнитофон. Много ли это дало вам? }
\people{Мы ничего не хотим упустить из нашего разговора, поскольку память наша, действительно, не совершенна.}
\soul{Тогда, как вы можете говорить, что вы умеете управлять, а не управляют вами? Вами управляет техника. Вы хотите вашу живую память заменить на искусственную.  Спрашивайте.}
\people{Обладаете ли вы способностями нашей цивилизации выражать своё понимание  Природы, Творца, через  искусство, картины, скульптуры, фильмы, книги и другие проявления разума? КакиеКакие у вас заменяют…}
\soul{Да… У нас есть и книги и техника. Но… У вас – физически, у нас – другой план.}
\people{Вы создаёте книги?}
\soul{Далее… как вы понимаете понятие книг? Вы  видите только – бумага? Скоро вы будете называть, и уже называете, микроплёнки – книгами. (1994год) Далее, вы научитесь писать на кристаллах. И тоже будете называть их “книгами”. А теперь, подумайте! Сумейте продвинуться дальше! Какие книги были ранее, и какие сейчас? И какие будут? Шагните вперёд! Шагните, сумейте мечтать! Сумейте фантазировать! Если вы даже сейчас пытаетесь записать на кристаллах  называете это книгами, почему же мы должны писать, в вашем понятии, вашими книгами?}
\people{Скажите, мне немного понятнее становиться ваш мир. Значит, среди нас есть творцы, которых вы уважаете, которых вы…}
\soul{Мы говорили о Единстве, мы говорили об одинаковости. Если б мы были совершенны, не похожие на вас, мы бы не сумели прийти  к вам. Но мы и не похожи. Потому и защита мозга.}
\people{Скажите, а зачем вам книги, если у вас нет памяти?}
\soul{Книги? Книги – для Вас! Для подобных.}
\people{Но, мы их не можем прочитать.}
\soul{Разве?}
\people{То есть, вы сейчас нам можете продиктовать нам, какой-нибудь…}
\soul{Мы спрашивали у вас о младенце. Вспомните…  Далее. Вы говорите “реинкарнация”. Говорите – вы… Почему же тогда вы не сможете сделать аналог… Книга пишет  о прошлом. Реинкарнация говорит о прошлом… Вы сами ответите?}
\people{Получается, мы – листки вашей книги, так сказать ?}
\soul{Спрашивайте…}
\people{Скажите… “Книги” – это по понятиям… }
\soul{Вы и есть “Книга”!}
\people{…это “виртуалы”?}
\soul{Нет! Мы говорили вам, “виртуал” – это физика. Вы говорите о высших материях. Вы говорите о тонких полях. Согласитесь, понятие поля ,  это уже физика. Даже мечтая, вы не можете перейти на духовное. Вы хотите сравнить с физикой. Даже, в вашем понятии, энергетические точки, вы говорите “физика”. Если физика, то почему же ваш лотос над головой вы не могли сфотографировать, или как-то воспринять по-иному? Объясните это. }
\people{Но, наше зрение не позволяет видеть этого физически. Мы должны видеть, в нашем понятии, “духовно”. }
\soul{Поймите. В вашем понятии, лотос, имеющий основные  тысячу “лепестков”, которые находятся в постоянном движении и не двигаются… Не мыслями, и не мысли их двигают… Как вам объяснить то Единство, если каждый лепесток отвечает за вашу мысль и, в то же время, каждая мысль двигает этими лепестками? И вы не сможете понять это Единство. Далее… И вы не сможете увидеть это физически. Ибо, НЕТ физики в тех лепестках.}
\people{То есть, лотос – это не носитель?}
\soul{В вашем понятии, пусть будет “ носитель”, если мы вам скажем – не существует его физически. И вы не сможете понять другого и принять другое понятие. И, тогда мы посеем в вас сомнение и вы будете уже сомневаться , что, “ а есть ли он вообще?”  Пусть будет – “ физически”.}
\people{Скажите, а вот этот “ носитель” в нашем понятии, он, так сказать…э-э-э … Его можно разрушить или он поддерживается… Как бы “ самообеспечивается” без разрушения?}
\soul{А теперь подумайте…  Мы говорили вам, что он не физический. Как вы можете разрушить   “ нефизическое”?  Объясните мне пожалуйста, на какие составные части вы хотите его разрушить?}
\people{Тогда, надо вернуться к Началу Начал, к Силам.}
\soul{Вспомните. Мы говорили вам о Иерархии!}
\people{Ну, да. Вначале было, действительно, Сила,. Потом  про-Материя, Материя…}
\people{Так вот, Сила… Она делима или не делима?}
\soul{Вы не должны вернуться к той Силе. Вы должны прийти к ней, прийти , с новым “ грузом”. Вы же САМИ пришли сюда ( в физический план. Прим.) и  сказали: - Я хочу многое “Новое”! Я хочу воспитать себя! И создали себе тело. А теперь, испугавшись, вы кричите: - Не могу! Не хочу! И, даже, не верите в самого себя… Когда мы говорим вам “ Поверьте себе” , мы говорим вам о тех силах. Мы не говорим о ваших клетках (биологич.) ,. Мы не говорим о вашем мозге. Вы пришли и забыли, что приходили. Вы родили Силу и забыли о ней. Что ж вы? Пришли искать  и забыли, что ищите…}
\people{Ну, такова, наверно, так сказать “дань”, что мы приобрели тело, забыв о том, что мы создали Силу.}
\soul{Дань?!}
\people{Ну-у, видимо существует и Доброе и Злое . Вот это вот, перераспределение…}
\soul{Мы говорили вам о Единстве…  И мы говорили вам о Силе. Сила – не имеет понятия “Добра”и”Зла”. Вы, получив материю, вспомните иерархию,- только тогда начали говорить , есть Доброе, есть  Злое. Назовите мне какой-нибудь предмет, который был бы только “злым” или только “добрым “. Назовите, и я тогда скажу вам – Лгу!}
\people{Ну, видимо, те силы, которые губят нашу душу… Это - “Зло”.}
\soul{Да? А как вы понимаете “губить вашу душу”?  Если вы не знаете о ней ничего и вы её чувствуете, что “ губят”? Скажите мне. Как вы можете это объяснить?}
\people{Мы пытаемся  это чувствовать,  через  собственное..}
\soul{Вы пытаетесь кого-то обвинить, но только не себя. И вы говорите: - Злое губит меня! А что оно губит в вас? Душу, которую вы не можете увидеть?  А как же вы тогда увидели, что она губится?}
\people{Ну, через какие-то следствия.}
\soul{Какие следствия? Болезни? А тогда подумайте, как вы живёте, как вы зарабатываете болезни. И что, это пришли “ чёрные силы” и заразили вас, или вы пришли сами и получили это? И, как вы можете говорить о разрушении, не видя того? Это ваши “фантазии”. Далее… Есть борьба “Добра” и ”Зла”. Но, только не в вашем понятии. И как вы можете знать, делают вам “Добро” или ”Зло”? Если вы – отец и у вас есть дочь, и вы бьёте её за проступок – это “Добро” или ”Зло”?}
\people{Это “Зло”!}
\soul{А вы подумайте! }
\people{Нет. Это бывает с “Добром”.  Это,чтоб она не делала. Кстати, у тебя есть дочь? ( обращается к предыдущему спрашивающему. Прим.)}
\people{У меня две!}
\people{Вот видишь, тебя уже знают!}
\people{Ну-у… Я, так сказать, этому не удивляюсь.}
\people{Можно продолжать? (обращается к  оппоненту ) Как вы выражаете свои…}
\soul{Вы хотите продолжить?}
\people{Так…}
\soul{А что поняли вы из сказанного? Вы будете слушать не нас, а записи…}
\people{Вы правы…(вздыхает.прим) Нам легче вести разговор, когда там есть канва какая-нибудь…}
\soul{Мы говорили вам, мы – будем играть в ваши игры. Мы будем говорить вашим языком.  Мы будим вас. И будем делать, как вы хотите.  Ибо,  вы – дети, и нам приходится поддакивать вам…}
\people{Ну, а если вы не будете… Ну… мы вас попросим не поддакивать… Что мы… Как это будет выглядеть? }
\soul{"Ничто''.}
\people{Ну, слышать самого себя мы будем, да?}
\soul{Вас просто не будет. Поймите… Не вы владеете Миром, - Мир владеет вами.}
\people{Естественно.}
\soul{И Мир, к сожалению, порой играет с вами … И делает вам много пощады.}
\people{Да-а… Это уж точно…}
\people{Лично нам трудно поддерживать дискуссию, так сказать в “свободном полёте”. Мы пытаемся заранее подготовиться, копить какие-то вопросы… И не сразу не всегда умеем реагировать… Ну-у… Может быть научимся.}
\soul{Поймите, мы не хотим вас обидеть, мы не хотим вас унизить… Мы когда-нибудь говорили, что вы “ниже” нас? }
\people{Нет. Никогда.}
\soul{Но, мы никогда и не говорили, что мы “выше” вас. Вспомните. Вы же говорите “ высшие”, “низшие”, не подобные вам… почему? Почему вы унижаете или возвышаете себя? Почему вы не хотите увидеть себя, как есть? Почему вы слушаете нас и верите нам, но не верите себе?! Почему вы задаёте вопросы, на которые знаете ответы? И вы примете только те ответы, которые ближе вам. Другие же, вызовут в вас сомнения,  и вы будете говорить: - Я здесь не верю, здесь может быть ложно. Мы согласны и делать и это. Мы пришли к вам – сомненьями.}
\people{Кстати, мы всё с большим доверием относимся к вам, как к информаторам нашим.}
\soul{Это плохо…}
\people{Может быть подмена незаметная, да?}
\soul{Нет. Мы не говорим о подмене.  Мы говорили вам о фанатизме. Запомните! Фанат- не умеет думать! Мы хотим через сомнения заставить вас думать! Вы говорите о логике – и нелогичны. А, иногда, бываете слишком логичны… Когда вам говорят – просто ЛЮБИТЕ, вы начинаете рассуждать логически. И, когда вас просят рассуждать логично, вы почему-то не умеете делать это. }
\people{Да…  Ну, всё равно, для нас эти сеансы, своего рода учения. Это уроки учёбы и, может быть, надежды. }
\soul{Далее. Вы говорили о безопасности контактов. Мы подскажем вам способ быстрого вывода. Но, не злоупотребляйте им. Вы должны раскрытой ладонью провести от лба до подбородка переводчика. И тогда, практически  быстро он сможет прийти в себя. Но, пожалуйста, делайте это только в крайних случаях.}
\people{Хорошо. Спасибо большое за такой совет. }
\people{Скажите, а  это каждый может делать или определённый человек?}
\soul{Ложитесь.}
\people{Нет, если провести по …}
\soul{Любой.}
\people{То есть, моё присутствие не столь важно уже?}
\soul{Н-да… К сожалению, у вас не развит… Мы говорили о вас. О вашей невнимательности. Вы не можете дать вовремя счёт, ибо, говорите “не  замечаете”. Вспомните, он пытается поднять шар. И вы должны здесь помочь, - дать счёт.}
\people{А-а-а… Я думал…}
\soul{Вы же, не обращаете внимание на то. Вы должны быть внимательны во всём. Поймите, мы не хотим принести вам вреда, и любая помеха может принести вам непоправимое.}
\people{Хорошо. Мы перестали умываться после сеансов. Это тоже может навредить нам?}
\soul{Умывайтесь только тогда,  когда мы говорим вам. Хотя…  Можете делать это постоянно. Это не будет мешать  вам. Но, если вам сказано, - сделайте.}
\people{Угу…}
\people{У нас много вопросов. Может быть они не все хорошие, но… Как вы выражаете…}
\soul{Для нас нет понятий “хороший” ,”плохой”. Они только для вас. Задав “плохой” вопрос, вы потеряете “хороший”. И это отразиться только на вас.  Далее… Вы говорите “ нет ничего нового”. Тогда, вы противоречите себе .  Здесь вы не логичны. Вы говорите “нет нового” и приходите и разговариваете с нами. Что же вы делаете? Прочитываете “ старую книгу”?}
\people{Нет. Это мы в спорах так говорим. Одним кажется, что нет ничего нового…}
\soul{Нет. Вы прочитываете “ старую книгу”. Да, вы знаете всё. Всё, что вы говорили, всё что вы слышали – вы знаете сами.}
\people{Обладаете ли вы знаниями о энергообеспечении  организма человека  без уничтожения животного и растительного мира?}
\soul{Нет, здесь вы не сможете не сделать этого . Мы говорили вам, физическое не сможет питаться духовно.}
\people{То есть, стремиться к вашему образу жизни нам не пристало?}
\soul{Вы не стремитесь к нашему образу. Если мы говорим с вами, то, значит, вы уже придёте. Если я вам скажу, что с вами разговариваете вы же сами, вы поймёте меня?}
\people{Не-ет… Пока, не очень… Но, у нас есть гипотезы и, даже попытки, ограничить питание,  перейти на  вегетарианское питание, не питаться целыми несколькими днями в неделю. Это правильно?}
\soul{Пожалуйста, делайте это. Делайте.}
\people{Продуктивно ли это?}
\soul{Мы вам скажем… Это продуктивно только в том случае, если вы верите. Согласитесь, вы создавали лагеря и “ практиковали” там голодание. Много ли это принесло пользы?}
\people{М-да… Верно…  Так выходит, что идея автотрофности человечества, она не верна? Говорят, что только на путях автотрофности человечество  может выжить и дальше развиваться, как космическое сообщество. Это не верно?}
\soul{Как вы можете представить? Вы перестали есть, вы перестали пить…}
\people{Да. Но-о-о…}
\soul{В вашем понятии, космическое общество – общество скелетов?}
\people{Но Вернадский, Фёдоров… Они именно на этой точке зрения стояли.}
\soul{Нет… будьте внимательны. }
\people{Так-так…}
\soul{Будьте вы внимательны! Если вы сейчас – физические,  ну, как же вас накормить  духовно, если вы не воспримите?  Вы даже не можете понять что такое духовное, а  уже хотите испробовать это! Как же вы это получите, если вы даже не знаете, где это найти?}
\people{Но, мы имеем в виду  другие виды энергии. Может быть мы будем питаться какими-то таблетками, которые нас  энергетически будут поддерживать?}
\soul{Хорошо… Представьте, вы питаетесь таблетками. Зачем тогда вам нужен желудок? Здесь природа сделала ошибку?  И вы опять скажете: ”природа (!) сделала ошибку,  и мы обманули её и нам не нужны многие органы.” Вы скажете- это так.}
\people{Да….}
\soul{Далее… Да, вы перейдёте на таблетки.  Будет такое время. Но вы… И будет  время, когда будете тайком … желать обыкновенной пищи.}
\people{У-у… Интересно… А скажите…. Аппендицит…}
\people{1-2-3-4-5-6-7-8-9 (идет счет)}
\soul{Спрашивайте.}
\people{Аппендицит – что  это за орган? О нём идёт много споров. Можно ли без него жить или он какую-то функцию выполняет, если в организме ничего лишнего нет?}
\soul{Вы не внимательны, мы вам только что произнесли ответ.}
\people{Ну-у… Повторите. Что мы… Да, ничего в организме лишнего нет. Тогда, почему можно жить без аппендицита?}
\soul{Вы можете жить и без ног и без рук. Вы согласны?}
\people{Согласен. Но, всё-таки, какая функция за аппендицитом, оставшаяся, или нами не распознанная? Вы не можете сказать или вы не знаете?}
\soul{Мы скажем, вами не познана, и, скажем так же, - ``сторож''.}
\people{Сторож… От переедания… Так…}
\soul{В вашем понятии, - от переедания. Вы хотите сказать, - удалите аппендикс, и…}
\people{Это одна из версий.}
\soul{…вы не будете переедать?}
\people{Та-ак…}
\people{Он является “сигналом” к тому, что раньше ты много переедал, а…}
\soul{Тогда объясните, почему у детей происходит воспаление? От переедания?}
\people{Я думаю, что от ошибки хромосомного аппарата.}
\people{Тут мы не знаем функции аппендикса и вы нам, похоже, не можете объяснить или не хотите.}
\soul{Давайте, мы вам объясним сейчас всё, что вы зададите… И что это будет вам? Вы скажете:  “Здесь – изучены, здесь – уже получены. Пойдёмте дальше.”  И что получиться? Вы заглянули в конец задачника, нашли ответ, а задачи не научились решать.}
\people{Ну, тогда вопрос…}
\soul{Да, в вашем понятии – аппендицит - от переедания. А почему вы говорите о множестве версий и не можете соединить их все? И, даже ту, что это не нужно.}
\people{Ну, если бы мы соединили, - мы б, наверно, сделали какое-то открытие. Были б  учёными.}
\soul{Согласитесь, что вы пришли к тому времени, когда всё новое создаётся на стыках ваших наук.}
\people{Да… Бионика…}
\people{Всё понятно.}
\soul{Соедините все версии на ваш вопрос. И, даже ту версию, что он не нужен. Сумейте соединить и вы получите ответ.}
\people{Спасибо за подсказку.}
\soul{Далее… Запомните, если к вам придут и станут говорить конкретно то-то и то-то, - БОЙТЕСЬ того!}
\people{Вы предупреждали.}
\people{А-а…Ясно. Тогда, об организме человека…. Действительно ли…}
\people{1-2-3-4-5-6-7-8-9  1..(счёт)}
\soul{Спрашивайте.}
\people{Действительно ли, что раковая болезнь “запускает” в мозге человека какой-то “ центр” когда хочет уничтожить собственное тело по каким-то причинам? Чаще всего, по негативным причинам.}
\soul{Тогда объясните мне, - человек, который идёт  топиться, он болен раком?}
\people{Нет.  Нашёл другой способ.}
\soul{Здесь тоже есть цель. Здесь есть “центр” уничтожения. Почему ж вы тогда решили…}
\people{Это – такая гипотеза…}
\people{Это, скорее всего, когда человек сам не понимает, что…}
\soul{Информационный голод!}
\people{Что-о?}
\people{Информационный голод.}
\people{А ваша версия рака? От чего…}
\soul{Теперь, представьте… У вас есть люди, которые забыли, как и что делать… Что вы делаете с ними? Вы называете их “больными”. Почему вы не можете сделать аналогию и здесь? В вашем понятии, две новые болезни – это, всего лишь, клетки забыли, что им делать, и как. И они переходят на другую программу или создают хаос.}
\people{Верная картина. А как научить делать то, что…}
\soul{А как вы научите клетки, если вы не можете со-организоваться сами?}
\people{Угу. То есть, у гармоничного человека и клетки разбалтываться не будут. Верно?}
\soul{Н-да…}
\people{То есть, ну-у… В стадии, когда человек уже болеет, можно найти - чем…}
\soul{Вы нашли. Вы нашли, минимум два способа. И отвергли их! И мы называли вам один из этих способов. И что? Это привело? И что вы сделаете, если мы назовем вам то-то и то-то? Вы сможете доказать это другим? Если люди, обладающие  боОльшим  опытом в медицине не смогли сделать этого, как сделаете это вы?}
\people{Ну-у, мы не берёмся…}
\soul{А что мы сделаем, если мы дадим вам рецепт? Что вы будете делать?}
\people{Ну. Мы попытаемся…}
\soul{Что вы “попытаетесь”?}
\people{Донести через прессу.}
\soul{Кто будет слушать вас? Вы говорили о Добре и Зле… А теперь. представьте… Можете ли сделать так, что бы вам поверили? Далее…  Если мы вам дадим рецепт, да, мы вам сделали добро…  Но, для вас, мы сделали  зло. Ибо, вы, увидев сопротивление, непринятие, в вашем понятии – “истин”, что получили вы – вы потеряете себя… Вы перестанете доверять всему человечеству. Вы найдёте себе множество врагов, которые не верят вам. И, тогда, можно назвать это “добром”? Мы говорим с вами…}
\people{Мы не такие фанатики.}
\soul{Мы говорим с вами – лично для ВАС.}
\people{Но, если бы была интересная подсказка, мы бы могли заинтриговать читателей, в газетах, рассказами…}
\soul{Мы говорили вам…}
\people{Так.}
\soul{… о двух способах, которые могут это сделать. И было множество, уже выздоровленых этим способом. Ну и что? Это привело? Что даст ваша подсказка? Только неверие в ваши силы. Неверие в ваши СОБСТВЕННЫЕ силы. Далее… Может ли “совершенное Добро”, придя в низший план, сделать себе “Зло”?}
\people{Наверное, нет.}
\soul{А вы подумайте, может ли “Совершенное Добро”, придя, скажем, в ваш мир, сделать себе “Зло”? В вашем понятии.}
\people{Себе? Может. Запросто!}
\people{То есть, отдать свою, мягко выражаясь, “энергетику”, на подавление “Зла”?}
\soul{Вы говорите : “энергетику”…}
\people{Ну, в нашем понятии.}
\soul{Мы говорим вам проще. В вашем понятии, к вам пришёл Иисус. Он пришёл к вам “совершенным добром” … И получил что?  Огонь! Сожжение! }
\people{Как сожжение? Его же распяли…}
\soul{А вы подумайте, о каком ``огне'' говорим мы…}
\people{А-а…}
\soul{Далее… В вашем понятии – он принёс себе зло. Согласны?}
\people{Ну, да.}
\people{Ну, согласны. Но-о, всё-таки, он оставил след. И люди, которые верили ему…}
\soul{Но! В вашем понятии – он сделал себе Зло! Согласитесь.}
\people{Верно.}
\soul{Для того, что бы доказать вам, что он занимался медленным самоубийством – много ли надо доказательств? Он пришёл к вам Добром. Что же сделали вы? Вы, уничтожили его. Так что же получается, рассудите логически… Он пришёл добром, и сделал себе  зло. Правильно? В ВАШЕМ понятии – он сделал себе Зло. В ВАШЕМ!  И как вы тогда можете объяснить и понять, ГДЕ есть разница Добра и Зла? Поймите! ДОБРО приходит, это не значит, что она будет добра всем и вся. Для многих, она будет, как ЗЛО! Но, для многих, для “других” – может стать Добром. Вот вам – Единство Добра и Зла! И, вот вам – отсутствие его! Спрашивайте!}
\people{То есть… Можно прийти к выводу, что ваши познания в понимании природы, нам ничем не помогут… И эти достижения ваши будут бесполезны для нас?}
\soul{Мы – не имеем физики. Так зачем же мы будем спускаться на этот уровень, который уже был пройден нами? Поймите! Если мы будем говорить вам о “рецептах”, то мы будем больше похожи на знахарей. Не больше. Ибо, мы обладаем другими способами “вылечить”. И будем стараться вылечить примитивными  для вас способами, чтобы вы не смогли понять, как это делается и не  привести тогда  себе вред. И тогда, мы будем просто “знахарями”. Мы будем пытаться быть “ хирургами с кухонным ножом”. Хотя, имеем и “ скальпель”, но боимся показать его вам.}
\people{То есть, наше сотрудничество желательно на духовно-нравственном уровне? Так?}
\soul{Мы говорили вам об инопланетянах. И говорили вам, что МЫ – не ОНИ.  Далее… Что вы ищете? Вы ищете Знания. А духовных ли знаний вы ищете?}
\people{Скорее всего, не очень понимаем это.}
\people{1-2-3-4-5-6-7-8 (идёт счет)}
\soul{О ВЕРЕ… Мы говорили, что ВЕРА может сделать многое. А теперь представьте… Те же ваши болезни, можно лечить и химически, лечить ВЕРОЙ… И в вас есть,- даже в вашем теле, есть все химические элементы способные и вылечить и вашу же болезнь. И вам, порою, слишком редко бывает нужно химия “ извне”. Она есть вся в вас! И не мне вам это доказывать.}
\people{С этим, вот, вопрос…  Это вы имеете в виду уринотерапию?}
\soul{Не только. Вы говорили вначале – о ВЕРЕ. Подумайте.}
\people{Вы говорите о химических веществах, которые…}
\soul{Мы говорим вам, - и, будьте внимательны,-  что в вашем же теле есть все элементы, способные вылечить любые болезни! А вы говорите мне – “ уринотерапия”…}
\people{Ну, тогда нам необходимо…}
\people{Как их включить?}
\people{Да?}
\soul{Вы должны суметь “включить”, как вы говорите. Для этого нужна ВЕРА. Согласитесь. Мы приводили вам пример – операционный стол. Человек, не желающий жить – вряд ли выживет. Даже, чем бы его не лечили.}
\people{Ясно.}
\soul{И, чаще, поддавшись болезни,- вы говорите “ наказан”, - не разумеете, что сами наказали себя. Или – решили испытать. Поймите. Вы  - называете себя человеком. Не видите, что в себе, в вас, - есть другой человек. Намного могучее вас. А всё остальное – всего лишь “тело”. И того человека вы называете “ душой”. И думаете, что любая болезнь – наказание. Разве? Тогда вспомните ваши уроки физкультуры. Для тела было не очень приятно делать многие занятия. Это было “наказание”? Или обучение?}
\people{Можно?(спросить.прим.) Представьте ситуацию -  кто-то из нас смертельно заболел. Вот из присутствующих которые с вами беседуют…}
\soul{Мы не будем помогать вам в этом.}
\people{Вы не будете помогать?}
\soul{Нет. Ибо, вы будете идти уже нашей дорогой. Поймите. Вы боитесь. Боитесь за свою “одежду”. Вы боитесь умереть по той причине, что вы боитесь потерять вашу “одежду”, какой бы она ни была, красивой или нет.}
\people{Нет, ну, может быть не сделаны в жизни самые главное… Дела не сделаны…  Поэтому нам жалко уходить.}
\soul{А вы решили… В чём ваша “жалость”? В том, что вы “не успели сделать дело”? И всё? И Мир остановился? В том “жалость” ваша, ваших “дел не успевших”???}
\people{Ну… ``ты не “ выразился”"…}
\people{Близкие будут страдать. Горе.}
\soul{Горе? Ваше горе об умерших, – это, всего лишь, эгоизм. Эгоизм. Вы переживаете не за человека который умер, а за то, что “вот он умер – ушёл от нас”! Вспомните ваши “ переживания”! Вы  переживаете – себя, а не того, кто “умер”. И это есть - ЭГОИЗМ.}
\people{Это справедливо, но, наверно, наши…}
\people{1-2-3-4-5-6(счёт)}
\people{Скажите, вот у нас такое ощущение… Мы к вам очень расположены… Если бы у нас попросили помощи медицинской или какой либо… Я думаю, что мы предприняли все возможности, чтоб вам правильно сделать рецепт или  помочь. А вы о нас достаточно равнодушно судите… Ну, умираете,- и ладно…}
\soul{Разве? Как в вашем понятии “Равнодушие”? Вы спрашивали о Добре и Зле губящее душу… И мы вам говорили, что вы не знаете, где Добро и где Зло. И как вы можете узнать  доброе мы сделаем или нет, если продлим жизнь вашу? А может быть, умирая, вы найдёте другую дорогу? Более.  А мы – спасём вам жизнь… И вы останетесь здесь и потеряете ту дорогу. Почему вы решили, что вам нужно помогать в этом?}
\people{Значит, вам виднее…}
\soul{Подумайте! Вы говорили о судьбе… Мы говорим вам:  Да, у вас есть судьба. И мы не имеем права её менять. Но мы и говорим вам о праве выбора. Если вам суждено, мы не имеем права вмешиваться. Хотя и могли бы. Но, тогда – мы придём к вам Злом.}
\people{То есть, от вас ожидать просьбы об оказании помощи…}
\soul{Мы говорим вам, что физически – вы нас не интересуете. Если в этом видите равнодушие, пожалуйста. }
\people{Ну, допустим, вам захотелось  духовно от нас получить помощь . То есть передать нам интересный материал, нужный для человечества, в виде книги.}
\soul{Нет.}
\people{Вы так же не обратитесь к нам за такой помощью?}
\soul{Подумайте. Мы разговариваем с вами с какими целями? Подумайте!}
\people{Научить думать.}
\soul{Научить думать?}
\people{Нет, ну может родиться из этих бесед книга, или статьи, по крайней мере. Мы не исключаем этого. }
\soul{Поймите. Если вы говорите “ научить думать”, здесь есть и ответ. Разве вы можете написать книгу не умея думать?}
\people{Ну, можно проконстатировать  все эти … Буквально переписать и думать не надо, в принципе… Можно. Такой вариант не все…}
\soul{Нет… Чтобы задать те вопросы, вы должны думать.}
\people{Ну, всё правильно. Тут идёт развитие.}
\soul{И есть только качество. Но это уже другой вопрос. И даже, если вы сможете ``один к одному'', как вы говорите “не думая” всё повторить для других, почему вы решили, что кто-то другой, читая вас, не будет думать?}
\people{Хм… Верно. Вопрос такой… Ска… 1-2-(счёт)}
\soul{Задайте по три вопроса каждый.}
\people{Скажите… А вот раз мы умираем, и вроде как должны “не горевать”, а когда, значить, рождается человек … По идее, мы должны плакать, что он пришёл сюда “ страдать”, так сказать.}
\soul{Нет…}
\people{Нет?}
\soul{Мы говорили вам, что вы горюете о себе, а не о умершем. Будьте внимательными.}
\people{Тогда у нас будут отсутствовать все реакции на боль, на добро… На всё.}
\soul{А вы считаете – реакция “боль” и “всё” – только относительно себя? Вы даже не можете представить, как можно переживать за умершего человека… Именно за УМЕРШЕГО, а не то, что вы ПОТЕРЯЛИ его.}
\people{То есть, надо ставить себя на его место, чтобы понять его, в общем-то.}
\soul{Вы никогда не сможете поставить себя на чьё-то место. Вы всегда остаётесь своим “Я”. И всё остальное - всего лишь только игра.}
\people{Ну, тогда не понятно, как мы можем кого-то любить? Потому-что мы любим тоже, так сказать, “для себя” там… Или как-то…}
\soul{Разве?}
\people{Ну-у-у… Приблизительно конечно.}
\soul{Тогда объясните мне. Когда у вас рождается любовь, - а у вас были такие моменты,- вы думали о себе?}
\people{Да-а…  О себе - я, действительно, не думал…}
\soul{Зачем вы спрашиваете тогда? И как мы сможем объяснить  не познавшему любовь, что такое }
  Любовь? Согласитесь, что этого нельзя сделать. Подумайте.  Мы говорим о бессмертии души, и вы говорите, что души могут слиться. Так что ж получается? Ведь слиться может что-то плохое? Тогда душа может погибнуть? Нет. Вы всегда остаётесь самим собой. Ваша сущность  всегда – ваша. Ваше “Я”, всегда будет Вашим “Я”. Но, истинное “Я”! Вы же – ваше тело принимаете за то, истинное. И, потому, говорите: - Я вжился! Я вжился, я такой, как он! Хотя, вы может быть похожи “одеждой”, эмоциональным фоном. Сущность – останется всегда одна.
\people{Ну, мы же говорим - “мы чувствуем душой”… Как это понять?}
\soul{Если вы говорите “ чувствую душой”,  это уже – ложно. Ибо вы говорите на физическом уровне. Когда вы будете чувствовать душой, вы не будете говорить: “ я чувствую телом”…}
\people{Мы выражаемся просто так. Словами.}
\soul{…вы будете забывать о нём! А в ваших “выражениях” и есть – ошибки! Если бы вы так не выражались, то, значит, вы бы этого и не делали! Ибо, вы бы и не знали понятия, как выразить.}
\people{А скажите, вот вы говорите, что вы идёте впереди нас и делаете ошибки…}
\soul{Мы не говорили в прямом смысле. Мы говорили, что мы на другом уровне. И мы говорим вам, что вы придёте на него. Но нельзя говорить; “выше”, ”ниже”, ”ближе” и ”дальше”.}
\people{”Другой”, скажем так…}
\people{Вот, (в связи) с этим делом, вопрос…  Вы видите всё… Но, “видеть” и ”управлять” – это не одно и то же…}
\soul{Нет.}
\people{С точки зрения “управления”… Силами, которые вы видите, последствиями, которые вы видите,  вы – можете управлять?}
\soul{Мы – часть той силы. Но мы осознаём себя. Вы – тоже часть той силы. Но вы – не видите её. В вас эта сила - хаос. Мы же можем управлять этой силой. Иначе – мы – это та же самая сила, но – познавшая себя.}
\people{Скажите тогда, управление этой силой вызывают у вас какие-нибудь ошибки или нет?}
\soul{Не ошибается только тот, который ничего не делает.}
\people{Скажите, а ошибки эти вот… Они для нас губительны, или мы являемся…  Все их не постигнем – ваши ошибки…?}
\soul{Чаще, мы ошибаемся “выше”.}
\people{То есть, они на нас не влияют?}
\soul{Не-ет, они на вас влияют. Мы же говорили вам о Единстве. Влияет всё. Но… Есть “степень влияния” . Согласитесь, что если мы не будем влиять “вниз”, но будем влиять “вверх”, то, далёкий “верх”, всё-таки, на далёкий “низ” даст отражение.}
\people{Тогда, вот мой последний вопрос… Скажите. Вот, ваши ошибки, это… Являемся ли мы вашей ошибкой?}
\soul{Нет… Мы же говорили вам : - Вы разговариваете с самими собою. И, если вы  задаёте вопрос, то вы же и отвечаете на него! И – каждый из вас делает  так! Мы говорили вам во “множестве”, и мы говорили вам, как “личность”. Мы говорили вам, что мы – одна “личность”. Вспомните! Но, мы и говорили вам, что всё ЧЕЛОВЕЧЕСТВО, это и есть – Вы. Вспомните и соедините эти “узлы”. Задающий вопрос – слышит  собственный ответ. }
\people{То есть, мы, создавшие эту ошибку (человечество), а вы – ушли вперёд?}
\soul{Будьте внимательны. И подумайте, ЧТО было сказано.}
\people{Скажите. Почему вы вынуждены прервать сегодня так рано контакты?}
\soul{Вынуждены не мы, - вынуждены вы.}
\people{Из-за того, что посредник-переводчик себя плохо чувствует?}
\soul{Нет… Мы находимся в эмоциональном мире… И, заметьте, НАХОДИМСЯ там, но не ЖИВЁМ!  И мы видим общий план. И мы видим ваш план, как отрицающий, на данный момент, возможность вести далее.}
\people{Нет. Мы, как раз, расположены беседовать. Давайте тогда продолжим. Исключите ваши условия о трёх вопросах.}
\people{Скажите, это не я ли являюсь помехой в этом  в этом контакте? (приглашённый на контакт гость.прим.)}
\soul{В вас говорит гордость? }
\people{Нет… Ну, вы говорите….}
\people{Скромность говорит в нём.}
\soul{Тогда подумайте. В прошлых контактах вас не было. Тогда мы должны были бы не прекращать и разговаривали бы до сих пор!}
\people{Ну, вы же говорите, “У вас мало времени на этот контакт”?}
\soul{Поймите…  Вы же устроены “физически” и ваш мозг создаёт защиту. Ваш мозг – может устать, и многое не принять. Вы устаёте  даже эмоционально. Устают даже ваши эмоции. Вспомните ваши переживания.}
\people{Ясно.}
\people{Понятно.}
\soul{И вы говорите об условиях.  Поймите, мы не ставим условия. Но, если мы будем продолжать, то вам будет сложнее понять, и только лишь.}
\people{Нет, я бы ещё продолжил. Хотя бы…  Вот, что нам интересно… есть ли структура и какова организация вашего общества, которое вы представляете? Или вы – индивидуум, который  в обществе не задействуется?}
\soul{Вспомните и прослушайте, что мы говорили вам, буквально пять минут назад. И вы найдете ответ. И вы найдёте нашу “структуру” и вы найдёте, кто разговаривает с вами. Неужели нам приходится повторять то, что уже слышали вы? И повторять, когда уже было сказано вашим языком! На ВАШИ вопросы – отвечаете ВЫ САМИ. Что я могу сказать вам более? Что вы и разговариваете сами с собой?}
\people{То есть, это наше подсознание? Вот, нам это и трудно понять!}
\soul{А тогда подумайте, как переводчик может говорить вашим подсознанием?}
\people{Тогда, получается, что вам никто не мешает в нашем контакте? Вас никто не контролирует.}
\soul{Вы мешаете сами… Мы говорили вам о наказании. Вспомните.  И мы говорили вам, что мы наказываем себя.}
\people{Значит,  закон “не навреди” действует.  Да?}
\people{Неся Добро нам…}
\soul{Поймите. Вы не можете понять одного, - мы на другом плане. И мы говорим вам: - Если вы задали вопрос… Если лично Вы задали вопрос,  значит, лично Вы на него и отвечаете! }
\people{Но, этот вопрос нам, как раз, даёт новизну. Я задаю вопросы, на которые не знаю ответов и, вдруг, оказывается , я прекрасно знаю эти ответы. Вы что-то… Ну-у-у…}
\soul{Подумайте! Подумайте…}
\people{То есть, этот контакт просто раскрывает нашу возможность?}
\soul{Нет… Вы просто не можете понять. Представьте, что вы двигаетесь столь быстро, что изображение ВАШЕ осталось позади, а Вы ушли вперёд. И вы обернулись и сумели увидеть себя. Нельзя же сказать, что вы раздвоились? Согласны?}
\people{Трудно сказать…}
\soul{Трудно? Хорошо. Эхо. Вы не можете представить скорость света, потому что вы не можете представить, как расставить изображения – хотя, делаете это всегда постоянно. Тогда, давайте говорить о звуке, - эхо! Что родило эхо? Вы же и родили. Согласны?}
\people{Да.}
\soul{И вы же слышите.}
\people{Верно.}
\people{Если оно отражается.}
\soul{А теперь подумайте. Я вам скажу, что вы - эхо.}
\people{Ну, ладно. Мы подумаем.}
\people{То есть, если я не получил ответа на функции аппендикса, то я его и не знал. Если б я предполагал хотя бы, то я получил бы ответ.}
\soul{Нет. Мы говорили вам, что вы знаете всё. Заметьте, и складите всё вместе. “Я” всегда говорил вам… “Мы” всегда говорили вам… Вы знаете всё, что было сказано… Теперь, я вам сказал более – ВЫ отвечаете на собственные вопросы. Далее… Я говорил вам, и Мы говорили вам, что ВЫ, есть Мы. И Мы, есть Вы. Почему вы не можете понять и сделать дальше шаг?  Мы сказали вам достаточно многое.}
\people{Э-э-э…. Вот… Вопрос, вот, к эху. Дело в том, что эхо…}
\people{1-2-(счёт.)}
\soul{Здесь вы можете понять,  почему нужен перерыв. Если бы мы с вами вели разговор “вначале”, вы бы уже смогли быть опытней. Запомнив  закон Вселенной, что …  Дайте счёт….}
\people{Обратный? Какой счёт?}
\soul{Вы не внимательны.}
 Конец записи.
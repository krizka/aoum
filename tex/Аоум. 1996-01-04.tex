Аоум. глава 4-01-96г
Георгий Губин
\people{**}
04-01-1996г.
\people{Сегодня 4-е января 1996-го года. Мы проводим сеанс.  Геннадий, Оля и Геннадий Белимов.}
\soul{Часы.}
\people{Часы?}
\people{Сейчас убираем последние часы, которые  остались в комнате.  Они на вас действуют ритмом своим не хорошо?  Мы чувствуем, что мы опять с нашими первыми представителями общаемся.  Мы очень благодарны вам за то, что, по-видимому,  вашими усилиями мы смогли выйти и на другие миры… общества,  слои,  астрала миры  или что. Это не без вашей помощи. Но теперь, мы хотели бы расспросить, во-первых, как вам это удалось? Почему пришла эта мысль, всё-таки, свести нас с иными мирами, с иными представителями цивилизаций…}
\people{В прошлом.}
\people{В прошлом?  Да?  Как? Как вам пришло это?}
\soul{Мы пытались показать  вас.}
\people{Нас самих в прошлых воплощениях, так что ли? }
\soul{Каждый голос, что слышали вы, один и тот же. }
\people{Но в разных тогда ситуациях? Допустим, если ситуация была с эльфами, гномами,  то есть  получается, что вы показали нас  в наших фантазиях, в  нашем фантазийном мире? Так?}
\soul{Да. Но только вы живёте там. Вы же фантазируете, значит, и мир - ваш. }
\people{Так. Ну, он реален, он тоже размножается, он имеет право,  длительность существования? Так?}
\soul{Он реален даже физически. }
\people{Ну, например, как-то мы встретили в Волжском девушку эльфа, которая якобы вспоминает те миры, это, с вашей точки зрения,  реальная ситуация или чисто надуманная? }
\soul{Реальная. Относительно вас может быть и ложь. Мы говорили вам о лжи. Вспомните.}
\people{Ну, дело в том, что меня поражает то, что незнакомые люди прежде, начинают переписываться и вспоминают свои прежние  контакты, свои прежние жизни. Хотя, например, в произведениях других фантастов этого быть не может.  Потому, что произведения других фантастов -  это чисто их выдумка и никакой переписки в их фантазиях быть не может. А здесь встречается…}
\soul{Тогда,  вспомните ``нападение марсиан''.}
\people{Ну, понятно. Герберта Уэллса. Ну, там… там могла быть переписка, наверно.}
\soul{Мы говорили вам о множестве богов. Приводили примеры. Существует и бог журналистики. Потому, что вы мечтали, вы построили его.}
\people{Ну, мы это идею не развивали особенно.}
\people{То, что мы придумали, оно уже существует, подкрепленное  множеством  других…}
\soul{Вы же говорили, что каждая мысль материальна.}
\people{Ну, мы говорим, но  ещё это в полной мере не ощущаем. И не можем пользоваться.}
\people{Воспринимаем. Это очень трудно для  восприятия.}
\people{Скажите, а вот ситуация, допустим, с тем же Сергеем Ивановым, из прошлой жизни, 1916-го года…}
\soul{И снова - имена?}
\people{Имена, не надо? Но, он-то говорил!  Потому-то, мы думали, что это может быть…}
\people{1922-й год, 18 мая было, да? }
\people{Был контакт. Как вы эту ситуацию объясните?}
\soul{Мы когда-то говорили вам о множество контактов. И  говорили, что и в прошлом, и даже вы были переводчиком. }
\people{В прошлой жизни, да? }
\soul{В вас остался страх, вы должны помнить. Мы говорили, что когда-то мы сделали ошибку и позволили вам помнить. }
\people{Ну, может быть мы ещё не готовы.}
\people{Скажите, а почему вот так получается, такое ощущение, что мы в одном времени, как бы? Это что, как это объяснить? }
\soul{На время делите вы. А мы всегда говорили вам о его несуществовании. Мир един,  вы знаете об этом, но не можете понять. Может быть в едином мире время?}
\people{Нет.}
\soul{Только физически.  Мы говорили вам, что вы можете создать машину времени, но вы никогда не сможете повторить один к одному – это будет просто попытка или взгляд со стороны. }
\people{Скажите, а  вот если мы можем с прошлым, допустим… с прошлым каким-то образом… на прошлом контакте…побывали  в прошлом. А  в будущем?}
\soul{А в будущем - побывал он.}
\people{Переводчик?}
\soul{Вы не согласны?}
\people{А! ``Он''. То есть…}
\people{Да. }
\people{Он побывал…был ошарашен тем, что мы ему сказали. Надо ли, вообще-то, говорить о том, что будет война, о том, что умер их вождь?  Или это опасные сообщения? И почему мы не получаем, допустим,  через переводчика прогнозы на будущее? Ведь они очень интересны человечеству.}
\soul{Представьте, что делаете вы. Вы говорите, что мечты его не сбудутся, что мир, в котором он живёт и который он помогает строить, - разрушится. Представьте, что делаете вы, какую приносите боль своими знаниями, своим желанием. И теперь  представьте, если  он будет говорить о том, что вы ему дали знать. }
\people{Ну, да.}
\people{Его, конечно, или один вариант - в психушку отправят, или воспримут, как вруна, лгуна. Так? Мы, в общем, этим портим, да, человеку его существование?  }
\people{Не только….}
; …
\people{Ну, мы поняли по вашему молчанию, что так нельзя. Но с другой стороны, поставив себя на его место, вот лично я очень хотел бы узнать, а что ждёт нас на пороге 2000 года?  Ведь много об этом идёт толков, неожиданностей. То же предсказание Нострадамуса… Кто их оплевывает,  считает, что это белиберда, а мы, вот уфологи,  допустим,  мы видим, что они, в основном, сбываются. Значит, 1999 год для человечества чем-то опасен?}
\soul{Предсказания верны только тогда, когда они сбудутся. И  когда вы уже сможете понять, что это сбылось. Тогда вы признаете предсказание верным.}
\people{Но если у Нострадамуса большинство предсказаний, всё-таки,  сбывается, значит 1999 год  или порог 2000-го года человечеству грозит какими-то необычными потрясениями, так?}
\soul{Вы идёте на обман. }
\people{То есть, вы нам совершенно ничего не хотите из прогнозов говорить? }
\people{Не надо. А зачем?}
\people{Хорошо, давайте продолжим. Вот, на нас вышли ``санитары'', так называемые. Цивилизация, которая чувствует боль землян. Как вы можете охарактеризовать эту субстанцию, эту ситуацию, как мир?}
\soul{Их можно было бы назвать ``утешителями'', но нельзя сделать этого. Для вас, это обидно. }
\people{Почему?}
\soul{Почему? Потому, что вы считаете, что вы строите мир, и больше никто. И по этой причине вы отвергли все остальные миры.}
\people{Лично мы не такие уж… одиозные.}
\people{Ну, в общем, о человечестве речь идёт. Ну, потому, что нас призывают всегда к единству, а… }
\people{Ну, давайте по цивилизации этих ``утешителей'' или ``санитаров'' . Почему они невидимы и неощутимы для нас, хотя они говорят, что они такие же, как мы, и живут, существуют в трехмёрном мире?}
\soul{Вы замечаете поля?}
\people{Нет, не замечаем.}
\soul{Замечаете электрический ток?}
\people{Нет.}
\soul{Нужны объяснения?}
\people{А-а. То есть, они примерно, как поля выглядят? Полевое состояние? Не материальное?}
\people{А у нас тоже такое состояние, только другое немножко.}
\people{А как вот объяснить, что они смещены по фазе? Можете нам пояснить это? Это иная частота колебания?}
\soul{Нет, та же.}
\people{По времени?}
\soul{Да, можно сказать и так. Это будет близкий ответ. }
\people{Но они - телесны? }
\soul{Да.}
\people{Плотные тела имеют, да?}
\soul{Да. }
\people{Почему же мы их не видим?}
\people{По времени смещение.}
\people{Смещение по времени? }
\people{Скажите, а они вот, как  впитывают в себя наши эманации?  В тело, или в свои, ну, излучения, тоже у них ауры есть, допустим, какие-то ещё? }
\soul{Это их жизнь. Это не работа – они живут тем, что питаются вами.  И вы, и вы…}
\people{1…}
\soul{А меня уже считать учат. (шёпотом)}
\people{Это кто?}
\soul{А вы кто? (шёпотом)}
\people{Ты - Вася? Ты - Вася?}
\soul{А где Петя? (шёпотом)}
\people{Петя где? Пети сегодня нет. }
\people{А кто вас учит считать?}
\soul{Монахи. (шёпотом)}
\people{А-а! Это какой век?}
\people{Какая сейчас погода у вас?}
\soul{А у нас всегда одна и та же погода.}
\people{У вас нет дождя?}
\soul{Вода?}
\people{Да, с неба вода. Дождь идёт?}
\soul{Есть.}
\people{А почему одна и та же? Солнце светит?}
\soul{А-а-а! У нас ночью всегда вода.}
\people{С неба? С неба ночью вода всегда?}
\soul{Сверху. }
\people{Сверху.}
\people{А вы живёте в пещере?}
\soul{Да. }
\people{Глубоко под землёй?}
(перепад в записи)
\soul{Я их не люблю, потому, что они бьют больно.}
\people{Монахи бьют вас?}
\people{Они вас бьют? Вам сколько сейчас лет?}
\soul{Ещё не знаю.}
\people{Ну, вы взрослый? }
\soul{Да. У меня две жены.}
\people{Ух ты!}
\people{Так…}
\people{А как они узнали, что вы обладаете такими свойствами - общаться с другими мирами?}
\soul{Они  пришли и сказали, что дадут мне землю. }
\people{И вы согласились, да?}
\soul{А вы б не согласились?}
\people{Ну, так они обманули и землю не дали вам, всё-таки?}
\soul{Они никогда не обманывают!}
\people{Ну, землю вам дали?}
\soul{Дали.}
\people{А вам удалось их ожидания выполнить? То есть, вы связались с иными мирами, вы рассказали им много интересного?}
\soul{Не знаю, но они меня после стали учить.}
\people{Чему?}
\soul{Ну… я со следующего посева буду знать, что такое ``две тьмы''. }
\people{Две тьмы? А-а, это мера! Да?  Мера измерения? Вы, что выращиваете - зерно, пшеницу, да?}
\soul{Не знаю.}
\people{Ну, вам дают зерно?}
\soul{Да.}
\people{Землю сами обрабатываете? Или на быках, или на чём?}
\soul{Жёны.}
\people{Жёны обрабатывают? А кто копает землю?}
\soul{Копать?}
\people{Да. }
\people{Сеет кто? }
\soul{Зачем? Жёны. }
\people{Куда? В землю?}
\soul{А я откуда знаю!}
\people{Это другой мир.}
\people{Вы на Земле живёте? У вас  солнце светит всё время, да?}
\soul{Светит.}
\people{Сколько урожаев в год вы снимаете? Вы живёте в тёплых краях?}
\soul{А я ещё не знаю, что такое год. }
\people{Ещё не знаешь?  А монахи тебя счёту учат, да? И…да?}
\soul{Да.}
\people{До скольки ты умеешь считать? }
\soul{Одна тьма.}
\people{Одна тьма. А две тьмы?}
\soul{Учат.}
\people{Будут потом, да?}
\people{А что такое одна тьма? Объясни нам, пожалуйста.}
\soul{Ну, это когда урожая может хватить на трёх жён. }
\people{А в течении какого времени? До следующего урожая?}
\soul{Да.}
\people{А когда следующий урожай будет?}
\soul{Когда пройдут дожди.}
\people{Ну, это долгое время придётся ждать?  Недели, две, три? Месяц?}
\people{Счёта нет у них. (шёпотом)}
\soul{Не знаю.}
\people{Не знаете?}
\people{А монахи только тебя учат счёту?}
\soul{Да, наверное.}
\people{А они часто тебя берут к себе?}
\soul{Петя помог. Теперь, часто.}
\people{Ну, Петя…  Вот у вас русские имена. Вы живёте в Руси?}
\people{Нет. Скажи, вот как ты соседа своего называешь?}
\soul{Никак.}
\people{А как к нему обращаешься?}
\soul{А у меня нет соседа.}
\people{Ну, а к жене?}
\people{Назови, как зовут твоих жён?}
\soul{Ауба.}
\people{А вторая?}
\soul{Уана.}
\people{Уана? А кого бы ты хотел в третью взять?}
\people{А как они тебя называют? Жёны как тебя называют?}
\soul{Они не могут меня называть.}
\people{Почему?}
\soul{Они женщины.}
\people{А женщинам нельзя  называть мужчин?}
\people{А-а… мужчин нельзя? А ты мужчина?}
\soul{Да. }
\people{Скажи, а монахи чему ещё тебя, кроме счёта, обучают?}
\soul{Они учат меня рисовать.}
\people{Рисовать?}
\people{Буквы или рисунки?}
\soul{Себя!}
\people{Себя? И у тебя получается?}
\soul{Бьют.}
\people{Бьют, если плохо получается? А скажи, чем рисуешь?}
\soul{Камнем.}
\people{Камнем?}
\people{И по чему? По камню?}
\soul{Да. }
\people{По песку? Или по камню?}
\soul{Нет, они приносят воду, которая горит, а из неё потом берут камни. А они рисуют. }
\people{По камню? По камню рисуете?}
\people{Типа мелков наверно.}
\soul{На камне.}
\people{И они на камнях тоже рисуют?}
\soul{Они?}
\people{Да.}
\soul{Не-ет.}
\people{Только ты рисуешь?}
\soul{Ну, они же монахи. Они приходят и просто учат. }
\people{А  у тебя свой дом есть? }
\people{В пещере. (шёпотом)}
\soul{Да.}
\people{Он из камней или в земле вырыт?}
\people{Пещера.}
\people{Пещера?}
\soul{Что такое ``вырыт''?}
\people{Вырыт - это углубление в земле.}
\soul{Нас за это бьют, если мы будем вытаскивать. }
\people{Нельзя,  да? А монахи, как живут? У них большие помещения?}
\soul{Я там не был.}
\people{Тебя туда не приглашают? А куда же тебя берут?}
\soul{А туда нельзя. Они приходят.}
\people{Они к тебе приходят? И говорят, что… надо с кем-то поговорить, да?}
\soul{Они мне дали землю и  обещали третью жену!}
\people{Понятно. }
\people{А это когда было? Когда ты с Петей познакомился?}
\soul{Да.}
\people{А что они тебе сказали после этого?}
\soul{Не знаю. Они меня били сперва. }
\people{Скажи у вас сейчас царь или король, или  кто-то есть, кто вами правит? }
\people{Главный.}
\soul{Монахи.}
\people{Только монахи? А у монахов, кто главный? }
\soul{К ним прилетают боги. Но их никто не видел.}
\people{А у каких морей вы живёте? Там есть леса моря, большая река, есть? }
\soul{Море есть. Но мы приходим только после трёх урожаев, один раз. }
\people{Один раз в году?  Значит, три урожая в год получается у вас.  Скажи, а в прошлый раз, когда ты познакомился с Петей, монахи пришли… Ой, монахи… Что потом? Много времени прошло после этого?  Ты урожай снимал или нет? Землю давали тебе? }
\soul{Два.}
\people{Два урожая? После того, как ты с Петей…}
\soul{С новой земли.}
\people{С новой земли, которую тебе дали за Петю, да?}
\people{Два урожая с новой земли? А в чём ты одет? Вот мы сейчас не видим, в чём ты одет?}
\soul{Одежда? }
\people{Что на тебе?}
\people{Ты голый или?.. }
\soul{По…по…повя…}
\people{Повязка?}
\soul{Да-да!}
\people{А монахи во что одеты?}
\people{Они так же закрытые ходят все?}
\soul{А один хотел их увидеть, но он потом… он слепой.}
\people{Ослеп? А жёны как одеты? Тоже с повязками?}
\soul{Да.}
\people{А они у тебя красивые?}
\soul{А зачем я буду не красивых брать?}
\people{Хорошо. Но ты в пещере нашёл тёмной, там же не было видно?}
\soul{Почему тёмной? Нам давали огонь. }
\people{Вы огнем пользуетесь, да?  Он у вас горит в масляных лампадках или лучиной? Или костёр?}
\soul{Монахи приносят воду, а она горит.}
\people{О-о! Вот оно чё!}
\people{Нефть.(шёпотом.)}
\people{А скажи у вас какой цвет тела, у твоих жён? }
\soul{А я ещё не знаю, что это такое. }
\people{Вася, а в этот раз они, что тебе обещали, когда ты пришёл говорить с Петей?}
\soul{Жену.}
\people{Ты хочешь ещё одну жену? }
\soul{Да. У меня есть земля,  и я могу кормить.}
\people{А ты можешь кормить, а  не они тебя кормят? Они же сеют? Они же выращивают пшеницу?}
\soul{А зерно-то  я даю!}
\people{А-а! Они выращивают, собирают и приносят тебе, да? А ты этим заведуешь. А ты распоряжаешься, кому дать поесть, а  кому чего?}
\soul{Не понял.}
\people{Ну, они собирают урожай. Кто собирает? }
\soul{Я.}
\people{А! А они сеют, да? Жёны сеют, а ты убираешь?}
\soul{Наверно.}
\people{А зёрна, куда ты деваешь?}
\soul{Отдаю жёнам.}
\people{Сразу всё?}
\soul{Зачем всё? Монахи говорят, что давать, а что нет. А потом, мы ещё трём на камне и их можно есть. }
\people{Муку сделать.}
\people{А где вы готовите? На очаге? Еду готовите вы или едите так, просто, после того, как на камне… }
\people{Жарите или печёте хлеб?}
\soul{Не знаю, о чём вы говорите.}
\people{Ну,  на камнях вот вы разваливаете зерно?}
\soul{Да.}
\people{А потом, что с ним делаете?}
\soul{Сушим.}
\people{И всё? А едите как?}
\soul{Как едим? Просто едим!}
\people{Это же?}
\people{Ну, вы печёте, наверное? }
\people{А вы на огне что-нибудь делаете с едой?}
\soul{Два урожая проходит и уже можно есть.}
\people{А монахи ещё чего тебя учат, кроме  рисовать и считать? Ещё чему учат?}
\soul{Они произносят какие-то…слова,  а я должен повторять.}
\people{Повтори несколько слов для нас.}
\soul{Нельзя.}
\people{Почему? Это секрет?}
\soul{Они меня будут бить.}
\people{Нет, они не будут. Мы обещаем, что они не будут бить. }
\people{Мы попросим, чтоб они не били.}
\people{Назови несколько слов.}
\soul{Ауба.}
\people{Так. Ещё?}
\soul{Не буду. }
\people{Хорошо. А у вас есть скот? Коровы, животные.}
\people{Нет, мы уже спрашивали. У монахов есть. А вообще расскажи, как ты живёшь с жёнами? Как у вас жизнь идёт? Как вы проходите? }
\people{Как у вас день проходит?}
\soul{День? }
\people{Да. Ну, вот, как вы живёте? }
\people{Что вы днём делаете?}
\soul{Мне они объяснили, что такое день, но я забыл.}
\people{Ну, это когда светло, когда солнце светит. Когда вы не спите.}
\soul{Ну, сейчас приходят монахи. Солнце встаёт  – приходят они. Они меня сперва бьют, а потом учат. }
\people{А зачем бьют, они не говорили? }
\soul{Они не разговаривают.}
\people{А как же они тебя учат?}
\soul{Они, просто, бьют.}
\people{А среди монахов есть кто-то главный, да?  Ты его боишься?}
\soul{Нам нельзя бояться монахов.}
\people{А как вам сказали, что нельзя? Откуда ты знаешь?}
\soul{Мы должны их любить. }
\people{А кто  у вас объясняет это? }
\people{У него имя есть?}
\soul{Есть. Старейшина.}
\people{Старейшина, да?}
\soul{Они говорят ему, а он говорит нам. }
\people{А-а,  у вас есть старейшина!}
\people{Он умный?}
\people{Или старый?}
\soul{Говорят, что я тоже буду.}
\people{Будешь старейшиной, да?  Ну, мы надеемся на это, потому, что ты хорошо учишься. Это скоро будет?}
\soul{Обещали - две тьмы. }
\people{Две тьмы?}
\people{Через две тьмы?}
\people{Одна тьма – это два урожая.}
\people{Долго, да?}
\soul{Долго!}
\people{А ты ещё не седой человек?}
\people{А у них нет волос. Лысый.}
\people{Лысый?}
\people{Да. Скажи, а  опиши, как вы выглядите. Как выглядит твоя жена? Вот что у неё есть: руки, ноги, голова, глаза какие, всё, нос?  }
\soul{Есть.}
\people{Расскажи, какая она? Можешь это сделать?}
\soul{Не знаю, жена есть жена.}
\people{А вот, дети… }
\soul{У тебя нет жены?}
\people{Есть. У нас есть… Скажи, а дети у тебя есть?}
\soul{Есть.}
\people{Много?}
\soul{Один. }
\people{А в прошлый раз говорил, что два было.}
\soul{Два было. }
\people{Одного монахи забрали?}
\soul{А я Пете говорил уже.}
\people{Что говорил Пете? }
\soul{Что они приходят и забирают.}
\people{А, да.}
\people{И выращивают твоих детей у себя? Так что ли? }
\people{Они не знают.}
\soul{Не знаю.}
\people{А у тебя девочка или мальчик был… Остались?}
\soul{Девочка, мальчик?}
\people{Ну, да. }
\people{Ну, как жена или как ты?}
\soul{Как я.}
\people{Мальчик,  да?}
\people{Вы с женой часто спите? Говорили?}
\people{Детей, когда делаете? }
\soul{Приходят монахи и говорят, что пришло время. }
\people{Делать детей? Скажи, можешь описать, ну, как вы живёте? Вот, ну, днём приходят монахи, вас кормят. Как кормят? Из чего?}
\soul{Не-е. Они только приносят зерно, а я отдаю жёнам.  Они говорят, что сушить…я ложу на самый…}
\people{Камень?}
\soul{Куда садится солнце. }
\people{На горизонт? На стол?}
\soul{Камень. Солнце сушит. Проходит два урожая, и я могу уже есть. Но это я женам не даю.}
\people{А кто ест это?}
\soul{Я-я!}
\people{А они что кушают?}
\soul{Урожай.}
\people{Свой?}
\soul{Почему это - свой? }
\people{А какой?}
\soul{Мой!}
\people{А у вас нет травы?  Трава есть у вас?}
\soul{Трава?}
\people{Да.  Или одни камни?}
\soul{Камни есть. }
\people{А что-нибудь из камней бывает, такое зелёное? Растет? Трава, деревья?  Кусты?}
\soul{Красное.}
\people{Красные?}
\people{Может это другая планета? (шепотом)}
\people{Нет, монах нам сказал планета Земля.(к Белимову) Расскажи, как вы живёте, как ваше племя живёт? У вас же есть главные…}
\people{У тебя есть же такие же, как ты? Соплеменники есть же? }
\soul{Много.}
\people{Дружно живёте или соритесь друг с другом?}
\soul{Сориться нельзя, придут монахи.}
\people{Дружно живёте?}
\soul{А если вас будут бить, вы будете сориться?}
\people{Нет, не будем.}
\soul{Что тогда  спрашиваете? }
\people{А вы поёте песни? Смеётесь? Часто смеётесь?  У вас весело живётся? }
\soul{Да. Приходят монахи, а старейшина рассказывает от них и..ии… ирории.}
\people{Истории?}
\soul{Да.}
\people{Вам нравятся эти истории? }
\soul{Не всегда.}
\people{А какие, например?  Ну-ка, расскажи нам хотя бы одну историю.}
\soul{Ну, они нам рассказывают, как приходят боги, и что придёт время, и боги нас заберут к себе. Тогда  мы будем уже монахами.}
\people{Вам этого хочется?}
\soul{Э-э! Конечно!}
\people{А почему хочется монахами стать?}
\soul{Потому, что они бьют.}
\people{А-а, ясно.}
\people{А вы, может быть, плохо себя ведёте? А вы никогда не били, вот, не нападали на них?}
\soul{Тьфу, на вас!}
\people{Нельзя, да, нападать? У вас один Бог или много?}
\soul{Кого?}
\people{Бог.}
\people{[Шепотом] К монахам приходит Бог. Скажи, а у вас есть рассказы, вот вам рассказывают старейшины, которые… когда раньше ещё люди жили. }
\people{Расскажи, какой рассказ особенно запомнился. Нам это очень интересно.}
\soul{О монахе.}
\people{Что - монахи?}
\soul{Что он был такой же, как я, пришли боги и научили его.}
\people{И он что?}
\soul{А потом…забрали….}
\people{Куда забрали?}
\soul{На…}
\people{На небо?}
\soul{Да. Чего-то сделали с ним, он стал большой, пришёл, а мы были ещё дикие. И он стал нас учить. Он научил нас сажать, научил жён… (считать, видимо. Прим.)}
\people{Обеспечивать, да?}
\soul{…а потом, опять ушёл. }
\people{Вы знаете? Монахи помнят, как он ушёл?}
\soul{А мы ждём его.}
\people{Он ушёл на небо?}
\soul{Да.}
\people{Это было на какой-то огненной колеснице? }
\soul{Говорят - огонь.}
\people{Огонь был?}
\people{Скажи, а вот ты рисуешь, когда на камне, что ты рисуешь? Можешь рассказать?  Себя…А как ты себя представляешь?  Или ты видишь себя где-то?}
\soul{А они мне дают камень, а меня там видно.}
\people{В камне?}
\people{А-а! И ты себя видел? Видишь?}
\soul{Видел.}
\people{Ты бородатый?}
\soul{Чего?}
\people{А волос  у вас нет?}
\soul{Это чё? }
\people{Ну, вот, нет.}
\soul{Тра…травы - нет. (имеется в виду волосы.прим.)}
\people{“Травы'' нет… Скажи, вот ты себя рисуешь, вот, что у тебя есть, вот, как ты себя  рисуешь?}
\people{Голову? Да? }
\soul{Да.}
\people{Уши?}
\soul{Нет, уши не обязательно.  А нужна только голова, ноги и руки.}
\people{А глаза – это важно? }
\soul{Глаза у нас только старейшина может рисовать.}
\people{А-а! А нос? }
\soul{Нос?}
\people{Что такое нос, не знаешь, да?}
\soul{Не-ет.}
\people{А зубы у тебя острые?}
\soul{Зубы? Есть.}
\people{Зубами ешь? Пищу пережёвываешь?}
\soul{Да.}
\people{У тебя есть зубы?}
\soul{Есть.}
\people{Все? Много?}
\soul{Ну, я не знаю. Ну, я же не вижу в камне зубы!}
\people{А почему не видишь? Ты можешь, ты можешь раздвинуть рот и увидеть зубы.}
\soul{Я не пробовал. }
\people{Попробуй. Увидишь свои зубы, какие они есть. Скажи, а… Скажи… Всё?}
(сбой )
\soul{У кого-то слёзы, и снова нам работа…}
\people{У кого слёзы? У бывшего нашего персонажа, которого мы слышали? У кого слёзы? }
\people{Это вы - предохранители, да? }
\soul{Да, вы так называете нас.}
\people{А вы как себя называете?}
\soul{Как вы называете. Мы же предложили вам.}
\people{А-а… Ну, да… Скажите, это вы опять? Вы же, да?}
\people{”Санитары'' - так наверно?}
\soul{Вы хотели спросить – тот ли?}
\people{Да, да, да. Тот же? Ну, как назвать… То же существо или… Не знаю. Мы себя называем людьми, а вы как себя называете? Ну, если… Как мы можем понять, кроме как ``предохранители''?}
\soul{Мы тоже можем назвать себя людьми.}
\people{Можете рассказать подробнее о вашей среде обитания?}
\soul{Говорите.}
\people{Нет . А вы … У вас тоже Солнце есть? Луна?}
\soul{Да.}
\people{Деревья? Цвет какой?}
\soul{Всё красное.}
\people{Всё красное? И Солнце - красное? То есть – смещённое… А скажите, деревья у вас какого цвета?}
\soul{Всё – красное.}
\people{Всё - красное…}
\people{У вас других цветов нету?}
\soul{Есть, но это редко.}
\people{А небо? Не голубое?  Тоже – красное?}
\soul{Нет… Небо больше похоже на синеву, но… скорее, наверно, прозрачное…}
\people{На небе есть звёзды в ночное время?}
\soul{Нет, звёзд мы не видели.}
\people{Не видели?}
\soul{Это ваш удел.}
\people{А вы знаете о том, что мы видим звёзды, да?}
\soul{Да, конечно.}
\people{Скажите, а у вас … ну, кто…хм-м… Ну, будем говорить так – кто вами управляет? Ну, то есть вот…}
\people{У вас есть общество, сообщество, коллектив?}
\soul{Да, конечно.}
\people{У вас техника какая-то есть? Или что-то…}
\soul{Техники нет.}
\people{Вы живёте естественно, да? То есть, мы вот… у нас…}
\soul{Как  - ``естественно''? Мы тоже воюем. Так же, как и вы.}
\soul{Какими орудиями вы воюете? Пики, стрелы, мечи у вас?}
\soul{О, нет!}
\people{Руками да? Или какими своими способами?}
\soul{Нет. Мы просто вспоминаем противника и стараемся его изменить.}
\people{В лучшую или в худшую? То есть, что бы его убить, да? Если убить можно.}
\soul{Не-ет.}
\people{А как?}
\soul{Убиваете вы! Мы стараемся – просто изменить и сделать подобным себе.}
\people{А-а-а!}
\soul{И тогда - он мне уже не враг.}
\people{Вон как… А вы имеете на это право? Или вы, просто, тоже вот, как и мы… своенравные, будем говорить так…}
\soul{Вы знаете, меняются времена… Когда – можем, когда – нет. Меняется власть – меняются и законы.}
\people{У вас есть государство?}
\soul{Да.}
\people{И правитель? Как он называется? Президент? Царь?}
\soul{Наверно, назовём вашим именем.}
\people{Президент, да?}
\people{Или, лучше – царь? Или – король?}
\soul{Ну-у… мы не знаем различия.}
\people{У вас есть семья, семьи?}
\soul{Да.}
\people{У вас по одному мужу или жене, да?}
\people{Моногамия?}
\soul{Ну… мы, вообще-то не имеем таких связей, как вы.}
\people{А какие?}
\people{А семейных, таких, связей нет, как у нас  – всю жизнь жить, да?}
\soul{Нет, у нас есть семьи. Но у нас нет того способа…}
\people{Рождения, общения, так что ли?}
\soul{Да.}
\people{Вы не физически общаетесь друг с другом?}
\soul{Мы просто хотим ребёнка.}
\people{Так…}
\people{И он появляется, да?}
\soul{Нет. Мы должны его долго, сперва, рисовать в себе. Выбирать самые лучшие качества. Потом – приходим к королю или - вы говорите ``царю”…}
\people{Так.}
\soul{Он смотрит нас и разрешает. (теряется контакт.прим.)}
\people{1…}
(переход контакта на разговор с Мабу)
\soul{Петя!}
\people{Петя… Петя… Пети нет, но мы… Вот, как ты голос…}
\soul{Опять нет?}
\people{Нет. Есть другой.}
\soul{А когда же он бывает?  Уже прошло три урожая, а его всё нет.}
\people{А ты с ним… Он тебе был очень памятен? Он тебе был очень интересен, да?}
\soul{Если б я с ним не разговаривал, у меня не было б земли, и не было бы трёх жён.}
\people{А ведь ты не только с Петей разговаривал, но и со мной!}
\people{Это недавно ты стал разговаривать?}
\people{У тебя уже три жены? }
\soul{Недавно… }
\people{Тебе дали, да?}
\soul{Уже прошло две ``тьмы''!}
\people{О-о! Тебе дали за то, что ты с Петей разговаривал, ещё одну жену?}
\soul{Не ``дали'', а я выбрал!}
\people{А-а… Выбрал…}
\people{А она – молодая?}
\soul{Зачем мне старая?}
\people{Красивая?}
\soul{Зачем мне уродка?}
\people{А ты что, хочешь сейчас уже и четвёртую жену получить? Будет это у тебя?}
\soul{Нет.}
\people{Пока  хватит?}
\people{А зачем тебе столько много жён? Что бы они работали?}
\soul{Чтоб они меня кормили.}
\people{А-а! И ты много ешь?}
\people{Ты - толстый?}
\soul{Не-е… Я не толстый, я сильный!}
\people{Сильный? А вы кроме как в пещерах, никуда не ходите?}
\soul{Почему? Монахи водят нас на реку.}
\people{И всё?}
\people{Река большая? Далеко она?}
\soul{Рядом.}
\people{Она широкая?}
\soul{Мы берём воду.}
\people{Воду там берёте?}
\people{Пьёте, да? А кто ходит за водой?}
\soul{Жёны.}
\people{А в чём воду переносите?}
\soul{В шкурах.}
\people{В шкурах?!}
\people{В шкурах?}
\people{В чьих?}
\soul{А нам дают монахи. А они ``травянистые''. (покрыты шерстяным покровом. Прим.)}
\people{Травянистые?}
\people{Так она же вытечет оттуда! Почему она не вытекает?}
\soul{Не-е-е… Не вытекает.}
\people{Не вытекает? Монахи сами делают шкуры такие, да,  что воду можно носить?}
\soul{Да.}
\people{А вам часто приходится ходить на реку за водой?}
\soul{Это проблема жены.}
\people{Это они? Вот, тебя обещали сделать старейшиной,  скоро это будет?}
\soul{Не знаю. Я знаю только, что  они никогда не нарушают своих обещаний. Но, хотя, уже перестали бить.}
\people{Тебя бить перестали? А рисовать… Ты что-нибудь нарисовал ещё?}
\soul{Даже стали со мной разговаривать!}
 
\people{И о чём они говорят?}
\soul{Ну, они стали давать мне имя.}
\people{Как? Какое имя? Как они тебя называют?}
\soul{Моба.}
\people{Моба…}
\soul{Моба. Моба!}
\people{Мы рады, что у тебя хорошее имя.}
\people{Ты всё ещё рисуешь на камнях?}
\soul{Рисую.}
\people{И что ты теперь рисуешь?}
\soul{Я нарисовал старейшину, ему понравилось, и он мне решил подарить… э-э…}
\people{Что подарить?}
\soul{В которую можно смотреться.}
\people{А-а! Зеркало! У нас называют это – зеркало. Это очень большое для вас… Ценность, да?}
\soul{Если б у меня было зеркало, я был бы богатый!}
\people{А-а. Скажи, а что ты ещё-то научился рисовать на камне? Ну-ка, расскажи.}
\soul{А мне приносили чего-то, а я должен был её нарисовать.}
\people{Ну, и ты нарисовал?}
\soul{Потом, меня водили в священную пещеру, я должен был там всё запомнить и нарисовать.}
\people{И что ты запомнил? Там интересно было для тебя?}
\soul{Непонятно.}
\people{Непонятно? Там были какие-то картины или сосуды стояли? Что там было?}
\people{Там на камнях тоже было что-то нарисовано?}
\soul{Да. Только очень мелко и я ничего не понял.}
\people{А тоже - одним камнем? Или там разные цвета были?}
\soul{Цвета?}
\people{Ну, вот, как Солнце – одно, это цвет – светлый, а ночь – это тёмный цвет, а камни – это жёлтый цвет. Как там? Разные были..?}
\people{Расскажи, Моба, что тебе поразило в пещере?}
\soul{Меня?}
\people{Да. Что поразило?}
\soul{Вот, в чём мы возим воду… А там есть такое, что можно видеть воду.}
\people{Видеть?}
\people{Воду? В чём?}
\soul{Как она там лежит.}
\people{Прозрачная? Она прозрачная? Сосуд?}
\people{А-а. Как она в воде она лежит.}
\soul{Твёрдая вода, да.}
\people{А-а…}
\people{Твёрдая вода?}
\people{Стекло, стекло, стекло. (к Белимову. Прим.)}
\people{А ты не щупал, она холодная может, нет?}
\soul{Э-э! Меня бы тогда били, если б я трогал.}
\people{Трогать нельзя. Что ещё тебя поразило? Что ты расскажешь своим соплеменникам, когда станешь старейшином?}
\soul{Ой, когда я стану старейшиной… Скорей бы уж! (с вожделением. Прим.)}
\people{Что ты расскажешь тогда? Что ты людям своим…}
\soul{А что разрешат монахи, то я и расскажу.}
\people{Ну, про пещеру… Что ты там ещё запомнил? Ну, ладно, прозрачная вода. Как она где-то хранится, - что ещё? Там были повозки?}
\soul{Что-то такое – громкое…}
\people{Громкое?}
\people{Звук?}
\soul{Подходишь, говоришь тихо-тихо…а она громко-громко. Даже уши болят.}
\people{У-у… Аппарат какой-то.  Может это от инопланетян осталось? У них же прилёты были. (Шёпотом Ольге)}
\soul{И много-много меня.}
\people{А-а! Зеркала.}
\people{Зеркала, ясно.}
\soul{Не-е… Когда говорю.}
\people{А-а! Эхо! - Много тебя.}
\soul{Не-е… Эхо у меня в пещере живёт, а там другое! Там – большое, а у меня маленькое.}
\people{А-а… А в пещере этой видно небо?}
\people{Там светло?}
\soul{Светло.}
\people{А что там, огни горят что ли? Небо не видно там, в пещере?}
\soul{Небо не видно.}
\people{Небо не видно, а светло, да? А что там? Огонь горит?}
\soul{Нет.}
\people{А как? Просто светло и всё?}
\soul{А я не знаю.}
\people{Просто светло и всё? Просто светло, да?}
\soul{Светло.}
\people{А у тебя в пещере темно?}
\soul{Темно, если нет огня.}
\people{А огонь у вас, как… Огонь всё время? А чем вы огонь…ну… питаете? Чем питается огонь?}
\soul{Старейшина приносит воду. }
\people{И наливает, да?}
\soul{Когда я буду старейшиной, я буду носить! Хочу – принесу, хочу – нет.}
\people{Ой, зачем же так? Нужно всем. Всем нужно.}
\soul{Что бы меня все любили.}
\people{А ведь любят не за то, что бы ты наказывал без причины.}
\soul{Да, конечно! Если б мне старшина не давал, я б его и не любил.}
\people{А-а!}
\people{Скажи, а вот эта вода, которую ты будешь скоро иметь, как старейшина,  она имеет запах? Плохой запах?}
\soul{Нос болит.}
\people{Она прозрачная эта жидкость?  (шёпотом.)Может спирт или что…}
\people{Тёмная? Прозрачная? Долго горит?}
\soul{Да. Вода! }
\people{Вода. И долго горит она?}
\soul{Долго.}
\people{А..у…дым  чёрный идёт от неё?}
\people{Дым идёт? Нет?}
\soul{Не-е…}
\people{Или только свет? А  ты знаешь, что такое дым?}
\soul{Облако.}
\people{Правильно.}
\people{Облако, всё правильно. Ну, чёрный дым или белый?}
\soul{Нет дыма!}
\people{Нет дыма вообще?}
\people{Скажи, а вот ты не с нами разговаривал, – с Петей разговаривал, тебе третью жену дали, а теперь ещё что тебе обещают дать?}
\soul{А я теперь сам.}
\people{Что  - ``сам''?}
\soul{Разговариваю.}
\people{А-а… Ты сам с нами… А монахов нет?}
\soul{Нету.}
\people{А-а. Ты сам захотел с нами разговаривать?}
\soul{Мне нравится.}
\people{Тебе нравится. Скажи…}
\people{А, вот, сколько детей у тебя уже? У тебя был один. Одного ребёнка забрали, сколько сейчас у тебя детей?}
\soul{Один.}
\people{Так один и остаётся? Он уже подрос?}
\soul{Я жену только взял.}
\people{У тебя ж было три! (к Ольге)Или уже перерыв, видимо? Тогда другой был.}
\people{Скажи, вот, как ты с нами разговариваешь? Что ты сделал, что бы с нами разговаривать?}
\soul{Ложусь на шкуру, которую мне дал монах.}
\people{Ага. Да. И дальше?}
\soul{Укрываюсь ею.}
\people{Да.}
\soul{Ауба берёт огонь.}
\people{Ага…}
\soul{Подносит. Я смотрю на него, а потом разговариваю с вами.}
\people{И ты нас слышишь, да, что мы говорим?}
\soul{Ну, да.}
\people{Ты нас хорошо слышишь?}
\soul{Не всегда.}
\people{А наши слова тебе всегда понятны? Или есть и непонятные слова?}
\soul{Мне много не понятно, но я откуда-то догадываюсь.}
\people{А-а…}
\people{А-а… Ну, скажи, вот слово ``маразм'' ты понимаешь, что это такое? Нет?}
\soul{У нас есть один такой.}
\people{Какой?}
\soul{А он всё время чего-то…}
(конец стороны кассеты)
\soul{(Моба)Я сделаю куклу жены…Буду бить кукла,  жена будет больно.}
\people{А зачем куклу обязательно бить?}
\soul{А  монахи всегда так делают.}
\people{Бьют?}
\soul{Если кто-нибудь убегает, они делают куклу …}
\people{А-аа, и?}
\soul{А потом он сам приходит…}
\people{Сам.}
\people{А бывает, что до смерти забивают? Кто-нибудь умирает у вас? Бывает?}
\soul{Умирают.}
\people{А, вот, когда человек умирает, глаза закроет и все.}
\soul{Монахи забирают.}
\people{Скажи, а вот как ты ребенка своего воспитываешь? Что ты ему говоришь?}
\soul{Они его оставили, потому что он не годный им.}
\people{А-аа, он плохо говорит? Он плохо движется? Да?}
\people{А вы разговариваете?}
\soul{А он марап.}
\people{Марап? А как это? Объясни что это такое?}
\soul{Не люблю говорить.}
\people{Урод? Маразм?}
\soul{Он мне все равно нравится.}
\people{Он тебе все равно нравится. Ну, конечно, он же родной человек, конечно. Он ползает или ходит уже?}
\people{Он ходит?}
\soul{Не-е, не умеет.}
\people{А что? Он ползает?}
\soul{Не он только сидит.}
\people{Сидит ещё? Он маленький совсем, да?}
\soul{Не он большой.}
\people{Большой? А, он не может ходить, да? Не умеет? Тебе его жалко?}
\soul{Как не умеет? У нас все умеют ходить.}
\people{А он сидит. Как сидит? Сейчас сидит?}
\soul{Да он же марап.}
\people{Тебе его жалко, значит. А почему ты себе ещё ребёнка не родишь?}
\soul{Монахи не приходили.}
\people{Монахи отличаются от вас внешним видом? Они что, лысые, с волосами, с бородами?}
\soul{Не знаю. Они в шкурах.}
\people{В шкурах? Монахи в шкурах одеты?}
\soul{Да. Только без травы. }
\people{Без травы….А скажи…}
\soul{Видеть их нельзя.}
\people{Вася, а вот скажи…}
\soul{А зато я их уже слышал два раза!}
\people{А что они тебе сказали?}
\soul{А они говорят только с при…при…при…}
\people{Что?}
\soul{Ну,  когда ты им нравишься.}
\people{Ну, ты им, видимо,  понравился? Ты им нравишься, наверное, да? Они часто тебя к себе берут?}
\soul{Они даже водили меня в священную…}
\people{Пещеру?}
\people{И что же ты там запомнил? Про воду?}
\soul{Твердая вода, гром…}
\people{Гром был? Свет?}
\soul{А чё - свет? Светло и светло, а чё тут такого?}
\people{Скажи, а что там ещё в пещере…полы…что там когда ходишь – какое оно – гладкое, или как камни?}
\soul{Противное. Я там два раза падал.}
\people{Скользкое?  Запинался?}
\soul{Ну, как после дождя.}
\people{Скользко значит. А стены? Стены влажные, скользкие?}
\soul{Ой, я не дотрагивался.}
\people{Ну, какие они, гладкие стены, как пол?}
\soul{Не-е.}
\people{А какие?}
\soul{Разные.}
\people{Скажи, Вася…или Моба…}
\people{Моба.}
\people{Ты часто выходишь на разговор с другими? Часто ложишься около костра? Ещё с кем -нибудь разговариваешь? Кроме Пети.}
\soul{С вами.}
\people{А ты знаешь, как нас зовут?}
\soul{Не-е.}
\people{А почему ты с нами разговариваешь?}
\soul{Мне нравится.}
\people{Тебе интересно, да?}
\people{А сколько ты голосов слышишь? Какой голос слышишь?}
\soul{Как у жены.}
\people{А ещё, какой голос слышишь? Как у тебя, да? Нас двое, как ты думаешь?}
\soul{Чего ``двое''?}
\people{Нас двое с тобой разговаривают? И жена и муж - два.}
\soul{Не понял.}
\people{Ну, вот я жена, да? Я – жена.}
\soul{Моя?}
\people{Нет, не твоя.}
\soul{А как я могу разговаривать с чужой женой?}
\people{Нельзя с чужой женой разговаривать?}
\soul{Меня за это будут бить.}
\people{Скажи, Моба, что ты у нас хочешь спросить?}
\soul{Моба!}
\people{Моба, извини. Что ты нас хочешь спросить?}
\people{Может тебе интересно - какая у нас жизнь, как мы живём? Ты никогда не спрашивал у нас. И почему мы с тобой разговариваем. Тебе интересно?}
\soul{Не знаю. Я не думал ещё.}
\people{Ну, подумай. }
\people{О чём ты нас хочешь спросить? }
\people{Как мы выглядим, хочешь спросить?}
\soul{Монахи  говорили, что как они.}
\people{Правильно, да, как они. Только мы одежду не носим такую длинную. Мы ходим, как ты, в повязке. Так же. А ещё что? Как монахи говорили? Что говорили о нас монахи?}
\soul{Они говорили, что вы колдуны.}
\people{Что мы колдуны? А колдуны что делают?}
\people{А колдуны для вас плохие или хорошие?}
\soul{Ой, я не знаю.}
\people{А что делают колдуны? }
\people{Вы их боитесь колдунов? Что они могут сделать с вами?}
\soul{Они могут зажигать воду…}
\people{А-а… }
\people{Да.}
\soul{… а мы не можем. Ещё, они могут делать шкуры.}
\people{Да, всё правильно.}
\soul{А ещё, у них есть палки, они ими нас колют, а мы потом долго болеем. А они говорят,  чтобы не болели. Дураки. (наверно им делали прививки.прим)}
\people{А палки не большие такие, блестящие? }
\people{Как солнце сверкают?}
\soul{Не-ет. Как камень.}
\people{Как камень? }
\people{Ну, вам больно потом после этих уколов? }
\soul{Мы потом болеем.}
\people{У вас кровь потом течёт?}
\soul{Нет. Кровь не течёт.}
\people{А куда они вас бьют? Куда они вас колют?}
\soul{В руку и сзади.}
\people{Сзади – это на чём сидишь, да?}
\soul{Нет. Под рукой.}
\people{А они не говорят, зачем это вам делают?}
\soul{Чтобы не болели.}
\people{А вы всё равно болеете, да?}
\soul{Если б не кололи, мы бы не болели.}
\people{У вас большое племя?}
\soul{Племя?}
\people{Да.}
\people{Ну, сколько людей рядом с вами живёт в пещерах? Много пещер?}
\soul{Я вас не обзывал.}
\people{А как вы себя называете? Вы люди?}
\soul{У нас так зовут животных.}
\people{Каких животных? Они же только у монахов.}
\soul{Мы- то их видим…}
\people{А как вы себя называете, соседей, деревню?}
\soul{Не знаю.}
\people{Ну, много у вас  пещер? Много, вот, где люди живут?}
\soul{Много.}
\people{А где они? А, вот, камни высокие, большие? До неба?}
\soul{Да. Я пробовал подняться, а меня поймали монахи и побили.}
\people{А там, наверху, где камни, там что? }
\soul{А там, говорят, Бог живёт.}
\people{А там, на верху большого камня есть светлое что-то?}
\soul{Что там есть?}
\people{Ну,  вот, на верху, белое, как зеркало, есть что-нибудь?}
\soul{Небо.}
\people{Небо и всё? Камни и небо, да? А у них острые вершины?}
\soul{А что такое ``острое''?}
\people{Ну, как укол. Вот, когда вам делают… укалывают вас, это ``острое'' называется.}
\soul{Это надо пойти посмотреть…}
\people{А в пещере ты на чём спишь?}
\soul{Шкура.}
\people{А раньше спал, когда тебе монахи шкуру не давали?}
\soul{Как спал… Так и спал.}
\people{На  камне, да?}
\soul{На камне.}
\people{На камне? И травы даже не дают?}
\soul{Трава только на шкурах.}
\people{А-а…А песок? На песке-то мягче спать. Удобнее. Лучше.}
\soul{А что такое ``песок''?}
\people{Песок – это мелкие-мелкие камни. Это, как зерно на камне разбиваешь зёрна… Разбиваешь, а  она такая белая, да?}
\soul{Красное.}
\people{Красное? А потом… Потом - ешь. А это -  можно камни так мелко-мелко тоже размолоть и на них можно спать, если их много-много будет.}
\soul{А зачем, когда можно спать просто на большом камне?  На нем даже удобней.}
\people{Да? Ну, да, вообще-то.}
( сбой контакта)
(на связи другой персонаж)
\soul{Удалось…}
\people{Что удалось?}
\soul{Анис, я уже слышу их…}
\people{Кого?}
\people{Скажите подробней, кого вы слышите? Нас?}
\soul{Вас.}
\people{А кто вы?}
\soul{Я? Анур.}
\people{Анур? }
\people{Анур? И ты хотел с нами связаться и у тебя вдруг сейчас получилось?}
\soul{Анис пришёл.}
\people{И что сказал?}
\soul{Сказал, что будешь сейчас разговаривать.}
\people{Ты в это не верил, а теперь убедился, что получилось.}
\soul{Почему ``не верил''? Анис никогда не врёт.}
\people{У тебя первый раз получилось?}
\soul{Да.}
\people{Ты хочешь  нас о чём-то спросить? Задавай вопросы.}
\soul{Хочет Анис.}
\people{Анис? Так.}
\people{А может он тебе говорить, а ты - нам?}
\soul{Он мне пытался объяснить, как всё происходит, чтобы мы могли друг друга понять. Он говорит, что у вас другой язык и совершенно другие понятия.}
\people{Всё правильно.}
\soul{Но, слава Гею, Анис достаточно умён, чтобы заставить меня говорить с вами.}
\people{Кто ты? Национальность какая?}
\soul{Что значит ``национальность''?}
\people{Народ какой? }
\people{Какая нация, племя? Где ты живёшь?}
\soul{Здесь я живу.}
\people{А у вас там тепло?}
\soul{Да.}
\people{У вас стоят пирамиды или горы? }
\people{Жилища у вас какие?}
\soul{Круглые дома.}
\people{Круглые? }
\people{А как вы себя называете?}
\soul{Люди.}
\people{Как? }
\people{Люди? Вас много?}
\soul{Да.}
\people{Сколько? Сотни, тысячи?}
\soul{Около восьми.}
\people{Чего?}
\people{Человек или тысяч?}
\soul{Тысяч.}
\people{У вас большое село? Большой город такой?}
\soul{Нет. Нас окружает море.}
\people{Море окружает?}
\people{Вы на острове живете? }
\people{Оно теплое, море?}
\soul{Да.}
\people{Вы плаваете по нём на кораблях?}
\soul{Нет, кораблей у нас нет, но мы можем плавать.}
\people{Как? }
\people{Вплавь? На чём, на плотах?}
\soul{Нет, сами.}
\people{Сами? Купаетесь?}
\soul{Мы неплохие пловцы.}
\people{И далеко вы можете плавать?}
\soul{На соседние острова.}
\people{И рядом много островов?}
\soul{Много.}
\people{Ага. Они богаты растительностью?}
\soul{Да. Мы добываем там пищу. Наш остров скуден. На нём стоит большой город, просто негде расти. Один дворец занимает всю вершину.}
\people{О! У вас дворцы есть?}
\soul{У нас есть дворец. Зачем нам множество дворцов? У нас один правитель.}
\people{А простые люди живут где?}
\soul{В стенах дворца.}
\people{Он большой дворец? Огромный?}
\soul{Да, он занимает всю вершину.}
\people{А как зовут правителя?}
\soul{Адонис.}
\people{Адонис? Он хороший правитель, он умный?}
\soul{Конечно.}
\people{Он пожилой, или молодой?}
\soul{Нет, он молодой. Он всего лишь только второй год, как мы выбрали его.}
\people{Вы выбирали его?}
\people{А сколько ему лет? У вас есть счёт?}
\soul{Да.}
\people{Сколько ему лет по вашему счёту?}
\soul{Двадцать восемь.}
\people{А тебе сколько?}
\soul{Тридцать один.}
\people{Ты женат?}
\soul{Нет.}
\people{А почему не женат?}
\soul{Слава  Гею, повезло.}
 
\people{А что это плохо? Женитьба тебя как-то сковывает?}
\soul{Я знаю много друзей женатых, это все равно, что быть скованным.}
\people{Скажите, а вы знаете на какой планете вы живёте?}
\soul{Земля.}
\people{Земля? Она  круглая?}
\soul{Кто?}
\people{Земля круглая?}
\soul{Не знаю.}
\people{А у вас есть солнце и луна?}
\soul{Конечно.}
\people{(Ольга Белимову)Конечно.}
\people{(Белимов Ольге)Нет, луна не всегда была. (к Ануру)Скажите, а из чего сделан ваш дворец?}
\soul{Камень.}
\people{Обработанный, белый?}
\soul{Нет, больше красный.}
\people{Вы его обжигаете, или делаете из скал, вытачиваете?}
\soul{Нет. Середина дворца – это скала, только обработанная, а снаружи мы брали металл и камень и соединяли их.}
\people{А металл, какой?}
\soul{Металл смерти.}
\people{А как, как название?}
\soul{Говорят, в других странах за него убивают.}
\people{Золото наверное? Неужели золото!}
\soul{Я не знаю.}
\people{Скажи, а какого цвета у вас небо?}
\soul{Красное.}
\people{А солнце?}
\soul{Красное.}
\people{Все красное? А звезды у вас есть?}
\soul{Конечно.}
\people{И тоже, красные?}
\soul{Почему же?}
\people{А какие?}
\soul{Они, просто, яркие.}
\people{А цвет, какой?}
\soul{Есть и синий, есть и красный, есть и белый. Моя звезда, как говорит Анис, зеленая.}
\people{А сколько лет Анису?}
\soul{Сорок два.}
\people{Он уже взрослый. А чем вы занимаетесь, ты и Анис? Какие у вас профессии, что вы умеете?}
\soul{Я?}
\people{Да.}
\soul{Я леплю горшки.}
\people{Делаешь горшки?}
\soul{Я учусь делать их.}
\people{Из глины или из металла?}
\soul{Глина.}
\people{У тебя красиво получается это?}
\soul{Пока не очень.}
\people{А Анис, что делает?}
\soul{Он приближенный Гею.}
\people{Объясни нам, пожалуйста, кто такой Гей?}
\soul{Бог.}
\people{Да, так он, получается, как монах? Жрец?}
\soul{Да.}
\people{Он  умный, Анис?}
\soul{Я не встречал глупых жрецов.}
\people{А скажи, как вы с ним нашли общий язык? Ты ж горшки, вроде, мастерил. }
\people{Вы встретились как-то? Почему он понял, что ты умеешь выходить на контакт?}
\soul{Он пришёл ко мне и сказал, что вылечит меня.}
\people{Ты болел?}
\soul{Да.}
\people{Чем?}
\soul{Я ``терял себя''.}
\people{На время уходил куда-то?}
\people{Сознание терял, это у нас так называется. Отключался, да? То есть, не помнил ничего?}
\soul{Да. Я терял себя, когда спал, а когда просыпался, то не мог ничего вспомнить, и всегда было больно.}
\people{И говорил  необычные вещи, наверное, да?}
\people{И чего-нибудь рассказывал?}
 
\soul{Я же не помню и не знаю.}
\people{А жрец?}
\soul{Он пришел и сказал, что вылечил меня.}
\people{И что дальше?}
\soul{Теперь, я ученик.}
\people{У него?}
\soul{Нет, я леплю горшки. Зато я уже не теряю  себя.}
\people{Теперь ты уже не теряешь себя. А раньше ты чем занимался?}
\soul{Ничем. Я болел.}
\people{У тебя были родители? Есть родители?}
\soul{Есть.}
\people{Они богатые люди? Или чем занимаются?}
\soul{Ну, у нас такого понятия – богатый, бедный.}
\people{У вас почти все одинаковые? Кто работает, тот что-то…?}
\soul{Ну, немножко не так. Есть приближенные, есть белые…}
\people{Цвет кожи у них белый?}
\soul{Нет, мы называем их так – белые, потому что они занимаются нефизическим трудом. Они пишут стихи, рисуют картины, сочиняют оды, развлекают нас. Есть зеленые…}
\people{Кто это?}
\soul{Это такие, как я.}
\people{И у вас основное занятие какое?}
\soul{Зеленые?}
\people{Да.}
\soul{Это больные люди. Их, просто…Заботятся о них, как заботились и обо мне. Кормят, поят и стараются научить.}
\people{Но их не уничтожают?}
\soul{Что значит  ``уничтожать''?}
\people{Ну,  не убивают, как ненужных?}
\soul{Это было бы глупо.}
\people{Было бы глупо, да?}
\people{Вы воюете, с кем-нибудь, когда-нибудь? Войны бывают?}
\soul{Нам не с кем воевать.}
\people{Вы мирно живёте?}
\soul{А с кем воевать нам?}
\people{А соседние острова?}
\soul{Соседние острова? Там есть черные, но они слишком слабы, чтобы воевать с нами.}
\people{А у вас разве есть какое-то оружие, или вы мысленно можете других пугать?}
\soul{Монахи… как вы говорите… Жрецы или монахи… как вам будет точнее? }
\people{Так, и что они?}
\soul{Они, говорят, могут. Но я ни разу этого не видел.}
\people{А вот, сейчас Анис, что он вам сказал? Чтобы разговаривали с кем-то, или как?}
\soul{Он накрыл меня туникой и долго давил на глаза.}
\people{И что дальше?}
\people{Потом,  у вас получилось, и вы вышли на нас.}
\soul{Сперва  мне было больно, я видел огни.}
\people{Какие? Разные?}
\soul{Разные. Потом, он сказал: ``видишь голубой?''. Я сказал: ``вижу''. Тогда он перестал давить и сказал: ``Открой глаза, но чтобы голубой остался''. Я открыл, я вижу его.(Аниса.прим) И его. (цвет.прим)}
\people{И голубой свет?}
 
\soul{Конечно.}
\people{Опиши его, Аниса. Какой он из себя? Мы его не видели. }
\people{(Ольга Белимову тихо) Сейчас голубой вокруг него.}
\people{Ты не можешь описать его? (к Ануру.)}
\soul{Он выше меня.}
\people{Ну,  примерно, рост какой вот? }
\people{У вас есть метрическая система – два метра, метр..?}
\soul{Я ещё только начинаю учить.}
\people{А, вот, примерно сколько локтей?}
\soul{Локтей?}
\people{Да. Ну, вот рука до локтя, сколько примерно?}
\people{Он высокий человек?}
\soul{Девять, десять.}
\people{А ты?}
\people{Пониже?}
\soul{Ну, где-то на локоть и ещё половину.}
\people{Понятно. Спасибо. А цвет кожи, какой у него и тебя?}
\soul{Красноватый.}
\people{А растительность на голове есть? Борода?}
\soul{Бороды нет. А волосы есть.}
\people{А волосы, какие? Кудрявые? Прямые?}
\soul{У него?}
\people{Да.}
\soul{Белые.}
\people{Белые?}
\soul{И кудрявые.}
\people{А-а… Белые, как золото, как метал?}
\soul{Да. }
\people{А у тебя?}
\soul{У меня грязные волосы.}
\people{Черные? Тёмного цвета?}
\soul{Нет.}
\people{Просто грязные? Не моешь?}
\soul{Не-ет, зачем же. Цвет грязи.}
\people{А, ясно цвет тёмный.}
\people{А глаза у вас, какие? Цвета какого? Разные бывают цвета? Нет?}
\soul{Я, вообще-то, видел мало людей своей болезни, но у всех, кого я видел, они голубые.}
\people{Голубые глаза… А у вас женщины красивые?}
\soul{Да.}
\people{Они высокие тоже? Пониже мужчин?}
\soul{Нет, есть и выше.}
\people{А у них волосы длинные или короткие?}
\soul{А это трудно понять. Они их собирают в пучок и прячут.}
\people{Под что прячут? У них косынки какие-то, головные уборы?}
\soul{Да.}
\people{Какие?}
\soul{Красные.}
\people{Не цветом, а, вот, форма… Форма какая?}
\soul{Как мой первый горшок.}
\people{А-а…А одежда, какая? }
\people{Он ушёл…}
(срыв) 
1-2-3…
\people{Скажите, сейчас, видимо благодаря вам, мы несколько этапов  жизни переводчика узнали. Скажите, вот, самый первый этап – когда он был с тремя жёнами, не мог считать, и  когда помогали ему  монахи познавать мир, скажите, какое время было на Земле? Это очень раннее, да?}
\people{Ну, примерно.}
\soul{По вашему, да. Но, благодаря вам, он добился даже семи жён. }
\people{Семи жён!?}
\soul{Сбылись его мечты. Как говорит, его сделали стариком.}
\people{Старейшиной?}
\soul{Нет.}
\people{Стариком?}
\soul{Стариком. Это значит, дать жить больше, чем другим.}
\people{То есть, благодаря контактам с нами, он стал великим человеком по-своему, так?}
\soul{Да.}
\people{Благодаря своей внутренней сути, наверное? }
\people{Но он сам тоже оказался необычным, умным человеком? }
\people{Это тот, который сейчас перед нами лежит? Который переводчик наш - это он?}
\soul{Это уж сами гадайте.}
  
\people{Хорошо, спасибо большое!}
\soul{Для чего же мы всё устраиваем, чтобы потом всё разжевать? Нет смысла в спектакле.}
\people{А можно ли так сказать, что вы – это наше будущее? Вы – это ``будущее'' нашего переводчика? Можно так сказать?}
\soul{Относительно вас? Нет. Это было бы скорее ``настоящее''.}
\people{Скажите, с монахом, с первым человеком, что за пещеру ему показывали? Это действительно была пещера, кладезь каких-то знаний, может быть, инопланетных?}
\soul{Он будет ещё разговаривать с вами. Будьте внимательны, мы же говорили вам о семи жёнах. }
\people{Прекрасно!}
\soul{Его и расспросите.}
\people{Мы так и хотим! Вот, нам надо заранее знать, какое это время. Это что, начало цивилизации человеческой?}
\soul{Думайте и решайте сами. Побудьте в наших ``шкурах''.}
\people{Да, да, мы понимаем, как вам с нами тяжело разговаривать, именно на этом примере, когда хочется иной раз и бросить разговор, потому что он наивен. Наверно и мы такими представляемся вам, к сожалению. Но давайте иметь терпение друг к другу, нам очень интересно всё равно общаться. Но, скажите, по его мнению – наши боги приходят, по его мнению, откуда-то, и прилетают и уходят… Это – восприятие…}
\soul{Всё спросите у него.}
\people{Хорошо, ладно. Тогда, следующий этап – с кем Анур?  Это, что за времена, хотя бы?}
\people{Это Атлантида, если по-нашему?}
\people{И мы хотели бы об Атлантиде Анура очень подробно расспрашивать, у нас есть  чего спросить. Мы, просто, сейчас не успели.}
\soul{Вы должны были бы заметить,..}
\people{Что цвет одинаковый, да? Красный.}
\soul{… что все, разговаривающие с вами, в какой-то мере были больны.}
\people{Да-да.}
 
\soul{И это их отличало от общей массы, потому они и могли разговаривать. Вы  должны были знать из истории, что монахи всегда выделяли их.}
\people{Да.}
\soul{И последние, как вы говорите, двести лет, уже не нужны больные, потому что уже монахи не ищут их. И вы растёте, всё-таки.  Что такое ``больной'' в вашем понятии? Это не могущий  себя контролировать, а раз так, значит, он более восприятен к внешнему. Вы согласны?}
\people{Согласны, да.}
\soul{И это помогает и облегчает говорящим, но вы растёте. Вы говорите, что не помните, но не помните только вы. Есть память, а значит, и есть знание.}
\people{Но сейчас напрасно монахи отошли от такой практики? Они, видимо, себя ограничивают в чём-то, в  познании? Они зациклились? Догматически воспринимают?}
\soul{Монахи потеряли свою власть.}
\people{А-а, они уже стали не теми монахами, которыми были раньше? Раньше  монахами были люди, которые более, ну, знающие…}
\people{Просвещенные. Они несли знания…}
\soul{Раньше - были монахи, а сейчас, просто - ``одевающие их одежды''.}
\people{Ну, ясно.}
\people{Просто, они потеряли, видимо, своё значение для человечества, как полезное?}
\soul{Нет… Просто, монахами становится кто угодно. Тот же, кто должен стать им, он не может, потому что его оттолкнули. Потому, что для вас теперь, сила важнее. Сила! И способность заплатить!}
\people{Вот как. Да, все правильно, этим мы отличаемся в наше время.}
\people{Вот скажите, мы так поняли, что если раньше только редкие больные могли выходить на такие контакты и являлись интересными информационно для монахов, то сейчас становится большее количество людей, способных на это выйти. Мы знаем, даже в нашем городе достаточно много контактеров, которые что-то получают, какую-то информацию. }
\soul{У вас множество способов войти в контакт. Один из них – болезнь, другой, это когда, как вы говорите, не выдерживают нервы, когда жизнь бьёт вас и вы уже на грани срыва. Тогда, контакт может быть предохранителем или, наоборот, погубить вас, но это уже одержимость. Здесь такие тонкие грани, что нельзя уже определить. И каждый из вас, в принципе, болен. В принципе, каждый из вас уже одержим. И все мечты ваши, вы стараетесь их выполнить любыми путями. Разум ваш  - уже одержим, потому что возомнил себя и сделал себя владыкой мира. Потому, вы говорите о Боге, как о пустом. Бога вы понимаете как разум, сверхразум и даже не можете вспомнить о чувствах. А что для вас чувство? Это что-то ``низкое''. Вдумайтесь, ведь это становится так.}
\people{Да, точно.}
\people{Но это, конечно, деградация человечества, это наше упущение. Это плохо.}
\soul{Это не деградация, это одна из ступеней, разветвление. А куда вы пойдёте? Какой путь выберете? Да, вы можете погубить. Мы, когда то говорили вам, когда вы спрашивали о конце света. Конец света вы представляете обязательно физически, и не знаете, что каждый из вас однажды уже умирал, а вы это называете -  перелом.}
\people{Перелом…}
\people{Разве? Мы такую терминологию не применяли…}
\people{Не знаю, вы, может быть, обо мне говорите? Я…}
\soul{Мы говорим, о всех. }
\people{О всех…О людях, вообще. В общем, да?}
\soul{Каждый из вас, можно сказать, умирал. В каждом из вас - двое, трое, десять. Мы приводили вам примеры, какие вы бываете в автобусе, какие вы бываете дома, какие бываете на работе. И вы не можете найти себя. Приходит время, когда вы сбрасываете все эти шкуры и становитесь самими собой. Очень трудное время, потому что вам очень трудно разговаривать с другими, потому что вы чувствуете ихнюю ложь, ихние шкуры. Вот вам – ``переломный возраст''. А потом, вы одеваете новые одежды…}
\people{И новое по-новому, да?}
\soul{Правильно. Вы же придумали: `` В волчьей стае по-волчьи выть''.}
\people{Угу…}
1…2…
\people{Что? Оно яркое, светлое?}
\soul{Погреться, понежиться…. И встретить восход его…. А сейчас, будь оно проклято,  жгёт и жгёт…}
\people{Очень жарко? Почему? Приблизилось светило к вам? }
\soul{Когда же будет конец-то…}
\people{У вас нет воды даже? Вы не можете скрыться от солнца? А в пещеру, почему не уйдёте в пещеру? }
\soul{Миражи…}
\people{Мираж?}
\soul{Миражи…}
\people{Нет, это не миражи.}
\soul{Я потерял дорогу. Хотел увидеть горы…  мираж…}
\people{Какой мираж?}
\soul{Они же говорили, пустыня обманчива. Дурак…}
 
\people{У тебя воды не осталось, да? Ты бредишь?}
\soul{О, я уже и голоса слышу… Нормально…Сколько ж мне ещё то осталось?}
\people{Ты давно ушёл из города в пустыню? Сколько ты идёшь?}
\soul{Интересно, с  кем я разговариваю? Сам с собой?}
\people{Нет.}
\soul{Да-а, похоже, я уже идиот.}
\people{Нет, нет… Ты в пустыне один? Мы тебе поможем.}
\soul{Галюники мне помогут! Нормально… Мне бы воды…}
\people{Верблюд есть? }
\soul{Воды бы мне…}
\people{(между собой: Это у него галлюцинации)}
\people{Это не галлюцинации, мы люди, мы тебе поможем! Мы слышим тебя. Надейся! Ты выживешь.., Слышишь? Ты выживешь! Ты еще немного пройдешь и увидишь воду.}
\soul{Солнце… }
\people{Ты давно ушёл из города, скажи нам?}
\soul{С кем это я..? Может даже и лучше… Вроде как не один помру.}
\people{Как тебя зовут?}
\soul{Что я, уже самого себя не помню?}
\people{Скажи нам. Мы другие.}
\soul{Сергей…}
\people{Сергей? А фамилия? Иванов?}
\soul{Точно… сам с собой.}
\people{Ты в пустыне, ты воюешь? На войну ушёл?}
\soul{Нет, любопытничать пошёл.}
\people{Сергей, не переживай. Это нормальное явление, тебе плохо не будет. Вспомни, вспомни, мы уже с тобой разговаривали. Это было в 22 году, 18 мая, вспомни. Сегодня, какое число, день, месяц?}
\soul{Да, действительно уже теряешь… Уже всё поперепутал.}
\people{У тебя был друг Андрей? Ты учился на РАБФАКЕ?}
\soul{Я даже не знаю, что такое РАБФАК.}
\people{Скажи, какой сейчас год? Скажи вслух.}
\soul{Двенадцатый.}
\people{Двенадцатый год?}
\people{Так ты ещё маленький.}
\soul{Хм, маленький…}
\people{Сколько тебе  лет?}
\soul{Одиннадцать.}
\people{Одиннадцать…}
\people{Где ты живёшь-то, в каком городе? }
\soul{Деревня…}
\people{Тебя же отец в три года привёз в Москву. Мать у тебя умерла, ты это знаешь?}
\soul{Мать я не знаю.}
\people{А отец?}
\soul{Отца я помню. Но я помню его только постоянно пьяным, и он меня постоянно гонял. Потом он меня как-то выгнал из дома, и я пошёл. Благо лето было, лапти с собой взял да ушёл. Потом меня кто-то подобрал. }
\people{Тебе было горько, жалко, ты голодный был?}
\soul{Да, похоже, я действительно умираю, если вспоминать начинаю. }
\people{Нет, ты выживешь, У тебя просто сегодня состояние такое, ты не здоров сегодня.}
\soul{Меня подобрали… а что было дальше-то?}
\people{Тебя увезли куда-то? Мальчик, ты в какой пустыне сейчас?}
\soul{Меня подобрал…Чем же я болел-то?}
\people{Чем ты болел? У тебя было отключение сознания?}
\soul{Какое-то слово интересное забыл.}
\people{Оно большое слово, длинное? Эпилепсия?}
\soul{Там был доктор, он бил по щекам и говорил: ``Ты будешь жить, только начни улыбаться''.}
\people{Всё правильно! Он правильно говорил. Вспомни болезнь!}
\soul{Потом, меня отправили в детдом. Это я помню. Но я не могу вспомнить, сколько мне было лет. }
\people{Скажи, ты чем болел: холера, чума, высокий жар у тебя был, да? }
\people{У тебя болезнь часто повторяется, или ты  один раз болел?}
\soul{Нет, я заболел ей один раз. Я помню… в деревне меня хотели сжечь.}
\people{За  что?}
\soul{Чтобы я никого не заразил, Это я помню.}
\people{Дифтерия?}
\people{Цинга?}
\people{Чума? Или холера?}
\people{Тиф?}
\soul{Что же было дальше?..}
\people{Расскажи, что было дальше? Рассказывай. Больше рассказывай.}
\soul{Метрика… У меня было две метрики. В одной мне поставили пятый год рождения, а во второй поставили прочерк.}
\people{Почему?}
\soul{А я ничего не помнил от этой болезни. Я помню только отца, - был пьян.}
\people{А кем он работал?}
\soul{Я помню, помню… Я помню себя ещё молодым.}
\people{Маленьким?}
\soul{А когда доктор разбудил меня и вылечил, я уже был старый. Мне даже кличку-то дали ``Старик''.}
\people{А почему?}
\soul{У меня был седой волос и старые глаза.}
\people{Хотя ты ещё был маленький по возрасту?}
\soul{Ну, по одной метрике мне был пятый год, по другой прочерк. Уж как тут скажешь, какого я всё-таки года. Знал бы я ту деревню, мог бы увидеть этого отца. Интересно, искал ли он меня?}
\people{А сейчас, ты где заблудился, что за пустыня-то?}
\soul{Интересно, с кем я всё-таки  разговариваю?}
\people{Ты разговариваешь с людьми, которые тебе хотят помочь.}
\soul{Что-то я не вижу людей.}
\people{Но слышишь ведь. Ничего страшного, так бывает.}
\soul{Это уже галлюцинации…}
\people{Нет, это не галлюцинации, это правда. Что ты нам хочешь рассказать? Расскажи что-нибудь, тебе будет легче. Мы выведём тебя к воде, выведем к людям. Расскажи. }
\people{У вас война идёт с германцами? С немцами.}
\soul{Я  помню… Ганса.}
\people{Ганса? Немца?}
\soul{Да, он был немец. Он мне показывал какой-то порошок и говорил, что он победит нас – русских.}
\people{А-а…Нет, он русских не победит. }
\people{И ты поверил ему, что он победит?}
\soul{Я видел, как действует этот порошок.}
\people{Как он действует? Он горит?}
\soul{Нет.}
\people{Взрыв?}
\soul{Нет, просто он  его поджёг, мы убегали, он на меня надел какую-то маску, а потом все животные умерли. Я тогда, помню, сильно испугался, схватил у него банку и бежал. Я хотел её сбросить в реку, а он меня догнал и избил.}
\people{Сильно избил? }
\soul{Это я помню…}
\soul{Банку отобрал? Успокойся, мы знаем, что нас немец не победил.}
\soul{Кого - нас?}
\people{Русских. Ты же русский?}
\soul{Наверное.}
\people{Если ты Сергей Иванов, значит ты русский?}
\soul{Имя мне дали в детдоме.}
\people{Это не важно. Но ты же в России живёшь.}
\soul{Вообще- то я москвич.}
\people{Ну, вот, а как ты в пустыню-то попал?}
1..2..
\people{Кто? Кукла?}
\soul{…было бы не плохо конечно, но вряд ли. Знаешь, Андрей, у меня есть одна идея….}
\people{Какая идея?}
1..,2..
(сбой контакта)
\soul{… кто-то оговорился…}
\people{Что?}
\soul{Да я не верю этому.}
\people{Чему? Говори, говори!}
\soul{Толька?}
\people{Говори! Ты кто?}
\soul{Подожди-ка. }
\people{Ну. Спрашивай.}
\soul{Из-за этого стеклянного шарика!?}
\people{Ты слышишь нас?}
(сбой контакта)
\soul{…Алё,  барышня! Я не слышу, алё!…Сергей! Алё!…}
\people{Говори, говори, тебя слышно. Я – Сергей! (попытка Белимова втянуться в монолог переводчика.прим)}
\soul{… А какая мне разница? Лишь бы жизнь нормально можно было… }
\soul{…какая глупость. Вам что, делать больше нечего? Ну, что из-за этого изменится…?}
\people{А зачем убивать себя?}
\people{Ты прав, ты правильно думаешь.  Это ничего не решило и не решит.}
\people{Говори. }
\people{Заблудился?}
\soul{Вот железная дорога и иди по ней. Зачем метаться? Будет интересно, вернёшься, посмотришь. А если бегать туда- сюда, всё равно всё забудешь и попутаешь.}
 (Конец записи)
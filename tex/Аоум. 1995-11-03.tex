Аоум. глава 27-я 03-11-1995г
Георгий Губин
  
Аоум. глава 25-я
Георгий Губин
  Оно разрешает ответить ``да'' и позволяет сказать ``нет''.
  Оно даёт свободу свободным и снимает всякие обязательства с любви.
  Оно распахивает окно после того, как захлопнется последняя дверь.
  С ним связаны все приключения и авантюры на свете, вся радость, слава и смысл жизни.
  Оно приводит в действие буксующий двигатель эволюции.
  Оно свивает из шепотов и вздохов кокон для гусеницы.
  Его произносят молекулы перед тем, как соединиться в цепочки.
  Оно отделяет мёртвое от живого.
  Его не способно отразить ни одно зеркало.
  В начале было слово, и слово это было: ВЫБОР!
                                                      Том Роббинс.
 
\people{**}
\people{Мы собрались почти через год, после того, как был наш последний сеанс. Мы не знаем,  как этот перерыв подействовал на вас, но для нас, это была пауза, во время которой мы хотели подумать над тем, что мы от вас получили, какую информацию, переварить и так далее. Не знаю, полезным этот перерыв был… Возможно, не полезным. Вы слышите нас? Можно задавать вопросы? }
\soul{Часы и зеркало. Тем более, вы знаете, о зеркале, и не могли догадаться.}
\people{У меня эта мысль мелькнула. Оно слишком сильно мешает? Может быть, мы успеем сейчас убрать?}
\soul{Далее.  Часы. Ритм задаёт ``переводчик'', а не часы.}
\people{Значит, сейчас убираем часы. Зеркало, наверное, сложнее будет убрать.}
\soul{Чего вы боитесь? И почему вы лжёте? Лжёте даже себе?}
\people{Зеркало можно убрать? Закрыть можно его? (между собой. прим.)}
\soul{Спрашивайте далее.}
\people{Хорошо.  Скажите,  пожалуйста, как вы чувствуете, перерыв в наших сеансах был благоприятен для нас и, немного, для вас?}
\soul{Вы, про себя, спрашиваете у нас.  Почему же вы не спросите самих себя?  Хотя бы, даже если солжёте, вы, всё-таки, попытаетесь ответить.  Мы говорили вам, что мы никогда не покидали вас. И лишь только вы - слышите или не можете.}
\people{Ясно. У нас набрались вопросы от наших коллег из других городов, и, в частности, мы сегодня хотели бы посвятить сеанс вопросам Вадима Черноброва  из г. Москвы…}
\soul{Мы говорили вам, о именах.}
\people{А… Извиняюсь. Мы больше не будем упоминать. Вот такой вопрос, он, правда, правда, поначалу, не участвовал в сеансах, поэтому будут повторяться вопросы.  Его интересуют - кого вы представляете: цивилизацию, общество или отдельных индивидов?}
\soul{Мы говорили вам, что мы ваши дети и ваши родители. В этом мире, нас нет. Вы придумали новые и называете их ``виртуалы''. Ну, пусть будет так.}
\people{А какие цели преследуются вами на земле, если они всё же преследуются?}
\soul{Вы имеете привычку преследовать. Вы, обладая инстинктом самосохранения, уничтожаете себя, а значит, и нас.}
\people{То есть, мы уничтожаем и вашу цивилизацию?}
\soul{Если мы – это вы,- мы можем быть отдельны?}
\people{А вот, есть ли цели у вас, при общении с нами, с конкретными людьми, которые в этой комнате?}
\soul{Вы зовёте – мы приходим и отвечаем вам.}
\people{Тогда, следующая группа вопросов, которая волнует уфолога. Это о норме, о доверии и доли дезинформации. Можно ли всецело доверять вашим сообщениям и, вообще, вам?}
\soul{Нельзя доверять ничему. Давайте договоримся. Мы ведём беседы, но не уроки. Далее, мы говорим не ``мы''. Мы не можем говорить – это говорит ваш переводчик. Почему мы не оперируем цифрами? Почему мы не употребляем слова, которые знает переводчик? Почему? Потому, что мы не хотим тревожить его память. Мы не хотим, как вы говорите - ``гипноз'', когда человек полностью подвластен кому-то. Нет, мы просто показываем картинки, и он их рассказывает. И потому, много, в вашем понятии, будет дезинформации.}
\people{То есть, это из-за того, что словарный запас используется - не правильно понимается какая-то картинка. Только из-за этого, да?}
\soul{Мы  находимся на поверхности его. Мы не можем и не имеем права входить, иначе - это будет насилие. Потому,  мы не можем отвечать на многие вопросы, связанные с логикой, когда переводчик должен мыслить. Тогда он будет отвлечён и потеряет нас. Если же мы войдём глубже, он не сможет нас потерять, но это будет уже не он.}
\people{То есть, это может повредить его психическое здоровье? Помешать ему, да?}
\soul{Спрашивайте.}
\people{А считаете ли вы, что ваша информация будет нами воспринята адекватно, т.е. правильно? Мы подготовлены для получения такой информации? Как ваш вывод за этот год?}
\soul{Если вы будете говорить, что напрасно беседуете, значит, вы обижаете и нас, и себя.}
\people{Нет, оно конечно, не напрасно. Понимаем ли мы адекватно, так, как вы бы хотели, чтоб мы вас поняли?}
\soul{Мы не добиваемся, чтобы вы поняли нас. Мы добиваемся только одного - чтоб вы начали думать.}
\people{Скажите, а, вот, разрешаете ли вы использовать полученную от вас информацию в каких-либо благих целях? Или вы накладываете определённые ограничения на это?}
\soul{Вы когда-нибудь замечали их? }
\people{Нет. Но это задаёт вопрос другой человек. Мы не видели ограничений, и мы, действительно, могли публиковать и ответы, и вопросы. }
\soul{Это ваши проблемы. Это будет на вашей совести. Только и всего. Применяйте, как хотите. Но, есть одно ``но''. Когда мы не можем дать такой информации, которая могла бы повредить не лично вам, а кому-то другим.  Лично вам, -  нас это не сильно волнует, ибо вы сами пришли и согласились. Зачем должны страдать другие, в вашем понятии?}
\people{Как я понял, были моменты, когда вы могли бы что-то сказать, и эта информация могла помешать кому-то, да? Такие моменты бывали? Понятно. Тогда, вопросы о пространстве-времени. Что такое, для вас, пространство времени? Сформулируйте это.}
\soul{Вашим языком?}
\people{Хотя бы - нашим.}
\soul{Есть несколько формулировок. Множество. И все они не верны, и все они верны. Вы забываете только одно, представить себя в этом пространстве и времени. Вы отделяете себя от всего мира. И потому, любое ваше представление, о пространстве будет ложным, хотя, вы являетесь частью этого пространства. Но вы забываете об этом. Время? Что в вашем понятии ``время'', если вы живёте в разных временах? Когда вы ожидаете – сколь томительно идёт, когда вы торопитесь – сколь быстро оно спешит! А вы говорите ``время''… Время – это, всего лишь скорость вашего восприятия мира в пространстве. Только и всего. Возьмите часы, которые будут идти во множество раз дольше, но вы об этом не будете знать, и вы будете жить в 2 раза дольше по этим часам. И наоборот.}
\people{Это - уже астрология…астрономия…}
\soul{Зачем же. Зачем? Знает ли  ребёнок, о часах? Знает ли он, о времени? А столько много он получает информации, и сколь долго он живёт по отношению к вам! Его 5 минут равны вашим 10-ти, а, бывает, и годам.}
\people{Ну, это субъективное понятие.}
\soul{А что такое ``время''? Для вас, весь мир субъективен. Вы измеряете мир - чем? Только тем, что вы напридумывали. Вы согласны, что все ваши приборы, все ваши науки – это выдуманное вами? И в природе вряд ли существуют они.}
\people{Угу. Скажите, а можно ли пространство рассматривать отдельно от времени? }
 Они, действительно, независимы друг от друга во всех случаях?
\soul{Они всегда независимы.}
\people{Всегда независимы.}
\soul{Мы же говорили вам: пространство не меняется, а вы изменяете время. Теперь представьте себе - пространство - и нет времени. Что это? Это и есть  мечта одной из религий.}
\people{Вечный рай? }
\people{Скажите, а является ли пространство-время четырёхмерным, как принято нашей официальной наукой?}
\soul{Это ваша мера. И мы вам только что говорили об этом.}
\people{Оно не мерно?}
\soul{Давайте скажем так: весь мир, в вашем понятии, субъективен. Весь мир ложен, ибо лжёте вы - сами себе. Вам лгут глаза, вам лгут ощущения. И чтобы вам было легче жить с этой ложью или хотя бы упорядочить её, вы придумали линейку и так далее. Вот он - ваш мир, измеренный вами, и, в то же время, не понятый.}
\people{А есть какой-то выход из этого положения, которое мы сами напридумывали?}
\soul{Не быть ``машиной''.}
\people{А вы считаете, мы, всё-таки, близки к этому? К машине, к роботу, да?}
\soul{Нет, вы близки более к человеку. Ибо вы уже ведёте себя, как машина.}
\people{Мы даём себе задания просто-напросто.}
\people{Скажите, а если говорить, о мерностях пространства-времени, - они равнозначны между собой?}
\soul{Нет. }
\people{А у времени есть мерность?}
\soul{Вы, сперва  поймите, что хотите спросить.}
\people{Да, тут сложно, потому что вопрос задавал учёный другого профиля. Тогда,  вопросы чисто о времени. Что такое, по вашей формулировке, время?}
\soul{Для вас – это скорость восприятия окружающего мира. И оно никогда не бывает постоянной. Вы скажете: ``Как же так? Мы имеем одни часы, и они всегда верны на всём земном шаре''. Да, ЧАСЫ – верны. Но, неверно ваше восприятие. А вам нужны не часы,- нужно ваше понятие времени. И, пока вы не откинете всю ту механику, вы не сможете найти, что такое ``время''. И вы не сможете никогда изменять её тогда, когда вы хотите и насколько хотите. Вы этого всего хотите добиться механикой. Да, скоро будет время, когда только чистой механикой вы сможете перемещаться. Но, будет ли это перемещением? }
\people{То есть, время не имеет мерности, по вашему мнению? Да?}
\soul{Нет.}
\people{Но, ведь оно задаётся вращением Земли вокруг Солнца. Вот оно и ``время'' получается. Круг обернулся – вот и получаем ``время''.}
\soul{Да. Потому что вы сделали меру - вокруг Солнца. А почему бы вам не взять время полёта Галактики на любой определённый угол? - и вы будете называть ``временем''. Назовите - ``год социума''.  Привыкните к этому. Разве многое изменится?}
\people{А, вот, имеет время, как (имеется гипотеза), скорость хода или плотность?}
\soul{И то, и другое.}
\people{Имеет?}
\soul{Конечно.}
\people{А как же ``плотность''? Объясните.}
\soul{Плотность? Это можно объяснить на примере ребёнка. Когда ребёнок может воспринять информации столь много или столь мало, при одном и том же интервале времени. Вот вам и ``плотность''. Далее…}
  1-2-3-4
\people{Скажите, а имеет ли время ``длительность''? }
\soul{Если считать вашими науками, оно имеет все измерения. И время, также, может иметь множество измерений, как вы говорите, ``в пространстве''.}
\people{Не обязательно ``секунды'', да?}
\soul{Ну, давайте скажем так: вы - трёхмерны в мире, но время есть и трёхмерное, и больше, и меньше,- если брать вашими мерами.}
\people{Скажите, а время имеет другие характеристики?}
\soul{Все. Время, в вашем понятии - время – это материя. Это, всего лишь, вид существования материи. Почему же материя не может иметь все параметры? Вы, время, можете измерять чем угодно!}
\people{Так что, изменяя время можно изменить и материю?}
\soul{Нет, вы будете изменять материю – и тогда будет изменяться время. А если точнее – ваше понятие о времени. Вы же нам только что говорили о движении Солнца и Земли. Давайте изменим орбиты! И, в вашем понятии, изменится время. И, поверьте, ваши секунды уже не будут те.}
\people{А скажите, время обратимо или нет? Оно только вперёд движется, вектор его?}
\soul{Вы можете перемещаться в пространстве?}
\people{Можем.}
\soul{Почему же нет?}
\people{Значит, и время может?}
\soul{Только вы придумали ``прошлое'', ``настоящее'' и ``будущее''. Вы вспомните прошлый контакты. Мы говорили вам, о времени, о его понятии. И говорили, что, для  нас, нет времени. Мы говорили вам, что мы живём столь малое мгновение, что вы ещё не придумали этих цифр. И, в то же время, наша жизнь столь длинна, сколько ещё не было Вселенной. Вам нельзя это понять, ибо вы меряете количеством. Вы говорите: ``Много ли параметров времени?'', а пользуетесь только одним, - ``количество''. И всё.}
\people{А будущая наука придёт к пониманию? Уже приходит или нет?}
\soul{Приходит. Но, опять же, - физика.}
\people{А это - внутреннее понимание? Работа внутренняя, будем так говорить?}
\soul{Да, но, чаще, вы под ``внутренней работой'' понимаете ``лень''.}
\people{А скажите, время является причиной энтропии; старения, разрушения, увеличения хаоса?}
\soul{Нет.}
\people{А что? Может быть, энтропия является причиной…}
\soul{Вспомните, мы же говорили вам, что является вашей старостью. Забывание! Но, согласитесь, если обладать хорошей  памятью, то столь долго надо будет жить, чтоб забыть! А есть и дети, что стареют уже …}
  (счёт)
\soul{Зеркало…}
\people{То есть,  мешает зеркало?}
\people{Закрыть надо? Убрать можно? (убирают зеркало)}
\people{Всегда ли выполняется во времени принцип ``стрелы времени''? То есть, направление - только вперёд?}
\soul{Мы же говорили вам: вы задавали только что вопрос, и задаёте снова, только другими словами. Вы можете перемещаться во времени  сколь угодно и как угодно. Когда вы научитесь владеть действительно временем, а не только его понятием. Мы же говорили вам, что время для вас - есть. И, даже для вас, вы можете перемещаться вперёд и назад. И когда вы перестанете быть тем, кем вы являетесь сейчас, вы вообще не будете спрашивать  и не будете вообще понимать, что такое время, потому что вам это будет уже не нужно.}
\people{А это придёт? Такими будем?}
\soul{Будем надеяться.}
\people{Вот скажите, а вот существуют ли процессы, текущие в обратном направлении времени?}
\soul{И вы снова повторяетесь.}
\people{Угу. Ладно…}
\people{А можно назвать это ``инволюцией''?}
\soul{А что такое ``эволюция''?}
\people{Инволюция.}
\soul{И что такое ``инволюция''? Это придумано только вами. Только и всего.}
\people{Ну, термин…  Ну, мы же общаемся…}
\soul{Давайте скажем так: деградация… Что в вашем понятии - ``деградация''?}
\people{Рзарушение. Хаос.}
\soul{Прекрасно. Но, когда-то, вы  говорили, о ``космических весах''. Если где-то что-то разрушается, значит, где-то что-то на другом полюсе,- наоборот. Это можно назвать деградацией?  Или, может быть, проще - изменение … (теряется)   Спрашивайте.}
\people{Скажите, пожалуйста, какие сейчас чувства переживает ``переводчик''? Что мешает ему?}
\soul{Вы, и он.}
\people{Мы мешаем невнимательностью к ответам? Несобранностью какой-то?}
\soul{Зачем же? Он чувствует вас, а вы боитесь.}
\people{Отвыкли уже от контактов, наверное.}
\soul{Мало того, что у вас страх, вы ещё почему-то ощущаете стыд. Задавайте вопросы.}
\people{Хм. Хорошо, а можно ли самому двигаться во времени с такими скоростями без помощи технических средств?}
\soul{Да. Вы делаете первые попытки во снах.}
\people{Ага…А двигаться во времени с помощью техники? Удаётся? И будет возможно?}
\soul{Будет. Но мы и не завидуем  вам, о тех временах. Далее, вы не можете себя представить без плоти, и поэтому, нужна физика. Для того,  чтобы что-то где-то сделать и сказать: ``да, я там был'' - вам нужна плоть. Почему?  Но, ведь человек  - это не плоть и не кусок мяса. ЭТО  - в вас. А вы хотите перенести тело! Ну, зачем? Зачем? Когда вы идёте куда-то, вы берёте с собой груз? Вы выбираете только то, что вам нужно. И полёты во снах, вы считаете нереальностью. И хотите создать технику, которая могла бы перемещать ваши тела. Ваши оппоненты утверждают, что ``вы не были ни на каких планетах, ибо мы только что видели, как вы лежали или сидели в кресле. Куда же вы тогда летали? ``  Вам надо перемещать плоть. Но зачем? Сколько трудов, сколько боли и несчастья принесло вам только то, что вы не можете отделить себя от плоти?}
\people{Скажите, а зачем тогда существует проявленная Вселенная, будем так говорить, и физический человек?}
\soul{Вот и живите в этой Вселенной и понимайте, для чего вы живёте в ней. Простите, к вам пришёл Иисус  Христос – в вашем понятии Бог, - и не смог доказать кто он. Неужели это сможем доказать мы? И сможем ли мы объяснить вам? Сможем? Если вы сами не знаете, как объяснить, хотя бы то понятие, которое вы …. }
\people{Да-а… Ещё вопрос нашего коллеги.  Всё-таки, для Землян время состоит из ``прошлого'', ``настоящего'' и ``будущего''. Эти понятия равнозначны, по сравнению друг с другом?}
\soul{Нет. Можете сказать ``настоящее''? Это столь короткий миг, что нельзя сказать, о прошлом, о будущем, и о настоящем. Вы можете говорить о прошлом только тогда, когда это уже прошло, когда вы уже можете вспомнить. Настоящее?  Назовите мне мгновение? Будущее?  Назовите мне ``будущее''. И где тогда живёте вы, если ``в прошлом'' - уже прошло, если настоящего - нет, ``будущее''  ещё не пришло? В каком времени вы живёте?}
\people{Парадокс. А вы думаете, в каком времени мы живём?}
\soul{Мы вначале говорим вам, о виртуалах….  (в комнате  лает собака) Ваш страх останавливает нас. Был страшен не лай, а было страшно то, что вы боялись, что этот лай может остановить.}
\people{То есть, любые эмоции передаются и вам, и переводчику? Настоящее, это есть процесс перехода в будущее и прошлое, или это есть стабильный момент времени?}
\soul{Мы, вам - только что спрашивали,- отвечали. ``Прошлое'' в вашем понятии – то, что будет. ``Будущее'' – что будет. А вы можете предсказывать ``будущее''?}
\people{Ну,  мы бы хотели этим овладеть и, кстати, это одно из интересных проявлений психики.}
\soul{Хорошо, вы можете предсказывать прошлое? Именно предсказывать?}
\people{Нет.}
\soul{А вы пытаетесь и постоянно это делаете. Когда вы что-то вспоминаете, вы это вспоминаете с таким количеством подробностей, которых не было никогда. Вот - ваши фантазии. И вот вам  - изменение прошлого. Вот вам - и виртуальность прошлого. Настоящего, его нет, и это говорили только что мы. Будущее? Вы умеете фантазировать и там. И создаёте себе будущее. Тот же виртуал. И ваше счастье, что вы живёте столь плотно друг с другом, что вы живёте, всё-таки, в одном мире, а не во множествах различных и были бы одиночкой.}
\people{Получается, что прошлое одновариантно. То есть, неизменно. Так?}
\soul{Нет! Прошлое вы меняете, как хотите. Всё зависит только от ваших воспоминаний и от политики. Вы говорите ``история''? Ну, тогда вспомните сколько раз она менялась, хотя бы на ближайшие года.}
\people{Да, это верно.}
\soul{Ну, так, где же настоящее и прошлое?}
\people{Ну, тогда будущее тем более многовариантно?}
\soul{Давайте скажем так: вы создали машину времени. И вы хотите попасть в прошлое. По вашим представлениям и по подготовке к этому прошлому вы, всё-таки, будете читать литературу. Хотя бы приодеться к этому времени. Вы попадёте именно в то прошлое, которое представляете, именно в тот вариант мира, который вы знаете.}
\people{Но не в реальное прошлое?}
\soul{Нет. Ибо миры ваши столь параллельны и столь одинаковы и расхожи лишь только в мелочах. В мелочах, которые делаете вы. Вы создаёте множество миров. Вспомните старые контакты. Вы можете сейчас сидеть, можете сказать одну фразу, можете сказать другую, можете вообще уйти. Вот вам ``множество будущего''. Вы согласны? Хотя, ``прошлое'' было одно. Это - всего лишь ``настоящее''. Это -  возможность изменения вариантов ``будущего''. А если быть точнее, ВЫБОР одного из вариантов будущего. Вот вам – ``настоящее''. А в прошлое вы не можете вернуться именно в то, которое было для вас истинно, которое, вы именно, прожили, ибо вы не сможете настроиться.  Вы говорили, о вращении Земли. Вы создаёте … (теряется).}
\people{Спутники?}
\soul{Вы создаёте корабли. Вы хотите свою плоть перенести на одну из планет.  А не проще ли было бы настроиться на неё? Пускай даже на ложные её понятия, пусть хотя бы так. Хотя бы в её прошлое, то, которое придумано вами. Но вы не сможете его придумать полностью. Ибо, всё-таки, есть и пространство. Пространство вы не можете изменить столь сильно, как вы балуетесь со временем. И, придя в это прошлое, пространство будет корректировать вас.  Как бы будете сканировать по этим параллелям, пока не найдёте, как вы говорите ``резонанс''.}
\people{Скажите,  пожалуйста, всё-таки, будущее - оно предопределено или оно многовариантно?}
\soul{Настоящее – это выбор будущего. Вариантов. Мы вам говорили это только что.}
\people{Но предопределено с определённой точки?}
\soul{"Предопределено'' - только тогда, когда вы пошли к гадалке и вам сказали: ``будет то-то и то-то''. Да, она сыграла роль бога. Она вас направила по своему пути. Поэтому говорили  вам: ``лучше не знать будущего, ибо вы будете идти только одной дорогой''. Она может быть и не верна.}
\people{Так свободы выбора не существует? Будущего.}
\soul{Мы же говорим вам, что никогда никто не отнимал у вас свободы! Вы всегда были свободны. И никогда вы не были рабами или кем-либо. Вы свободны даже в выборе, быть вам богом или не быть! Хотя, вы вроде бы уже и есть, судя по вашей жизни, но вы выбираете. Вы объявляете себя судьями, объявляете себя богами. И тут же говорите, что нет свободы. Как вы можете совместить такие вещи? Не проще ли сказать так: что вы столь ленивы, что даже обвинить себя не хотите! ``Мне было так предопределено''. ``Я не виновен''. Так вы хотите сказать?}
\people{Бывает это. Да. Лишь бы не бороться. Скажите, а можно ли путешествовать во времени без помощи технических устройств?}
\soul{Вы должны к этому стремиться.}
\people{А мы стремимся пока, чтобы с техникой путешествовать?}
\soul{Вы же не можете расстаться с плотью. Вам обязательно нужно перевезти свою плоть. А когда вы во снах – вы преодолеваете множество пространств… Вы просыпаетесь – ``чепуха'' - и вся ваша реакция.}
\people{Ну а с техникой,  путешествия,  во плоти, не имеют перспектив или, всё-таки, есть перспективы?}
\soul{Ищите. Вы должны идти своей дорогой. Если мы вам скажем ``не надо'' или ``надо'',  ``давайте сделаем так и по-другому'', простите – вот тогда вы - рабы. И вы будете следовать нашим советам. Сперва,  вы будете сомневаться. Иногда, будете делать по-своему, не следуя нашим советам. Но это скоро перейдёт к вам в привычку. Тогда вы будете спрашивать, о каждом шаге у нас - что делать вам. Нужно ли нам это?}
\people{Нет. Мы, как раз, вот, год перерыва был, чтобы не закабаляться, не зомбироваться…}
\soul{Не лгите себе.}
\people{А выход для человечества из создавшегося положения есть? Мы, действительно, должны понять что-то, наверное, чтобы избавиться от технократии, что-ли…}
\soul{Да, тут не надо устраивать революций и крушить всё подряд. Вы любите переходить в крайности. ``Или давайте думать только над техникой, или давайте будем голыми в пещерах. Но, только давайте все это сделаем сразу''.}
\people{Мера должна быть во всём, наверное? Или технические должны быть…}
\soul{Вы растёте. Вы не можете заставить ребёнка делать то, что он ещё не умеет, ибо он ещё не вырос. И когда вам говорят: ``вы недостойны'', или  подобное,  отлучают вас от церквей, простите, но вы растёте! И поэтому нельзя говорить, что вы грешны, или вы злы, или вас надо уничтожить, или отлучить от церквей. Кто вы, отлучающие?  Кто?  Вы говорите: ``по подобию Божьему''. Вы – подобие, да, вы – Бог.  А остальные, раз вы их отлучаете?}
\people{Скажите. Вот, целый цикл интересных вопросов. Дальше…Существует ли возможность оказать воздействие на прошлое, на ход истории?}
\soul{Да. Здесь фантасты придумали множество вариантов. Но, не учли одно ``но''. Изменив прошлое, конкретно вы, вы уже  не попадёте в то будущее, или, если хотите - ``настоящее''.}
\people{То есть, есть какие-то ограничения,  на  такого  рода  воздействия? Может быть, типа космического закона?}
\soul{Законы - только у вас. У вас , у людей. Какие могут быть законы у Вселенной, которая просто живёт? Как она может сама себе создавать законы? Вы можете сами себя заставить дышать/не дышать,  или двигать клетками только строго по закону, который придумаете вы? Зачем? Вы, просто живёте.}
\people{Скажите, а возможно ли спасение людей, погибших в прошлом, если это очень хочется?}
\soul{Если вы можете изменять прошлое, почему же нет? Но мы же говорили вам: вы не попадёте в ту реальность, из которой пришли. Вы попадёте, в другую. В этой реальности человек никогда не погибал в прошлом, он всегда жил и умер своей смертью. Вы поняли нас? }
\people{Ну, так… Смутно.}
\people{Зачем всё это делать?}
\soul{Ну, почему же? Есть множество из тех времён.  Да, вы хотите кого-то спасти – вы уходите из настоящего в прошлое, спасаете, и возвращаетесь.  И, вот, во время возврата, вы уже забываете то настоящее, которое было ``первым'', из которого прибыли. И в этот мир, вы уже приходите, где этот, вами  спасённый человек, никогда не погибал, и где историю уже тысячи лет изучают о том, как вы спасли этого человека!}
\people{М-да… Словом, лучше в прошлое не соваться из будущего…}
\soul{Ну почему же? Для вас это будет не вредно совсем. Для любого из вас.}
\people{Почему?}
\soul{Ну, потому что вы попадёте в то время, в котором вы будете считать, что вы в нём всегда и жили. Вы сами переходите множество миров и не замечаете этого. Как можно сказать, что вы – именно Вы, именно сейчас, и именно в этом настоящем времени? А, может быть, совсем недавно был другой? Из другого варианта. Вы не исключаете это?}
\people{М-да… Существуют ли параллельные миры? Вопрос уже говорился, но, всё-таки…}
\soul{Множество. Есть миры, которые не относятся к вам. Множество никогда не соприкасаются с вами никак. И никогда не встречали и никогда не имели общих точек. Но, всё-таки, большинство миров - имеющие общие точки. И это связано со временем.}
\people{Со временем именно, да?}
\soul{Мы ж говорили вам только что, что вы можете, в настоящем, выбрать один из вариантов будущего. Ну, вот это, настоящее, и будет точкой общей истории,- давайте скажем так, в терминах фантастики,-  дальше -  идёт расхождение этих миров. Два мира, имея лишь общую точку, стали отличаться на что-то одно. Или вы сказали одно слово или сделали другое движение, а дальше - лавина. И, потом, эти миры расходятся, и становятся столь неузнаваемы, что придя в тот мир, путешествуя, вы не узнаете себя, и будете говорить ``инопланетяне''.}
\people{Ага…}
\soul{И столь трудно вам будет найти ту общую точку, где вы были едины. Но, так как вы не одни, - параллельных миров множество. Множество тех, которые созданы не вами, а другими. И не землянами, а кем-то другими. Их столь большое множество, что здесь нельзя говорить о цифрах – нет в этом смысла.}
\people{Но, получается, что они образуются и исчезают постоянно?}
  
\soul{Давайте так, - бесконечно.}
\people{Бесконечно.}
\soul{И, в то же время, каждый из этих миров - конечен. Он имеет начало, он имеет и конец.}
\people{А параллельные миры отличаются друг от друга какими-то параметрами именно времени? Да?}
\soul{Нет, пространственными изменениями во времени. Мы же вам объясняли только что. Вы сделали одно движение, а не другое. Это больше относится не ко времени, а больше - к изменению ``пространства'', если хотите, ``во времени''. Вы можете менять пространство во времени. Но не можете сделать наоборот. Вы не можете изменить время в пространстве, хотя постоянно занимаетесь этим. Мы говорили вам, - когда вы ждёте или торопитесь.  Но это столь маленький  интервал, что он не может дать больших изменений в вашем мире. Далее. Хаос. Кто-то не торопится, а кто-то другой ждёт. И всё это компенсируется. Вот если бы торопились все вы,- все,- тогда бы вы могли это изменить! А так – хаос. Беспорядочное движение времени вперёд-назад, вперёд-назад.}
\people{Можно задавать вопросы?}
\soul{Спрашивайте.}
\people{А скажите, возможна ли связь между этими параллельными мирами? Перемещения, перелёты между ними? Или это только в сознании, получается, что можно?}
\soul{И в сознании вы, чаще, делаете это хаосом, случайностью. Вы  не можете руководить своими снами. И бывают сны столь бестолковые, что  вы не можете понять, что это вообще было?  Это, чистый хаос. Нельзя сказать, что вы во снах постоянно перемещаетесь. Нет. Есть просто игра, ``мусор'' в вашем понятии. Но давайте скажем так: мозг выполняет свою работу. Он очищает себя, приобретает что-то новое, приводит себя в порядок и так далее. Да, вы когда-нибудь создадите способ. Но, правда, не в параллельные миры. Вам, сперва, надо научиться перемещаться в пространстве так, как вы хотите, - свободно. Вы же, всего лишь только - привязаны, скованны цепью гравитации. Согласитесь.}
\people{А, вот, если я, вот, сны не вижу вообще? Это признак чего? Я забываю быстро их или …  Что это такое?}
\soul{Ничего. Если вы будете говорить что это плохо, это даст.(последствия. прим.) Если не будете - ничего. Мы говорили вам, что понятия добра и зла – это все только ваши понятия. И если к вам придут и скажут, что это очень вредно, и вы поверите в то, и, действительно, будет очень вредно. И - наоборот. Давайте скажем так: это не значит, что вы не летаете, не перемещаетесь и ничего. Это не значит, что вы не помните. Есть люди, которые действительно не видят снов, и они их не помнят, потому что они их не видели. Но вы же, вспомните прошлый контакты,- вы же сказали, что вы полгода занимались их изучением. Вспомните, мы говорили вам, что важны не те сны, которые помните или которые виделись, а внутренняя работа мозга. А те картинки, }
что показывает он вам, это всего лишь, если хотите, грубо, но если быть точнее  - это всего лишь ``реклама'', что мозг, всё-таки, работает и чтобы вы ему не мешали. Вы же ещё маленький, и ему приходится играть с вами вашими же играми.
\people{Скажите, а границы между параллельными мирами могут, как бы, утоньшаться?}
\soul{А их и нет.}
\people{У-у…Хорошо. Время на микроуровне связано со спином элементарных частиц?}
\soul{Да. Так же, как и связано Солнце с вращением планет. Но, только, мы же говорим вам, что это всё ВАШИ понятия.}
\people{Скажите, а есть ли связь времени с электрическими зарядами? С магнитным зарядом?}
\soul{Если вы создаёте очень мощное магнитное поле, то это будет трудновато вам  сделать без электрических полей. И наоборот. Да, вы можете изменить. Поверьте, изменять время – это так просто!}
\people{То есть, это может сделать сам наблюдатель? Действия наблюдателя будут этому помогать? Или какие-то другие…}
\soul{Вы читаете книгу – время летит. Вы отбросили книгу – время ползёт. Вот вам простейшее изменение. Физически?  … (теряется)}
  (счёт)
\people{Скажите, что мешает переводчику сейчас? Зеркала нет.}
\soul{Зеркала нет?(удивленно) Ладно, спрашивайте далее.}
\people{Хорошо, ещё более сложный вопрос. Время, на макроуровнях, связано с вращательными процессами. Есть такая гипотеза, что только из-за вращения.}
\soul{Мы только что говорили вам о том. И не только. Любое изменение скорости – это есть изменение времени. И наоборот, изменение времени – это есть изменение скорости. Вспомните, что такое фотон?  Это частица… или нет?}
\people{Частица.}
\soul{Частица? Или волна?}
\people{Да, или волна.}
\soul{А где частица?}
\people{Но ведь всё – материя.}
\soul{Нет, давайте будем конкретными. Фотон – это волна или частица?}
\people{Сейчас, по-новому, вроде, волна.}
\soul{Или частица?}
  (смех)
\soul{Ваши эмоции …(теряется).}
\people{Мешают. Ясно. А переводчик знает, что это волна или…}
\soul{Теперь представьте, как вы можете всё это разделить, хотя это всё вместе? Частицу, которая проявляет волновые… }
\people{Свойства?}
\soul{Как вы думаете, может быть такое?}
\people{Наверное.}
\soul{Вот вам и … (теряется ) Ваш переводчик слишком технически подкован, и он старается вмешиваться. Мы же говорили вам, что мы не имеем права ворошиться, в вашем понятии в ``чужом белье''. Мы находимся снаружи. Мы показываем только картинки. И все движения, – его движения. Мозг ставит защиту – мы говорили вам. Мозг – эгоист, одиночка, ибо он физический. И когда он черпает энергию со всего мира, он считает, что, всё-таки, он и только он, а всё остальное – пустяк. Это не значит, что вы мозгом выполняете дурную работу. Такого нет понятия. Только вы придумали ``дурное'' или ``хорошее''. Мозг, просто,  выполняет работу. И - верно или нет,  это будет всего лишь только один из вариантов вашего существования. Только и всего.}
\people{Скажите, а связано ли время с мыслительной деятельностью человека?}
\soul{Да.}
\people{А с деятельностью высшего разума?}
\soul{А что вы понимаете под ``высшим разумом''?}
\people{Ну, что-то над нами есть, всё-таки.}
\soul{Вы до сих пор не верите ещё, что он есть, и в то же время, ищете его реакцию.}
\people{Ну, это субъективное мнение - верить или не верить. Мы ищем объективное обоснование,  и, видимо, считаем, что он, всё-таки есть.}
\soul{Мы говорили вам, что вы должны стремиться к изменению времени именно не ``физикой''. Чем же вы можете изменять изменение ``физики''? Вот вам, пожалуйста, и мысль. Хотя,  мысль, пока,  рождается только у вас ``физикой''. Вы можете сказать скорость химических реакций, происходящих в вашем мозгу?}
\people{Нет.}
\soul{Как вы думаете?}
\people{Быстрее света, по крайней мере.}
\people{Ну, мысль - вообще в физическом мире считается самым быстрым.}
\soul{Но, вы же считаете, что больше скорости света не существует ничего? И, вдруг…}
\people{Мысль.}
\people{Ну, сейчас приходим к выводу, что нет. (что скорость света не самая быстрая)}
\people{Мыслью можно любое пространство преодолеть за мгновение.}
\soul{Да. Только не будет доказательством, что вы там были. И поэтому, у вас нет смысла преодолевать эти пространства. И потому - вы ленивы, и потому, это не умеете уже делать. Потому, что это всё будет, только для вас. Вы же - хотите похвастаться. Всего лишь только сказать всем, что ``я был там''. И если вы говорите, что вы там были и вам не верят – вы обижаетесь и начинаете сочинять, как бы вам это сделать ``физикой''. Вот вам – себялюбие. Ну, а потом, конечно, вы придумываете множество причин ``для блага человечества'' нам нужно то-то, то-то, то-то. Хотя, это, всего лишь, себялюбия или какая-то древняя забытая обида. Но вы найдёте множество красивых фраз, чтобы оправдать себя. И найдутся множество глупцов, которые и поверят вам. И вы будете вместе создавать новые }
 (обрыв)
  … исправлять. Потом, вдруг окажется, что вы бы не хотели, чтобы у вас прошлое было таким. ``Вот если бы не было того-то, того-то, то, было бы, наверное, лучше. Давайте, это исправим''. Поехали, исправили. Приехали. Опять не то. И так и будете всю жизнь заниматься только тем, что будете перекраивать. А что ж тогда будет для вас ``настоящим''? Что будет истинно, для вас? Если вы дойдёте до того, что вместо карманных калькуляторов, у вас будут машины времени. ``Мне не понравилось – давайте изменим'' - даже по пустякам. Вот прекрасное будет время.  И тогда наступит такое время, когда вы поймёте, что машины времени не нужны! И с помощью этой же машины вернетесь в то, когда первым была создана эта машина и уничтожите её. И тогда, будете жить себе спокойно, а не изменяя её по пустякам. Вы согласны?
\people{Это будет? Или…}
\soul{Почему, если б вы опоздали на автобус, то почему бы не изменить прошлое, чтобы на этот автобус попасть вовремя? Потому, что вам лишние 5 минут было б тяжело подождать. Вы согласны?}
\people{Да, такое бывает у нас.}
\soul{Вот теперь представьте, что будет. И наступит время, когда вы полностью от них откажетесь.}
\people{Да, это,  наверное, как детям не интересно играть уже с игрушками. Бросают, принимаются за новые, в конце концов, потом бросают все игрушки.}
\soul{Вы, просто,  заблудитесь. Вы, просто, заблудитесь и уже не будете знать где кто! Вы уже не будете знать настоящего, уже не будете знать отправной точки. В этом году была революция, теперь её нет. Теперь стали другими. Было - одно, стало- другое. Вы будете менять всё, что хотите, всё, что нравится.  По вкусу, по цвету, по настроению. И что это будет? Вот вам и хаос. Вот вам разрушение мира, вот вам и ``конец света''. А вы представляете, что придёт кто-нибудь и уничтожит вашу Землю. Зачем? Вы это сделаете сами прекрасно, и даже не используя оружие. Зачем? Зачем вам изобретать что-то более сложное, когда вы можете прекрасно запутать себя до такой степени, что вы просто потеряете себя, разбросаете по всем мирам, по всем временам все ваши частицы и уже не сможете это собрать? Вот вам и ``машины времени''.}
\people{Хорошо. Тут, по времени, ещё много вопросов. А не является ли время источником всей вообще энергии во Вселенной?}
\soul{Нет.}
\people{А может ли время питаться энергией звезды?}
\soul{Нет. Давайте скажем так: нет силы, которую вы могли бы назвать ``временем''. Есть только изменение пространства во времени и наоборот – изменение времени в пространстве. Это столь тесно и столь раздельно, что мы не можем даже примерно объяснить людям, что это значит. Вы же не знаете, что такое пространство!  Как мы можем объяснить вам, что такое пространство, чем оно заполнено?}
\people{Мы, о пространстве ещё поговорим отдельно. А, вот, по ``времени'', продолжим. Может ли время отражаться от чего либо? Хотя бы ответы - ``да''/``нет''.}
\soul{Да.}
\people{Может отражаться?!}
\soul{Вы подумайте, мы же говорили, что время, это, всё-таки, может проявлять и свойства материи.}
\people{Аккумулироваться время может где-либо?}
\soul{Да.}
\people{И как? Пример тогда.}
\soul{Пример?}
\people{Да.}
\soul{Хотя бы - ваш мозг. Он только и занимается этим.}
\people{М-м-м, аккумулирует, значит, время…  А, вот, распространяться в пространстве время может?}
\soul{И - да, и - нет. Дело в том, что пространство может существовать без времени. В одной из восточных религий,  это является основой их веры, когда нет времени, но есть, всё-таки, пространство. А пространство выдумано только для того, чтобы опять положить куда-то вашу плоть. Потому что вы с ней даже здесь не можете расстаться. И, потому, существуют, специально для вас, существуют пространства, где нет времени.  Где нет понятия ``сегодня/завтра''. Вот вам, ``вечный рай'' и ``вечный ад''. Ваши слова.}
\people{Может ли время распространяться со сверхсветовой скоростью, или оно распространяется мгновенно?}
\soul{Давайте не будем говорить о времени и про распространение. Это абсурд.}
\people{Хорошо. Можно ли провести какую-либо аналогию, сравнивающую время? с какими либо известными нам явлениями? С какими-то геометрическими фигурами…}
\soul{Всегда привыкли говорить  ``река времени''.  И, даже объясняя путешествия в прошлое,-  приводите пример, что вы плывете против течения, - не имеет время ``направления течения''. Не имеет.}
\people{Но, может, оно спиралевидное?}
\soul{Нет. Это было бы уже ``направление''. Вы согласны? И поэтому, не нужно времени распространяться с какой либо скоростью. Не нужно времени куда-то перемещаться. Ему не нужно аккумулировать, ему не надо отражаться. И, в тоже время,  время может проявлять свойства материи. И, тогда, оно будет отражаться. И, тогда, оно будет притягиваться. Тогда, оно будет влиять на звёзды и звёзды на неё. И, при этом, вы не будете ничего знать о том. Потому, что вы будете изменяться вместе с этим временем. Ибо в вашем мире, время и пространство, пока что неразделимы. Если изменилось время – изменилось пространство; изменилось пространство – изменилось время. И вы уже не сможете, ибо находитесь в этой системе, которая двигается… И потому, вы не сможете зафиксировать движение этого изменения. Вы поняли?}
\people{Ну…тяжело. Надо б ещё…}
\soul{Для того, чтобы увидеть, что изменилось, вы должны находиться вне системы. Вне этого времени. Геометрически? Ну, хотя бы, не в этой солнечной системе.}
\people{А время, можно связывать с присутствием какого либо разума? Не нашего, не человеческого?}
\soul{Вы могли бы назвать разум ``не человеческий''? Хотя бы представить его. Хотя бы немножко. Хотя бы на мгновение вы можете его представить?}
\people{Ну… Трудно, но мы думаем, что есть какой-то разум.}
\soul{Вы не можете представить никакой разум, кроме своего. Вы просто будете его гипертрофировать.  Или резать по частям, и говорить, что всё отдельно, ибо, просто вы не можете сочинить то, чего нет. Зачем вы задаёте вопросы, которых вы даже не можете сочинить?}
\people{А человек в состоянии оказывать какое-либо воздействие на процессы времени?}
\soul{Вы повторяетесь.}
\people{Да. А эти процессы времени могут оказывать воздействие на человека, опасно ли это?}
\soul{Да, и множество. Если вы не знаете эти примеры - пожалуйста: ваше самовозгорание, старость, которая приходит в младенческом возрасте. И есть даже случаи более редкие, ибо из хаоса порядок, всё-таки, сделать тяжелее, когда, вдруг, старый человек начинает молодеть. Пожалуйста, - ваши летаргические сны, когда приостанавливается деятельность организма. Вы - посторонние наблюдатели, и, потому, вы можете видеть изменения. Человек, попавший в этот сон – он не видит. Он не видит себя. Он не понимает, что идёт время, ибо, для него, это - настоящее. Вы можете назвать, для себя, ``будущее''? Лично,  для себя. }
\people{Да, нет.}
\soul{Нет. Потому, что для себя - вы постоянно живёте настоящим. Вы согласны? Для этого нужен посторонний наблюдатель. Да, любой из вас является посторонним, но только вас выручает то, что вы живёте очень тесно. Ваши поля соединяются столь сильно, что трудно разделить их. Хотя, ваши экстрасенсы умудряются видеть их отдельно. Вы можете видеть только расплывчатые границы. Вы можете видеть их комбинации, но вы не можете видеть сами их границы. Потому, что вы живете в столь плотном мире и столь информационном, что уже нельзя разделить каждого из вас. И, потому, мы говорили вам, что если вы все будете торопиться, все до одного, тогда вы измените время, и оно будет идти быстрее. И наоборот. А, чаще, вы торопитесь, а сосед – нет. Какие могут быть изменения, если вы действуете, как в басне? (басня Крылова ``Лебедь, рак и щука''. прим.) Спрашивайте далее.}
\people{Хорошо. А вот, имело ли время начало?}
\soul{Мы говорили же вам, что все миры имеют начало и конец.}
\people{А произошло ли это в результате большого взрыва? Или какой-то другой процесс был?}
\soul{Это уже зависит от отправной точки, от той общей точки, где уже произошли ваши миры. Да, есть миры, множество миров, которые произошли от взрыва, есть множество других, которые произошли от других причин, и множество, множество, множество.}
\people{А вообще, был ли физически - большой взрыв?}
\soul{Давайте скажем так:  сперва, была Тьма. А бог был. А где же он был? Во тьме? А что такое Тьма, если не было ещё Мира, и он был сотворён только, им… Вы можете мне сейчас ответить на эти вопросы?}
\people{Нет, мы пока не можем. Существовало ли что либо, до большого взрыва?}
\soul{Но мы  же вам сказали. Мы сказали это только что.}
\people{Тьма….}
\people{Ну, будем говорить так: бог всегда есть. Он всегда, в любом состоянии, но он есть.}
\soul{Давайте скажем так: начало… Можно, конечно, зафиксировать и начало. Но что было до Начала, нельзя уже говорить, потому, что это иные понятия, это совершенно другая физика, это всё совершенно иное! И даже примерно нельзя будет объяснить, что было, когда была Тьма. Нельзя этого объяснить. У вас нет этих слов. У вас нет аналогов. У вас нет этих представлений. Ваш мозг не может воспринять то, что он не может сравнить с чем-то! Выйдите на море, и назовите мне расстояние на море, если вы не имеете, как на суше, отметок. Это вы здесь можете сказать: ``да, здесь километр, здесь – два''.  В море вы не можете этого сделать, ибо у вас нет уже меры. У вас нету точки отсчета. Так же и здесь – ваш мозг никогда не возбуждает  то, чего не было. И если он, вдруг, попадет в то ``иное'', хотя бы во Тьму, – пусть будет названа так,- он перестанет существовать.}
\people{Скажите, а расширяется ли наша Вселенная?  Ведь сейчас большие споры вокруг этого идут.}
\soul{Да.}
\people{Так она и в дальнейшем будет расширяться? Или, всё-таки, обратный процесс…}
\soul{Мы когда-то говорили вам в одном из контактов. Вспомните.  О шарике. И вы находитесь на поверхности шарика, и его раздувает. Вы помните?}
\people{Да, может быть.}
\soul{Может быть…}
\people{Так вселенная растёт,  как растёт человек - из маленького - в большой?}
\soul{Можно сказать так, что вы находитесь в середине шара, кто-то надувает шар, и вы видите его стены и говорите, что - да, Вселенная расширяется.  И, в то же время, она может уменьшаться. Находясь в постоянном объеме, вы будете уменьшаться, и  будете говорить: ``Смотрите, она расширяется!'' Вы согласны?}
\people{А на ваш взгляд, Вселенная расширяется или будет момент, когда она будет сужаться?}
\soul{Мы говорили вам, что у нас иная физика. У нас, слава богу, ничто не расширяется!}
\people{Вселенная, как физическая Вселенная расширяется, допустим…}
\soul{Да, мы вам отвечали.}
\people{А будет ли конец у нашего времени? У земного.}
\soul{Конец есть у любого мира. Согласитесь, что  то Солнце, когда-то - рано или поздно - погаснет. Вот вам и физический конец. А пока вы считаете, что вы – это плоть, считайте, что это будет конец и ваш!}
\people{Скажите, а животные, ведь они не знают,  что они рождаются, живут и умирают. Почему они тоже стареют и умирают?}
\soul{Разве не знают об этом? Вы знаете, что думают животные?}
\people{У них же разума нет.}
\soul{Вы решили, что у них нет разума? Потому что вы решили, что вы умнее, и вы созданы по подобию бога?  Животное - не было создано? Кстати, по какому же подобию, тогда, было бы создано животное? Вы можете мне сказать? Это была фантазия бога?}
\people{По нашему подобию, наверное.}
\soul{По вашему подобию? Почему же он тогда не похож на бога?}
\people{Почему ж не похож? Это же не внешность. В принципе-то, внешность не влияет.}
\soul{Прекрасно. Внешность не влияет! Но тогда почему у вас есть разум, а у животного – нет? Если внешность не влияет? А каком разуме говорить? О возможности создавать что-то? О возможности создавать оружие, чтобы убивать друг друга? О возможности создавать технику, которая помогала бы вести беседу, ибо у вас так слаба память, и вы так ленивы что-то запоминать. Вы это считаете разумом?}
\people{Ну,  животное и человек ведь различаются чем-то?}
\soul{Чем? Чем?!}
\people{Мыслительные процессы не те.}
\soul{Не те?}
\people{Уровень.}
\soul{Если вы считаете, что вы мыслите, всё-таки, химией, простите, что у животных другая химия? У животных - другая химия?}
\people{Хорошо, животные обладают разумом. Но, наверно, разница-то какая-то существует между человеком и животным в этом отношении?}
\people{Более примитивный, скорее всего, разум у них.}
\soul{Ну, назовите мне эту разницу! Назовите!}
\people{Мы можем создавать книги.}
\soul{Прекрасно. Есть множество племён, которые не умеют создавать книги и музыку. Но вы же, всё-таки, их называете человеками. Только что, ``отсталыми''. Так в чём разница между животным и человеком?}
\people{Животное вообще не может создать…}
\people{Ну, творческий процесс…}
\people{Творческие процессы, конечно, не те у них.}
\soul{Хорошо, давайте говорить о творческом процессе. Что в вашем понятии ``творческий процесс''? Как вы можете проявить ваш творческий процесс, чтобы вас поняли другие? Нарисовать картину?}
\people{Нет. Не обязательно.}
\soul{Создать скульптуру? Написать стихотворение?}
\people{Ведь творчество не только в искусстве или в науке проявляется, но и в жизни проявляется…  Воспитать ребёнка, ведь это тоже…}
\soul{А как вы его воспитываете?}
\people{Ну, как воспитываем – дело другое. Может быть, больше натворим, чем сотворим.}
\soul{Давайте по-другому. Животные не воспитывают своих детей?}
\people{Воспитывают,  конечно.}
\soul{И в чём разница? В творчестве? Назовите мне элемент творчества в воспитании щенка или человека.}
\people{Ну, вы хотите сказать, что разумы равны?}
\soul{Нет. Мы хотим услышать, от вас,  разницу между человеком и животным.}
\people{Ну, если не брать внешность, конечно, во внимание… наверное,  у животного потенциал меньше, в какой-то мере, чем у человека.}
\soul{Да? Но, вы же, сами говорите, что есть люди, которые имеют очень маленький потенциал. Даже у самого падшего человека потенциал выше, чем у животного, да? Так? Тогда, почему же вы говорите, что вы – животные, а не человек? И вы называете множество людей, которых вы причисляете к животным. Потенциал?}
\people{Ну, наверное, человека мы называем животным только тогда, когда у него нет действительно человеческого разума – что ли….}
\soul{Вы не правы.}
\people{Мы не правы, не правы, да. Животные не принимают такие формы, как человек в некотором виде. Это верно.}
\soul{Вы отличаетесь от животных, самое первое, тем, что вы придумали это отличие. Вы же сказали, что вы подобны Богу. Только вы и больше никто! А животное было создано прихотью чьей-то. А по какому подобию – не известно. Далее. Вы отличаетесь от животного тем, что вы имеете эгоизм. Вы – себялюбы. У животных - только инстинкт самосохранения, чтобы сохранить себя и подобных себе. А вы имеете этот инстинкт и устраиваете войны, хотя, вам ничто не угрожает. Жадность. Обладают животные жадностью? Той страстью, которая может уничтожить весь мир, лишь бы было всё ваше?}
\people{Нет.}
\soul{Нет. Обладают ли животные той ненавистью, с помощью которой вы можете уничтожить весь мир только из-за того, что вы обижены на кого-то одного?}
\people{Нет.}
\soul{Нет. Обладают ли животные религиями?}
\people{Нет.}
\soul{Здесь вы не правы.}
\people{Обладают?!}
\people{В каком смысле вы понимаете ``религию''? Т.е. связь с высшими мирами у них существует, и они больше видят, чем мы…}
\soul{”Связь с высшими мирами'' или ``религии''? Это - совершенно разные вещи!}
\people{Ну, если понимать религии – это связь с высшими…}
\soul{Нет.}
\people{Нет? А что под ``религией'' тогда понимать?}
\soul{Религия – это всего лишь одно из видений мира. В него входят все ваши мечты и желания. И вера в Истинное Высшее и в Бога – это не религия.}
\people{Скажите. Если, завершая, всё-таки, дискуссию о животных… Скажите, пожалуйста, вы контактируете с животными? И с какими вам больше нравится контактировать животными?}
\soul{В вашем понятии, контакт – это умение слышать и разговаривать. В нашем понятии, нет контакта. Мы никогда не покидали вас. И мы когда-то говорили вам, что мир столь един, что зря вы делите себя и животных на разные миры. У вас даже название - ``животный мир''. Вы поделили. Мы не имеем делений. У нас другая физика. Когда-то вы спрашивали, есть ли у нас животные и любим ли мы их. Мы вам говорили: ``да''. Но это не значит, что это именно те животные, которые есть у вас. Это другое понятие. У нас другая физика, совершенно другая, но она у нас есть. Но, для вас, эта физика считается тем, одним из миров,  куда вы хотите прийти. И когда вы придёте в наш мир, вы разочаруетесь. Потому, что в нём тоже надо будет работать. В нём тоже есть много бестолковости, падений, греха, добра и зла. Мы похожи с вами, и мы говорили вам об этом. И мы - тоже ищем. И,потому, говорим вам: ``Мы не боги. Боги – вы, а мы – в вас''. Далее…}
  (счёт)
\people{Что вы хотели сказать?}
   
\soul{Спрашивайте.}
\people{Наверное, мы посвятим отдельно разговор  животным и растениям, поскольку, действительно, мы  недопонимаем  и контакты с животными и растениями у нас должны в дальнейшем развиваться. Может быть, вы нам подскажете пути?}
\soul{Пока вы будете говорить о ``контактах'', вы не поймёте ничего. Ибо, вы уже подразумеваете разделение.}
\people{Да, вы правы.}
\people{Скажите, а животное  не помешает?  Допустим, если сейчас собака войдёт сюда, это не помешает общению нашему?}
\soul{Нет.}
\people{Можно запустить её?}
\soul{Это ваши проблемы.}
\people{Хорошо. Скажите, а может ли быть конец жизни во Вселенной?}
\soul{Да, мы вам говорили, что физический конец будет обязательно для каждой Вселенной.}
\people{А конец разума во Вселенной может быть?}
\soul{Если вы не научитесь и не поймёте, что разум - не обязательно ``во плоти'', то …}
\people{Вы, почему замолчали?}
\soul{Спрашивайте.}
\people{Скажите, а вот известен случай попадания человека из будущего. Именно из будущего. Это действительно было?}
\soul{Да. И не только одна.}
\people{Да, не одна.}
\soul{И есть наоборот, когда исчезает человек и попадает в будущее. Или в прошлое. Есть множество этих вариантов.}
\people{А действительно ли, что на других планетах, в виду настроенности зрения человеческого и приборов на другое время, астронавты будут видеть совсем не то, что там есть на самом деле?}
\soul{И – да, и - нет. Потому, что вы попадёте в один из вариантов. Параллельных. Только и всего. Даже здесь, вы видите ложный мир. Придя туда, вы тоже увидите ложный мир. Только что он не будет параллельным, по отношению к вам. Это будет одно и то же в вариантах, видение на одной планете и здесь. Нельзя говорить ``это ложно или нет''. И когда вы говорите, что мы попадём на другую планету и будем видеть ложно,- это ложь. Ибо вы будете в одном из вариантов, любой из вариантов будете верен для вас. И нет того истинного мира. Нет… Либо, будет один из вариантов, потому что вы  сами – один из вариантов, А ваше ``Я'', именно истинное ``Я'', пока что спит.}
\people{То есть, я хочу спросить: вот, прилетим мы на Марс. Вот, что мы там увидим?  Один из вариантов параллельного мира или действительную картину, что сейчас там осталось?}
\soul{Вы увидите по отношению к своему миру, своей параллели.}
   
\people{И, в то же время, на Марсе может быть совсем другой мир?}
\soul{Вы говорили о параллельности миров, и тут же, не можете догадаться, что есть параллельность миров и там, на Марсе.}
\people{Ну, мы уже догадываемся, потому что …}
\soul{Давайте скажем так… Да, вы правы. Придя на планету, которая имеет другие параметры, вы будете её видеть ложно.}
\people{Своим взглядом? То есть - как мы привыкли?}
\people{Да. У нас совсем другие приборы…}
\soul{Вот это и есть ``параллельность миров''. Поймите, вы пришли на Марс, имея приборы, как говорите, которые настроенные на ваше время. Вы видите один мир, а Марс-то вертится с другими понятиями, и у него другое понятие ``времени''. Вот вам - параллельность. И, в то же время, мы сказали вам ложь.}
\people{Почему?}
\soul{Почему? Потому, что вы измеряете скорость и изменения любых параметров этой планеты вашими же приборами. Вы согласны? Тогда о каком ``изменении'' вы можете говорить?  О каком измерении можете говорить, если вы находитесь внутри этой же системы? Мы говорили вам: чтобы увидеть изменения, вы должны находиться, хотя бы, вне солнечной системы. И если вы проводите измерения Земли и попадаете на Марс, вы увидите ту же самую картину, что вы видели с Земли. Разве только-что поближе. Только и всего. А для того, чтобы увидеть иное другое измерение, вы должны попасть на другую планету и измерить это другое. Вы говорите ``центр Вселенной''. Даже геометрически и логически вы можете догадаться, что этих центров множество, великое множество. Ибо, по отношению к вашей Земле, у вас один центр. Переместитесь на другую планету, и переместится центр. Вы согласны?}
\people{Да, скорее всего. А скажите, может ли иное время служить причиной гибели астронавтов? То есть, быть несовместимым с человеческой…}
\soul{Почему же только астронавтов?}
\people{А кого ещё?}
  
\soul{Вспомните возгорания хотя бы.}
\people{Да. Вот пастух сгорел. Это, действительно…}
\soul{Спрашивайте далее.}
\people{Скажите, часто рассказывают, что во время трагических ситуаций, время замедляется. Меняется время или меняется его восприятие?}
\soul{Да. Это и есть проявление инстинкта. Вы привыкли воспринимать инстинкт, что это ``животное'', ``противное'', ``низшее''. Хотя всю жизнь живёте инстинктом. Но только что придумали множество имён, с  множественными характеристиками, и всё это называете характером, нравом, настроением и т.д. Но, стараясь сохранить себя. И потому, он изменяет восприятие. Согласитесь, для окружающих не было никакого изменения времени. Вы согласны?}
\people{Да.}
\soul{Только для вас изменилось время. Ибо вы стали быстры для того, чтобы у вас хватило того же времени спастись. Но, чаще, вы только наблюдаете, потому что мозг в испуге -  в вашем понятии, мысль об испуге, - и он имеет их столько много, что ему не хватает больше энергии, чтобы вы сделали какие-то лишние движения, чтобы спастись. Далее. Вами же сказана пословица: ``пьяному море по колено''. Согласны?}
\people{Да.}
\soul{Человек в нетрезвом виде, падая, имеет меньше шансов разбиться, чем трезвый. Вы согласны?}
\people{Да.}
\soul{Причина? Причина та же. Что вы не мешаете мозгу решить ту проблему. И не начинаете шевелить, в вашем понятии, мозгами - как бы упасть получше. Нет понятия ``испуга''. Потому что испуг, это всего лишь тормоз, который хочет остановить вас.  Испуг – это, когда мозг ``закрывает глаза''…  Вы поняли?}
\people{Понял. Поэтому, в разных ситуациях, для одного, время меняется тысячекратно, а у другого – неизменно.}
\soul{Потому и говорят вам: ``доверяйте. Доверяйте инстинкту, а не унижайте его''. И потому говорят, что иногда полезно умение ``не думать''. Когда вы падаете, вы представляете множество картин, как вы упадёте, столь быстро! Вы уже даже представляете – вот вам и ``будущее''. Предвидите в будущем, как вы будете падать, как вы будете умирать и множество, множество, множество. И столь сильно это представляете, что очень трудно переубедить кого-то и поменять вариант. И когда вы падаете неожиданно, вы просто не успеваете создать все эти варианты. И, потому, мозг может спокойно выбрать наилучший, а не созданный вами, в вашем понятии, разумом.}
\people{Скажите, а когда-нибудь мыслительные процессы тоже станут инстинктом?}
\soul{Давайте скажем так, что сперва  появилась, всё-таки, мысль. Вы согласны? Что, сперва было, всё-таки, слово. А чтобы произнести слово, нужно было научиться мыслить.}
\people{Но ведь человеческая суть ещё в таком зачаточном… Вы сами сказали, что мы спим, т.е. ``истинный человек'' наш - спит.}
\soul{И что же? Давайте попробуем и разбудим вас. Что будет? Ничего. Ваш разум испугается, и вы потеряете его совсем. Только и всего. Разум считает себя владыкой. Зачем ему сейчас лишняя ``революция''?}
\people{Так, человек должен, наверное, набрать тогда  какую-то сумму опыта…}
\soul{Давайте скажем так: разум не должен бороться. Он должен сам прийти к тому, что пора ``будить''.}
\people{Ну, хорошо. Вот, считается, - ``триединство'', - любовь, разум и воля. Это триединство в какой-то мере, главные его принципы, да? Зачем их разделять? Одно без другого инвалидом будет, наверное?}
\soul{Да. Но вы уже разделили, называя и придумав им имена.}
\people{Ну, наверное, люди так… (пытается объяснить почему разделили)}
\soul{Да, вам  легче. Вы любите рассуждать разделено, потому - вы не можете нарисовать истинную картину. Вы придумал слова, только для того, чтобы эти разделения предоставить другим. Вы говорите: человек отличается тем, что у него есть разум. Под разумом вы подразумеваете возможность говорить. А что такое слово?  Это ``разделение''. Просто, деление мира  на столь мелкие кусочки, что вам не хватает … (теряется)}
\people{Скажите, был случай, когда женщине упорно снился сон, о другой жизни в другой семье, о действиях в той семье. И, вдруг, он сбылся. Она всё знала, о той семье и т.д.  Как вы охарактеризуете эту ситуацию?}
\soul{Человек очень хорошо ориентируется во времени.}
\people{Так если у человека вещие сны – он хорошо ориентируется во времени?}
\soul{Да. Он ощущает, как вы говорите, ``течение времени''. Это первое. Второе, - это может быть насильственное, когда кто-то извне ведёт вас именно к тому. Когда вас лишают выбора вариантов будущего. Это может быть третье. Когда вы сами стремитесь именно к этому, это было в мечтах. Есть и четвёртое… Когда ``будущее'' пришло к вам, в ваше ``прошлое''.  Но тут же, на четвёртое налагаются первые три. Ибо будущее может прийти насильственно, специально кем-то,  или вызвано вами  и так далее, далее. Вы поняли?}
\people{Да.}
\soul{Спрашивайте.}
\people{Скажите, сейчас что-то мешает переводчику? Как охарактеризуете его действия?}
\soul{Вы можете объяснить, что вы чувствуете сейчас?}
\people{Переживания за него.}
\soul{Или страх?}
\people{Может быть и страх. Страх мешает очень, да?}
\soul{Самые сильные эмоции у вас, это страх и жадность.}
\people{Жадность во всём?}
\soul{Во всём. Даже жадность во страхе. Если уж бояться, так бояться, так действительно бояться. Даже в этом вы проявляете жадность. Или страх - бояться так, чтоб умереть от страха.}
\people{Скажите, ваш мир лишён страха? И лишён жадности?}
\soul{Нет. Что вы!}
\people{Тоже есть?}
\soul{Там, правда, другие меры. Мы жадны до другого. И, конечно, не только хорошего, но и плохого.}
\people{А можете назвать, что вы жаждете?}
\soul{Ну, если у нас нет общих точек в физике, как мы можем вам объяснить? Представляете наш труд - показать картинки ``переводчику'', не обладая физикой? Мы должны хорошо изучить ваш физический мир и найти аналоги, которые мог бы переводчик понять и нарисовать. }
\people{Угу. Ну, понятно. Ладно. }
\soul{Тем более, мы не используем его память. Мы не можем вмещаться в его… (теряется)}
  (счёт)
\people{Переводчик не устал? Есть какие-то у него мотивы прекратить сеанс?}
\soul{Спрашивайте.}
  
\people{Давайте, о времени… Немного вопросов. О пространстве. Что такое пространство? Сформулируйте ещё раз.}
\soul{Когда-то вами было сказано: пространство и материя. И вы соединили, назвав одной из теорий. ``Материя – это искривление пространства во времени''. Помните?  }
\people{Да.}
\soul{Здесь вы ложны.  Пространство – это возможность изменения ``физики'' во времени. Мы говорили вам, что есть пространства, не имеющие понятия о времени. Вы помните? Как вы думаете, попав в ``вечный рай'', о котором вы рассуждаете, вы будете там изменяться? Нет! Вы же там вечны – зачем же вам изменяться? Каким вы пришли, таким и останетесь. Вот вам - религия. Что такое религия? Это одно из видений мира и мечты, навеянные, в вашем понятии, ``злыми силами''. Когда вы искажаете мечту, да в столь безобразную картину, что потом сами отвергаетесь от них. Вот вам и ``пространство, не имеющее времени''. Есть пространство, и есть время. Они неразделимы. Это ваш мир. Мы не можем разделить здесь отдельно и то и то. Если будет какое-то изменение в пространстве, то произойдёт изменение и во  времени. Давайте скажем так: если вы сейчас передвинете стул, время будет течь по-другому. В нём будет столь мало изменений, что вы это не заметите. Если же сдвинуть всю вашу планету, тогда заметите и вы. Вы согласны?}
\people{Да.}
\soul{Но только с одним условием: если вы будете находиться извне. Хотя бы спать в этот момент. Ибо, когда вы спите, есть другое понятие о времени. Вы живёте в другом времени. Как бы, уже находясь извне, вне системы.}
\people{О наблюдателях тогда. Когда человек наблюдает, у него … то есть, за событиями или за другим человеком, то у него понятия о времени другие, чем у него, когда у него это происходит, так?}
\soul{Чаще - нет. Он старается подстроиться. Чем более он внимательно будет наблюдать, тем более будет ближе к времени, за кем наблюдает.}
\people{То есть, вы имеете в виду - отстранённый наблюдатель?}
\soul{Отстранённый наблюдатель? Вы можете это понять? Как вы можете ``отстранённо наблюдать''? Вы ж ничего не увидите и ничего не запомните.}
\people{Нет. Например, когда чувствуешь события, которые происходят? Или просто наблюдаешь за какой-то преградой? (прим. хочет сказать, что есть ли разница если ``живёшь'' ситуацией или просто не вовлечён в неё.)}
\soul{Давайте скажем так: для того, чтобы вам понять какого-нибудь человека, вы должны шагать в ногу с ним. То бишь - быть во времени в том, в котором существует и он. Вы же, хотите прилететь на Марс и увидеть истинное? Вы забываете, что вы – целый мир, и чтобы увидеть вам истинно -  тоже надо быть настроенным именно на это время.  Вы никогда не замечали, что человек, которого вы множество лет знали, вдруг, в один прекрасный момент становится неузнаваемым, и вы видите, что это совершенно чужой человек, совершенно иное лицо. Разве не было у вас такого?}
\people{Угу, было.}
\soul{Как вы это объясняете? Тем, что вы пришли в другое время, по его понятиям. Вы – театр. Вы мир, в котором вы играете, вы играете роли. Вы прекрасный резонатор. Вы прекрасно усиливаете то, что приходит к вам. Любое изменение вы прекрасно усилили и укрепили эти изменения. Вы садитесь в автобус, и вам портят настроение… Очень просто и легко, одной фразой - вам испортили. Вас изменили. И у вас уже - другое понятие, о времени. У вас совершенно другое понятие, о мире. Или вам сказали доброе слово, и вы тут же, резко, изменились. Вы согласны, что вместе с вами изменилось восприятие мира? Но не изменился же весь мир? Вы согласны? Изменились только вы. Изменилось время, с которым вы стали воспринимать этот мир. Изменились те количественные качества, которые вы можете охарактеризовать, как ``добрый'' и ``злой''. Если вы злы, давайте скажем так: проявления времени – волновые. Давайте представим и разделим мир, как делите вы,  -  на доброе и  злое. И вспомните, колебания. Вспомните, свою историю. Давайте считать так, что более нуля – добро, менее нуля – пусть будет зло. Вы понимаете? А теперь представьте, что вы настраиваетесь или на ``верхние'' или на ``нижние''. Или ваши колебания (пусть будет так) – вы же имеете колебания тоже? - накладываются на эти колебания добра и зла. Вот вам и ``пожалуйста”: вы воспринимаете или только доброе, или только злое, или ничего.}
\people{А что лучше – ничего не воспринимать?}
\soul{Ничего? Это равнодушие. Это голое равнодушие. Это уже нельзя назвать даже ``человеком  низшего''. Самое страшное –  это равнодушие. Пусть вы будете делать неправильно, но  у вас есть энергия. Рано или поздно вы поймёте, что вы совершили ошибку и будете исправлять. Ошибается всегда только тот, кто что-то делает и никогда не ошибается равнодушный и ничего не делающий человек. Так что лучше уж идите и ошибайтесь, но, идите. И не ждите, когда к вам придёт праведник или какой-нибудь проводник, и скажет: ``пойдёмте за мной''. Вы пойдёте его дорогой и всё. Вот вы и ``овцы''.}
\people{Можно мы вернёмся к пространству. Есть ли мерность у пространства?}
\soul{Да. }
\people{И какая?}
\soul{Вы же разделили на тройное, и дальше и дальше. Для вас, оно есть. Действительно, вы же можете попасть в четвёртое измерение, вы согласны?}
\people{Согласен.}
\soul{Значит, есть измерения.}
\people{Ну, вообще, оно бесконечное или скольки-то мерное?}
\soul{Будете расти вы – будут расти и ваши измерения. И всегда, всегда вы будете говорить, что есть понятие, о измерениях. Немножко подрастете – и вы уже освоите четвёртое и более тонкий, как вы говорите ``мир''.}
\people{А скажите, какой наилучший способ перемещения в пространстве?}
\soul{В плоти? Или истинного вашего ``Я''?}
\people{Хотя бы, в плоти.}
\soul{В плоти? Вы когда-то говорили о червоточинах.  Вот вам лучший вариант.}
\people{Червоточина?}
\soul{Задающий вопрос, должен понять. (задавала вопрос до этого женщина)}
\people{Скажите, а понятие ``нирваны'' тоже?  Это понятие времени? Или отсутствие его?}
\soul{Это пространство, не имеющее времени. Иными словами, это смерть.}
\people{А они считаются, как наилучшие. (состояния)}
\soul{Да, наилучшие, когда уже возрождаешься и не приходишь мучиться в этот мир. Многие применяют это так: смерть. Мы же вам скажем: что это остановка. И эта остановка столь длительная и столь коротка, что действительно может привести к смерти, если вы не поймёте, что пора уходить. И поэтому, для многих, это смерть. И чаще всего – это смерть.  Ибо, попасть туда очень просто! Для этого достаточно, даже не ``верить'', а просто соблюдать все обряды – и вы попадёте. Но если вы только и соблюдали обряды, чтобы попасть туда – вы умерли.}
\people{Человек должен работать всегда и везде, во всех абсолютно состояниях, чтобы …}
\soul{Да, вы красиво работаете, но не делаете этого.}
\people{Я имею в виду человека в любом состоянии его сознания, если можно так сказать.}
\soul{Да, но мы говорим вам, что вы знаете всё, что должны делать и никогда этого не делаете, и находите множество причин этого не делать. Почему вам говорят, что нету реинкарнаций! (церковь. прим.) Почему? Потому, что боятся, что вы просто-напросто скажете: ``а у меня в следующей жизни ещё будет время''. Потому,  приходится  и говорить вам о том, что его нет. }
\people{Это благо для человека сейчас - незнание?}
  
\soul{Далее. Вы опять же делаете из корысти. Ибо, если вы будете непослушны ему.( речь о служителе церкви. прим.) Ибо вы скажете, что ``я могу исправиться в другом времени''. Только не думайте, что мы говорим о Боге. Мы говорим, о ваших земных ``богах'', придуманных вами. То, конечно, ему надо вас стращать, пугать, ибо одно из самых сильных – это страх. И вас надо запугивать, запугивать постоянно. С помощью страха вы можете сделать всё, что хотите. Вы даже в рай идёте только из-за страха, из-за страха не попасть в ад! Абсурд, не правда ли? Но вы живёте именно этим абсурдом!}
\people{Скажите, мы правильно представляем, что перемещения в пространстве лучше всего и более качественны, если будет искривление пространства использоваться, мы правильно понимаем?}
\soul{Мы говорили вам, о червоточине. Используйте, используйте. И, причём, вы уже почти нашли! Но, мы не будем говорить вам, иначе - какой смысл? Вы должны найти всё сами. Ибо, если вам дать готовый ответ, вы перестанете работать. Вы скажете: ``Мы это уже знаем, давайте пойдем дальше''! И вы будете изучать новое. А как вы можете изучать новое, если вы не знаете старого? Вы начнёте совершать ошибки. Лавины ошибок! И есть, в вашем понятии, контакты, когда приходят и говорят вам и дают чуть ли не готовые формулы. И что в итоге? В итоге - это один из способов  погубить вас. Вас очень легко погубить – сделать вам подарок!}
\people{Понятно.Скажите, возможно ли… Мы даём обратный отсчёт. Мы даём вам обратный отсчёт.}
  9-8-…
  (Конец записи)
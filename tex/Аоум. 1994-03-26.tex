Аоум. глава 13-я 26-03-1994г
Георгий Губин
\people{**}
\soul{Каким богам поклонялись древние?}
\people{Солнцу, грома, молнии.  Многим богам.}
\soul{Тогда, давайте короче, - Богу?}
\people{Природным явлениям.}
\soul{Согласен. А что было далее?}
\people{Мы, к сожалению, не  помним, что было далее.}
\people{Тогда, что-то извратилось. Стали уже не природе поклоняться, а, видимо, выдуманным,  персонализированным Богам.}
\soul{Нет. Вы дали им имена. Вы поклонялись силам природы и дали имена. И дали им свои характеры. Добрые боги, злые.}
\people{Ну, это отражало сущность?}
\soul{Всё отражает сущность. В любой лжи, есть правда.}
\people{Что вы этим хотите сказать нам? }
\people{Что только отношение лично каждого влияет на ``злой/добрый''?}
\soul{Мы говорили о богах. Вы подумайте, - эволюция богов? Согласитесь, у вас их было множество, и стал - один.}
\people{Т.е. объединение происходит?}
\people{Но, это не правильно, что один стал Бог?}
\soul{Для вас, он один, для другого народа - он другой. Но, всё равно, один. Подумайте. И, до сих пор, у вас, есть язычники.}
\people{Ну, а кто более прав?}
\soul{А как вы думаете, что такое Бог? Или кто?}
\people{Сложно сказать. }
\people{В одной книжке я прочитал, что весь мир был сотворен из человеческого сердца, которое взорвалось от избытка любви, чувств.}
\soul{Нет. Все было гораздо проще.}
\people{А вы знаете, как было?}
\soul{И вы знаете, но вы не помните.}
\people{Да.}
\soul{Всё было проще. Не было богов, и все жили. Появились боги, в вашем понятии, и, теперь, вы просто существуете. Боги ваши – это ваша мечта. Мечта стать ими. Ибо вы были ими и забыли. Вы пришли сюда и забыли. Забыли, что были сами, когда-то, богами. Когда-то, сами создавали миры. Пришли – потеряли многое. И стали мечтать в легендах, в религиях. Конечно, доля истины есть. Вы соединитесь, и будет вам Истина.}
\people{Ну, это, видимо,  было чем-то обусловлено… наш такой ход эволюции?}
\soul{Нет. Это была, всего лишь, ваша глупость.}
\people{Глупость?}
\soul{Это был, всего лишь, ваш страх. Страх, что забыли. И вы стали выдумывать прошлое.}
\people{Кто же были нашими такими глупыми предками?}
\soul{Вы же и были, если вы бессмертны. Согласитесь.}
\people{Н-да…Но сейчас мы, может быть, становимся умнее, находим эту ошибку? И, видимо, понимаем, что не сделали, то, что предназначалось, что цель, нами, потеряна. Эволюции на Земле. Мы правильно мыслим? Идём в этом направлении?}
\soul{Вы хотите сказать, спросить, ``Умнее ли мы сейчас или в прошлом?”}
\people{Да, хотя бы так.}
\soul{Вспомните, средневековье. Во имя бога вы уничтожали. А сейчас? Что вы делаете сейчас? Вы создали своих богов, но не называете их богами. Вожди - ваши боги. И вы, во имя их, делаете то же самое. Только что более цивилизованно.}
\people{Не далеко ушли, короче.}
\people{Но, вот, я сегодня хотел задать вопросы по истории нашей России.Вы готовы поддержать это? }
\soul{Мы говорили о единстве, и говорите о истории России… А вы подумайте, всмотритесь в историю и найдите, - большая ли разница между историей России и историей любой другой страны?}
\people{Ну, у меня такое ощущение, что на долю России выпали очень больше испытания. Куда больше, чем, например, Франции, Англии…}
\soul{Да? Почему вы тогда не вспомните другую народность? Согласитесь, ей больше досталось.}
\people{Евреи?}
\soul{Далее, мы говорили вам: духовный центр, когда-то, был ими. Теперь, он будет у вас. Но не все будут в нём . Правильно, вы говорите ``экзамен''. Это жестокий экзамен, в вашем понятии. Ибо, вы не знаете, за что бьют вас.}
\people{Да, не знаем. Наверное, это на пользу. Бог бьёт того, на кого возлагает большие  надежды… Но, это мало утешает.}
\soul{Потому и больно вам, что нет утешения. Согласитесь, любимым достается более.}
\people{Т.е. надо терпеть и к этому примириться? Оправдывать надежды…}
\soul{Нет. Нельзя говорить о примирении. Вы должны бороться, но не против Бога. Бороться, чтобы вас не били. Чтоб не заслужили того. Вы же, ищете виновных не в себе, обвиняете кого угодно, даже бога, но не себя.}
\people{Скажите, пожалуйста, вот, у нас исчезла кассета. Вы не подскажете, куда она, всё-таки, делась?}
\soul{Нет. Поймите, мы говорили вам: вы рабы, и, потеряв, потеряете многое, ибо не всё помните, и не всё будет понято вами. Будет исковеркано. Всмотритесь в себя, всмотритесь в эволюцию контактов, и всмотритесь в историю – вы найдете здесь много общего. Вы услышали и записали, прослушали и как-то поняли. Если не слушали, то поняли по-другому. Согласитесь, что, когда-то, пришёл истинный бог, но прошло время и исказило его. Что делаете и вы. Даже здесь.}
\people{Поэтому, мы и стараемся сохранить все записи с вами…}
\soul{Мы говорили не о том. Мы говорили, об аналогии. Подумайте - всё во всём. И, тогда, у вас не будет вопросов. Вы будете видеть во всём. Вы же говорите: это - другое, то – другое. Не видите единства, хотя, говорите, что поняли. Поймите, как вы живёте, вашу эволюцию. Хотя бы эту жизнь, раз не помните прошлую. Она повторяет один к одному всю историю Земли.}
\people{Повторяет в нашей группе?}
\soul{Мы говорим о каждом, мы не говорим лично, о вас. Давайте договоримся так: что бы не было – мы будем говорить ``я''- про себя. Про переводчика будем говорить - ``он''. Про всех будем говорить - ``вы-вы'', про вас – просто ``вы''. Вы поняли?}
\people{Ну, тогда можно продлить вопросы? Получается, вы знакомы с теорией Маркса, Энгельса, о фазах социализма, переходе капитализма в социализм?}
\soul{Ответьте мне, прежде, чем зададите: ваша религия, о которой вы будете спрашивать, создана  мозгом? Сознанием? Или душой?}
\people{Скорее всего, это умозрительные построения. Игра ума.}
\soul{Это ваши мечты. Мечты, - но вы не боретесь за них. Вы хотите изменить физически всё, но не духовно. Как вы можете представить, в вашем понятии, социализм и далее, оставаясь прежними? Давайте, будем шутить. Представьте, что сейчас объявили, как вы говорите ``при коммунизме''. Согласитесь, о коммунизме вы в первую очередь приводите пример, что всё бесплатно. Далее, вы не можете привести. Вот? представьте хотя бы это - всё бесплатно. Что вы будете делать?}
\people{Ну, вероятно, каждый будет стремиться набрать, как можно больше.}
\soul{И опять станете нищими.}
\people{А как же тогда лозунг ``от каждого по возможностям – каждому по потребностям''? Что, он не верен?}
\soul{Какие ваши возможности? И когда вы будете брать, брать и брать, вы будете следить за своими возможностями? У вас будет одна возможность - брать, брать и брать.}
\people{Ну,  скажите, переход от формата капитализма к социализму, теперь, к коммунизму, это - закон общества?}
\soul{Это ваши законы. Вами выдуманные. Для вас, они реальны. Для вас, они настоящие. Для вас. Ибо, вы живёте по этим законам. Потому, они и настоящие. Всмотритесь, всё идет по кругу: вы пришли и пошли далее. А теперь смотрите, что было, и что стало. Вспомните историю. И вы сможете сказать, какой будет следующий строй, если вы будете продолжать так жить дальше. У вас опять будут феодалы, у вас, опять будут рабы. Хотя, они есть и сейчас. Да, это так. И только две вещи могут помешать этому.}
\people{Какие?}
\soul{Первая, это взрыв, созданный вами. Другая – взрыв, созданный вашей душой. Тогда, вы поймёте, что политика не нужна вам. Политика – это уже разделение. Вы можете представить себе - общество духовных и политику, вместе? Вы, как-то, спрашивали…}
1-2-3
\people{Почему переход от социализма к коммунизму не произошел, как предполагали теоретики коммунизма?}
\soul{Мы же говорили вам - это ваши мечты. Многие ли ваши мечты сбываются, как вы хотели? Далее. Мы говорили вам, только что, о политике. И вы придумываете множество стрОев, но сами не изменились. Какими вы были, древними, так и остаетесь сознанием. Вами правит эгоизм, и там, и в эти времена. И, потому, любой строй будет ложным и неверен вам.}
\people{Получается, и не надо стремиться к улучшению отношений между людьми через строй.}
\soul{Разве? Вы считаете, политика улучшает отношения между людьми?}
\people{Ну… Как механизм.}
\soul{Да? И вы здесь умудряетесь применить технику? Тогда, объясните - у вас все друзья политические?}
\people{Нет.}
\soul{Вы вспоминаете  о политике в разговоре  с другим?}
\people{Нет.}
\soul{Тогда, о чём говорить?}
\people{Мы страдаем от негодной политики}
\soul{Вы страдаете, от того, что не умеете увидеть себя в других. Вы не можете понять единства. Вы делите на высших и низших. И потому - политика. Вы хотите разделить весь мир по своему умыслу. Вы считаете… Кто-то, из вас, был крупным политиком. И потому, нам трудно разговаривать.}
\people{“Из нас'' – в смысле? В общем вы берёте?}
\soul{Вспомните.}
\people{Ну, вопросы политические нас интересуют. Идеалы социализма понятны и привлекательны. Они сродни идеалам христианского общества. Почему они оказались несостоятельными?}
\soul{Вы говорите: ``они сродни ``. И, в то же время, ваш социализм отрицал Христа. В чём сходство? Согласитесь, что это была, просто борьба религий,  только и всего.}
\people{Но, смотрите, если рассматривать зарождение социализма в мировой истории, то был ли изначально этот путь тупиковым?}
\soul{Нет. Любой путь никогда не был тупиковым. Вы создаете тупик и приходите в него. Вы прекрасно могли бы жить при феодализме, не придя в тупик. В любом строе вы могли бы прекрасно жить. Но мы же говорили вам: политика, это всего лишь… И вами сказано: механизм. Механизм, который может ломаться, который неодушевлен. Поймите это. Мы не отрицаем политику, ибо вы живёте в этом мире, и потому, она вам нужна, и без неё вы не можете жить. Вы придумали, и для вас, это пища. Для многих, это средство выживания, и далее. Вы поймите, политика – тупикова, если не будет одухотворена.}
\people{Это  мы сейчас, видимо, и наблюдаем в России.}
\soul{Скоро, скоро, мы говорили вам, даже при вашей жизни вы увидите изменения, и, дай бог, чтобы не была тупиком и эта.}
\people{Социалистическая революция в Германии была задушена и подавлена. Почему России пришлось пройти этот путь до полного краха?}
\soul{Нет, вы не знаете полного краха. Вспомните. Вспомните страну, которую вы так бесчётно ищете.  Можно сказать, там был социализм. Вот она потерпела крах.  Вы же - нет.}
\people{Шамбала? Шамбала потерпела крах?}
\soul{Нет.}
\people{А в какой стране?}
\soul{Подумайте.}
\people{Израиль?}
\soul{Нет, мы говорили о стране, которую вы ищете и не можете до сих пор найти. Вы ищете Израиль?}
\people{Нет. Мы ищем следы… Интересуемся Беловодьем, Шамбалой.}
\people{Вы не можете конкретно сказать?}
\soul{Нет, вы подумайте сами. Вы сказали о крахе. Разве Шамбала не существует сейчас?}
\people{Мы же не знаем, какие там внутренние процессы происходят… Они, когда-то одобрили политику Ленина, писали письмо Махатм. И, в то же время…}
\soul{Кто одобрил?}
\people{Ну, Рерихи передали, якобы, от Шамбалы послание Далай-ламы. И это было одобрением. Вдруг, выясняется, что это был путь тупиковый. Что Ленин – мечтатель… Не правильно, видимо, строил…}
\soul{Нет, мы говорили вам, что любая дорога не тупиковая, и вы сами ведёте к тупику. Идеи были верны. Но, вспомните, вспомните, это уже история ваша. Это жизнь ваша. Как постепенно она превратилась в тупик. И найдите, в чём здесь причина? Разве не новые были идеи? Идеи были верны, но вы забыли одно ``но''. Вы хотели изменить весь мир, но не себя.}
\people{Но, почему тогда не только в России, но и в других странах произошёл спад и несостоятельность социализма? Если у нас были вожди не очень умные, то в других странах могли быть и талантливые….}
\soul{Давайте, будем откровенны. Те страны, о которых вы сейчас говорите, были всего лишь марионетками вашей страны. И если хозяин кукол испорчен, то и кукла не будет подчиняться и будет нарушать. Согласитесь.}
\people{Но, это печально , что страны могут, как любые соседи, простые индивидуумы вести себя в политике, как простой смертный человек.}
\soul{А вы подумайте, вы здесь только что ответили сами. Посмотрите страну, как индивидуум, и вам будет легче понять. Возьмите карту. Возьмите страну, как один человек. Далее. Каждая страна имеет, в вашем понятии, душу. Это - всё общество, все жители этой страны… И тогда уже можно будет говорить об индивидуумах её. Но, согласитесь, составляющие её - влияют на качество.}
\people{Да.}
\people{А сейчас, мы догадываемся, что нельзя было разрушать старое, наработанное капитализмом. Надо было лучшее брать из этой формации и внедрять в новые социалистические отношения. Так ли мы понимаем? Правильно?}
\soul{Согласитесь, не зная старое, его использовать, - новое не сделаешь. Это одна из ваших ошибок. И вы хотели начать с нуля. А с какого нуля? В вашем понятии, даже кухарка могла править страной. Согласитесь, что это звучит смешно.}
\people{На какой стадии развития социализма в России были совершены главные, кординальные ошибки? На стадии ``разрушения всего до основания''?}
\soul{В начале.}
\people{Это не было связано с цареубийством? Это огромный грех на страну упал.}
\soul{Нет. Это одна из причин, но не более.}
\people{Ну, да. А это могло быть связано также с разрушением церкви и религии? Это болеее, наверно,существенное значение?}
\soul{Это было существенней, чем цареубийство. Далее, согласитесь, убивая царя, вы убивали и церковь. Далее, вспомните, Я говорю вам,вспомните,- в ваше время и вашими руками было сделано и это. Да, каждый из вас, когда-то, разрушал и строил храмы.}
\people{Наверное. А вы сейчас видите в нас какую-то созидательную силу? Есть у вас надежда на нас, лично, что мы что-то сможем, более полезное?}
\soul{Никогда не задавайте никому таких вопросов. Это говорит только о неверии себе. Если мы вам скажем: нет у вас этих сил – что вы будете делать? Вы скажете: ``ну и шут с ним''. А теперь подумайте, если мы вам скажем: есть у вас эти силы – вы думаете, что вы начнете созидать? Я думаю, что нет. Вы скажете – ну, значит, хорошо. Придёт время –  и начнём. Мы говорили вам, и вы писали: нет падшего человека, достигшего дна. Нет. На вашей Земле - нет.}
\people{Вы хотите сказать, что на других Землях есть?}
\soul{Мы говорили вам, но вы забывчивы, что есть миры и ``ниже'' вас. Не вы первые, не вы последние.}
\people{Ну, вот, поэтому, я, лично, стремлюсь к этим встречам, чтобы получить ответ на свои сомнения, на свои мысли. Вот, вы сейчас сказали, я это уже понимаю без особых сомнений.}
\soul{Но будьте осторожны. У вас было сказано: - ``благими намерениями…'' (прим.}
имеется в виду поговорка ``благими намерениями выстлана дорога в ад''). Согласитесь, что любое добро может перерасти во зло. Всё зависит, что вы поймёте, и как вы поймёте. Мы б могли бы в вас придти и объяснить всё так, чтоб вы поняли. Но, простите, тогда вы ничто не будете знать более, и уже, встретившись с новым, растеряетесь, и вам будут снова нужны ``контакты''. Вы должны идти. Идти сами. И мы, всего лишь приходим к вам, не советом, а только сомнениями. Ибо, сомневаясь, вы начинаете думать.
\people{Вот, я и чувствую положительность этих контактов, в отличие от других, с которыми я знаком.}
\soul{Нет ни одного контакта, чтобы приносили только вред. Даже если к вам пришли злые, в вашем понятии, духи, в этом есть тоже польза. В этом вы можете увидеть каково зло и уже будете знать направление, как бороться.}
\soul{Скажите. У меня вопрос по этой теме, насчёт зла. Вы, в принципе, сказали, что добра и зла не существует?}
\soul{У вас - есть.}
\people{А-а. Понял.}
\soul{У вас. Ибо вы не знаете единства. В вашем понятии, есть добро и зло. Мы говорили вам, что нет такой личности, который был только злым или только добрым. И мы вам как-то доказывали вашей же логикой. Согласитесь.}
\people{Да.}
\people{Были ли ошибки в построении социализма, связанные с заблуждениями отдельных личностей, т.е. Ленина, Сталина…}
\soul{Мы говорили, что вы удивительны. Мы говорили вам, что одна личность может изменить весь мир и ,при этом, не обязательно быть гением, чтоб наломать дров. Чаще, у вас гении были только в том направлении, чтобы ломать, но не строить. Назовите мне хотя бы одного вашего вождя, который бы начинал с того, чтобы строить?}
\people{Да. Даже последние; Горбачёв и Ельцин, в основном, тоже, разрушали…}
\soul{О чём мы говорите? Вы опять понимаете физически. Разве мы говорили вам, о строить дома, строить заводы? Подумайте. Вы говорите о физическом. Мы же говорим об идеологии. Это ваше слово. Ваше. От слова ``идея''. И подумайте, тогда, о значении. И ваши вожди имели свои идеи. Они не разбирались, правы они или нет. Они знали лишь одно – другие идеи не верны.}
\people{А был ли бы путь социализма в России более успешным, если б сохранилось и упрочилось влияние религий и церкви?}
\soul{}
 Зачем вы говорите ``было бы''? Всё уже прошло. И вам надо начинать с этого мгновения. Давайте не будем говорить о прошлом. Давайте о настоящем. Далее, согласитесь, если все ваши жизни, все ваши века были связаны с религией, как можно убрать?
\people{Да. Но при этом она, всё-таки, чаще всего консервативна и поэтому, порой, и реакционна.}
\soul{Ваша религия – это, всего лишь одна сторона политики. Только и всего.}
\people{Могла ли идея социализма мирно ужиться с доктриной христианства? Теперь-то мы понимаем, надо было бы в этот путь.  Почему не получилось? Из-за ошибок личностей?}
\soul{Подумайте. Пришла одна личность и сказала: Я не верую! И подхватили массы. И при этом, пришли другие, в вашем понятии – ``не с Земли'', и помогли. Подумайте. Разве мог один человек так заворожить всю страну, говоря глупость? Вспомните, вспомните ваших вождей.}
\people{Какие силы тогда им помогли?}
\soul{В вашем понятии – чёрные. Вспомните вашу войну, и вспомните… И вы поймёте, что здесь или гипноз, в вашем понятии, или что-то ещё.}
\people{Может быть именно сейчас, церковь сможет поддержать и по-новому осветить идеи социализма? повышение роли Церкви, в настоящее время, – положительный инициатор?}
\soul{В политике - да. Она стала вести себя более активно в политике. А посмотрите, в духовном плане?  Как отрицала, так и отрицает. Она говорит о духовности, и любые движения души показывает, как ``бесовщину''.}
\people{Да, вы правы. Сейчас кажется грубой ошибкой большевиков - ликвидация самого трудолюбивого слоя крестьянства. Так называемых ``кулаков''. Вы разделяете эту точку зрения, и почему?}
\soul{В вашем понятии, кулак и бедняк. Согласитесь- не работая, не станешь кулаком и не станешь богатым. У вас же, - это ваша, ваша шутка. Как вы говорите? Вспомните.Ваша революция была - ``чтобы не стало богатых''.}
(счет)
 
\soul{Спрашивайте.}
\people{И все стали бедными… Да.}
\people{Почему в библии написано: ``блаженны нищие, ибо их есть…"}
\soul{Простите, в Библии было сказано о политике?}
\people{Нет. Там ни слова о политике.}
\soul{Мы же говорили вам, и говорили в начале. Политика – это, всего лишь механизм,- не более. И это - ваше отношение. Вспомните начало. Мы же говорим вам: политика без души – ничто. Это, всего лишь машина, которой вы не умеете управлять, и она вас заведет куда угодно, но только не к цели. И, к сожалению, многие, в вашем понятии, ``иные миры'', действуют вашей же игрушкой, вашей же политикой - против вас.}
\people{Но мы же, сейчас, не можем выбросить политику, как нечто…}
\soul{Мы не говорим вам ``избавиться''. Вспомните, приходил Христос, и вы также говорили ``мы не можем''. И что же? Как было так и осталось. И потому к вам придут мечём и огнём. Чтобы тогда вы поняли. И тогда, среди вас, будут звери-люди. Чтобы увидеть их, вы посмотрели на себя и подумали: ``не похож ли я на него''? И потому, вам трудно. И будет труднее. И потому, вы сделаете другой выбор и начнёте войны. Войны против себя. Согласитесь, воюя с другой страной, вы воюете с собой. Вы лишаете себя всех прелестей мира. Вы ищете войну, ибо не знаете, как быть. Вы говорите ``политика''. Да – политика ведёт войну. Машина, не умеющяя управлять. Согласитесь, не в споре рождается истина, - в диалоге. Вы же, мало кто спорите..спорите…}
(обрыв записи)
\soul{Ваша политика глупа, и потому - воюют. Мы говорили вам: да, идёт битва за вас. Но не только в этом понятии. Да, вашей политикой управляете не вы. Это вами управляют. Политика ваша – это, всего лишь, нити, связывающие вас с ``иными''.}
\people{Вы, сейчас, так говорили о войнах, как будто их видите. И, видимо, переводчик от этого страдает. Это что, действительно ближайшее наше будущее?}
\soul{Спрашивайте.}
\people{Вы на это не можете ответить, да? Видимо есть закономерность, в том, что социализм был дан человечеству в виде тяжкого испытания. И, как ``путь'', для нас, оказался ошибочным. Но, почему? Ведь идеи его близки к христианским?}
\soul{Нет. Ничто вам не надо в наказание. Ничто. Вы сами превращаете в наказание. Да, были идеи прекрасные, но, получая их, были искажены.}
\people{Но, может быть, в третьем тысячелетии мы примем социализм уже правильно, сменив политиков или мышление своё?}
\soul{Вы придёте, но не будете называть это социализмом. Не будете. Мы говорили вам, что уже при вашей жизни, даже сейчас, уже идёт изменение. Но не подумайте, что мы говорим о материальных благах. Мы говорим вам о духовном. Нас, и вас, тем более, должна интересовать духовная революция. Вы же спрашиваете, говорите и мечтаете, о физическом.}
\people{Да, это есть такое. Мы слишком привязаны к этому. Понимание материального блага для нас, подчас, однозначно и духовному продвижению. Вот, вы отмечали гениальность и великое значение Ленина для человечества. Почему же его идеи оказались несостоятельными? }
\soul{Вы или не слушали или неправильно поняли. Мы не говорили о гениальности. Мы говорили о гениальности суметь разрушить, о гениальности - доброе принести в зле. Далее. Никогда не говорите, что Ленин или ваши, другие, были контактёрами. Хотя, вы можно сказать и так, ибо нет человека, который бы не был с чем-то где-то связан. Далее. Да, он хотел многое. Но не видел. Согласитесь, сейчас, услышав всё, что здесь говорится и говорилось, и вы начнёте действовать, вряд ли вы будете действовать правильно.}
\people{А верно ли, что социализм оказался исторически преждевременным, что внедрению его идей помешала биологическая несовершенная природа человека с его хватательными, эгоистическими инстинктами?}
\soul{Мы же отвечали уже вам. Далее, нельзя говорить о биологических. Неужели вы думаете, что ваша душа – биологическая? И будет когда-нибудь другой биологический уровень, когда вы не будете хватать? Пока у вас есть биологический уровень – вы будете хватать.(уровень сознания. прим.)}
\people{Но, душа у нас, у каждого, не такая жадная, как тело?}
\soul{Душа не может жадничать просто по той причине, что всё, что вокруг – это её.}
\people{А почему именно Россия была избрана для эксперимента с социализмом?}
\soul{Вы так говорите, как будто мы уже говорили вам, что кто-то экспериментирует над вами. Осталось только спросить вам, кто это сделал.}
\people{Почему не другая страна, не США, Англия? Именно мы весь путь прошли.}
\soul{Мы говорили вам: вы не избраны, вы не любимы. Любимы все. Иначе бы, нарушили сами себе. Мы говорим вам, что вам надо сделать первый шаг. Согласитесь, кто-то должен сделать первый. Вы сделали первый шаг. Сделали,-  почти в ту сторону, но свернули. Свернули. И,- да, многое сделано было ``чужими'', в вашем понятии, чёрными силами. Но, согласитесь, это ваша вина, что они могут действовать на вас.}
\people{И успешно действовать.}
\soul{Далее. Ваше счастье. То ваше счастье, что вы не видите иных миров, иначе не было бы здесь вас. Все вы были б больны, ибо не готовы видеть всё. Поймите, вы пришли сюда с целью научиться, но вы были слишком хвастливы и сказали: ``я сделаю это с завязанными глазами''. И что же? Вы ослепли, вы забыли. Теперь не можете найти. Но в каждом из вас есть частица, и вы называете это душой, которая помнит всё. Она пытается пробиться сквозь ваше сознание, сквозь ваше подсознание. Подсознание понимает её, но не может, не может поверить во всё, ибо сознание блокирует. Далее, вы скажете: как же собака или иное животное? Она же не имеет сознания? Значит, должно быть умнее нас? Вот и подумайте, каково её сознание. Да, она тоже имеет сознание. Вы спрашивали, умнее ли дельфины? Умнее ли другие? И мы говорили вам, что мы не говорим, кто умнее, и кто глупее. Поймите, если душа обладает всем, то отдавая - она не теряет. И когда вы это поймёте, что отдавая, вы не теряете, тогда вам будет легче отдать, и тогда, вы не будете говорить ``политика, политика''.}
\people{Вот у нас ощущение, что Гера всё больше понимает своё подсознание, мы правильно это чувствуем?}
\soul{Скажите пожалуйста, как чувствуете вы, его подсознание?}
\people{Я его не чувствую.(Белимов)}
\soul{Если бы вы могли чувствовать подсознание, то тогда мы бы говорили уже о единстве.}
\people{Нет, ну у Геры что-то пробуждается. Он понимает больше, чем… я-то скептик и …}
\soul{Поймите, есть люди, считающие, что если скептик – то это плохо. Нет, очень много открытий и, в вашем понятии, гениии были скептиками. И потому искали и находили. Мы говорили вам о фанатизме и говорили вам о сомнениях. Скептизм движет вами. Скептизм в каждом из вас. Вы не верите – хотите пощупать. И потому ищете множество путей.}
\people{Но я понимаю, что изменение природы человека в сторону любви к ближнему, альтруизма, крайне долог и медлителен? Неужели только через 1000 лет будут какие-то изменения и люди смогут вновь, довольно успешно придти к воплощению идей социализма?}
\soul{Почему вы говорите такие большие сроки? Мы говорили вам: да, можно вечность потерять на то, можно и мгновение. Подумайте. Когда вы рождались, вы не помните, но согласитесь, что ребёнок не думал о социализме. Когда вы будете умирать, вы не будете об этом думать тоже. И тогда получается два момента в жизни: рождение и смерть. Тогда вы правдивы, всё остальное – вы лжёте. Мы говорили вам, что ребёнок рождается чистым. Да, в нём есть прошлые воспоминания, и они создают характер, будущий характер. Но, в этот мир он приходит для того, чтобы старое исправить. Вы же - не исправляете, а получаете новое. Ибо вы, вы врёте даже себе. Вы не верите себе и лжёте даже себе. Если вам что-либо не нравится, вы придумаете множество причин оправдать себя.}
(счёт)
\soul{Понятие – параллельные миры. Подумайте, что вы называете параллельными мирами? Это мгновения. Мгновения, которые могли бы стать другими. Согласитесь, что вы можете, что-то подумать иначе, что-то сделать иначе, вы можете нажать кнопку или не нажать –  это будет один из параллельных миров. Согласитесь, что ваши сны – параллельный мир. Ваши мысли – параллельный мир. Тогда, где ж находится ваша душа?}
\people{Как долог ещё путь понимания другими людьми этого? Озадачивает это. Гнетёт нас, исследователей аномальных явлений, что другие никак не могут к этому придти, что мы для них кажемся странными людьми.}
\soul{Простите. Первооткрыватели всегда были странными. Вспомните Христа. Неужели он был всем знаком? Он был более странный, чем вы. Но это одно из испытаний. Это - один из способов борьбы, борьбы с вами, с Христом. Признать его ``не таким''.}
\people{Понятно. Ну, а вдруг, мы заблуждаемся, а наши оппоненты правильно понимают историю?}
\soul{Давайте скажем так: нет среди вас на Земле ни одного, кто понимал бы правильно. Обязательно что-то где-то, но искажено. Разница лишь только в том, что больше или меньше. Но достаточно искажено так, чтобы не было видно правды.}
\people{Поддерживаете ли вы нас в наших исканиях, в наших поисках, в ответах на аномальные явления? Вы поддерживаете нас?}
\soul{Посудите логически, если бы мы вас не поддерживали – как бы мы пришли к вам?}
\people{Вот теперь, после краха социализма в мире, видно, что путь ряда стран к благоденствию и процветанию, оказался более правильным. Это тоже входило в задачи эксперимента с построением социализма?}
\soul{Мы говорили вам, что нет эксперимента. Далее, назовите мне страну, которая бы прекрасно сейчас бы жила. Не говорите мне о материальном, скажите о духовном. Чем больше материального вы нашли, тем меньше духовного. Посмотрите, вы мечтаете жить, как живут другие. А вы знаете их душу? Вы знаете, как живут они ``в себе''?}
\people{Ну, ощущения тех, кто встречался: они более радостны, более добры, потому}
что у них нет проблем экономических…
\soul{Давайте сделаем так: возьмите газету и почитайте статистику. Статистику вашу и их. Возьмите любую страну, которая, в вашем понятии, живёт прекрасно, и почитайте хотя бы уголовную хронику. Их, и вашу. И подумайте.}
\people{Ясно.}
\people{А можно ли жить богато и при всём при том быть человеком с душой, так сказать? Ведь не деньги портят человека, а отношение человека к деньгам портят его.}
\soul{Отношение. }
\people{Да.}
\soul{Живите. Мы не говорили вам стать бедными. Это у вас была такая цель. Мы же говорили вам: станьте богаты духовно. И дай бог, если у вас будет и в карманах.}
\people{Согласитесь, если у тебя в кармане много, то у кого-то не хватает. Как это понять?}
\soul{А вот вам и единство. Как вы понимаете единство? Только в этом?}
\people{Это одна из проблем единства.}
\soul{Поймите, одна из ошибок в вашей эволюции в том, чтобы всех уравнять. Что вы}
сейчас предлагаете уравнять всех, тогда всем будет одинаково плохо или одинаково хорошо.
\people{Получается, не надо уравниловки?}
\soul{Что вы хотели уравнять? Мы говорим вам: вы хотите уравнять материально. Мы говорим о единстве, говорим о духовности. Вы же мне говорите о деньгах. У вас много, у кого-то мало. Вот, подумайте, и решите этот вопрос. В конце концов, мы не пришли к вам шпаргалкой. Пришло время экзамена,и жестокого экзамена. Почему же мы должны вам подсказывать, даже это?}
\people{Экзамен-то не только для нас. Для всех стран, для всего общества.}
\soul{Давайте не будем говорить о всей стране. Мы ж говорили вам: вы хотите изменить всю страну, но не себя. Сумейте изменить себя, и тогда вы увидите многое  то, что не видите сейчас. И тогда вы увидите, если вы изменитесь в лучшую сторону, вы увидите, что это не так уж всё и страшно. Вы увидите, что духовно возросло, возросло во множество раз. Если вы же повернете в плохую сторону, то вы увидите: да, преступность увеличилась – как страшно жить.}
\people{Да, вы правы. Будем стремиться к улучшению духовного климата, хотя бы в себе самих.}
\soul{Поверьте, измените себя. Измените, и тогда вы будете знать, как делать это другим. Если вы не умеете рисовать, как вы научите делать это других? Как вы научите искусству, которым не обладаете сами?}
\people{А вот измениться духовно''… Что вы посоветуете для этого? Больше думать, читать? Какие труды, именно, читать? Или верить, да?}
\soul{Верить? Тогда церковь вам скажет, только верить, если скажет вам читать, то, только литературу, в их понятии, духовную. Нет, вы должны делать всё. Всё, что помогает вам вырасти. И, придя в церковь, вам скажут: читайте только духовную литературу. Согласитесь, вы видите литературу эту духовную? Если бы она была духовной и истинно духовна, разве  могли бы вы сложить её в сторону?}
\people{Т.е. то, к чему лежит душа – это, значит, духовное?}
\soul{Мы же говорили вам: верьте во что угодно. Вы спрашивали о шёпоте. Мы же вам говорили: да, сейчас у вас 2 перста, 3 перста, раньше было 5 и было 10. Что, в вашем понятии, изменилась вера? Изменился Бог? Что изменилось? Изменился только символ.}
\people{Я продолжу. Старый капитализм в России не уходил без боя. Оправдана ли была жестокость большевиков в борьбе за власть, когда противники уничтожались, и было пролито много крови.}
\soul{Мы говорили вам, о диалоге. Вспомните. Мы говорили вам, любое убийство с любыми целями не оправдывается никогда. Мы спрашивали вас: можете ли вы убить человека, который может уничтожить весь мир? Вы же не смогли ответить мне.}
\people{Мне кажется, я бы смог, если бы знал конкретно, что этот человек может уничтожить мир.}
\soul{Тогда вы уничтожите весь мир за место него. Подумайте, чем вы лучше его? Он может уничтожить весь мир, а вы можете уничтожить его. Значит, вы сильнее его! С какой целью вы будете это делать? Спасти весь мир? }
(счёт)
\soul{Когда-то спрашивали: почему Будда ушёл, оставив детей?}
\people{Да.}
\soul{Теперь подумайте. Если человек осознал, что Всё - есть родное, разве он покидал своих детей?}
\people{Это я понимаю. Но, тот, кого он покинул, не знает же этого и для него - он принёс боль.}
\soul{А вы можете привести пример, хотя бы один, из своей жизни, когда вас правильно поняли? Когда вы сделали добро и вас поняли, что вы сделали добро именно так, как вы хотели? Назовите мне хотя бы один пример, когда вас понимают так, как вы хотите.}
\people{Если буквально так, то не понимают, практически, никогда. Приблизительно - бывали случаи.}
\soul{Приблизительно,- вы делаете всё. Это и есть - ложь.}
\people{Кто, для нас, является Анна? (это ещё один участник контактов, первый раз про неё говорилось в контакте от 31-01-1994), кармически, как-то, она связана? Долги перед ней….}
\soul{Кармически? Да.}
\people{А долгов – нет у нас?(кармических. прим.)}
\soul{Что вас беспокоит? Что вас заставило задать вас этот вопрос?}
\people{Когда вы говорили, первый раз, что - найдя четвёртого…}
\soul{Нет, давайте откровенно, назовите, вначале, что вас волнует. Что вас волнует, о долгах.}
\people{Ну т.к. я не помню прошлой жизни, я не знаю…}
\soul{Хорошо, вы не помните прошлой жизни. Если мы сейчас вам скажем, кто кому что должен, это что-то изменит? Вы не сможете отдать или взять эти долги, ибо не помните.}
\people{Я не говорю вам конкретно: что, где. Я просто говорю: должен или нет?}
\soul{Давайте будем откровенными, что заставило задать вас этот вопрос?}
\people{Не знаю… Желание такое.}
\soul{Желание? Тогда подумайте, откуда оно у вас пришло. Подумайте, вспомните. Только будьте честны. И вы найдёте ответ.}
\people{Меня интересует вот что… Вы уже много раз упоминали о какой-то личности, которая может уничтожить весь мир. Вас этот вопрос, видимо, волнует. Действительно такая личность уже есть на Земле? И у него действительно есть эти нити, чтобы уничтожить мир?}
\soul{А вы подумайте. Даже в ваше время они были. Почему им не быть сейчас? Вспомните ваших вождей. Разве не могли они уничтожить весь мир? Вы представьте - если бы что-то или кто-то, не помешало им, и тогда бы Хиросима была бы …. }
(счёт)
\soul{Ваше счастье, что вы не помните. Вы не помните даже эту жизнь. Ваше счастье, что вы не помните все, и потому, не видите будущего.}
\people{Но, ведь есть среди людей те, которые уже поняли это? Они более интересной жизнью живут?}
\soul{Мы говорили вам, о памяти. Поймите, если бы вы помнили бы всё, вам было бы очень трудно. Ибо вы бы видели и начало и конец.}
\people{Так, понятно…}
\people{Угу. Ну, ладно… По России. Почему Бог оказался безучастным, наблюдая невероятные жестокости со стороны большевиков и коммунистов…}
\soul{Бог - это палка, следящяя за детьми и наказывает их? Почему вы обвиняете Бога и, при этом, не выполняете его заветы? Тогда в чём же его вина? Пробовали ли вы …?}
\people{Нет.}
\soul{Тогда в чём же вина его? Далее. Почему же он должен придти и отвечать за вас? Почему он должен исправлять ваши ошибки? Почему? Вы - родители, разве не должны заставлять детей исправлять их ошибки самим?}
\people{В этом смысл?}
\soul{Смысл? Обвиняйте себя. Прежде всего, себя.}
\people{Хорошо. Скажите, родись Ленин в России сейчас, он бы действовал по-другому в продвижении своих идей?}
\soul{Согласитесь, вы не видите свои ошибки. В его понятии, были бы будущие ошибки. Конечно. }
\people{По-другому, да? Долго скрывалась безжалостная и жестокая сущность Ленина. Эта его черта была как-то оправдана в то жестокое время?}
\soul{Вы говорите ``безжалостно''. Вы говорите с такой ненавистью, будто он вам что-то сделал.}
\people{Нет, я просто констатирую.}
\soul{Согласитесь, чтоб удержать пост столь высокий, надо иметь характеристики гораздо выше. Согласитесь, надо быть или очень добрым, или очень злым. Лишь только тогда можно удержаться на вершине. На вершине вашей власти. Спрашивайте.}
\people{Понятно.}
\people{Скажите, я в прошлый раз перекрестил его, и вы спросили: ``знете ли, что это за символ?'' Я ответил, что ``спаси и сохрани''. А, по-вашему, прав я? Или кто что в него вкладывает, так он и помогает?}
\soul{Нет. В нём - многое. Да,в нём есть ``спаси и сохрани'', и в нём же есть ``убей''. Вспомните. Вы крестите ребёнка и, этим же символом,- закапываете человека. Теперь подумайте. Множество.}
\people{Ну, а если я только верю в ``спаси и сохрани'', он мне так и поможет?}
\soul{Вы верите в то? Вы – верите?}
\people{Ну, не всегда.}
\people{Ну, почему… Я верю.}
\soul{Вы верите сознанием?}
\people{Ну, пока, может, мозгом.}
\soul{Научитесь верить сознанием. Вы пришли в этот мир и живёте сознанием. И потому, ваше сознание должно обладать тем же умением, что имеет и подсознание и далее. Только тогда вы можете понять единство и далее. Мы говорили вам о Христе и о семи телах. И он будет говорить вам о них. Вспомните. Согласитесь, что будут говорить только зная о них и владея ими. Вы же, не можете владеть всеми. Для вас они - хаос. Они борются между собой. И вы их называете ``сомнениями'', ``тревогами'', ``страхом'', ``чувствами'',- чем угодно.}
(счёт)
\soul{Действительно, была личность, которая создала Землю. Да, она пришла и строила. Строила Землю, создавала здесь жизнь. И приходили друзья и помогали… И были они… В вашем понятии, вы говорите: 7 дней. Нет. То было 7 друзей, и каждый из них создавал своё. Разделение труда – в вашем понятии. Но, пришёл день, когда они поссорились и ушли. Вы это называете… (прерывается контакт)… }
\soul{В один из дней была создана Луна. Щит. Мы говорили вам об этом, (…) как у вас в сказках. Вы называете кометами. Земля же,и мы говорили вам, имеет дух. Мы говорили вам – вспомните. Человек имеет душу. Город имеет комплекс душ, живящих в городе. И это, тоже - душа. Душа города. Далее – душа страны. И, согласитесь, значит, есть и душа Земли.}
\people{Душ, живущих на Земле?}
\soul{Вы их называете проще.}
1-2-3-4-5-6-7
\soul{(…) Вы интересны тем, что хотите многое познать. Ваше сознание опровергает то, что не верит, и ,всё же, с интересом наблюдает.}
\people{Я так понял – это первый шаг.  Раз наблюдает, значит, что-то пытается понять?}
\soul{Да. Но, чаще всего, оно пытается не понять, а корректировать по-своему. Что такое - вода? Это пепел. Вспомните химическую формулу. Вода – это пепел. Подумайте. Тогда, кто вы, если… ваше тело - это пепел. И спросите, какой огонь жжёт этот пепел?}
\people{Какой огонь?}
\soul{И спросите, как вы создаёте пепел. И в чём разница вашего пепла, и пепла вашего тела. Ваш пепел – это смерть. Вы сжигаете, убивая. Мы говорили вам… }
(счёт)
\soul{В одной из параллелей, ваша Земля уже умерла. В других параллелях – Земля цветов. И мы приходим и туда, и туда. Мы приходим и пытаемся зародить жизнь и там, и там. Не много разницы между вами и этими планетами. Про вас можно сказать: да, вы планета пепла. Но и про вас же, можно сказать: да, вы планета цветов! Вспомните. Cмотря как. Далее, вы можете выбирать, и никто не запретит вам. Это ваше право выбора, какой параллели придти к вам.  Вы как-то спрашивали, ``зачем умывание водой?'' Это - один из способов забыть. Забыть, что было видено. Или мы оставим травму. И тогда- мы пришли, получается, злом. Не забудьте сделать то.(имеется ввиду умыться после контакта. Прим.)}
(обрыв записи)
\soul{И вы называете это ``любовью'' и ``страхом''. Согласитесь, любовь и ненависть имеют одни корни. Согласитесь, рождение и смерть – один корень. Умирая здесь, вы рождаетесь там. Но согласитесь, умереть от старости или от убийства – разные вещи. Вам не грозит старость. Скорее всего, вы умрёте от того, что будете убивать друг друга. В том ваше несчастье, в том. Но, мы и говорили вам – вспомните, вы – планета цветов.}
(перепад)
\people{Переводчик плохо чувствует себя? Так мы ощущаем? Что с ним происходит?}
\soul{Мы говорили вам, о сознании, пытающемся увидеть.}
\people{А картины неприглядны, да?}
\soul{Мы говорили вам, о вашем ``счастье''.}
\people{Но почему вы нам предрекаете смерть не от старости, а от войны? Вроде бы никто не хочет воевать.}
\soul{Разве мы предрекали? Мы вам и говорили, что вы и планета цветов. Просто мы}
нарушаем вето. Вето не вмешиваться. ``Погибнут? Пусть погибают''. Нет! Мы пришли, чтобы помочь вам.
\people{Может, тогда лучше уехать за границу? Или скрыться в тайгу, пожить это время? Нет? Вы не посоветуете?}
\soul{О чём говорите вы?}
\people{Ну, если мы не хотим воевать и убивать своих близких.}
\soul{Неужели вы думаете, что вы можете скрыться? Скрыться… Вы сможете скрыться от себя? Назовите мне одну точку хотя бы, где вы можете скрыться от себя? Простите, даже в морге  вы этого не сделаете. Спрашивайте.}
\people{Вчера позвонила одна жительница нашего города – у неё пропал 18-летний сын, уже 2 недели. Она спрашивает о его судьбе и что его побудило уйти. Вы можете ответить? Я обещал ей что-то узнать.}
\soul{Мы могли бы это сделать, мы бы это сделали с удовольствием, но мы не имеем на то право. Поймите, мы не должны говорить вам конкретно ``то-то'' и ``то-то''. Поймите, мы пришли к вам, нарушая многие правила. Мы пришли к вам помочь вам. Поймите, вы находитесь так далеко от всех, что вы … (обрывается)}
(счёт)
\soul{Что вы где-то писали - в заповеднике. Поймите, в вашем физическом мире множество миров, и множество из них хотят занять ваше место. Идёт борьба. Борьба среди физических. И идёт борьба, и мы приходим, и помогаем вам удержаться, помогаем вам не погибнуть. Мы говорили вам, да, вы бессмертны, но, вспомните, мы же и говорили вам, что вы можете убить себя.}
Согласитесь, бессмертный может убить себя. Только он может убить себя, и больше никто. Но, согласитесь, что могут придти и ``уговорить'' сделать его это. Тогда, вы убьёте себя. Мы же приходим к вам остановить вас. Мы говорили вам об аналогии. Посмотрите на вашу жизнь и поймёте, что ваша жизнь на вашей планете, это миниатюра жизни всего космоса.
\people{Тогда, моё предположение, что этот парень ушёл, чтобы сделать самостоятельный выбор. И может быть, для него это, может быть, не плохо. Я могу сообщить это родителям?}
\soul{Мы не будем говорить вам ничто. Поймите, это сделает больно и нам, и ему.}
\people{Неужели его судьба будет трагичной?}
\soul{Мы не говорили о трагичности. Мы вам говорили о невозможности. Мы не можем отвечать вам ``то-то'' и ``то-то''. Поймите, вы, называя имена, вы ставите маяки, по которым к вам же придут другие. К вам же, но уже с другими целями. Поймите это! И потому не будем говорить вам имена. Вы говорите, можно называть имена умерших. Но подумайте,- имена умерших? А может быть, этот человек уже сейчас живой? И имя носит другое. Но, он-то живой! И маяк будет светить. Идёт борьба. Борьба, в вашем понятии, ``светлых'' и ``тёмных''. Но это только в вашем понятии. Тут всё гораздо сложнее. Поймите, одни говорят – не трогайте, пусть живут сами, другим – просто интересно на вас смотреть. Третьим, четвертым и далее – у каждого свои цели. У каждых свое понятие о вашей жизни, а кто-то, просто, приходит обогатиться за ваш счет. Или наоборот.}
1-2-3-4
\soul{Мы же, пришли к вам не физически, но мы используем физику, чтобы разговаривать с вами. Мы используем ваши тела, ваше сознание, ваше подсознание. И ваша душа видит это и хочет помочь, но слишком много груза, слишком много тел. И мы, всего лишь ничтожная единичка, которая хочет помочь этой душе освободиться. Вы, всего лишь сторонние наблюдатели. У вас внутри идёт война, а вы просто смотрите. ``Как интересно, интересно, что будет дальше? А если попробовать так и так'', вместо того, чтобы выбрать место среди этой битвы, перейти на чью-то сторону и воевать. Согласитесь, если к вам пришло чёрное, в том вина ваша, а не чёрных, ибо вы притянули их, ибо вы были способны принять их, и потому, вы, наказуемы, не они. Они, всего лишь хищники. Они, в вашем понятии, всего лишь комары, летящие на вашу плоть. Но, согласитесь, комар виноват в этом или вы? Но есть и Cолнце, что освещает ваше тело и даёт ему тепло. Согласитесь, если солнце, если им долго, слишком долго пользоваться, это Cолнце уже сожгёт ваше тело. Тело, которое ранее согревало…}
(конец сеанса)
Аоум. глава 16-06-94 г
Георгий Губин
16-06-94г
\people{Сегодня 16 июня. Мы собрались вновь, на сеанс связи. Готовы задать вам вопросы. Вы готовы отвечать?}
\soul{Спрашивайте}
\people{Хорошо. Мы уже некоторые вопросы задавали вразброс, всё-таки пол-года с вами разговариваем. Но сейчас нам надо систематизировано. Мы просим вас ответить на вопросы, вот, о духе, о душе, о жизни после смерти. Может быть такое? В чём различие состояний ``духа'' и ``души''? На ваш взгляд… Вы слышите нас?}
\soul{(…)}
\people{Мы хотим узнать, в чём различие на ваш взгляд, понятий ``дух'' и ``душа''?}
\soul{На наш или на ваш?}
\people{Ну…}
\soul{Есть ли в том разница? Да, душой вы заметите то, что в вас. Дух - извне. Вы должны прийти к духу.  Вы поняли?}
\people{Интересный ответ.  Что из себя представляет дух, и каково его воплощение, и есть ли оно? }
\soul{Но вы придёте. Если нет - к кому же пойдёте?}
\people{А как оно воплощается? Как оно выглядит?}
\soul{Простите, вы знаете это лучше нас. Вы же ищете. И вы имели. И вы ушли. И мы говорили вам. Вы  не помните.}
\people{Мы лишены памяти, к сожалению. Памяти наших перевоплощений.}
\soul{Лишены? Или не хотите? В том есть разница. Многие из вас помнят более, более, чем многие другие…}
\people{Я считаю…}
\soul{Вы не хотите, вы боитесь. Страх. Мы говорили вам, что самое сильное чувство в вас, к сожалению, страх. Страх потерять эту жизнь, пойти к другой. И вы думаете любопытство гонит вас узнать о других жизнях? Нет - страх. Вы ищете место ``поудобней'' в вашем понятии - подсознательно.}
\people{Ну, мы бы хотели этими контактами воспользоваться как раз для того, чтобы суметь много понять в этом мире и научиться пользоваться, - то, что вы нам советуете: убрать страх, раскрепоститься, суметь выходить на свое подсознание, на свою пост-память. Вы можете нас этому научить?}
\soul{Пока вы будете говорить ``воспользоваться'', мы ни чему вас не научим, и только будем вредить вам, не более. Ибо вы сами будете вредить себе, а винить - нас. Всмотритесь, что вы делаете? Вы спрашиваете и принимаете ответы только те, что нравятся вам, но не те, что нужны. Вы говорите ``ложь правдива'', только меряете через себя. Где же разум ваш? Разумом был отвергнут ваш бог. Разумом! А вы говорите о душе, о духе. Вы не познали душу. Душу… А хотите прийти к духу. Не правда ли, что вы похожи на тех, кто хочет перепрыгнуть, не видя, что взади, не помня, что было ранее? Как же вы научитесь ``ходить'', если когда-то умели, но забыли и не хотите вспомнить? Спрашивайте.}
\people{Хорошо. Тогда, что из себя представляет душа? Действительно ли, что она состоит из лептонов и весит от 2 до 7 граммов?}
\soul{Нет. Вы подумайте, что вы говорите? - ``Душа'' - и вы говорите о ``физике''. Тела ваши - да. Но тела ваши, всего лишь ``реакция'' души. А теперь представьте, какова ``реакция'' духа? А вы говорите ``грамм''!}
\people{Угу. Так, что из себя представляет ``душа''? Это - наши тела что ли?}
\soul{Нет. Вы не внимательны, мы вам ответили только что. Ваши тела - составляющее души. Разберитесь в телах ваших, и тогда будете знать. Мы говорили вам о луковице. Вы помните?}
\people{Помним.}
\soul{..Если будете очищать её. В вашем понятии, - она исчезнет. Но понятие, понятие ``луковицы'' осталось. Вот так же и душа ваша.}
\people{Хорошо. Откуда душа изначально появилась?}
\soul{Появилась? Вот вы говорите о начале и конце. Зачем же тогда мы говорили вам множество, о бесконечности времени, если вы спрашиваете ``начало'' и ``конец''? Почему вы говорите ``Бог'' и делаете его подобным себе, но не наоборот? ``Бог создал вас''. Этим вы говорите, что вы создали Бога! Подумайте, логически рассудите. Бог имеет ваши эмоции. Он умеет ненавидеть, он умеет любить, как и вы. Чем же тогда от вас отличается? Только силой? Подумайте, до чего же вы унизили бога.}
\people{Ну, это, видимо, по незнанию, а не специально.}
\soul{По незнанию? А как вы можете показать - ``по не знанию'' или ``не специально'', если вы сами говорите, что не помните и не знаете? Подумайте.}
\people{Угу. Мы продолжим…}
\soul{И мы говорили вам, о сознании, и когда-то, кто-то из вас сказал: ``дьявол - сознание''. Нет, вы не правы. Сознание – всего лишь только орудие дьявола, но не сам дьявол. Потому и говорят вам ``откройте двери'', в вашем понятии, души. Откройте! То есть не будьте рабами. Не будьте рабами дьявола. А чаще, вы и есть - дьяволята.}
\people{Угу. Ну, вот, вы ушли от ответа на вопрос ``откуда душа появилась''. Ну, тогда ответьте - душа эволюционирует?}
\soul{Если мы ``ушли'', тогда вы более невнимательны, чем вы думаете. }
\people{Ну, вечность… Ну, ладно, может тогда - душа эволюционирует? Она может расти, вбирать в себя новые знания, новые качества, наполняться какой-то энергией?}
\soul{Тогда подумайте и вспомните, вам же было сказано, и вы знаете - ``Большая душа'', ``Распахнутая душа'' - ваше выражение. Ваше. Неужели они родились  на пустом месте?}
\people{Ну, мы могли считать, что это просто метафора такая красивая.}
\soul{Метафора? Ваша жизнь, чаще, метафора. А то, что вы принимаете за ``метафору'' – то и есть Жизнь. Почему вы ищете новое, а жизнь прошлая, даже если вы не помните, но оставили знание, вы пренебрегаете ими и ищете новое, и этим отрицаете то, что было сказано вам ранее. Ну что ж, вы дойдёте, и даже на библию вашу скажете: ``старо, нам нужно новое''. Но забыли, что всё новое строится на старом, и вы должны быть подготовлены к новому. Вы же - не знаете. Если что говорят вам, удивляетесь - ``Как это нам ново! Неужели об этом даже говорили в прошлых веках?!'' А может быть ``в прошлых веках'' говорили, а сейчас, ``даже'', ещё говорят, но вы не поняли.}
\people{Вот, мы задаём ещё наивные вопросы, видимо, на ваш взгляд. Какова форма души или\и духа? Это она человекоподобна, или, допустим, она, как шарик выглядит, сфера какая-то, которая внутри нас? Как она всё-таки выглядит?}
\soul{Давайте скажем так. В вашем понятии, она не имеет формы и может приобрести любую, ибо, когда в вашем понятии, умирает, и душа покидает,- поймите, в вашем понятии ``душа покидает тело'', - нет, просто оно создаёт новые составляющие и не больше. Но будем говорить вашим языком. И сознание видит, видит душу, но так, как ей это больше нравится, и не больше того. Сознание придает ей форму. Сознание. Подумайте, как можно говорить о вечности души, если она имеет форму, если она имеет физические размеры и далее далее? Любая физика, рано или поздно, гибнет. Подумайте!}
\people{Скажите, а разум души существует?}
\soul{Да. Но не тот, что вы называете ``сознанием''. Не тот. }
\people{Оно более совершенно?..}
\soul{Чаще, вы слышите разум в боли, боли сердца, предчувствия.  Ибо, тела ваши не знают его язык и переводят по-своему. Перевод грубый, но всё-таки помогает вам.}
\people{Скажите, а существует ли среди духов иерархи?}
\soul{Духов? Да. }
\people{А среди душ уже такого нет? Они все равны?}
\soul{Поймите, поймите и будьте внимательны. Мы же говорили вам, что все миры подобны вам. Все! Есть только более ``выше'' и более ``ниже'', по вашим понятиям. Для многих, ваша жизнь является раем, а для многих, ваша жизнь - ад. Так же одну из жизней вы называете ``адом'', и одну из других жизней вы называете ``раем''. Вы поняли?}
\people{Да. Хотя это непривычно. Скажите, к чему стремится человек в качестве духа?}
\soul{В качестве духа? Человек? Это несовместимо.}
\people{Угу. А в качестве души у него есть цель впереди, к какой он стремится?}
\soul{Поймите, вы говорите ``человек''. Но вы понимаете это только как ``человек'' - как себя, как вашу плоть, но даже в слоге том есть вам понятие, что ж такое ``человек''. Подумайте, - ``вечный дух'', - а вы смертны, плоть ваша смертна. И пока вы не поймете истинное слово ``человек'', мы будем говорить вам ``человек и дух - разные. Несовместимо!''. И когда пойдёте выше, гораздо выше, тогда мы уже будем говорить  вам - ``человек и дух - одно!"}
\people{Скажите, есть ли степени совершенства у духов, и какие они?}
\soul{Мы же говорили вам, о подобии миров. Подумайте. И вами же сказано ``черная душа'', ``светлая душа''.  Вами. Опять ``метафора''?}
\people{Хорошо. А духи различаются по своим  качествам? Есть ли добрые, злые, равнодушные?}
\soul{Мы же ответили вам только что.}
\people{Ага, понятно. Надо полагать, что злые духи охотнее ``опекают'' и снисходят к людям, если мы чаще с ними, вроде как, встречаемся.}
\soul{Нет, охотней вы зовёте. Поймите, они блуждают, они ищут. Ищут, как хищники. А вы - хорошая добыча, ибо вам легче поверить в зло, чем в добро. Посмотрите на жизнь вашу. Что вы скажете о жизни?  ``Плоха''. Малейшая неприятность - и вы называете ``беда'', а если к вам придёт большое счастье, вы не заметите. Вы скажете ``маленькая радость - бед больше'', вот и подумайте. Вы, в вашем понятии ``ловите'', вы ``ловцы'', но не они.}
\people{Но у нас создается ощущение или впечатление, что добрые духи как-то неохотно общаются с человечеством.}
\soul{Неохотно? Может быть, вы закрываете двери? Неохотно? А вы вспомните, вы же читали Библию, и вам было сказано ``откройте двери'', а что делаете вы? Вы хотите, чтобы к вам пришли добрые, а вы в это время лежали на диване и попивали чай! Подумайте, вы - заслужили то? Вам проще творить зло. Проще, ибо для этого не надо много энергии, ибо она в вас, в вас, и ей легче управлять. }
\people{Меняются ли и улучшаются ли духи со временем своего развития?}
\soul{Зачем же тогда будут вам нужны веры?}
\people{Губить.}
\soul{…Если бы не менялись, зачем вам нужны веры? Зачем тогда к вам приходят? Потому и приходят к вам на смерть, на погибель. Отдают своё, чтобы вы росли. И даже если маленькое зернышко будет, маленькое, ничтожное - не затопчите, не сумеете. }
\people{Знаете… Мы…  Никто из нас не был в потустороннем мире, в реальных ситуациях. И вот, нас очень много вопросов интересует, как же там что происходит? Душа после смерти. Кем становится душа в мгновение смерти?}
\soul{В мгновение? Энергией. Вот вам и ``грамм''. У вас есть физики, пусть они ответят вам, что происходит, когда уходит и приходит энергия. Хотя бы вспомните Постулаты Бора.}
\people{Скажите, а энергия у разных людей умерших, видимо, разная? Кто-то накопил много хорошего, и у него энергетика высокая, а у другого - совсем низкая, он жил не правильно. Это зависит от способа жизни?}
\soul{Давайте не будем говорить о количестве энергии души. Давайте говорить о качестве. О качестве.}
\people{Качество… Угу… Но она действительно может быть разная, у разных людей - качество энергии?}
\soul{Ну, вы же разные. Вы же разные! Посмотрите на себя. Посмотрите на ваши качества. Посмотрите.}
\people{Праведник обладает более качественной и высокой энергией, и это ему как-то ``засчитывается'' на свете?}
\soul{Давайте спросим, ``праведник'' - относительно вас? Или относительно того мира, куда он уйдёт?.. Многие ваши ``праведники'' грешны более, чем вы.}
\people{Ну, на общечеловеческий взгляд? Сравним с Серафимом Саровским. Он для нас - ``вершина''.}
\soul{И что мы должны сказать?}
\people{Ну, высокая у него энергетика? Он - действительно накопил? И, может, далеко вознесётся… На какие-то другие уровни…}
\soul{Накопил? Поймите, если вы будете копить, то вы ни куда не вознесетесь, ибо вам дано даром. Вы должны помнить это. И даром должны отдавать. Но не копить, и - ни в коем случае. Поймите, чем более вы копите, тем более теряете, тем тяжелее вам будет подняться. Даже со стороны физики. Подумайте. Ибо, ваш мир тяжелее, чем тот, и значит, энергия ваша - тяжелее. Подумайте. Вы должны отдавать. Отдавать. И отдавая, вы будете получать более. Когда-то вами было сказано ``Природа не любит пустоты''. Вот вам, один из вариантов, почему отдавая, вы приобретаете.}
\people{Ну, это важное ваше замечание..}
\soul{Поймите, чем чище будет, в вашем понятии, чем густее, чем больше отдадите, тем больше войдет. Да, будет приходить к вам и злое и доброе, в вашем понятии, но Вы должны отсеять и выбрать. Если пришло к вам злое - не уничтожать, нет. Ибо тогда, вы будете не лучше тех злых. Должны, в вашем понятии, преобразовывать. Ну, а чтобы было ещё легче  - воспитывать вас…}
1-2-3-4
\people{Хорошо. Сохраняется ли индивидуальности души после смерти человека?}
\soul{Да, конечно. Иначе - зачем было говорить о карме?}
\people{Но индивидуальность сохраняется именно последнего воплощения, или же какого-то яркого предыдущего воплощения?}
\soul{Нет. Все.}
\people{Он может…}
\soul{Не ``может'', а ВСЕ. Все жизни ваши запечатлены, все жизни ваши вы помните у нас. Все, кем бы вы ни были когда-то и где-то.  Всех. И будущие - тоже.}
\people{А как можно кармические долги отдать не только через физическую боль, а другими какими-то способами? Есть возможность?}
\soul{Да, конечно. И почему вы говорите только о физической боли? Почему?}
\people{Ну, потому, что говорят - карма через болезни может проявляться. Старые долги отдает человек.}
\soul{Нет. Здесь много ошибок. В вашем понятии, те болезни, что получаются при рождении. Нет. Чаще, те болезни вы получаете и от родителей. Родители же - создали их сами. Сами. Кармой, которая в этой жизни. А физические боли… Неужели вы думаете, что физически вы можете избавиться? Нет, духовно. Духовно. И чем выше будете духовно, тем меньше будет влиять на вас карма, ибо вы отдадите больше. Мы же говорили вам - ``Сумейте отдать!''. Сумейте отдать и устраните, в вашем понятии, карму.}
\people{То есть, больше накапливать доброты и больше отдавать. Таким образом, карма…}
\soul{Не накапливать. Нет. Ни в коем случае.}
\people{Нет, не так я… вернее - развивать, да?}
\soul{Развивать?}
\people{Отдавать. }
\people{Отдавать, да.}
\soul{Поймите, если вы будете развивать физически, а у вас есть множество способов того, много ли будет вам? Да, вы можете физически развить, в вашем понятии, ``чакру'' до больших и огромных размеров, которые  могут занимать не то что вашу землю, но и более. Но это будет всего лишь искусственное и не более. И потому, говорят вам о душе. О душе. То, что делаете незадумываясь  – то истинно.}
\people{Скажите, пожалуйста, ну, для многих жителей  России, например… К примеру - схоластный  человек является (…)  Как бы вы оценили его опыт души?}
\soul{Давайте будем говорить так - оценять должны вы. Не мы. Поймите. Вы хотите стать марионеткой? Если вам скажем  ``то-то'' и ``то-то'', и вы будете верить нам? Вы опять пропустите через свое сито и сделаете выводы. И тогда, чем больше будет подобных вопросов, тем толще будут нити, что связывают нас и вас. И тогда, вы будете без воли. Подумайте, что вы делаете. Потому и говорим вам этим языком, не другим. Чтобы не было вас без воли. Вам вести.}
1-2-3..
\people{Но вам такая личность…}
\people{Но его душа сейчас - отдыхает?}
\soul{Нет. Нет душ, которые отдыхают!}
\people{Она трудится?}
\soul{Вы хотите отдыхать, но не души ваши. Поймите, для многих ``рай'', как вы говорите, более, более труден, чем вы живёте сейчас. Ибо душа ваша будет более и более трудиться.}
\people{Ну, не могли вы конкретно проследить путь именно этой души?}
\soul{Нет. Мы не будем прослеживать ни что. Поймите, мы говорили вам, что мы видим целое. Мы приходим к вам, как к мозаике, и потому, не называем имён. И вы когда-то нашли один из ответов, чтобы не привлечь имени. Далее,-  поймите, душа ваша не отдыхает никогда. Как вы можете представить рост души и отдых, когда она не растет? Подумайте.}
 
\people{(…)}
\soul{Далее, если хотите, то скажем вам - рожден человек. }
\people{Это - о только что заданном вопросе?}
\soul{Да}
\people{Уже воплощён?}
\soul{Спрашивайте далее.}
\people{Слава богу. Что уносит с собой душа, когда уходит отсюда с земного плана?}
\soul{Что? Ваш опыт.}
\people{Он ей помогает?}
\soul{Вам - не помогает?}
\people{Скажите, а вот этот накопленный опыт… Мы будем сюда возвращаться, до тех пор, пока всё-таки все не придут к душе к своей?}
\soul{Нет. Мы же говорили вам об иерархии. Подумайте, в чём заключается она? Не в званиях, что имеют у вас, а в силе, качестве тех сил. Да, есть множество, что приходят сюда, в вашем понятии - с бесконечности, и не могут подняться выше. О таких вы говорите - ``человек без души''. И есть те, кто только рожден и уже поднимается выше.  Ибо мы говорили вам, что сознание ваше не является главным. Вот, потому и дети, как вы говорите, ``святые''.}
\people{А вот если имеет место многократное такое перерождение душ, то откуда берутся всё новые и новые души? Ведь численность населения растет на планете.}
\soul{А теперь подумайте, если есть множество миров, почему бы нет? Далее, вы должны знать, в вашем понятии ``биомасса Земли'' не менялась уже очень давно! Вот вам и ответ.}
\people{А душа обладает способностью как-то делиться, размножаться?}
\soul{Подумайте, мы вам только что сказали. Мы вам сказали о биомассе. Тогда, давайте сделаем проще. Вспомните Теорию Относительности Эйнштейна. Вы помните? Помните? Всего только 2 пункта, и что они дают! }
\people{Какие пункты вы имеете в виду?}
\soul{…Такие простые, и что они дали? Вся физика ваша основывается на тех пунктах. Вы помните?}
\people{Скорость и масса что ли… Что?}
\soul{Вспомните. Вспомните, всего лишь только 2 правила. Всего 2 правила изменили вашу науку. Так же хотим, чтобы и вы умели делать то. Если мы говорим о биомассах, подумайте, откуда же берутся новые?}
\people{Видимо,  размножаются.}
\people{Скажите, душа может, так сказать, ``падать'' с потерей энергетики в более низшую физическую область?}
\soul{Если есть в вашем понятии ад и рай, значит - да.}
\people{Скажите, вот, из фактов перерождения человеческих душ, логически следует, что, в общем-то, наши дети не есть НАШИ дети. Просто, души совсем других людей, которые были в прошлом. Вот само…}
1-2-3-4…
\people{…Само осознание этого факта, что наши дети - есть не наши дети, в общем-то, может иметь довольно-таки далеко идущие последствия, для человечества, если оно это будет понимать. Может быть это страшно, или это не страшно? - Осознание этого.}
\soul{В том состоянии, в котором вы? Страшно. Потому, что вы не умеете любить всех, и вы делите на ``своё'' и на ``чужое''. Когда же вы будете видеть, что мир един, вы даже не сможете спросить ``чужие это дети или нет?'', ибо, для вас всё будет Родное и всё Единое. В состоянии, в котором вы находитесь здесь? - Да, это страшно. И потому, есть Силы, что запрещают вам делать то. Запрещают. Ибо - что будете делать? - ``Не моё!''. Вы же – делите,  дЕлитесь.}
1…
\soul{Спрашивайте.}
\people{Есть мнение, что душа после смерти возвращается в некое ``Вселенское Целое''. Так ли это?}
\soul{Если заслужили то, то - да. Подумайте, вы противоречите себе, вы спрашиваете: ``возвращается на землю или нет'', и говорите о ``целом''.}
\people{Ну, то есть это ``целое'' везде присутствует, в принципе. Никуда не уходить - уже будешь в целом? Надо только познать это?}
\soul{Поймите, в вашем понятии, существуют иерархии. В вашем понятии.}
\people{Да.}
\soul{Есть законы, но не ваши, которые ``заставляют'' идти выше. Если хотите, даже не ``заставляют'', ибо душа знает эти законы и подчиняется им не силою, как это делаете чаще вы. И потому, она сама выбирает, куда прийти. Сама выбирает, но по законам. Даже если она уходит в соседний мир, где законы не те не действуют, она всё равно делает по-своему, не зная о них. Ибо душа не обманывает себя в вас. В вашем понятии Инстинкт, Первородный Инстинкт. Но не тот, который вы называете сейчас.}
\people{Да}
\people{Скажите, сейчас какие-то были помехи. Они какого качества? Может съёмка мешает или свет, или что?}
\soul{Спрашивайте.}
\people{Хорошо, продолжим. Отделение души от тела - мучительно ли оно?}
\soul{Нет. Страшнее - ожидание смерти.}
\people{А как осуществляется отделение души от тела?}
\soul{Давайте скажем так. Душа не отделяется от тела, а скидывает, скидывает. Меняет свою энергетику. Подумайте,- душа… тела… Мы же говорили вам, что тела ваши - составляющие. Если хотите, это всего лишь – волна. Волна души.}
\people{Этот процесс мгновенный или постепенный?}
\soul{Вы создаете более точные приборы. Вы измеряете более короткие времена. Но это бесполезно. Тогда вы увидите, что нельзя измерить, нельзя, ибо это столь короткое мгновение, столь тонкие они, что не зафиксируете. И оно равно вашей жизни. О тех мгновениях говорим мы.Поймите, нет их. Нет. Нет ``настоящего'' и нет ``будущего'', и вся жизнь ваша - это мгновение, только лишь мгновение. Когда вы поймёте это, тогда вы пойдёте выше, и удивительно будет то, что это сделает ваша наука.}
\people{Пока мы скептически настроены к нашей науке, она…}
\soul{Мы говорили вам о…}
\people{О времени. Хорошо.}
\soul{О времени… Спрашивайте.}
\people{Органическая жизнь тела обязательно прекращается с отделением души?}
\soul{Нет. Есть, есть моменты, когда тело продолжает жить. ``Живые мертвецы''. И мы говорили вам ``люди без души'', их есть множество, множество среди вас. Ибо осталась энергия, осталась… Но не осталось то, что родило их. Рано или поздно, по закону даже вашей Физики, энергия там иссякнет. Но сколько она ещё дел сделать может. Спрашивайте.}
\people{Видит ли душа в миг смерти тот мир, куда ей суждено войти?}
\soul{Тот мир? Нет, видит более. Подумайте сами. Вам говорят о коридорах, о свете. Разве это мир? Нет, это множество.  Сады смерти и жизни вы не найдёте, не найдёте. И потому, наука ваша придёт и скажет - ``Нет времени''.}
\people{Скажите, а эволюция души имеет какое-то сходство с развитием бабочки из кокона?}
\soul{Нет. А вы подумайте, бабочка, рано или поздно, как бы она себя не вела, все равно была когда-то гусеницей, и  гусеница  подчиняется  всего лишь  законам природы и станет  бабочкой  в любом  случае,  что бы она ни делала - в любом понятии.}
\people{Хорошо. Какое чувство испытывает душа, когда осознает себя уже в Мире Духов?}
\soul{Всё зависит от того, как вы прожили жизнь вашу. Поймите, душа увидит правду, скинет одежду лжи, которой вы оправдываете свои поступки. И тогда, она уже будет смотреть на свою жизнь.}
\people{Ну, кому легче? Тот, который верил, что загробная жизнь есть, и он сразу ``включается'' в эту жизнь или тот, который совершенно отрицал, не верил, как это чаще всего у нас и бывает.}
\soul{В этом нет значения, ибо нет сознания того, нету - что не верило.}
\people{А уже сразу…}
\soul{Поймите же, качество души зависит не от того, что верило в загробную или нет, верило в бога или нет. Если вы даже не верили в бога, но, в вашем понятии, выполняли все Заповеди - бог примет вас. Поймите, вы его очеловечиваете и говорите, что он знает и любит славу.}
\people{Это, конечно, примитивное суждение. Мы так не считаем. Так. Сразу ли…}
\soul{Вот. Техника ваша…}
\people{Ну, мы хотели бы включить свет, потому что техника этого требует. Это плохо для переводчика?}
\soul{Уже ответили.}
\people{Спасибо большое. Скажите, сразу ли встречает дух тех, кого он знал на Земле, и которые умерли до него?}
\soul{Имеет ли  это значение - ``сразу или нет''. Лучше бы вы спросили - ``встречает или нет?''.}
\people{Так. Ну, хотя бы.}
\soul{Вы говорите о времени.}
\people{Встречает?}
\soul{Да.}
\people{А в любом случае?  Или только тех, которые, ну, переживали друг о друге, хотели этой встречи?}
\soul{Давайте скажем так - вы встречаете, давно невидевших,  здесь? }
\people{Да}
\soul{Вы - встречаете? Почему душа не может сделать того? Почему?}
\people{Это примерно так, да?}
\people{Встречает лишь своих друзей или близких людей?}
\soul{Близких. Близость не зависит от родства, ибо многие родственники более далеки.}
\people{Встречает одна душа или может быть несколько?}
\soul{Множество. Вы же говорите о друзьях.}
\people{Скажите, имеет ли особенности, для души, мгновенная насильственная смерть в плане осознания себя и понимания свершившегося? Или это безразлично?}
\soul{Мы же говорили вам о Переходе смерти, жизни, как о мгновении, а вы говорите…. Вы умираете медленно - да. Притом, сам переход мгновенен. Будьте внимательнее. Мы говорили вам о науке.}
\people{Скажите, вот в наше время уже довольно распространена догадка о жизни после смерти. Вот почему, на ваш взгляд, умершие по-прежнему исключительно редко дают о себе знать в подтверждении своей жизни после смерти?}
\soul{Хорошо. Давайте скажем это так - почему более низким мирам вы так редко являетесь? Почему? Вам что, лень поговорить? Вы подумайте, мы же говорили вам о рае и аде, и относительности. Вы помните? - Что жизнь ваша, для многих - рай или ад. И жизнь ваша, для многих – потусторонняя, ибо ``ниже'' вас - умирают и приходят к вам. Вы умираете - и идёте ``выше'' или ``ниже''. Но почему ж вы тогда не разговариваете? Вы - в потустороннем мире! Почему не разговариваете?}
\people{Для кого-то мы - потусторонний мир. Понятно.}
\people{Ну, если бы я умер, я бы постарался, может, своим друзьям…}
\soul{Вы когда-то уже умерли. Из другого мира вы пришли в этот мир. Относительно того мира - вы живёте в потустороннем. Где же ваше ``обещание''? Может в прошлой жизни вы тоже ``обещали''?}
\people{Хороший ответ. Спасибо.}
\soul{Далее. Вы вспомните, вспомните сны ваши. Неужели к вам никогда не приходили во снах ваши родственники, умершие и далее? Неужели того не было? Подумайте…}
1…
\soul{Спрашивайте.}
\people{Скажите, пожалуйста, вот вы неоднократно упоминали некоторые понятия из того, что мы называем Физикой, в частности, постулаты Бора, применительно, вот, к понятию души. Честно говоря, я не очень понял, какое отношение постулаты Бора к этому имеют. Ведь, в принципе, они говорят об устройстве атома: квантование орбит, переход… излучение  энергии при переходе с орбиты на орбиту. Какое это отношение имеет строение атома к тому, о чём вы вели речь до этого?}
\soul{Первое,- вы не внимательны. Мы говорим вам, об энергии. Далее. Если мир един, как мы разделим строение атома?   А вы сами состоите из тех же атомов и разделяете. Может вся жизнь ваша - это всего лишь колония атомов - тело ваше. Подумайте и посмотрите. И подумайте, вы говорите и спорите, о биополях, говорите, есть они или нет. А вы посмотрите атомы ваши, излучают или нет? И тогда вы можете доказать, что существуют ваши поля. Возьмите Физику вашу и примените тела ваши, ибо физика и тела ваши - одно. И если атомы излучают и принимают, то значит и тело ваше…}
1-2-3…
 (Обсуждение контакта)
 
 
\people{Ну, какие ощущения были? Вот мы не много сейчас беседовали, наверно минут 30 так, ну, по сравнению… сравнительно немного. Вот. Какие у тебя были ощущения? Что ты видел. Ты несколько раз прерывался. У тебя что-то, может насморк мешает? Что у тебя?}
\people{(Харитонов) Да, точно - насморк. Дышать нечем.}
\people{Дышать нечем? Так, что ты помнишь из наших вопросов что-либо?}
\people{(Харитонов) Да вроде как помню, а повторить я не смогу.}
\people{Ну, картинки опять были или цвета может, ощущения какие-нибудь?}
\people{(Харитонов) Да нет, ощущений нет вроде (…)}
\people{Ну, вот ты не помнишь, мешал тебе свет? Мы включали свет и мне кажется ,у тебя то ли глаза резь была, то ли насморк. Ты открывал глаза, это впервые у тебя было, чтоб ты открывал глаза.}
\people{В полной темноте открывал глаза.}
\people{Ну, вот, я смотрю, насморк тебе сильно мешает видимо. Да?}
\people{(Харитонов) Ну, вообще - что-то помню…}
\people{Вот ты рассказывай… Ну что давайте сделаем перерыв? Или продолжим? Как сейчас по состоянию его - или тогда прекратим сеанс.}
\people{Время восемь. Мы без пятнадцати семь начали. Час с лишним.}
\people{(Харитонов) Не… ну, вот такое, по сравнению с ``раньше'' мне как-то больше что-то ощущал.}
 Помню что-то больше.
\people{Больше помнишь?}
\people{(Харитонов) Ну, раньше как-то было так, знаете…  Вот встал - ничего не знаю ни чего не помню, а сейчас, что-то вроде как типа … знаете… вот, сон какой-то видел, вроде бы и помнишь, а вроде бы и не знаешь, как его рассказать.}
\people{Ты как кино посмотрел, а сразу  пересказать не можешь, да?}
\people{(Харитонов) Ну, да. Типа, может того.}
\people{Ну, вообще-то, я тебе скажу, ответы были хорошие. Конкретика даже кое-какая была. То есть мы удовлетворены сеансом, конечно. Ну, ладно, давайте отдохнём тогда, сделаем перерыв, а там ясно будет.}
 (Конец записи)
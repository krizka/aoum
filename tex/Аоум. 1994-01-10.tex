Аоум. глава 3-я 10-01-1994г
Георгий Губин
В поздний час пожелавший отрешиться от мира сказал:
Нынче к богу уйду я, мне дом мой обузою стал.
Кто меня колдовством у порога держал моего?"
Бог сказал ему: ``Я''.
Человек не услышал его.
Перед ним на постели, во сне безмятежно дыша,
Молодая жена прижимала к груди малыша.
"Кто они - порождения майи?'' - спросил человек.
Бог сказал ему: ``Я''.
Ничего не слыхал человек.
Пожелавший от мира уйти встал и крикнул: ``Где ты, божество?"
Бог сказал ему: ``Здесь''.
Человек не услышал его.
Завозился ребенок, заплакал во сне, завздыхал.
Бог сказал: ``Возвратись''.
Но никто его не услыхал.
Бог вздохнул и воскликнул: ``Увы! Будь по-твоему, пусть. 
Только где ты найдешь меня, если я здесь остаюсь?''

    Рабиндранат Тагор.

10-01-1994 
  * Вы попросили  привести четвертого. Так?

\soul{ВЫ ПРАВЫ.}
\people{Я не расслышал.}
\soul{Вы правы. Дайте счёт.}
\comment{1,2,3}
\soul{Вы правы.}
\people{Так, кажется, я нашел четвертого и даже убежден в этом. Это Олег, так?}
\soul{Вы убеждены?}
\people{Да. Он рулевой.}
\soul{Он изменил вашу жизнь?}
\people{Да}
\soul{Вы ищете четвертого? }
\people{Я нашел его.}
\soul{Я подскажу вам: вы все изменили друг другу жизнь. Вы были вместе, и, даже сейчас, но вы не видите этого. Скажу более - вы все убийцы и все жертвы.}
\people{Понятно. Это я подозревал.}
\soul{А теперь, о книге. Мы говорили с вами, что четверо - это один.}
\people{Да.}
\soul{Вы теперь поняли, кто пишет книгу?}
\people{Ага, все четверо.}
\soul{Спрашивайте. Вы спросите, как долго мы пишем?}
\people{Да нет, зачем?}
\soul{Будьте внимательны в ваших ответах.}
\people{Стараюсь. Да, ещё такой возникает вопрос - в принципе, все люди меняют жизнь друг друга в той или иной степени?}
\soul{Да, вы правы. Но существуют повороты, решающие судьбу. Это первое. Подсказка в этой жизни. В прошлой жизни вы меняли её не подсказками. Вы поняли меня?}
 \people{В прошлой жизни мы меняли жизнь друг друга не подсказками. То есть, вы скажите нечто, только тогда, когда мы приведем вам этого четвёртого?}
\soul{Мы скажем больше.}
\people{Всё понятно. Хорошо.}
\soul{Но, найдя четвертого, у вас будут другие вопросы. И это уже не подсказка.}
\people{Ну… вообще, конечно, наверняка глупо, но вы говорите, что нас сейчас четверо, в этой… Здесь, в этом месте нас четверо.}
\soul{Вы обобщаете.}
\people{Ну, нас, с переводчиком, уже четверо, вместе с переводчиком?}
\soul{Нет. Будьте внимательные в ответах, я предупреждал вас. Я вам не скажу сейчас ничто. Послушав кассету, обратите внимание, до того, как я вас предупредил.}
\people{Хорошо. Спасибо. Вы сейчас картинку сможете ему давать? }
\soul{Да.}
\people{Положение улучшилось с его мозгом?}
\soul{Нет}
\people{Продолжает ухудшаться, да?}
\soul{В нашем понятии - да.}
\people{Ну, в вашем понятии, в смысле, мы вас и спрашиваем.  В прошлом контакте меня удивила ваша интонация, когда я сказал: а поможет ли музыка, вы, по-моему, так немножко ошалело сказали: ``да, конечно''.}
\soul{Мы работаем в мире эмоций.}
\people{Но эта интонация в ответе… то есть,  надо слушать музыку, чтобы было лучше?}
\soul{Нет, каков вопрос, таков и ответ.}
\people{Понятно.}
\soul{Вам помогает музыка?}
\people{Да. }
\soul{Зачем спрашиваете?}
\people{Понятно. В  3-м томе …он весь прочёл? Вы знаете, что он знает, в 3-м томе?(имеется ввиду 3-й том  книги - прим.)}
\soul{Он не прочёл его весь.}
\people{Значит, вы не имеете представление, что в 3-м томе?}
\soul{Мы не входим в это. Это ваше личное, и мы не имеем права. Мы говорили вам о законах.}
\people{Да вот, я не понял. Есть, какие-то правила, которые нельзя говорить? Т.е. мы пока еще не доросли до этого. Ну, в принципе, правильно.}
\soul{У нас есть законы, в которые мы не вправе вмешиваться, не нарушив их.}
\people{Ясно. Но, как вас понять, если вы говорите, что вы знаете не больше чем переводчик? Так ведь вы говорили?}
\soul{Мы говорим то, что позволяет он. Мы говорили вам - мозг закрыт.}
\people{Скажите, как он мог вам позволить… когда он не имеет представления, что делал товарищ Белимов, третий, в 71-77 годах?}
\soul{Разве? Мы говорили о  единстве 4-х в одном.}
\people{А-а-а… То есть, все знают обо всём? О всех четверых? Получается - так?}
\soul{Вы, не всё знаете о себе? }
\people{Ну, я не всё помню о себе, скажем так. То есть, принцип голографии, когда голографическую пластинку разрезают на тысячу частей, но на каждой этой части воспроизводит целое, тут верна? }
\soul{Да.}
\people{Хорошо, третий – это Белимов. Это я правильно понял?}
\soul{Мы говорили вам, что вы должны найти всё сами. Не будьте глупы и не считайте нас – это первое. Второе – помните о поправке.}
\people{О какой?}
\soul{Вы забывчивы.}
\people{Это на этой кассете? А, ну это я помню, что мы коренным образом изменяли жизнь друг друга.}
\soul{Нет, послушайте внимательно, мы не будем повторять.}
\people{Что-то я ещё хотел спросить.. Без записей трудно.. Какую-нибудь дезинформацию вы нам давали специально, чтобы мы рассекретили, что это дезинформация?}
\soul{В вашем понятии - нет.}
\people{А в вашем?}
\soul{Всё - дезиформация. Нам приходится говорить вашим языком.}
\people{Нет, я имею в виду конкретные ответы на конкретные вопросы. Когда вы нам ответили конкретно. }
\soul{Вам отвечали не конкретно?}
\people{Конкретно – это когда, допустим: да/нет. Ну-у, отвечали  конкретно.}
\soul{Вы вели так диалог?}
\people{А-а-а… Это он вёл так диалог?}
\soul{Вы ранее вели так диалог. Что вам не понравилось?(т.е. первые контакты, знаками руки .Велись с августа 93г. По дек.93г. )}
\people{У меня создается впечатление, что с одной стороны вы не знаете наше будущее, но, с другой стороны, вы говорите, что если бы был человек, то'' он отказался бы слушать то, что ему скажут''. Это случилось.}
\soul{Мы не говорим вам прямо.}
\people{Естественно. Но, откуда вы могли знать, что есть человек, который откажется слушать то, что ему скажут. Откуда?}
\soul{Думайте.}
\people{Это знает переводчик?}
\soul{Нет, это знаете и вы.}
\people{Т.е. все наши мысли где-то там… ну, не где-то… тут у вас, скажем так, видны?}
\soul{Думайте.}
\people{Ясно, не скажете.}
\soul{Задавайте.}
\people{Извините, про символ фашистского креста хотел бы продолжить.}
\soul{Продолжить?}
\people{Да.}
\soul{Мы говорили с вами?}
\people{Я лично не говорил, говорил товарищ Белимов. Я так же под вас подстраиваюсь, так как мы едины, то считайте, что говорил я.}
\soul{Мы не знаем такого символа – это первое. И вы сделали вывод о единстве – это второе.}
\people{Извините, вы не знаете такого символа?}
\soul{Мы не знаем, почему вы сделали вывод о единстве. Будьте внимательны.}
\people{Скажем так, единство взглядов - это единство?}
\soul{Нет.}
\people{Нет?}
\soul{Вы встречались с единством взглядов?}
\people{Да, безусловно.}
\soul{Приведите.(пример. прим.)}
\people{Ну, допустим, так вам конкретное имя надо?}
\soul{Нет, вспомните, чтобы вы не спорили. Любой спор – это уже не единство – это первое. И второе: вы не одинаковые. Даже, говоря о голографии, вы не повторите с осколков полной картины.}
\people{Да, она будет размыта, всё правильно.}
\soul{Иначе нам бы не пришлось бы говорить о соединении.}
\people{Угу, ясно.}
\soul{Для вас здесь нет новизны? Или для вас это старое?}
\people{Вот я и говорю,- новое. Структура та же, а цвет другой.}
\soul{Вас приведут в тупик подобные рассуждения. Живите на этой песчинке. Вокруг вас их множество, вы не хотите их видеть - это ваши проблемы.}
\people{Это понятно.}
\soul{Но не задерживайте других. Вы делаете это - минимум, троим.}
\people{Лично я делаю? }
\soul{Лично вы. Мы говорили с вами о соединении. Мы дали вам часть ответов?}
\people{Н-да, но вы сказали - лично я, и я это  привел ко всем четверым потом.}
\soul{Вспомните поправку на  магнитофоне, и вы найдете третьего.}
\people{Простите, 3-го или 4-го?}
\soul{Мы говорили о третьем.}
\people{Вы говорили о 4-м, насколько мне известно.}
\soul{Послушайте кассету…}
\people{Хорошо.}
\soul{…и вы поймёте – это первое. Второе – счёт придуман вами. Здесь нет понятия ``первого'', ``второго'' и ``четвёртого'', но нам удобней говорить с вами так. Хотя, для нас, это вредно.}
\people{А есть какой-то компромисс, чтоб и для вас не вредно, и мы поняли?}
\soul{Нет. Нам с вами тяжело работать и нам приходиться жертвовать, разговаривая с вами. Но это не та жертва, о которой вы ведёте речь.}
\people{Т.е. от вас, что-то уходит, так скажем.}
\soul{Нет. Сложнее работать.}
\people{Скажите, вас трудности пугают или наоборот, удесятеряют силы, так скажем?}
\soul{Как ответите вы, если я вам задам подобное?}
\people{Я отвечу, что - когда как. Когда удесятеряет силу, а когда пасуем.}
\soul{Всё зависит от индивидуума. Вы спрашивали, существуют ли у нас индивидуумы. Мы говорили вам о быстротечности нашей жизни.}
\people{Да.}
\soul{Мы скажем вам более: мы не прерываем контакт, ибо, с вами говорит одна личность. Вы поняли?}
\people{Угу.}
\soul{Сопоставьте с подсказкой, но помните: самое главное, что вы увидите в этой записи - это подсказка и конец записи.}
\people{То есть, вы уже знаете, какой конец?}
\soul{Да.}
\people{Интересно. Хорошо.}
\soul{Ибо, он будет создан нами, не вами.}
\people{Хорошо. Если это самое главное то… Значит вы…}
\soul{Вы пришли к ответу сами, но, вы любите подтверждение других.}
\people{Верно.}
\soul{Вы хотите увидеть мир чужими глазами. Это ваша…}
\people{Ошибка?}
\soul{Это больше даже чем ошибка.  }
\people{Я к чему говорю, если это будет самое главное в нашем разговоре, то… и говорить то не о чем больше? Если нам только это надо узнать. }
\soul{Разве?}
\people{Ну, согласитесь, если остальное будет всё ``ветер'', так скажем…}
\soul{Вы всё считаете ``ветром"? Вы читаете книги.}
\people{Да.}
\soul{Что важно вам там? Если вы будете жить на голом пестике, не имея света и листьев, вам будет нравиться этот дом?}
\people{Угу, понятно. Значит вы, в прошлом говорили, что были в Шамбале всё-таки, да? Т.е. прошли через неё?}
\soul{Вы будете там.}
\people{Это вы тоже говорили, я помню. Вы были, да? Т.е. все были? Но вы ушли оттуда?}
\soul{Нет. Мы не говорили, что были там и ушли оттуда. }
\people{Нет, но…}
\soul{Если я говорю вам: вы были там, то остальное получите из поправки.}
\people{Угу. Помню, на прошлом контакте, когда 3-й был, из нас 4-ых, я так понял, и он сказал насчет Шамбалы, вы сказали ``мы были там'' и сейчас связь, типа того, поддерживаем.}
\soul{Вы не внимательны. Мы говорили, что ОН будет там.}
\people{Ага, ну ладно. Кассету не долго поднять.}
\soul{Это первое. И второе, - если вы говорите, что были где-то, вы можете иметь(ввиду) не только себя, но и ваш народ. Индивидуально - нет}
\people{Ну, я понял. Допустим, к вам кто-то пришел, кто был в Шамбале, так? }
\soul{Нет. Вы забываете о единстве.}
\people{Я, чё-то, не свяжу два узла…}
\soul{Если вы свяжете, у вас будут другие вопросы, и вы будете на другом уровне контакта.}
\people{Наверное. Четвёртый, которого я нашёл, боится ,значит, что будут побочные последствия,  не в лучшую сторону, от этих контактов. У него страх. Что вы нам посоветуете?}
\soul{Всё зависит от вас. И мы можем принести побочные явления, если вы хотите того.}
\people{Да? А-а… Ну, я понял, это насчёт нерва в зубе? Это ваша работа?(после контакта у переводчика перестал болеть зуб.прим.)}
\soul{Нет. У вас хватает сил сделать это и самим.}
\people{Угу. }
\soul{Мы, лишь только  корректируем. Мы говорили вам.}
\people{А-а-а. Ну да, ну да. Т.е. вы, определенного рода…}
\soul{Если мы скажем, что вы будете молодеть и у вас будут расти новые зубы, это не значит, что это делаем мы, это будете делать вы.}
\people{Т.е. у нас хватит сил на всё, что угодно? Понятно. Ну, я к этому пришёл , только не знаю, как эти силы… научиться управлять и вообще знать где это, что это. Вот.}
\soul{Вы выходите на контакты, вы активны, мы пассивны. Ибо, мы имеем законы не вмешиваться, но, мы не противимся, и, скажем более, - рады вам.}
\people{Понятно. Я думал насчет этого, что, грубо говоря, ``не лезь туда, куда тебя не просят"?}
\soul{Мы не говорили этого.}
\people{Ну, это я сказал…}
\soul{Мы говорили, что вы активны. Мы не говорили о вреде. Всё творит не деяние.}
\people{Мысль. Ну, это я уяснил.}
\soul{Какую мысль вы несёте, имея контакт.  Цель?}
\people{Лично я? Контакт?}
\soul{Да. Ваша цель?}
\people{Ну, по своей природе я ненасытен в знаниях, так скажем, и, соответственно, делаю кое-какие выводы из того, что я знаю. И, потом проверяю их в жизни. }
\soul{Вы подгоняете. (под ответ. прим.),}
\people{Возможно. Однако, получается довольно часто. }
\soul{Вы искусны в этом. Вы живёте в мире лжи. Не бойтесь ошибиться. Не ошибается лишь тот, кто ничего не делает. Да, вы можете споткнуться идя. Но, или вы встанете сами, или вам помогут. Помните это.}
\people{Ясно.}
\soul{Вы должны идти надеждой.}
\people{С верой, скажем так.}
\soul{Да.}
\people{И, если упал, не прекращай  свою веру?}
\soul{Нет. Иначе, это не была вера – это был лишь только обман. Ложь самому себе. Вы караетесь не за ложь соседу,  вы караетесь за ложь себе.}
\people{Понятно.}
\soul{Мы говорили вам, лжи не существует, и скажем более: лжи нет, но есть предательство.}
\people{Понял, наверное.}
\soul{Вы предаёте себя. Вы можете сказать многое, и мы поверим вам.}
\people{Я помню, что вы говорили. Вопрос такой: есть теория вибрации, резонанса. Вы знаете о такой?}
\soul{Спрашивайте.}
\people{Вот, есть такая теория резонанса.}
\soul{Ответ был дан вам, и был дан в блокнотике. Спрашивайте.(до контактов переводчик вел в блокноте изречения, приходящие в голову при рассуждениях на эзотерические темы. прим. ) }
\people{Хорошо. В этой теории, кстати, один друг, 4-й кстати, испытал это состояние и весьма напуган,так скажем, разными непонятностями, которые с ним происходили. Там увидел кубы, треугольники и всё это в цвете и после этого сеанса встал, посмотрел на зелёный цвет, и это был не такой зелёный, как сейчас виден, а намного содержательнее, скажем так, ярче.}
\soul{Спрашивайте.}
\people{Вопрос такой: это происходит действительно из-за резонанса, скажем так?}
\soul{Это резонанс. Не будем говорить о дефектах.(восприятия и декодирования мозга.прим.)}
\people{Понятно.}
\soul{Мы говорили вам о деянии и мысли. Если вы ищете зло, вы найдете его, если вы несёте добро, то найдете добро. Если не ищете ничего, то ничего и не найдете.}
\people{Можно ли искать добро, если я злой, и я ищу добро? Я хочу найти его и у меня есть мысль измениться. Т.е. переполюсовать себя можно?}
\soul{Вы обязаны сделать это. Мы говорили вам о реинкарнации. Они созданы для того, чтобы учиться. Учиться на своих ошибках. }
\people{Да, но как учиться, если ты не помнишь, как ты это сделал?}
\soul{У вас есть сердце. }
\people{Ясно.}
\soul{Оно помнит всё – это первое. Второе, какое у вас отношение к людям, – в этом вы можете найти отгадку прошлого.}
\people{Ясно.}
\soul{Ничто не проходит бесследно, будьте внимательны. Вы не хотите, вы боитесь знаний и поэтому не находите их.Да, мы говорили вам,помнить всё - наказание. Наказание, если  вы не имеете сил помнить всё. }
\people{Так, вот опять же, я попытаюсь поймать вас на противоречии. Вы сказали, что все мы были и жертвой и, так сказать, противоположностью. }
\soul{Вы были и убийцами и жертвами, и, поверьте, все.}
\people{Т.е. друг от друга?(принимали смерть.прим.)}
\soul{Мы говорили о четверых.}
\people{Я и говорю насчёт 4-х, применительно ``друг от  друга''. Значит, каждый из нас помнит всё. Тогда, почему он, конкретно он, переводчик, 2 раза помнит вещи не его?}
\soul{В вашем вопросе и ответ, он помнит только 2 раза. И, согласитесь, если мы дадим ему свободу помнить, когда он был не жертвой, то мы не сможем работать с ним вообще.}
\people{…}
\soul{Вы оплатили долги меж собой.}
\people{А-а-а. Значит, сейчас будет заключительная стадия какая-то. Я так понимаю, в этой жизни именно, да?}
\soul{Всё зависит от вас.}
\people{Должна быть, скажем так.}
\soul{Мы вам подскажем далее. Вы не внимательны, вспомните последние фразы, когда вы стирали. Повторите их.}
\people{Сейчас?}
\soul{Да.}
\people{``Андрюша, что ж ты? Мне же больно.''}
\soul{Далее.}
\people{Далее? Далее он ничего не говорил.}
\soul{Разве?}
\people{А-а-а! ``Что ж ты делаешь! Ты ж мой лучший друг''.}
\soul{Далее.}
\people{Ещё далее? Не помню.}
\soul{``Вы убиты будете так же.'' Вы не помните разве?}
\people{А.Честно говоря, не помню, но вы напомнили. Верно. Т.е. в этой жизни получается… Что-то я не свяжу… Чтож, мы, вроде-как, должны харакири друг другу сделать, чтобы уж примириться совсем?}
\soul{Вы нелогичны. Мы говорили о прошлой жизни. Хорошо, давайте рассуждать вашими мерами. Согласитесь, речь шла о царе. Вы помните?}
\people{Помню.}
\soul{И небольшое согласие приводит к смерти.}
\people{Не-согласие, простите.}
\soul{В каком случае это может быть?}
\people{Ну, когда это подпольщики, наверно. Боязнь, что он предаст.}
\soul{Вы слышали фразу о предательстве?}
\people{Нет.}
\soul{Тогда подумайте, почему такая реакция.}
\people{Т.е. кто-то, не соглашаясь с кем-то… Т.е. мы, не соглашаясь между собой… Это приводит к таким печальным последствиям, так? Одно несогласие?}
\soul{Давайте по-другому.}
\people{Давайте. Я, видимо, не правильно подумал. }
\soul{Вы договариваетесь о чем-то, достаточно крупном, могущем изменить вашу жизнь, и, чаще, в летальном исходе. И если с вами кто-то не согласен, чем крупнее, ваша задача,  тем больше шансов, что вы уберёте. Вы опять не поняли?}
\people{Ну,я так понял, что они пытались, договаривались убить царя, а он был не согласен, поэтому они его… такая реакция произошла.}
\soul{Теперь вспомните, что произошло с остальными. Вы не можете вспомнить? Тогда догадайтесь. (пауза) Вы все были казнены. Но, не будьте горды – не вы убивали царя. Вас убила ваша мечта.  Вы мечтали о власти, и ваша же власть убила вас. И вспомните последние слова.}
\people{Так, ну, это уже что-то проясняется. ``Вас же так тоже убьют''-  да? Тогда смысл как бы в привидении сюда четвертого вроде как теряется. Вроде, вы нам, как бы говорите:- живите спокойно и всё. Живите, как живёте.}
\soul{Мы говорили о песчинке, мы говорили о голографии, мы говорили о пестике, без листов. Почему вы не помните? Вы хотите помнить прошлое, но вы не помните настоящего. Как вы можете решать будущее?}
\people{Понятно, нам ещё далеко. Такой вопрос: теоретически, случись так, что не получится привести сюда четвёртого, мы обязаны будем это сделать в следующей жизни?}
\soul{Мы говорили вам, что вы обязаны привести 4-го?}
\people{Нет, не говорили, но это я так говорю.}
\soul{Давайте говорить о вас, вы должны найти друг друга,- и мы придём к вам.}
\people{Ну, это я уже догадался, только в каком ракурсе вы придете, я не понял. }
\soul{Вы найдите, и мы придём к вам.}
\people{Ну, я-то нашёл. Нам надо всем четверым вместе собраться или по одиночке приводить друг друга?}
\soul{Мы не говорили о приходе, умейте думать. }
\people{А-а-а, в жизни короче.}
\soul{Мы говорили вам, мы не скажем много, мы сказали более, чем могли. Мы нарушаем наши же законы.}
\people{Понял я в каком смысле.}
\soul{Вы поняли?}
\people{Да, надо короче…}
\soul{Вы не поняли.}
\people{Сдружиться надо…}
\soul{Но, мы понесём наказание за попытку, чтобы вы поняли.}
\people{Не надо, наверное.}
\soul{Это не вам решать.}
\people{Хорошо. }
\soul{Вы говорите о риске контактов. Мы рискуем более.}
\people{Не надо рисковать,наверное. Не стоит.}
\soul{Вы не правы.}
\people{Ну…Вам виднее. }  
\comment{(пауза)}
\soul{«Ищущие! Что находите? Рождающие цели, приходите в мир и теряете. Что с памятью вашей, забывшие, не только цели, но и себя? Потерявшиеся в мире бытия. Вы скажете, ищите Бога, забывши, что Бог создал Вас же. Не видите правды в истине, создали своих Богов по разуму Вашему». Вам продолжать? }
\people{Нет. Зачем?}
\soul{Почему же? Это было писано вами. Вы не помните? Мы спрашиваем, отвечайте. «Вы создали Бога по разуму вашему и молитесь теням Истины, придёт Истина во плоти и будете палачами в рясах» Вы не помните?}
\people{Нет, не помню.}
\soul{Вспоминайте! Вспоминайте!}
\people{Хорошо. }
\soul{Почему вы не помните? Вы не помните ничто. Почему?}
\people{Я не знаю. Я не знаю!  Раз я не помню – я не знаю!}
\soul{`` Ищущие – что находите?  Рождаете цели, придя в мир – теряете. Что с памятью вашей? Вы создали Богов по разуму вашему и покланяетесь теням Истины. Придёт Истина во плоти и будете палачами в рясах. Ибо не по Вашим мерам. Ибо, могучее Вас, и потому, страшное Вам.'' Вспомните! Вспомните! Это ваши слова!}
\people{Я так понял, это отрывок из его книги, которую он читал. Во сне.}
\soul{Мы говорили о вас.}
\people{Трудно разобраться, когда вы говорите ``о нас'', а потом по отдельности.}
\soul{Вы невнимательны. ``Пигмеи, боящие  величия, но молитесь сим великанам, желаете воззреть их и алкать с ними. Придёт мечта Ваша ,но неузнаваемая Вами. Вы скажете:'' Память слаба, не помнит былое. Скажем: Но вы не помните и настоящего, ибо хотите верить только в то…'' Вы вспомните! }
\people{Я где-то это читал, припоминаю.}
\soul{Вспомните! }
\people{Хорошо. Но, я читал это здесь, в этой жизни!}
\soul{Вспомните! Вопрос в том, чтобы вспомнить.}
\people{Хорошо. Теперь у меня есть цель, мне надо вспомнить.}
\soul{Так, давайте снова.}
\people{Вы три раза уже повторяли.}
\soul{Пожалуйста, не перебивайте.}
\people{Хорошо.}
\soul{Мы будем повторять вам множество раз, пока вы не вспомните. Мы пришли к вам вернуть вашу память. А вы не помните… Так вот слушайте:}
\soul{«Ищущие! Что находите? Рождающие цели, приходите в Мир и теряете. Что с памятью Вашей, забывшей не только цели, но и себя? Потерявшиеся в мире бытия. Вы скажете – ищите Бога, забывши, что Богом создано ваше. Не видящие правды в Истине, создали своих Богов по разуму Вашему. Поклоняетесь теням Истины. Придёт Истина в плоти, и будете палачами в рясах. Ибо не по вашим мерам. Ибо, могучее вас, потому страшное для вас. (пауза) Пигмеи, боящие величия, но молитесь сим великанам. Желаете воззреть их и алкать с ними. Придёт мечта ваша, не узнаваема вами.

 Скажете: ``Память слаба, ибо не помнит былое''. И не поверите, что не верили и настоящему, ибо помните только то, что хотите помнить, а не то, что потребно. Вы ищите новое, опираясь на не понятое старое. Вы создали Богов по мерам Вашим, но не разумеете, что Боги Ваши не сильнее Вас, ибо, это творение Ваше. Забывши, что он создал Вас по подобию своему, но не вы Его. Слепцы, ищете подобных себе! Непохожее – инородно для Вас, незамечаемое вами. Глухая плоть не слышит более, чем может».  Вы помните, что должны вы сделать? Не забывайте это. Вы помните, что вы должны делать после контакта?}
\people{Помню.}
\soul{Не забывайте это.}
\people{Я не забываю.(умыть лицо. прим.)}
\soul{«Вы ищете новое, попирая не понятое старое. Вы создали Богов по мерам вашим и не разумеете, что Боги ваши не сильнее вас, ибо это творение ваше. Забывшие, что он создал Вас по подобию своему, а не Вы Его. Слепцы! Ищите подобных себе…. Непохожее – инородно вам, незамечаемо. И если вы все – пуп Вселенной,- не увидите другой. И много ли видит пуп ваш?  Скажете: несёте крест свой. Всмотритесь, не те ли дрова, что сожгут мир Ваш? Не те ли узы, что распяли Христа? Не тот ли яд, что закрыл глаза Будды? Не те ли слова, что мешали услышать суры? Может ноша Ваша лишь бревно для колесницы Рамакришны? Бог создал Вас! Вы создали ад и всю его рать. Вы – войны той рати. Да, Вы велики! Но всё величие - во лжи Вашей. Вы - любовь, вросшая в плоть. Вы – страдание, под маской сладострастья. Но вы и меч.  Могуча острота его!  Чьи руки владеют им? Что принесут они? Горе, счастье? Нет, Вы – не первые, но и не шестые. Вы – не последние. Вы опора других и Вам решать, быть Вам ``дорогой'' или ``верстовым столбом''. Вы вольны - убиваться,  родиться. Вы выбираете ворота в рай или ад. В том нет Божьей воли, то вина Ваша - не более. Ибо глухи Вы - не слышите. Вы – соль Земли, но и червь, пожирающий её. Вы – жар Вселенной… но холодным бывает тот жар. Вы – жизнь, но летите на крыльях смерти. Вы – лотос, но и болото, топящее его. Вы – ось колеса. На плечах Ваших крест, что клином для многих. Вы – царь над шутом. Но, чаще, шут – отражение в разбитом зеркале. Вы скажете – религиозны. Нет среди Нас Ваших религий! Нет Богов, вами выдуманных. Нет еретиков, ибо не созданы меры. Можно жить не попирая, и везде будет Бог! Вы скажете – ``Мрачен надеждой услышанным быть''. Пророки приходят надеждой,- убиваете их и рождаете ``любовь"! Не услышаны будем – посеем семя раздора …  Не услышаны будем – посеем зерно раздора. Ибо в битвах крепки и правдивы! Ибо разбуженный зверь – разумнее. Скажете: «спокойно со спящим». Да! Но придёт пробуждение и лучше разбудить самим, будучи подготовленны. Не страх движет Нами, ибо крепки опоры для глухого слепца. Не жалость кормит Вас, ибо для Вас – унижение. Любовь! Но для многих та, что держит домашнюю тварь. Вы скажете – горды.  Разве  гордостью  приходим к Вам, убивая себя в Вас? Говорим вашим языком. Вы скажете – гневны?  Разве мать гневается  сыном? … Мы кричим, Вы – не слышите. Мы приходим – Снами, Озарением, Любовью, Талантом и Гневом! Но то, лишь мгновенья прозрений, ибо не можем разбудить вас. Вы спросите – добро ли, зло несём? Если царствует зло над Вами,-  злом придём. Добро несёте – добром! Мы – Сила, но слабы разумы в руках ваших, ибо не хозяевами приходим, но и не гости ваши. Да, вы – ничтожны, ибо познали Величие. Да, Вы – глупы, ибо познали гениев. Да, Вы – слепы, ибо видели. Да, вы – глухи, ибо слышали уста свои. Да, Вы ищите, ибо теряете. Вы – соль Земли. Но не будьте ею на ранах наших! Мы устали от вас – но любим… Скоро… скоро… придёт Новое время, другая жизнь, но не будьте слепцом - увидьте! Найдите! Войдите! Мы не прощаемся с вами, ибо не покидали вас. Но не услышите более, пока не поймёте Прошлые, Сказанные и Прожитые». Вы получили что хотели?}
\people{Да.}
\soul{Вы всё поняли? }
\people{Не совсем и не всё.}
\soul{Я сказал вам, что вы будете вместе. Я сказал вам, что вы должны найти. Я сказал вам, что вы должны помнить. Вы забыли. Создайте, создайте снова. И помните:- дело…дело…дело…(сбой контакта. прим.) Дайте счёт…}
\people{1-2-3-4-…18-19-24}
\people{Ну,допустим,  вспомним мы, опишем. Дальше нам надо её издать? Я так понимаю… Чтобы люди, кому надо, прочитали что им надо.}
\soul{Мы говорили вам, не понимайте буквально. Ибо многие книги были изданы после (смерти. прим.). И мы не знаем, как будет у вас, ибо, мы начинаем,- вы заканчиваете.}
\people{Далее, что вы ещё можете сказать относительно того ,как её писать?}
\soul{Вы пишите. Это ваша книга! Я должен прийти к вам и спросить: - Как ВЫ её писали? А вы меня спрашиваете! Вы писали! Вы писали. И если вы забыли её в прошлом – то вспомните её в будущем. Вы говорили: - вспомнить будущее. Если вы не можете вспомнить прошлое.}
\people{Да.}
\soul{…Если вы не можете вспомнить прошлое. Правильно, или не правильно? Вспомнить будущее. Уйдите от этих понятий, уйдите. Если вы не можете вспомнить прошлое, - создайте будущее!}
\people{Ясно. }
\soul{Говорите, говорите что угодно. Хотите пойте, хотите - сочиняйте стихи. Уйдите, дайте уйти ему! Мы не подвластны ему, он не подвластен нам.} (это конец записи на кассете. – наложен на начало. прим.)
\soul{…Фактически, мир эмоций подчиняется вам. Вы не изменяете его и порой глобально. Вы интересны тем, что один индивидуум может изменить весь мир. Разве вы не можете привести пример?}
\people{Да,- глава государства меняет всю страну..}
\soul{И вы удивительны тем, что человек не далекого ума может совершить это, имея лишь только власть.}
\people{Да, наша беда и наш..}
\soul{Интересно, мы с вами находимся уже 15000 лет вместе, но не можем вас понять.}
\people{Ну, если мы сами себя не можем понять…}
\soul{Вы же непредсказуемы. Вам даже трудно указывать на будущее. Мы можем только предполагать, ибо в любое мгновение вы можете изменить всё.}
\people{Вот почему сказано:'' Судьба, лишь спутница твоя, не дай ей властвовать  тобою''. Правильно?}
\soul{Вашими словами – да.}
\people{Фактически, у человека судьбы нет, он её делает сам?}
\soul{И да и нет.}
\people{Т.е. есть направление…}
\soul{Нам трудно найти аналогию очень трудно работать с «переводчиком». У него большой фон.}
\people{Понятно, что делать? }
\soul{Вы не сможете ничем помочь. Мы должны только выиграть время и постараться уйти от этого.}
\people{Понятно,-  ``заговорить зубы'', короче. Давайте, поговорим.}
\soul{Мы будем говорить или отвечать на вопросы?}
\people{Как вам удобнее.}
\soul{Это ваши проблемы.}
\people{Ладно.}
\soul{Мы хотели бы многое  узнать о вас в духовном плане. Физически вы нас не интересуете. Скажите немного о себе – то, что вы можете? Я имею в виду не конкретно Вас, а о вас.}
\people{О людях?}
\soul{В ваших понятиях -  о людях. Давайте сделаем наоборот – я буду задавать вам вопросы, а вы будете отвечать. }
\people{Давайте. }
\soul{Кто такой человек?}
\people{Человек… «чело»-«век», т.е. можно расшифровать… не знаю, насчет ``вечного духа'' тут мало кто слышал… короче - ``живет целый век'', в принципе, сто лет.}
\soul{Раньше, под словом ``век'' подразумевалась бесконечность. Ибо раньше 100 лет считалось большим числом. Согласитесь, что 1000 лет назад, возраст 60 лет считалось очень много.}
\people{Да.}
\soul{И вы можете смело перевести человек как ``бессмертный''.}
\people{Да, раньше жили 25-30 лет. Что ещё вас интересует? }
\soul{Говорите.}
\people{В духовном плане, как говорится, разные индивидуумы бывают. Но, в принципе, у каждого, лично по моим наблюдениям, с  любым, как говорится, по нашим понятиям, низко павшим, можно вполне по-человечески поговорить и он понимает. Практически со всеми. Только, бывает, мозг ставит запреты сердцу, т.е. не принимает сердечные рассуждения…}
\soul{Вы правы, сущность одна, у любого человека. И затрагивая эту сущность, мы можем с ним говорить, и не имеет значения кто он.}
\people{Я тоже к этому выводу пришёл. Иногда, на правду, действительно, обижаются, ну, в нашем понятии ``правду''. }
\soul{Здесь вы жестоки.}
\people{Да, операцию тоже делают больно, но это необходимая боль.}
\soul{Необходимо?} (резкий переход)
\people{Если рассуждать с вашей позиции, то можно вообще плюнуть на всё и жить в своё удовольствие…}
\soul{Значит, инквизиция была необходимостью?}
\people{Скажем так, она оставила самых стойких, самых мудрых. }
\soul{Нет. Она оставила глупцов.}
\people{Хм… Тоже верно. }
\soul{Именем Бога вы творили работу Дьявола. Вы согласны?}
\people{Согласен.}
\soul{Вот вам один из ответов. Что лучше, добро или зло? Подумайте об этом. }
\people{Это правда. Добром творили зло. Правильно. А ``злом творить добро'' – это как? Пример какой-нибудь, пожалуйста.}
\soul{Зло творит добро?}
\people{Ну, сказано было: ``Что лучше добро творящее зло, или зло, творящее добро"?(это было написано в блокноте переводчика ещё до контактов. прим.)}
\soul{Вопрос был задан, чтобы ответили вы.}
\people{Понятно.}
\soul{Вы сейчас не в состоянии на него ответить. Мало кто это может сделать. Найдя ответ, вы зададите новые вопросы, и снова будете искать ответ.}
\people{А вы знаете, что лучше? В вашем… в нашем мире.}
\soul{У нас свой мир, у нас свои понятия. }
\people{Ну, по вашим понятиям.}
\soul{По нашим понятиям, - мы тоже ищем ответ. Мы с вами одинаковые, не  забывайте об этом. И мы тоже идём дорогой. Нам с вами по пути, но мы не ведём вас.}
\people{Понятно.}
\soul{Ибо, если мы скажем вам, то, значит, мы идём своей дорогой. У вас - своя, у нас – своя, – первое. Второе,- не всегда благородно. Иногда, приходится оставлять путника в беде. }
\people{Так значит…}
\soul{Существуют законы, которые нельзя нарушать. И даже и с вашей и с нашей точки зрения, они, порой, жестоки. И лишь только со временем мы можем понять, почему эти законы существуют, и уже - другое понятие о жестокости.}
\people{Ясно. Вопрос, конечно, сложный, долгий, можно решать вечно.}
\soul{У вас хватит времени.}
\people{Да, спешить некуда.}
\soul{То, что сейчас сделали мы – жестоко, но у нас есть возможность с вами говорить.}
\people{Я понял. Так это всё-таки сделали вы? (пауза) Без вас, тоже это может быть? }
\soul{Да.}
\people{Ясно. }
\soul{Вы живёте хаосом. И в хаосе бывают моменты… }
\people{Понятно.}
\soul{Вы, бываете чаще - неуправляемы, и вы  знаете, к чему это приводит. Вы не умеете управлять. Вы спрашивали о сумасшедшем. Да, сумашедший, в какой -то мере контактёр, но он - первое – хаос. }
\people{Да. Понятно. Этим, всё сказали.}
\soul{Мы говорили вам, что мы находимся в эмоциональном плане и, тем более, мы сейчас границе его. Вы спрашивали, мы вам отвечали. Мы ушли. С ним тяжело работать – первое. Второе, согласитесь, часть мозга отвечающие за расчёты, находится не в том месте, где находимся мы. Мы говорили вам, что ваш мозг закрыт. И заметьте, чем больше идут контакты - тем больше он закрывается. Что нелогично. Мы рассчитывали на адаптацию. Мы рассчитывали на привыкаемость, а он всё больше закрывается. И он нашёл очень интересный способ, – он уходит в прошлое.}
\people{Вижу. Что же нам делать, по-вашему? Может, существую какие-то земные способы разблокирования, приостановления блокировки? }
\soul{Мы не знаем ответ. Поймите, мы стараемся войти,- мозг воспринимает это чужеродным. И, чтобы уйти от этой реали, где находимся мы, он уходит в прошлое. }
\people{Вы в прошлое не можете уйти?}
\soul{Мы не можем уйти вместе с ним. }
\people{И вам трудно работать.}
\soul{Это самая лучшая защита, которую он мог придумать. }
\people{Н-да. }
\soul{По-другому сказать? Мы хотим войти в мозг, а он исчезает. И его нет.}
\people{Так, значит, наступит  момент, когда это всё дело прекратится, да? Но, если вы  не найдёте, и мы не найдём способов?}
\soul{Конкретно с ним,- может быть. Но, мы ведём множество контактов, и такого не случалось. Причины? Причины в его зарядах. Мы не можем этого пока понять. И мы также анализируем ситуацию. Поэтому мы просили вас о 4-м. У нас свой закон, и мы не можем назвать его.}
\people{Ясно. Вы как-то узнаете кто этот 4-й? }
\soul{Мы знаем, но здесь мы не дадим даже намеков. }
\people{Т.е. даже - мужчина, женщина, - вы не скажете?}
\soul{Нет. Мы не скажем. Уже всё было сказано ранее.}
\people{Хорошо.}
\soul{Дальше. Мы пошли на всё это для того, чтобы соединиться конкретно с ним.}
\people{Вы с ним соединились чтобы, или я с ним соединился?}
\soul{Мы - конкретно с ним. Но, говоря о нём, мы имеем в виду и вас. }
\people{Ну, да,-  через него и с нами .}
\soul{Нет. Здесь гораздо глубже.}
\people{Капните, поглубже.}
\soul{А вы не догадываетесь?}
\people{У меня откроется такая ``крыша''? Ой! Извиняюсь, - контакт. }
\soul{Хорошо. Ваша версия? }
\people{Допустим, лет через 5 так и я мог бы лежать на его месте.}
\soul{Нет. Вас сейчас 4-ро. Но вы были один.}
\people{М-м-м, --- ``вас сейчас 4-ро''? Нас сейчас двое. C вами - трое.}
\soul{Вас четверо.}
\people{Белимов - третий. Вот, он задавал вопросы. Надо найти четвертого… и мы соединимся. Соединившись, мы образуем, наверное, какую-то силу, которая преодолеет барьер мозга, я так думаю. }
\soul{Вдумайтесь. Сейчас вас 4-ро, но раньше вы были один. ВЫ БЫЛИ ЕДИНЫ. Четверо, и один. ОДИН разделился на ЧЕТВЕРО. Мы можем сказать только одно: Он  РАЗДЕЛИЛСЯ  НА  ПРОТИВОПОЛОЖНОСТИ.  Это может дать вам подсказку.}
\people{Противоположности… так…}
\soul{Мы сказали более .  Продолжим далее. }
\people{Так. }
\soul{Дайте счёт.}
 1-2-3-4

\soul{Задавайте.}
\people{Так. Музыка можем чем-то помочь? Эмоциональный план улучшить? Или…}
\soul{Во время контакта - нет.}
\people{Ну, это я понял. А до контакта? }
\soul{Да, конечно. }
\people{Т.е. можно процесс приостановить. Какая музыка, в каком стиле? Какая была?}  
 1-2

\soul{Слушайте своё сердце. }
\people{Извините, моё сердце может говорить, что эта музыка хорошая, а он может воспринимать её как ``туфту''. Что это? }
\soul{Вспомните, что мы говорили 3 минуты назад. Но, не говорите сейчас!}
\people{Понял. Так… ясненько. Что ещё можно сказать?}
\soul{Задавайте. Вас было четверо. Пусть так, - сила одного вмещала четырех. Как нам с вами работать?}
\people{Нас было четверо, здесь? Да? }
\soul{Вы не поняли. Был один человек, скажем так; сверхчеловек. Он разделился на 4-х человеков.  Вы должны найти составляющие и соединиться в нём. }
\people{Интересно, а как мы соединимся? Духовно, да?}
\soul{А вы себе можете представить физическое соединение? }
\people{Представить-то  можно и физическое. Факт, что это не то… Т.е. мы должны стать друзьями или продолжить дружбу?}
\soul{Вы найдёте сами ответ. Вы встречались, и не раз, но вы не нашли правильный вывод. Это ваши проблемы.}
 *Понятно. Тут кругом наши проблемы, только у нас голова уже не варит от этих проблем… Я тогда просил, чтобы вы передали на ту планету, что называется Вифлем. Вы передали? (привет.прим.)

\soul{Вафлем.}
\people{Почему? Написано было ``Вифлем''.}
\soul{Вафлем.}
\people{Не важно. Так да, или нет? }
\soul{Нет.}
\people{Не передадите? }
\soul{Передадим.}
\people{Но не сейчас?}
 -Не было контакта. Мы с ними контактируем так же, как и с вами. Мы говорили вам об этом.

\people{О них вы что-нибудь можете рассказать? Ну, там -  организации, иерархия, техника, есть ли, нет?}
\soul{У них есть иерархия. Да, и у них есть техника.}
\people{Не в нашем понятии?}
\soul{Есть и вашем понятии.}
\people{Базы какие-нибудь у нас тут есть на Земле? }
\soul{Множество.}
\people{У них тут тоже семьи живут, как люди?}
\soul{Да. Чаще, они имеют форму людей.}
\people{Ясно, т.е. они могут структурно изменяться, так сказать, физиологически? Исчезать они могут? В мановение ока – раз!- и нет их.}
\soul{Представьте двухмерный мир…}
\people{Понятно. Они преследуют свои цели познания?}
\soul{И да, и нет.}
\people{Тоже ясно.}
\soul{Вы, помогая кому-то, какие преследуете цели?}
\people{В смысле?}
\soul{И те и другие.}
\people{Можно надеяться, что назад они нас не потащат, вернее нас  тащат оттуда , откуда мы ушли.  Так скажем,- мы – дети, оторвались от всего мира, и пошли куда-то не в ту сторону. Они вернулись, чтобы…}
\soul{Худшего не принесут, но вы можете и лучшее обратить в худшее.}
\people{Да-а… Тут всё возможно…}
\soul{Но это уже будет не их вина. Но, они будут нести ответ. }
\people{Т.е. мы их подведем? Ну, можем подвести?}
\soul{Вы делаете это удивительно просто! Вы бессмертны, и так легко можете убить.}
\people{Тут проблемы нравственности, скорее всего, в стране.}
\soul{Нет. Нравственность, мораль – это всё выдумано вами. Это, всего лишь, отговорки . Вы живете в мире лжи. Да, мы говорили, что лжи не существует, но ложь – это,  когда лгёте самому себе. Вы можете мне солгать, что угодно, и я могу это принять за правду. И это нельзя назвать ложью. Но, относительно вас, вы знаете, что это ложь.}
\people{Верно.}
\soul{Научитесь не лгать себе.}
\people{Хорошо, а можно, если уж мы так заврались, можно ли эту ложь использовать в благих целях для нас же самих? Т.е. допустим…}
\soul{ЗЛО, творящее ДОБРО?}
\people{Получается, уже добро, а не зло…}
\soul{Хорошо, вспомните своих пророков, которые ложью делали добро. Вы можете их вспомнить?}
\people{Ясно. Вопрос исчерпан.}
\soul{И вспомните о кривом единстве. Мы не религиозны, но религия – один из способов видения мира и она не лжёт. С вами легче говорить религиозным языком. Вас искушает Сатана. И очень искусно. К сожалению, вам порой это нравится.}
\people{Да… то есть,. все плотские наслаждения-  это от Сатаны, если переходить уж на язык религиозный?}
\soul{Нет. Всё должно быть в меру. Нельзя казнить свою плоть, но и нельзя развращать её. Диалог - истинное начало.}
\people{Т.е. брать не больше того, что тебе надо.}
\soul{Вы правы.}
\people{Насчет цивилизации дельфинов, вы сказали. Они же едят рыбу, но это тоже живое. Что ж, получается, что это тоже… зло в меру, так скажем?}
\soul{Злак тоже живой. Вы творите зло, поедая его?}
\people{Верно. Сейчас, допустим, откажусь я от еды, от питья. }
\soul{Вы пожинаете воздух. }
\people{И от дыхания.}
\soul{Хорошо, вы принимаете энергию. Она, тоже, живая. Вы её перерабатываете.}
\people{Т.е. можно фактически питаться энергией?}
\soul{Если так рассуждать, то вы убиваете  постоянно. Так, что не будем говорить о дельфинах. А как вы считаете, чем питаетесь вы? Что бы вы ни ели, что бы вы ни делали,- вы потребляете энергию. Всё остальное лишь несёт её.}
\people{И мы можем потреблять её больше чем надо, или менее того?}
\soul{Верно. }
\people{Этим мы не делаем великое зло или великую доброту? Я о постах.}
\soul{Любая чрезмерность - это зло. И пост в меру. Можете назвать мне религию, которая гласит о посте, так, что вы должны умереть? Если вам не по силам 40 дней, вы можете, не прекращая поста, принимать пищу, но, разумно. Это - первое. Второе, будьте внимательны, Христос пришёл не для того, чтобы взять добродетелей и святых. Он пришел для вас, грешников.}
\people{Это я понял. }
\soul{Это символ. Символ вашей ``дороги''. Вы должны стать, как Христос. Но, сперва, вы должны стать самими собой, и, лишь только тогда идти далее. Будьте внимательны. Вы знаете ответы на ваши вопросы. И более,- вы читали эти ответы, и будете читать. Но, вы не внимательны. Вы не можете всё соединить. Вы читаете и видите, запоминаете, только те строки, которые вам понравились, или, которые подтверждают вашу правду. Всё остальное, вы опускаете. }
\people{Ну, не всегда. Мы возвращаемся ещё иногда. Потому что, если что-то где-то замечено было ,то, если память помнит …}
\soul{Вы можете и извращать. }
\people{Можем. В угоду нам же самим… Мы хотим…}
\soul{Вы лжёте самим себе, вы живёте в мире лжи. Возьмите 3-й том.}
\people{Скажите, в мире лжи, можно не лгать? }
\soul{Можно. Вы опять не можете воссоединить всё в единое. Мы говорили о Христе. Он лгал? }
\people{Нет. Он не лгал.  Ну, я думаю… }
\soul{Дайте счёт. }
\people{Обратный?}
\soul{Вы хотели этого?}
\people{Да. Я так подумал, что уже…} .
\soul{Дайте счёт.}
 
\comment{9-8-7-6-5-4-3-2-1}
\comment{(Конец записи)}

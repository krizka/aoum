Аоум. глава 31-я 01-04-1996г
Георгий Губин
\people{Мы проводим сеанс.  Геннадий, Оля и Геннадий Белимов. Сейчас убираем последние часы, которые  ещё остались в комнате.  Они на вас действуют ритмом своим?  Мы чувствуем, что мы опять с нашими первыми представителями общаемся.  Мы очень благодарны вам за то, что, по-видимому,  вашими усилиями мы смогли выйти и на другие миры… общества,  слои  астрала или что. Это не без вашей помощи. Но теперь, мы хотели бы расспросить , во-первых, как вам это удалось? Почему пришла эта мысль, всё-таки, свести нас с иными мирами, с иными представителями цивилизаций в прошлом?  Да?  Как? Как вам пришло это?}
\soul{Мы пытались показать  вас.}
\people{Нас самих в прошлых воплощениях, так что ли? }
\soul{Каждый голос, что слышали вы, один и тот же. }
\people{Но в разных ситуациях. Допустим, если ситуация была с эльфами, гномами,  то есть  получается, что вы показали нас  в наших фантазиях, в  нашем фантазийном мире? Так?}
\soul{Да. Но только вы живёте там. Вы же фантазируете, значит, и мир - ваш. }
\people{Так. Ну, он реален, он тоже размножается, он имеет право,  длительность существования? Так?}
\soul{Он реален даже физически. }
\people{Ну, например, как-то мы встретили волжскую девушку эльфа, которая якобы вспоминает те миры. Это, с вашей точки зрения,  реальная ситуация или чисто надуманная? }
\soul{Реальная. Относительно вас может быть и ложь. Мы говорили вам о лжи. Вспомните.}
\people{Ну, дело в том, что меня поражает то, что незнакомые люди прежде начинают переписываться и вспоминают свои прежние  контакты, свои прежние жизни. Хотя, например, в произведениях других фантастов этого быть не может.  Потому что произведения других фантастов -  это чисто их выдумка и никакой переписки в их фантазиях быть не может. А здесь встречается…}
\soul{Тогда,  вспомните ``нападение марсиан''.}
\people{Ну, понятно. Герберта Уэллса. Ну, там… там могла быть переписка, наверно.}
\soul{Мы говорили вам о множестве богов. Приводили примеры. Существует и бог журналистики. Потому, что вы мечтали, вы построили его.}
\people{Ну, мы это идею не развивали особенно.}
\people{То, что мы придумали, оно уже существует, подкрепленное  множеством  других…}
\soul{Вы же говорили, что каждая мысль материальна.}
\people{Мы говорим, но  ещё это в полной мере не ощущаем. И не можем пользоваться.}
\people{Воспринимаем. Это очень трудно для  восприятия.}
\people{Скажите, а вот ситуация, допустим, с тем же Сергеем Ивановым, из прошлой жизни, 1916-го года…}
\soul{И снова - имена?}
\people{Имена, не надо? Но он-то говорил!  Потому-то мы думали, что это может быть…}
\people{1926 год, 18 мая было, да? }
\people{Был контакт. Как вы эту ситуацию объясните?}
\soul{Мы когда-то говорили вам о множество контактов. И  говорили, что в прошлом… и даже вы были переводчиком. }
\people{В прошлой жизни, да? }
\soul{В вас остался страх, вы должны помнить. Мы говорили, что когда-то мы сделали ошибку и позволили вам помнить. }
\people{Ну, может быть мы ещё не готовы.}
\people{Скажите, а почему вот так получается, такое ощущение, что мы в одном времени, как-бы? Это что, как это объяснить? }
\soul{На время - делите вы. А мы всегда говорили вам о его несуществовании. Мир един,  вы знаете об этом, но не можете понять. Может быть в едином мире - время?}
\people{Нет.}
\soul{Только физически.  Мы говорили вам, что вы можете создать машину времени, но вы никогда не сможете повторить один к одному – это будет просто попытка или взгляд со стороны. }
\people{Скажите, а  вот если мы можем с прошлым, допустим… с прошлым каким-то образом… на прошлом контакте…побывали  в прошлом. А  в будущем?}
\soul{А в будущем - побывал он.}
\people{Переводчик?}
\soul{Вы не согласны?}
\people{А он, то есть…}
\people{Да. }
\people{Он побывал…был ошарашен тем, что мы ему сказали. Надо ли, вообще-то, говорить о том, что будет война, о том, что умер их вождь?  Или это опасные сообщения? И почему мы не получаем, допустим,  через переводчика прогнозы на будущее? Ведь они очень интересны человечеству.}
\soul{Представьте, что делаете вы. Вы говорите, что мечты его не сбудутся, что мир, в котором он живёт и который он помогает строить, - разрушится. Представьте, что делаете вы, какую приносите боль своими знаниями, своим желанием. И теперь  представьте, если  он будет говорить о том, что вы ему дали знания. }
\people{Ну, да.}
\people{Его, конечно, или один вариант - в психушку отправят, или воспримут, как вруна. Так? Мы, в общем, этим портим, да, человеку его существование?  }
\people{Не только….}
; …
\people{Ну, мы поняли по вашему молчанию, что так нельзя. Но с другой стороны, поставив себя на его место, вот лично я очень хотел бы узнать, а что ждёт нас на пороге 2000 года?  Ведь много об этом идёт толков, неожиданностей. То же предсказание Нострадамуса… Кто их оплевывает,  считает, что это белиберда, а мы, вот уфологи,  допустим,  мы видим, что они, в основном, сбываются. Значит, 1999 год для человечества чем-то опасен?}
\soul{Предсказания верны только тогда, когда они сбудутся. И  когда вы уже сможете понять, что это сбылось. Тогда вы признаете предсказание верным.}
\people{Но если у Нострадамуса большинство предсказаний, всё-таки,  сбывается, значит 1999 год или порог 2000-го года человечеству грозит какими-то необычными потрясениями, так?}
\soul{Вы идёте на обман. }
\people{То есть, вы нам совершенно ничего не хотите из прогнозов говорить? }
\people{Не надо. А зачем?}
\people{Хорошо, давайте продолжим. Вот, на нас вышли ``санитары'', так называемые. Цивилизация, которая чувствует боль землян. Как вы можете охарактеризовать эту субстанцию, эту ситуацию, как мир?}
\soul{Их можно было бы назвать ``утешителями'', но нельзя сделать этого. Для вас, это обидно. }
\people{Почему?}
\soul{Почему? Потому, что вы считаете, что вы строите мир, и больше никто. И по этой причине вы отвергли все остальные миры.}
\people{Лично мы не такие уж… одиозные.}
\people{Ну, в общем, о человечестве речь идёт. Ну, потому, что нас призывают стать  единством, а… }
\people{Ну, давайте по цивилизации этих ``утешителей'' или ``санитаров'' - почему они невидимы и неощутимы для нас, хотя они говорят, что они такие же, как мы, и живут, существуют в трехмерном мире?}
\soul{Вы замечаете поля?}
\people{Нет, не замечаем.}
\soul{Замечаете электрический ток?}
\people{Нет.}
\soul{Нужны объяснения?}
\people{А-а. То есть, они примерно, как поля выглядят? Полевое состояние? Не материальное?}
\people{А у нас тоже такое состояние, только другое немножко.}
\people{А как вот объяснить, что они смещены по фазе? Можете нам пояснить это? Это иная частота колебания?}
\soul{Нет, та же.}
\people{По времени?}
\soul{Да, можно сказать и так. Это будет близкий ответ. }
\people{Но они - телесны? }
\soul{Да.}
\people{Плотные тела имеют, да?}
\soul{Да. }
\people{Почему же мы их не видим?}
\people{По времени смещение.}
\people{Смещение по времени? }
\people{Скажите, а они вот, как  впитывают в себя наши эманации?  В тело, или в свои, ну, излучения ну, тоже у них ауры есть, допустим, какие-то ещё? }
\soul{Это их жизнь. Это не работа – они живут тем, что питаются вами.  И вы, и вы…}
\people{1…}
\soul{Они меня уже считать учат. (шёпотом)}
\people{Это кто?}
\soul{А вы кто? (шёпотом)}
\people{Ты Вася? Ты Вася?}
\soul{А где Петя? (шёпотом)}
\people{Петя где? Пети сегодня нет. }
\people{А кто вас учит считать?}
\soul{Монахи. (шёпотом)}
\people{А-а! Это какой век?}
\people{Какая сейчас погода у вас?}
\soul{А у нас всегда одна и та же погода.}
\people{У вас нет дождя?}
\soul{Вода?}
\people{Да, с неба вода. Дождь идёт?}
\soul{Есть.}
\people{А почему одна и та же? Солнце светит?}
\soul{А-а-а! У нас ночью всегда вода.}
\people{С неба? С неба ночью вода всегда?}
\soul{Сверху. }
\people{Сверху.}
\people{А вы живёте в пещере?}
\soul{Да. }
\people{Глубоко под землёй?}
(перепад в записи)
\soul{Я их не люблю, потому, что они бьют больно.}
\people{Монахи бьют вас?}
\people{Они вас бьют? Вам сколько сейчас лет?}
\soul{Ещё не знаю.}
\people{Ну, вы взрослый? }
\soul{Да. У меня две жены.}
\people{Ух ты!}
\people{Так…}
\people{А как они узнали, что вы обладаете такими свойствами - общаться с другими мирами?}
\soul{Они  пришли и сказали, что дадут мне землю. }
\people{И вы согласились, да?}
\soul{А вы б не согласились?}
\people{Ну, так они обманули и землю не дали вам, всё-таки?}
\soul{Они никогда не обманывают!}
\people{Ну, землю вам дали?}
\soul{Дали.}
\people{А вам удалось их ожидания выполнить, то есть, вы связались с иными мирами, вы рассказали им много интересного?}
\soul{Не знаю, но они меня после стали учить.}
\people{Чему?}
\soul{Ну… я со следующего посева буду знать, что такое ``две тьмы''. }
\people{Две тьмы? А-а, это мера! Да?  Мера измерения? Вы, что, выращиваете зерно, пшеницу, да?}
\soul{Не знаю.}
\people{Ну, вам дают зерно?}
\soul{Да.}
\people{Землю сами обрабатываете? Или на быках, или на чём?}
\soul{Жёны.}
\people{Жёны обрабатывают? А кто копает землю?}
\soul{Копать?}
\people{Да. }
\people{Сеет кто? Куда,  в землю?}
\soul{Зачем? Жёны. А я откуда знаю!}
\people{Это другой мир.}
\people{Вы на Земле живёте? У вас  солнце светит всё время, да?}
\soul{Светит.}
\people{Сколько урожаев в год вы снимаете? Вы живёте в тёплых краях?}
\soul{А я ещё не знаю, что такое год. }
\people{Ещё не знаешь?  А монахи тебя счёту учат, да? И..да?}
\soul{Да.}
\people{До-скольки ты умеешь считать? }
\soul{Одна тьма.}
\people{Одна тьма. А две тьмы?}
\soul{Учат.}
\people{Будут потом, да?}
\people{А что такое одна тьма, объясни нам, пожалуйста?}
\soul{Ну, это когда урожая может хватить на трёх жён. }
\people{А в течение какого времени? До следующего урожая?}
\soul{Да.}
\people{А когда следующий урожай будет?}
\soul{Когда пройдут дожди.}
\people{Ну, это долгое время придётся ждать?  Недели, две, три? Месяц?}
\people{Счёта нет у них. (шёпотом)}
\soul{Не знаю.}
\people{Не знаете?}
\people{А монахи только тебя учат счёту?}
\soul{Да, наверное.}
\people{А они часто берут тебя к себе?}
\soul{Петя помог. Теперь, часто.}
\people{Ну, Петя…  Вот у вас русские имена. Вы живёте в Руси?}
\people{Нет. Скажи, вот как ты соседа своего называешь?}
\soul{Никак.}
\people{А как к нему обращаешься?}
\soul{А у меня нет соседа.}
\people{Ну, а к жене?}
\people{Назови, как зовут твоих жён?}
\soul{Ауба.}
\people{А вторая?}
\soul{Уана.}
\people{Уана? А кого бы ты хотел в третью взять?}
\people{А как они тебя называют? Жёны как тебя называют?}
\soul{Они не могут меня называть.}
\people{Почему?}
\soul{Они женщины.}
\people{А женщинам нельзя  называть мужчин?}
\people{А-а… мужчин нельзя? А ты мужчина?}
\soul{Да. }
\people{Скажи, а монахи чему ещё тебя, кроме счёта, обучают?}
\soul{Они учат меня рисовать.}
\people{Рисовать?}
\people{Буквы или рисунки?}
\soul{Себя!}
\people{Себя? И у тебя получается?}
\soul{Бьют.}
\people{Бьют, если плохо получается? А скажи, чем рисуешь?}
\soul{Камнем.}
\people{И по чему? По камню?}
\soul{Да. }
\people{По песку? Или по камню?}
\soul{Нет, они приносят воду, которая горит, а из неё потом берут камни. А они рисуют. }
\people{По камню? По камню рисуете?}
\soul{На камне.}
\people{И они на камнях тоже рисуют?}
\soul{Они?}
\people{Да.}
\soul{Не-ет.}
\people{Только ты рисуешь?}
\soul{Ну, они же монахи. Они приходят просто учат. }
\people{А у тебя свой дом есть? }
\people{В пещере. (шёпотом)}
\soul{Да.}
\people{Он из камней или в земле вырыт?}
\people{Пещера.}
\people{Пещера?}
\soul{Что такое ``вырыт''?}
\people{Вырыт , это углубление в земле.}
\soul{Нас за это бьют, если мы будем вытаскивать. }
\people{Нельзя,  да? А монахи, как живут? У них большие помещения?}
\soul{Я там не был.}
\people{Тебя не приглашают? А куда же тебя берут?}
\soul{А туда нельзя. Они приходят.}
\people{Они к тебе приходят? И говорят, что… надо с кем-то поговорить, да?}
\soul{Они мне дали землю и  обещали третью жену!}
\people{Понятно. }
\people{А это когда было? Когда ты с Петей познакомился?}
\soul{Да.}
\people{А что они тебе сказали после этого?}
\soul{Не знаю. Они меня били, сперва. }
\people{Скажи у вас сейчас царь или король, или  кто-то есть, кто вами правит? }
\people{Главный.}
\soul{Монахи.}
\people{Только монахи? А у монахов, кто главный? }
\soul{К ним прилетают боги. Но их никто не видел.}
\people{А у каких морей вы живёте? Там есть леса моря, большая река, есть? }
\soul{Море есть. Оно приходят только после трёх урожаев, один раз. }
\people{Один раз в году?  Значит, четыре урожая в год получается у вас.  Скажи, а в прошлый раз, когда ты познакомился с Петей, монахи пришли  и  что потом? Монахи… Что потом? Много времени прошло после этого?  Ты урожай снимал или  нет? Землю давали тебе? }
\soul{Два.}
\people{Два урожая? После того, как ты с Петей…}
\soul{С новой земли.}
\people{С новой земли, которую тебе дали за Петю, да?}
\people{Два урожая с новой земли? А в чём ты одет? Вот мы сейчас не видим, в чём ты одет?}
\soul{Одежда? }
\people{Что на тебе?}
\people{Ты голый или?.. }
\soul{По…по…повя…}
\people{Повязка?}
\soul{Да-да!}
\people{А монахи во что одеты?}
\people{Они так же закрытые ходят все?}
\soul{Ну, один хотел их увидеть, но он потом… он слепой.}
\people{Ослеп? А жёны как одеты? Тоже с повязками?}
\soul{Да.}
\people{А они у тебя красивые?}
\soul{А зачем я буду не красивых брать?}
\people{Хорошо. Но ты в пещере нашёл тёмной, там же не было видно?}
\soul{Почему тёмной? Нам давали огонь. }
\people{Вы огнем пользуетесь, да?  Он у вас горит в масляных лампадках или лучиной? Или костёр?}
\soul{Монахи приносят воду, а она горит.}
\people{О-о! Вот оно чё!}
\people{Нефть.}
\people{А скажи у вас какой цвет тела, у твоих жён? }
\soul{А я ещё не знаю, что это такое. }
\people{Вася, а в этот раз они что тебе обещали, когда ты пришёл говорить с Петей?}
\soul{Жену.}
\people{Ты хочешь ещё одну жену? }
\soul{Да. У меня есть земля,  и я могу кормить.}
\people{А ты можешь кормить, а  не они тебя кормят? Они же сеют? Они же выращивают пшеницу?}
\soul{А зерно-то  я даю!}
\people{А-а! Они выращивают, собирают и приносят тебе, да? А ты этим заведуешь. А ты распоряжаешься, кому дать поесть, а  кому чего?}
\soul{Не понял.}
\people{Ну, они собирают урожай. Кто собирает? }
\soul{Я.}
\people{А! А они сеют, да? Жёны сеют, а ты убираешь?}
\soul{Наверно.}
\people{А зёрна, куда ты деваешь?}
\soul{Отдаю жёнам.}
\people{Сразу всё?}
\soul{Зачем - всё? Монахи говорят, что давать, а что нет. А потом, мы ещё трём на камне и их можно есть. }
\people{Муку сделать.}
\people{А где вы готовите? На очаге? Еду готовите вы или едите так, просто, после того, как на камне… }
\people{Жарите или печёте хлеб?}
\soul{Не знаю, о чём вы говорите.}
\people{Ну,  на камнях вот вы разваливаете зерно?}
\soul{Да.}
\people{А потом, что с ним делаете?}
\soul{Сушим.}
\people{И всё? А едите как?}
\soul{Как едим? Просто едим!}
\people{Это же?}
\people{Ну, вы печёте, наверное? }
\people{А вы на огне что-нибудь делаете с едой?}
\soul{Два урожая проходит и уже можно есть.}
\people{А монахи ещё чего тебя учат, кроме ; рисовать и считать? Ещё чему учат?}
\soul{Они произносят какие-то…слова,  а я должен повторять.}
\people{Повтори несколько слов для нас.}
\soul{Нельзя.}
\people{Почему? Это секрет?}
\soul{Они меня будут бить.}
\people{Нет, они не будут. Мы обещаем, что они не будут бить. }
\people{Мы попросим, чтоб они не били.}
\people{Назови несколько слов.}
\soul{Ауба.}
\people{Так. Ещё?}
\soul{Не буду. }
\people{Хорошо. А у вас есть скот? Животные?}
\people{Нет, мы уже спрашивали. У монахов есть. А вообще расскажи, как ты живёшь с жёнами? Как у вас жизнь идёт? Как вы проходите? }
\people{Как у вас день проходит?}
\soul{День? }
\people{Да. Как вы живёте? }
\people{Что вы делаете?}
\soul{Мне они объяснили, что такое день, но я забыл.}
\people{Ну, это когда светло, когда солнце светит. Когда вы не спите.}
\soul{Ну, сейчас приходят монахи. Солнце встаёт  – приходят они. Они меня сперва бьют, а потом учат. }
\people{А зачем бьют, они не говорили? }
\soul{Они не разговаривают.}
\people{А как же они тебя учат?}
\soul{Они, просто, бьют.}
\people{А среди монахов есть кто-то главный, да?  Ты его боишься?}
\soul{Нам нельзя бояться монахов.}
\people{А как вам сказали, что нельзя? Откуда ты знаешь?}
\soul{Мы должны их любить. }
\people{А кто  у вас объясняет это? }
\people{У него имя есть?}
\soul{Есть. Старейшина.}
\people{Старейшина, да?}
\soul{Они говорят ему, а он говорит нам. }
\people{А-а,  у вас есть старейшина!}
\people{Он умный?}
\people{Или старый?}
\soul{Говорят, что я тоже буду.}
\people{Будешь старейшиной, да?  Ну, мы надеемся на это, потому, что ты хорошо учишься. Это скоро будет?}
\soul{Обещали - две тьмы. }
\people{Две тьмы?}
\people{Через две тьмы?}
\people{Одна тьма – это два урожая.}
\people{Долго, да?}
\soul{Долго!}
\people{А ты ещё не седой человек?}
\people{А у них нет волос. Лысый.}
\people{Лысый?}
\people{Да. Скажи, а  опиши, как вы выглядите. Как выглядит твоя жена? Вот что у неё есть: руки, ноги, голов, глаза какие, всё, нос?  }
\soul{Есть.}
\people{Расскажи, какая она? Можешь это сделать?}
\soul{Не знаю, жена есть жена.}
\people{А вот дети… }
\soul{У тебя нет жены?}
\people{Есть. У нас есть… Скажи, а дети у тебя есть?}
\soul{Есть.}
\people{Много?}
\soul{Один. }
\people{А в прошлый раз говорил, что два было.}
\soul{Два было. }
\people{Одного монахи забрали?}
\soul{А я Пете говорил уже.}
\people{Что говорил Пете? }
\soul{Что они приходят и забирают.}
\people{А, да.}
\people{И выращивают твоих детей у себя? Так что ли? }
\people{Они не знают.}
\soul{Не знаю.}
\people{А у тебя девочка или мальчик был… остались?}
\soul{Девочка, мальчик?}
\people{Ну, да. }
\people{Ну, как жена или как ты?}
\soul{Как я.}
\people{Мальчик,  да?}
\people{Вы с женой часто спите? Говорили?}
\people{Детей, когда делаете? }
\soul{Приходят монахи и говорят, что пришло время. }
\people{Делать детей? Скажи, можешь описать, ну, как вы живёте? Вот, ну, днём приходят монахи, вас кормят. Как кормят? Из чего?}
\soul{Не-е. Они только приносят зерно, а я отдаю жёнам.  Они говорят, что сушить…я ложу на самый…}
\people{Камень?}
\soul{Куда садится солнце. }
\people{На горизонт? На стол?}
\soul{Камень. Солнце сушит. Проходит два урожая, и я могу уже есть. Но это я женам не даю.}
\people{А кто ест это?}
\soul{Я-я!}
\people{А они что кушают?}
\soul{Урожай.}
\people{Свой?}
\soul{Почему это - свой? }
\people{А какой?}
\soul{Мой!}
\people{А у вас нет травы?  Трава есть у вас?}
\soul{Трава?}
\people{Да.  Или одни камни?}
\soul{Камни есть. }
\people{А что-нибудь из камней бывает, такое зелёное? Растет? Трава, деревья?  Кусты?}
\soul{Красное.}
\people{Красные?}
\people{Может это другая планета? (шепотом)}
\people{Нет. Монах нам сказал что  планета - Земля. Расскажи, как вы живёте, как ваше племя живёт. У вас же есть главные…}
\people{Соплеменники, как ты, есть же? }
\soul{Много.}
\people{Дружно живёте или соритесь друг с другом?}
\soul{Сориться нельзя, придут монахи.}
\people{Дружно живёте?}
\soul{А если вас будут бить, вы будете сориться?}
\people{Нет, не будем.}
\soul{Что тогда  спрашиваете? }
\people{А вы поёте песни? Смеётесь? Часто смеётесь?  У вас весело живётся? }
\soul{Да. Приходят монахи, а старейшина рассказывает от них и..ии… иории.}
\people{Истории?}
\soul{Да.}
\people{Вам нравятся эти истории? }
\soul{Не всегда.}
\people{А какие, например?  Ну-ка, расскажи нам хотя бы одну историю.}
\soul{Ну, они нам рассказывают, как приходят боги, и что придёт время, и боги нас заберут к себе. Тогда  мы будем уже монахами.}
\people{Вам этого хочется?}
\soul{Конечно!}
\people{А почему хочется монахами стать?}
\soul{Потому, что ОНИ бьют.}
\people{А-а, ясно.}
\people{А вы, может быть, плохо себя ведёте? А вы никогда не били, вот, не нападали на них?}
\soul{Тьфу, на вас!}
\people{У вас один Бог или много?}
\soul{Кого?}
\people{Бог.}
\people{[Шепотом] К монахам приходит Бог. Скажи, а у вас есть рассказы, вот вам рассказывают старейшины, которые… когда раньше люди жили. }
\people{Расскажи, какой рассказ особенно запомнился. Нам это очень интересно.}
\soul{О монахе.}
\people{Что - монахи?}
\soul{Что он был такой же, как я, пришли боги и научили его.}
\people{И он что?}
\soul{А потом…забрали….}
\people{Куда забрали?}
\soul{На…}
\people{На небо?}
\soul{Да. Чего-то сделали с ним, он стал большой, пришёл, а мы были ещё дикие. И он стал нас учить. Он научил нас сажать, научил жён… (считать, видимо. Прим.)}
\people{Обеспечивать.}
\soul{…а потом, опять ушёл. }
\people{Вы знаете? Монахи помнят, как он ушёл?}
\soul{А мы ждём его.}
\people{Он ушёл на небо?}
\soul{Да.}
\people{Это было на какой-то огненной колеснице? }
\soul{Говорят: огонь.}
\people{Огонь был?}
\people{Скажи, а вот ты рисуешь, когда на камне, что ты рисуешь? Можешь рассказать?  Себя…А как ты себя представляешь?  Или ты видишь себя где-то?}
\soul{А они мне дают камень, а меня там видно.}
\people{В камне?}
\people{А-а! И ты себя видел? Видишь?}
\soul{Видел.}
\people{Ты бородатый?}
\soul{Чего?}
\people{А волос  у вас нет?}
\soul{Это чё? }
\people{Ну, вот нет.}
\soul{Тра…травы - нет. (имеется ввиду волосы.прим.)}
\people{Травы нет. Скажи, вот ты себя рисуешь, вот что у тебя есть, вот как ты себя  рисуешь?}
\people{Голову? Да? }
\soul{Да.}
\people{Уши?}
\soul{Нет, уши не обязательно.  А нужна только голова, ноги и руки.}
\people{А глаза – это важно? }
\soul{Глава у нас только старейшина может рисовать.}
\people{А-а! А нос? }
\soul{Нос?}
\people{Что такое нос?  Не знаешь, да?}
\soul{Нет.}
\people{А зубы у тебя острые?}
\soul{Зубы? Есть.}
\people{Зубами ешь? Пищу пережёвываешь?}
\soul{Да.}
\people{У тебя есть зубы?}
\soul{Есть.}
\people{Все? Много?}
\soul{Ну, я не знаю. Ну, я же не вижу в камне зубы?}
\people{А почему не видишь? Ты можешь, ты можешь раздвинуть рот и увидеть зубы.}
\soul{Я не пробовал. }
\people{Попробуй. Увидишь свои зубы, какие они есть. Скажи, а… Скажи… Всё?}
(сбой )
\soul{У кого-то слёзы, и снова нам работа…}
\people{У кого слёзы? У бывшего нашего персонажа, которого мы слышали? У кого слёзы? }
\people{Это вы - предохранители, да? }
\soul{Да…}
 (Конец записи)
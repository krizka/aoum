Аоум. глава 14-я 01-04-1994г
Георгий Губин
\people{**}
 У человека есть жена. Он недоволен ей.
 Ему как будто бы нужна получше, поумней.
 Но если счастья нет внутри и в сердце пустота,
 Хоть всех вокруг перебери: не та, не та, не та.
 У человека есть страна. Он недоволен ей.
 Ему как будто бы нужна получше, посильней.
 Но если счастья нет в груди и в сердце пустота,
 Все страны мира обойди: не та, не та, не та.
 У человека есть судьба. Он недоволен ей:
 Потери и за жизнь борьба, не всё как у людей.
 Но если счастья нет внутри и в сердце пустота,
 Все судьбы мира рассмотри: не та, не та, не та.
 Но если счастья семена взошли и расцвели,
 По нраву жизнь, друзья, страна и все края земли,
 И прекращается борьба, хоть сотни лет живи,
 Не воспрепятствует судьба течению любви.
                      Ярослав Климанов
\people{**}
\people{Сегодня 1 апреля, мы вновь собрались для контакта с вами. Сегодня у посредника, у нашего ``переводчика'' хороший день. Он получил квартиру. Правда, на первом этаже и не очень доволен. Может это скажется на его эмоциональном сознании. Интересно будет почувствовать. Теперь будем задавать вопросы. Вы готовы отвечать?}
\soul{Спрашивайте.}
\people{Мы прервались в прошлый раз на судьбе России, на судьбе нашей страны. Сталин – антипод Ленина или, как мы склонны думать, его продолжение, его развитие?}
\soul{Нет. Они совершенно разные.}
\people{Так какие цели ставил тогда Сталин, отличные от Ленина? Нам показалось, что он просто продлил идеи Ленина, только в более жёстком плане.}
\soul{Согласитесь, можно быть разными людьми, но нести одни идеи. Далее, вы говорите, антипод. Как вы понимаете это? Вы сами, порой, добиваясь что-то, считая правдой, нарушаете ваши же условия.}
 (счёт)
\soul{В вашем понятии он находится в сознании?}
\people{В нашем понятии, он, наверное, в полубессознательном состоянии.}
\soul{Нет. Сейчас он находится в полном сознании. И он будет помнить всё.}
\people{Интересно.}
\soul{Можете спрашивать даже его.}
\people{Мы подумаем над вопросами ему. Хорошо… Это сегодня такой эксперимент вами?}
\soul{Нет. Это ваше право…}
\people{Хорошо. Я продолжу. Многими признаются заслуги Сталина в создании огромной и влиятельной империи СССР. Эти заслуги можно признать более важными, чем его кровавую роль ``палача народов''?}
\soul{Мы говорили вам, - от цели…}
\people{От цели… Вы слышите нас сейчас?}
\soul{Переводчик старается отвечать сам…ибо многое помнит.}
1-2-3-4-5-6-7-8
\soul{Никакая цель не оправдывает средства.}
\people{Хорошо.}
\soul{Далее. Согласитесь, не только у него были …}
\people{Да… Можно ли, на ваш взгляд, считать Сталина гениальным человеком? Или это просто монстр, жестокий интриган с гипертрофированной жаждой власти?}
\soul{Давайте скажем так, многие из вас - обыкновенные люди, но когда к ним кто-то приходит извне, вы их называете гением. Далее, мы не говорим, что Сталин, в вашем понятии, был контактёр. Но, если вы так скажете, вы будете не далеки от истины.}
\people{Это свойственно многим людям?}
\soul{Это одержимость. }
\people{Одержимость?}
\soul{У вас есть такое понятие,- ``одержимость''.}
\people{Маньяк, короче.}
\people{Сталин уничтожил интеллектуальную элиту России.}
\soul{Нет, вы это сделали ранее.}
\people{Ещё при Ленине, да? Когда ранее?}
\soul{В вашем понятии, в революцию.}
\people{В революцию… Но, всё-таки, он своими ГУЛАГами, лагерями подорвал генетический фонд нации. Каким силам космоса это было угодно? Почему светлые силы оказались безучастными, наблюдая те истребления генетического фонда России?}
\soul{Почему вы решили - безучастны? Были безучастны вы. Вы шли на убой. И не очень-то сопротивлялись этому. Почему? Беды ваши, и вы их получаете по своим заслугам, а обвиняете кого-то.}
\people{Кого-то. Ясно. За какие грехи русские получили тиранию Сталина?}
\soul{Мы говорили вам, – идёт борьба. Это первое. Второе. Согласитесь, что, чем выше человек, тем ниже он упадёт. }
\people{М-да…}
\soul{Подумайте. Женщина. Вот вам пример. Она может быть самым добрым существом и самой злой, и ни один мужчина не сравняется с ней. И так народ ваш – женщина.}
\people{У нас есть ощущение, что, доведись испытать всё, что пришлось на долю россиян, другим нациям, менее стабильным, например финнам, норвежцам, итальянцам… их страны бы уже рухнули и ассимилировали. Мы почему-то пока  живём, хотя и прозябаем. Говорит ли это о силе и жизнестойкости нации?}
\soul{Мы же вам сказали: чем больше сил, тем больше вы поднимитесь, либо тем ниже упадёте, смотря в какую сторону примените её. Далее, мы не говорили вам, что вы избранные. Но… многие из вас избраны. Мы говорили вам о вашей России, что вы будете духовным центром. Идёт жестокое обучение, идёт жестокая борьба, и к вам приходят вожди ваши, чтоб уничтожить вас. Но они не знают того, и  считают, что стали …(обрыв контакта)}
1-2-3-4- 5-6-7
 
\soul{Многое, что творите, – думаете, что делаете сами, хотя, кто-то это делает вашими руками, вашими мыслями. Мы говорили вам о виртуалях. Почитайте, почитайте заключение экспертизы.}
\people{Какую экспертизу?}
 1-2-3
 
\soul{Спрашивайте.}
\people{О какой экспертизе вы говорите?}
\soul{Конечно, после смерти. Разве мог быть диагноз, и он вам известен и опубликован при его жизни?}
\people{Диагноз состояния мозга? Или чего?}
\soul{В вашем понятии – психологического здоровья.}
\people{А-а-а… Да-да.. это говорит о том, что он был параноик. Это вы подтверждаете тоже?}
\soul{Поймите, многое сделанное вами, Всего лишь,  вашими руками и не более. Вы их принимаете за свои идеи. Далее. Вы родили монстра. Вы. Ваш образ.}
 1-2-3-4
\soul{Подумайте, если б он не был рождён вами, разве мог бы он владеть вами? Подумайте, какую власть он имел?}
\people{Понятно. Но почему не смогли противостоять  при жизни?}
\soul{Зло! Почти всё зло, что было в стране вашей – воплотилось, воплотилось в личность. И та личность раскидала это зло. И до сих пор получаете. И долго будете получать.}
\people{Ясно. Ну, несмотря на уверования, что России предопределено быть духовным вождём человечества, у нас не исчезают сомнения в этой миссионерской способности русских. Мы слишком деградируем, на наш взгляд, если говорить по большому счёту. Мы жадны, ленивые, завистливы, агрессивны в массе своей. Где ж корни духовного обновления?}
\soul{Мы говорили вам: чем больше сил, тем больше упадёте, или поднимитесь. Поймите! Почти вся, в вашем понятии, энергия находится у вас же. В вашей России! И потому, используя её,  впадаете в крайности (обрвывается)}
 1-2-3-4-5
\people{Можно задавать вопросы далее?}
\soul{Далее. Не говорите о всех. Говорите о себе. В том и ошибка ваша, что вы говорите о всех, хотите весь мир изменить. А себя – нет. И вы весь мир меряете относительно себя. И в том ошибка каждого из вас. Согласитесь… (обрывается)}
 1-2-3-4-5-6-7
\soul{Почему-то вашему переводчику очень не нравится это слово.}
\people{Какое?}
\soul{“Согласитесь''.}
 1-2-3-4
\people{Найдите другую замену. ``Подумайте''. (как альтернатива. прим.)}
\soul{Спрашивайте.}
\people{Наблюдая историю других стран; США, Франции, Англии и др., мы убеждаемся, что там эволюция развития идет более цивилизованно, без колоссальных человеческих жертв. Получается, что там люди умнее и человечнее, чем в России?}
\soul{Мы говорили вам, почитайте уголовную хронику. Далее, как вы меряете? Если бы она прозябала бы так же, как прозябаете вы, вы бы не заметили тех перемен. Вы бы не заметили той духовности. Вы, за физическим благополучием, не видите их падение.}
 (обрыв)
\soul{Не чистый контакт … }
\people{Кто вмешивается?}
\soul{Переводчик находится в полном сознании. И потому, старается корректировать нас более, чем мы можем .}
\people{Скажите, может быть, тогда лишите его сознания?}
\soul{Тогда, мы применим насилие.}
\people{Ну-у-у…}
\soul{Поймите, мы играем по вашим правилам. Мы же пришли к вам,  и потому - вы хозяева. Мы, всего лишь, гости. (по правилам сознания. прим.)}
\people{Но переводчик слишком вмешивается. Все равно он не запомнит дословно диалога, а наши перебивки будут мешать течению разговора.  Ну, ладно, это вам сейчас судить… Давайте попробуем так. Если будут слишком большие помехи, то, может, изменим что-либо?}
\people{Скажите, пожалуйста, а вы когда-нибудь сможете придти вообще в сознание? Вот, как мы сейчас? Прям, чтоб мы знали, что вы пришли. }
\soul{Тогда подумайте. Ваши мысли, это что?}
\people{Это, то же самое, да?}
\soul{Подумайте. Мы говорили вам о начале мыслей. Мы говорили вам – найдите и подумайте. В вашем понятии, множество мыслей и разного качества… Подумайте. Смотря на качество, вы сможете понять, кто к вам пришёл. Вы, мы, или иные…}
\people{Понятно. У меня такой вопрос… Надо какие-то критерии, от чего мерить… }
\soul{Критерии находите вы. Мы говорили вам, если мы будем вести вас, вы никогда не научитесь сами.}
\people{Хорошо. Вопрос такой… Мне надо ваше мнение знать на такую ситуацию, - вот, допустим, я так услышал или сделал сам вывод, что  ``Всё в мире разрешено, лишь бы другим не мешать''? Пусть каждый двигается, куда он хочет, полная свобода действий, но, чтобы не мешать другим.}
\soul{Это говорит о том, что вы опускаете руки – ``пусть будет всё, как есть''.}
\people{Но, извините…}
\soul{Вы должны бороться. Бороться, но не попирая. Подумайте и вспомните – мы говорили о войне, мы говорили о споре. Вспомните.  Вы можете доказать свою истину тремя путями: физически,…}
\people{Психологически.}
 1-2-3
\soul{Тремя путями; физически, что чаще вы делаете, эмоционально, - чувствами, – это вы делаете тоже часто, и, просто, неся информацию. Книга даёт вам эмоции? Вы их создаёте сами. Книга даёт вам только информацию, всё остальное вы делаете сами. Вы переживаете, если книга… Ну, согласитесь, что книга-то несёт только информацию. }
\people{Да-а…}
\soul{И каждый, как примет эту информацию… Это и есть - множество  видений. Это и есть - параллельные миры. Даже вы, живя с друзьями, не видите, что они живут по-другому, они вас не понимают. Вот вам и параллельные миры. Вот вам множество миров среди вас. И вы сами живёте во множестве. Вы изменчивы, вы – вода. Сегодня вы одни, завтра – другие. Вот вам,  параллельные миры. А вы хотите, минуя их, уйти в другие, которые, действительно, можно назвать параллельными мирами. В физическом смысле. Но, вы подумайте, с чем вы придёте туда? Придёте, как экскурсанты? Дайте счёт… }
  (счёт)1-2-3-4-5
\people{Если смерть людей,  это освобождение души из тел-клеток, то Россия много своих сынов освобождала для космических дел. Скажите, надо ли радоваться этому?}
\soul{Как вы можете радоваться смерти? Что в вашем понятии ``смерть''?}
\people{Это потеря, конечно, для родных/близких, но, может быть для ``космического сообщества'' - это приобретение?}
\soul{В таком случае все ваши враги, что приносили смерть, – приносили пользу космическую. Ну, давайте, все пойдём на эшафот, что же здесь останется? Пустыня? Мы все уйдём куда-то выше? Почему вы хотите уйти в лучшие миры, не оставаясь здесь?}
\people{Понятно. }
\soul{Не предательство ли это?}
\people{Т.е. наши дела, в основном, на Земле должны вершиться?}
\soul{Там, где живёте, там и вершите. Если вы придёте в другой мир, что будете делать там? Вы опять попадёте в ту же среду. И вы опять захотите уйти оттуда. Ибо не будет понят и тот. Вы не можете понять тот мир, где родились. А хотите уйти в другой. Если вам трудно здесь, вы хотите уйти, где легче. Это трусость. Это предательство.}
\people{Хорошо. Я, наверно, повторяюсь… Вот, тут есть мнение, что Бог лучшим своим чадам даёт испытания. То же происходит и с русскими?}
\soul{Мы отвечали вам. Но,  поймите, испытание – не наказания. Наказания - ничто не приносили, кроме страха. Спрашивайте.}
\people{По теории получается, что социализм – это более передовой и человеколюбивый строй. Почему же он потерпел неудачу не только в СССР, но и во всём мире?}
\soul{Давайте говорить о вас. Вы говорите ``передовой''. В чём?}
\people{Ну… в том, что он проповедует равенство людей…}
\soul{Равенство? Да, вы сказали: каждому - по труду. Вы делали то?}
\people{Видимо, не делали. Ну, мы не делали в Союзе, так может, в Чехословакии это делали? Почему везде он рухнул?}
\soul{Вы, и сказано вами, не выполняете 10 заповедей Господних и обвиняете Господа в своих бедах. Вы - не выполняете идеи социализма и обвиняете его, и говорите: ``чем он плох?''. Что вы выполняли из того?}
\people{Т.е. мы не подошли близко…}
\soul{Вы говорили о равенстве. Да, вы сделали равенство. В вашем понятии, все стали бедными. Далее, вы были бы похожи, и это вам говорит уже ``переводчик''… У вас было более похоже…(возможно, о строе. прим.)}
 1-2-3-4-5-6-7
\soul{Пусть он скажет, пусть.}
\people{**– Социалистический капитализм. (Переводчик (сознание))}
\soul{Вы не согласны с ним? Да, у вас были верные идеи, но вы их не исполняли и в том беда ваша. И давайте не будем вам говорить, в чём ошибка. Ошибка ваша – во всём. Ибо, в каждом и везде вы ошиблись. Назовите хотя бы одну из ваших идей, которую вы выполняли. Даже, вы говорите ``интеллигенция''. Но разве слово ``интеллигент'' не было у вас ругательством?}
\people{Всё верно. Но нам не верится, что капитализм с его эксплуатацией людей, более человеколюбив и богоугоден, чем коммунистическая формация. В чём мы заблуждаемся? Ведь капитализм, действительно, сейчас царствует на Земле.}
\soul{Тогда, давайте проведем аналогию. Библия и капитализм. Посмотрите внимательно. Увидите  там многое похожее. Там говорилось об ``овцах''. Вы же, про капитализм говорите - ``рабочие''. В чём разница? Далее,  эксплуатируете…}
 1-2-3-4-5-6
\soul{Как вы понимаете – ``эксплуатация''? В таком случае, здесь нет ни одного свободного человека, ибо все эксплуатируемые. Нет воли. Нет. Ибо вы зависите от  обстоятельств. Вас эксплуатируют обстоятельства. Подумайте. }
\people{Ну, да.}
\soul{Тогда, что ж получается, – все вы рабы? Далее, ваши ошибки. Мы говорили вам: религия, вера и Бог – разные вещи. Вы подумайте, что говорит ваша религия. Она говорит о послушании. Она говорит вам, чтобы вы стали рабами. И подумайте, вы имеете детей. Вы хотите, чтобы они были вам рабами?}
\people{Нет.}
\soul{Почему же тогда ваши религии говорят это?}
\people{Да, мы чувствуем несоответствие.}
\soul{Далее, вы говорите об эксплуатации. Вы работаете, и вы получаете… и вы вспоминаете об эксплуатации только тогда, если мало получаете.}
\people{Понятно.}
\people{Ну, да, в принципе. Кнутом нам никто не бьёт по спине.}
\people{Мы убедились, что социалистическому строю под силу грандиозные дела. Мы выиграли жестокую Великую Отечественную Войну, возвели огромные плотины, освоили целину, первыми вышли в космос. Трудно поверить в несостоятельность, хотя бы в будущем, такого строя. Что скажете на это вы?}
\soul{Мы скажем вам, что вы были одержимы. И потому всё делали. Подумайте, многим ли вы были хорошими?}
\people{Да.}
\soul{Вы создали плотины – и что же? Вы решили природу выкроить по своим меркам. Что было потом? Что вы делали? Зачем вы строили эти плотины? Зачем выходили в космос? Для души? Или, быть может, для тела?}
\people{Ну, это извечное стремление человечества открывать новое что-то. Но, главное, что социалистический строй сумел сделать это первым.}
\soul{Скажите, что вы открыли нового, перегородив реки? Да. Вы открыли русла, они стали сухими и вы смогли теперь понять, что такое голод. Смогли понять, что такое экология. Это, то новое, что вы   искали?}
\people{Мы получили дешёвую электроэнергию, и, благодаря этому, прогресс у нас  произошёл.}
\soul{А может быть - смерть?}
\people{Смерть?}
\soul{Может быть, это и есть ваша медленная ``дешёвая'' смерть? Подумайте, что вы будете делать скоро, если вы будете продолжать те ми же темпами? И подумайте, если бы ваш социалистический строй двигался такими же бы темпами? Он так же перекрывал бы плотины, так бы летал бы в космос, оставаясь при этом обыкновенными людьми – кухарками. Согласитесь, что вас надо было бы вовремя остановить, иначе бы вы столько ``наворочали''!}
\people{Ну, это и было сделано в 90-х годах, да? Нас остановили?}
\soul{Мы говорили вам о душе. Мы говорили вам о душе города, о душе планеты и далее. И когда мы говорим вам: вы сами остановились. Мы говорим о душе вашей планеты. Многие индивидуумы понимают истинное положение ваших открытий. И, видя, страшились того. Вот вам сообщества; индивидуум, город, страна, планета. Вы, и мы говорили вам это часто, вы владеете своим миром, вы владеете собой. И если вам скажут: душа планеты остановила вас, это тоже не ложно. Ибо вы частица той души. И ваше решение, ваш голос был важен.}
\people{Скажите, а если он не сыграл большой роли? Ну, мало таких людей. Их никто не послушал.}
\soul{Нет таких людей. Среди вас, в вашем сознании, есть люди ниже и выше вас. Подумайте, мы говорим о сообществе и называем ``душой планеты''. Как вы можете сказать, что в этой душе есть частицы, которых никто не слушает? Где же единство, где же целостность? Даже физически вы можете доказать, что такого невозможно. Подумайте.}
\soul{1-2-3}
\soul{Спрашивайте.}
\people{А вот, скажите… К каким слоям относятся те индивидуумы, которые уже понимали пагубность социалистических … ну, крупных преобразований планеты. Это ученые, политики или философы? Кто, всё-таки, в большей массе?}
\soul{В большей массе? }
\people{Да.}
\soul{Все. Все слои понимали. Разница только в том, что чем выше пост, тем больше он скрывал свои чувства…}
 1-2-3-4-5-6-7-8-9
\soul{Было больше тех, кто боялся, боялся выступить. И чем выше пост, тем больше страха, ибо больше ответственности. Нельзя говорить о процентах.}
\people{Но кто, всё-таки, был впереди по интеллекту, учёные или философы?}
\soul{Почему вас интересует это? Почему? Что заставило вас задать этот вопрос?}
\people{Ну, у меня есть элемент недоверия к учёным.}
\soul{Ваши учёные управляли. Ваши. А кто дал им команду, чтоб они…?}
\people{Политики или философы.}
\soul{Политики. Тогда подумайте кто выше? Политики не учёные. Они навязывают свои идеи учёным. Что делают учёные? Они поступают не лучшим способом. Они выбираютсамый наилучший, как угодить политике. И ваши науки – всего лишь…политика.}
 (обрывается).
 1-2-3
 
\soul{Вспомните. Вспомните вашу медицину. И вспомните её запрет. Вспомните. Переводчик подсказывает нам. Да. Вспомните Сахарова, вспомните Чайку. Вспомните. Полезное пропадает…}
 (обрывается)
\soul{Обман. Политики лгут, учёные лгут в угоду политикам. Кто меньше всего будет лгать? Подумайте. Обыкновенный человек, не имеющий пост. Ему нечего терять.}
\people{Вот, они-то и двигали до прогресса умственного развития и понимания вот этого.  Простые люди, да?}
\soul{Все вы просты, только должности у вас разные. И вы, вашу должность принимаете за сущность, хотя, это, всего лишь, одежда.}
\people{Скажите, может быть, несостоятельность социалистического строя в том, что он, провозглашая благо для всего сообщества, оказался жестоким по отношению к отдельным индивидуумам, просто к людям. Мы правильно поняли?}
\soul{Социалистический строй был жесток, или всё-таки были жестоки вожди и люди? Как вы понимаете? Как вы разделяете ``социалистический строй'' и ``людей''?}
\people{Вы правы.}
\soul{Социалистический строй был жесток, или всё-таки люди? Которые, под флагом социализма, творили грязные дела.}
 1-2-3
\soul{Спрашивайте.}
\people{Вы правы, надо было мне по-другому задать вопрос. Видимо, люди. Но, странное дело, во всех странах, а их было много – почти половина, на земном шаре, везде получалось, - люди ошибались одинаково. Нигде соц. строй не показал свои преимущества. Почему так получилось дружно?}
\soul{Поймите, все вы одинаковы. Да, вы имеете разную одежду, вы имеете разную национальность, но вы же все одинаковы. В вашем понятии, ваша цивилизация, она везде одинакова. Где-то больше, где-то меньше  всего лишь только техники. Духовны, как это  будет обидно вам слышать, духовно, пока что, больше там, где меньше ``цивилизации''.}
\people{Состоятельны ли на ваш взгляд идеи коммунизма и социализма? И почему ``да'' или почему ``нет''?}
\soul{Почему вы спрашиваете нас? Это было придумано вами, и мы говорили вам, и вы спрашиваете дальше.}
\people{Но вам со стороны виднее наши ошибки.}
\soul{Мы должны прийти и спросить к вам, что у вас плохого и что хорошего? Что помешало вам или, что двигало. Вы спрашиваете нас. Почему? Вы же знаете ответ. Знаете ответы на множество вопросов. Вы знаете их, но не доверяете себе. Прежде, чем задать вопрос, записать, вы уже ответили на него. Ответили. Вспомните. И вы, лишь хотите только наше подтверждение, не верите себе.}
\people{Понятно. Но нам нравились идеи коммунизма, но почему они рухнули все? Это ставит нас в тупик.}
\soul{Вам нравятся идеи коммунизма? Тогда скажите, многое ли вы сделали, чтобы они жили? Много ли сделали ваши… (теряется связь).}
 1-2…
\soul{Спрашивайте.}
\people{Т.е. мы мало сделали? И, поэтому, винить отдельно строй не имеем права. Так видимо?}
\soul{Вы больше похожи на спортсмена, который жаждет добежать первым, а что по сторонам, это никого не волнует. Вы бежите, толкаетесь. Вокруг вас падают, а вы бежите. Ваша цель, – финиш, вот ваша задача. Вы похожи на бегунов, которые заботятся только о себе и лишь бы первыми добежать. Вот вам, ваш коммунизм, ваше понятие коммунизма. Что вы можете сказать о коммунизме? Вы говорите об идее? А много ли вы их знаете, читали ли вы полностью всё, что писалось о коммунизме?}
\people{Нет, конечно.}
\soul{И все ваши произведения прошли через политику. И многое было вычеркнуто. А ведь и в этих трудах было сказано о коммунизме и плохое. Сами авторы говорили и указывали ошибки в начале. Указывали, к чему может привести. Если б всмотреться, то вы бы нашли, что они были правы. Вы же, пропустили через политиков. И они, и вы часто это делаете, всё, что не угодно – убираете. И потому, ложно. Мы говорили вам: ``истина'' и ``правда'' – в чём разница? Правда – это отражение истины. Отражение в зеркале, которое столь криво, и кривится оно вашими эмоциями, вашими понятиями о долге, вашим понятием о … (обрывается).}
 1-2-3
 
\soul{Спрашивайте.}
\people{Хорошо. Насколько мы информированы, западные страны не имеют задуманной далеко наперёд совершенствования общества, за исключением, может быть, провозглашённого стремления обеспечить всех по потребностям. Так что, в этом их и успех? В этом и перспективы их общества?}
\soul{Почему вы задаёте такие вопросы? Вы боитесь отстать? В Вас  больше страха. Страх остаться последним. Вот что заставляет задавать этот вопрос.}
\people{Нет.}
\soul{Вы похожи на ревнивую девицу, которая говорит ``почему социализм лучше, капитализм – хуже, но почему при капитализме лучше живут''? В вас говорит страх. Поймите, любая политика, любой строй – всё будет утопией, утопией, если вы не изменитесь духовно! Ещё в давние времена уже мечтали, как вы говорите, о ``коммунизме''. И что же? Мечтали о строе. О cтрое, но не о людях! Никто никогда не говорил об индивидууме в этом строе. Всегда говорили ``строй, строй'', ``народ, народ'', и никогда не говорили о человеке, никогда не говорили о чувствах. А если и говорили, - вспомните – как? Да, вы будете равны, не будет денег, и далее, далее, – только так вы говорили о людях. Как же вы можете говорить, почему было плохо, если изначально вы приняли, приняли искаженно. Приняли искаженно, и, мало того, исказили ещё больше. Исказили, в вашем понятии, о долгах, о любви.  Вы…(теряется связь).}
 1-2-3-4-5-6-7-8
 
\soul{Спрашивайте.}
\people{Хорошо. Судя по Нострадамусу, развал СССР был предопределен. Но сейчас об этом знают многие люди. Почему, исторически, развал был необходим? Исторически был неоходим.}
\people{Мы же говорили вам о душе планеты. Подумайте. С вашим фанатизмом, вспомните как вы… Вы же придумали пятилетки. Теперь представьте, если их будет множество. За одну только пятилетку вы наворочали столько, что до сих пор не можете расхлебать! А представьте, что если это будет 100, 1000 лет ? Что будет?}
\people{Угу. В этом наши ошибки… Ясно.}
\soul{Спрашивайте.}
\people{Скажите, пожалуйста, может, если избрать путь ближе к природе, это будет правильно?}
\soul{Вы опять возьмите крайности. И у вас это уже есть. Да, вы говорите: ``Уйдём к  природе, уйдём в пещеры''. И что? Вы опять вернётесь в средневековье. И опять вернётесь и станете дикими. Зачем же? Мы говорили вам - вы рабы техники. Да, вы – рабы. И мы говорим вам: перестаньте быть ими, перестаньте и растите духовно. Вы должны вырасти духовно. Вами же сказано - ``родиться заново''. Какой бы строй к вам ни пришел, и кто бы к вам ни прилетел – он не изменит вас. Он не даст вам лучшее, если вы не изменитесь сами. Любой строй, любой строй - вы могли бы жить. И потому, изменившись, вы бы не создавали никаких строев. Вы бы просто жили. Просто жили. Подумайте. И вспомните животных. Хотя бы дельфинов. Подумайте, почему вы строите заводы? Потому, что вы не можете уже без них. А когда-то ведь могли! Вспомните, совсем недавно вы могли существовать  без заводов, и вы жили, и жили прекрасно. И вас было много. А сейчас вы стали не нужны. И убрав их (заводы), вы погибните. Почему? Потому, что вы стали их рабами. И подумайте, почему те же дельфины не строят заводов? Они просто живут.}
\people{Но они могли бы построить заводы?}
\soul{Зачем? Зачем им нужны протезы?}
\people{Я к этому и клонил, что они в природе живут, они часть природы, и природа о них естественно беспокоится. Чтобы они не исчезли сами.}
\soul{Подумайте, если вы сейчас уберёте всю вашу технику, вы погибните. Мы же говорили вам. К сожалению, вы должны теперь идти медленнее, вы будете идти медленнее. Да, у вас произойдёт духовная революция. Духовная. Поймите никакая другая, только духовная. И когда вы изменитесь, у вас будет другое отношение к технике. Совершенно другое. И вы не будете уже её рабами. Вы будете её строить. Да, вы будете могучей технократической страной, но не рабами её.}
\people{Но ведь чтоб строить технику требуются ресурсы. Мы больше будем выкачивать из земли. Т.е. будем наносить природе непоправимый ущерб?}
\soul{Вы слепы и не видите множество энергии, что вокруг вас. Вы не видите и берёте то, что уже лежит. Подумайте, вы говорите об энергии. Тогда, представьте дикаря, который добывал энергию огнём. Если бы ему сказали, что существует электричество, что бы он вам сказал? Он бы вас не понял. А теперь подумайте, может вы тоже дикарь с электричеством и ваш костёр, всего лишь, электрический? И будет более другие энергии, чем вы сейчас понимаете.}
\people{Мы подозреваем, что так оно и есть.}
\soul{Подумайте, мог ли дикарь видеть электричество, солнечную энергию? Мог он видеть?}
\people{Нет.}
\soul{Нет, и потому не мог использовать.}
 1-2-3
\soul{Спрашивайте.}
 
\people{Т.е. получится, что мы уйдём от того, чтобы губить Землю и будем пользоваться тем, что вокруг ``бесплатно'', так сказать?}
\soul{Да, вы привыкли бороться. Вместо того, чтобы подойти и попросить – вы отбираете. Для того, чтобы взять, просто взять,  вы не можете – вам надо завоевать. Вспомните про ваш строй: ``завоевали'', ``победили'' – это ваши слова. Мы когда-нибудь говорили; ``взяли'', ``отобрали''? У природы отобрали то-то, то-то; победили природное то-то, то-то; завоевали у природы то-то.?  Вспомните. На кого же вы похожи?}
\people{А природе не безразлично как мы говорим: завоевали, победили, или попросили?}
\soul{Разве мы о словах говорили? Мы говорили, как вы действуете. Подумайте, вы плотины строили – что, вы просили? Или может быть, вы просто взяли? Взяли, не навредив никому. Может, вы завоевали? Вы же, вы сказали: `` И великая река Волга, окована цепями''. Цепями. Вспомните. О какой свободе говорите вы? И говорите вы, что рвали эти цепи.  Вспомните ваши лозунги. Вы сами создаёте их.}
\people{Но ведь, от перестановки мест слагаемых сумма-то - не измениться. Рыба-то будет гибнуть - на нерест не идёт. И, опять же, получается, по-любому, попросили мы или не попросили - рыба будет гибнуть.}
\soul{Если бы вы жили бы в природе, вам не нужно было бы строить те плотины, вы могли бы взять и так. Вы же, страдаете гигантоманией. Социалистический строй только построил такие плотины – вот вам пример. Почему же при капиталистическом строе энергии не нужно столько, сколько у вас? Нет, – у вас гигантомания. Вы, всегда хотели больше. Если строить, так большое, если завоёвывать, так всё.}
\people{У нас были возможности…}
 1-2-3-4
\people{…мы могли концентрировать…}
 (идет сбой записи)
\soul{… или о рабах?}
\people{Ну, капиталисты не могли строить огромные плотины, а социализм мог выделять… Ну, конечно, за счёт обнищания народа, понятно.}
\soul{Капитализм не делал из рабочих рабов. Вы же, и ваше всё, построено рабами. Разница лишь только в том, что они носили имена вместо номеров и назывались ``рабочими''. Всё остальное – рабы.}
\people{Ясно. Крушение СССР повлекло войны и беды, которые мы сейчас наблюдаем. Неужели в общепланетарном масштабе это благое дело?}
\soul{Мы не говорили вам о благом. Мы говорили вам о ноже. И, какое ``благое'' вы видите в этом? Какое? Сколько крови вы принесли бы не только своей стране, но и другим? Это благое? Вы пришли с караваем хлеба в одной руке, а в другой держали меч. И если отказывались брать хлеб ваш, вы рубили мечом. Вот ваша доброта.}
\people{Ну, вот смотрите, если рассматривать общую ситуацию в мире. То уход со сцены СССР снижает независимость третьих, развивающихся стран. Они легче попадают в зависимость от капиталистов, монополий. Мы не правы?}
\soul{Представьте, вы сейчас находитесь в капиталистической стране. Ведём этот контакт.  Теперь, подумайте, вы – капиталист, и вы спрашиваете тот же вопрос. Только вместо ``социализма'' говорите –“капитализм”: ``Вот,  мы теряем капиталистические страны, капиталистический строй, и потому  - независимость…''. Вы можете провести далее аналогию?}
\people{Ну, сейчас, вполне возможно, что без всяких усилий любая страна, то же США, завоюет Гаити, или другую какую страну, и Советский Союз не сможет вмешаться, как раньше бы он вмешался. Получается, действительно, ``большой полицейский'' может творить что угодно. У других стран теперь нет защитников.}
\soul{Давайте проведем ту же аналогию. Вы в капиталистической стране и вы - капиталист. И вы будете говорить теперь о социалистах: ``Любая социалистическая страна захочет завоевать…'' и далее-далее.}
\people{Понятно.}
\soul{Увидьте, прежде всего, себя. Со своих понятий.}
\people{Интересно, в силу каких понятий ещё мы мерить можем?}
\soul{У вас много понятий, и, хоть они ложные – это ваши понятия, вы подчиняетесь им. Вы решили, что вы лучшие, остальные строи – хуже. Вот была эта ошибка, почему вы погибли. Вместо того, чтобы идти в ногу, вы стали бороться только из-за того, что там другой строй. То вы говорили: ``пролетарии всех стран, соединяйтесь'', и в то же время, тех же пролетариев уничтожали, потому что, они живут в другом строе.}
\people{Вы сказали, что человек живёт только тогда, когда он борется. Что ж, мы, если б мы не боролись – померли что ли?}
\soul{Как вы понимаете ``боретесь''?}
\people{Ну, впрямую и понимаю.}
\soul{Впрямую? Впрямую - ваша борьба – это физически, это драка, это войны?}
\people{Ну, не только, но и преодолевать трудности.}
\soul{В чём трудности были у вас? Почему вы физически преодолеваете физически? Если к вам идёт навстречу человек, и вы считаете, что это преграда – вы его уничтожаете только из-за того, что он идёт не в вашу сторону, носит не ваше имя и не похож на вас. И вы это называете борьбой? А может быть, это просто убийство? Убийство, чтобы самим дойти.}
\people{Хорошо. А теперь, посмотрите на другую сторону вопроса,- идёт человек, чтобы вас убить, а вы встречаете его с распростёртыми объятиями. В результате получаете ``нож в спину'' или в сердце. И что?}
\soul{Что? }
\people{Да.}
\soul{А как вы можете получить ``нож в грудь…''(теряется)}
 1-2-3-4-5-6-7-8
\soul{У переводчика есть интересная идея…}
\people{Он не может высказать?}
\soul{Если вы сумеете и встретите, он не будет вашим убийцей. Поймите, подобное притягивает подобное. Это закон, и не вам его менять. Если вы ждёте с распростёртыми руками, и где-то в глубине думаете, как бы он не ударил в грудь вам…. Он ударит.}
\people{Ясно.}
\soul{Далее, … (теряется)}
 1-2-3-4
\soul{Вы говорите -  не помните прошлую жизнь?}
\people{Да, к сожалению.}
\soul{А что заставило задать этот вопрос? Может быть вина ваша?}
\people{Может быть.}
\people{Продолжим. Если правы те, кто называл Советский Союз ``империей зла'', то сейчас, после развала СССР, обстановка политическая, эмоциональная, моральная, на планете Земля улучшилась. Вы отмечаете это улучшение?}
\soul{Нет.}
\people{А почему?}
\soul{Под каким миром вы понимаете ``лучший мир''? Если мы вам скажем: какими вы были, такими и остались. Мы вам говорили: какими вы были в средневековье, такими и остались. Вы не изменились. Физически – да, вы изменились сильно. Духовно, - нет. По каким критериям вы можете сказать ``лучше'' или ``хуже''? Почему ваши критерии вы навязываете нам? Почему мы должны говорить ``лучше'' или ``хуже''? По каким мерам вы меряете?}
\people{Ну, может вам лучше наблюдать со стороны. Вы в эмоциональном мире.}
\soul{У нас нет мер ваших. Поймите,- нет! Мы говорили о единстве, вы же говорите хуже/лучше. Как мы можем сказать, хуже у вас или лучше? Мы должны быть вашими мерами?}
\people{Ну, послушайте. Кончилось противостояние стран. То есть, ушла эманация зла, между  друг другом, страха перед друг другом.}
\people{Борьба сменилась сотрудничеством…}
\people{Должно было улучшиться состояние эмоциональное в мире?}
\soul{Это, всего лишь, ``ожидающий зверь''.}
\people{Что? …}
\people{Ожидающий зверь? А почему?}
\soul{Скоро, скоро вы поймёте сами, найдёте сами ответ и будете переживать его. Мы говорили вам, помните, о возможности конца света, в вашем понятии. А нужен ли был бы конец света, если бы всё вы жили дружно, если было бы понято, если бы у вас было сотрудничество? Зачем тогда вам ваши ``дубинки'', если вы сотрудничаете. Зачем?}
\people{Страх войны-то ушёл. Опасения-то уменьшились. Как мы сейчас будем друг против друга воевать, если мы идём одним путём и американцы и русские?}
\soul{Разве? У вас держится только на тех ``дубинах''. Вы боитесь быть уничтоженными. Вы боитесь, что та ``дубина'', другая, у него, - может оказаться сильнее. И потому, вы в перемирии. Подумайте, не звучит ли это глупо? Но, всё-таки, это говорите вы. Вы живёте так. Для того, чтоб не было войны – надо создать больше оружия, более мощное. А вы живёте так и делаете так. Что вы боитесь? Что вам мешает развязать войну? Ответный удар.}
\people{Сейчас, кстати, к вооружениям мы  достаточно прохладно относимся…}
\soul{Далее, вспомните Иуду. Вспомните. Разве он шёл войной? Нет, он помогал ей. Он помогал, но предал.}
\people{Роль США примерно такая же будет? Да?}
\soul{Вот вам, пожалуйста, пример,- что лучше, что хуже. Почему вы не спросили  роль СССР?  Потому что вы не из США?}
\people{Потому что мы исчезли как СССР…}
\soul{Нет. Вы не исчезли. В этих словах было  больше смысла, чем вы даже знали.}
\people{Минутку! Поясните, пожалуйста, последнее. ``Больше смысла, чем вы даже…''?}
\soul{Вы говорите, что не существует СССР, а мы говорим: ``вы остались''. Вы остались и ещё соединитесь. Далее. В этих словах было больше смысла, чем вы знали.}
\people{В смысле, что мы соединимся или, что мы исчезли?}
\soul{Мы говорим о буквах, но не о словах.}
\people{Это тайна? Вы не можете сказать? Почему три ``С'' чем-то значительнее?}
\soul{У вас есть люди, и вы их знаете, которые вам могут ответить на это.}
\people{В нашем окружении? Хорошо, надо потом спросить… Скажите, возродятся ли идеи коммунизма, как на то надеется видоизменённая компартия?}
\soul{Мы множество раз отвечали вам на эти вопросы. Мы отвечали вам о любых партиях, отвечали вам о любой стране, а вы продолжаете и продолжаете спрашивать, спрашивать. Почему вы спрашиваете то, что уже известно, что уже было вам сказано? Почему не услышано вами? Поймите, мы говорили вам: в любом строе вы можете жить, лишь бы жили по-человечески. Будьте человеком! И всё! Став им, вы не будете говорить о строях.}
\people{Возможно, после бесед с вами, мы начнём в наших статьях это проповедовать, пропагандировать, но нам надо досконально разобраться в правоте или не правоте наших мыслей.}
\people{А что вы считаете ``человеческим'' из таких вот; зависти, самомнения, навязывания своего мнения другим, допустим даже в разговоре; или наоборот, полная свобода, согласие со всеми, непротивление злу, как проповедовали, что ``не противься злому''? Типа того,- зло само себя накажет. У меня, честно говоря, в Библию больше веры, потому, что вы противоречите, иногда, сами себе. В одном контакте  одно говорите, а в другом,  всё наоборот. Опровергаете себя.}
\soul{Приведите пример ``наоборот'', – первое. Второе. Вы, и мы говорили вам, соглашаетесь только тогда, когда это вам угодно. Далее. Вы спрашиваете, о человеческом в чувствах. Мы говорим вам: будьте человеком. И сказано в той же Библии, о которой вы говорите, как быть человеком. В чём противоречие? Далее. Будь человеком – почему вы сразу говорите об отдельных качествах? Почему вы говорите о зависти или, наоборот, о любви? Почему вы делите? Почему, если даже вам в вашей Библии же сказано: ``человек - сын Божий''. Подумайте. Даже здесь говорится о единстве. Вы же разделяете. Вы же говорите: зависть, любовь, зло, гневность и далее. И вы называете это человеком. А почему вы не вспомните ``любовь к матери''? Почему не вспомните и не скажете: ``это человек''? И вы даже не замечаете того. Когда к вам приходят с любовью, вы не замечаете. Вы не называете его человеком, вы говорите: ``он ненормальный''. Все чувства, если они открыты, вам чужды. Вы пугаетесь их и говорите: ``ненормальные'', ``больные'', ``слишком эмоциональные'', ``нервные'' и далее, далее…}
\people{Есть такое, да.}
\soul{Далее. Вы что, хотите услышать рецепт? Рецепт о человеке? Конкретно? Вам нужно конкретно сказать все чувства, которые должны быть у человека? По процентам, количеству? Тогда вы будете знать, что такое ``человек''?}
\people{Нет, не утрируйте понятия. Я хотел сказать такое. (обрывается запись).}
 (продолжение)
 
\people{…в принципе, своими мыслями, которые навивают ему чувства. Допустим, считает, что это неправильно. И всё – под откос пошло целое учение Маркса, допустим. Потому, что подумал не так, посчитал, что так нужно будет всем. Вот я про эти чувства и спрашиваю. Если он тоже был человеком, тут, в принципе, тогда, ничего такого нечеловеческого и нету, раз присуще всё человеку, все чувства, от самых низких, до самых высоких.}
\soul{Мы же говорили. Говорили о силе. И говорили, чем больше той силы, тем больше опуститесь или тем меньше. Вспомните. Вспомните. Да, человек, потому и человек, что он обладает большей силой, чем остальные. Он может подняться ``выше'' любого животного или опуститься ``ниже'' любого животного. В том и сила его. Потому, он и человек, что обладает большей энергией. Большей. И потому, в нём будет злости и любви больше, чем у любого  животного.  Он опустится или поднимется выше любого. И это есть - человек. Это. А вы говорите - чувства. Тогда подумайте: у животных они тоже имеются, и тоже есть любовь, и тоже есть зависть. Они - человек?}
\people{Они - тоже человек. Вы сами говорили, что всё – один человек.}
\soul{Тогда вспомните: вы спрашивали, а мы отвечали вам разницу между животными и вами. Вспомните. Мы говорили ``всё едино''. Вспомните. Но, и мы говорили вам, что вы обладаете энергией большей. Почему вы не помните? В чём вы видите противоречие? В том, что вы забываете. В том, что вы, что говорим, принимаете по-своему. Перекладываете на свои слова, и потому, непонятные вам.}
\people{Согласен. Может быть элемент…}
\soul{Приведите. Приведите…}
1-2-3-4-5-6-7-8-9
\soul{Вы говорите сознанием вашим. А вы знаете сознание ваше? Сегодня вы - один, завтра – другой. Сегодня у вас одна цель, завтра - другая. Выдаёте и рассуждаете сознанием вашим. Логически. А приходят к вам не в сознание. ---Сознание не может обладать энергией. --- Сознание радуется, в вашем понятии, знаниям. Вот вам. Вы говорите: вожди тоже были людьми, тоже человеки, и ничего нечеловеческого у них не было. Да. Но вспомните о Христе и о Гитлере. Далее. Как вы можете понять, если вы не знаете начала? Не знаете начала своих знаний. Как вы можете говорить об этом? Вы спрашиваете у нас, кто мы, и спрашиваете, кто - вы. Даже про себя ищете ответ от других.  Вы не можете понять себя. Сегодня, вы добродушный, завтра – ненавидите себя. Вот, ваше  противоречие. Вы - вода. Вы – непостоянство. Вы хотите увидеть и осознать всё.   Когда-то, вы пришли и сказали нам:  ``Я пройду сам!'' Завязали себе глаза и что же? Ослепли. Забыли о той повязке и не снимаете её, и кричите. В вас есть личность, которая видит этот порядок. Вы не помните. Вся жизнь ваша, чтобы познать. Да, это интуиция. И вспомните, что может делать она. Что может сделать человек с закрытыми глазами? Много ли он поймёт?}
\people{Ну, а как открыть их? Вы даже говорите, что мы  даже не знаем. Вы говорите, что мы не знаем эти силы. Естественно, если у нас глаза завязаны, мы её не видим просто. Вы нам намекаете, что есть у нас эта сила, а мы даже не знаем ту сторону, где её искать. Внутри себя её искать?}
\soul{Вы знаете. Должна видеть душа ваша. Ничто, ничто не сможет вам помочь, если вы не сможете увидеть сами. Подумайте, к вам приходил Христос. И что же? Сколько  противоречий было найдено в нём и  упрёков? Слепцы! Ну что? Нашли? Нашли? И любовь придёт к вам в душу  и ничего не сможет сделать, ибо закрыты вы. Вспомните.}
\people{Да, я помню.  ``По вере вашей и воздастся''. }
1-2-3…
 
\soul{Спрашивайте.}
\people{Вопрос такого рода. Скажите, а вот Христос, он был, в принципе, человеком? Он же не из другого мира пришёл? Ну… в предыдущем воплощении. До ``Христа''. }
\soul{Мы говорили вам, что даже ваше подсознание знает. То, ваша душа. Да, Христос пришёл к вам. Он был человечнее вас. И он человечней любого из вас. Вот и вся разница. Мы говорим вам, что он был человек, и говорим вам, что вы – человек. Мы говорим вам, о человеческом в вас.}
\people{Вопрос такой. Вот вы мне сказали, что вы ``гуляете по другим мирам'', типа того, то с чем вы придёте, то и будет вам. Если всё едино, в принципе-то, если с ``большой горы'' смотреть, все параллельные миры, это одно и то же, то, почему бы не прийти туда, посмотреть, какая жизнь там творится?}
\soul{Как же туда можно придти вашими плотями?  И это же вами писано, туда можно прийти только тогда, если, как раз, не видеть его. Как мы можем привести вас? Как вы можете придти туда, где вы уже находитесь? Вы слепы, и для вас нет единства. Нет. И потому мы приходим и говорим вам про стены, где вы находитесь. Не познавши единства, вы будете, как слепой щенок, искать выход. Искать, вместо того…}
1-2-3…
\soul{Вы создали множество религий, вы создали множество наук. Многие из них уже умерли, есть сейчас, и будут новые. Вы создали множество направлений. Духовных направлений. И по любому из них не идёте. Вы не видите, о чём говорят они. Да, вы понимаете и говорите ``я понимаю душой, что это верно, это направление, и, да, надо идти по нему''. Но ведь не идёте же, не идёте! Если приходите в церковь, вы молитесь, не очень-то и верите тому. Ибо, если приходите и говорите:  Боже, за что мне такое наказание?  Подумайте, уже даже этим вы обвинили Бога, но не себя.  Что же тогда? Придёт к вам и будет помогать? И, получается, что он согласился с вами, что он наказал вас. Даже здесь, в малом, вы лжёте, не зная об этом. И вас не будут ругать за ложь, ибо вы ребёнок со своими понятиями. Но и не будут поощрять. И с этой ложью вы будете идти по ложным дорогам. К вам будут приходить, и к вам приходят и мыслями, и в книгах, и приходят наяву и во снах, и говорят вам.  И что? Вы  не верите. Вы считаете это ложью, фантазией и далее и далее. Потому что у вас другое, своё понятие о правде. Свои и только. И кто бы к вам ни пришёл, кто бы ни пришёл, пускай придёт к вам даже Бог, вы не поверите ему. Вы будете слышать только то, что вам больше подходит, что вам больше нравится.}
\people{Вы так нарисовали образ ``Фомы неверующего'' в большом масштабе.}
\people{Ну, мы с этим сталкивались уже.}
\soul{Ну, вы же и есть. Вы же и есть. Как вы можете говорить, что вы верите? Как вы можете говорить, если вы задаёте столь наивный вопрос! Вы говорите: ``я познал единство'', и тут же опровергаете его вопросами. Как же вы его познали? Если б вы познали то единство, вы бы разговаривали с нами. Разговаривали и были б нами. Тогда, можно было бы говорить, что вы познали единство. В этом же вам  требуется ``переводчик''. Переводчик имеет своё понятие о единстве. И, тем более, сейчас, когда он находится в сознании, он старается корректировать.}
\people{Почему наши учителя, силы космоса, вовремя не указали нам на явные ошибки коммунистической доктрины? Кратко, если можно.}
\soul{Давайте не будем говорить об этом. Подумайте, мы же говорили вам о любом строе и только что говорили вам о боге и о вашей молитве. И только что говорили вам, что вы наказываете себя. Вы говорите - ``космические учителя''. А кто, кто верил им? Вспомните. В вашем строе, если говорить о ``космических учителях'', что было с ними?}
\people{Неверие…только сейчас, и то, мало кто верит. В чём ошиблись Рерихи, когда поддержали Ленина, заявив, что он чуть ли не посланник Шамбалы?}
\soul{А вы никогда не ошибались, поддерживая кого-то?}
\people{Точно так же и Рерихи, да? Хорошо. Какими силами, если не самой Шамбалой, инспирировано письмо в поддержку Ленина?}
\soul{Нет. Мы вам скажем, что Шамбала письма не пишет. Ни одного письма ещё не было оттуда. Были всего лишь попытки, перевести мысли, несущие это послание. Поймите, переводчик, тот же переводчик пишет это письмо.}
\people{Думая, что это от Шамбалы, да?}
\soul{Почему же? Это может быть и от Шамбалы. Но это может быть искажённо.  Далее. Вспомните, кто были они? Люди. В первую очередь люди. И у них тоже так же, как и у вас, только что направление немного другое. Иное. Воспринимали больше и лучше, но они тоже ошибались. Нет среди вас тех, кто никогда не ошибался. Нет. В том счастье ваше. Счастье, что вы можете ошибаться, даже если это плохо, что ж.}
\people{За это понесём наказание какое-то.}
\soul{Мы говорили вам о наказании?}
\people{Ну, религия говорит об этом.}
\soul{Вы сами себя накажите. Никто другой. Никто другой. Вы.}
\people{Ну, правильно – обратная связь!}
\soul{Далее, мы говорим вам: ваше счастье, что можете ошибаться. И мы говорим  вам, что вы человек. Человек может ошибаться. Не может ошибаться только машина.}
\people{Тогда, минутку! Такой вопрос… А за что же тогда, если мы можем ошибаться, и это, в принципе ``человеческое'', вот  ``получать'' за свои ошибки? Или это неизбежное, всё-таки? Это не грехи?}
\soul{Вы говорите о единстве. Мы вам только что отвечали. Мы говорите о многом, и не поняли. Ну, хорошо, если вы сейчас будете идти и стукнетесь об угол дома. Что, угол дома наказал вас?}
\people{Ясно.}
\people{Был ли Сталин тем посланцем тёмных сил…}
\soul{Уже спрашивали…}
\people{Уже спрашивали? Ладно… Но Сталина не было в других социалистических странах, а результат один – крах. Значит не в личностях дело, а в порочности всей системы? Так ли это?}
\soul{Мы же говорили вам. Ну, почему вы повторяетесь? Ну, почему? Мы же вам говорили и в прошлом и сейчас. Далее, мы же говорили вам, даже о личности скажем - все остальные были, всего лишь, вашими марионетками.}
\people{Ну, вы говорили, что на отрицательных примерах социализма…}
 1-2-3-4-5
\soul{Мы говорили вам, что мы не физические. Мы не имеем физической  материи. Никакой, которая известна или не известна. И потому мы приходим к вам и используем вашу физику, более тонкую, которая есть в вас…}
\people{Но, она-то с вами контактирует? Грань имеет какую-то общую?}
\soul{Мы говорили о единстве. Конечно. Конечно, иначе, как мы могли б говорить чтоб вы росли  духовно,  если ваше физическое не имело никаких граней с тем духовным?  Как бы мы тогда говорили вам? Какое могло бы быть рождение духовности? Далее. Мы используем самые тонкие тела. В вашем понятии - мысль. Вы можете сформировать свою мысль такою, какой она пришла, высказать её вслух?}
\people{Да, вряд ли.}
\soul{Вот и подумайте, сколько вы искажаете.}
\people{Но, ведь вы сказали, что у вас тоже есть высший план, так сказать, ``духовный''?}
\soul{Мы говорили вам, что мир не имеет границ. Мы говорили вам о бесконечности. Мы говорили вам о бесконечности мира, как вы можете поставить границы?}
\people{Понятно.}
\soul{Мы говорили вам, что в нашем понятии есть и наши ``тела''. Давайте говорить так – другая физика. }
\people{Ну, давайте.}
\soul{Вам будет легче это понять. У нас тоже есть ``тела'', и мы тоже ищем.}
\people{Ну, ладно. Мы ищем, кто мы тут, зачем, что мы тут творим, не можем понять. Как слепые котята. А что вы ищете?}
\soul{Мы говорили вам о подобии. Мы - идём далее.}
\people{Вы уже нашли?}
\soul{Вы, возьмите аналогию  себя. Вы мечтаете то-то и то-то. Вы ставите себе цель и когда достигаете её, что делаете? Ставите выше.}
\people{Вы, так же?}
\soul{И далее, далее, далее. Разница только в том, что мы видим цель. Мы - ставим цели те, которые видим.}
\people{А-а-а…}
\soul{Вы же, ставите цели, не зная какие. Если быть точнее – вы не ставите их никаких. Вы просто говорите:  Пойду наугад. И всё. И идёте. Хорошо, если вы наткнётесь на что-то хорошее. Хорошо. Но, может быть и обратное. Вы и не ставите цели. Назовите мне цель? Духовную цель. Нет.  Вы просто идёте наугад – ``А что, если так? А что, если так?'' Что есть тогда – ваша физика, если вы не ощущаете и не видите её? Мы говорили вам, что вы обладаете энергией, но не умеете управлять ей. Разница в том, что мы - умеем. Умеем управлять большей частью, гораздо большей, чем вы. И потому мы можем подняться намного выше, в вашем понятии, но и также можем упасть намного ниже.}
\people{А были у вас такие индивидуумы, которые падали?}
\soul{Они были даже у вас. Они приходили к вам.}
\people{Это всё взаимосвязано?  То есть, кто-то падает у вас – кто-то падает у нас, одновременно?}
\soul{Нет. В вашем понятии существуют космические весы. Уравновешивают добро и зло. Нет таких весов. Представление о весах уже опровергает единство.}
\people{Что-то я не пойму, как у вас может падать кто-то, если вы сами сказали, нет добра и зла. Это нам пока не понять.}
\soul{Почему же? Пожалуйста, приведите сами. У вас есть множество аналогий, они есть у вас. Подумайте.}
\people{Т.е. делая кому-то добро, кому-то мы зло одновременно? Так что ли?}
\soul{Хорошо, давайте сделаем так: ``подняться высоко'' - в вашем понятии - Христос, Будда и далее. ``Упасть низко'', в вашем понятии, - ваши вожди, и тот же Гитлер. Пожалуйста, вот вам пример. И там и там - человек. Вы видите разницу? Вы видите, где здесь ``падение'', и где здесь ``повышение''? Вы видите здесь?}
\people{Один, Гитлер, так сказать, у власти проявлял насилие, а второй, просто… Как сказать? Христос…}
\soul{Тогда подумайте. Подумайте и ответьте: есть бог и антибог, есть Христос и антихрист. Подумайте. Вот вам и ``подъём'' и ``падение''. Вот вам ``вершина'' и ``дно''. Есть бог и антибог. Подумайте. И подумайте, ваша религия уже допускает слабость Бога, неверие ему. Раз есть Антихрист, раз есть антибог, это уже - неверие в Бога, неверие в его могущество и далее. Подумайте. Вы можете доказать это логически. Не вам это делать. И тогда вы обвиняете кого-то, вместо того, чтоб обвинить себя. Вот вам падение и возвышение. А мы же говорим вам, что мы подобны вам. Подобны. Мы не имеем только физики, не более. И мы говорим вам: мы не выше и не ниже вас, ибо мы меряем своими мерами, своими. Поймите. Многие, многие из вас, среди вас считаются ничтожными. Они намного  выше, чем вы думаете. Намного выше многих из вас. Хотя, для вас они - ничтожества. Есть многие, которые выше вас, в вашем понятии, но они, всё-таки, намного ниже. Аналогия? Пожалуйста! Сталин. Когда-то, он был возвышен, и как опустили сейчас. Ленин … продолжать вам? И это ваши гении. Вы их называли гениями. Теперь, как вы их называете? Если вы их и называете гениями, так только ``гениями зла''. Хотя, раньше, это был ``отец народа''. Теперь – ``гений зла''. Смотрите, какие противоположности! Вы маятник, маятник. Сегодня говорите одно,  завтра - совершенно другое. Сегодня…}
 1-2… 
(Конец контакта)
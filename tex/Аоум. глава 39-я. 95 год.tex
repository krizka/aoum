Аоум. глава 39-я. 95 год
Георгий Губин
Белимов Корреспонденту: Вы не волнуйтесь, мы участвуем уже достаточно долго. Правда, без Георгия может будут трудности.
Корреспондент: Я представлюсь, а потом? И как мне представляться? 
Ольга: Они просят не называть имен.
Корреспондент: Я право не знаю, без имен, в неизвестности…
Переводчик: Спрашивайте.
Корреспондент Белимову: Он что все время нас слышал?
Переводчик: Спрашивайте.
Белимов: Хотелось бы продолжить наши диалоги с участием м…
Корреспондент: Извините, мне надо назваться?
Переводчик: Спрашивайте.
Белимов: Я понял, вы не против посторонних. Скажется ли это на переводчике? И, если скажется, то как?
Переводчик: Что волнует вас? Разве это произошло в первый раз?
Белимов: Ну, незнакомые люди…
Переводчик: Скажите честно, что вас беспокоит?
Белимов: Здоровье переводчика, не повлияет ли это на продуктивность контакта.
Переводчик: Прошлый раз вас беспокоила только продуктивность. Вы растете.
Белимов: Хорошо. Вам известно что-нибудь о регрессии? Действительно повторное переживание отрицательных эмоций вылечивает людей? Какова технология гипнотического воздействия? Возможен…
Переводчик: Что вы хотите услышать от нас? Подробную инструкцию?
Корреспондент: Вы подтверждаете, что регрессия не очередной псевдонаучный вымысел?
Переводчик: Разве? То, чем сейчас вы заняты подтверждено наукой, или является хотя бы гипотезой?
Белимов: Вы о контактах, или о регрессии?
Переводчик: Мы говорили вам, что любое движение Души приветствуется. И если путешествие в прошлую жизнь принесло вам пользу – дерзайте. Чаще, первые шаги осуждаются не только наукой, но и ближними, друзьями.
Белимов: Ваш ответ положителен, тогда есть смысл развивать новое направление.
Переводчик: Если бы мы сказали – нет, вы бы перестали этим заниматься?
Белимов: Скорее всего, да.
Переводчик: Мы просим идти своей дорогой, по нашей тропе нет смысла идти и вам и нам. Мы ответили вам?
Корреспондент: Простите, но я совершено вас не поняла, на какой вопрос вы сейчас ответили?
Переводчик: Что в вашем понятии регрессия? Путешествие в прошлое? Или в вами прожитое? Подумайте, если возможно отдельной личности вернуться в ``свое'' прошлое, значит и народ, страна, Земля могут повторить ``пройденное''.
Белимов: А это возможно? Если Россия проведет регрессию, то сможет избавиться от ``болезней''?
Корреспондент: Если такое случиться, разве не нарушится политический баланс?
Белимов: Почему?
Корреспондент: Россия получит излечение, другие страны автоматически станут аутсайдерами.
Переводчик: Увы, такое невозможно.
Белимов: Высшие Силы оберегают человечество от дисбаланса? Россию специально удерживают на среднем уровне…
Переводчик: Вы никогда ни в чем не виноваты – злые силы, теперь и Высшие не дают вам развиваться? Вам можно позавидовать – вы всегда невиновны.
Корреспондент: Мы судим по вашим ответам.
Переводчик: Вы говорите за нас, решаете за нас. Мы постоянно говорили вам о вашей свободе, но вы не слышите нас. Второе: вас здесь всего двое, и вы разошлись во мнениях. Вы считаете, что жители всей страны найдут общий язык? И причем тогда здесь Высшие Силы?
Корреспондент: Но высшие силы управляют нами. Объясните тогда о Вавилоне.
Переводчик: Разве желание излечиться привело к смешиванию языков?
Корреспондент: Люди хотели стать подобно Богу.
Переводчик: Нет! Вы всегда были подобны Богу, вспомните Библию. Никак и никто не может карать вас за правду. Человек возжелал стать Богом. Ничего не зная, ведомый одной гордыней. Самое тяжелое, самое ответственное – быть Богом. Вы думаете, Бог злодей, что не пустил людей на небеса?
Корреспондент: Мне кажется, что Бог или давно покинул нас, или… Заратустра говорил о смерти Бога.
Белимов: Действительно, простому человеку трудно верить, когда на него валятся все беды.
Переводчик: Когда у человека беда – он или призывает бога в помощь, или проклинает его. И то и другое скорее говорит о вере.
Корреспондент: Меня Геннадий Степанович немного просветил об использовании речевого аппарата переводчика. Скажите, вы используете и его словарный запас?
Переводчик: Мы всего лишь показываем ему наши миры. Он подбирает слова и старается передать свои чувства через речь.
Корреспондент: Я не хочу никого обидеть, но косноязычность вызывает некоторое недоверие к вашим ответам.
Переводчик: Что вы предлагаете?
Корреспондент: Вы умеете обижаться?
Ольга: Переводчик обиделся.
Переводчик: Мы не имеем ваших чувств. Хотя некоторые похожи на ваши. Ваши чувства ограничены, в них отсутствуют малые оттенки и имеют излишнюю прямолинейность.
Белимов: Примитивны!
Переводчик: Сейчас вы обиделись.
Белимов: Мы должны прислушиваться. И если свыше сказали ``дурак'', значит…
Переводчик: Ничего это не значит. Только одно – вам трудно признать себя ``дураком'' и с облегчением согласитесь, если вас обзовут другие. Тем вы успокаиваете свою совесть. А гордыня принимает за оскорбление, если вас обзовет более глупый.
Белимов: Если я буду прислушиваться ко всем и допускать оскорбления, меня примут или за труса или за сумасшедшего.
Переводчик: Спрашивайте.
Корреспондент: Сейчас много пишут о не стандартном мышлении. Как вы думаете, правильно ли что шизофрению не следует лечить, а надо пробовать найти общий язык? Да, о языке: вы можете ответить сами, без переводчика?
Переводчик: Давайте попробуем ваш лексикон.
Корреспондент: Не поняла. Я буду сама отвечать?
Переводчик: Дайте счет от 22 и до 28. Счет даете только вы. Но, пожалуйста, не забудьте после ощущения тепла в груди, дать обратный счет от 28 до 22.
Корреспондент: Я боюсь.
Белимов: Мы поможем.
Переводчик: Мы повторимся – счет дает только спрашивающий.
Корреспондент: Гена, спрашивай. Я боюсь.
Белимов: Мы были не готовы к такой ситуации, тем более человек присутствует впервые.
Переводчик: Хорошо, посчитайте только от 22 и до 24. Повторяйте счет до ощущения тепла. И не забудьте сразу же дать обратный счет.
Примечание: После уговоров со стороны Белимова, и опасений от Ольги, Корреспондент начала считать от 22 до 24. После пятого повтора она начала ощущать горение в груди, о чем с испугом сообщила Белимову. Белимов посоветовал остановить эксперимент. Корреспондент дала обратный отсчет и после пятого повтора:
Переводчик: Остановитесь, зачем вы отсчитываете более, чем насчитали. Повторите вопрос одним словом.
Корреспондент: Сумасшествие.
Переводчик: Всегда существуют минимум две реальности. Одна, ваша, основана на ваших чувствах, характере ваших верований, допусках и запретах, субъективных и объективных причин, на личном знании и незнании. Вторая реальность независима от ваших понятий и мыслеформ, создана природой и физикой, мощью всех выше перечисленных причин, без ваших усилий, но трудом многих других. Как видите они (реальности) практически имеют один закон (законы) и, всё же, совсем различны. Вы же интуитивно стараетесь совместить их в одну реаль. Но, основываясь как и на своем опыте и знании, так и на ошибках и заблуждениях, чаще просто трансформируется первая ваша реальность. И чем успешнее вы это сделаете, тем ближе (в вашем же понятии) будете к истине. Ну а если, что бывает очень редко, первая и вторая совпадут, то будем говорить об истинном видении Мира и Его содержании. И, что удивительно для вас, творящих эту реаль, обе могут оказаться несостоятельными. Кстати, заметьте пожалуйста, мы не говорим о лжи, ибо это определение здесь неуместно. Подумайте сами: мозг благодаря своим чувствам, памяти и, наконец, привычным и мыслями, создал свою реаль. Для постороннего скорее реаль даже и не близка, но только для постороннего. Ваш же мозг, пусть и в заблуждении, относительно себя всегда находится в истинном положении. И работает в истинном (для себя) Мире. Потому и дурак никогда не признает себя дураком и глупец уподобится гению. Короче? – истина истине рознь. Иное отношение мозга приводит к дискомфорту и разрушении гармонии, отражающее сумасшествием в других реалиях. И если вы, услышав и прочитав эти строки, объявите Нас лжецами или пророками, то не сможем, да и не имеем таково права, обидеться на вас. Не можем и порадоваться за старую, но уже отрихтованую относительно нас, вашу реальность. Зато с радостью встретим вашу мысль, даже сомнение и догадливость о взаимосвязи этих реалий и признании (отвержении) вашего полноценного участия в их создании. В принципе, не так и важно, верите ли вы во что-то или категорически отрицаете. И то и другое проявило действие и изменило равновесие. И чем активнее вы отрицаете, тем больше вы подтверждаете отрицаемое в его сосуществовании в вашей реальности. Но не проводите аналогии с верованием во что либо. Ибо отрицая, вы уже уверовали. И наоборот, веруя, вы, естественно, что-то отрицаете. Все противоречия ваших осознаний создает вашу и только вашу личную реаль. Мы же вынуждены повториться – ни слова о лжи. А как часто вы набиваете себе шишки, сталкиваясь с другими реалями. Как часто вы жалуетесь, что вас не понимают, как изменился мир, соответственно его восприятие на вас. Тогда вы говорите; как я ошибался, заблуждался, я никому не нужен…. Обвиняется весь мир (ослепший, жестокий, несправедливый). Вам трудно, очень трудно менять себя, свои привычки, понятии и догмы. Личная реаль острыми углами ранит и вас и других, чужих. В ней появляются, оппоненты, глупцы, враги. Друзья становятся предателями. Весь Мир верх ногами. Мы ответили вам?
Корреспондент: Не знаю, трудно принимать такой объем, когда думаешь только о боли в груди.
Белимов: Мы что-то не так сделали? Мы забыли об инструкции?
Переводчик: Умойтесь проточной водой и все пройдет.
Белимов: Вы говорили, что не можете воздействовать на нас физически. Что тогда произошло сейчас. Вы внушили боль?
Переводчик: Первое: не путайте боль и тепло. Второе: перечитайте вами написанное.
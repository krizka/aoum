Аоум. глава 30-я 02-01-1996г
Георгий Губин
 Разглядывать под микроскопом
 Я стал однажды капельку воды.
 Напрасны не были труды:
 Я множество живущих скопом
 Существ миниатюрных увидал.
 Какое зрелище чудесное для взора!
 Я начал наблюдать и скоро
 Законы их, обычаи узнал,
 И даже обнаружил у бактерий
 Немало суеверий.
 Ту каплю, где живут они,
 Считают эти крошки центром мира,
 Подобного себе придумали кумира,
 Решили: капля их — важнейшее звено,
 Погибнет мир с ней заодно…
 Смешно? Но в сущности мы столь же эфемерны,
 Масштабы же Вселенной непомерны,
 И, право, не могу сказать я, чтобы
 Мы, люди, значили в ней больше, чем микробы.
                       Пьера Лашамбоди
\people{**}
 
02-01-1996
\people{2-е января 96 года. Вы нас слышите?}
\soul{…Этим уже не вернёшь.}
\people{Вас можно спрашивать?}
\soul{У вас есть этот камушек. Поймайте его и не опустите.}
\people{Какой камушек?}
\soul{У кого-то кристалл, а у кого, просто, булыжник. Маленькие, большие, разные. У кого-то показано на нём, у кого, просто, отдаёт в нём. У каждого есть камушек. У кого-то кристалл, и грани его чистые,  а у кого-то с гнильцой, меж  капелькой воды и капелькой слёз. Слушайте в ней лишь.  Если б вы могли слушать… Уже не мечтаем, чтоб говорили. Слушайте, хотя бы, близко сидящего, но вы боитесь.}
\people{Боимся?}
\soul{Мечтаете о новых дорогах, а боитесь их. Ходите по проторенному  и повторяете. Говорите - новая жизнь… Да все они у вас одинаковы. Даже гибнете одинаково. Память? Но, при чём здесь память? Просто не желаете и всё. (Вздыхает).}
\people{Гена, это ты говоришь?}
\soul{В каждой жизни  только одно имя,  которое можете произнести вслух,  остальных имён вы не знаете, остальные имена для вас - пустой звук. Даже любимого человека, называя по имени, вы не ощущаете его. }
\people{А кто вы?}
\soul{Вы не умеете пользоваться именами. Наверное, и  хорошо. Потому, что все ваши пожелания не всегда добры. Хорошо, что они не всегда доходят до адресата.}
\people{А кто вы? С кем мы разговариваем?}
\soul{Да мы и сами-то  не знаем имени своего. Когда-то вы придумали нам имя, потом быстро забыли, а потом стали называть его в рассказах. Назвали это ``фантастикой'', назвали это ``преданием''. Как только вы нас не называли;  и ведьмами, и эльфами, и гномами.  А мы сами не знаем своё имя. Называете нас ``ведьмами'',  значит, будем вести себя, как ведьмы. Называйте ``гномиками'' –  будем маленькие.}
\people{Кстати, вы извините, но слово ``ведьма'' в древности  обозначало ``красивая женщина'' что ли… Не так?}
\people{“Ведать''.}
\people{Нет.}
\soul{Да нет, вы не правы. }
\people{Ну, тогда не прав тот словарь.}
\soul{А вы пользуетесь тем словарём? Называя сейчас, вы что имеете в виду? Похвалу?}
\people{Нет, я имел в виду изначальный смысл этого слова.}
\soul{Значение и смысл этого слова у каждого свои. Если вы называете ругательским, то, значит, и будет руганью. Любое слово, как произнесено будет вами, а произносите вы их по-разному – это зависит от вашего назначения, настроения, а значит, и  меняется смысл этого слова.}
\people{Ясно.}
\people{Скажите,  а мы,  в прошлый раз, были…ну…  с вашим…  Ой.. (идёт вздох. прим.)}
\soul{Да  не были мы с вами в ``прошлый раз''.}
\people{Это другие? Я имею в виду…}
\people{Совершенно новые, что ли?}
\people{…существа вашего… Вашей цивилизации, что ли… Или, как сказать?}
\soul{Давайте скажем,  что мы просто с вами заново разговариваем. }
\people{Заново… }
\soul{Да-а. Мы когда-то разговаривали с вами, но, как вы говорите `` не в этой жизни''.}
\people{Не в этой жизни?}
\people{Значит, в той жизни, тоже был клуб, тоже контакты? Тоже, там что-то… }
\soul{Нет. Это было о-очень давно.}
\people{А-а! Это наверно, когда люди могли… могли… общаться ,что ли, с существами другого…}
\soul{Это ваш камушек. Вот он, ваш камушек.}
 
\people{Чей?}
\soul{Ваш.}
\people{Мой?}
\soul{В нём  много светлого,  но тёмного, почему-то больше. Посмотрите, что окружает его – красное, жёлтое.  В прошлом, когда мы разговаривали с вами, он был светлее. Светлее. Вы потеряли его. Вы отказались от того камня. Он был поднесён вам в подарок. Первое,  что вы сделали, - вы  попытались его разглядеть всевозможными путями.  Тогда вы его стали изучать, и он потерял смысл, он стал просто ``камушком''. Он стал просто собирать вас, чтобы повторится в этом камушке вами. А  вы забыли про него… но, он-то - помнит. Он никогда вас не покидал. Из жизни в жизнь он приходит к вам. И в этой жизни приходит и дарит вам камушек. А вы -  ``Какая безделушка!''. Даже не смотрите в его сторону, потому что, при этом, дарящего  вы не уважаете.  Вы станете бояться.}
\people{Вас?}
\soul{Вы просто разговариваете и всё. Стараетесь говорить то, что первое… А всё, что размышляете – уже не уверены. }
\people{А почему ж тогда размышляем, интересно?}
\soul{Вы размышляете-то - только кажется. Дана вам нить, а вы её – в клубок. Вы же не можете управлять мыслями и путаетесь, путаетесь. Пришла, как вы говорите, `` идея'', а вы тут же на неё –'' А если так, а если так? А может быть и этак?''   И что? Сама идея потерялась, само зёрнышко,  укутана уже грязью  ваших мыслей, ваших забот.}
\people{Ну, может мы, действительно, что-то не понимаем в своей жизни, но как-то стараемся…}
\people{Может, мы не теми методами стараемся понять это?}
\people{Да.}
\soul{При чём здесь методы? Вы просто плывёте по течению и всё. Когда вы стараетесь оторваться от общего движения, вы, тут же, начинаете бояться, чтобы не стать ``белой вороной''. Кода-то вы лечили травами, не вылечили  одного, испугались в себе, сказали - ``Наверное, не верно'', а когда умер другой – вы перестали заниматься этим.}
\people{(Гера) Это обо мне же? }
\soul{Ну, зачем же?  Разве вы сейчас спрашиваете?}
(Гера)* А-а. Извиняюсь. 
 (Ольга) *Я, наверное, и сейчас точно так же боюсь. Но ведь… смерть человека,… это довольно неприятное, будем говорить так, мягко говоря, ``явление''.
\soul{Оно неприятно - для вас. Вы эгоистичны.  И смерть любого понимаете, как потерю чего-то своего. Да, конечно, смерть печальна, и радоваться нельзя. Но вы переживаете саму потерю, а не смерть. Что вы больше никогда не увидите и, и…}
Счёт 1-..
\people{А-а,  ну, если наши врачи будут экспериментировать вот так, не боясь, так сказать, на всех? Вот, как у нас недавно был такой случай…}
\soul{Ну, зачем вы берёте крайности? Вся жизнь ваша – ``эксперименты''. Как бы вы могли найти новое, если не экспериментировали? Но, только когда вы это делаете без души – вот, что страшно. Даже если будете делать всё верно, но без души, - вряд ли вы  вылечите. Первое, что надо сделать, научить человека жить, дать ему желание жить. Это самое первое. Если он не хочет жить, если он не верит в то, что он будет жить, какие могут быть лечения? А вы скажете: есть множество людей, которые не хотели  умирать, но умирали. А вы всмотритесь  в них –  они со страхом ждали смерти, со страхом говорили, что не хочу умереть, со страхом. Значит, всё-таки, они были настроены на смерть, вы согласны?}
\people{Угу.}
\people{Ну, вы нам  может быть, что-то посоветуете? Каким образом тогда… Как нам избавиться от рассудка, вот от этой рассудочной деятельности излишней?}
\soul{Мы разговаривали. Говорили нам, предупреждали нас, что вы будете повторяться, будете спрашивать те же вопросы, и будете перепроверять тех или других. Говорили нам, чтобы не обижались на вас, и говорили, что бы мы с вами говорили, как с детьми. Но мы не согласны с ними, не согласны,. Они многое прощают вам. А мы бы не хотели.  Мы хотели бы быть пожёстче с вами. Вы же не прощаете нас?}
\people{Мы ж не знаем, с кем мы говорим, в принципе,  и где мы вас не прощаем?}
\soul{Нет, всегда, всегда вы нас не прощали. Всегда нас не уважали, всегда нас били. Когда вы называли нас ``эльфами'', то обязательно почему-то злыми. Когда вы называли нас ``ведьмами'' –  то обязательно - зло и каннибализм. Смешно. Это же ваше, а вы приписывали нам. }
\people{А почему? Эльфы в сказках всегда очень добрые, вообще-то.}
\soul{Да? Добрые… }
\people{Ну, в общем,  мы, конечно, - люди, -  мы уже так про себя поняли. Мы, люди - существа странные.  Но, тогда почему  всюду говорится: - Вы - люди, вы  - человеки, у вас будущее необычное какое-то, не как у других существ на этой Земле? Ну, мы как бы ``первые'' что ль.}
\soul{А как вас встряхнуть? Ну, вы хотите быть первыми, ну, мы и будем говорить, что вы первые.  А как вас ещё встряхнуть? Если вам сказать, что вы ещё далеки, то многие из вас плюнут на всё  и останутся, где были.  А когда видна цель – оно вроде как легче идти.}
\people{Скажите,  а как вы с нами общаетесь, как вы переводите голос переводчика?}
\soul{Говорит он, а мы картинки.}
\people{Картинками тоже. }
\soul{Только один есть способ разговаривать с ним. Нам сказали, и мы следуем их советам.  Зачем, зачем нарушать привычное? Вы…вы….}
(Сбой контакта)
\people{Скажите, а вы сказали, что вы эльфы, ну, там… мы вас называем,  там ведьмы… там… Может вы та сила, которая  даёт  людям вот это, так сказать, искусство перевоплощения, полёта … Я не знаю, что…ведания там всякого, провидения, предвидения? }
\soul{Да нет. Это всё ваше. А мы,- просто одна из ветвей. Только и всего. Был один зародыш. Один.  Одно яйцо. Разбилось оно на множество ветвей. Древо жизни – оно имеет множество, множество ветвей, на каждой ветви есть множество листиков.}
\people{А вы можете воплотиться человеком? Ну, родиться человеком?}
\soul{Да.}
\people{Значит, вот эти девочки, которые утверждают, что они эльфы, вполне возможно, они могут быть ими?}
\soul{Ну, нет, конечно. Чаще , всего – это  игра. Мы не можем умереть, как вы.  Просто взять и умереть. Чтобы мы могли умереть – вы должны нас забыть. И не вспомнить. И, тогда, мы умрём. Мы живём только тогда, когда вы помните о нас. Вы о нас не забываете, поэтому у нас смерть -  это очень редко.  А многие просто приходят и играют. Играют с девочками, как говорите вы, и девочкам это нравится. Они воспринимают эту игру. А потом, когда мы уходим, то нравится продолжать. Им не хочется терять то, что уже получили  и продолжают играть.}
\people{А вот, вы – бессмертны, или всё зависит от человека? Вы только что сказали…}
\soul{Да, мы смертны, но это очень редко, потому, что вы помните нас. Помните. Иногда, вы называете нас ``чертями''. А вы знаете, что такое слово ``чёрт''? Что это множество, множество веществ, и всё  зависит от  того, как вы назвали, с каким настроением, если хотите. Назовете со злостью – будет вам чёрт, назвали с радостью – будет маленький бесёнок, который будет помогать вам просто  жить и веселиться. Всё зависит от того, как вы произнесли. Сказали ``Бог'', довольно равнодушно, вскользь – этот Бог вас уже не заметит, пройдет равнодушно. А если будете в проклятиях его вспоминать – Он вам проклятиями и придёт.}
\people{То есть, есть некто, так понял -  субстанция… Эдем, что ли…или Солярис…Ну, то, что  откликается…}
\soul{Давайте скажем так, - фэнтези. Мир фэнтези, мир фантастики. Вот в этом мире живём мы. Мир, где исполняются все ваши желания. Желания  ваши разные, конечно, поэтому у нас есть и чёрные и белые, есть страны страшные, есть и…(теряется)}
\people{…эльфов там, чертей? Простите, конечно.}
\soul{Нет, мы уже другие.}
\people{Другие – это кто?}
\soul{А вы с нами уже разговаривали. }
\people{В прошлый раз?}
\soul{Говорите.}
\people{В прошлый раз мы с вами разговаривали?}
\soul{Ну, да.}
\people{А в этот раз?  Кто были?}
\people{Вы с ними, вообще, как-то контактируете в вашем мире?}
\people{Вы - одни…Одно?}
\soul{О нет! Нас теперь множество. Первые, что говорили с вами, собрали нас всех поболтать с вами. А зачем это им нужно – мы не знаем. Но они хотят, чтобы мы с вами все поговорили. А нас много.}
\people{Много кого? Мы уже сейчас теряемся.}
\soul{У-у! У нас тут даже есть переводчик с прошлой жизни.}
\people{Так он там?}
\soul{Так что вы можете прекрасно заняться спиритизмом.}
\people{Зачем?}
\people{Кто вы, вообще? Вот те назвались. Они  там ``’эльфы”… Которые мы называем ``эльфы'', ``гномы”…}
\soul{А как мы в прошлый раз назывались, вы уже забыли?}
\people{Предохранители.}
\people{Утешители. }
\people{Предохранители.}
\people{Ну, не важно. Короче, - те.}
\people{Мы разговаривали…  Вы можете их назвать? Вы их знаете? Вы с ними в контакте?}
\soul{Да. Они живут в мире желаний.}
\people{В мире желаний.}
\soul{Этот мир - чисто ваш. Он полностью выдуманный вами. Довольно-то весёлый мир, очень разнообразный. У вас столько множество желаний,  что в этом мире целый бардак. Мы сами можем придти туда и заблудиться. Постоянно меняются страны, постоянно меняются короли.}
\people{И все, кто угодно, да? Кто кем хочет стать.}
\soul{О, это очень зыбкий мир!}
\people{А, ну, теперь поняли.  Скажите, как они вышли на переводчика?}
\soul{Те, что разговаривали с вами, пригласили всех.}
\people{А… те, которые, первые наши друзья?}
\soul{Да.}
\people{Приятно со всеми поговорить.}
\people{Переводчику не может это повредить каким-то образом?}
\people{Ему всё равно, с кем мы там контактируем?}
\soul{Ну, как вам сказать, всё равно или нет?  Главное, чтобы не было страха. Испугается – придут и другие. И хотя мы стараемся не пугать вас, но всё может быть. Но те, как говорите ``высшие'' - они следят. Следят за нами, и не дадут мне сказать ничего лишнего.}
\people{А где же ваша свобода?}
\soul{Свобода? А как вы понимаете ``свободу''? Болтать, что хочешь и когда хочешь? И не обязательно -  нужно это  или нет? }
\people{Скажите, а,  ну, вот этот мир наших фантазий: эльфы, гномы, черти и так далее …Примерно, что …}
\soul{Давайте скажем так:  этот мир не создан вами, вы просто изменяете его. Есть какая-то основа, первый корень, а вы потом насаживаете деревца. А что уж посадили – вам-то уж и расхлёбывать. }
\people{Скажите, мы после физической смерти уйдём туда?  Пройдем все те миры, смотреть, что натворили? }
\soul{Туда есть множество путей. Каждому из вас –  разные дороги. Те, кто не верят – попадут в никуда. Тоже мир. Страшный. И есть миры, раи и ады. Если вы верите в ад - и попадете туда, верите в рай – попадете в рай. Вот только с некоторыми поправками.  Не думайте, что если вы будете думать о прекрасном, но будете творить гадкое – это не значит, что вы попадёте в мир красоты. Вы попадете в мир гадости, и он будет красив.}
\people{Скажите, вот, в прошлый раз с нами разговаривали другие. Вы говорили, что одни живут пол- часа, другие, у вас, кому повезёт,- год.}
\people{Это вы, да?}
\soul{Мы.}
\people{Это вы же были?}
\soul{Мы.}
\people{Скажите,  а почему вот, в прошлый раз…ну, рука переводчика …у него другой немножко ритм был? Сейчас ритм и амплитуда побольше.}
\soul{Ну, мы же говорили вам, что нам сложно,  мы слабее. Но, мы же - учимся. }
\people{Ну, значит, выходит, что те,  с которыми мы раньше разговаривали… не управляют, а как бы… в общем-то, как родители следят и за вами, и за нами, будем так говорить. Да?}
\soul{Да нет.}
\people{Нет?}
\soul{Здесь все немножко сложнее. }
\people{Ну, это может быть примитивно, конечно, сказано. Ну, в общем-то, они следят, чтобы…}
\soul{Нет. }
\people{Нет?}
\soul{Когда-то они пришли к нам. Мы, даже сперва не поняли их, что хотели они. У нас тоже были, как вы говорите, ``контакты''. Ближе, ближе. Потом, наконец, мы смогли их увидеть. Мы считали, что это они, а потом они нас удивили и убедили, что мы увидели лишь только своё,  а их увидеть нельзя. }
\people{Своё понимание о них, да?}
\soul{Они совершенно из ``другого теста'', поэтому совершенно невозможно увидеть, невозможно. Мы не можем их увидеть. Вы не можете их увидеть. Вся материя.}
\people{Скажите, а у вас… Вот мы, люди, развиваемся…У нас есть какая-то там цивилизация…}
Счет -1-2-3
(Далее – кусочек внутреннего монолога из прошлой жизни переводчика. 12хх-й год)
\soul{Но, я  же не виновата. Меня продали. Кого я теперь должна называть матерью, ту, что приняла меня или ту, что родила?  Ту, что научила меня жить …или ту, что продала?}
(10 секунд тишины)
(Обрыв).
\soul{Спрашивайте.}
\people{Мы с кем разговариваем?}
\soul{Вы уже заблудились?}
\people{Да нет. Прошлый раз Вы говорили: ``Говорите ``, а сейчас Вы говорите: ``Спрашивайте'', как с теми, с которыми мы раньше разговаривали. }
\soul{Вот и спрашивайте.}
\people{Так, всё-таки, вы нам объясните, если можно конечно, объясните, вот, как вы устроены? Есть ли у вас технические средства? Как вы вообще живёте, можно от вас узнать?}
\people{Чем  питаетесь?}
\soul{А вы будьте внимательней, мы говорим: ``Спрашивайте''.  Вы уже не узнаёте нас?}
\people{Ой,  мы очень рады. Я, во-всяком случае, рада.}
\soul{Да?}
\people{Почему?}
\soul{А что вы боитесь?}
\people{Мы боимся? Мы не боимся. Нам… Не знаю, может и боязно…Как-то, всё-таки…}
\soul{Привычка.}
\people{Да, наверно.}
\soul{Вы хотите разговаривать с нами, а с другими не хотите. И начинаете всё сначала. И вы боитесь, потому что не знаете, что спрашивать. }
\people{Да вы у нас уже, как родители. Мы, так…за вашу ``юбку'' прячемся. Нас пугают. Мы считаем, что это может привести к не очень приятным последствиям у переводчика или ещё что-то… Поэтому, мы больше за переводчика…(волнуемся./ прим./).}
\soul{Зерно.}
\people{Что?}
\soul{Если сидит в переводчике ``зерно'', то  оно вырастет. Сидит ``зерно добра'' – какое может злое прийти? Если и придёт, то ненадолго удержится в нём.}
\people{Скажите, а какие вопросы вы задаете, разговаривая с другими? Вот, о них спрашиваете или что, о себе разговариваете?}
\soul{Вы хотите,-  какой  мы ведём разговор с другими?}
\people{Да.}
\soul{Прежде, чем они пришли к вам?}
\people{Да.}
\people{Нет. Допустим с теми, которыми никто ещё из ``наших'' ещё  не разговаривал. Ну, с вашими ``высшими''. Вы ж не просто их спрашиваете там… Вы же, у них сами что-то спрашивать должны? Мне, просто, любопытно.}
\soul{Да, конечно, среди нас, тоже есть,  как вы говорите, ``контактёры''. Мы тоже спрашиваем, мы тоже имеем тела, вы должны это помнить – только совершенно всё другое. Совершенно в другом мире. Физика.}
\people{А законы?}
\soul{Законы? Законы выдуманные вами. У нас тоже есть законы, которые  выдумали мы. В этом мы одинаковы. Подобны. Поэтому разговариваем-то с вами. А с совершенно иными, совершенно, не мы, ни вы не можем  говорить. Когда-то, вы пытались нас называть инопланетянами. Это было бы обидно для нас. Мы - совершенно другой мир. А в вашем понятии ``инопланетяне'' это обязательно что-то, но просто с другой планеты. }
\people{Понятно.}
\soul{Мы, в вашем понятии, не имеем планет, не имеем  Вселенной, хотя, в наших понятиях, всё это тоже есть. Но только совершенно иное.  Вы говорите о мире огня, ``может вы оттуда''. О, нет! Мир огня – это тоже материя, это только одна из ступеней. Вы должны были бы помнить, что мы говорили, что проходили ваши ступени, и вы идёте за нами. Вы помните?}
\people{Да.}
\soul{Но мы не хотим, чтобы вы шли нашей дорогой, и вы это должны помнить тоже. У вас своя дорога, у нас своя. Цель? Одна ли цель? Да мы и сами-то не знаем её, мы сами-то не видим.}
\people{Скажите, а вот мы перед вами разговаривали, которые называют себя  ``предохранителями”…  Вы с ними имеете контакт, как мы  поняли, и давно, наверно. Они существа, как они сказали, такие же, как и мы. Такой же физики. А мы их не видим, не слышим, не замечаем.}
\people{Это более разреженная физика что ли?}
\soul{Ну, они сказали – ``фазы''.  Пусть будет так. (имелся ввиду сдвиг фаз между мирами. Это приближенный аналог./прим/)}
\people{Скажите, а они развиваются по другому пути, если можно так сказать? Или как?}
\soul{Они очень печальны. Вы - можете уйти с Земли, вы свободны в выборе. Они не могут, они останутся на Земле всегда.}
\people{А когда Земля исчезнет?}
\soul{Они должны убирать за вами всё. Они называли себя ``предохранителями''. Но они могли бы назвать себя и ``уборщиками'', и ``ассенизаторами'' и кем угодно! Они были бы правы. Потому что им приходится, приходится впитывать в себя всё то, что творите вы. Для чего? Да  чтобы было меньше сору. А вы стараетесь, стараетесь, даёте множество им пищи. Вы их убиваете, не задумываясь даже. Полчаса и час - так они сказали? (среднее время их жизни./прим./)}
\soul{Да. }
\soul{А есть и мгновение. Очень редко, кто может прожить более, в вашем исчислении, десяти лет. Это очень  большая редкость. }
\people{Скажите, а они как рождаются, умирают? Дети у них есть?}
\soul{Да. }
\people{И эти дети тоже такую работу выполняют,  как они, или до какого-то определенного возраста? }
\soul{Это не работа. Это их жизнь.}
\people{Нет… Ну, да. Такая жизнь… Дети – тоже, так вот?}
\soul{Они же тоже живут.}
\people{И бывает, что они не становятся взрослыми?}
\soul{Да.}
\people{А семьи у них существуют, если есть дети?}
\soul{А вы подумайте, - плачет ребёнок,- кто будет его успокаивать? - ребёнок. Потому, что им легче понять друг друга. Взрослый всегда ставит себя выше ребёнка и не хочет опускаться до его восприятия. Потому, ребёнка успокаивает ребёнок. Женщин успокаивают женщины.}
1-2-3
(перепад)
 
\people{(Гера)Мы?}
\people{(Гера и Лена) Люди. }
\soul{С кем я буду говорить?}
\people{(Ольга)А кто вы? }
\soul{Сергей.}
\people{(Ольга) Сергей? Был Сергей? }
\people{(Гера) Это тот, я так понял, который переводчик, да?}
\soul{Я не знаю, о чём вы говорите.}
\people{(Ольга) Сергей? Который?}
\soul{Я. Сергей.}
\people{А фамилия?}
\soul{Иванов.}
\people{(Гера) Год рождения?}
\soul{1905-ый.}
\people{(Ольга) А вы где сейчас?}
\soul{А вы  кто?}
\people{(Гера) А мы в 1996-ом.}
\soul{Какой?}
\people{(Гера и Лена) 96-ой.}
\people{(Ольга) Вы где?}
\soul{Как - где?}
\people{(Ольга) Вы где живёте? В каком городе?}
\soul{Москва.}
\people{(Гера)Живёте?}
\soul{Живу. А вы кто?}
\people{(Гера) А мы кто? Мы люди,  вроде, пока ещё. }
\people{(Ольга) Мы живём тоже в городе. В городе Волгограде. Вы знаете такой – Волжский?  Город такой знаете? }
\people{(Гера) Царицын. }
\soul{Царицын? Волг.. какой?}
\people{(Гера) Волгоград – город на Волге. Его переименовали.}
\people{(Ольга) Город Волжский.  Сталинград  знаете?}
\soul{Нет.}
\people{(Ольга) А Волгоград?}
\soul{Нет.}
\people{(Гера)А Царицын?}
\soul{Нет.}
\people{(Гера) И Царицын не знаете? }
\people{(Ольга) Сколько вам лет?}
\soul{Восемнадцать. А кто вы?}
\people{(Ольга) Мы - люди.}
\people{(Гера) Вы нас видите, или как? Или слышите?}
\soul{Слышу.}
\people{(Гера)А что вы сейчас делаете, вообще? Спите, не спите?}
\people{(Лена) Как вы попали сюда?}
\soul{Я не знаю, меня положили.}
\people{(Гера) Куда?}
\soul{В гипноз.}
\people{(Гера) А-а.}
\people{(Ольга) Сергей Иванов? }
\people{(Гера)А отчество? Извиняюсь.}
\soul{А кто вы?}
\people{(Гера) Меня зовут Георгий.}
\soul{Мне сказали, что бы я не очень-то, потому, что могут быть и зле. А кто вы?}
\people{(Ольга) Да мы не знаем, может мы и злые, а может быть и нет.}
\soul{Какой вы говорите год?}
\people{(хором) 996-ой.}
\soul{Это прошлое.}
\people{Прошлое?}
\soul{996-ой.}
\people{(Гера) 1996-ой, не восемьсот, а девятьсот девяносто шестой. Это будущее для вас, а для нас, это настоящее.}
\soul{Мне трудно вас понять.}
\people{(Ольга) Ну… Вы русский?}
\soul{Да.}
\people{(Ольга) У вас есть семья?}
\soul{Да.}
\people{(Ольга) Мать, отец?}
\soul{Да.}
\people{(Ольга) И жена?}
\soul{Я не хочу с вами разговаривать.}
\people{(Гера) Ну, ради бога. Мы не требуем, в принципе. Но если уж вы разговариваете….  Мы  просто хотим узнать…}
\soul{А как вы можете доказать, кто вы?}
\people{(Гера) А как вы можете доказать, кто - вы?}
\people{(Ольга) Вы нас не видите, не ощущаете, только наш голос слышите?}
\soul{Да, слышу.}
\people{(Ольга) А как вы слышите? Чем? Внутри себя? Внутри? Ушами или внутри себя?}
\soul{Не знаю, надо у Андрея спрашивать. Он меня уложил.}
\people{(Ольга) Андрей?}
\people{(Лена) Ну, Андрей уложил в гипноз его.}
\people{(Ольга). А как его фамилия?}
\soul{Нет, вы мне  сперва  ответьте,  кто вы?}
\people{(Гера) Георгий Губин. Георгий Губин, могу паспорт показать, но вы не увидите.}
\soul{Вы из будущего?}
\people{(Гера) Да, мы из будущого. Для вас - из будущего, а для нас - это настоящее. Вы, для нас - в прошлом.}
\soul{В прошлом. Значит, вы знаете историю нашу?}
\people{(Гера) Конечно! }
\people{(Девушка) Но мы не скажем.}
\people{(Гера) Тебе восемнадцать лет, значит, революция уже произошла, да? }
\people{(Ольга) Двадцать третий год – гражданская война закончилась?}
\soul{Двадцать второй.}
\people{(Гера) Ну, да, двадцать второй.  Возможно.}
\people{(Ольга)А… число?}
\soul{Восемнадцатое.}
\people{(Ольга) Восемнадцатое, а месяц?}
\soul{Май.}
\people{(Ольга) Май? Восемнадцатого мая тысяча девятьсот двадцать второй год.}
\people{(Гера) А у нас, позавчера был новый год, тысяча девятьсот девяносто шестой год, с чем вы нас можете и поздравить, в принципе. Ну, просто праздник – Новый Год. }
\people{(Ольга) Скажите, а Ленин жив ещё?}
\soul{Да.}
\people{(Ольга) А он где сейчас? }
\soul{А какой у вас сейчас строй?}
\people{( Ольга) Социалистический, перестройка идёт.}
\people{(Гера) Сейчас, скорей всего, где-то - капитализм начинается. А может быть социалистический капитализм, а может капиталистический социализм. Сейчас непонятно, бардак идёт.}
\soul{Вы меня запутали.}
\people{(Гера Мы сами  тут запутались. }
\people{(Ольга) Понимаете, у нас был строй социалистический, до какого- то времени, до 1985 года, затем у нас произошли некоторые события, которые привели к такой вот революции, бескровной, будем так говорить. Сейчас мы в процессе перестройки, только не знаем, куда мы перестраиваемся, но куда-то мы перестраиваемся.}
\soul{А… советской власти нет?}
\people{(Ольга) Почему? Как таковой, официальной, её нет но, всё осталось, как и было, практически.}
\soul{А Владимир Ильич?}
\people{(Ольга) Владимир Ильич умер в 1924 году.}
\people{(Гера) Его убила, эта… Каплан, да?}
\people{(Ольга) Да. }
\soul{Как она его убила?}
\people{(Ольга) Не убила, она его ранила отравленной пулей, и он умер. }
\people{(Гера) Это у нас все дети знают, в принципе. В школах учили. }
\people{(Лена) Сейчас уже не учат.}
\soul{Что-то вы попутали всё.}
\people{(Ольга) Нет.}
\soul{Это уже было, но он же не умер!}
\people{(Гера) Ах, да! Он два года же это… его же два года держали, он же не сразу же умер.}
\people{(Ольга) Как –  умер?}
\people{(Гера) Не умер он. (Сергею) Каплан уже стреляла, да?}
\soul{Да.}
\people{(Ольга) Он умер в тысяча девятьсот…}
\people{(Гера Ольге) Погоди!  Двадцать второй год восемнадцатое мая, да?}
\people{(Ольга) Да, он умер в двадцать четвёртом году.}
\people{(Гера Сергею) Да, в двадцать четвёртом году он умер, вернее - для вас ещё умрёт, называется. }
\people{(Ольга) Да, ещё не умер. }
\people{(Гера) Для вас - он ещё жив. }
\people{(Ольга)  А вы кто? Вы комсомолец? Вы знаете, что такое комсомол?}
\soul{Да, знаю.}
\people{(Ольга) Вы комсомолец?}
\soul{Нет.}
\people{(Гера)  Верующий?}
\soul{Да у нас уже и нет верующих-то.}
\people{(Ольга) Вы учитесь?}
\soul{Да.}
\people{(Ольга) В институте? В медицинском?}
\soul{Нет, РАБФАК}
\people{(Ольга) РАБФАК?  А вы решили  гипнозом заняться?}
\soul{Андрей.}
\people{(Ольга) Андрей. Как его фамилия?}
\soul{Не знаю.}
\people{(Ольга) А вы его так плохо знаете?}
\soul{Нет, мы с ним просто учимся.}
\people{А-аа.}
\soul{Подождите, не понятно всё-таки… Девяносто шестой…}
\people{(Ольга) Девяносто шестой год. }
\people{(Лена) Тысяча девятьсот девяносто шестой.}
\people{(Гера) Скоро двухтысячный, через четыре года. Следующее тысячелетие будет.}
\soul{А кто победит?}
\people{(Гера)  А в чём? В гражданской? Или в чём? В гражданской?}
\people{(Ольга) В революции?}
\soul{А! У вас она только идёт…}
\people{(Ольга)  Нет, мы уже прошли всё. }
\people{(Гера) А в чём победит-то?}
\soul{И снова –  царь?}
\people{(Все дружно) Нет, у нас нет царя, у нас президент.}
\soul{Президент?}
\people{(Ольга) Раньше он назывался председатель Верховного Совета СССР. }
\people{(Гера) Союз Советских Социалистических Республик. }
\people{(Ольга) А, ещё нет союза, да?}
\people{(Гера) Давайте, по-одному, а то он не поймёт ничего.}
\soul{Президент. Подождите… как - там? (имеет в виду как у западных стран)}
\people{(Ольга) У нас называется президент. В капиталистических странах тоже называется президент.}
\soul{Да.}
\people{(Ольга) Да, и у нас теперь тоже теперь президент.}
\people{(Гера) А семьдесят лет вот…}
\soul{Подождите, это что же это, всё напрасно? (Революция) Прим.)}
\people{(Гера) Нет, напрасно ничего не бывает.}
\people{(Ольга) Это не напрасно.}
\people{(Гера) Просто оно дало, какие-то результаты, и оно вылилось вот в это.}
\people{(Ольга) Понимаете, скоро будет,… была война с немцами, с тысяча девятьсот сорок первого по тысяча девятьсот сорок пятый. Вы знаете?}
\soul{Опять?}
\people{(Ольга) Была война.}
\people{(Гера) Да, поять. В четырнадцатом, в шестнадцатом была. По какой-то там, ну, не важно… и будет с сорок первого по сорок пятый. }
\people{(Ольга) Будет коллективизация в тридцать первом году начинаться, будут колхозы. }
\people{(Гера) Сталин придёт к власти, будут большие репрессии, значит, интеллигенции.}
\people{(Ольга ) Всех революционеров…}
\soul{Сталин… (Вспоминая)}
\people{(Ольга) Джугашвили…}
\soul{Не знаю.}
\people{(Ольга) Сталин, он…  }
\people{(Гера) Иосиф Виссарионыч.}
\soul{Нет, не знаю.}
\people{(Лена) Он их ещё не знает. }
\people{(Гера)не знает, да?}
\people{(Ольга) Он будет, после смерти Ленина вашим генеральным секретарём коммунистической партии.}
\soul{А когда умру я?}
\people{(Все в месте) Этого мы не знаем. Вы для нас, уже, так сказать, уже…}
\people{(Лена) Из прошлого.}
\people{(Гера) Из прошлого. Мы не знаем. Может, вы  живы и до сих пор будете, при нашем времени. }
\people{(Ольга) Может, вы живы ещё? }
\people{(Гера)  В принципе, 905 год… 96-й… Девяносто один год вы, возможно, проживёте, хотя… }
\people{(Ольга) А вы всегда живы, вы знаете об этом?  Человек не умирает.}
\soul{В смысле? }
\people{(Ольга) Человек никогда не умирает. }
\people{(Гера) Человеческое - не умирает никогда! }
\people{(Лена) Человеческая душа. }
\people{(Гера) Тело умрёт, а душа будет жить.}
\soul{Да это вы Попенса начитались.}
\people{(Гера) Да это не Попинс.}
\people{(Ольга) Чего? Мэри Попинс?}
\people{(Общий смех) (Гера) А что, у вас тоже была Мэри Поппинс, в ваше время?}
\people{(Ольга) Так, а почему мы разговариваем из прошлого..?}
\soul{Мне давал  Андрей читать Папенса, но я не очень-то  разделяю его. }
\people{(Ольга) Скажите, вы из интеллигентной семьи? }
\people{(Гера) Кто ваши родители?}
\soul{Нет.}
\people{(Ольга) Нет?}
\soul{У нас этим даже  не хвастаются.}
\people{(Ольга) Нет, ну, причём тут ``хвастаться''? }
\people{(Гера)  Мы - просто так.}
\people{(Ольга) Вы на РАБФАКЕ  учитесь, а раньше, где учились?}
\soul{Я не учился.}
\people{(Ольга) Вообще?}
\soul{Да.}
\people{(Гера) Вы коренной москвич?}
\soul{Нет.}
\people{(Ольга) Вы из деревни?}
\soul{Из деревни.}
\people{(Ольга) Из какой деревни?}
\soul{Ой, я не знаю.}
\people{(Ольга) Как не знаете?}
\soul{Отец привёз меня, когда мне было три года, а потом, отца убили.}
\people{(Ольга) А мать?}
\soul{А мать я никогда и не знал.}
\people{(Гера) А как же ты… с трёх лет… ?}
\soul{Так меня воспитывали просто.}
\people{(Гера) Кто?}
\soul{Детдом.}
\people{(Гера) А раньше, он назывался не детдом, а как правильно сказать?}
\soul{Нет, так и назывался.}
\people{(Гера)  Это при церквях было, да?}
\soul{Почему?}
\people{(Ольга) Ну, вы же родились в 905-м году.}
\soul{Да нет, не было при церквях. Почему?}
\people{(Ольга)Не обязательно, выходит. }
\people{(Гера) У меня плёнка кончается.}
\soul{Подождите.}
\people{(Ольга) Да, говорите.}
\soul{Война с немцем…}
\people{(Ольга) Да. В 1941 году 22 июня началась война.}
\soul{И сколько?}
\people{И продлилась…}
 (кончилась кассета)
(продолжение)
\people{(Ольга) … и технически, по сравнению с вашим [временем]. Вы даже не знаете. У нас летают в небе самолёты.}
\people{(Гера)  Ракеты летают в космос, на Луну. }
\people{(Ольга) На Луне были люди. Вы знаете об этом?}
\soul{Где?}
\people{(Ольга) На Луне.}
\soul{Самолёты-то и у нас летают.}
\people{(Ольга) А?}
\soul{Самолёты-то и у нас летают, но как самолёты могут долететь до Луны?}
\people{(Ольга) Нет, не самолёты, это уже другие приспособления. }
\people{(Гера) Вы Циолковского помните? Читали? Нет?}
\people{(Ольга) Знаете Циолковского? }
\soul{Нет.}
\people{(Гера) Нет? Вот он будет родоначальником космического авиастроения. }
\people{(Ольга) Он придумает ракеты, которые могут донести корабль до Луны. Ну… не только до Луны, на  орбиту Земли вынести.}
\soul{Это пушка?}
\people{( Гера) Нет, это, скорее, похоже на снаряд, в котором стоит двигатель. }
\people{(Ольга) Сам, ну… как пушка. Пусть будет так.}
\soul{И что, на Луне и правда, живут?}
\people{(Ольга) Нет, На Луне никого нет.}
\people{(Гера) На Луне нет таких, как мы, то есть человекообразных. В принципе, там кроме камней и кратеров ничего нет. }
\people{(Ольга) Там жизни, такой, как здесь, нет.}
\people{(Гера) Но, там своя жизнь, другая. Мы её не видим.}
\soul{Не видим?}
\people{(Ольга) Но она не… Да, это то, что…}
\soul{А вы знаете доктора Папенса.}
\people{(Ольга) Паркенса?}
\soul{Папенс.}
\people{(Ольга) Как?}
\soul{Папиус.}
\people{(Гера) Папюс?}
\soul{Да.}
\people{(Ольга)А-а! Папюс. Доктор чёрной магии. Ну, мы слышали, читали, немножко.}
\soul{Вот, у меня сейчас его издание за 1898 год, и что вы хотите сказать - что он пишет правду?}
\people{(Ольга) Нет, он не всегда правду писал, но в общем,.. может быть доля какая-то и есть, правды, да. (срыв)}
(Между собой в ожидании)
\people{(Ольга)Так интересно с людьми разговаривать. Ничё себе.(смеётся) }
\people{(Лена) Сейчас опять кто-нибудь придёт. Все, он ушёл.(о Сергее)}
(вышел на контакт другой) 
\soul{Я вас устрою?}
\people{(Ольга)  Кто?}
\soul{Ну, давайте-с, спрашивайте, и мы поотвечаем.}
\people{(Ольга) А вы кто?}
\soul{Хм, кто мы? Ага,.. а  как вы хотите, чтоб мы назвались?}
\people{(Ольга) Мы не хотим, как вы себя,..  мы, например люди, а вы кто?}
\soul{А я Вася.}
\people{(Гера) Ну, а я Петя. }
\soul{Спрашивайте, спрашивайте.}
\people{(Ольга) А-аа! Ничего себе! (Гере) Это, Гера, надо что-то с этим делать… кто это ещё?}
\people{(Гера) А мы не хотим с вами говорить.}
\soul{Как - не хотите? Почему? Чем я вам не понравился? }
\people{(смех)}
\soul{Со всеми говорите, а со мной не хотите?}
\people{(Гера) Ну, хорошо, о чём ты хочешь спросить?}
\soul{По-моему, вы спрашиваете, а я  отвечаю.}
\people{(Гера) Ну, хорошо, в каком времени вы живёте?}
\soul{Мы? В настоящем.}
\people{(Лена) Ух! Ничего себе! }
\people{(Гера) В настоящем? А вы нас ощущаете как-то?}
\soul{Ну, да.}
\people{(Гера)А что я сейчас сделал?}
\soul{А шут вас знает, что вы сделали. Вы спрашивайте,  мы отвечаем. }
(Смех)  между собой совещаются продолжать или нет
\soul{Давайте, давайте спрашивайте. Спрашивайте. Ну, чего вы испугались? Сказали, что мы можем поболтать минут пять, вот мы и болтаем с вами, чего вы ещё хотите.}
\people{(Гера) А, ну, давайте, пять минут, в принципе это… Скажите,  у вас технократическое развитие цивилизации? Нет?}
\soul{Нет.}
\people{Эфирное, да?}
\soul{Какое ``эфирное''?}
\people{Не ``кефирное'', а ``эфирное''. }
\soul{Не знаю.}
\people{А как у вас с юмором?}
\soul{Нормально.}
\people{В почёте?}
\soul{Да как обычно. А у вас что, не в почёте?}
\people{Смотря какой. }
(обращаясь к Ольге)  Я не знаю, писать? Нет? (вести ли запись или выключить магнитофон./прим./)
\people{Пиши, пиши!}
\people{А у вас женщины есть?}
\soul{Есть. }
\people{А вы с нами  каким  образом контактируете?}
 
\soul{Нам сказали, что мы можем поболтать. И всё.}
\people{(Смех)  Ха! Поболтать! Ничего себе!}
\soul{Спрашивайте! Спрашивайте!}
\people{А вы кто, кто? Кто они?}
\people{Вася какой-то.}
\soul{Вася? Пусть будет Вася. Только спрашивайте.}
\people{А что спрашивать-то? Мы не знаем, с кем мы разговариваем.}
\soul{А я тоже не знаю, с кем я разговариваю.}
\people{Ну, вам сказали, можно поболтать. А кто сказал?}
\soul{Можно поболтать. О чём хотите болтайте.}
\people{А кто вам сказал, что можно поболтать?}
\soul{К нам приходят тут какие-то.}
\people{Кто – ``какие-то''?}
\soul{Они ``богами''  нас называют. Ругают нас, что мы…}
\people{Что?}
\soul{Ну, не так живём. Не так делаем. Даже короля нашего и то уже обхаяли.}
\people{(Смех) В каком году сейчас…(Смех)}
\soul{Ну, этого я не знаю… В настоящем.}
\people{В каком году вы живёте? Какой у вас год?}
\soul{Не знаю.}
\people{Счёта нет? Системы счёта.}
\soul{Счёт? }
\people{Да. }
\soul{Ну, счёт есть. }
\people{Вы считаете? Сколько вы живёте?}
\people{Сколько лет вы живёте?}
\soul{О-о! Да это у нас считают только вот монахи.}
\people{А вы по национальности кто?}
\soul{А что это такое?}
\people{А монах, кто такой?}
\people{Негры есть?}
\soul{Монахи – служители бога. Вот они и считают. Составляют календари. А нам-то зачем? Они приходят и говорят нам:  Пришло время сеять. Вот мы и сеем.}
\people{А вы у них служите, да?}
\soul{Да, нет.}
\people{А что вы сеете?}
\soul{Как что? Пшеницу.}
\people{В какой стране вы живёте? Как вы себя называете?}
\soul{Не знаю.}
\people{Русичи вы? Или…}
\soul{Люди.}
\people{Ну, а как русские или хохлы, или кто там?}
\people{Викинги.}
\soul{Не знаю.}
\people{Ну, а вообще,  на какой планете вы живёте, вы в курсе?}
\soul{Земля.}
\people{Планету вы знаете, значит, да? А национальность – нет? Этого не может быть!}
\soul{Ну, не знаю. У нас нет.}
\people{А читать, писать умеешь?}
\soul{Ну-у…Мне не говорил никто.}
\people{А почему тебя зовут Вася?}
\soul{Не знаю. Мне сказали – назовись так, я и назвался. Какая мне разница?}
\people{А окружающие твои тебя как зовут, когда хотят обратиться?}
\soul{Ко мне пришли монахи и сказали, что ``ты будешь разговаривать'', я и разговариваю. А за это они мне обещали дать землю.}
\people{Много?}
\soul{Ну, мне хватит.}
\people{А с кем, они сказали, ты будешь разговаривать?}
\soul{Нет. Просто сказали, будешь разговаривать. Положили меня, подняли зачем-то руки. И всё. Потом всё. И вы заговорили, и я заговорил. }
\people{А они сейчас через тебя слышат нас? Да?}
\soul{А откуда я знаю?}
\people{А-а, ты не знаешь.}
\people{А…ты крестьянин? Парень ты?}
\soul{Парень. }
\people{Сколько лет?}
\soul{Не знаю. Это надо спрашивать у монахов.}
\people{Жена есть?}
\soul{Есть. Две.}
\people{Две жены даже? Ничего себе…}
\people{В смысле, две? Одна не официально, одна официально, да?}
\soul{Не понял.}
\people{В церкви венчались? Нет?}
\people{Нет. Церкви – нет? Да?}
\people{Как нет? Монахи - есть, а церкви - нет?!}
\people{Монахи могут быть, церкви ещё не было. }
\soul{Монахи.}
\people{Монахи? А как они одеты?}
\soul{Ну, как одеты…Ну, закрывают лица.}
\people{Балахоны. }
\soul{Нам нельзя видеть их лица. Они всё же `` приближенные к Богу''. }
\people{А–а! А у вас какие дома? Крыша остроконечная?}
\people{В монастыре.}
\soul{А я не понял слово?}
\people{Острая? Крыша какая?}
\soul{Да, не знаю. У нас нет… Что такое – ``крыша''?}
\people{Дом, где ты живёшь…Есть у тебя дом?}
\soul{Да.}
\people{Ну, на ней крыша, чтобы от дождя…}
\soul{Да  нет крыши.}
\people{Как нет крыши? Ты не построил крышу?}
\soul{А зачем? }
\people{Как?}
\soul{Пришли монахи, дали мне пещеру - я и живу. А что такое – ``крыша''?}
\people{Пещеру???}
\people{Оба-на! Это когда же было-то? А что ты ешь, Вася? Чем питаешься?}
\soul{Иногда, нас угощают вкусненьким.}
\people{Сам себе на пропитание не зарабатываешь?}
 
\soul{Почему? Я же выращиваю. Для чего я с вами разговариваю? Чтоб получить землю.}
\people{А-а!…}
\soul{И тогда я уже начну уже иметь своё. Тогда смогу завести третью жену.}
\people{А зачем вам столько?}
\people{Подожди. Вася, а вообще, какие у вас имена бывают?}
\soul{Я не знаю, нас не называют по именам.}
\people{Никого?}
\soul{Да нет. Это монахи пришли и сказали: назовись, если спросят. А вы сразу и спрашиваете.}
\people{А в чём вы одеты? Вы сами.}
\people{Солнце светит у вас? }
\soul{Светит.}
\people{Жарко?}
\soul{Жарко.}
\people{А снег бывает?}
\soul{Снег? Я не знаю, что это такое.}
\people{Так, ясно…Это восток. }
\people{Нет, не восток. Может быть юг.}
\people{Море есть у вас?}
\soul{Море? Да. }
\people{А река?}
\soul{Есть.}
\people{А как она называется?}
\soul{Не знаю.}
\people{Просто река и всё?}
\soul{Ну, только в  реку мы можем подойти, а дальше нас не пускают. Есть море. Но, к морю мы приходим только один раз в году.}
\people{Зачем?}
\soul{А я не знаю. Нас умывают и читают молитвы.}
\people{А корабли по морю ходят?}
\soul{Корабли? }
\people{Да.}
\soul{А это что?}
\people{А лодки? Чтоб по морю можно было плавать.}
\soul{Плавать???}
\people{Да.}
\soul{Вы чё? С ума сошли что ль?!}
\people{( Смех.)}
\people{Нет. А ты?}
\soul{А как можно плавать по морю?}
\people{Ну, как же - птицы плавают. }
\people{Чайки.}
\people{Ну, которые живут около моря. Видел, как птицы плавают, ну,- утки?}
\soul{Я же не  птица, как я могу плавать?}
\people{Ты можешь плавать, если… Лес есть у тебя? Лес? Дерево?}
\people{Деревья есть?}
\soul{Не знаю.}
\people{Любые деревья, где яблоки растут, груши, апельсины мандарины. Есть что-нибудь?}
\soul{Нет.}
\people{Вы чем питаетесь? Пшеницей?}
\soul{Нам дают зёрна –  мы выращиваем.}
\people{А вы ещё… а раньше, что выращивали?}
\soul{Я ничего не выращивал. А у меня нету земли.}
\people{А вы разводите овец там… или  каких-то животных?}
\soul{Монахи - да, они имеют.}
\people{А как же вы живёте? Питаетесь чем?}
\soul{Они нам приносят.}
\people{Скажите, а у вас кожа чёрного цвета  или белого?}
\soul{Кожа?}
\people{Кожа, да. Ваша кожа. }
\people{Ваша жена, она как выглядит? Жена, какая у вас цветом: тёмная или светлая? }
\people{Как солнце или как ночь?}
\soul{До солнца ей далеко, но и до ночи тоже.}
\people{Средняя? Как песок? Цвет, приблизительно, - как песок?}
\soul{Нет, темней.}
\people{А волосы?}
\soul{Волосы?}
\people{Волосы. Там есть волосы?}
\soul{Нет. А зачем?}
\people{А у вас, есть волосы?}
\soul{Да вы что?!}
\people{А у кого-нибудь есть волосы?}
\soul{У нас бывает, что-то растёт, но монахи за это карают.}
\people{Как карают?}
\soul{Наказывают. Оставляют без пищи.}
\people{Это рабы. Вы – раб?}
\soul{Кто?}
\people{Раб?}
\soul{А что такое ``раб''?}
\people{Раб, это тот, кто работает на хозяина.}
\people{Кто в подчинении находится у других людей, и не имеет своего ничего. Делает только то, что прикажет ему хозяин.}
\soul{Нет, тогда я  не раб. У меня есть две жены, и у меня будет земля.}
\people{А-а… Скажите, а у вас местность, пустыня?}
\soul{Нет.}
\people{А животные, вот какие? Верблюды есть?}
; Я не знаю, что это такое.
\people{А какие животные есть вообще? Назовите.}
\people{У монахов, какие животные есть?}
; Не знаю.
\people{Вы их не видели?}
; Мы их видели, и мы не знаем их имён. Монахи не говорят нам. Они только приходят и кормят нас.
\people{А как выглядят  животные у монахов? Как выглядят? Вы опишите их.}
; Ой… Ну,.. не знаю вообще-то.
\people{Ну, высокие, низкие, толстые-тонкие, шерстяные-лысые, рога есть, хвост, копыта, крылья? Что есть?}
; А! У них есть то, чего запрещают нам монахи.
\people{Волосы. Да?}
; Ну, не знаю. Наверно.
\people{Белого цвета или чёрного?}
; Не знаю.
\people{Ну, как ночь или как солнце?}
\people{А вы часто моетесь? Вы моетесь в речке?}
; Да, монахи следят за этим строго.
\people{Что бы вы умывались. А сколько вас? Ну, вы живёте с жёнами, а ещё есть кто-нибудь, есть рядом? Живёт кто-то рядом ещё?}
; Да. У меня двое детей.
\people{А ещё кто-нибудь? А вы откуда пришли?}
; Мать была, но её  забрали монахи.
\people{А ещё какие другие люди, которые не родственники вам? Есть?}
\people{Друзья есть?}
\soul{Друзья?}
\people{Он не понимает.}
\people{Родственники или просто соседи?}
; Соседи есть.
\people{Есть?  И где? Они тоже в пещерах живут?}
; А где ещё можно жить?
\people{А монахи где живут?}
; В пещерах.
\people{Тоже в пещерах? А как они о Боге говорят? Что такое Бог?}
\people{Это что-то ``высшее''  какое-то там. Они приходят, учат монахов, говорят, что им делать, а монахи уже учат нас.}
\people{Скажите, вы знаете, что существуют другие народы, которые далеко-далеко живут за морем?}
; За морем? 
\people{Да. Нет, не знаете? А монахи только мужчины или женщины?}
; Нам говорили, что море – это конец света. Кто там может жить? Умершие. А-а! Да-да! Нам когда-то говорили, что там есть умершие. Но это -  по ту сторону моря.
\people{А скажите,  монахи только мужчины или женщины есть?}
; Мужчина-женщина, это как?
\people{Вы мужчина, у вас есть дети.}
; А-а, нет! Только мужчины!
\people{Только мужчины, а женщин нет, да?}
; Что, я свою жену сделаю монашкой?
\people{Но сказали, что вашу мать увели в монашки?}
; Нет. Её увели монахи.
\people{Что они с ней сделали?}
; О, это тайна! Это их тайна.  Пришло время и забрали.
\people{Умирать?}
; Придет моё время - заберут и меня.
\people{А! Монахи забирают, когда умирать надо? Старых, да?}
; Да, нет.
\people{Всяких забирают и молодых и старых?}
\soul{Да.}
\people{А что они с ними делают? Это тайна? Они их приносят в жертву Богу?}
; Не знаю.
\people{Скажите, откуда  приходят Боги? Как они вам сказали?}
; О-о! Это страшно! Мы все прячемся, когда они приходят. Потому, что очень много огня и нам будет больно.
\people{Огня? Как - много огня?}
; Если мы вовремя не спрячемся, то нас убьют.
\people{Кто? Бог? Бог убьёт?}
; Да. 
\people{За что?}
\soul{Мы сгорим.  Монахи приносят нам огонь. А боги сами, как огонь.}
\people{А откуда они приходят? С неба?}
;  Да. 
\people{А при этом бывает звук какой-то такой - громкий?}
; Да.
\people{На что он похож: на протяжный или очень резкий хлопок?}
\people{Гром, да?}
; Не знаю, но ушам больно.
\people{Долго больно или сразу раз и всё?}
; Мне потом жена что-то говорит, а я уже не могу услышать.
\people{А откуда вы знаете, что это боги приходят?}
\people{Монахи говорят.}
; А кто ещё так придёт?
\people{А как они – светлые? Как в очаге огонь?}
; Монахи говорят: тот, кто увидел Бога, не может жить.
\people{И они убивают?}
; Да никто не видел! Мы видим только, как они прилетают, и то, чтобы  потом никто знал.
\people{А на кого они похожи  боги?}
; Не знаю.
\people{На людей?}
; Да не видал я их!
\people{Скажите, а монахи видели Бога?}
; Ну, если они к ним прилетают?
\people{Скажите, вы можете рассказать, как живут монахи?}
; Не знаю. Они приходят утром, дают нам пищу и уходят. Потом приходят и ведут нас к реке, умывают нас. А потом ещё говорят, когда приходит время делать детей или сеять.
\people{А-а. И даже делать детей они говорят, когда нужно, да?}
; Да.
\people{А когда?  Ночью или днём?}
; Да нет, они просто приходят и говорят, что пришло время.
\people{А-а… И что вы делаете?}
; Что делаем…Делаем детей! Что же мы можем делать-то?!
\people{А это приятно?}
; Ну, с первой женой я бы этого не сказал.
\people{Это у вас вторая жена?}
; У меня их две. 
\people{А, сразу две. }
\people{А сколько можно иметь жён?}
; Да сколько хочешь.
\people{Сколько хочешь или сколько скажут?}
; Да нет, наверно, сколько хочешь.
\people{А когда монахи не приходят и не говорят, что пришло время делать детей, то вы их не делаете  сами? У вас нет такого желания?}
; Нет, у нас просто нет детей.
\people{Как нет детей?}
; Они приходят и говорят: пришло время детей. Тогда и рождаются дети.
\people{У жены рождается дитё, ребёнок?}
; Ну, да.
\people{А у неё большой, большой живот, да?}
; Да.
\people{А потом появляется маленький ребенок, который кричит?}
; Да.
\people{А почему мужчины не рожают детей?}
; Ха! Не хватало мне ещё этого! А на что мне тогда жёны?!
\people{А-а! (Смех) А вокруг вас, может быть, ещё кто-то живёт, рядом?  Может село какое-то? Кто-нибудь ещё живёт?  Вы гуляете?}
; Да, далеко - живут.
\people{Тоже такие же люди, как вы?}
; Я  как-то раз ночью подсматривал. Но это,  чтоб  монахи не знали. Вы им не говорите. Там тоже есть дом монахов. И видал людей.
\people{Таких же, как вы?}
; Да.
\people{И в таких же одеждах?}
\soul{Да.}
\people{Вас одевают? У вас одежды длинные? До земли?}
;  Зачем? Было-бы неудобно.
\people{А какие у вас одежды?}
; Повязка.
\people{А-а! Повязка! А у монахов длинные одежды?}
; Они все прячутся.
\people{А! Их не видно, да? Ничего не видно, ни ног,  ни рук?}
;  Ничего не видно. У нас есть один, говорят, он  видел лицо. 
\people{Монаха?}
\soul{У него выкололи глаза.}
\people{Скажите, как ваши дети разговаривают? Вы как учите их говорить? Никак,  или они сами…}
\soul{Да. }
\people{… и  воспитывают их. }
\soul{Потом, приводят их…}
\people{Когда они вырастут?}
; Не-ет, они ещё должны  пройти крещение, только тогда они становятся мужчинами.
\people{А женщины? А девочек куда?}
;  А девочек забирают сразу, и мы их больше не видим.
\people{Никогда не видите?}
; Нет.
\people{Они их в жёны кому-нибудь отдают?}
; Не знаю.
\people{А кого вы в жёны берёте тогда?}
; Женщин.
\people{А где они?}
; Как где? Они с нами живут.
\people{А мальчики на ком женятся тогда?}
; А мальчики не женятся.
\people{Как не женятся? }
; Как может жениться мальчик?
\people{Нет! Когда он вырастет. Когда мужчиной становится.}
;  Мужчиной становится? О! Это целый ритуал!  Вы думаете, это так просто? У нас есть пещера, нас запускают туда. Вот, как было у меня… Нас было четверо, которые должны были стать мужчинами, нас запустили в пещеру, закрыли её, и там мы должны были найти жён.
\people{Жён найти? Так вот их куда забирают? И вы их нашли? Ощупью?}
; Я нашел целых две!
\people{(Смех)  А хотите нас что-то спросить? Вам интересно, с кем вы разговариваете? Вы можете у нас что-нибудь спросить, может быть, мы вам что-нибудь скажем интересное для вас?}
;  Мне интересно – лишнего я что-нибудь не сказал?
\people{Да нет!}
\soul{Монахи – они - ой и сердитые!}
\people{Сердитые.}
(Переводчик вышел в сознание./прим/)
; Вы давали большой счёт что ль? 
\people{Какой большой счёт? Нет. До девяти от одного.}
\soul{Да? Интересно так, вот… вроде бы себя чувствуешь, а…Оригинальное вообще-то ощущение. }
\people{Что?}
\soul{Цвета какие-то синие… Вот не пойму, такое, знаешь,  раздвоение какое-то. }
\people{Синие цвета?}
;  И там и там, вроде как. Вроде бы ещё в контакте и вроде б не в контакте. Оригинально!
\people{Да, да, давай!}
; Ну, так хоть спросите что-нибудь!
\people{Спрашивайте.}
; Вы спрашивайте!
\people{Как себя чувствуешь?}
; Да ничего, нормально.
\people{На клапан не давит?}
; А шут его знает. Я его, думаешь, чувствую?
\people{Где последний раз был-то? Ты в курсе, где ты был? }
\soul{В смысле?}
\people{Ну, с кем ты контактировал? Ты слышал их?}
; А-а… Нет.
\people{Совсем никого?}
; Да нет.  Ну, так, картинки я какие-то помню, так… местами.
\people{Что там происходило? Огонь, пламя, гром, молния, что-то такое?}
; Да нет, я не помню.
\people{А кто ты? А ты кто? Как тебя зовут?}
; Я.
\people{(Смех) А кто это ``я”-то? Нет ну ты скажи кто ты?}
; Ну, Гена я.
\people{А фамилия?}
; Харитонов.
\people{(Смех) }
\soul{Ну, дожили! Забыли!}
\soul{Ты можешь нас продиагностировать? Ты нас видишь?  Вот, я встану.}
; Да я и так вас вижу, можете не вставать.
\people{Видишь?}
; Да, такими огоньками.
\people{Синие? Нет? Не синие?}
; Разные.
\people{Разные ; это как?}
\people{Где у меня сейчас видится? Какими огоньками?}
; Ну, в области сердца. Искорки такие, голубые.
\people{О-о! Ну-ну!}
\people{А что это? Хорошо, плохо?}
; А откуда я знаю? А вот вокруг головы, как пелена, какая-то такая молочная какая-то. Типа киселя что-то.
\people{Угу}
\people{Белая?}
; Ну, да.
\people{А у нас?}
\people{Нет,  погоди, а что ещё? Ну, ещё где-то есть  что-то?}
; Да.
\people{Коленки, ноги, почки, печень? Что?}
; Сейчас, подожди, соображу. У меня рот болит. Такое ощущение, что я нахохотался. 
\people{Да, ты… это…было смешно. Какая у него аура цветом?}
\people{Общий фон видишь?}
; Да такое что-то… такое, синевы немножко, но…, но больше вообще-то, такого…что-то такого между красным и желтым. Что-то такое вот.
\people{Странно.  А у нас? }
\soul{Зелёная.}
\people{Эх ты! Позеленела! А она была розовая раньше.}
; Не-е. Зеленая. 
\people{В области чего-нибудь, светится, может быть? }
\soul{Да. Что-то там светится!}
\people{Давай!}
; Да как я могу сказать, что там может  светиться такое? Неудобно сказать-то!
(смех)
\people{Ну, каким цветом, мы поняли.}
\people{Зеленый.  Он сказал.}
; Вся, в общем, зеленая, а так и красного и всякого есть. А вот… Ну, как сказать? Между ног, что ли? 
\people{Ну, говори! Говори!}
\soul{Красная.}
\people{Ну… это естественно. Я на этом работаю.}
\people{Не-ет. Это же между ног. Это же ``муладхара”… ``Кундалини''. Красное, конечно.}
\people{Ну, как… типа, как… ощущение такое – как наполнение крови, что-то такое…}
\people{Так… По-моему,  нам прервали…}
(переводчик вышел из сознания. Контакт с другими.)
; …  разговаривать с вами.
\people{А кто  - вы?}
\soul{Извините, если что не так.}
\people{Ничё, ничё..}
\people{А кто вы?}
; Меня называют ``Носителем жёлтого”
\people{Желания?}
; Желтого. Я отвечаю за жёлтый цвет.
; Носитель. Желтый цвет.
\people{А носитель жёлтого…Это значит жёлтый цвет – это цвет желаний? Нет?}
; Не-ет. Это цвет  боли или разочарования.
\people{А. Ну, когда… У нас даже есть песня такая про жёлтые цветы. А скажите, если у человека жёлтая аура, значит, он разочарован?}
; Ну, чисто желтый-то  я никогда не видел. Нас много, вообще-то.
\people{Оттенков жёлтого, да?}
; Не-ет. За все оттенки жёлтого отвечаю только я. Все жёлтое приношу я.  У меня есть друг,  он… правда, иногда мы с ним часто ссоримся, ну…он  отвечает за красный.
\people{А друг – кто это?}
\people{А что это - ``красный''?}
; Ну, это эмоции,  очень яркие эмоции.
\people{Красный, да?}
\soul{Это - та же боль. Та же боль.  }
\people{То же – боль?}
\soul{То же – боль. Ярость. Потому он, наверное, постоянно на меня кидается.}
\people{А вы с ним друзья?}
; Друзья.
\people{А вы на него не обижаетесь?}
; Да нет. Просто он меня дразнит, говорит: ``А ты все время разочарован!'' Ну, может быть.
\people{Разочарован?  А где вы живёте?}
;  Ну, как сказать где? Вообще-то -  в вас.
\people{В нас? А вы осознаёте, что вы в нас живёте?}
; Мы разукрашиваем вас.
\people{А-а, не мы рисуем вас, а вы нас рисуете?}
\people{То есть, вы наши эмоции? Нет?}
; Да нет, мы, скорее всего, реакция ваших эмоций. Вы что-то подумали – сразу меняются краски, а мы должны успевать за вами. Мы должны восстановить их или,  когда говорят нам, – изменить полностью. Но это очень редко бывает.
\people{А мы вас обижаем?}
; А как вы нас обидите?
\people{Ну, вот одни говорят нам, что мы их убиваем. А мы не знаем даже.}
; Да нет.
\people{Вы не против нас, ничего такого не имеете? }
; Да как мы можем иметь против? Мы же с вами-то живём вместе!
\people{Скажите, вот, ``желтый''… Почему ``желтый''  именно здесь, сейчас, а не ``красный''?}
; Ну, просто, меня попросили с вами поговорить и всё.  А друга просить -  он очень горячий. 
\people{А другой цвет? Допустим, ``зеленый''?}
; Я не знаю. Ну, может и зелёный с вами говорить.
\people{А что обозначает?}
; Зелёный? Этот цвет я знаю плохо. Ну, вообще-то, как мне говорил, это, скорее, боль за что-то чувственное. Но, вообще-то, я точно не знаю.
\people{Угу. А ``синий''?}
; ``Синий'' – это холод. Мне приходилось, как-то раз, убирать его. Это холод  и  ум.
\people{Ум. А, понятно.}
\people{Холодный ум.}
\people{То есть, это  ум. Он без эмоций. А фиолетовый?}
; Фиолетовый? Вы, знаете, я видел его, но никогда не подходил ближе.
\people{Вы его боитесь?}
; Да нет. 
\people{А! У вас просто контакта нет. Контакта с ним нет, да? А оранжевый?}
; Оранжевый рядом со мной. Мы всегда работаем вместе.
\people{А как вы себя сознаете?}
Счет -1-2…
(Переводчик снова в граничном состоянии сознания. /прим/)
\people{Что?}
\people{С кошку ростом. Большой, серебристый. }
\people{Интересный? Незнакомый?}
; Не понятно. 
\people{Незнакомый?}
\soul{Я даже не могу понять, где у него перёд, а где зад.  Что-то такое… пушистое такое. Но вот не даёт, паразит, дотронуться.  }
\people{Серебристый?}
; Да.
\people{Голубой? Серебристо-голубой?}
; Да нет, белый. Скорее всего - так… белого…Типа серебра что ли…
\people{Куда ушёл? Ушёл?}
Конец контакта.
(Идет обсуждение контакта)

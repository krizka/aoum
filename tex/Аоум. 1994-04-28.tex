Аоум. глава 19-я 28-04-1994г
Георгий Губин
 
\people{**}
 28-04-94г   `` Далее, поймите, мы говорили вам о сознании и говорили, о битве, и говорили, что сознание ваше не видит, и потому битва идёт, битва идёт за вас и в каждом из вас, и вы являетесь и той обороной, и той защитой, и той силой. Силу даёте вы – для борьбы. Вы кормите силы зла и силы добра. Вы, сегодня, можете быть противником зла. Завтра или, может быть, даже сегодня, вы будете уже его сторонником. Вы меняетесь, вы не постоянны. И каждая мысль ваша имеет эту энергию и даёт, и даёт. Вспомните. Вы же говорили о вампирах. И вы знаете – нет, не вампиры приходят к вам. Вы, вы рождаете. Вы даёте силу. Вашими силами воюют. Вашими. Нет иных''.
 
\people{**}
\people{Сегодня 28 апреля. Вы знаете, что случилось после последнего сеанса с нашим переводчиком? Ну, по крайней мере, он рассказывает, если вы не знаете, что к нему сюда на это место пришли двое: старик и молодой парень. Скажите, пожалуйста, кто это был?}
 
\soul{Мы говорили вам, о вас, как о множестве. И, подумайте, сколько вас - в вас? }
\people{Но это были, видимо, не реальные люди?}
\soul{Мы уже говорили вам, и вы говорите: ``мысль материальна''. Подумайте, да, физически – не реальны. }
\people{Но чьей мыслью рождены эти люди? Переводчика, или кого-то из нас? Мы, например, ни сном, ни духом…}
\soul{Будьте внимательны, мы говорили, о вас, как о множестве. Подумайте, кто приходил к вам? Вы же, вы же. Каждый из вас носит в себе многое. Если хотите – пусть это будут ваши мысли. Мысли есть добрые, есть злые. Подумайте, в вас – множество. И это множество приходит к вам и говорит вам, и многое множество приходит, чтобы пугать вас – ``своё''. И есть те, кто пугает вас, для того, чтобы вы двигались дальше. Мы говорили вам, что самое сильное чувство в вас – страх. И потому, потому и доброе и злое будет приходить, и будет воздействовать на то, что болит в вас.}
\people{Нам показались… Переводчику показались эти люди агрессивного плана. В отличие, скажем, от вас. Это действительно такие? }
\soul{Поймите, каждый миг, миг, даже короче, чем вы можете представить, идёт борьба, и в той борьбе рождаются и погибают многие. Вы же, сознание ваше, не видит того.}
\people{Скажите, а это опасный симптом для нас и для переводчика, или это ``так'', можно пропустить?}
\soul{Подумайте, если враг приходит открыто, то, значит, это последнее, что он может сделать.}
\people{Получается, что это враг, и он в таком образе пришёл. Может быть, надо отнестись очень осторожно к этому и внимательно?}
\soul{Поймите, и враги и друзья созданы вами. Вашим понятием. Вот и приходит понятие ваше к вам. И если вы поддадитесь страху, своему же страху, что будет с вами? Вы будете задушены и погибните.}
\people{Ну, переводчик, видимо, правильно реагировал. Он не испугался, и настроен продолжать наши сеансы, несмотря ни на какие угрозы…}
\soul{Если б только, в вашем понятии, храбрость говорила им, было бы правильно.}
\people{А что им ещё руководствует?}
\soul{Поймите, вы упрямы.}
\people{Упрямство? Ну, может быть это и не такое плохое качество, но из чувства самосохранения…}
\soul{Смотря какие корни.}
\people{Что бы вы порекомендовали из чувства самосохранения переводчика?}
\soul{Мы говорили вам, создайте один ствол. Вы же не можете,- вот вам и ``гости'', и мы приходим к вам в гости, вместо того, чтобы быть вами. Они же, приходят к вам в гости для того же.}
(сбой)1-2
\people{А какие корни у него, у его упрямства?}
\soul{Давайте спросим, какие корни у вас? Вы можете сказать? Если вы не можете сказать о себе, к чему вас интересует иное? Если вы не можете понять себя, вы хотите понять других? Почему? Может быть проще - и вы хотите увидеть себя в других?}
\people{А почему пришли именно старец и молодой человек? Кто они, если так разбираться?}
\soul{Мы говорили вам, о врагах, о друзьях, но если вы скажете ``враги'',- победа их.}
\people{Как можно защититься? Вдруг они каждый раз будут приходить? Это вообще-то назойливо. Есть способ защиты?}
\soul{Есть.}
\people{Какой?}
\soul{Способ - не делать ничего сознанием. Ибо когда вы хотите,- хотите сознанием, и ничего не получите. Поймите, в вашем понятии есть инстинкт, и не тот, что вы ругаете – другой. Дайте ему, дайте ЕМУ сделать своё дело. Если вы будете вмешиваться… (теряется контакт).}
\people{Скажите, почему именно только переводчику являются эти люди, а ни к кому из нас ни сном, ни духом не являются и не беспокоят?}
\soul{Разве? Когда-то вы были на его месте, и когда-то вам приходилось быть. И вспомните, не вспомнив ничего физически - что приходило к кому-то из вас и указывало на ошибки? И что сделал он? Он ушёл. Он ушёл в себя.}
\people{Вы не о монахе?}
\people{Это который в степи замёрз?}
\soul{Вспомните, мы говорили вам, и вы же говорили: нельзя уйти от себя, если в вас сидит та сила. И уйдёте от неё, если будете бороться. Бороться её способами, той силы, - она победила вас. }
\people{Прошлый раз был очень неожиданным для нас. Большим откровением. Вы много намёков дали на наши прошлые жизни. Мы не всё, конечно, поняли. Но, нельзя ли было ещё и вот такой намёк: вот вы говорили о писателе, который писал книгу о монахе, сказку о падающей звезде. Вы можете хотя бы букву назвать его фамилии? Или, давайте я буду перечислять азбуку, а вы скажете: ``да, вот эта буква''.}
\soul{Нет. Вспомните, мы говорили о созвучии. Вспомните. Далее, вы говорите о намёке. Что даст вам, если скажем имя вам? Что? Тогда не будет поиска. Тогда не будет ошибок. И не будет побед. Далее… (теряется контакт на пару секунд) …И вспомните, разве монахи носили свои имена? Представьте - те времена, русский монах, и русское имя. И много ли было имён?}
\people{Т.е. это был писатель-монах? И у него совсем была другая… Тем более нам тяжелее.(имеется ввиду – узнать имя его. прим.)}
\soul{Вы не внимательны. Мы сказали: ``пусть вспомнит другой''.}
\people{Ну, нам, значит, не под силу…}
\people{Монах? }
\people{Я, наверно? }
\soul{Поймите, то время было - время других имён. Вы имели своё, данное вам матерью. Но, уйдя, (в монахи. прим.) вы взяли другое. Поймите и подумайте – вы же хотели уйти, хотели уйти от себя, хотели уйти от мира. Вспомните. И потому - не хотели быть узнаваемы и взяли себе другое имя. Далее, мы вам говорили …}
(сбой)1-2-3
\people{Что вы нам говорили? Вы мне говорили…}
\soul{Спрашивайте далее.}
\people{В общем, нам в этой ситуации трудно будет разобраться. Мы ж не можем всю отечественную литературу поднимать…}
\people{Но он ушёл не только от матери, он сменил национальность и ушёл заграницу. Так было?}
\soul{Да. Вы гуляли по свету. }
\people{А мать, какой национальности у него была?}
\soul{Мы вам говорили, что вы ушли от матери? Или быть может, вы ушли от дел своих? Может быть, вы перестали быть переписчиком и просто ушли? Может, вы ушли, ибо гоняла вас ``нечистая''? Вами сказано: ``нечистая сила попутала''. Вспомните. И вы хотели уйти, хотели уйти от позора, ибо объявили сумасшедшим вас. Почему невнимательны вы? Почему? Если искать будете – найдёте. Найдёте. Ибо не потеряно то имя. Если б было потеряно, не говорили бы  вам.}
(сбой)1-2-3-4-5-6
\soul{Поймите, если дадим вам готовое, что даст это вам? Что? Ничто. И вспомните, вспомните, вы говорили о роли переписчика. И далее, вспомните. И поверили ли вы в то? Поверили? Нет. А если бы, было бы даже такое и нашли бы сами, было бы больше веры, ибо вы пережили бы. Пережили! Голые факты не дают чувств. И если дают – негативны они. Поймите то.}
\people{Вы можете сказать, как присутствующая девушка связана с нами кармически? Какое у ней прошлое было?}
\soul{Вы даже не можете представить, как тесен ``клубок''! И подумайте, вы можете прожить вместе всю жизнь, и та нить будет всего лишь рядом, и не вашей. А может быть, всего лишь миг встречи, но тот миг больше даст вам. }
1-2-3-4-5-6
\soul{Спрашивайте.}
\people{Так вот, наша собеседница, которая рядом сидит, с нами связана как-то?}
\soul{Связана. Но не так сильно.}
1-2-3
\people{Она кем была, хотя бы в недавней жизни? Поэтом, литератором или монахиней? Кем?}
\soul{Няней. }
\people{Няней?}
\soul{Да. }
\people{Есть знаменитые няни.}
\soul{Нет. В вашем понятии - отцова мальчишки. }
\people{Какого мальчишки, простите?}
\soul{Далее, далее и далее…(просьба задавать вопросы далее. прим.)}
\people{О вашей цивилизации, о вашем обществе. Вам необходимы элементы самозащиты, а значит, может быть, наличие военных структур? Ответьте, пожалуйста.}
\soul{Нет, в вашем понятии, у нас нет оружия, ибо мы сами – оружие. Подумайте. Мы говорили вам об энергии. Если мы говорим, что мы энергия,- а когда-то мы пытались вам сказать то, но не поняты были вами,- зачем нам оружие, если мы являемся им?}
\people{У вас нет никаких врагов?}
\soul{Врагов? Мы говорили вам недавно, что если вы скажете ``враг'', то, значит, победил враг тот, над вами. Нет, в вашем понятии, есть ``оппонент''.}
\people{Скажите, а есть защитные структуры и средства у других цивилизаций? Хотя бы в нашем околоземном мире.}
\soul{Множество. }
\people{Защитных структур?}
\soul{Далее, подумайте, мы говорили вам о единстве, вы говорите о врагах. Да, есть те, кто недовольны нами и воюют с нами, но не вашими методами. Здесь методы другие. Есть более жестокие. Есть более мягкие. }
\people{Скажите, вам известно о действиях оборонительных структур в других цивилизациях околоземных в последние годы?}
\soul{Да. }
\people{Это были серьезные акции?}
\soul{Да. }
\people{Они как-то повлияли на землян?}
\soul{Более чем вы думаете. Посмотрите на себя и подумайте, что заставило вас вершить дела ваши, и что заставило вас повернуть вспять?  Подумайте и посмотрите, что делаете. Мы говорили вам, что вы ``маятник''. Вся земля ваша – маятник. Посмотрите. Вы говорите: ``Россия. Страдает Россия''.  Одна ли лишь? Вглядитесь, вглядитесь - был подъём, теперь - опускаетесь. И не многие выживут, не многие поймут и не многие увидят. И многие ослепнут, желая увидеть. Увидеть то, что не суждено видеть. В вашем понятии, вы говорите – мечты. Вы говорите ``оборона'', но не умеете понять, и потому, против вас -  не нужна.}
\people{Вы часто говорите, о том, что нам грозят серьезные катаклизмы и кровь. Недавно прозвучала речь верховного правителя церковного. Тоже он сказал, что многие из нас не доживут до следующей Пасхи. Откуда он знает эти подробности?}
\soul{Подумайте.  Приходят к вам враги ваши, угрожают вам. … трудно ему будет…если вы уверуете, будете знать его…(речь идёт о том, что врагу будет труднее нас одолеть, если мы будем его знать.  прим.)}
\soul{Гораздо будет сложнее…}
\people{Мы можем продолжить? }
\soul{Спрашивайте. Вот вам. (кончается кассета и мы отвлекаемся на эту проблему. прим.)}
\people{Рабы техники. Ясно. Скажите, против кого приходилось применять оборонительные структуры? Против землян или против каких-то других цивилизаций? Мы-то не почувствовали…}
\soul{И то и другое. Далее, поймите, мы говорили вам о сознании, и говорили о битве, и говорили, что сознание ваше не видит, и потому битва идёт. Битва идёт за вас, и в каждом из вас. И вы являетесь и той обороной, и той защитой, и той силой. Силу даёте вы – для борьбы. Вы кормите силы зла и силы добра. Вы сегодня можете быть противником зла. Завтра, или может быть даже сегодня, вы будете уже его сторонником. Вы меняетесь, вы не постоянны. И каждая мысль ваша имеет эту энергию и даёт, и даёт. Вспомните. Вы же говорили о вампирах. И вы знаете – нет, не вампиры приходят к вам. Вы, вы рождаете. Вы даёте силу. Вашими силами воюют. Вашими. Нет иных.}
\people{У нас провалилась доктрина, по которой  ``от каждого по способности - каждому по потребности''. Как иные цивилизации решают эту проблему? И решали ли они, другие цивилизации?}
\soul{Да. Есть такие лозунги и на других. Но если вы говорите, у вас провалилась, почему же тогда у других может не получиться? Подумайте. Следовали ли  вы тому? Следовали? Следовали ли вы вашим законам, следовали ли вашим заповедям? Нет.}
\people{А вообще-то, решаема эта проблема в целом? Проблема потребительства. Может быть её как-то решили с помощью генной инженерии, т.е. изъяли из душ жажду наживы, которая сейчас человечеству, фактически, вредит?}
\soul{Неужели вы думаете, что все чувства ваши в генах? Все чувства ваши в физике? Да, мы говорили: ``эмоции – это мир физический''. Но, вспомните, мы говорили вам, что вся физика ваша - это всего лишь отзвуки души. Далее, вспомните. Говорили вам о преступниках. Вы помните?}
\people{Что-то припоминаю.}
\soul{``Что-то''! И подумайте  далее, далее, далее…}
 
\people{Мы чувствуем, что человечество не совершенно, потому что у него есть такая неистребимая черта, как жажда наживы. Жадность. У других цивилизаций нет этого?}
\soul{Поймите, вы – целое. Целое. И это целое вы раздробили на множество кусочков. На кусочки ``добра'' и на кусочки ``зла''. На кусочки ``щедрости'', ``жадности'' и далее, далее. И все эти кусочки, бьют вас же. Те же осколки сидят в вас и ранят вас. Вы же даже не соединились. И стать совершенным, совершенным… Вспомните. Вспомните ваши заповеди, вспомните вашу Библию. Вспомните. Вы же разбились, разбились на множество, на столь множество, что не узнаёте своё же. К вам приходят другие, - вы говорите ``не моё - чужое'', и начинаете, начинаете. Вот вам, начало ваших бед, когда вы сказали: ``не моё - чужое''.}
\people{А скажите, как другие цивилизации решают проблему профессии? Вот, допустим, никто не хочет быть шофёром, строителем, асфальт укладывать, а хотят все творить, писать, сочинять или сочинять музыку. Как это решается у других цивилизаций?}
\soul{Давайте скажем так: что многие хотят, всё-таки, стать шофёром. Подумайте. Не многие хотят только работать. Далее, мы говорили вам, о подобных вам. И представьте - да, есть, в вашем понятии, кто не хотел … (сбой) }
1-2-3-4-5-6-7-8
\soul{Спрашиваете об ``иных'' или подобных себе? Вы спрашиваете о нас или, о подобных? О подобных, - то же, что и у вас. У вас – машины, у них - другое. Какая разница, шофёр или что иное?  Далее… Да, вы все хотите писать, вы все хотите рисовать и далее, ибо вы, вы, и в вас есть начало - начало творения, и в каждом оно больше или меньше.}
\people{Но нет элемента насилия в других цивилизациях, чтобы индивидуумы занимались только тем, чем положено?}
\soul{Давайте сделаем так: вы спрашиваете или ``выше'' или о ``подобных себе''. О подобных – возьмите и оглянитесь, посмотрите на себя.}
\people{А если, ``выше''?}
\soul{"Выше''? Да, есть и насилие там, ибо тогда это был бы совершенный мир. Но, то насилие – насилие психическое. Здесь же, у вас, бьют физически, физически. Вы же, понимаете, как психическое. Подумайте. Подумайте, что делаете вы! Вы рабы не только техники, вы рабы физики, рабы, столь страшные, что не можете даже понять, как уйти от того. Мы вам когда-то говорили ``попробуйте, попробуйте мечтать не о физическом, а о духовном''. Вспомните. Вы не можете сделать того. Не можете. Ибо вы живёте в этом мире столь глубоко, что трудно, трудно будет уйти вам отсюда. Вы говорите: ``как же, как же мне покинуть это тело'', и говорите: ``если это тело умеет любить''. Даже в одном этом понятии, что ``тело умеет любить'' уже вы отвергаете, что есть любовь, любовь без тела. Вспомните, вы говорили об искусстве. Вы помните? Как можно покинуть этот мир, - тогда не будет искусства. В вашем понятии, в нашем мире нет искусства. А вы подумайте, и проведите хотя бы аналогию в вашем мире, посмотрите на искусство наскальных и посмотрите сейчас. И подумайте: неужели ранее наскальные рисунки не были искусством? Сейчас вы создаёте множество, в вашем понятии, ``культур искусства'', и не можете подумать, что будет, если вы уберёте хотя бы одну верхнюю оболочку. В вашем понятии - не будет ни любви, ни чувств, ни более. А может быть наоборот? Наоборот. Те, кого вы называете …(сбой)}
1-2-3-4-5-6
\soul{Мы говорили вам о телах, и что в каждом теле вы должны умереть. Вспомните. И вспомните, что душа ваша - в одеждах. Многих одеждах. И потому, не видите мир, не видите настоящее, ибо преломляют одежды ваши. Подумайте. Подумайте, что будет душа видеть, видеть то, без ваших ``розовых очков'', или то, без ваших ``одежд''? Она будет видеть то, что есть, то, чём живёте. Сейчас же - живёт в телах ваших, и потому видит тела ваши. Тела же ваши – физические…(сбой)}
1-2-3-4
\soul{И вспомните, вспомните детство ваше, вспомните чувства ваши. Вы же не знали, не знали, о иных мирах. Вы не знали многих понятий, которые знаете сейчас. Вы не знали, что существуют инопланетяне и далее, далее. Что есть множество техники. Подумайте, вы не знали, что такое ``биоэнергетика'' и многое, многое. Но вы же, вы же говорили о том! Говорили. И вы называли это ``детскими фантазиями''. И кто-то, в вашем понятии - ``взрослые'',- пришли и растоптали фантазии ваши. И сейчас, многие из вас, делают то же. И потому губится в вас, и потому, одежды ваши, когда вы родились, были тонки и просвечивали, и можно было увидеть. Душа ваша могла видеть сквозь них. Ибо тела ваши были молоды. Молоды. Но, с годами, в вашем понятии – ``цивилизация'',-  тела ваши твердели, каменели, и всё меньше и меньше видит душа ваша. И когда-то, если не остановитесь, если не поймёте, станете камнем. Камнем. И уже другие, в вашем понятии  ``расы'', придут на Землю ту, и увидят камни, и скажут: ``Вот те камни – предки наши''. Подумайте, когда то была и такая религия. Помните? И будут ходить по вам. Вы же - будете камнем. Бойтесь того! Потому приходят к вам и говорят вам: ``Остановитесь!''. Если б это было …(теряется контакт) (имелось ввиду,  что не приходили бы, если бы это было не так. прим.)}
\people{Хорошо, спасибо за подробный, развёрнутый ответ. Но, вот, как подобная проблема профессии и занятий решается у вас, в вашем обществе?}
\soul{Подумайте, вы говорите о профессиях. В вашем понятии ``профессия'', это физика. У нас нет физики. У нас нет профессий. Подумайте, какие у нас могут быть профессии, если мы не обладаем никакой техникой? Зачем она нам? Если мы можем везде быть. Вспомните. У нас нет психологов, ибо у нас нет памяти. Вы же говорите о психологах, но не ведаете, что психолог ваш – память. Память. Все болезни ваши – память, ибо не помните памятью. Вы же, стараетесь лечить иное. Вы приходите к врачу и говорите то-то и то-то, вас мучает что-то. И вас лечат, лечат физически - не память вашу. Вам было бы проще и надёжней, если бы вас заставили вспомнить, чтобы вы пережили заново, но, сделали правильно. И тогда, не было бы болезни той. Вы же - лечите физически. И потому…(сбой)}
1-2-
\people{Скажите, что, у вас нет профессий? Но есть же у вас великие композиторы, художники, писатели… т.е. люди духовного такого плана.}
\soul{Вы повторяетесь. Мы ответили вам. Да, есть и у нас искусство. Есть.}
\people{Мы как-то с вами контактируем в этом плане, или вы отдельно идёте, мы – совсем отдельно?}
\soul{Как вы можете представить единый мир, в котором бы всё было бы ``отдельно''?}
\people{Т.е. ваше и наше искусство имеет общие точки соприкосновения, да?}
\soul{Поймите, в вашем понятии - искусство, имеет один корень. Подумайте, один корень. Чем вы создаёте, чем? Сознанием? Тогда спросите любого и спросите себя. Вы пишете научную статью о физическом. Cогласны?  Чисто о физическом. И что – пишет сознание ваше? Оно, всего лишь даёт вам знаки. А что дало вам знаки те? Подумайте.}
\people{Нас тревожит вопрос, или как-то волнует - нас, землян, кто-то пытается колонизировать или нет? Или закон не даёт колонизировать? И почему до сих пор не колонизированы?}
\soul{Подумайте, мы говорили вам, о подобных вам. Вы не знаете космических законов. Значит есть  множество подобных вам, которые не знают их тоже. И потому будут приходить, и, как вы говорите - делать вас рабами. А вы посмотрите на себя и подумайте. Посмотрите, вы - рабы техники. Рабы! А теперь подумайте, кто вам дал технику? Вы скажете: ``инопланетяне''. Нет. Вы, вы создали технику. Вам помогли. Вы, своими руками, как вы говорите ``роете могилу''. Своими руками! Зачем вас колонизировать, если вы уже рабы? Вы – рабы. И не страшен рабовладелец, подобный вам. Страшиться вам надо себя, ибо вы рабы себя! Вот самое страшное. Вот вам и ответ. Вы несёте… (теряется контакт)  (ответственность за всё, что с нами происходит. прим.)}
1-
\people{Скажите, вы не можете нам дать хотя бы основные законы космоса. Первый, второй, третий…}
\soul{Вы, хотя бы вспомните свои. Научитесь выполнять свои.}
\people{Они, наверное, близки и к 10ти заповедям… или есть какие-то особые положения?}
\soul{Близки? Вы знаете законы те, знаете. И вы называете это ``чувствами'', ``предчувствиями'' и не можете найти слова им. Но, подумайте, неужели космос ваш говорит словами вашими? Неужели, законы те, написаны вашими словами? Даже у вас на Земле, их - множество. Множество языков. Каким же языком написать, законы ваши?  Есть, есть один язык, когда-то мы говорили, универсальный, - то душа ваша. Душа.}
\people{В наших газетах промелькнуло тревожное сообщение, что якобы одна из монахинь ясновидиц, 8 лет назад сообщила, что на Юпитер упадёт гигантская комета, в результате образуется огненный шар, который уничтожит Землю.  Конечно, её никто не слушал. Мы - люди недоверчивые, земляне. Но, недавно астрономы открыли неизвестную комету, которая несётся в сторону Юпитера. Первая часть предсказания этоq монахини нашла научное подтверждение. Комета, по расчётам учёных, должна столкнуться с Юпитером между 16м и 22 июля … года.}
\soul{Поймите, и проведите аналогию, что - ваш ребёнок, вы обучаете его, и он не понимает вас – что делаете вы? Вы создаёте условия, в которых мог всё-таки понять. Вспомните, как вы учите детей плавать. Вспомните. А теперь, проведите аналогию далее. Но если мы не можем заставить вас видеть то-то или что-то, не можем заставить вас уйти, уйти от первого тела, - может быть можно сделать и так.}
\people{Т.е. это будет насильственное, скажем так, благо?}
\soul{В вашем понятии - насильственное благо. Ибо будет в вашем понятии - сознание. Поймите, ваше сознание – это первый плод, потому, оно будет бороться и кричать, кричать. Вы же, боясь, забыли о душе. Вы говорите: ``всё в Божьих руках''. Вы же говорите. Не мы. И вы же, сами себе противоречите и сами боитесь. }
\people{Значит, вы подтверждаете, что именно на эти числа выпадет этот день?}
\soul{Поймите, поймите, не все из вас погибнут, в вашем понятии. Да, не все. Кто-то уйдёт раньше, кто-то уйдёт позже. Потому и приходят к вам. Приходят к вам и говорят вам: уйдите, уйдите сами. Вы же поймёте - о самоубийстве. Нет, мы говорим чисто о психическом. Мы говорим вам ``уйдите от этого тела, уйдите душой, уйдите''.  Но, не делайте так… Не надо насилия.}
\people{Недавно, под утро, под восход солнца, я был на какой-то планете, где мои одноклассники были. Такой прекрасный мир. И я помахал им рукой и сказал, что ещё вернусь. Это в астральном мире я был или в ментальном?}
\soul{Давайте скажем так: как вы понимаете ``астральный'' и ``ментальный''? Видите ли вы эту разницу? Что для вас значат эти слова? Что? Почему вас интересует, в каком это было мире, если вы не знаете разницы той? Подумайте. Подумайте, как вы спрашиваете. }
\people{Да, возможно я неправильно сказал, но я хотел другое сказать.}
\soul{Подумайте, мы говорили вам о словах ваших, и о значениях их.}
\people{Да.}
\soul{Если вы задаёте вопрос, задаёте словами его, мы будем отвечать вам за слова услышанные, заданные вами словами, но не будем отвечать на мысли ваши.}
\people{Хотя вы и мысли слышите, я так понял, да?}
\soul{Подумайте. Далее, мы говорили о  вас, как о множестве. Но почему же вы не можете даже понять. Даже ваша логика или сознание не отрицает того, что вы можете одновременно быть во множестве, если вы множество. Подумайте. И подумайте: вы идёте по улице и, в вашем понятии, вас что-то тронуло. Быть может, кто-то из вас, в вас, пришёл и рассказал, где был. Но сознание не слышит, ибо нельзя передать словами. Вспомните ответ.(…)}
\people{Да, я понял.}
\soul{… и вы принимаете, как чувство.}
\people{Что-то я не совсем понял. С одной стороны, вы сказали что нам надо сознанием дорасти до всего, значит, осознать всё. Своё подсознание, весь мозг. С другой стороны, вы сказали, что это сознание надо этими чувствами победить, так сказать.}
\soul{Подумайте,  что говорите вы.}
\people{Ну, возможно, не правильно…}
\soul{Подумайте, мы говорили вам. Да, сознание ваше – всего лишь, доли. И потому, мало что понимает. И вспомните, мы когда-то говорили вам, вы спрашивали, что сознание не нужно. Мы же говорили вам: вы должны осознать себя. Осознать всего. В нашем понятии, вы должны быть сознательны. Мы говорили вам, что вы боги. Да, но только не сознанием. Нет, не боги. Подумайте, в сознании ли вашем умение управлять?}
\people{Скажите, в наше время есть или могут ли быть ``звёздные войны'' между цивилизациями?}
\soul{Они были и есть. И вы, многое называете новым или подобным.}
\people{А какие законы позволяют сохранять мир во Вселенной?}
\soul{А вы подумайте. Если мир един, что может в нём быть? Может он разрушить сам себя? Подумайте.}
\people{Для землян, это ``Бог''. А у вас это как?}
\soul{Подумайте, может исчезнуть один индивидуум. Может исчезнуть целые планеты. Могут. Но, подумайте, даже у вас есть закон сохранения. Каждый из вас обладал энергией. Да, она не исчезнет никуда. Она преобразуется, преобразуется в что-то иное. Погибает одна планета – рождается новая. Подумайте. Когда-то кто-то был уничтожен, в вашем понятии, в физическом плане. Благодаря тому, у вас пришла жизнь. Подумайте, подумайте. Да, могут уничтожить и вас. Вам подобные. Значит, где-то, будет новая жизнь, ибо будет, в вашем понятии, свободная энергия. Вам же, надо быть там. Ибо, вы говорите ``бессмертна душа'', мы же говорим вам: да, бессмертна, ничто не может вас уничтожить, только вы сами. И подумайте далее. Подумайте и проведите аналогию. Вы же, вами же сказано, что созданы вы по подобию божьему. Подумайте, значит, вы подобны ему. Значит, вы обладаете энергией. Энергией Бога. Но, вы не умеет управлять. Вы – хаос. И в этом хаосе вы можете плодить и хорошее и плохое, ибо - хаос. И не будете знать, что сделали.}
\people{Вы считаете, что Библия дана нам, чтобы научить, как сохранить и спасти свою душу. Это так?}
\soul{Подумайте.  Подумайте. Да. Но вы, делаете ``шаг в сторону''. Вы говорите что ``да'',- Библия – ``да'', Коран – ``нет'', и далее. В том ошибка ваша. Вспомните, мы говорили вам, о врагах. Если вы будете использовать методы врагов против врагов, значит, враги победили вас!}
\people{На эту тему вот такой вопрос недавно человек задал, - `` А как быть, если в тебя целятся, и вот если ты его убьёшь – то это зло или метод самозащиты просто''?}
\soul{Чаще - беда ваша, ваша вина. Подумайте, ``подобное - к подобному''. }
\people{Поэтому и заповедь есть: ``не противься злому'', да?}
\soul{Вспомните, мы же говорили вам об оружии, о силе зла. Вспомните. Подобное – подобному. Мы говорили то же, то же, но другими словами. Подумайте, если вы будете использовать методы врага, чем вы лучше него? Вы можете объяснить мне? Чем? Вы скажете: ``цель иная''?  Нет. Тогда получается - ``цель оправдывает средства''. Подумайте, у вас доли, доли сознания, но достаточно, чтобы сделать выводы самим. Подумайте. Здесь не нужно каких-либо знаний. Здесь нужно всего лишь уметь логически думать, сопоставлять.}
\people{На данном этапе нашего развития лучше воспринимать веру и всё, что духовно преподносится так, как оно есть, чем вообще быть неверующим?}
\soul{Подумайте, дело не в вере, а в том, как живёте вы. Подумайте, ребёнок рождается ребёнком. Вы говорите, он безгрешен. Но, ведь он же не верит, он же не знает ещё о верах ваших. Подумайте. А он - безгрешен.}
\people{Вы сказали: одни уйдут рано, другие – поздно в связи с этим метеоритом. }
\soul{Мы не говорили вам ``в связи с метеоритом''. Мы говорили вам, как один из способов, что рано или поздно, могут сделать и так. Мы говорили вам ``подобное … (имелось ввиду выражение :”подобное притягивается подобным''. прим.)}
\soul{…Вы же говорили о ``звёздных войнах''.}
\people{Понятно. }
\soul{Далее. Мы же говорим вам…}
1-2-
\people{Чем объясняются перерывы переводчика?}
\soul{Вами. Вами. Физикой. Сознание ваше. Ибо часто, вспоминая ваше сознание, оно слышит и откликается.}
\people{Скажите, переводчик правильно чувствует, что необходим сейчас перерыв в наших беседах?}
\soul{Разве чувствует?}
\people{Ну, так, вроде, склоняется к тому, что надо прерваться.}
\soul{Мы говорили вам, что это ваши проблемы. Мы приходим, когда зовёте. Но, да, будет время, когда мы скажем вам ``прощайте, прощайте''.  Мы придём к вам только тогда, когда услышите. Вы же не слышите. Вы же, кто каждый из вас, и много раз, разговаривал с нами, вы забыли и не слышите. Лично вы, когда-то, ``легли'' в вашем понятии, и увидели ``прошлое''. Вы видели многое, многое. Сознание ваше, испугавшись, сказало себе: ``Нет, нет, не ляжу более''. Вы, в той жизни, ложились, ибо не знали, ибо не знали, что уже дали, в вашем понятии, себе ``установку''. Вы пришли в эту жизнь и уже не можете лечь, ибо вы защищаетесь, защищаетесь, не зная ничего. Вы говорите, вампиры приходят к вам, сосут энергию вашу. Но подумайте, то же самое вы можете сказать и о святом, который придёт к вам. Ведь он заберёт у вас ваше тёмное, чёрное. Вы ощутите пустоту и скажете ``отобрал, отобрал -  вампир''. Подумайте, чаще, чаще вы делаете так: вы, почувствовав что убирают от вас, вы не разбираетесь плохое/хорошее, вы просто говорите: ``вампир''.}
\people{Были ли аварии НЛО на Земле или это выдумки?}
\soul{Много. И мы говорили вам и отвечали более конкретно. Не помните ранее?}
\people{Но обломки хранятся где-то в Америке или у нас может быть, в России?}
\soul{А вы подумайте. Ничто не исчезает бесследно.}
\people{Я, вот, прямой вопрос хочу задать: есть ли в Актюбинске такое хранилище, лаборатория такая?}
\soul{Мы не будем отвечать на подобные вопросы. Подумайте, подумайте, что движет вами? Почему создаёте, почему пишите, почему ищете? Сомнения ваши, сомнения. Но, подумайте, если получите доказательства всего, что вы будете делать? Вы перестанете. Вы скажете: ``Да, это есть. Зачем мне более?'' Поймите, если мы скажем ``нет того'', вы скажете: ``если нет, зачем я тогда это делаю, если нет того?''. Подумайте, вами, вами движут сомнения. Подумайте, если бы мы не сомневались, многое б не делали. Многое. И потому говорим вам: вы не машины, вы не машины, ибо вы умеете, умеете ошибаться, умеете сомневаться. И бойтесь того времени, когда не будет никаких сомнений вам. Бойтесь того времени, когда не будет чувств вам. И тогда, вы не будете совершать ошибки. Тогда,  вы – просто ``машины''. }
\people{(Белимов) - Да, я это чувствую интуитивно. Скажите мне вот так вот, по-дружески, стоит ли мне пробиваться в Актюбинск?}
\soul{Мы вам ответили только что.  Много - написано вами. Подумайте. И подумайте,- даже во лжи есть правда. Сказано вами, многое, многое, что слышано и писано вами – ложь, в вашем понятии. Ложь. Но - ложь в этом мире. Подумайте,- мысль материальна. И что бы вы ни подумали, что бы ни сочинили, а мы как-то говорили вам, вы не умеете фантазировать, но мы вам и говорили, что вы творите множество миров. Подумайте, если пришли к вам и сказали: ``я был там-то и там-то''. Да, в этом мире он солгал. В другом, как угодно назовите его: мыслеформы и далее,- он был там! Ибо, он, считая, что сочинял, увидел. Увидел. Но, сознание ваше знает, что не было того. Не было того в этом  мире.  Потому и говорим вам: вспомните о лжи. Вспомните.(счёт) }
1-2-3
\soul{Вами же сказано: ``станьте детьми''. Вспомните. Вспомните, к вам приходили и говорили. Если будете внимательны, то вы найдёте слова те не только Христа, но и ранее. И подумайте, и будьте внимательны. Вы говорите: в каждом, в каждом есть всё. Но не видите того. Будьте внимательны. Будьте! Аналогии вам приводили только что.  Вы спрашивали, мы отвечали вам о детях.  Подумайте, даже здесь, даже здесь есть многое, многое, как вы говорите, `` космическое''. Спрашивайте.}
\people{Скажите, как у вас решена или решается проблема добра и зла, проблема преступности. Вы покончили с этими понятиями?}
\soul{Нет. Мы когда-то говорили, что мы подобны вам, но только ``выше'' вас и не имеющие множества тел ваших, только и всего.  И вспомните, мы говорили вам, вы спрашивали о преступниках, в психических планах. Вы помните? И вспомните, что было и что спрашивали - начало. Вот вам, и преступники, вот вам, и добродетель. }
\people{Т.е. у вас есть такие же понятия?}
\people{Каковы мы - такие и они.}
\soul{Подумайте, вы спрашиваете первый вопрос, и мы отвечали вам: вот и  преступники, вот и добродетели. Подумайте, но не называйте, не говорите: ``Я слепой''. То ошибка ваша, ошибка. Это реально, это реально даже более, чем вы можете представить. О подобных – возьмите и оглянитесь, посмотрите на себя.}
\people{А если, представить всё - реально, но в других мирах. Вы говорите ``предсказали будущее'', к вам приходят и говорят. И – случается, или нет. И не думаете, что в другом мире это может быть случилось или наоборот.  Подумайте, вы приходите, и вы видите множество миров. Множество. Но не знаете об том и говорите, как об одном, здесь. Вот вам и ошибки предсказаний. Вот вам и совпадения. И, быть может, может каждый из вас, уже множество раз перешёл в миры параллельные, не заметив того. Подумайте.}
\people{Вы говорите, чтобы мы прислушивались больше к своей интуиции, к внутреннему голосу, вашему шёпоту. А если это не всё совпадает с вашими словами во время этих контактов, кому больше доверять, интуиции или вам?}
\soul{Доверяйте себе. Доверяйте себе, ибо, если мы скажем: ``доверяйте нам или кому-то другому'', не будет успеха. Всё равно не будете доверять.}
\people{А вообще, можно ли покончить с преступностью?}
\soul{Да. }
\people{Но, может, не с чем будет сравнивать, и мир не будет развиваться?}
\soul{Подумайте. Возьмите и прекратите. Прекратите быть преступником сами. Вспомните, вспомните себя. Посмотрите на себя, сколько в вашем понятии, сделали преступлений, и грешите. Спросите себя. А потом, когда вы будете истинно святым, даже в вашем понятии, тогда, возьмитесь за других. Вы же, вы же будучи грязны, хотите очистить других. Вам же было сказано об энергии. Помните?}
\people{Хорошо. Как вы наказываете своих оступившихся особей, если таковые есть?}
\soul{Многие приходят к вам. Но, из выдающихся - тоже приходят к вам. Поймите, чем больше энергии, тем жёстче экзамен, тем жёстче требования. И, порой, в вашем понятии, ``герой'' и ``преступник'' встречаются в одном мире, ибо ``преступник'' пришёл, как наказание, ``герой'', в вашем понятии, пришёл в тот же мир - для испытания.}
\people{Возможно ли в отдалённом будущем, если каждый будет совершенствоваться, жизнь землян без преступности вообще? }
\soul{Это было б прекрасное время. Прекрасное. Но, всегда, в вашем понятии, всегда, пока вы не поймёте о едином - будут в вашем понятии ``преступники''. Подумайте. Пока существует личное ``я'', ``я'' возомнившее себя - будут преступления. Подумайте. Даже преступник кричащий, что он совершает во имя мира… Нет. Он совершает во имя себя. Пока ваше ``Я'', собственное ``Я'', будет выше иного – будет преступность.}
 
Конец 1-й кассеты 
Начало второй  кассеты (Без Белимова)
\people{(Г) Скажите, можно задавать вопросы по прошлому или нежелательно?}
\soul{Нежелательно, но вы можете задать.}
1-2-3…
\people{(Г) Скажите… Вот, девушка… Вы сказали, что она была няней. Я так понял это - моей няней. Правильно ли я понял?}
\soul{Нет.}
\people{(Г)  Нет… У нас там возникли разногласия насчёт одного слова, сказанного вами… Так и не поняли…}
\people{(Ш) Чьёй же она няней была?}
\people{(Г)  Да.}
\soul{А вы подумайте. В прошлой жизни, двое из вас были ``во дворе'' и один из вас, как вы говорите  ``при дворе''. Подумайте.}
\people{(Ш) ``При дворе'' монах был? Нет?}
\soul{Нет, в то время не было монахов.}
\people{(Ш)Но это был он?}
\people{(Г)То же лицо?}
\soul{Да.}
\people{(Г)Хорошо.}
\soul{Далее. Мы говорили вам о ``ранее”… И подумайте, няней могли иметь только в богатых сословиях.}
\people{(Г)Да. Следовательно, двое из нас были, скажем так, богатыми,  состоятельными людьми…}
\people{(Ш)Значит, это была няня не моя.}
\soul{Мы не говорили вам, что няня была ваших детей. Вы были знакомы. Будьте внимательны, мы говорили вам о тесноте. (тесноте кармических связей. Прим.)}
\people{(Г) Ага… Ну, я так понял, всё это, что в то время было в Петербурге, сейчас все люди, так сказать, перемещаются почему-то вот, к Волге… Это не случайно ведь? Явно? Да? Поволжье?}
\soul{Во-первых, мы не говорили вам о Петербурге. Далее. Вы опять себя возвеличили. В вашем понятии – Поволжье? … Спрашивайте далее. }
\people{(Г)Тогда, не понятно, в каком ``дворе'' мы были…}
\soul{Думайте.}
\people{(Ш)Но, это может быть и скотный двор…}
\people{(Г)Да…}
\people{(А)Вы ж самостоятельно по нему гуляли – вам говорили …}
\soul{Думайте.}
\people{(А) Не обязательно царский двор.}
\soul{Спрашивайте далее.}
\people{(Ш)Боярский двор. (как вариант. Прим.)}
\people{(Г) Скажите, вот наш ушедший товарищ хотел про Луну задать вопрос. (Б. ушёл. Прим.) Что вы максимум можете сказать о ней?}
\soul{Спрашивайте.}
\people{(Г)Вы сказали как-то, что там жизнь кишит, мы просто не видим. То есть, наше сознание не допускает, ставит преграды, чтоб мы её видели. Я правильно понял?}
\soul{Вы поняли правильно.}
\people{(Ш) Но, а есть же такая… вид жизни, где либо в солнечной системе, которую мы можем увидеть? }
\soul{Есть. И вы её увидите даже на Луне.}
\people{(А) А вот сейчас, чтобы увидели нашим…? (обычным зрением, прим.)}
\soul{Вы были там?}
\people{(А) Не знаю… Я не знаю. Непонятный вопрос…}
\people{(Г) А биологическая?}
\soul{Вы, все - были ли вы там? (уточнение вопроса на не понятый Анной. Прим)}
\people{(Г) На Луне? Да нет… Я не помню.}
\people{(Ш) Мы видали фотографии, которые сделаны были на Луне.}
\people{(А) Мы видим её отсюда.}
\soul{Все-все – были на Луне? (имели ввиду все наши тонкие тела наверное. Прим.) Или вы считаете, что если вы сделали несколько шагов - вы знаете всё? Простите, у вас есть множество, где могут спуститься, сделать несколько шагов и сказать, что у вас на Земле нет жизни.}
\people{(Г)Так, ещё такой вопрос, насчёт сознания и инстинктов… Что-то я не понял… Сознание было когда-то тоже инстинктом? Просто оно возвеличило себя что ли?}
\soul{Нет. Сознание всегда было сознанием, инстинкт был всегда инстинктом. Разница только в том,- и будьте внимательны, мы вам говорили о пятнадцати тысяч (лет назад. Прим.) и дали понять вам, что ранее вы пришли и знали всё… }
\people{(Г)Да, это так.}
\soul{Подумайте. Сознание ваше – было могуче. Гораздо! Вы пришли в этот мир. Вы пришли в этот мир и стали жить в нём по его правилам. В его же правилах, в его природе – сознание было слабым, и вы – потеряли. Вы это называете ``потерять опыт'', ``потерять навык''.}
\people{(Г) Угу.}
\people{(Ш) Можно сказать, что теперь сознание и подсознание поменялись своими местами?}
\soul{Нет.}
\people{(Г)Но, раз мы… Раз так мир этот устроен, что, живя здесь, сознание мало…}
\soul{Вы пришли, чтоб изменить его.}
\people{(Г) А-а! }
\soul{Мир изменил вас.}
\people{(Г) Видать, мы слабы были… Вот.}
\people{(Ш) Наверно, это невозможно сделать.}
\soul{Мы когда-то говорили вам, что мир изменил вас. Вы же – приходите в другие миры, изменяя по вашим мерам. Вы говорите: ``наш бог”… Одно только это понятие – и вы делите мир на множество. Вы говорите ``наш бог'', значит, есть и другие боги. Потому вам и множество, потому и войны ваши.}
\people{(Г) Да…}
\soul{Далее… Вы, куда бы ни пришли, куда бы вы не придёте, всегда будете мерять мир через себя. Вы мир видите через себя. Через ваше сознание, не через чувства… Те же, кто видит через чувства – вы их называете ``сумашедшими'' и создаёте лечебницы. Далее… Вы, придя сознанием, которое всего лишь доли (%) много ли увидите в мире том? И будете менять его. Будете менять его потому, что вам не будет нравиться многое. Вы не будете знать что и к чему. Вы же не знаете мир, сознание ваше не знает того, потому что не знает причин. Вот вам – проблемы иные. Вы их сейчас называете ``экологией''. Подумайте и сопоставьте. Куда б вы ни пришли – вы будете менять, менять по своему, вместо того, что б прийти и жить. Просто жить.}
\people{(Г)  Ну, а если мы придём в тот мир, где другие меняют и, так сказать, нам это  не понравится?}
\soul{Но вы и пришли в тот мир, где меняют другие, и изменились сами. Изменились в другую сторону. Будьте логичны. Мы же сказали вам, и только что. Только что сказали.}
\people{(Г)То есть, надо быть самим собою во всех мирах? В смысле,  как… че-ло-веческим.}
\soul{Будьте самим собой. Будьте. Вы же – носите множество одежд. Множество ``масок''. Вы, всегда, каждый миг – ``другой''. Вы меняетесь, и меняетесь хаотично, ибо – игры… Игры – зависят от обстоятельств. И потому, кто вы? - Марионетка, которая подчиняется хозяину обстоятельства.}
\people{(Г) Да… Обстоятельства  сильнее нас пока…}
\people{(Ш) Скажите, вот вы насчёт ``Поволжья'' так немножко иронично сказали… Не следовало ли нам сейчас подумать о том, что тот двор, который вы упоминали, был тоже на Волге?}
\soul{Подумайте. Мы вам говорим - Поволжье, не о (…) . Далее… Вспомните. Вспомните Христа и искушение дьявола… Он тоже пришёл, и тоже сказал, - Христу, не вам,- о том, что он единый и могучий и этим хотел соблазнить его. Неужели вы думаете, что если вы будете говорить, что вы, только вы правы… и только вы, только вы  можете изменить мир… Измените ли? Нет, мир изменит вас - сломает или ещё хуже.}
\people{(Ш) Но, я хотел просто уточнить местонахождение ``двора''. Он был тоже на Поволжье?}
\soul{Ищите.}
\people{(Г) А вот ещё такой вопрос… Мне не так давно попались стихи одной девушки, жившей тоже в Поволжье… Вот… Случайно они мне, вроде как, пришли, но случая в жизни нет – я хотел бы знать связь. Почему они мне попали ко мне? Переводчик знает, о чём речь.}
\soul{Тогда, подумайте, подумайте и спросите собеседницу вашу, знает ли что либо она, слышала ли что-то либо ранее, не вспоминала ли что?}
\people{(Г)Тут, рядом сидящую?………  Молчание – знак согласия , вообще-то. Ну, а если она не вспоминала?}
\soul{Пусть вспомнит. Но, не утешитесь – не её.}
\people{(Г)Я и так понял, что это не её. А какая связь со мной, я не понимаю…}
\soul{Простите, вы всегда и везде хотите видеть связь? И во всём?}
\people{(Г)Ничего ж случайного не бывает, вы же сами сказали.}
\soul{Тогда и найдите сами, если ничего не бывает случайного. Вы знаете этот закон. Что ж, исполняйте.}
\people{(Г) Отталкиваться же надо от чего-то…}
\soul{Отталкиваться? Вам дано многое, и вы оттолкнулись? Сдвинулись вы?}
\people{(Г) Скажите, а вот, имя той, которая писала, и имя нынешней…}
\soul{Спрашивайте далее.}
\people{(Г) Спасибо.}
\people{(Ш)Я до сих пор не могу это понять точную грань… Ну, она, конечно не существует – точная грань… И всё-таки… Где граница между предсказанием будущего и грань, как вы говорите, кодировки, зашифровки… ``’это вы рабами становитесь предсказанного'' (цитата высказанная на прошлом контакте. Прим.)}
\soul{Чаще всего, вы исполняете по той причине, что любое предсказание является кодировкой, в вашем понятии. И чем сильнее, в вашем понятии, предсказатель, чем он именитее, тем успешней его предсказание. Тем успешнее. Подумайте. Если его знаете только вы и больше никто – он может только изменить вас. И то, если вы будете верить. Если быть точнее, то не предсказатель предсказывает вам - вы даёте себе установку, по словам предсказателя. И подумайте, если предсказатель не является вам авторитетом, много ли совпадёт? Подумайте и вспомните, были у вас примеры. И теперь представьте, если придёт иной, более могучий, много знающий… Подумайте, если он скажет вам и другим… У вас были ``предсказатели'' – вы называли их ``вождями''.}
\people{(Ш) Мы называли их ещё и ``пророками''.}
\soul{Вот и подумайте.}
\people{(Ш) Так, получается…}
\soul{Подумайте! ``Вожди'' ваши – ваши вожди. Вы их создали! Не мы. Согласитесь, вы бы могли менять их, если бы хотели. Если бы вы…(сбой контакта. Прим.)}
1-2-3
\people{(Г)Скажите, а где грань между ``установкой'' и ``действительно будущим'', так сказать, которое уже в будущем случилось и просто человек знает об этом?}
\soul{Давайте скажем так – Вам скажут то-то и то-то, но вы можете придти туда множеством путей. Множеством. Подумайте.}
\people{(Г) Верно.}
\soul{Эти пути - уже выбираете вы сами. Далее. Мы говорим вам, о параллельности. Не все же предсказания сбываются, даже у ``великих''. Подумайте.}
\people{(Г) Да.}
\soul{Может быть, просто не была названа ваша параллель, а иная? Далее. Чаще – предсказание ваше, всего лишь кодирование. Кодирование вами. Подумайте. Кто-то придёт, и если будет заставлять силой вас делать что-то, то вы не будете делать того, ибо горды. Но можно сделать проще и хитрее – придти, намекнуть, и далее… дело пускается так, что вы будете вынуждены сделать то, что не сделаете никогда по принуждению.}
\people{(Г) Да-да, есть такое – психологическое давление.}
\people{(Ш)Так, значит нам не стоит всерьёз относиться к пророчествам библейским? К тому же – о конце света, в Откровении…}
\people{(Г)…Иоанна Богослова.}
\people{(Ш)Так что?}
\soul{Не стоит. Будьте внимательны. Мы вам сказали… Мы сказали: ``Станьте подобны богу.'' Подумайте. Вспомните. Вспомните ваше библейское. И вам же говорят о конце света. В том говорят, как о судьбе. Вы же принимаете - впрямую. В вашем понятии действительно будет конец света? Физически? Поймите, каждый из вас… Каждый из вас умирает и рождается в ЭТОЙ жизни - много раз! }
\people{(Ш) В моём представлении, конец света идёт, как раз, с представлением, о новом этапе эволюции человека.}
\people{(Г) Человечества.}
\soul{Давайте так – у каждого из вас свой срок. Подумайте. Подумайте. Мы говорим о боге, как о силе добра, и тут же говорим, что бог принесёт зло. Зло в ``конце света'' и будет суд ваш. Здесь вы противоречите себе. Здесь нет вашей логики. Нет. Почему… Почему? Если будете внимательны и сможете пережить всё, что сказано, всё, что слышано вами, - вы найдёте, где ложь и где правда. Подумайте, мы говорили вам – библия ваша писана вами – людьми, и потому искажена. И есть другие силы, более могучее вас, и они ``искажают'' вас. Что движет вашей рукой? – Мысль. Вы же не знаете начало мысли и потому не можете управлять этим началом. Кто-то из вас приходит и искажает вас. Подумайте. Так же, многое написано  искажённо.}
\people{(Ш)Ну, а кто ж тогда диктовал мысли пророкам? Даниилу, там, или Матфею… Да кому угодно.}
\soul{Вы говорите о пророках. Вы называете имена – Моисей.. и далее… Так почему же тогда, когда пришёл Христос, никто не понял? Подумайте. Ибо ждали, ждали так, ``как написано'', впрямую  – `` во славе'' и далее.. Пришёл к вам плотник и не признали его…  И придёт к вам другой, придёт  к вам Христос и вы опять не узнаете его, ибо не будет совпадать по вашим понятиям. Далее. Вы говорили ``иносказательно”… Вы говорите ``иносказательно'', а понимете буквально…Как вы тогда найдете? Как вы найдёте спасителя вашего? Подумайте, и вспомните историю Христа. Вспомните! Многие не признали его, ибо у многих – не сбылось предсказание о Христе. Подумайте.}
\people{(Ш)Но вы уверены, что этим пророкам не являлись никакие видения, ни какие-то голоса или шёпот, как вы говорите, не было контакта, как сейчас происходит?}
\soul{Как вы можете… Как вы можете говорить так?! Тогда вспомните, вспомните, как был писан Коран…Вспомните. Разве видел он бога? Он только слышал. Подумайте. Тогда, в вашем понятии, не существует снов, или сны теперь -  это игра вашего мозга. Подумайте. Да, здесь вы правы, это игра вашего мозга… Но правила той игры – не ваши.}
\people{(Ш)Но вы постоянно говорите, что ``вами написана библия''. Я как раз привёл пример, что не нами она была написана, а с чей-то помощью всё-таки.}
\soul{Да. Но вы не внимательны, мы говорили вам о ``чёрных'' и ``белых''. Подумайте. Вы говорите о боге о добром, и тут же говорите о боге злом, о карающем. Подумайте. Вы говорите: ``Уверуйте, уверуйте в меня и тогда будет вам Царствие небесное. Иные же – не будут там.'' Может это ваш эгоизм и не более? И вы дали эгоизм свой богу, очеловечили его своими чувствами. Подумайте.}
\people{(Ш)Ну, дайте ключ к разгадке тех моментов, где было продиктовано ``свыше'' именно и где ``человеческое'' что-то, низменное.}
\soul{Простите, что вы просите? Что выпросите, чтоб мы залезли в душу вашу и стали копошиться там? Что вы делаете? Вы никому не должны позволять делать то. Никому. Вы должны разобраться. Вы! Кто бы не пришёл к вам,- если не готовы,- он не убедит вас. Подумайте. Подумайте,  зерно растёт в подготовленной почве – в иной не будет расти. И если вы говорите ``я стал веровать в бога”… Если вы… Если… Поймите, если бы не была подготовлена почва та – ничто, ничто не заставило бы вас веровать, даже сам бог. }
\people{(Г)Можно ли сказать так,- вы говорили ``во снах вы не помните, где вы бываете'',- получается, душа - она летает по разным телам, когда ``из этого тела'', так сказать..?}
\soul{Давайте, скажем так… Она гуляет по миру. По миру, где нет понятия о времени, и потому она может уйти и в ``прошлое'', и в ``настоящее'', и в ``будущее''. Она гуляет, не зная понятия ``расстояния''. И потому, для неё нет ``вперёд”/”назад'' и ``вверх”/ ``вниз''. Она просто ``гуляет''. И  видит иное, совершенно иное чем ваше сознание, чем ваши глаза и далее. То – мозг ваш. Мозг строит картины подобные.(подобные тому, что видит душа.прим.) Как вы делаете…Приведите  аналогию – можете вы сказать, что такое любовь, словами?  И скажете ли мне? И пойму ли я вас? И поймут ли вас другие? Нет, вы не скажете, - вы будете подбирать слова, и это будет всего лишь аналогия, грубая копия, и вы будете знать это. И подумайте, - одни и те же слова – ``я люблю вас'' вы можете сказать любому, и лишь только один вас поймёт. Подумайте, понимает он слова ваши или чувства ваши? }
\people{(Г)Понятно, что человек живёт в мире чувств, а мозг рисует ему ``картинку”… А как, вот так вот, ``назад в детство'' вернуться всё-таки?}
\soul{Назад в детство… А вы подумайте. Вы приходите и  говорите `` Я люблю вас''. И если вы готовы опошлить – то будут всего лишь слова. Подумайте и возьмите пример влюбленного. Возьмите и подумайте, что двигает вами в то время? Что? Ничего! ``Ничего'' и сознание ваше – оттого и ``ребёнок''! Вы же, говоря слова те, включаете сознание, ищите слова, которые могли бы подойти лучше. Вам кажется, что так будет сказано лучше и лучше… Подумайте. И возьмите - ребёнок не знает слов ваших, - подумайте, - как же он думает? Как же он думает, подумайте… как Вы думаете? - словами. Словами! И в вашем понятии -  ребёнок не умеет думать, ибо не знает ваших слов… Тогда, как думает животное? Неужели собака думает только ``гав-гав'' в вашем понятии?}
\people{(Г)Ну, тогда зачем нам даны язык, глаза, уши? Вообще - тело зачем нам, получается? - Чтобы жить!}
\soul{А мы  и говорим вам – ``обострите''. Ваши глаза могут увидеть больше, гораздо больше. Ваши уши могут слышать гораздо более, чем вы слышите сейчас. Ваше тело слышит всё! Ваши глаза, ваши уши – тело! Вот и подумайте.}
\people{(Ш)Но вы иногда говорите, что если ``вы не можете что-то делать, то это не значит, что это нельзя сделать''. Ребёнок грудной, который не знает, что такое ``горячее'' и ``холодное'' – он возьмётся за раскаленную сковородку, но всё равно ожёг будет - он же не знал, что она горячая.}
\people{(Г)Да, что этого можно бояться.}
\soul{Потому и говорим вам, что и сознание нужно. Сознание, но – сознающее всё. Неужели вы слова Христа о детях понимаете так буквально, что должны даже сознанием стать ``дитём''? Чувствами, чувствами вы должны стать как дети, сознанием - как глубокий старик.}
\people{(Г)Но для этого, наверно, надо и жизнь пройти, чтобы…}
\soul{Пройти или прожить? Чаще - вы проходите или пробегаете, потому ничего и не видите и не слышите. А если что и было слышано, то забыто или искаверкано. }
1-2-3-4-5-6
\soul{Спрашивайте.}
\people{(Г)Скажите, как вы относитесь, вообще, к тому, что вот, бытовые проблемы просто не дают ``глаз открыть'', потому что всё время надо кого-то кормить… с чем-то… Уходишь в проблемы и …}
\soul{Мы говорили вам в начале, что вы – рабы обстоятельств и чаще, те обстоятельства создаёте вы. Множество из них - создаёте вы. Вы создаёте себе трудности, чтоб потом с успехом их преодолевать. И гордитесь, когда преодолели, хотя было бы проще – не создавать их. Вы гордитесь ушедшим. Ушедшим от жизни. Говорите ``монахи'' и далее…Гордитесь ими, ибо они ушли и не мучаются вами и презираете тех, кто рядом с вами, кто мучается рядом с вами, кто живет рядом, помогает жить вам - тех вы не видите, тех вы не слышите. }
\people{(Г)Ну, всё-таки, я не то хотел спросить…}
1-2-3
\soul{Вы говорите ``хочу многое новое”… Что вы хотите? Чаще всего вы ищете себе оправдание. Вы боитесь согрешить, показаться плохим и ищете оправдание. Многое – больше, больше чем вы думаете, - вы всего лишь оправдываетесь. Оправдываетесь, и не более. И потому, создаёте себе обстоятельства, чтоб оправдать, оправдать свое предательство.}
\people{(Г)А если не ``предательство'', то опять ``растворишься в проблемах'', чтобы, так сказать, других тянуть…}
\soul{Что поняли вы? Вы говорите ``других тянуть”… Вы не можете вытянуть себя и говорите о других. А быть может, чаще, вы ``топите'' других, а не ``тянете''? Вы предаёте себя, а значит и многих других, и ищете себе оправдание, множество оправданий. Вы, любовь воспринимаете, как попытку оправдаться. Вы лжёте себе всегда. Да, бывает очень редко, когда вы признаетесь, что вы врёте, что вы лживы, но вы быстро забываете. }
\people{(А) Ну, а как же… Как же определять в самом себе эту ложь, эту правду? Самому себе.}
\soul{У вас есть сердце. У вас есть чувства. Что вы делаете с чувствами? Что? Вы их давите, давите ``словами'', давите ``обстоятельствами''. Вы боитесь их раскрыть, чтобы не поранили вас и создаете, создаёте множество, множество ``брони'', чтобы защитить себя. Становитесь более равнодушным и далее и далее… или агрессивными. Подумайте. Посмотрите, что вы делаете - вы говорите ``создаю защиту'' – вот вам. Тем более сейчас. Подумайте. Вы боитесь, что придут и натопчут вам. Хозяйка, призывающая гостей, - она же открывает двери и не думает, что потом убирать придётся? Подумайте. Вы же, чаще, - приоткроете двери, - а вам ``дует''. И вы боитесь, боитесь и закрываетесь поглубже. А вы говорите ``чувства!'' Открыли ли вы сердце, чтобы говорить так? Открыли ли вы чувства? - Нет. Чаще - вы выбираете слова, чтоб передать свои чувства.}
\people{(А) Как же передавать чувства?}
\soul{А вы подумайте. Подумайте, мы говорили вам о влюбленных. Мы говорили и приводили пример - ``я вас люблю''. Пройдите, скажите многим эту фразу! Поймут ли вас? Нет. И только один, услышав от вас  то же слово, ту же фразу - поймёт вас и будет любовь. Вот и подумайте, в чём разница, если были одни слова…}
\people{(Г)А если почему-то любовь проходит… Это значит что? Мы ставим защиту?}
\soul{Любовь проходит? А быть может не было любви той? Быть может всего лишь был ``сквозняк''?}
\people{(Г)То есть – открыли дверь, хлопнули и закрыли…}
 
\soul{Нет.  Когда вы откроете полностью, любовь не уйдёт. Любовь не может уйти. Не может. Поймите. Всё остальное – всего лишь влечение. Если вы говорите ``ушла любовь'', это говорит ваш эгоизм, это вы просто нашли ещё один повод оправдаться.}
\people{(Г)Да, но у нас есть такие понятия, как о ``морали''.}
\soul{Мораль и любовь, - что говорите вы? Что?}
\people{(Г)Я говорю о тех законах, которые существуют здесь, пока.}
\soul{Что вы называете ``законами''? А Законы Любви вы забыли, а о ``морали'' помните! Что такое мораль? Это всего лишь способ жить. Жить в этом мире. У вас есть хорошая пословица - ``в волчьей стае по-волчьи выть'', - вот ваша мораль.}
\people{(Г)Верно. Пойти против этих законов - значит быть изгоем общества. И куда идти?}
\soul{Что вы боитесь? Вы боитесь, что обстоятельства эти будут бить вас. Страх вас водит, страх. Вы хотите жаждать… Вы хотите попасть в рай, но как-нибудь полегче и желательно на лифте!}
\people{(А)Скажите, а мы прошлый раз спрашивали про цвета, вы  сказали, что у меня синий цвет, холодный и почему-то сказали ``искусственный''. Вот, как это понять?}
\soul{Поймите. Очень многие очень многое вы получаете с рождения. Очень многое вы приобретаете живя. И если вам что-то не хватает, вы создаёте иллюзии. В этих иллюзиях купаетесь и получаете иные. То - ваши мечты. Вы желаете быть такой-то и такой-то. Но, где-то в глубине души вы знаете, что это ложь, и вы не такие. И вы создаёте, вы создаёте множество… Подумайте. Вы находите,- а ищущий всегда найдёт, желающий всегда увидит то, что хочет… Мы ответили вам?}
\people{(А)Да.}
\people{(Ш)Вы как-то раз сказали, чтоб я конкретно обратился к известным нам экстрасенсам.}
\soul{Обращайтесь ли вы ко всем известным вам экстрасенсам? Далее. Вы хотите спросить, что обозначает он? (символ разбитого креста. прим) А вы подумайте. Подумайте о прошлой жизни… Подумайте, - монах, шаман… Мы говорили вам, что это одно и то же. }
Вы можете представить священника - шаманом? Вы можете представить себе?
1-2-3-4
\people{(Ш)У меня мало известных экстрасенсов, поэтому я практически ко всем обратился, кого знал. Вы сказали, что у меня должен был быть символ ``крест'', никто из них не увидел его.}
\soul{Ищите. Ищите, и кто скажет вам… И будьте внимательны, подумайте, если человек скажет вам то, что вы услышали, -человек, значит, знает вас более, чем иные, чем другие. Далее. Вы хотели знать, что это значит? – Искажённая вера. Понятие желания веровать, но когда-то… И вспомните, мы говорили вам, что когда-то вы пришли в церковь молиться и через 20 лет пришли, чтобы разрушить её.}
\people{(Ш)Вы говорите ``искаженная вера'', но вы сами сейчас искажаете веру, которую мы понимаем по библии.}
\soul{Как мы можем исказить вашу веру, если она искаженная уже? Мы исказим лучше или хуже? Лучше или худше? Подумайте. Вы исказили, пришли и исказили мы… К лучшему или к худшему? Ближе вы стали или нет? Далее. Мы – исказили веру вашу? А во что веруете вы? Во что? Вы себе веруете более, чем другим.}
\people{(Ш)Но вы сегодня на контакте так и сказали, что надо так и поступать - верить себе больше.}
\soul{Да, но вы забыли о ``сердце'' и о ``чувствах''. Зато хорошо помните о сознании.}
\people{(Г)Да, нам трудно определить, где от  чего идёт. Мне кажется, что весьма возможно, что если мы не можем объяснить, откуда это идёт, это из сердца.}
\soul{Разве?}
\people{(Ш)Из подсознания?}
\soul{У вас есть множество других, которые могут придти и сказать. И многие из них, в вашем понятии, черны. А вы подумайте. Подумайте – тогда многие, совершающие убийство – в невменяемом состоянии. Что, сердце им подсказало сделать то?}
\people{(Г)Инстинкт самосохранения. Есть, кстати, такой?}
\soul{Да, инстинкт самосохранения?  Многие убивают не из-за того, что ``опасность''.}
1-2-3-4-5-6-7-8-9
\soul{Спрашивайте. Спрашивайте далее.}
\people{(Ш)Если б у вас была возможность обратиться к человечеству, что бы вы передали бы?}
\soul{Человечеству? }
\people{(Г)Да, всем.}
\soul{Ко всем…Будьте человеком, тогда будет человечество. Сейчас же вы больше похожи на иное.}
\people{(Г)Скажите на такой вопрос…Вы можете физически проявиться в нашем сознании?}
\soul{Физически в вашем сознании?}
\people{(Г) Ну, чтоб мы увидели, в смысле.}
\soul{Физически в вашем сознании – да, можем. Но тогда мы будем болью.}
\people{(Г) Поясните.}
\soul{А вы подумайте, когда вы делаете большие перерывы, мы и подобные нам проявляются в вас как боль, и вы ищете способ унять боль эту. Вы придумываете множество, вы находите  множество путей, чтобы уйти от боли той.}
\people{(Г)Но это - как воспоминание о своих грехах, сознание себя? Да?}
\soul{То – ваше забытое, - мысль совести, мысль чувств…Но  не та, что рождена моралью вашей.}
\people{(Ш)Скажите, если б на месте данного переводчика лежал бы кто-нибудь другой, ответы были бы точно такие же?}
\soul{Нет.}
\people{(Ш)Вы отдаёте отчет, что говорит сознание, подсознание конкретного человека?}
\soul{Да. И потому, мы не говорим им всему миру.}
\people{(Ш)Скажите, в ответах как-то сказывается его начитанность, какой-то запас?}
\soul{Мы пользуемся его эмоциями, и потому, его знаниями, его словами.}
\people{(А)То есть, у другого человека могли бы быть другие аналогии?}
\soul{Да.}
\people{(Г)Но суть оставалась такая же самая?}
\soul{Да.}
\people{(Г)Ну, всё, что нам надо было услышать.}
\people{(Ш)Вы, значит, не можете сейчас, например, заставить переводчика говорить, например, на английском языке? Вам-то – нет разницы?}
\soul{Вы должны были бы заметить, что мы не употребляем никаких терминов, хотя переводчик знает их множество. Мы находимся в той области, которая наименее защищена. Мозг ваш, сознание - ставит защиту, и чем выше логика, в вашем понятии – ``терминология'', тем сложней было бы нам и тем более было бы искажено. Мы же приходим к вам, вашем понятии - на более низкий уровень, не обладающий большими знаниями, понятиями и тем более терминологией, и потому говорим вам сами. И подумайте, то, что вы называете для него ``специальной терминологией'' для кого-то другого - обычная речь, и потому, другой будет лежать и говорить по вашей терминологии. И много ли тогда поймёте? Вы придумали слишком много новых слов и не знаете их значение.}
\people{(Г)Да, вот нам , наверно, не хватает человека, который смог бы всё объяснить. Не хватает, я думаю, человека, хотя бы на Земле, чтобы объяснил наши заблуждения. Потому, что мы в этих заблуждениях ``купаемся'', плодим их ещё больше, и конца и края я этому не вижу, честно говоря.}
\soul{Поймите, кто бы к вам не пришёл, вы будете вариться в себе. Всё – в себе. Относительно ваших знаний, относительно вашей планеты… Подумайте, когда пришёл Христос  - уже был бы новый мир, был бы рай. Но его не поняли. А вы подумайте, - его не поняли, а что же вы хотите от других? Каждый из вас, кто бы ни пришёл, у вас – свой. Вы добавляете свои ``приправы'' и что не нравиться вам откидываете, не спрашивая, ложно то, или истинно, ибо ему ``не нравиться''. Потому и ложь . Вы искажаете, искажаете столь много, что, порою, теряете истину.}
\people{(А)Почему мир нас так изменил? Мир-то один, а изменил, ну… не одинаково. Значит, у кого-то осталось больше сознания, у кого-то меньше всё-таки?}
\soul{Подумайте, мы вам сказали только что, каждый меняет по-своему. Вот вам и ``множество'', вот вам и ``множество миров. Каждый воспринимает по-своему, мы вам сказали только что.}
1-2-3-4-5
\people{(Ш)Вы говорили, что дети более близки к природе, так сказать, изначальной и нет ничего случайного. Я когда шёл на этот контакт, то  услышал просьбу своего старшего сына, который сильно хотел узнать, какую жизнь он живёт. Это, случайно,- не случайно? Не могли бы ему ответить, что ему передать?}
\soul{А может быть вы виноваты в том интересе? Подумайте. Далее. Поймите, мы скажем вам, скажем вашему сыну, придут другие и другие… и - что? - Не та ``энергетика'' – и то-то и то-то Далее. Каждый живёт своей жизнью. Вы же пришли её изучать. Изучая, не помните. Вы должны найти сами. Сами. И подумайте, раньше, когда вы были сыном, вы интересовались, какой жизнью живёте вы? Вы говорили о количестве?}
\people{(Ш)Нет, тогда этих понятий не было.}
\soul{Не было. Тогда вы, всё-таки, думали, что когда-то может быть и жили. Вы даже представляли себя взрослым. Вспомните ваше детство! Вы не знали о количестве. Не знали. Сейчас, дети ваши говорят о количестве. Вы подумайте, что вы делаете?!  Даже здесь, даже в понятие чувств вы вставляете математику и науку. Вы объявляете  - любовь это ``химия''. Вы говорите не о качестве жизни, а об их количестве.}
\people{(Г)Поставлено с ног на голову всё. }
\people{(Ш)Про качество - это будет второй вопрос.}
\soul{Что же вы считаете ``правильно''? Прожить ``много''?}
\people{(Ш)Многие хотели бы узнать ответ на этот вопрос `` сколько он живёт''. Я именно упомянул сына, потому что речь шла о детях много. Я о сыне задал вопрос, не случайный ли его интерес? Ясно дело, что его интерес вызван мной и моими рассказами. Но ведь случайностей-то не бывает в этом мире?}
\soul{Ну, что же вы тогда спрашиваете? Вы говорите ``случайностей не бывает'', и нас же спрашиваете `` случайно или нет?''.}
\people{Вот, я понимаю, что это не случайно и задаю вопрос – какую жизнь он живёт?  Я хочу выполнить просьбу своего ребёнка.}
\soul{Вы говорите о количестве, не о качестве?}
\people{(Ш)Второй вопрос будет о качестве.}
\soul{Поймите, если вы нас спросили о качестве, вы могли бы уже знать будущее сына. Пусть будет 8-я, вас устроит то?}
\people{(Ш)Я бы ему это передам. }
\soul{Что вам в том? Что?}
\people{(Ш)Ребёнка интересовал именно этот вопрос, наверно, не случайно. Придёт время, и он задумается и о качестве.}
\soul{Возьмите, спросите его и пусть скажет  любое число, которое первое придёт в ум.}
\people{(Ш)Хорошо.}
\soul{Вы же говорите, ничего нет случайного. Тогда подумайте. Подумайте, не вами ли, открывая страницу первую, вы искали ответ? Не вами ли было то?(сделано. Прим.) И, чаще, находили…(гадание на первой случайно открытой странице книги. Прим.) }
1-2-3
\people{(Ш)Ну, вы говорите болезни даются нам, как наказания. Нет?}
\soul{Когда мы вам такое говорили? }
\people{(Г)Но это наши ошибки выливаются на нашем теле так.}
\soul{Ваши болезни, это всего лишь ``ошибки прошлого''.}
\people{(Ш)Ошибки прошлого… Потом вы говорили, что это с памятью связано. Что заболеваем там…}
\soul{Подумайте. Говоря о памяти, и о болезни, разве мы вам давали понятия о физических болезнях? Разве мы не говорили вам, что физические болезни - лечите физикой, психические - лечите психикой. Вы же делаете всё только через физику. В вашем понятии, можно управлять чувствами физикой, химией. И вы это делаете, и большинство - успешно, и потому - будет гибель вам.}
\people{(А)То есть физический путь он все равно никуда не приведёт, да? В любом случае.}
\soul{Подумайте, что будет искусственная любовь? Вам дают химию – ивы любите…И вы искусственно любите… Скоро вы придёте к тому и будете делать то. И это будет называться ``любовью''? Вы – хотите летать…Вы мечтаете летать и создаёте самолёты, - вы летаете? Нет, летает самолёт.}
\people{(А)А самим телом мы сможем летать? И в каком случае?}
\people{(Ш) Левитация. }
\people{(А) Да. Если очень сильно-сильно захотеть?}
\soul{Можете. Вы можете летать во всех этих телах. Вы можете иметь даже больше тел и будете  летать.}
\people{(А)Как это?}
\soul{А вы подумайте. Подумайте. Тогда, в вашем понятии, птицы, что летают сейчас, не обладают вашими телами,- они выше вас. Вы подумайте, что вы говорите!}
\people{(Г)У птиц крылья есть.}
\soul{У вас их нет? }
\people{(Г)Ну, это наверно надо очень уж быстро махать руками. }
\soul{Вы слепы. В вашем понятии – не крыльев у вас, ибо вы не видите их. Тогда подумайте, почему же ангелы ваши имели крылья? Подумайте.}
\people{(Г)Но ангелы не физические, не этого тела, не этой физики, тонкой.}
\soul{Подумайте.}
\people{(Г)Хорошо.}
\soul{Далее. Неужели вы думаете, что для того, чтоб летать нужны только крылья? Почему вы понимаете всё впрямую? Иносказательное вы принимаете впрямую. Прямое - вы принимаете иносказательно. Почему? Неужели вы думаете, что только крылья позволяют летать? Вы же читали и вы же приводили примеры оттуда, вспомните.}
\people{(А) Телом мы, вобщем-то, не мешаем…}
\people{(Г) Ну… сознанием летать, как представлением, - вы называете это ``фантазей''.}
\soul{В фантазиях? Тогда вспомните, вы же читали. В вашем понятии, чакра та летала крыльями? Мы же говорим вам – есть крылья у вас более могучие! Тела ваши не могут улететь в космос, вы же – можете.}
1-2-3-4
\soul{Подумайте. Даже здесь мы вам дали подсказку. Даже иносказательно мы дали подсказку, иносказательно мы вам дали понять, что вы можете летать, а вы говорите ``нет''. Вы помните ту подсказку? Тогда вспомните о воскресении.}
1-2-3-5-6-7-8-9
\people{(Ш)Скажите, а какую роль играют, всё-таки, чакры у человека? Как можно открыть верхние чакры?}
\people{(Г)Распределить энергию поровну.}
\soul{Вы должны знать, за что отвечает каждая чакра, тогда вы уже будете знать пути. Если вы знаете, что надо делать для развития нижней чакры, почему же вы не знаете, что делать для верхней? Что надо делать, чтоб открыть сердечную? Вы спрашиваете, за что отвечают чакры ваши,-  мы же говорили, что тело - это всего лишь реакция души.}
\people{(Ш)Можно сказать, что чакра отвечает за очередное тело. }
\soul{Вы уже можете по чакрам вашим сказать, кто человек, его характер и далее. Ибо не чакры влияют на характер, а характер влияет на них. Подумайте, чаще, вы же путаете. Вы хотите увидеть чакру, вы говорите ``увеличьте мне сердце, я буду лучше других''. Увеличили. Но вы не будете лучше любить. Не будете. Вы же должны сделать это сами. Поймите, не чакры управляют вашими характерами, а характер. Мы говорили вам, о ``лепестках'' что  двигают мысли ваши, и мысли ваши двигают ``лепестки'' те. Вспомните. И если вы хотите увеличить какую-то или уменьшить вашу чакру… Подумайте. Ответ очень прост - за что отвечает, то и делайте. Понятно?}
\people{(Ш)Смутно.}
\soul{Смутно? Смутно… Хорошо, назовите любую. Назовите любую и назовите, за что отвечает она.}
\people{(Ш)Вот, на неделе только узнал, что вторая отвечает за сексуальность человека. И как правило, экстрасенс, которая рассказывала, что у подавляющего большинства людей именно эта чакра открыта.}
\soul{А вы  подумайте, что заставило её сказать так. Почему она видит её более, чем другие? Может у неё проблемы с чакрой той?}
\people{(Г)Где же найти грань, у нас проблема или у неё?}
\soul{А мы же говорили вам: Верьте себе. Вы же будете верить кому угодно. Но если только совпадает с вашим и нравиться вам. Кто бы не пришёл к вам, к то бы не был пророк,- но если он скажет вам то-то, но вам не нравиться то,- вы не уверуете. И любому, любому мальчишке, поверите ему, если он скажет вам то, что вам нравится. Вы скажете - не случайно ли сказал вам то?}
\people{(Г)Да… (…)}
\soul{Спрашивайте.}
\people{(А)Это значит, что (…)}
\people{(Ш)Значит, если я поверю, что у меня открыты все-все  чакры, то я…}
\soul{Этого уже будет достаточно, если вы будете веровать. Сделайте проще – возьмите, в тихой обстановке, никому не мешая, и просто представьте каждую из них. И вы увидите. Какую-то одну из них вы увидите больше, другую меньше. Значит, ту, что увидели больше – она больше меньшей,- вот вам понятие ``фантазии''. Возьмите и представьте. Представьте, пока, их величины, не берите цвета, ибо сложно то, представьте их величины – и вы увидите разные. И вы будете знать, какие больше, какие меньше.  Но, придите к экстрасенсу и пусть он вам скажет и посмотрите - совпадают ли?  И будьте осторожны, здесь нельзя дать ответа о конкретности. Поймите, многие видят чакры те, в других, которыми страдают сами. }
\people{(Ш)Но, так или иначе, если у человека открыта одна чакра, он должен эту энергию, которая через неё проходит, стараться раскидать по всем чакрам, только изнутри как бы?}
\soul{Поймите, ошибка ваша, что вы хотите энергию той чакрой раскидать по всему телу. Но, поймите, вас пронизывает множество, множество энергий, вы же, порцию вашу, что заняла одна чакра, раскидаете на всех, и что? Что будет? Ну, представьте - ваша семья, и одна порция… Согласитесь, что если эту одну порцию разделить на всех, голодны будут все.}
\people{(Г)Да…у нас так приходиться  зарплату на всю семью раскидывать.}
\people{(А)(…) Что ж, вообще не делать?}
\soul{Сделайте проще, вы живёте в мире энергий, но не можете его представить. Почему? Почему вам надо всё ``от'' и ``до''? Ну, подумайте, мы говорили вам, представьте, какая чакра больше, какая меньше и начните увеличивать. Увеличивать – не уменьшать,- большую. И тогда, будет вам. Рано или поздно, вы получите нужные вам и заметите, что другие не уменьшились. Вы же, увидев, что одна большая, другие - малые, будете убирать большую. И что будет? Будут все малые. У вас хотя бы одна была развита, станут все одинаковые, не развитые.}
\people{Что делать, если я, например, даже не знаю, что такое чакра? Значит, их у меня нет?}
\soul{Разве? Вы не знаете многое - этого нет? Возьмите проще, - возьмите и поделите себя хотя бы  на семь частей. Возьмите и сделайте проще ещё - множество литературы, где нарисованы они, - возьмите и представьте, представьте. И не бойтесь, если не получится. Если вы будете бояться, что не получается, то не получится. Если вы будьте специально настаивать на том, чтобы получилось – и не получиться. Будьте проще, будьте проще. Вы же читали, и вы нас спрашиваете.}
\people{(Ш) Но,если я представлю и увеличу таким образом сердечную чакру - что это, следовательно, - я стану больше любить?}
(конец ленты).
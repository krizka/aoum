Аоум глава 2-11-96
Георгий Губин
\people{**}
 VG – 1996.11.02_-_01 
\soul{(Гена) …выделять его. Я не буду его… но, с первого взгляда, это кажется равнодушием. Что это стул, что другое что-то, понимаешь, и относишься ко всему одинаково. Нет, я просто его не выделяю. Я его не выделяю из-за того, что я чувствую единство во всём, и в том же стуле.}
\people{(Ольга) Гена, вот скажи, ты как-то на контакте, который был первым после перерыва, про ``чужого'' всё время говорил: ``Чужой! Чужой!'' Ты сейчас не можешь вспомнить, кто это был ``чужой''?}
\soul{(Гена) Ну, это был, наверно, не я.}
\people{(Ольга) Не ты? А-а… Ну, да.}
\people{(Белимов) В этом состоянии, ты чувствуешь в сыне своём какие-то особенности, черты может экстрасенсорные? Вот он видит в небе войну. Что это такое? Ты можешь сейчас ответ получить?}
\people{(Гера) Нет, да?}
\people{(Белимов) В этом состоянии - не может. }
 (Диалог с Мабу)
\soul{(Мабу) Надо говорить Мабуу.}
\people{(Белимов) Вот оно чё!}
\people{(Ольга) Мабу, я только подумала про тебя, и вот, опять встретились. (Смех) Ну, рассказывай, как у тебя жизнь?}
\people{(Белимов) Трудные времена, неприятности были у тебя, да?}
\soul{(Мабу) Чего это? Будут ``две луны'' и будут неприятности. Этого не было. }
\people{(Белимов) У тебя всё хорошо складывается в жизни, да? Ты радостный? Хватает корма, еды?}
\people{(Ольга) ``Корма, еды.'' (смеётся)}
\soul{(Мабу) ``Корма, еды?'' Нет, не хватает.}
 (Белимов) Маловато, да?
\soul{}
  (Мабу) Мало.
\people{(Ольга) Да ты чё? Ты так поесть любишь?}
\soul{(Мабу) Да.}
\people{(Ольга) А попить?}
\soul{(Мабу) Конечно, попить – тоже. Глупая! Что, я буду есть, а пить не буду что ли?}
\people{(Гера) А что ты пьёшь?}
\soul{(Мабу) Что пью?! То, что и ты пьёшь!}
\people{(Белимов) Воду, да. А в чём вы её храните, в чём приносите? Что-то ты не рассказывал никогда об этом.}
\soul{(Мабу) Рассказывал!}
\people{(Белимов) Да?}
\soul{(Мабу) У вас память плохая.}
\people{(Гера Белимову) В шкурах приносят.}
\soul{(Мабу) Да. А ещё  - делаем в камушках ды… (дыры прим. [не может сказать])}
\people{(Белимов) Дыры, отверстия да?}
\people{(Ольга) Углубление?}
\soul{(Мабу) Да. Вот. А там, потом, вода.}
\people{(Гера) А чем вы делаете?}
\soul{(Мабу) Камнем.}
\people{(Гера) Камнем по камню? }
\soul{(Мабу) А как же ещё? (удивлённо)}
\people{(Белимов) Это долго же, большой труд.}
\people{(Гера) Они же оба разрушатся.}
\soul{(Мабу) Если не умеешь, то разрушатся, а если умеешь, то это и недолго.}
\people{(Гера) А сколько?}
\soul{(Мабу) Чего ``сколько''?}
\people{(Белимов) Сколько вы тратите?}
\people{(Гера) Один день?}
\people{(Белимов) Две ночи, три? Сколько ночей проходят?}
\soul{(Мабу) Это же днём делают, а ночью спят! }
\people{(Гера) Ну, а надо говорить: ночь проходит. Сколько их прошло, прежде чем вы сделали?}
\soul{(Мабу) Оди-и-ин.}
\people{(Белимов) Так быстро?}
\soul{(Мабу) А что там долго-то делать-то?}
\people{(Белимов) А вы не пробовали из сырой глины делать? Потом - она высохнет, и будет очень хорошая вещь. Ты попробуй своим соплеменникам рассказать.}
\soul{(Мабу) А что такое ``глина''?}
\people{(Белимов) Глина – это вязкая такая, влажная земля.}
\people{(Ольга) Это, когда землю берёшь, и туда водички немножко наливаешь и размешиваешь, и такое…}
\people{(Белимов) Как тесто получается, вязкое что-то.}
\people{(Ольга) Получается, знаешь, когда зерно размалываешь, а потом, водичку туда добавляешь, - так же получается.}
\soul{(Мабу) И чё-ё?}
\people{(Ольга) А потом, делаешь такое…}
\people{(Белимов) Углубление, как бы – сосуд. На солнце или около костра поддержишь, и будет ``кружка'' такая или ``миска''. И очень удобно, ты своим соплеменникам расскажи, как…}
\soul{(Мабу) Глупые какие. Как я возьму? У меня же всё растечётся!}
\people{(Белимов) Есть специальный вид глины.}
\people{(Ольга) Смотри, Мабу. Берёшь землю, наливаешь туда…}
\people{(Белимов) Жёлтую, светлую.}
\soul{(Мабу) Подожди-подожди. Я так возьму. (пошёл за землёй прим.)}
\people{(Гера) На речке берёшь землю.}
\people{(Ольга) Берёшь землю, наливаешь туда водички, и, так, рукой…}
\soul{(Мабу) (перебивает) Ну, подожди!  Землю я взял, осталось только водичку взять.}
\people{(Ольга) Ну, давай, иди, возьми водичку.}
 
(Молчание)
\people{(Белимов) Мабу, ты пришёл?}
\people{(Ольга) Нет.}
 
(Обрыв) 
 (Меняется интонация переводчика)
 
\soul{Чтобы всех вас, как бы, поставить на верный путь, это значит, в вашем понятии - просто уничтожить, уничтожить невыгодное вам. Но по какому критерию будут выбираться, эти вот избранные? Кто будет избирать, кто будет судьёй, который будут решать - быть или не быть? Понимаете? Кто? Каждый из вас не откажется от этого. И, естественно, каждый из вас не поставит, потом, себя в ряд уничтоженных. (Теряется)}
 
(Снова диалог с Мабу)
  
\soul{(Мабу) Ну, принёс я водички, что дальше-то?}
\people{(Ольга) Смех.}
 (Обрыв)
\people{(Ольга) Углубление, ты в камне же делаешь углубление?}
\soul{(Мабу) Возьми землю! – Взял землю, взял воду. Где я буду делать углубление?}
\people{(Ольга) Теперь, наливай воду в землю. Чуть-чуть только.}
\people{(Белимов) Немножко налей.}
\people{(Ольга) И рукой, рукой так, как тесто, размешивай.}
\people{(Белимов) Пальцами мешай. И не жидкая, а вязкая будет. А из вязкого, потом, сделай вроде ``ладошки'' такой.}
\soul{(Мабу) Не получается.}
\people{(Белимов) Земля рассыпается видимо, да?}
\soul{(Мабу) Утекла.}
\people{(Ольга) Утекла? Вода?}
\people{(Белимов) Ты много, наверное… (много воды налил прим.)}
\soul{(Мабу) Земля утекла. }
\people{(Ольга) Земля утекла? }
\soul{(Мабу) И вода утекла.}
\people{(Ольга) А-а. Надо другую искать.}
\people{(Белимов) Ты пробуй ещё. Находи другую землю. Знаешь, лучше не тёмную, не чёрную землю искать, а светлую, жёлтого цвета.}
\people{(Ольга) Нет-нет, там другой цвет. Смотри, около речки…}
\soul{(Мабу) А чем вам не нравится камень?}
\people{(Белимов) Это – долго, это – неудобно, это - много труда тратить нужно.}
\people{(Ольга) Камень большой, его нельзя переносить. А если сделать сосуд из глины, то туда…}
\soul{(Мабу) А зачем его таскать!?}
\people{(Ольга) Как ``зачем''? А вдруг, понадобится?}
\people{(Белимов) В путешествие пойдёшь куда-нибудь.}
\soul{(Мабу) С камнем!?}
\people{(Белимов) С собой возьмёшь сосуд. Его можно сделать и длинненьким, круглым, называют  ``кувшин''. Вы потом его назовёте так. И можно хранить. А камень не сохранит.}
\people{(Ольга) Там зерно можно тоже хранить.}
\soul{(Мабу) Я, тогда, расскажу, что вы его обозвали. (назвали кувшином прим.)}
\people{(Гера) Кого? }
\people{(Белимов) Вот, когда-нибудь попробуй. Ты запомни эту мысль и пробуй-пробуй и однажды у тебя получится, и ты будешь признан, как большой учёный, или там…умный человек.}
 (обрыв) 
 (Меняется интонация переводчика)
\soul{Чем вызван ваш интерес? Именно бедствием, сожалением о них?}
\people{(Белимов) Просто жалко и цивилизацию. Шли-шли и вместо того, чтобы эволюционировать до…ну, как инопланетяне живут, гуманность полнейшая и прочее, мы переходим к самоуничтожению. Разве это путь человечества? Получается, что лучше…}
\soul{Ну, а кто выбирал?}
\people{(Белимов) Ну, хочется-то к лучшему-то идти. Совершенствуется же разум.}
\soul{А кто вам мешает? Идите! А вы? Что делаете вы? Вы попробуете там, попробуете здесь. Простите, вы уже – ``пятые'' и всё не напробуетесь никак.}
\people{(Ольга) Ну, мы должны, наверное, вот, я так думаю что…}
 (обрыв)
\people{(Белимов) Сеанс закончился неожиданным выходом Геннадия…}
 (обрыв)
\soul{Вы выбрали неудобное место…не сделали сразу?}
\people{(Ольга) А? Не сделали сразу?}
\people{(Девушка) Чтобы здесь не мешаться.}
\people{(Ольга) Нет. Но недавно здесь сидел человек. На этом же месте.}
\people{(Гера) Это зависит, какой человек на этом месте сидит, да? }
\soul{Прежде всего - зависит от вас. Удобно ли будет вам.}
\people{(Гера) Ага.}
\soul{Где вы  будете отвлекаться.}
\people{(Девушка) Спасибо.}
\people{(Ольга) Мы можем продолжать общаться с вами, да?}
\people{(Ольга) Мы, теперь, собрались в том составе, в котором мы там наделали делов, наломали дров. Вы не подскажете, как нам, так сказать, исправить свою ошибку?}
\soul{Вы же делали ошибки, вы и исправляйте! Мы говорили вам о настрое.}
\people{(Гера) Но, а вдруг, настроимся, так сказать, а сделаем не то, что надо?}
\soul{“Просто ошибки'' - не наказуемы… если ошибка сделана с равнодушием - это… Мы же говорили вам о равнодушии.}
\people{(Гера) Может быть, просто, незнание? Что это - ошибка?}
\people{(Ольга) Нет. Ну, мы, наверное, тогда, когда с переводчиком будем разговаривать. Может быть, на эту тему ещё поговорим. А сейчас, можно уже с вами беседовать?}
\soul{Мы ещё не беседуем?}
\people{(Ольга) Ну, мы как привыкли с вопросами, как всегда, какими-то, какую-нибудь тему, такую вот, затронем.}
\soul{Дайте трижды семь и восемь.}
\people{Скажите, пожалуйста, вы как-то говорили о монашестве, о монастырях. Мы, обычно, женские как-то представляем, а монашество, обычно - мужское. А женские монастыри, они всегда тоже были, как и мужские?}
\soul{Давайте скажем так, что деление на мужчины и женщины - это выдумано чисто вами.}
\people{(Ольга) Ну, да.}
\soul{И мужчины и женщины, прежде всего – люди. И то, и другое – человек. И давайте не будем говорить об Адамовых рёбрах. Это всё лишь - мужчина пытается поставить себя выше женщины.}
\people{(Ольга) Нет. Я имею в виду, женщина, она, как мать. То есть, она, так сказать, вынашивает, рожает детей, поэтому, может быть, мужчина в этом процессе как бы и не участвует.}
\soul{Ну, прекрасно! Теперь мы, давайте, начнём говорить о матриархате.}
\people{(Ольга) А-а. Ну, опять – то же самое. Нет. Я имела в виду как раз - всегда ли были только мужские? В смысле, вот, допустим, когда Мабу…(мужское монашество прим.)}
\soul{Изначально – да.}
\people{(Ольга) Только мужские, да?}
\soul{Да.}
\people{(Ольга) А… Ну, вот, а потом, уже… А вот, когда Мабу?  В его время были женские тоже? Или как?}
\soul{О, нет.}
\people{(Ольга) Нет? Только мужские?}
\soul{Тогда не было монахов, в вашем понятии.}
\people{(Гера). А-а! В нашем.}
\people{(Ольга) Они просто скрывали свой вид, чтобы их не пугались? Так?}
\soul{Вы уже спрашивали об этом. И вам, уже сам Мабу рассказывал.}
\people{(Ольга) А вот, скажите, как они… И среди них тоже и женщины были? Среди тех, кто скрывал свои лица. Ну, не лица, а вообще.}
\soul{Нет.}
\people{(Ольга) Одни мужчины? Поэтому, они и вымерли, что, в общем-то, наверное…}
\soul{Нет. Просто, тогда не было полов.}
\people{(Гера) Так они были…}
\soul{Да, они были…}
\people{(Гера)…”универсальными'', скажем так?}
\soul{Нет. Просто не было полов.}
\people{(Ольга) Ну, а как они, извините, конечно, размножались?}
\soul{А вы привыкли размножаться только между разными видами?}
\people{(Ольга) Да нет. Нам трудно просто сейчас понять. }
\soul{Зачем это вам нужно?}
\people{(Ольга) Ну, интересно.}
\soul{Просто - любопытство?}
\people{(Ольга) Да нет, не просто любопытство. Просто, в конце концов, разнообразие жизни… и хочется узнать, как все…}
\soul{Вы не знаете жизнь, в которой вы живёте. Познайте её. А потом, когда будете знать её, тогда уже займётесь другими жизнями, тогда уже любопытствуйте. А как вы можете сейчас представить ``абсолютно без пола''? Вы не сможете это сделать. Потому, что даже природа…}
\people{(Ольга) Да. Сейчас-то ``бесполых”…}
\people{(Гера) А почему эволюция вот так пошла - по пути разделения…}
\people{(Ольга) …Полов. Это же - эксперименты? }
\soul{Нет. Это - просто один из видов. Один из способов.}
\people{(Гера) Понятно.}
\soul{Вспомните. Сперва был Адам. Ему было скучно. Вы помните?}
\people{(Ольга) Да-да-да.}
  
\soul{И господь сделал из его ребра женщину, чтобы ему не было, как бы скучно. Чтобы ему было чем заняться, развлечься. И заметьте, вообще-то имелось  в виду - в духовном плане. }
\people{(Ольга) Угу.}
\soul{Лишь только когда это перешло на физический план, тогда - были изгнаны.}
\people{(Гера) Угу. Когда люди стали физическими, они потеряли что-то, что… }
\soul{Нет. Когда из духовного, вы превратили в физическое. Именно тогда Адам заметил, что он гол, а значит, заметил различие пола. Лишь только тогда, когда Господь пришёл, он сказал: ``Мне стыдно перед тобой, Господи. Я гол''. Вы помните?}
\people{(Гера) Да, я помню.}
\soul{Тогда Господь понял, что Адам освоил различие полов, и значит, нужен другой мир. Как вы думаете, какой пол имеет Господь?}
\people{(Ольга) Да, он бесполый. Пол только у нас вот, в этом мире есть, а род, наверное… Да?}
\soul{Вот, поэтому вы живёте в этом мире. И всё в этом мире имеет пол.}
\people{(Ольга) Скажите тогда, о родАх. Вы как-то говорили, что… (род влияет на нас, но не пол. прим.)}
\soul{Относительно этого мира? Относительно этого мира - вы имеете родА. Более примитивно - отцовская линия и женская линия.}
\people{(Ольга) А монада не имеет рода?}
\soul{У вас?}
\people{(Ольга) Да. Человек.}
\soul{У вас? Да. К сожалению, здесь, вы уже дали ей значение пола.}
\people{(Ольга) В смысле ``название'', да?}
\soul{Раз дали, значит, и есть. Поймите, этот мир – мир иллюзий, ваших иллюзий, и какие иллюзии создадите, то и будет. Захотите вы рай, будет ваша Земля раем, захотите ад, будет адом. А так, как вы не хотите все одно и одновременно, поэтому, у вас хаос. Есть и рай, есть и ад. (обрыв)}
  
\soul{Спрашивайте.}
\people{(Ольга) А дальше, мы не поняли.}
\soul{О монадах?}
\people{(Гера) Да, о монадах.}
\soul{Всё очень просто, вы тоже здесь их отождествили с полом - мужской и женский. Иными словами, вы создали ещё одни иллюзии - испорченный телефон.}
\people{(Гера) Ну и напортили мы тут…}
\soul{Если это мир иллюзии, то достаточно много. Практически - всё. Потому, вы не видите многое, чтобы это хотя бы не испортить. От вас приходиться прятаться. Прятаться. Иначе вы нарисуете множество картин, и они все будут неверны. А кому охота увидеть свою карикатуру?}
\people{(Ольга) Да. У нас… Один художник, у него есть целая серия произведений по поводу человечества и, будем говорить, не очень приятно смотреть на это. Давайте перейдём тогда уже на другую тему - о животных. Мы как-то начинали о животных разговаривать, ну, кое-что мы, может быть, действительно поняли. Вы нас убедили, что… не то, чтобы убедили, а дали хотя бы понять, мы, как-то, немножко высокомерно относились к ним. В общем-то, стали понимать, что мы с ними равноправны. В принципе, всё в мире имеет равные права, если так грубо выражаться.}
\people{(Гера) Разные возможности.}
\people{(Ольга) Да. Можно продолжить разговор о животных? Допустим, какая разница между животным допустим, и…ну, человеком?}
\soul{Мы говорили, примерно, какая разница.}
\people{(Гера) Ещё скажите в этом ключе, почему в Библии написано, что Христос говорил что: ``Вы любимее многих малых сих''?  Получается – животных, там, птичек. Это за что он нас так отметил? }
\soul{Почему? Потому, что у вас есть разум.}
\people{(Гера) А у них, нет что ли?}
\soul{У них есть разум. Но у вас есть разум, который может и должен осознавать самого себя.}
\people{(Гера) А-а.}
\soul{Что делаете вы довольно-то не часто.}
\people{(Гера) А любое  животное не осознаёт себя? Есть такие, которые хоть задумываются, там, в зачатке, что они  делают?}
\soul{Конечно, вы хотите спросить о животных цивилизациях, мы сейчас опять вернёмся к дельфинам. Нет. Животные, в какой-то мере, честнее вас. Да, оно тоже лжёт. Так же, как и вы. Только вы лжёте, порою, - просто так, даже когда не грозит вам смерть, животные же это сделают только тогда, когда ей что-то угрожает. Это первое различие. Второе различие – животное не создало себе богов никаких, и потому, у них нет ложных религий. Для него единый бог. Один.}
\people{(Гера) Пища, да?}
\soul{Разве?}
\people{(Лена) Человек, может быть?}
\soul{А почему тогда, простите, собака умирает без хозяина? Бог – пища? Для неё один бог. И она не видит его конкретно личностью. Бог для неё - это всё. И вы являетесь частью бога, почему животные, всё-таки, к вам не очень-то и равнодушны. И оно является частью бога. Оно этого не осознает, но живёт. Вы скажете: ``Как же так?! Лишь только человеку Господь дал монаду - искру божью. А что же тогда двигает животными? Почему они живут? Тогда, если так рассуждать, учёный мог бы создавать животных, сколько хочешь и каких угодно. А человека бы не мог, ибо он не может дать искры божьей''.}
\people{(Гера) Значит, человек не Бог.}
\soul{Человек не Бог, и в то же время - он создаёт себе подобных. Где же он берёт эту искру божью?}
\people{(Гера) Ему, получается, кто-то даёт.}
\soul{Ну, вы подумайте. Вы - не Бог, вы не можете дать монады, и, в то же время - вы рождаете себе подобных. }
\people{(Гера) Ну, значит, это монада, имеет свойство, там, делиться как-то.}
\soul{“Делиться'' - вот вам и опять ``пол''. Почему? Потому, что вы не имеете, больше, иного представления. Для вас - это деление, другого - вы не видите. В вашем понятии – пустота. Это пустота, там нет ничего. А мы вам говорили, что пустоты не существует, нигде и никакой.}
\people{(Гера) Да, но вы тут же говорили, что пустоты достаточно, в принципе.}
\soul{О какой пустоте мы говорили? О пустоте в вас? Достаточно. }
\people{(Гера) Угу.}
\soul{Мы - о пустоте духовной.}
\people{(Ольга) Скажите, вот мы живём в мире иллюзий, то есть, наше тело - это тоже иллюзия, то с чем общаемся, с чем сталкиваемся, то есть природа…}
\soul{Ну, не совсем же она иллюзия, в вашем понятии.}
\people{(Гера) Относительно вас, получается?}
\people{(Ольга Гере) Нет.}
\people{(Ольга) Действительно, как вы говорили нам: надо найти свои истинные тела, что ли, грубо говоря.}
\soul{Давайте скажем так: вы - это то, что вы пытаетесь назвать монадой. Это вы, именно вы. Всё остальное, это физика, тела. Если бы вы могли бы обладать, всеми своими чудесными свойствами, то вы прекрасно могли бы менять своё тело, менять, как вам удобно. Если вам удобно в данный момент быть кошкой, вы тут же станете кошкой. И когда это будет ненужно, вы опять можете превратиться и любую другую принять форму. Вы же, не умеете этого делать. И, наверное, это даже ``слава богу'', ибо вы ещё не готовы, от вас надо прятаться, ибо вы сейчас больше похоже на зверей. Всего лишь только два зверя: вы и крысы. Здесь вы похожи друг на друга. Вы можете убивать просто ради прихоти. Просто - чтобы это было интересно. Вы будете отрицать?}
\people{(Ольга) Да нет.}
\soul{Тоже самое можно назвать и крыс. Вот и подумайте эволюцию.}
\people{(Гера) А мыши? }
\soul{Что - мыши?}
\people{(Гера) Мыши, они, в принципе, с крысами…}
\soul{Мыши - это не крысы.}
\people{(Ольга) Скажите, человек домашних животных тоже для своей прихоти, тоже для своих, так сказать, целей… (использует прим.) Это же не случайно? Или, вот, живя с человеком, - человек их в рабство что ли взял, закабалил.}
\people{(Лена Ольге) Зачем заводят животных? }
\people{(Ольга) Нет. Ну, вообще - приручил же ещё в те времена, когда он приручил животных.}
\soul{Тогда - да. Тогда, это нужно было для выживания. А сейчас? Порой ``завести животных”-  это наоборот - мешает выживанию. Практически - вы одиноки.}
\people{(Лена) Да, может быть из-за этого.}
\soul{Даже прожив всю жизнь с близкими, и не поняв их, вы всё же одиноки. И вы ищете безмолвную душу, которая не могла бы вам воспротивиться, вы могли бы просто выговориться. И вы заводите животных. Лелеете даже больше, чем ребёнка. Вы любите его даже больше, чем родных. Почему? Почему вы это делаете?}
\people{(Гера) Бессловесная тварь потому что.}
\people{(Ольга Гере) Да нет!}
\soul{Тогда - вместо вас, любого,  было проще - завести животных, а не вас.}
\people{(Ольга) Не знаю. Наверное, мы – эгоисты. В том плане, что животные, в принципе, всегда нам отвечают…}
\soul{Животное беззащитно. Для многих, это уже важный фактор, ибо оно не может ответить вам. А вам надо выговориться. Хорошо, если только выговориться, а порой - просто почувствовать себя хозяином положения. Бывает и так, когда вы берёте себе действительно раба, и - да, действительно его любите, но множество – всё же - нет. Возьмите старушку - дети бросили, оставили её. Она заведёт себе собачку, чтобы просто, скоротать время. Вы же заводите друзей?}
 (Ольга) Ну, да. Заводим.
  
\soul{Для кого-то, животное  - это друг. И ещё, друг может обмануть, животное - нет. Это тоже…(тоже важный фактор прим.) (теряется)}
\people{(Ольга) Я про себя скажу. У меня есть и кошка и собака. Я вообще - всегда с детства любила животных, как-то с ними общалась. И я, будем говорить, завела не потому, что мне скучно или что-то, а ну… просто так вот.}
\soul{Они - чище. Вы видите, что они чище. Понимаете? Вы не боитесь, что вас обманут, что вас обидят. Это очень важный фактор для вас. }
\people{(Ольга) Да, это правда.}
\soul{Потому, что вы всю жизнь идёте с опасностью, что вас обидят, не поймут. И поэтому, вы заводите друзей среди животных. Животные ``заводят'' вас, ведь не всякое животное сживается, - порой и уходят. И потому, не берите насильно никого. Самое лучшее - это брать не ``породистую''. Ищете породу, смотрите ``паспорта'', - это не друг, это, всего лишь только ``престиж''. Вот, привязалось за вами животное, вот, возьмите его, и она будет верна вам, ибо оно само нашло вас, а она в этих случаях - умнее. }
\people{(Ольга) Скажите, может быть это, конечно, и грубо так… Если человек к тебе привязался, то - тоже  так же? Или человек, может, из выгоды привязаться? Ну, так… грубо.}
\people{(Лена) Ну, конечно, может.}
\people{(Ольга) Дети же, они всё-таки чище нас намного - взрослых. И они, действительно вот, привязываются к кому-то, даже к незнакомым людям, на улицах гуляют, кого-то они там выделяют.}
\soul{Да, безусловно. Это должно так быть. Вы должны искать подобных себе. Это звучит грубо, но это так. Дополняющих или противоречащих. Ведь есть друзья, которые противоположные друг другу во всём. Для чего? Чтобы ``не стоять на месте''. А для кого-то нужно, чтобы было ``полное совпадение''.}
\people{(Гера) Или - ругаться.}
\people{(Ольга) Скажите, об изгоях мы как-то говорили тоже. Вы сказали, что вообще, человек должен стать изгоем везде. Как - везде?}
\soul{Он не должен быть им. Он не должен чувствовать себя изгоем. Кем бы он ни был. Поймите, пришёл к вам Христос, неужели он чувствовал себя изгоем?}
\people{(Ольга) Да нет.}
\soul{Нет. Только вы, сделали его им. Он же не чувствовал того, ибо он действительно любил. А если вы будете любить истинно, то вы никогда не подумаете плохо о любящем, но гонящем вас.}
\people{(Ольга) Да. Наверное.}
\soul{И потому, мы никогда не говорили, что вы должны быть изгоем. Это - первое. И второе,- к сожалению, все вы - изгои, ибо вы все одиноки, ибо сейчас пришло время разделения.}
\people{(Ольга) Да, мы это заметили.}
\soul{Почему? Почему? (теряется) …. Самое страшное в жизни - это быть забытым и не иметь друзей.}
 (обрыв)
\soul{..а третьи стараются уйти в отшельники, чтобы было как можно меньше соблазна. Он чувствует, что он не может справиться. Понимаете? А это значит - уже есть неверие в Бога. Если вы не уверены в себе, не уверены в своих силах, вы уже, значит, и не верите Господу. Понимаете? Потому, что вы же знаете, что сделаны по его подобию и благодаря вот этому знанию  вы начинаете проклинать Бога во всех бедах, которые сотворили вы сами. И обвиняете его в слепоте, хотя сами слепы, в глухоте… Понимаете? А ведь это всё - ваше. И давайте скажем так, что, практически, все сто процентов бед  созданы, всё-таки, вами самими. Понимаете? }
\people{(Ольга) Да.}
\soul{И если кто-то извне пришёл и обидел вас, просто он пришёл на резонанс. Понимаете? Значит, вы уже излучали те колебания, которые позвали того же преступника, который мог бы обидеть вас. А для чего это нужно? Ну, во-первых, это чисто физическое. Понимаете? Потому, что в резонанс могут войти только две одинаковые частоты,- вы согласны же, да?}
\people{(Гера) А гармоники?}
\soul{Это - чисто физическое. Гармоника - это всего лишь только… как вам сказать… сила вот этого насилия извне.}
\people{(Гера) Угу.}
\soul{-Или это может закончиться самоубийством, или - просто ограблением, или - просто испугом. Понимаете? Вы же - начинаете обвинять кого-то. Простите, если у вас прекрасное отличное настроение, вы честны, у вас даже нет ни малейшего намёка, что вас сейчас может просто остановить кто-то и что-то вам сделать, у вас даже этого в мыслях нигде нету - и этого не произойдет, где бы вы ни шли. Понимаете? И даже если вы… Давайте, заметим так: вот, среди тонущих, много ли были съеденными акулами? Нет. Потому, что ему было сейчас не до акул. Понимаете? Он тонул. У него была причина одна – не утонуть. И акула, та же акула -  уже не могла придти к вам на резонанс. Понимаете?}
\people{А кровь как же? Допустим, ранка там или что. Они же чувствуют!}
\soul{Тогда, вы должны были бы вспомнить, что было очень много людей, которые были и ранены и теряли кровь в воде, но всё-таки не были съеденными, хотя и было множество акул. Понимаете? Вы помните легенду, когда Господь прислал в качестве еды оленя? Помните?}
\people{(Гера) В библии, да.}
\soul{Так? Всё было прекрасно, всё было хорошо, но вы видите, беда в том, что появилась и болезнь, которая косила, как и оленя, так и народ. И тогда, народ опять просил Бога избавить от этой болезни. И что сделал Бог? Он прислал волков. Вы помните?}
\people{(Гера) Нет, этого не помню.}
\soul{Есть? Есть. И волки, вы же сами говорите ``санитары природы''. Да действительно. И и… (теряется)}
\people{(Гера) Про волков там, ничего не помнишь?}
\people{(Белимов) Это уже ``всё'', наверное. А вот в этом состоянии…(сможешь ответить на вопрос прим.) Попытки Александра Лебедя - они положительные или отрицательные? }
\people{(Лена) (Удивление {просили же не называть имена прим.})}
\people{(Белимов девушке) Да не волнуйся! Уже 10 раз называл имена.}
\soul{Мне бы не хотелось говорить о политике, потому, что это всего лишь только, как бы сказать, - всё это да, всё это ложно. Политика она не одухотворённа, понимаете? И что такое ``политика''? Это просто один способов обмануть. К сожалению, это сейчас так. Понимаете? Политика - это бизнес. К сожалению, вы превратили его в бизнес. И нельзя сейчас назвать ни одного человека, который бы действительно ``болел'' и действительно хотел что-то сделать для родины. Всё – красивые слова. Какое бы имя вы сейчас не называли. Понимаете? А если конкретно о нём, то простите, из него может получиться только диктат. И то, не очень хороший, потому, что в истинно политическом смысле - он безграмотен.}
\people{(Ольга) Это точно.}
\people{(Белимов) Хорошее… (Хорошее замечание прим.)}
\people{(Гера) Может, им движет благие намерения?}
\soul{Никем сейчас, к сожалению, не движет. Сейчас всеми движет материальное благосостояние, от низших, до высших. Понимаете? Проблема и беда в том, что сейчас вы…Вам же говорили, что время экзамена пришло. А что такое ``экзамен''? Это очень трудная пора. Понимаете? И поэтому, всё, что вы имели внутри в себе, выходит наружу. Создаются такие условия, что всё, что есть в вас, то, что вы скрывали…(выходит наружу прим.) Понимаете? Вы можете прекрасно 20 лет жить добрейшим, прекрасным человеком, но где-то внутри вы всё-таки маньяк, убийца. Понимаете? И обязательно в эти времена экзамена это должно выйти, выйти наружу. А когда - просто время идёт, просто - течёт, и нет никаких вот этих переходов, этот человек может прекрасно прожить, внутри быть убийцей, но умрёт прекрасным семьянином и никогда никого даже не обидит! Понимаете? Но сейчас время… и заметьте, это везде сейчас происходит. Назовите хоть одну страну, где было бы сейчас прекрасно и хорошо. Нет! Понимаете?- Резонанс. Земля сейчас вошла, и приняла те колебания, чтобы отсеять ложь. Вы создаёте ложь, понимаете? Вы создаёте между собой ложь, но Землю-то обмануть нельзя, всё-таки - она вас родила. И вот, чтобы отсеять эту ложь…. Ну, вспомните! - Придти, собрать пшени..(теряется)}
\people{(Гера) Ну да-да-да. Помним.}
\soul{Беда  в том что, вот это вот, как вы говорите: небоязнь обнажить своё. Это и есть беда! Это значит - пришло время вседозволенности. Что хочу, то и ворочу. То есть, все законы, что были в вас, они все нарушены. А потому и будет множество болезней. Понимаете? Как можно уничтожить…? Представьте,- кому то нужно уничтожить человечество. Ну, необязательно ему это делать войной, понимаете? Взять вас и физически уничтожить. Зачем? Физически врага не побеждаешь, понимаете? Вспомните…}
\people{(Белимов) Чечню, да.}
\soul{Ну, зачем же? Давайте вспомним так: чаще всего, враг старался сломить волю. Понимаете? Не ``просто убить'' его физически, а сломить волю, для чего и брал он-то в пленных. Не для того чтобы делать из них рабов. Вспомните последнюю войну. Не так уж и много рабов было сделано из пленных. Понимаете? А именно - сломать его волю. Именно - психологически одержать победу, а не физически. Физически - очень легко вас уничтожить, это не даст ничего. Смысл? Потому, что вы снова можете появиться. Пусть это будет другое обличие. Понимаете? Допустим - та же самая обезьяна, которая сейчас обезьяна, в конце концов, она просто заменит ваше место. Понимаете? Ничто не изменится. Для того, чтобы уничтожить, надо вырвать с корнем. Как это сделать? Это нельзя сделать физически. Значит надо сделать в духовном плане, значит, вас надо просто развратить. Чтобы вы, сделали или множество новых законов, которые опровергали бы друг друга и создать хаос. Или наоборот - чтобы вы не признавали никаких законов, и будет тот же самый хаос, который вас же и съест. Или подкинуть вам какие-то ложные идеи, и идя по этим ложным идеям, вы будете уничтожать себя именно в духовном плане. Понимаете? Сколь множество религий, откуда пришли они и зачем они здесь нужны!? Для того, чтобы сделать вас рабами именно этой религии. Так? А для чего? Чтобы вы были воинами этой религии и бездумно выполняли все прихоти её. Понимаете? Вот вам и религиозные войны. Давайте пойдёмте дальше. Итак, задача вас уничтожить. Культ насилия. Очень просто. Просто надо дать вам понять, чтобы вы уверовали что ``чем сильнее, тем и правдивее''. Понимаете? Вот отсюда сейчас и идёт пропаганда насилия, ибо мафию можно победить не в духовном плане, а именно огнём, битвой, физически. Понимаете? И физически он доказывает свою правоту. Возьмите популярных ваших героев - всё они там делают именно на физическом плане. Давайте пойдёмте дальше. Что ещё можно сделать? Можно ещё добавить вам другие ценности, подать вам ценности ложные. И тогда, вы забудете прекрасные песни, и молодёжь будет одурманена именно простейшими-простейшими идеями. И Любовь, слово ``любовь'' у вас превратится просто… Понимаете? }
\people{(Белимов) В ``трахание, секс''. Да?}
\soul{Понимаете? А что у вас сейчас и происходит в искусстве? Понимаете - обесценивание ценностей! Что дальше!? Итак, мы добились того, что у вас есть культ насилия, а это значит, что вы уже можете, вы уже готовы уничтожить себе подобного. Плюс - ценности. Ложные ценности – они столь низки. А для того, чтобы молодёжь ещё и не думала, не напрягала свои мозги, потому, что это будет вредно для тех, кто хочет вас уничтожить, эти ценности будут преподаваться в таком очень простом, простейшем виде. И песни будут содержать всего лишь пару слов которые постоянно повторяются, а вы будете уже находить в этом : ``Ой, сколько здесь сказано!''. Понимаете?}
\people{(Белимов) Да-да-да.}
\soul{Давайте пойдёмте дальше. Ритмика. Та же самая - ритмика. Заметьте, что ритмика сейчас всё тяжелеет и тяжелеет. Возьмём хотя бы искусство. Идёт именно чистый ритм, и нет никаких плавных переходов. Нет именно той музыкальности. Грубо говоря, то, что делал раньше шаман с барабаном, то же самое происходит у вас в искусстве. Всё, то же самое. Разве только, что вы в хоровод не становитесь, и вы не видите этого шамана в маске. Понимаете? Что ещё? Что ещё нужно чтобы вас уничтожить? Обесценить понятие жизни! А здесь тоже есть множество вариантов. Итак, вы уже готовы убивать, но ещё не убиваете, ибо вы, всё-таки ещё понимаете, что это тоже человек. Вы говорите: - обесценить как? - Назвать его врагом. А вспомните 30-е годы. Человеку нужен был враг. Всегда и везде пропагандировалось что: ``Будь внимателен, около тебя может быть враг!'' }
\people{(Гера) Установку давали.(пси-зомбирование сознания. Прим.)}
\soul{Было? }
\people{(Ольга) Было.}
\soul{Было! Есть и сейчас! Но только это делается не столь открыто. Понимаете? Что ещё нужно, чтобы уничтожить вас? Что?}
\people{(Ольга) Да. Всё это делается, к сожалению, всё это делается…}
\soul{А вы? Что делаете вы? }
\people{(Белимов) Пока поддерживаем. Идём в русле.}
\soul{“В русле'' – вы бежите, вы не просто идёте. Вы бежите, вы торопитесь успеть, ``Я буду первым! Я буду избранным!''. Понимаете? Как вас обозлить? Как разбудить в вас зло? А оно есть у каждого. Очень просто. Поставить  вас в то положение. И что? И где, тогда, ваша доброта? Где ваше сочувствие? }
\people{(Белимов) Ну, да. Многие из нас уже готовы за автоматы взяться. Озлобление большое. А что тогда предпринимать сейчас надо людям духовного плана, всё-таки?}
\people{(Гера) Какой-то путь есть, такой, без кровопролития?}
\people{(Белимов) Или нас доведут до катаклизма, до коллапса - всё общество, и потом, начнётся уже расслоение, понимание того, до чего мы дошли.}
\soul{Тогда, вы должны были бы вспомнить Библию. Там же было сказано, что придёт время Сатаны. }
\people{(Белимов) Так оно что, как раз сейчас?}
\people{(Ольга) Конечно.}
\people{(Гера) А там же и сказано, что будет множество товаров, золота, корабли будут ходить кругом. Вот, пожалуйста!}
\people{(Белимов) Значит, сейчас пришло время Сатаны, время экзамена, время выбора. Это длительный период? У нас есть сведения что…}
\soul{Остановитесь!}
\people{(Белимов) Не надо, да?}
\soul{Давайте не будем, не предсказывать ничто. Ведь уже было вам сказано что: предсказывая события, и участвующие в этих событиях - уже лжёт. Понимаете? Извне кто придёт, он предскажет вам верно. Предсказатель же, который участвует в этих событиях, не может предсказать их достоверно. Понимаете? Вам же объяснялось это уже. А найдите здесь сейчас человека, который бы не присутствовал в этих событиях?}
\people{(Ольга) Нет.}
\soul{Его нет!  И поэтому, все эти предсказания могут только дать направления, но не больше. Они не дадут вам конкретно, что будет так это или так. Понимаете? Будет всего лишь только максимум совпадений, именно только направлений. Да! Вы идёте к катаклизму. Но будет этот катаклизм географический, физический, психологический или в духовном плане, этого нельзя говорить, потому что это всё взаимосвязано. Если это произойдет на геофизическом уровне, то простите, и психика ваша будет подорвана тоже.}
\people{(Гера) Ну, да, вообще-то.}
\soul{Если это произойдёт в духовном плане, то это будет обязательно и в геофизическом. Поймите, все катаклизмы создаются именно вами. Понимаете? И если, вы, напряжены, если вы больны, то, простите, вспомните Спитак (город в Армении разрушенный землетрясением. прим). Чем вам не доказательство? Что вам ещё нужно? А вы начинаете говорить, об итальянских ``сапогах'', которые должны исчезнуть, утонуть. Это глупо. Это всего лишь только ажиотаж. Это, опять же, создание ложных ценностей. И понимаете, если придёт предсказатель и скажет ``то-то, то-то'', - просто ещё одна ложная ценность и не больше. И все ваши попытки превратятся всего лишь только в одну из возможностей вас же и уничтожить. Вот когда говорят: благими намерениями. (Благими намерениями вымощена дорога в ад. прим.)}
\people{(Белимов) Экзамен должен привести к чему-то, каким-то выводам, они последуют?}
\soul{А вы его сперва сдайте, а уж потом посмотрим, что будут за выводы. }
\people{(Белимов) То есть - ещё неопределённость у нас с ситуацией, да? Неопределённость дальнейшая с человечеством? Я согласен вот с чем, - вот у нас экзамен суровый, обнажились чувства и люди разделились: кто-то ушёл в ад, погибли и так далее. Но может, останутся те, кто как-то стремится к улучшению Земной цивилизации, или все погибнут одновременно, так сказать, и правые и неправые - поганые люди? К чему тогда экзамен?}
\soul{Тогда  вспомните историю Вавилона. ``Найдутся хотя бы двое, - и будет спасён город''.}
\people{(Ольга) Содом и Гоморра.}
\people{(Белимов) Они не нашлись, да?}
\soul{Тогда вспомните. Вспомните это.}
\people{(Белимов) В Вавилоне не нашлось. А сейчас, может быть найдутся?}
\people{(Ольга Белимову) Вы так считаете?}
\soul{Старайтесь! Вам!}
\people{(Гера) Понятно. Скажите, насчёт большого парада планет, который ожидается…}
\people{(Ольга Гере) Ну, уже был. Зачем?}
\people{(Гера) Какой ``был''?! Он будет в 2003-ем, где-то вот так. Всё-таки - это совпадение, каких-то… в ``один столб'' всех этих вибраций, всех планет пойдут, в принципе..?}
\soul{Вам же говорили, что вы столь возвеличили себя, что решили, что любой ваш вздох, откликается звездой. Какие вы избранные, что звёзды специально созданы для того, чтобы руководить вами?}
\people{(Гера) Нет. Не о том я.}
\soul{Всё о том же, всё о том, что всё это вы превратили в обычную мистику. Понимаете? В религию. То, что было ценным - уже потеряно. Да, действительно, звёзды откликаются на ваши судьбы. А если быть точнее, то судьбы ваши связаны звёздами и наоборот. Нельзя сказать то или это. Но, поймите же, вы это превратили - в культ, вы превратили это - в театр! И множество гадалок теперь гадают ``если звезда повернула так, если так, то будет то-то, то-то, то-то''. А если ещё и совпало! А, как правило, всё это говорится общими словами, чтобы наверняка совпало. Вспомните, сколь много снов видели вы. И вспомните, какие вы помните! Те, которые совпали! А их было, всё-таки, гораздо меньше, чем было других. И вы создаёте культ. Вспомните: ``Если вы хотите, уничтожить меня и учение моё, то сделайте из меня религию''. - И всё! Этого будет достаточно!}
\people{А вы можете в этом состоянии сказать, какой психологический шок ожидает землян примерно через год? Какая-то общая болезнь, или какая-то ситуация, которая с человечеством сосуществует?}
\soul{Заметьте, что множество из вас, уже страдает некрофилией. Понимаете? Вы ожидаете что-то плохое. Некрофил! Некрофил, только и всего. Именно - некрофил. Понимаете?}
\people{(Ольга, Белимов) Да, смерть.}
\soul{Потому что, вы решили, что выход может быть только через смерть. Вот ваша жестокость. Это значит - что? Пропаганда, действует на вас, и вы не видите иного выхода, кроме как уничтожения плохих, и остаются только хорошие. Понимаете? А плохие – говорят  то же самое. Только, вы уже, должны быть уничтожены в ихнем понятии. Понимаете? Вот действие пропаганды. Это и есть оружие, которое… против которого вы бессильны. Понимаете? (теряется)}
\people{(Гера) То есть установки дают? Да?}
\people{(Ольга) Всё, это уже…}
\people{(Гера) Пока он выходит, лучше - ничего не говорить. (пока выходит из состояния прим.)}
  Обрыв.
\soul{…спрашивайте.}
\people{(Белимов) А сейчас, готовы сказать..? }
\people{(Ольга) Нет, просто мы уже как-то, во всяком случае, понимаем, что это вы, уже где-то…}
\people{(Белимов) Ну, вот с ситуацией, когда очень интересный монолог ваш был, о судьбе, которая нас волнует…}
\people{(Ольга) Нет-нет. Это не они.}
\soul{Как вы невнимательны. Вы не заметили, что когда разговаривали с обоими, то была разница в амплитудах. Когда говорили мы - было больше защиты. Когда говорил он – её было меньше. Сейчас же…}
\soul{(Гена) Как бы вам это объяснить… я не перебиваю их.}
\people{(Ольга) Угу.}
\soul{(Гена) Просто - не уступаю. Может быть из-за того, что просто – настырность, или вот - тема…}
\people{(Ольга) Угу. Ну, ты можешь дать им слово, так сказать?}
\people{(Белимов) Или перейдём на твои вопросы? Ты же интересовался перед сеансом. Тебя очень волнуют некоторые вопросы. Давай, их попробуем задать. Вот, нас действительно интересует, и тебя интересует, почему твоему сыну, первое, что в детском саде сказали, что он Сергей Иванов?  Откуда это пошло? Что это? Он действительно Сергея Иванова воплощение? Того.}
\soul{(Гена) Нет. Просто, дети более чувствительны, и он просто поймал мою мысль.}
\people{(Гера) Это абсолютно точный факт?}
\soul{(Переводчик) И дело в том, что было время, когда я постоянно думал о Сергее Иванове, о его сюжетах, да? И это так вжилось, что просто, он воспринял это. Воспринял, как какого-то другого ребёнка и немножко приревновал. И чтобы привлечь к себе внимание, он: - Папа! - Он же не видит другого ребенка, хотя он чувствует его. Понимаете? И он  это воспринимал, как к себе, и потому: ``Почему меня назвали Сергеем Ивановым?'' - То есть, не было прямой вот этой ассоциации на моего ребенка, что он Сергей Иванов, но он воспринял это так, как о нём.}
\people{(Гера) Ясно.}
\people{(Белимов) А ты что, в слух рассказывал сыну, про Иванова?}
\soul{(Гена) Нет.}
\people{Он просто подсознательно смог считать?}
\people{(Гера Белимову) Ну, мысленно можно считать, в принципе.}
\soul{(Гена) Да.}
\people{(Белимов) А-а, вот оно что.}
\soul{(Гена) Дети - они очень чувствительны к этим вещам.}
\people{(Белимов) Надо же! А если бы ты думал о какой-нибудь девушке с другим именем, то ты считал бы эту мысль, что ли?}
\soul{(Гена) Я этого не пробовал. Но, понимаете, дети - они очень хорошо чувствуют, и, допустим, вот эта вот начинающая вражда… Она только начинается между родителями в семье, да? Ребёнок это уже чувствует.}
\people{Скажи, вот допустим, твой сын чувствует тебя, а мать свою он чувствует?}
\soul{(Гена) Конечно.}
\people{То есть, для него в принципе больше/меньше связи с матерью, чем с тобой не существует, она одинакова?}
\soul{(Гена) Нет. Почему? С матерью, безусловно, связь будет больше.}
\people{(Гера) Угу.}
\people{(Ольга) Ген, вот скажи, можем мы, допустим, вот… Мы с предохранителями как-то вот разговаривали. Ну, ты помнишь, да?}
\soul{(Гена) Ну, что-то примерно. Примерно - с ними вы и разговаривали сейчас. }
\people{(Ольга) Ну, это не примерно, но, это другие, если можно сказать, ``маски''. Ну, так, да?}
\soul{(Гена) О, нет. ``Масок'' уже не было давно.}
\people{(Ольга) Может быть, это моё желание тогда сбылось? Я просто несколько раз уже думала, чтобы с ними поговорить.}
\soul{(Гена) Ну, это вообще-то не они были, но ближе.}
\people{(Ольга) Но ближе к ним. Ясно. Я просто подумала, что маски, в каком плане, - то есть, такие же жесты, жестикуляции, как, в общем-то, похожи.}
\people{(Гера) Скажи, насчёт степени защиты твоей, когда, так сказать, ``наши прежние''.. ну, сначала кто разговаривал, у них больше защиты, значит, они и чувствительней получается, чем вот, допустим, ты в сознании?  Или предохранители, потому что у них реакция защиты меньше. То есть, им меньше надо защищаться, потому что они…}
\people{(Ольга) Они ближе к нам.}
\people{(Гера) Да, ближе к нам.}
\soul{(Гена) Нет. Понимаете, как получается? Вот, кажется, в прошлый раз, да? Скажем, что не сейчас сеанс был, а до этого, да? - Было наоборот, когда говорили они - амплитуда была больше, а вот, когда говорил я - амплитуда была меньше. Потому, что я, вроде бы как, ближе к вам, поэтому, мне надо меньше от вас защищаться. А если быть точнее, защищаться-то также, но просто…Я… такой же, как и вы. Я знаю более точный способ, как с меньшей же энергией защититься от вас. Понимаете? Это - первое. Второе - здесь используется ещё защита своего организма, оно включается, понимаете? А вот когда они, они же говорили тогда, что они посторонние. Мозг воспринимает их, как посторонних, он ставит против них защиту, понимаете? И им приходится защищаться и от меня, и от вас. }
\people{(Ольга) Но это не ``предохранители'', это же наши ``первые''.}
\soul{(Гена) Понимаете в чём дело… Дело в том что, вот этот принцип ``не навреди!'', заставляет соблюдать очень много…}
\people{(Ольга) Этические правила, нормы. Да?}
\soul{(Гена) Да. И поэтому, чтобы не вмешиваться грубо и в то же время как-то всё-таки произвести диалог… не будет обманывать, не будет превращаться в какую-то… Хотя это можно, конечно, если не иметь физики, можно создать любую физику, понимаете? Придти в любую физику и под неё подстроится, потому, что не имеешь своего потенциала и можно легко воспринять любой другой. Можно сделать, так, что, в принципе, они будут частью меня, и мозг даже не сможет их идентифицировать, что они чужие или вообще пришли ещё откуда-то. Вообще - идентифицикации никакой абсолютно не будет. Но они этого не сделают.}
\people{(Лена) Ну, правильно.}
\soul{(Гена) Почему? Потому, что они тогда возьмут часть меня, и они с этой частью должны будут потом уйти.}
\people{(Гера) Скажи, если я не перебил, конечно… Не перебил? }
\soul{(Гена) Я слушаю.}
\people{(Гера) Скажи, пожалуйста, они когда-то говорили, что ищут способ с нами соединиться, что, так сказать, у нас какая-то сила. Сила чего? Сила действия в этом мире, что ли получается? То есть - физика? }
\soul{(Гена) Нет. Понимаете… дело в том, что мы привыкли думать чисто на материальном уровне, и вот когда речь идёт о воссоединении, мы, почему-то искажённо это воспринимает. Понимаете, мы должны воссоединиться в духовном плане, в чисто духовном, даже не в психологическом. В психологическом плане у нас много собеседников. Это когда мы - друзья. Понимаете? Чаще всего, к сожалению, друзья - это всего лишь только в психологическом плане, совместимость в психологическом, физическом плане. Понимаете? Вот, ещё что интересно, например, маленький ищет друга большого или большой ищет друга маленького, понимаете? По принципу покровительства, дополнения, как говорится физически, да? Ну, и соответственно, значит, и психологически, а в духовном плане очень мало друзей. Многие так вот и проживут, имея множество друзей, а  духовно не будет ни одного. И вот, соединение семьи должно быть именно в духовном плане, понимаете? Всё остальное, это будет просто - цирк, театр…}
\people{(Гера) Ложь.}
\soul{(Гена) Но это нельзя назвать ложью, потому, что в принципе мы можем не заметить этой лжи. Будет просто постоянное неудобство, понимаете, вот это нежелание, не уют, попытка уйти, попытка соврать, то есть - попытка просто подстроится, чтобы было как можно меньше вреда. А вот эта попытка подстроится под кого-то, это уже говорит о том, что нет духовного соединения. И когда говорят о соединении духовном, это может быть абсолютно, что угодно. Понимаете? В духовном - можно соединиться даже со стулом. }
\people{(Гера) Как это? }
\soul{(Гена) Не обязательно даже с человеком. Вот. Психологически…Психологически - я тоже могу соединиться с этим стулом. Понимаешь? Я просто люблю эту вещь, допустим, ``у меня любовь к этой вещи, у меня любовь к этим цветам, я люблю это платье''. Или вот, этот костюм больше всего ко мне подходит, я чувствую в нём более комфортабельно, более удобно. Понимаете? Вплоть до того, что даже будет излечивать этот костюм, понимаете? Ну, это в психологическом плане. А в духовном плане, это не значит, всё, что переживает стул, скажем, буду переживать и я - если на стул садятся, то и я буду чувствовать, что вроде как на него сели - и на меня сели тоже. Понимаете?  Нет, это, как раз, просто вот - наш вымысел. Просто действительно вот будет – единение. Понимаете? Не то, что я буду всем кричать, что это стул живой, понимаете? А вот именно… как бы вам сказать …но это трудно описать. Просто…}
\people{(Ольга) Да это трудно и…}
\people{(Гера) Ты будешь чувствовать себя стулом?}
\soul{(Гена) Нет, я не буду чувствовать себя стулом, и стул не будет чувствовать. Понимаешь?}
  (Конец первой части…. Начало второй части контакта)
VG – 1996.11.02_-_02 
\people{**}
\soul{Вообще-то, множество сонников врут. Понимаете, как составляются сонники? Просто берётся средне-статическое, несмотря на конкретные случаи и описываются. Вот и получается тогда, что, допустим, первого числа сон сбывается - тогда-то и тогда-то, да? 15 – го числа месяца, например, он не сбывается – пустой сон, да? А вот закономерность-то в чём? В принципе, закономерность существует, это безусловно. Но, чаще всего, понимаете, эта закономерность вообще-то ложная потому, что эта закономерность, опять же, придумана нами. Решили мы, что середина месяца - это вроде как - переломный момент, понимаете, начало месяца, это было вроде как - ``все большие дела начинаются с первого дня недели'',- так? ``А ещё бОльшие дела, только уже…'' давайте тогда по принципу этому, начинаем, как говорится ``с первого дня месяца'', да? Вот и получилось, что вроде как 15-ое число у нас вроде бы такое, знаете, и начало прошло уже, да? И уже дело к концу, то есть такой момент… И чисто психологически мы решили, что 15-ое число это, значит, все сны - это пустые сны. Понимаете?}
\people{(Ольга) Да.}
\soul{И вот, вы решили, что это будет так. Тем более сейчас, понимаете, сейчас в чём? Очень много…Пресса помогает, понимаете, вот этому вот массовому гипнозу. Смысл в чём? Чем лучше, чем красивее ты сейчас опишешь любую белиберду, но чем красивее она будет написана, знаете, она найдёт обязательно своих последователей, - один-два. Раньше, это было до десяток, ну, пусть будет - сто человек. Создастся община, которая рано или поздно потом всё-таки ``рассосётся''. А сейчас? Если есть возможность прессы! В итоге получается - пресса помогает. И вот, эта вот дикая идея, которая выдумана чисто автором,- не обязательно просто с дурных побуждений, а просто, скажем - его фантазией,- эта идея превращается и тиражируется. И что в итоге? В итоге - уже сто тысяч человек, купившие эту книгу, знают эту идею, и из них найдётся, наверняка, те 50 тысяч, которые скажут: ``Да, это верно!'',- и разнесут дальше. Правильно? }
\people{(Лена) Да.}
\soul{Вот, пожалуйста. Вот, разошлась ложная мысль. Раньше это было сложнее. А ведь, заметьте, раньше-то и религий было гораздо меньше же. Ну, сколько мы можем перечислить? Ну, давайте на время…. Ну, давайте возьмём - тысячу лет назад. Как вы думаете, сколько было религий по миру? Много или мало?}
\people{(Ольга) Ну, если можно по названиям. Можно так? Давайте это…}
\soul{Давайте без названий. Просто скажем много или мало?}
\people{(Ольга) Ну, не много. }
\people{(Гера) Штук пять.}
\soul{По мировым?}
\people{(Ольга) В Индии там где-то.}
\soul{По мировым – да. А так, каждое племя имело свою религию, и причём не распространяла её, правильно?}
\people{(Ольга) Да.}
\soul{Нечем было. Если какой-то одинокий попадал в плен к другому племени, вряд ли он мог поменять мировоззрение того племени и поменять религию и бога. Ну, хорошо, тогда смотрите, и вы всё-таки найдёте закономерность, что те разобщённые племена, в принципе, имели одних и тех же богов.}
\people{(Гера) Ну, у них какие-то связи были – культурные…ну, там…}
\soul{Ну, минуточку. Связи? Связи тогда были только пленными. И было время…}
\people{(Ольга) Особенно в Европе.}
\soul{Тогда, давайте посмотрим. Итак,  поклонники огня. А почему поклонники огня? Почему, именно это было распространено и довольно сильно?}
\people{(Лена) Огонь - жизнь, тепло, пищу можно разогреть.}
\people{(Гера) Страх.}
\people{(Ольга) Огонь - это тепло, это молния, которая разжигает…}
\soul{Ибо, это было самое сильное…Правильно? }
\people{(Ольга) Да.}
\soul{Правильно. И потому, бог огня мог в разных племенах быть совершенно разным. Если в одном племени это было наказуемо, а в другом, бог благодарил за то, что он это сделал, но, всё-таки, это был бог огня. Так?}
\people{(Лена) Ну, да.}
\soul{Дальше. То же самое - бог солнца. А, простите, а бог пня? А бог камня? А вот богов камней, было гораздо меньше, ибо камень не обладал той силой, который обладает огонь или солнце – той важной частицей жизни. И поэтому, бог камня или бог реки… Бог реки – да. Если вы живёте на этой реке, для вас очень важно, потому что река даёт вам пищу и воду. А для других племён, которые не жили на этой реке, другой бог. Пускай он будет ``речной'', но это будет - другой бог. И здесь, именно здесь хранятся корни религиозных войн. Ибо встречаются два племени двух разных рек:  -Нет наш бог сильней! – Нет, наш бог сильней!}
\people{(Гера) Потом выясняют, да?}
\soul{Выясняют. Но и практически, они выясняли-то не какой бог сильней, а просто им нужно было - завоевать эту землю. Только и всего. Но, для того, чтобы придать воинский дух, сюда обязательно нужно вмешать Бога. Какой из воинов пойдёт воевать, если Бог против него. Нет? - Нет. И потому - бог реки, и - обычная вражда, обычная драка за борьбу земли превращается в религиозную войну. И что в итоге? А итог таков - давайте возьмём… Назовите мне страну, которая никогда не меняла своих богов, даже будучи завоёванной другой страной.}
\people{(Гера) Индия.}
\people{(Лена) Мусульмане, наверное, какие-нибудь.}
\soul{Нет. Китай.}
\people{(Лена) А-а…Да, наверное.}
\soul{Китай, Япония. Ибо захватчики принимали их веру. Захватчики постепенно принимали веру. В чём сила? Что, бог оказался сильней или дух?}
\people{(Ольга, Лена) Дух, наверное.}
\soul{Вот! Итак, религиозные войны - это что?}
\people{(Ольга) Выяснение отношений…}
\soul{Это - всего лишь оправдание самой войны. Не будем же мы кричать: ``Давайте завоюем вот этот колодец, потому что в нём есть вода! А у нас нет воды – у нас колодец засорился''. Понимаете, вроде как-то и неудобно, как-то это уже и такие времена дикие, да? Всё-таки цивилизованные люди. - Нет, братцы, что вы? Бог! Бог сказал: ``Избавиться от иноверцев!'' - А это идея! За неё уже пойдёт больше.}
\people{(Ольга) В общем-то, что мы и делаем постоянно. Да, в общем-то, бесконечно оправдания какие-то там находим. Действительно, жадность наша доводит вот, до такого лицемерия. И лицемерим себе, обманываем себя, кого мы обманываем…}
\soul{Давайте вернёмся, в нашу страну. Итак – Россия. ``Народ, народные массы, кухарки - всё это правит страной''. А зачем нужен Бог?}
\people{(Ольга) Да…(смех)}
\soul{Знаете, в нашей конституции Богу места не предусмотрено. Там же не может быть, чтобы и, простите за выражение - ``коммунистический аппарат'', и ещё и Бог?! Да кто из них главнее-то?  - Да нет братцы, давайте уж бога в стороночку, и возьмём вот: ``Мы и будем хозяева Земли! Мы строители своей судьбы! И зачем нам Бог? В нашей теории он не укладывается''. - И что? Мы создаём новую религию.}
\people{(Ольга) Да.}
\soul{Слово ``Бог'' мы заменяем на…}
\people{(Ольга, Лена) ``Коммунизм''.}
\people{(Гера) ``Партия''.}
\people{(Лена) ``Партия'', да.}
\soul{“Партия''. Слово ``рай''?}
\people{(Ольга, Гера)  ``Царизм, коммунизм''.}
\soul{Прекрасно! Какое ещё слово осталось?}
\people{(Лена) Слово ``Бог''?}
\soul{Слово ``ад''.}
\people{(Лена) Ага.}
\soul{Что у нас будет ``адом''?}
\people{(хором) ``Капитализм''.}
\soul{Правильно. Так, простите, в чём различие?}
\people{(Ольга) Да, ни в чём. (вздыхая)}
\soul{Расцвет христианства - это времена инквизиции, расцвет коммунизма - времена репрессий. Одно к одному. Всё одно и то же. Всё повторилось! Просто мы стали на виток повыше, или мы может быть просто - поменяли имена. А это уж, кому как нравится. Итак, почему-то все решили, что чем выше по спирали, тем это, значит, и ``выше''. Так простите, а можно падать и ``вверх'', а можно падать и ``вниз''. В любую сторону можно падать. Ну, давайте не будем рассуждать, падаем мы сейчас или не падаем. Давайте пока остановимся на том, что мы…ну, пусть будет ``по спирали''. Пусть, если вам будет легче, пусть это будет ``спираль'', но не круг. Спираль, да-да, вы изменяетесь, конечно. А теперь, давайте попробуем, докажем, что это всё-таки спираль, а не замкнутый круг. Найдите мне различия.}
\people{(Ольга) Ну, спираль - она имеет продолжение…}
\soul{Мы говорим не об этом.}
\people{(Ольга) А о чём?}
\soul{Мы проходим урок геометрии?}
\people{(Ольга) Да, нет.}
\soul{Итак, найдите мне различия. Именно, времена инквизиции и времена расцвета коммунизма. }
\people{(Гера) А-а, ну, если так сказать: больше образованней это делать – эти витки.}
(Лена) Образованнее - убивать, что ли?
\people{(Ольга) Ну, единственное что, может быть, разница это будем говорить, что - науки… Так сказать уже, с научной точки зрения всё это начинают объяснять, ну, будем говорить так - времена репрессии.}
\soul{Тогда, вспомните, как была названа наука? Вспомните, это некультурное слово, что относится и к политике.}
\people{(Ольга) Угу. Наука - это оружие для чего-то там.}
\soul{О, нет. Мы имели в виду совсем другое. Давайте, назовём мягко - пуританство. Итак. Вы остановились на науке. Хорошо. Вы, сейчас научно можете доказать, что коммунизм – Бога нет, что коммунизм - должен и будет, и он неизбежен. Вы это можете научно доказать? - Нет. Как и не можете доказать и обратное. Ну, хорошо, тогда давайте вернёмся во времена инквизиции. Могла ли тогда наука доказать?}
\people{(Ольга) То же самое. Нет.}
\people{(Лена) Нет.}
\soul{Нет. Так что, различие в науке?}
\people{(Ольга) Нет. Не в науке.}
\soul{Ага. Вы дали версию ``цивилизованнее''.}
\people{(Ольга) На счёт цивилизации, здесь, я бы не сказала. Подразумевается, как я понимаю цивилизацию, цивилизация – это, в общем-то, ну, более образованнее – общество, как бы.}
\soul{Ну, да, конечно, ``более образованнее''. Какая разница, вас заливали в смолу или вас убивают электрическим током? Смысл тот же.}
\people{(Ольга) Ну, да. Нет, я не об этом.}
\soul{Но, это почему-то называется ``цивилизованнее''. Вы придумали, подумать только! – смерть! Какую смерть?!}
\people{(Ольга) ``Гуманную''.}
\soul{“Гуманную'' смерть! Видите ли, люди все умерли, а вещи остались. Здорово?}
\people{(Ольга) Здорово.}
\soul{Здорово. И это называется – ``гуманное'' оружие.}
\people{(Ольга) Да-а. Назвали бы не так.}
\soul{Так, гуманизм к чему проявляется у вас - к вещам или к людям?}
\people{(Ольга) Ну, выходит - к вещам.}
\soul{Ну, значит, вы ``гуманно'' отнесётесь к вещам, потому что вы их не полюбите. А для чего? Да, что бы вещи вам достались, да? Тот же самый грабёж, только в более массовом характере. Итак, вы не сумели доказать?}
\people{(Ольга) О более цивилизованности?}
\soul{Так, в чём же разница тогда?}
\people{(Ольга) Да, ни в чём.}
\soul{Хорошо, давайте так. Давайте вернёмся во времена Мабу. Итак, возьмём Мабу, и возьмём любого из нашего общества сейчас. И в чём же будет различие? Только что - тот образованнее? А как у вас - образованнее? }
\people{(Ольга) Да.}
\soul{Мабу умел прекрасно точить камни, он прекрасным был семьянином, он прекрасно умел делать то, что ему было нужно и необходимо для жизни. Что делаете вы и сейчас. Вы прекрасно управляете компьютерами, машинами, потому что вам это нужно, а от того, что у него нет машины, это не значит что он - дикарь.}
\people{(Ольга) Да-да, это тоже.}
\soul{Ну и что? Опять возвращаемся к ``гуманному’ оружию? Что вы называете дикарём: если он не умеет писать, если у него нет машины,  если у него нет техники, если у него нет завода. ``Ах! Животные – глупы! Как может дельфин быть умным, если он не строит заводов?'' - А может он умный от того что он их не строит!}
\people{(Ольга) Да, это верно.}
(Щелчок - вкл/выкл записи)
\soul{Когда-то римляне погибли от свинцовой посуды. Интересно, от чего погибните вы?}
\people{(Гера) От ядерного.}
\soul{А в чём разница?}
\people{(Гера) Да ни в чём. Погибнем – результат.}
\soul{В удобстве. Там хотели удобства - создали водопровод. Сейчас вы хотите сделать удобства - создаёте электропровод, если вам ближе, чтобы легче было сравнивать. Итак, мы пришли к тому, что – бег по кругу.}
\people{(Ольга) Выходит так.}
\soul{Выходит так. Так, как его разорвать? Как разорвать так, чтобы вы, потом не превратились в обломки, упавшие с этого обрыва? Как, вот этот вот круг превратить в спираль? А разрывать придётся. Иначе, если вы, не разорвав, сделаете спираль,- даже геометрически - вы получите замкнутую… спираль, а это значит, что поднявшись ``наверх'' вы не заметите, как вы опять опуститесь ``вниз''. А это значит, вам и - первые, и вторые, и третьи, и пятые.}
\people{(Гера) Цивилизации, в смысле - уже ``шлёпались'' в эту ``лужу''?}
\soul{Да, тот же принцип.}
\people{(Гера) Интересно, а вот… Ну, Рим, да. Ладно, так сказать, наша цивилизация ещё худо-бедно… А эти…как их…Ну, до нас цивилизация какая была?}
\people{(Ольга, Лена) Атлантида.}
\people{(Гера) Атлантида.}
\soul{Разве, не рассказывали вам о ней? У вас было очень множество интересных версий.}
\people{(Гера) Да, версий много…}
\soul{Оправдывающих атлантов или проклинающих их.}
\people{(Гера)А вот, истину-то мы не знаем.}
\soul{Вот, смотрите. Итак, давайте, рассмотрим  версии…ну, давайте скажем: версии прославляющие их. Во-первых, - зачем нужно было прославлять? Зачем вам сейчас современникам нужно прославлять атлантов?}
\people{(Лена) Ну, чтобы у них чему-то научится.}
\people{(Гера) Ну, вроде, они там умные были. Знали, умели много чего, чего мы не умеем.}
\soul{А-а! Чтобы вашу глупость – оправдать. Понимаете, вы же множество потеряли. Очень множество знаний потеряли из-за чего? Хотя бы - из-за создания и ``шаганий'' этих религий. Очень много вы потеряли, и теперь вы ищете оправдания. ``Вот атланты умели и если бы не этот несчастный метеорит, если их море не утопило, то мы сейчас были бы - Ого-го как!''  – То бишь, оправдание самой своей глупости, что ``если бы нам не помешали, мы бы уже христианство развивали бы где-нибудь на Марсе''.}
\people{(Гера) Это является в следствии…}
\soul{Мы не против христианства, а мы против ваших принципов распространения ваших религий.}
\people{(Ольга) Всё верно.}
\soul{Понимаете? Вы пришли, и насильно…}
\people{(Ольга) Навязываем.}
\soul{Итак. Оправдываете, вы для чего? }
\people{(Ольга) Ну, чтобы нам удобнее было.}
\soul{Да-а?}
\people{(Гера) Верить, наверное, в то, что делаешь.}
\soul{Ну, зачем же?}
\people{(Гера) Раз уж, верить во что-то надо…}
\soul{Ну, зачем же? Просто вы хотите оправдать себя. Вы хотите оправдать, спрятать своё прошлое. Изменить не самих себя, а природу. Понимаете? Или богов. ``Боги воспротивились знаниям атлантов, ибо атланты стали знать более богов и хотели сравняться с ними.'' - Цитирую.}
\people{(Гера) Угу.}
\people{(Ольга) Неужели так? Я где-то, что-то не читала, чтобы атланты знания получили больше, чем боги.}
\people{(Гера) Не ``больше'', а сравняться.}
\people{(Ольга) Нет, не сравняться…}
\soul{Вы читали. Итак. Вот – одна версия. ``И боги, разгневавшись…'' - итог вы знаете. }
\people{(Гера) Ну, да.}
\soul{И для чего? Чтобы оправдать себя, оправдать вашу неповоротливость. И в то же время, создать какую-то маленькую цель, ибо, ``раз древние могли, то и, значит, и мы можем!'' А бедные атланты теперь чуть ли не летают по небесам, потому, что вы мечтаете теперь стать великими, вы говорите о духовном, а духовное в вашем представлении - это чтобы у вас ``росли крылышки'', чтобы вы могли перемещаться ``куда хочу, куда зачем'', посмотреть в магазине в Нью-Йорке. Так?}
(Смех)
\people{(Гера) Ну, да.}
\soul{Так. И при этом, желательно, бесплатно. А бесплатно, это если только самому полетать.}
\people{(Ольга) И ещё и украсть что-нибудь.}
\soul{Ну, это уже – телекинез.}
(Смех)
\soul{И вот, смотрите, -  вы, одушевляете атлантов, уже приписываете им то, что они никогда, бедненькие, этим-то и не страдали. Ну, а чтобы это не выглядело как-то…слишком уж так, похоже – что, это вот люди, да, - ``вдруг летают, а вот мы не летаем''. Что-то обидно немножко получается, понимаете,¬- какие-то древние летали, а мы не летаем! Ну, тогда давайте, мы их сделаем высокими и голубыми, чтобы они хотя бы немножко от человека отличались'', понимаете? ``Не наша же беда что у нас порода-то такая, вот они конечно зелёные там были, голубые, они поэтому летали'', понимаете? ``Ну, Бог нам этого ещё не дал. Ну, вот поголубеем, позеленеем и мы тоже летать начнём''.}
(Смех)
\people{(Ольга) Скажите, ну, вот смотрите, вот конечно теперь, у меня, например, смятения, то есть, какие-то противоречия, и сомнения тогда возникли, что мы читаем всякие там…}
\soul{Ну, подождите.}
\people{(Ольга) Ага, ну хорошо.}
\soul{Мы когда-то говорили, как говорится - возвеличили атлантов. Но существует же теории и унижающая их. Самое первое, что вы придумали…}
\people{(Ольга) ``Они почернели от греха''.}
\soul{А какой был грех? Самый близкий грех к вам.}
\people{(Ольга) Ну, наверное…}
\soul{Ах, ``они были сексуальные маньяки''. – О-о! В одном месте они приравниваются к Богам, в другой версии они - сексуальные маньяки. Интересно, какая из них верная? Интересно было бы знать. Ну, ладно, пока не будем говорить, какая из них верная. Итак, это сексуальные маньяки. А вот зачем мы их ругаем? А чтобы взять и показать – ``а вот мы-то не маньяки, мы по добровольному желанию!''.}
\people{(Гера) Согласен.}
\soul{Понимаете? ``Вроде бы, мы как бы и получше их – древних.  Ну, как? Они - дикари, а мы - современные люди. У нас вот – наука. Кто-то там говорил, конечно, что они там чё-то летали, но, это, скорее всего, на пароходах”…}
(смех) 
\soul{И всё. ``Ну, у нас тоже, и мы тоже сейчас умеем летать. Но, зато у нас нет вот этого сексуального маньячества, и, значит, нас Бог не покарает, и значит, мы будем жить вечно, и конца света не будет''. - Вот! Удобно?}
\people{(Ольга) Удобно.}
\soul{“А им так и надо! Пусть, окаянные, не делают чего не нужно''. - О-о! Так, теперь мы хотим узнать истинное.}
\people{(Гера) Да, желательно бы, конечно.}
\soul{Интересно. Сейчас, мы вам скажем то-то, то-то, то-то. И, это будет истина, провозглашённая  нами. А мы можем провозгласить эту истину в двух вариантах. Или так чтобы она вам не понравилась, и вы сразу же откажетесь от неё – это нам будет очень удобно потому, что вроде как вы нас уговорили, и, в то же время, мы тайну вроде бы как бы и не раскрыли, потому что вы тут же и забудете, потому что она вам противоречит, - а можем сделать наоборот -  набрехать вам с три короба, но так, что вам очень понравится, и мы создадим ещё одну Атланту, которая будет больше вам нравиться.}
\people{(Ольга) Да это наш опыт, наверно. Тоже истинный, который мы не знаем.}
\people{(Гера) А как можно ``истинно набрехать''?}
\soul{Можно. И очень даже просто. Достаточно просто поменять местами слова. И останется истиной…Что - истина в вашем мире? Что? – Одна из версий. Только и всего. }
\people{(Гера) ``Точка зрения'' получается.}
\soul{Да. Ибо, смотрите – светофор, -  горит красный цвет. А если мы превысим скорость? Как вы думаете измениться спектр?}
\people{(Гера) Ну, да. Немножко.}
\soul{Изменится. И не немножко, мы можем сказать, что он – зелёный. Уже был такой случай, когда уже полисмен был обманут этим. Так, какой же цвет всё-таки горел? - И тот и другой. Всё зависит от скорости вашего мышления, перемещения или чего угодно, чего хотите. Вот, смотрите, вы ищете рай. А вам говорят – рай был на Земле. Вот только Земля имела свойство не вращаться, вокруг своей оси. Так? Луна же не вращается.}
\people{(Гера) А-а! Ну, то есть, она была Луной до Солнца.}
\soul{И в итоге: вот вам - рай, вот вам - ад. И, соответственно, ад - обратная сторона. Так? Как вы думаете? - Смены климата  - никакого.}
\people{(Гера) Ну, да.}
\soul{Вот вам - рай, вот вам - ад. Строго, чётко - всё ограниченно, всё что хотите. Но, пришли внешние силы, вы назвали их - богами. Встряхнуло нашу матушку Землю, и она завертелась. И появилась смена климата, появились: года, месяца, дни, которых не было ранее… и, всё-таки, рай оставался. Почему? Да ибо, вы должны были бы помнить, что всё-таки Бог сделал за семь дней, а значит, всё-таки уже существовал календарь. }
\people{(Гера) Ну, да.}
\soul{Тогда, почему вы так невнимательны, и не можете в этих строках увидеть, что календарь всё-таки был создан уже? Подумайте сами: ``В начале была тьма, и дух носился над тёмными водами”… Помните?}
\people{(Ольга) Да.}
\soul{О ``времени'' не было сказано ничто, время не существовало. Не доставало, правда, и рая и ада. Давайте немножко дальше.}
\people{(Ольга) А подождите! А тьма? Это о чём говорит? О темноте, или о тьме как о ``много всего чего было''?  ``В начале была тьма''.}
\soul{Да. Можно, двояко думать, но дальше есть подсказка: ``И дал бог Солнце''. ``Да будет свет!''. Значит, речь, всё-таки, шла о тьме?}
\people{(Ольга) Да. Значит, солнца не было?}
\people{(Гера) Нет. Просто другая сторона Земли была, наверное, скорее всего.}
\soul{Интересно. Но, что было бы всё-таки верным?}
\people{(Гера) Ну, скорей всего - другая сторона Земли.}
\soul{А почему бы не так, что - не было Солнца, а Солнца не может быть по одной причине: или Солнце пришло, а потом возникла солнечная система, или Земля пришла в солнечную систему.}
\people{(Ольга) Да.}
\soul{А как вы думаете? С такой точки зрения вы не рассматривали Библию? Что может Земля извне пришла. Или может быть Луна, что более похоже, пришла извне, и вы были жителями той Луны? }
\people{(Ольга) Ну, это уже вы нам говорили.}
\soul{И, как говорится - ``упали с Луны''.}
\people{(Гера) Не зря же говорят: ``С Луны упал что ли?'' - А ведь в этом что-то же есть? }
\people{(Ольга) Ну, недаром же говорят, что Луна - это осколок какой-то планеты в которой…}
\people{(Гера) На осколок не смахивает вообще.}
\people{(Ольга Гере) Осколок другой планеты!  }
\soul{Нет. Это осколок.}
\people{(Лена) Осколок?}
\soul{Итак. У вас есть, всего восемь строчек. В эти восемь строчек - уже содержится множество информации. А теперь, давайте подумаем, кто написал эти восемь строчек? Ведь, простите, очевидцев в те времена наверняка не могло остаться.}
\people{(Гера) Да. Как говорится ``и слону понятно''.}
\soul{Сам Бог писал? Ну, как кроме него, вроде как больше никого и не было… }
\people{(Гера) Ну, это, наверное, сознание человека уже переработало информацию, уже записало это, относительно Земли.}
\people{(Ольга) Не человека, а кто был тогда.}
\people{(Гера) Ну, человек и писал.}
\soul{Угу, и значит, вы уже спорите. А вы не можете допустить, что четвёртая раса писала?}
\people{(Ольга)  Может. Почему? Может быть. Ну, человек ведь растерял все знания, которые раньше были ещё до человека.}
\people{(Гера) Вы хотите сказать, что на Земле было всё пока…}
\people{(Ольга)  И, в общем-то, он, наверное, то и…}
\soul{Вы называете себя - пятыми?}
\people{(Ольга)  Да-а.}
\soul{Значит, было уже четыре.}
\people{(Гера) Нет, я имел в виду, что четвёртые именно, вот, пришли сюда с Землёй. Если судить по версии, что она пришла. }
\people{(Ольга) Нет.}
\people{(Гера) В солнечную систему именно.}
\people{(Ольга) Нет. Ты не прав.}
\soul{А мы говорили вам о Луне ещё…}
\people{(Гера) Угу.}
\people{(Ольга) Луна пришла потом, после Земли.}
\people{(Гера) Да.}
\soul{Давайте уж договоримся. Люди уже были, но не было ещё Луны. Рано или поздно это учёные ваши докажут, что Луна потом пришла, всё-таки, позже. И вы сейчас ведёте споры, сколько лет человеку. Вы всё увеличиваете-увеличиваете его жизнь, но пока вроде как ещё и не дошли.}
\people{Ну, человек, наверное, как, существовал…но мы не можем назвать эту цифру.}
\soul{Ну, понимаете, учёных не интересует ``до человека''.}
\people{(Ольга) Ну, да.}
\soul{Его интересует прошедшие расы, его интересует именно сейчас, именно человек в этом вот обличии, но немножко изменённый. Но главное - найти папу. Кто же был первый папа. И когда же он родился, и сколько бы ему было бы сейчас. Вот что интересует учёных. }
\people{(Гера) Это Адама что ли ищут?}
\soul{Адам? А если вы ищите первобытного человека, это и есть Адам. Вы согласны?}
\people{(Гера) В принципе – да, наверное. Скажите, а вот до прихода луны, люди были на Земле…это люди были или всё-таки они…не то, что мы имеем в виду - людьми? Ну, может быть…}
\people{(Ольга) Ну, ведь, наверное, существует не только человек? Наверное, есть ещё какие-то другие живые жизни будем так говорить.}
\people{(Гера) То есть, есть человек – Человек, человек там – такой человек, ну, типа - человек….не знаю, как сказать… ближе к природе. Может, ``человек-животное'' было? То есть - согласовывали, гармонировали как-то с природой земной, а мы тут свалились с Луны и начали, вот, строить города там всякие…}
\soul{Давайте скажем то, что гармония с природой это ещё не значит вверх.(идёт развитие прим.) Понимаете, животные гармонируют с природой, но она вроде как и… не превращается в ``лучистое человечество''. Здесь не только гармония с природой должна быть. }
\people{(Лена) Но ещё - сознание.}
\people{(Гера) Ну, да. Сознание.}
\soul{Это - первое. И второе – ну, конечно, вы – цари. Вы не можете признать, что вы не единственные на этой Земле. Вы – цари. По настоящему – мно-же-ство. А кто ``вы''? Что ``ваше тело''? Что представляет ваше тело? – Сообщество мелких животных! Только и всего. Ваше тело, это всего лишь скопище клеток. И, заметьте -  ``клеток”…Как вы называете?  -  Такое унизительное слово… А?}
\people{(Гера) Амёба  что ли?}
\soul{Вспомните биологию.}
\people{(Гера) Инфузория?}
\soul{Ну, зачем ``инфузория''! У вас есть ещё – от слова ``первый''.}
\people{(Гера) Микробы?}
\soul{Микробы, это уже…}
\people{(Ольга) Бактерии?}
\soul{Вы вспомните.}
\people{(Гера) Молекула?}
\people{(Ольга) Нет.}
\soul{Вы, всего лишь, - сообщество клеток. Тех самых примитивных кле-то-чек. Всего лишь! Всё ваше тело. А вот, давайте, сравним вас и государство. }
\people{(Ольга) Ну, то же самое.}
\soul{Всё, то же самое. А кто правит этим государством? Кто? Кто-то ведь правит? Государство вроде бы не рассыпается, у неё существуют свои границы, - заметьте! А?}
\people{(Ольга) Да-да.}
\soul{Государство имеет свои границы.}
\people{(Ольга) Ум.}
\soul{Ум? }
\people{(Лена) Вообще-то - нет.}
\soul{Ум? Надо сказать, мы не заметим ни в одном правительстве ума. }
\people{(Ольга) Ну, да. Это мы тоже не заметили. (смех) Ну, выходит…}
\soul{Значит, существует какое-то правительство. Индивидуумы - те же самые человеки, но они, вроде бы как – жрецы, правительство этого государства. Если провести аналогию, вот общество клеток – человек, и несколько индивидуумов – клеточек, являются…}
\people{(Ольга) Главными, управителями.}
\soul{Управителем этого тела. И, вот эти вот главные клеточки, мы назовём ``монадой'', и, в то же время, ни под одним микроскопом мы их что-то не находим. Что-то не стыковывается, не правда ли?}
\people{(Ольга) Ну, почему эти клеточки мы монадой назовём?}
\soul{А что управляет государством? – Правительство. А что такое ``правительство''? – Несколько выбранных индивидуумов. Что такое человек? - Это ``государство'' клеток. И, значит, ими управляют по полной аналогии, ими управляют индивидуумы - несколько выбранных ``клеточек''. А раз так, то получается - вот эти клеточки и называются ``монады'', которые дают жизнь.}
\people{(Гера)  Ну, да ,в принципе. По аналогии.}
\soul{По аналогии. Что-то не получается?}
\people{(Ольга) Нет.}
\people{(Гера) Не получается, в том плане, наверное…}
\soul{Не получается. А раз не получается по полной аналогии и не получается и с государством. Так, кто тогда же правит вашей страной? Индивидуумы?}
\people{(Лена) Все участвуют.}
\soul{Все? И вы учувствуете в управлении?!}
\people{(Лена) Ну, а как же? Если все…}
\people{(Ольга) Мы даём им силы, энергии. Мы же их выбираем.}
\soul{Да, вы, конечно, их кормите, это понятно, но на ``кухарку, правящей страной'' вы не похожи.}
\people{(Ольга) Ну, да. Ну-у… принимаем участие.}
\people{(Лена) Какое-то же участие все принимают?}
\people{(Ольга) ``Кормим'', как говорится.}
\people{(Лена) Мысли, там, даём. Мысли - они же летают! Всё, о чём мы подумаем,  передаётся же!}
\soul{Прекрасно!}
\people{(Ольга) Нет, сообщество, опять же.  В унисон, когда все мысли поют. Да ведь? И выходит, вот, то, что имеем. Какое тело имеем, так мы и живём.}
\people{(Гера)Такое и нас имеет.}
\soul{Ну, это мы сейчас дойдём и договоримся до таких вещей… И всё-таки, вы решили, что вы управляете государством. Тогда, что же вы им плохо так управляете, что же тогда жалуетесь-то, что вы плохо живёте? Так, вы сами и не умеете тогда управлять!}
\people{(Ольга, Лена) Не умеем.}
\soul{А вами больше никто не управляет? А так, чисто физически - просто соединились и всё? Нет? ``Не подходит, нам это не нравится''. Значит, всё-таки, управление есть? Хорошо. Давайте, подумаем тогда так: если по полной аналогии с телом, с государством у нас что-то не получается, да? Значит, государством должна управлять какая-то монада, - государственная монада.}
\people{(Гера) Ну, вообще-то да.}
\soul{Так получается, да? Если человек - это набор клеток, и им управляет божественная искра монада, то, значит, и существует государственная искра монада. Ну, пускай называют её как-нибудь по-другому, но смысл тот же.}
\people{(Ольга) Ну, где-то так.}
\people{(Гера) Ну, общая аура скажем так.}
\people{(Ольга) Нет, не аура.}
\people{(Гера) Ну, энергия…}
\soul{А зачем тогда ваша общая аура?}
\people{(Ольга) Недаром же у нас есть: русские, французы, китайцы и так далее. Ведь, они мало того, что объединены общим, ну, так сказать, каким-то национальным, но не только внешними признаками, которые может быть, действительно, зависят только от природы, от климата, но ещё и внутренними какими-то…то есть как вот, семья, допустим, имеет свою психологическую, так сказать, совместимость.}
\people{(Гера) Да, в принципе, семья - это тоже государство,- мини.}
\people{(Ольга) Поэтому, и здесь тоже, государство имеет, так сказать, сообщество людей, которые, грубо говоря, ``притягиваются'' друг другу. То есть, имеют сходное что-то…}
\people{(Гера) Интерес.}
\people{(Ольга) Ну, интерес там…}
\soul{Итак, мы пришли к национализму. Так?}
\people{(Ольга) Ну, где-то так, да. Но мы не говорим ``национализм”…}
\soul{Значит, национализм существует, ибо существуют расы: ``это русские, это евреи, это китайцы''. И причём: ``эти умнее, эти глупее, эти вообще глупые.”}
\people{(Ольга) Нет, мы, как раз, не хотели сказать. Мы, как раз, говорим о том, что… ну, допустим, вот взять русского и взять китайца, да? Ну, чем… В принципе – внешне. Ну, там есть какие-то есть внешние признаки, а внутренние? Ну, если чисто психологически будем говорить.}
\soul{“Чисто психологически'' не затрагиваете, ибо, смотрите, будьте внимательны, глядите - чисто психологически, но не духовно. }
\people{(Ольга) Ну, чисто психологически, да.}
\soul{Итак, представьте, вы – русский, психологически - вы один, но попомните наше слово - если выучить иностранный язык и начнёте говорить по-иностранному, будете делать это замечательно и хорошо, то вы начнёте думать, как этот иностранец, ибо вы, чтобы знать его язык - вы должны стать им. И вы будете уже тем же иностранцем, именно в тот момент, когда вы будете говорить или думать на этом языке.}
\people{(Гера) Угу. Так, значит, резонанс…}
(щелчок)
\soul{О любви.}
\people{(Гера) О любви?}
\soul{Что вы можете назвать, как вы можете объяснить - что такое любовь? Давайте, первый вопрос: что вы можете назвать любовью?}
\people{(Гера) Ну, скорее всего - взаимодействие…взаимными чувствами друг друга, скажем так.}
\soul{Не обязательно.}
\people{(Ольга) Это не взаимное, не обязательно. Любовь это…присуща каждому человеку, не обязательно это взаимность.}
\people{(Гера) Я имею в виду, с точки зрения сознания как она описывается. Она может описываться и другими критериями.}
\soul{Но даже, с точки зрения сознания, все, что вы понимаете – не значит любовь. Когда сидят два профессора и разговаривают об одном и том же предмете, они прекрасно понимают друг друга, но нельзя сказать, что они любят друг друга.}
\people{(Лена) Да.}
\people{(Ольга) Любовь это, наверное… }
\people{(Гера) Симпатия какая-нибудь.}
\people{(Ольга) Нет. Это именно…}
\people{(Гера) Может, это всё положительное, что может чувствовать человек?}
\people{(Ольга) Притяжение что ли…это, как магнит…это чувство, конечно…это чувство.}
\soul{Или комплекс чувств? }
\people{(Ольга) Да, комплекс чувств…}
\soul{И, всегда ли полноценный этот комплекс? }
\people{(Лена) Нет, конечно.}
(Щелчок)
\soul{…понятие слепой любви.}
\people{(Лена) Да.}
\people{(Ольга) Здесь, конечно…наверное, любовь это…не когда,  действительно, слепая любовь. Грубый опять пример: когда мать своему ребёнку всё готова отдать, и всё сделать для него, но в итоге выращивает человека, будем говорить - не всегда хорошего.}
\soul{И он, мать, кстати, не любит. }
\people{(Ольга) Да, выходит что, то есть, мы…}
\soul{Иными словами, слепая любовь никогда не может родить - саму любовь.}
\people{(Ольга) И порой может быть, когда человек истинно любит, то он, наверное - выращивает, если можно так сказать, в другом человеке, если это человек или там что-то другое, допустим не только человек, это дерево может быть, ну всё. Он выращивает то, что действительно может дать другим любовь… ну,как…}
\soul{Ну, почему же? Простите, давайте скажем так: какой-то маньяк воспитывает своего сына или дочь, очень его любит, в прямом смысле слова, но выращивает, как вы говорите, из него всё-таки маньяка.}
\people{(Ольга) Да нет, я не то имела в виду.}
\soul{Тогда говорите более. Дальше, что в вашем понятии… Ладно, мы упустим это. Давайте возьмём другой вариант. Итак, что такое любовь, вы не можете сказать, формулу не можете описать.}
\people{(Ольга) Это же чувства, а чувства не всегда словами можно описать.}
\soul{Хорошо. Назовите мне какое-нибудь любое чувство, которое вы могли бы описать? Вы не можете описать комплекс чувств, ну, возьмите тогда одно любое чувство. }
(Гера) Ну, брезгливость.
\soul{Брезгливость. Опишите мне это чувство. Опишите, что это такое, чтобы я… Вот - я прилетел на вашу планету, я не знаю, что такое брезгливость, но я хочу знать, что это такое. Объясните мне, пожалуйста.}
\people{(Гера) Ну, это когда тебя отталкивает от чего-нибудь.}
\people{(Лена) Это отвращение.}
\people{(Гера) Да. То есть, ты стараешься уйти. Неприязнь.  Или исправить, чтобы было приятно.}
\soul{Нет, ну, подождите, вы уже тут назвали целую кучу чувств. Вы меня ещё больше запутали. Я вас спрашиваю: ``Что такое брезгливость?'', а вы ещё мне говорите, что существует такое: ``приятное''. Это тоже - чувство. Так, что такое брезгливость? Только не затрагиваете мне другие чувства, а то вы меня вообще запутайте. }
\people{(Гера) Хорошо. Это то, от чего ты хочешь уйти. }
\people{(Ольга) С чем тебе неприятно общаться, видеть там, ну, то есть – ощущать.}
\people{(Гера) C чем ты не хочешь встречаться.}
\soul{И это – брезгливость?}
\people{(Гера) Да.}
\soul{И всё?}
\people{(Гера) Ну, да. Если одними словами, то, приблизительно, так.}
\soul{“Приблизительно так'' - Хорошо. Вот, я вас очень люблю, но в данный момент я не могу, не хочу. Чтобы допустим вас не обидеть, да? Я хочу уйти от вас. Брезгливость такая, да? Это же брезгливость?}
\people{(Гера) Да, это брезгливость и да, между прочим.}
\people{(Ольга) Не правда.}
\people{(Лена) Нет-нет.}
\soul{По каким-то причинам, я чувствую, что вы хотите остаться один, я вас люблю, значит, я вас чувствую…}
\people{(Гера) Ну, да.}
\soul{Я вас оставляю одним, я хочу уйти от вас. Вы испытываете чувство брезгливости?}
\people{(Лена) Нет.}
\people{(Гера) Нет, вы не испытываете чувство, вы не хотите уйти. Раз вы любите, вы видите, что я хочу, чтобы вы ушли.}
\soul{Тогда не понятно ваше объяснение. Вы сказали, что брезгливость - это значит…}
\people{(Гера) А это то, что от чего лично вы хотите уйти.}
\soul{Ага. Значит, ещё что-то личное. А вы, всё-таки, конкретней, пожалуйста, скажите, что такое брезгливость? Отталкивание? Два магнита тоже отталкиваются друг от друга.}
\people{(Гера) Ну, получается, они брезгуют. А раз они брезгуют, значит, не желают встречаться.}
\soul{Итак, два магнита брезгуют между собой. А нежелание?}
\people{(Гера) А это и есть брезгливость.}
\soul{Это, тоже брезгливость? ``Я, вот, не желаю сейчас что-либо написать. У меня нет сейчас желания написать вам письмо''. - Брезгую, да? Так, получается, я брезгую сейчас писать письмо?}
\people{(Гера) Ну, я понял. Одним словом, конечно, не описать сейчас.}
\soul{А вы пытались это сделать только что.}
\people{(Гера) Ну, я попытался, я же знал, на что шёл.}
\soul{Итак, попробуйте ещё раз. Вы пытались взять другое чувство, ну, возьмите другое чувство, пожалуйста.}
\people{(Лена) Да нет. Я поняла, это невозможно. (смех)}
\soul{Вы не можете описать даже что такое одно чувство. А комплекс чувств, вы проще назвали – любовь.}
\people{(Лена) Ну, там же, в принципе, есть всё.}
\soul{Ну, давайте скажем так: ненависть. Это одно чувство или комплекс?}
\people{(Ольга, Лена) Комплекс.}
\soul{Хорошо. А брезгливость? Это одно чувство или комплекс?}
\people{(Лена) Ну, тоже - да.}
\soul{Тоже комплекс или тоже?}
\people{(Лена) Тоже - комплекс.}
\soul{Тоже комплекс. А отдельно, просто существует брезгливость? ``Вот я люблю, но у меня брезгливость''.}
\people{(Лена) Отдельно - не существует.}
\people{(Гера) Если брезгуешь, то не любишь.}
\soul{Тогда, давайте скажем так. Хорошо. Вот, смотрите. Ага! Брезгливость - это комплекс, да? Комплекс в комплексе? Что такое любовь? Это куча комплексов, или что? Не существует какое-то отдельное чувство, только это чувство и больше ничего, никаких комплексов?}
\people{(Лена) Нет. Не существует.}
\people{(Гера) Ну, почему? Когда, наверное, любишь – радуешься, например.}
\people{(Лена Гере) Радуешься чему? Что такое радость?}
\people{(Гера) Радуешься, не смотря ни на что, а  не ``чему''. }
\soul{Давайте скажем так, существует только чисто брезгливость и больше ничего, никаких больше чувств, только брезгливость?}
\people{(Ольга, Лена) Нет.}
\soul{Не существует. Отрицательная, так? Это эмоция отрицательная?}
\people{(Ольга) Да.}
\soul{Отрицательная.}
\people{(Гера) Угнетающая.}
\soul{Хорошо. Давайте так: гнев.}
\people{(Лена) Тоже отрицательная.}
\soul{Тоже отрицательная. Может быть одним чувством - просто гнев?}
\people{(Ольга) Ну-у, да. Гнев - это чувство одно.}
\soul{Давайте скажем так: я могу гневаться, чисто гневаться, как говорится, не обладая – брезгливостью там, ненавистью?}
\people{(Ольга) Можно конечно.}
\soul{Нет, я просто гневаюсь и всё.}
\people{(Ольга) Нет, в какой-то такой момент бывает действительно, когда вспышка такая гнева…}
\soul{Ага. Прекрасно. Значит, всё-таки, чувство гнева бывает одно, без комплекса?}
\soul{Давайте, тогда возьмем – брезгливости.}
\people{(Лена) Значит, тоже вспышка.}
\soul{Тоже вспышка. А мы, что договорились? Брезгливость - это комплекс. Без комплекса - не существует. Да?}
\people{(Ольга) Ну, да.}
\soul{А гнев существует и без комплекса, просто гнев. Можно гневаться на любимого, а можно гневаться и с брезгливостью или без брезгливости, можно гневаться любя и не любя, значит, отдельно - чувство всё-таки есть. Так получается?}
\people{(Гера) Ну, да, наверное. Только надо, наверное, отсеять от всех остальных.}
\soul{Ага. Хорошо.}
\people{(Ольга) Ну, вот это, как раз, всё одно за другим тянется - вот эти ниточки связывают, наверное, всё это…}
\soul{А давайте тогда так… Вот, смотрите, гнев может быть отдельным чувством и может быть комплексом, да?}
\people{Да.}
\soul{Ну, комплексом это понятно, когда вот человек который…вас принуждает, пробуждает в вас брезгливость, подходит к вам всё ближе и ближе, вы начинаете гневаться, и всё, вроде бы как всё, всё хорошо складывается, да?  Интересно, а может быть наоборот - вот человек гневается, аж до брезгливости? Благодаря гневу он рождает брезгливость.}
\people{(Гера) Можно.}
\soul{Можно. Значит, мы решили: брезгливость - это комплекс, гнев - это отдельное чувство, может быть комплексным, а может быть отдельным чувством. Брезгливость отдельным чувством не бывает? }
\people{(Гера) Ну, наверное, память из всех неприятных моментов и рождает брезгливость к ним, в связи с негативным их восприятием. }
\soul{Ну, хорошо. Вы мне только скажите: может ли брезгливость быть не комплексным, а чистым чувством?}
\people{(Ольга) Ну, может.}
\people{(Гера) Нет-нет. Она рождается из комплекса. Как же она может быть одним чувством?}
\people{(Ольга) Ну, почему? Я увидела, допустим, извините в туалете червяка, много вот этих шевелящихся червяков таких вот, и я испытываю чувство брезгливости. Разве это комплекс?}
\soul{Ну, прекрасно. Это, всего лишь только, как говорится - одно чувство, да? Не больше. А почему это вы, вдруг, на бедного червяка так ополчились?}
(Смех)
\soul{Что он сделал вам? Почему он вызвал в вас брезгливость? Что заставило вызвать это чувство брезгливости?}
\people{(Лена) Ну, скользкое, противное…}
\people{(Ольга) Вид, запах там, много чего…}
(Конец сеанса)
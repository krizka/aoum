Аоум. глава 20-я 12-05-1994г
Георгий Губин
 
 ``Если Вы не можете быть хорошим примером, то Вам просто придётся служить ужасным предостережением. © Кэтрин Эйрд"
\people{**}
\people{Сегодня 12-го мая. Мы вновь , после долгого перерыва,  собрались для встреч, для беседы. Скажите, ``переводчик'' очень трудно выходил на контакт. Это связано с новым местом или с новыми людьми, которые здесь присутствуют? Можете это объяснить?}
\soul{Нет.}
\people{Всё вместе? Скажите, что лучше для ``переводчика''? Место или…  Что больше влияет - место, или новые люди?}
\soul{Вам нужно настроиться.}
\people{То есть, нам труднее всем, сколько человек присутствует, надо настроиться, да?}
\soul{Простите, каждый из вас, разный. Каждый из вас…(теряется)}
\soul{1-2-3-4-5}
\people{Получается, если мы попробуем это сделать на большой аудитории, то контакта может  и не состояться? Так?  Ведь разные люди будут сидеть.}
\soul{Вы не правы.}
\people{Не правы? То есть, можно однажды попробовать?}
\soul{Мы говорили вам, мы слышим вас всегда.  Но слышите ли вы?}
\people{Да, мы не совершенны в этом. Мы согласны, что мы не всегда можем. Мы сегодня хотим задать серию вопросов о нашей Луне,  о нашей спутнице, которая нас издавна волнует. Вы готовы, кое-что ответить?}
\soul{Нет.}
\people{По Луне вообще?}
\soul{Готовы  ли  вы?}
\people{Ну,  посмотрим. И так; первое…}
\soul{Нет, мы говорим вам: готовы ли вы? Что задаёт вопросы ваши? Вы. Сознание ваше.}
\people{Ну, у нас жгучий интерес к нашей спутнице. Потому, что с ней связано много легенд и много неясностей, особенно в последнее время, когда там побывали космонавты.}
\soul{Вы, - сознание ваше,- разное, и потому,  интересуетесь. Мы же говорили вам, о физическом. Готовы ли вы?}
 (счёт)
\people{Не надо, наверное.}
\people{Хорошо. Другую серию вопросов зададим. Скажите, отличается ли земной астральный план от космического астрального плана?}
\soul{Поймите. Планы придуманы вами. Ваши планы, это всего лишь частица. То, что вы можете увидеть, ощутить. Мы говорили о множестве планов, что знаете вы, - физические. Выше, не ведаете, ибо сознание ваше не может ведать о духовном.}
\people{Но, мы будем к этому стремиться. Тогда, второе,- действительно ли утверждение, что земной астральный план,  это план иллюзий, воображения и фантазий людей?}
\soul{Мы говорили вам, что все версии ваши верны. Верны отчасти. Ибо все они ложны, или правдивы и далеки от истины. Вспомните. Далее. Вы говорите ``иллюзии''? А мы говорим: сознание ваше – иллюзии. Ибо сознание ваше - физическое, и хочет познать выше, но не выше физики.}
\people{Но, оно стремится к тому, чтобы узнать, что за пределами материального восприятия.}
\soul{Стремитесь? Стремитесь сознанием,  и  потому, не получаете. И потому, вы всегда правдивы тогда, когда приходит к вам неожиданость, приходит к вам любовь. Вы можете сказать, чем вы любите? Сознанием?}
\people{Не понятно чем. Наверно, душой.}
\soul{И подумайте. Подумайте! Вы же, не можете понять, что тела ваши – одежды. И их так легко менять! Так легко!}
\people{Может ли человек, в астральном плане, путешествовать в другие солнечные системы? Об этом, у нас ведётся много споров.}
\soul{Солнечные системы, или более?}
\people{В другие галактики.}
\soul{Простите, вы делите на'' другое'', `` иное''. Мы же, говорили вам: - физика ваша. Астралы – ваше.   И потому,  можете быть везде. Везде. Вся физика - ваша.}
\people{После смерти физического тела, каким образом продолжается совершенствование духовных возможностей человека?}
\soul{Если вы не избавитесь от верхних одежд, - вернётесь к этому. Вспомните. Мы говорили вам, о ваших, как вы говорите, семи телах. И, говорили о семи смертях, о смертях каждого из тел. Подумайте. Приведите аналогию. Подумайте о верхней одежде. Всегда ли вы носите её? Приходит время, когда снимаете. Вы же, - не хотите снять, ибо считаете плотью вашей. А мы говорим: одежды ваши легко менять. Легко!  И когда вы поймёте, и поймёте не сознанием, а более, только тогда вы их снимете и пойдёте далее.}
\people{Как астральный план соотносится с ментальным планом? Так же, как человек в физическом воплощении не знает о жизни после смерти, или, после смерти тела, человек изменяет своё понимание жизни?}
\soul{Хорошо, давайте скажем так. Аналогии ваших планов – таблица Менделеева. Вы сможете ответить далее  сами? Сами, себе ответьте. У вас есть тяжелый, и есть лёгкий. Но, всё это – физика. Физика. И, согласитесь, ранее, вы не знали о более тяжёлых. Вы согласны? И придёт время, когда для вас это будет пройденный этап. Тогда, даже сознание ваше будет знать, как вы знаете таблицу.}
\people{Говорят, что наша эпоха, это эпоха развития ментальных способностей человека. Так ли это?}
\soul{Нет. Вы горды. Каждый из вас решил, что  время ваше особенно. Подумайте. Неужели тысячу лет назад не был задан такой вопрос?}
\people{Но он до нас не дошёл. По крайней мере, ответ.}
\soul{Разве?}
\people{Ну, или мы, лично, не знаем.}
\soul{Простите. Не дошёл?  И чья вина в том?  Ваша или иных? Вы, чаще, обвиняете не себя, а учителей ваших.}
\people{Или предков,  которые не донесли, не сохранили ваши ответы. Или чьи-то ответы.}
\soul{Простите. Когда-то  и вы были предками. Что же вы не донесли?}
\people{Да. Виновны. Существует ли в астральном мире понятия религии, науки, искусства и т.д., то, что существует в материальном мире?}
\soul{Да. Мы говорили вам. Когда-то вы спрашивали о мире умерших, и мы говорили вам: это тот же, тот же мир, что и ваш. Но, подумайте и возьмите аналогию о вас. Даже у вас есть деления. У вас есть воздух, вода, земля. И согласитесь, везде жизнь разная, но она - одна.}
\people{Как воспринимается человеческим сознанием астральный мир после смерти физического тела?}
\soul{Сознание ваше умирает. Мы говорим вам: сознание ваше – одно из форм одежды. Сознание ваше умирает вместе с вашей одеждой. И теперь, подумайте, если вы смогли снять верхнюю одежду и остались в более тонких,-  сознание ваше, более тонко. Значит, знание – более. Согласны?}
\people{Ну, вот, познаём от вас.}
\soul{Вы, рождаясь, рождаетесь и получаете  сознание. Сознание – это физика. Физика  ваша. И все ваши жизни, вы их называете ``прошлым'', хотя не знаете, что даже здесь нет времени. Всё это едино. Столь едино, что нельзя говорить,– ``смерть  сознания''. Подумайте, каково противоречие?  И найдите здесь.}
\people{Вот, так называемые миры или планы, существуют в нашем сознании, как часть общего сознания человека или независимо от сознания человека они существуют?}
\soul{Тогда проведите аналогию. Приведите. И возьмите человека зрячего и нет, и полу (прим.  имеется ввиду плохо видящий) и подумайте.}
\people{Ну, объективно эти миры существуют, или, всё-таки, только в связи с сознанием человека?}
 
\soul{Простите. Что видит слепой, что видит зрячий и полу? Для полу,- многое, что видит зрячий, - не существует. Вы согласны? Ибо не видит и не признаёт. Слепой не видит и того. Многие из вас слепы. Хотя и живут в тех мирах и питаются ими. И не знают о том. Вы младенцы. Младенцы. И вы только начинаете жить. И вы уже хотите познать многое. Вы горды. Вы столь горды, что даже все чувства ваши преломляются той гордостью. Но не замечаете того.}
\people{Ну, вы знаете, мы одновременно и горды и готовы к самоуничижению. Поэтому, что превуалирует в нас тогда?}
\soul{А вы подумайте, может это одно из проявлений гордости?}
\people{А-а.  Да, верное замечание.}
\people{А как в этом разговоре, если вспомнить болгарскую ясновидящую, вот у нас есть, знаменитая. Как они в этом свете? Выше нас? Если мы – дети, то они тогда кто? Они видят многое.}
\soul{Мы говорили, что те миры, - те же миры, что и ваш. Вспомните, мы отвечали на подобное и ответим далее, если не помните: мы не имеем мер ``выше'' и ``ниже''.}
\people{И у нас – нет, вы, как-то, сказали в одном разговоре.  А она где была?}
\soul{Вот и подумайте.  Мы говорили о гордости, и вы хотите измерить  кто же ``выше'',  вы или они?  }
\people{Скажите, являются ли задачей человеческого существа, совершенствования своего подсознания, или ``подсознание'' - это заложено в человеке прошлыми существовавшими инстинктами?}
\soul{Как вы спросили? Вы спросили –  ``задачи'', и уже ``обязанности''. И вы говорите: ``инстинкты'', и уже,- `` животные''. И здесь, и здесь, думая так, вы не будете заниматься ни тем, ни другим, ибо вы считаете, себя выше инстинктов и не обязаны выполнять чьи-то обязанности. Мы говорили вам, о сознании, - пока вы хотите понять это сознанием и только чисто сознанием, – не поймёте ничего. Вы, просто,  утонете в море информации, и вы будете делать вывод относительно вашего сознания, уровня вашего сознания. И всё иное вы скажете: - ``Ложь''. Подумайте, вы принимаете правду только ту, что нравится или подходит вам более, а остальное для вас – ``ложь'' или ``другое''.}
\people{Скажите, а может ли человек влиять на развитие сознания в растительном,  животных мирах? И каким образом, если это возможно?}
\soul{Вы влияете, и влияют на вас. Вы едины. Проведите аналогию. Мы сегодня хотим дать вам множество аналогий, ибо вы много не понимаете. Придите в лес и подумайте, зададите ли вы тот вопрос там?}
\people{Так в том-то и дело, что не зададим. А, может, надо?}
\soul{Зачем спрашиваете тогда здесь?}
\people{Ну, нащупываем, где наши ошибки, как правильно себя вести.}
\soul{Вы, всмотритесь в себя. Вы не щупайте, вы, просто, постарайтесь жить. Вы же  постоянно хотите контролировать себя  и  любую мелочь принимаете за великое  и великое, - за мелочь.  Идите в лес! Вы получите ответы на те вопросы, и придите в ``каменный мешок''. Как вы думаете?  Влияет?}
\people{Как-то влияет.}
\soul{Спрашивайте далее.}
\people{Человечество по своему развитию, чувству и ума,  подразделяется на определённые группы. Как происходит существование и развитие людей с разным сознанием в астральном мире? Ведь не все равноценны по уму, по своему развитию чувств.}
\soul{Душой и эмоциями – все. И у вас есть только одно – сознание. Вы называете это ``умом''. Вот и подумайте. Мы говорили вам о других мирах. Как вы понимаете? Как вы делите их? Выше/ниже. То же, происходит там.}
\people{У нас ощущение, что…}
\soul{Да вы поймите: мир ваш и соседние миры ваши - не на много различаются от вас! Не на много! Подумайте, чем выше и, чем ниже, тогда будет более различий. Подумайте,  мир умерших, – это ваш мир. Вы живёте в том мире, живёте во снах, живёте даже наяву. Но не знаете того. Вы живёте в соседних мирах и не видите и не ощущаете их.}
\people{Ну, вот хочется выяснить про душу,  у нас такое ощущение, что вы не правы, что души у всех одинаковые и одинаково развиты. И есть ощущение, что у некоторых особей человеческих, вообще нет души, это преступники, это деградирующие.}
\soul{Мы говорим про души, а вы говорите о разуме. Подумайте.}
\people{А, то есть, именно разум управляет преступниками?}
\soul{Мы говорили о вас, о вашей плоти, как об одежде. Вы, можете получить любую. Многие приходят сюда в одежде хирургов,  другие - в одежде палачей. Есть одежда врачей, есть одежда священников, есть тюремные балахоны. И всё это, вы называете ``плотями''.}
\people{От чего это зависит, что каждый приходит по-разному?}
\soul{Всё зависит от того, как вы живёте, как вы жили ранее. Или в будущем. Здесь нет понятия о времени. Поймите. Земля ваша. Вы приводили множество версий о вашей планете и её назначении. Мы же говорим вам об одеждах ваших. Вы приходите на Землю, чтобы изменять те одежды. И если вы пришли в тюремном балахоне, подумайте, как вы вели себя ранее? Вы были праведником? Святым?}
\people{Нет, конечно. }
\people{Наверно, оправдано высказывание, что ``душу забрал чёрт''. Это о людях, которые вообще плохо ведут себя в обществе? Такое случается?}
\soul{Неужели вы можете представить, что есть ``чёрные души''? Есть только грязь, грязь на душах тех. Поймите: сознание, сознание ваше  живёт здесь. Сознание ваше двигает вами и вашими одеждами. Душой, вы чисты. Все души ваши чисты. Поймите, даже в религиях, вы не видите того, что вам говорят этим. Ибо Бог пришёл ко всем! Подумайте, неужели он пришёл бы в грязную, чёрную душу, в вашем понятии? Подумайте, - все вы чисты, но в душе. Всё остальное – грязь, созданная вашим сознанием и многим, многим другим.}
\people{Скажите… Вот, вы не любите упоминаний… но, вот, у Чикатило  была душа?}
\soul{Подумайте.}
\people{Ну, была, допустим. Сейчас, где она находится? Есть понятие ада для него, или он существует где-то в райских местах?}
\soul{Если мы скажем, что есть. Есть подобные и здесь, что вы ответите на этот вопрос? Вы говорите ``умер''. Вы не внимательны. Мы же говорили вам, что вы живёте в соседних мирах тоже. В вашем понятии, вы можете казнить человека, но он будет продолжать жить в этом мире. Подумайте. Он может продолжать. И вспомните о ``ключе”и ``ячейках''. Вспомните. И когда-то мы говорили, что чья-то смерть может стать для кого-то… Вы помните?}
\people{Типа ``крещения''?}
\soul{Вот и подумайте.}
\people{Значит, нет ада и рая?  Есть только высший и нижний  план?}
\soul{В вашем понятии, - да. Нет рая,  нет ада.  Но, для сознания, оно будет существовать. Поймите, если вы уйдёте в нижние, в более тяжёлые планы, а мы когда-то вам говорили, вспомните, - мы говорили вам, о камнях  и ``выше''.}
\people{Да.}
\people{Сохраняется ли личность человека после смерти?}
\soul{Да. То душа ваша.}
\people{А то сознание, с которым мы… }
\soul{А вы вспомните. Мы говорили вам противоречия. Мы говорили вам и одно, и другое. И, тут же, указали на противоречие. Мы говорили, что сознание ваше умирает с плотью, и говорили, тут же, что сознание ваше живо и бессмертно. Подумайте. И сравните, и вспомните,  мы говорили вам о единстве. И пойдёмте далее. Посмотрите, у вас в религиях есть падшие ангелы. А когда-то вы пытались доказать нам, что ниже упасть нельзя. Вы помните? И теперь  подумайте, что заставило их  упасть? }
\people{Потеря энергии, наверно.}
\soul{Сознание ваше, личность ваша. И не зря мы говорили вам о гордости. Не зря. И любой шаг вперёд, если он не сделан от сердца… То – гордость ваша.  Гордость.}
\people{Что такое гордость? Это, если шаг от сердца не сделал?}
\soul{Да, это когда вы думаете только о личности, как о себе.}
\people{А почему в некоторых религиозных верах не существует понятия ``cатаны''? В той же  ``Науке разума'', ``Вере бохаи''?}
\soul{А вы подумайте. Когда-то, мы говорили вам и пытались сказать, что нет добра и зла. Что вы сказали на это? Вы сказали, что живёте в этом мире, где есть и то и другое, а мы вам говорим: нет рая и ада. Согласитесь, если есть добро и зло, значит, есть рай и ад. Если мы вам говорим, что нет ни того, ни другого, значит, нет и зла, и добра. Мы говорим вам, живите, живите, но не сознанием, не сознанием, а выше. Сознание ваше придумало законы ваши. Придумало понятия о добре и зле, о свете и тьме, и многое, многое. Только сознание ваше.}
\people{Так. Мы поняли, что понятия времени в астральных мирах не существует.  А понятие ``пространства''?}
\soul{Нет, мы вам не говорили того. Мы вам сказали  другое.}
\people{Хорошо, а пространство в астральном мире существует?}
\soul{Да. Подумайте, мы же говорили, что ваш мир, что спрашиваете, подобные вам. Поймите. Соседние миры, в вашем понятии, не намного, отличаются от вас. Далее. Физическое всегда имеет понятие и времени, и пространства.}
\people{Существует ли в астральном мире подобие учебных заведений?}
\soul{Вы, повторяетесь. У вас  существует?}
\people{Существует?}
 
\soul{Всё, то же самое.}
 
\people{Они так же  похожи на наши институты, на наши школы?}
\soul{О-о! Вы даже не знаете,- похожи, даже более, чем вы можете представить. И, потому, вы можете разговаривать с ними. Мы когда-то говорили вам, что вы создаёте множество миров, множество ветвей. Подумайте. Вы же говорите: ``мысль материальна''. И подумайте далее. Мысль ваша, что-то родила, и появился новый мир. Может быть, в той мысли, в том мире, вы хотели пойти в одну сторону, а здесь вы пошли в другую, в третьем вы хотели остановиться, в четвёртом что-то ещё. Вот вам, и множество параллельных миров. Если взять дальше, то те параллельные миры, рождают ещё и ещё. Вы называете это ``цепной реакцией''. И представьте, соседний мир будет не на много отличаться от вашего, далее, далее и вы уже не узнаете и скажете, что это ``это иное''.}
\people{Может ли современный западный человек придти к состоянию просветления без познания мира через религию? То есть, из нас сделали атеистов, но можно ли без религии понять? К просветлению придти.}
\soul{Вы, говорили: - ``западный''? Далее.  Подумайте. Многие живут, не признавая религию, но живут, всё-таки, выше вас. Вы не можете вспомнить? Подумайте, многие верят, но хуже вас.  Вам привести аналогию или вспомните сами? В вашем понятии, было время дьявола. В вашем понятии, все не верующие - будут прокляты и будут гореть в аду. Это ваши слова. Неужели вы можете представить, что Бог, в вашем понятии, наделён гордостью?! Его очеловечили до той степени, что даже ваше низменное, есть в Боге. Гордость. Подумайте.  Бог так горд, что, если не верят в него …  Вы поняли?}
\people{Ну, здесь, может, тут другой механизм, какой-нибудь. Вера даёт какую-то новую энергию. Какое-то, может быть, озарение?}
\soul{Хорошо, тогда объясните мне, пожалуйста, почему ребёнок считается святым? Что, он верит? Он взрослеет и потерял? Подумайте. Подумайте, ведь это логика ваша! Спрашивайте далее.}
\people{Почему ребёнок считается чистым? Ведь есть закон кармы. И если он свои плохие недостатки переносит из жизни в жизнь, то он рождается, выходит, с  недостатками, с  отрицательными качествами, которые имел в прошлой жизни.}
\soul{Назовите мне хоть одну религию, где ``не ребёнок'' -  считалось бы святым. Вы можете назвать такую религию?}
\people{Нет. Он не считается.}
\soul{Вот и подумайте, вы имеете разные религии, но ребёнок свят. Ибо не успел в этом мире, как говорили когда-то вы, ``наследить''. Далее. Что вы знаете? Почему ребёнок пришёл и, не сделав ничего – ушёл? Почему? Может быть, ему хватило, хватило мгновения, чтобы снять все одежды или одну из них. Можно прожить сотни, тысячи лет и оставаться, оставаться в этих одеждах. А можно и не родиться. Только, в вашем понятии, выглянуть  и уже стать выше.}
\people{А может, нет закона кармы?}
\soul{Разве? Даже у вас, даже в вашей физике есть карма. Вы сами вспомните?}
 
\people{”Ничего из ниоткуда не возникает и …”}
\people{А-а! Не могли бы вы прокомментировать статью, которую мы сейчас прочитали вместе с ``переводчиком'' перед началом сеанса?}
\soul{Вы когда-то спрашивали. Вы уже спрашивали. Мы вам отвечали.}
\people{Но, это новая статья, и, конкретно из этой статьи,  хочу узнать там некоторые вопросы…}
\soul{Хорошо, задавайте. Поймите, вы ведёте запись, вы, когда-то спрашивали: ``почему вы отвечаете на вопрос, который слышите, а не тот, который я хотел задать?''.  Вы, помните? Но, простите, вы ведёте записи. Всё ли могут услышать то, что вы хотели задать, или они, всё-таки, слышат голос ваш? Спрашивайте.}
\people{В данной статье говорится, что учёные заметили такую вещь, что женщина может родить ребёнка не от того мужа с которым, конкретно сейчас живёт, а именно от старого, первого мужа или какой-то предыдущей встречи, которая давно была. Хотя, прошло уже после тех встреч, первых встреч её, сексуальных,  несколько лет.}
\soul{Подумайте, как вы задали вопрос. И пусть иной, кто не читал статью, объяснит, что он понял из вопроса вашего. Подумайте. Скажите проще, - можно ли родить незачатого? А вы вспомните, вспомните Библию вашу.}
\people{Ну, это слишком легендарно выглядит.}
\soul{Легендарно? Тогда возьмите, вспомните о рождении и почитайте внимательно вашу статью. Подумайте…}
\people{Это возможно, да?}
\soul{Будьте внимательны, вам же было сказано: ``ибо Святой Дух…''.}
\people{Чё-то мы не разобрались…  Давайте, снимаем этот вопрос.}
\soul{Снимать? Не понято вами?}
\people{Понятно. Тогда вопрос такой: это наверно через мать передалось, по наследству ей, Марии?}
\soul{Разве? Вы подумайте, вы читали и не поняли? Вы говорили и пытались задать вопрос: от первого мужа или от первого кто имел. Мы же говорим, что миры соседние, в вашем понятии, подобны вам. И вы  живёте и там и там. Вы не допускаете иное.}
\people{Это может быть даже во сне зачатие?}
\soul{Мы же говорили вам, что мать рождает плоть. Земля даёт вам плоть. Но мы и говорили вам, что рождает, всё-таки, душа. Душа оплодотворяет. Душа даёт, а не просто плоть. Подумайте. Неужели душа должна иметь обязательно плоть? Может, просто душа придёт к вам? }
\people{Ну, я тогда по-другому сформулирую свой вопрос. После того, как раньше ходили  фестивали, в Москве, например, где приезжало много иностранцев, тех же негров, спустя несколько лет после их отъезда, когда уже не было никаких половых связей наших женщин с неграми, они начали рожать негров. Как это объяснить?}
\soul{Есть множество причин. Одна из причин, это установка. Установка,  самогипноз. Поймите. Вы обладаете такой силой, что можете менять даже гены. И простите, когда женщина даёт себе установку, что может родить ребёнка без головы и родит, то она может создать и более. Есть и  то, есть и другое. Мы говорили вам, и вы,- о биополях, Вспомните, если два шара наполнить зарядом… Подумайте. Есть соединение и смешивание. Далее. Вы состоите, всё-таки, из множества тел, из множества одежд. Ваши верхние одежды могут оставаться в покое,  другие могут заниматься, в вашем понятии, любовью.}
\people{Ну, вы сказали на счёт страха, - установки. Но, ведь такое же явление было замечено и у животных. Что, у них осталось тоже страх, установка?}
\soul{А вы подумайте, мы же ответили и на этот вопрос. Тогда объясните мне, пожалуйста, когда вы спите, и вам снится что-то, и вы делаете какие-то действия. Вы мне объясните, какое тело совершает эти действия?}
\people{Тонкое, наверно.}
\people{Одно из тел.}
\soul{Не физические. Почему вы решили, что собаки не имеют других тел?  Тогда, подумайте, подумайте, почему кошка лечит вас? Подумайте. Почему тогда  вы говорите: ``животное чувствует человека''? Она может чувствовать, только по той причине, если обладает, тем же самым. В вашем понятии, вы горды столь сильно, что не можете даже представить, что среди животных есть экстрасенсы  намного сильнее вас. Разница только в том, что животные не афишируют, и не создают рекламу.}
\people{Хорошо. Духовность - понятие сугубо религиозное или более углублённое и расширенное понятие?}
\soul{Более, более даже, чем вы можете себе представить. Ибо, вы представляете, всё-таки, сознанием. Вспомните начало.}
\people{Существуют ли в астральном мире такие сугубо земные понятия, как; самовлюблённость, фанатизм, эгоизм, карьеризм, властолюбие? Ведь эти понятия идут от эмоционального плана и ума. Исчезают ли они со смертью физического тела?}
\soul{Нет. Если бы исчезали, не было б закона кармы.  Подумайте сами. Тогда бы не начинали всё с начала.}
\people{То есть, и там есть такие чувства, эгоизм…}
\soul{Есть. Есть во всех мирах. Ну, подумайте! Когда у вас был падший ангел, подумайте, если у него не было чувств, и было всё едино и далее, далее, как же тогда он упал? Подумайте. Здесь, вы даже логически можете увидеть многое. Увидеть. Но не будете жить там. Поймите. Многое вы видите, в вашем понятии, - приходите в масках. И  вы это называете чувствами, снами, настроением и далее, далее. Спрашивайте.}
\people{Если существует падший ангел, значит, есть понятия добра и зла?}
\soul{Мы же сказали, для вашего сознания, – да. В физическом мире, – да.}
\people{Скажите, а где храниться большее зло, в уме или в эмоциональном плане, физическом ли теле?}
\soul{В вас. В каждом из вас. Самое страшное зло, – это вы. Но вы же, и добро. Поймите. Вы – маятник, вы качаетесь с одной стороны на другую. Поймите. Падший ангел, тоже познал единство. Подумайте. Он тоже обладал могуществом, и огромным, в вашем понятии, как и у бога. Но, всё-таки, упал. Ибо личность его - осталась. Вот  и ответ. (теряется) Подумайте, разве встретите чего-нибудь иного? Вы можете назвать хотя бы одно существо, которое не было больше нигде, а только в человеке? Вы, можете мне назвать хоть один пример?}
\people{А вот  какую роль играет элемент азот в физической жизни человека? Ну, например, жидкий азот сохраняет долго биологическую ткань.}
\soul{Ткань. Простите, вы, когда-то спрашивали о тех, в вашем понятии, не имеющих душу. Вы, помните? Теперь, вспомните далее. Что ваша жизнь? Горение! Всего лишь, физическое горение. Вы – огонь. Разница только в том, что вы горите или медленнее или быстрее. Тогда, вы уже говорите о скорости: как быстро старею или ещё молодой. Подумайте, и вспомните, что нужно, для горения?}
\people{Кислород.}
\soul{Вы хотите сказать для сознания, что делает азот для духовного?}
\people{Мы просто предполагаем, что может быть, азот играет очень большую роль в жизни человека, поскольку, его и процентное содержание в атмосфере весьма велико.}
\soul{Подумайте, возьмите тело ваше и окружающую среду, и вы увидите, что разницы нет большой. Есть, всего лишь, только количество. Количество. Но не качество. Вспомните. Вы говорили вам об иерархии, мы вам говорили о корнях. Вы помните о более тонких и тонких. Посмотрите, возьмите камень и тело ваше. И разница -  количество, не более. Потому, и говорим вам, что всё живое, камни живые. Но там, меньше, в вашем понятии, энергии вещества. Вы же,- более гибки. Ибо вы состоите из более тонких и более сложных структур. Возьмите, возьмите напрямую, возьмите и отключите ваше сознание. И даже сознанием и логикой вашей, вашими науками ,- а в хотите познать наукой, и будете познавать только науками, находясь в мире физики,- даже здесь вы не можете найти многое вновь. Но, вы никогда не сможете найти то, что создаёт жизнь. }
\people{Скажите. Было время, когда физики не было или она параллельно с духовным миром всегда существовала и будет существовать?}
\soul{Тогда подумайте, мы говорили с вами много и много. Говорили вам, о бесконечности, вы же  уже говорите о времени. Мы вам говорили о существовании,- что не существует ни прошлого, ни будущего. А вы говорите: ``было ли когда-то''. Далее, вы приведите аналогию, возьмите и объясните мне, что такое ``настоящее''? И возьмите, пожалуйста, будьте добры, назовите мне число, - длительность вашего настоящего. Назовите.}
\people{Да… трудно… это вообще не возможно.}
\soul{Тогда подумайте. Может, тогда не существует и вас, если даже нет настоящего?}
\people{Ну, допустим, какой-то промежуток…}
\soul{Какой? Назовите мне, пожалуйста, его длительность.}
\people{Ну, парадоксально, конечно…}
\people{Ну, вот сейчас отрезок времени – настоящее…}
\people{Мгновение, наверное…}
\soul{Если вы сможете назвать ``настоящее'' ваше, тогда - да, – есть и ``прошлое'' и ``будущее''. Но, вы не можете сказать ``настоящее''. Ибо,  каждый миг, и даже более, для вас, уже ``прошлое'',- подумайте,- создаёт для вас ``будущее''.  Вот ваш парадокс и разрешается. Как разрешается он просто! Его не существует, если нет понятия о ``времени''.}
\people{Человек имеет аурическое свечение, вокруг тела. А как в астральном мире видят людей в физическом воплощении? По этой ауре или обычным, визуальным способом?}
\soul{Давайте приведём снова аналогию. Как вы видите магнитные поля?}
\people{Мы их не видим.}
\soul{Тогда, почему должны видеть вас?}
\people{А как они ощущают…}
\soul{Хорошо, давайте так, вы не видите магнитны поля, но, всё-таки,  вы знаете это. Вы воздействуете? Вы можете нарисовать их картинку?}
\people{В какой-то мере, да.}
\people{Скажите, вот читал я книгу'' Жизнь поле смерти'', и там написано, что люди, когда выходят из тела, они видят и себя и других. И видят нормально, сквозь них проходят.}
\soul{А вы подумайте, что вы видите во снах? Неужели, вы видите, что ваше, ваше тело рождает идеи и какие-то действия есть лишь хороший эскиз? Может оно видит другое тело? И может, тогда, тело, подумайте, ваше тело, первое,  оно не видит тоже? Вот и подумайте, можно видеть только то, что имеете сами. Или, тонкое может видеть более грубое. Грубое  не увидит тоньше.}
\people{Именно так происходит зрительное восприятие людей из астрального плана на физический? Они нас могут видеть? Да?}
\soul{Возьмём аналогию.  Вы, видите свет? Видите свет или, всё-таки, воздействие света на ваши клетки? Вы можете сказать, что вы видите свет?}
\people{Отражение… Отражённый…}
\soul{Вы, не видите никакой. Свет ваш видят, всего лишь ваши нервы. Подумайте.}
\people{Вы к нам обращаетесь, как к душе, а не как к сознанию, я так понял.}
\soul{Вы, спрашиваете о физическом. Мы и говорим, о физическом. Вы, видите только реакцию, реакцию на свет. Далее. Возьмите и увеличьте. Увеличьте. И что? Вы не увидите  того света, не увидите даже реакцию. И возьмите,  уменьшите, – не увидите того тоже. Подумайте, мы когда-то говорили вам, что вы слепы. Вот, и здесь, не важно, действительно, не важно, ибо вы не видите саму энергию, а только реакцию. Реакцию ваших нервов. Далее, даже реакция та работает в малом диапазоне, – в сознании. Сознание ваше, определяет только малую его долю. И подумайте, что любое, любое излучение, всё-таки, тело ваше принимает. Вы согласны? Вы не видите радиации, однако, она убивает вас. Вы не видите то, но она убивает вас. Хотя, сознание не видит того тоже. Мы же говорим вам, что вы говорите о личности, как о сознании. Вспомните, вспомните, многие не могут понять, что говорим вам, ибо не были в прошлом. ( на прошлых контактах.  прим.) Но вы должны знать. Мы говорили вам о теле, что видит и знает всё. Вы помните?}
\people{Наше подсознание что ли?}
\soul{Разве? Вспомните, вы спрашивали, а мы говорили вам, что тело ваше, видит всё. Вы, не помните?}
\people{Да-да. Даже рука, или что-то такое… Мы только не поняли – чем?}
\soul{Вот и подумайте, что тело – одежда ваша, знает более, чем ваше сознание. Каков удар по вашей гордости? }
\people{Мы переживём.}
\soul{Разве? }
\people{Нам надо овладеть этим?}
\soul{Нет, вы не можете пережить того, и, потому, ищете. Потому ищете и хотите избавиться от одежды. Потому и приходите, потому и спрашиваете, как избавиться от неё.}
1-2-3….
\people{Вот, эти трудности, которые испытывает ``переводчик'', и мы их замечаем, они, всё-таки, с чем связаны? С тесной комнатой, может быть? Может, надо в другом помещении?}
\soul{Более связано – с вами. }
\people{Из-за нас? Да?}
\soul{Мы когда-то говорили вам, что разговаривая с вами, мы должны прийти в ваш эмоциональный план.}
\people{Ну, видимо, мы, сейчас, не очень соответствуем, да?}
\soul{Спрашивайте далее.}
\people{Скажите, цвета ауры…  Что они означают? Означают; здоровье или нездоровье человека, его эмоциональный настрой, или  что-то сугубо человеку присущее, они означают?}
\soul{Больше, всё-таки,- ваше сознание,- если хотите, более, – игра слов. Подумайте, - красный цвет – огонь. Но ведь огонь может согреть, а может и сжечь. Подумайте. И когда вы говорите, а чаще, вы так и делаете, у вас больше красок, больше ``жара''. И тогда, всё  зависит от понятий - ``жар''. Иной, в вашем понятии, ``экстрасенс'' скажет: Вы, преобладая красным цветом, обладаете тем-то, тем-то, тем-то. Любовью, и далее, далее…  Другой же скажет:  ``Вы опасны.  Вы можете обжечь''. Где же правда?}
\people{Ну, и те и другие не правы, да? }
\soul{А может любой экстрасенс видит вас через себя и потому искажает через себя?}
\people{А вот злые мысли, не доброжелательные мысли, они как-то окрашивают ауру человека?}
\soul{Конечно! }
\people{Сильно влияют, именно те, на астрал? }
\soul{Да. Но нет среди вас тех, кто б знал истинность  причин и не мог бы их не преломить. Подумайте, одно и то же слово – даёт разные ассоциации. Подумайте, если вы говорите ``Здравствуйте'', - говорите сегодня, сейчас – у вас одно настроение и другое понятие этого слова.  Есть время, когда вы говорите это со злом. То же слово, но уже другое понятие. Возьмите и найдите мне хотя бы двух одинаковых людей, которые одинаково прочитали или увидели что-то. Поймите, если к вам приходят, если к вам приходят с ответом, вы должны, прежде, знать характер того человека. Поймите, если картину нарисовал святой, приходит грешный, что делает картина та? И что скажет грешник, увидев её? Он скажет: ``Сосёт'', - ваши слова.}
\people{Да.}
\soul{Придёт иной, - в вашем понятии, - праведник, -  скажет: ``Даёт энергию''.}
\people{Значит, в нашем мире всё субъективно. Вообще нет ничего объективного?}
\soul{Вы же, сами спрашивали -  мир иллюзий. А мы говорили вам: сознание ваше – иллюзия. Вспомните. Спрашивайте далее.}
\people{Скажите, 4-е измерение - это и есть астральный план?}
\soul{Нет.}
\people{А что это?}
\soul{А мы, вам когда-то пытались…}
\people{Объяснить, да?}
\soul{Мы говорили, что вы живёте - в пяти и семи. (измерениях и мирах. Прим.)}
\people{Может ли физическое тело, стать бессмертным? Вообще, достижимо ли это?}
\soul{Да.}
\people{В какой ситуации?}
\soul{А вы подумайте.}
\people{А нужно ли, чтоб оно стало бессмертным?}
\soul{Это всё зависит от вас.}
\people{Тогда вообще не имеет смысла истину искать, если всё субъективно? Невозможно тогда?}
\soul{Разве? Чем  ищете? Мы говорили вам, если вы будете искать сознанием – вы ничего не найдёте. Подумайте. И  мы говорили вам, и, пожалуйста, объясните - ваши чувства, рождает что? Сознание?}
\people{Наверно, что-то духовное и душевное.}
\soul{Ну, вы подумайте, подумайте, много ли было сознания, когда вы были, в вашем понятии, в ``любовной горячке''? Много ли было сознания в том?}
\people{Да… Сплошное бессознание.}
\people{Только сердцем, значит, можно понять?}
\people{А что для человека лучше, жить сердцем, или, всё-таки, умом, как многие живут?}
\soul{В вашем мире, жить сердцем? Только сердцем? Подумайте, что будет, и как вы примете его?}
 
\people{Сплошные удары будут.}
\soul{Хуже. Даже хуже.}
\people{И, тем не менее, надо же согревать этот мир своим сердцем. Правильно же?}
\soul{Да. Но, подумайте, для чего, всё-таки, вы пришли в этот мир?}
\people{Для того, чтоб согреть этот мир. Проявить в нём свою любовь. Каждый по-своему смотрит на мир.}
\soul{И вам дали сознание. И вам дали сознание, которое погасило огонь вашей любви? Простите. Даже логически, здесь что-то не совпадает.}
\people{Сознание мы сами приобрели перед жизнью?}
\soul{Зачем? Если вы уже обладали видением сердца. Зачем же тогда было приобретать сознание? Чтобы погубить себя? Подумайте. Даже здесь вы не логичны.}
\people{А почему разделение у сознания и сердца?}
\soul{Нет здесь раздела. Сознание разделяет. Сознание решило и говорит о своей исключительности. Только сознание. Далее. Вспомните. Мы говорили вам, что вы приходите в одеждах, во множестве их. Вспомните. Мы приводили вам пример, что многие из вас приходят в одеждах врачей, палачей, тюремных балахонах и далее, далее. Вы, помните? Вот и подумайте, зачем, многие из вас, пришли в этот мир?}
\people{У нас ощущение, что человек, который живёт только умом, ну, например, политические деятели, они, в общечеловеческом плане, проигрывают, нежели те, которые живут сердцем, душой. Наверное, те, которые сердцем живут, они ближе к идеалу человека, и к эволюционному своему росту, да?}
\soul{Давайте скажем, что не существует такого раздела, где может жить только душой или только умом. Поймите. Даже, в вашем понятии, роботы имеют свой характер.}
\people{А верно ли, что внутренние и внешние противоречия являются источником его (эвол. роста. прим.) в параллельности? }
\soul{Нет. Чаще, это всего лишь ``тормоз''. Поймите, вы похожи на басню… Вы помните? И что из того? ( ``Лебедь, рак и щука”- басня Крылова. прим. ) Вы всю жизнь создаёте противоречия, создаёте трудности, чтоб с успехом их преодолевать. Нельзя ли  было б делать проще? Не создавать этих трудностей и идти прямой дорогой. Мы вам говорили, и говорили сегодня, в вашем понятии, что вы…}
1-2-3-4-5.
\people{Хорошо. Вы можете объяснить, может хоть коротко, что такое душа в человеке, в вашем понятии? Вы тоже это не знаете, да?}
\soul{Да поймите, что если вы узнаете, что такое душа, то вы уже не будете человеком, и не будете искать ничего. Поймите, мы вам говорили только что, о бесконечности. Неужели вы можете представить конечность души? Ну, подумайте, чтоб познать свою душу – значит, она должна быть конечна. Вот и подумайте.}
\people{Хорошо. Скажите, умирает ли астральное сознание точно так же, как и физическое, через определённое время, и отчего зависит продолжительность жизни астрального тела человека?}
\soul{Мы вам говорили минимум о семи телах, в которых умереть  должны, в каждом. Каждый. Вы помните?}
\people{Угу.}
\soul{Вот вам пример. И, простите, можно ли сказать о длительности вашей жизни? Кто-то и вас живёт, всего лишь год, кто-то  сто лет, кто-то тысячу. Разве можно сказать конкретно? А вы ведь, спрашиваете о подобной мере.}
\people{Хорошо. И мы в своём физическом сознании, и вы, и другие, так называемые ``тела человека'', составляют единый организм? Существующий одновременно и вместе, не отделённые разной мерой сознания, или мы не зависимы друг от друга?}
\soul{Давайте скажем так: вы живёте во всех мирах. Но только сознание ваше видит какой-то один. Подумайте логически. Если мы говорим, что у вас есть семь тел, то есть уже существует семь миров, более тонких, тонких и тонких. Вы, согласны? Но, сознанием-то вы определяете только в этом. А далее, идёт более тонкое сознание. Значит, вы живёте уже одновременно во всех. Минимум - в семи. Возьмите далее. Возьмите далее, и тогда получится - вы живёте во всех, но, только сознание ваше, находится на одном уровне. И представьте, что где-то, в одном из тех миров, астральных или далее, тоже ведётся контакт. В вашем понятии - лежит астральное тело, и ведёт контакт с более тонкими. Более тонкое лежит,  тоже, где-то ``переводчик'', и ведёт более и более с тонким и далее, далее. Подумайте. Вы, и религии ваши призывают вас – уйдите в более тонкие миры. Хорошо,- вы ушли. Но у вас там будут религии те же. И будут говорить вам: уйдите в более тонкие ещё. Вы, согласны? Ваша логика.}
    
\people{Скажите. Мы мало знаем, вообще-то, о вас, как ни странно. Но, не являетесь ли вы, так называемой ``дэва-эволюцией'',  с которой человек тесно связан? И как наша жизнь влияет на развитие ``дэва-эволюции''?}
\soul{Давайте скажем так. Давайте напомним вам. Ведь мы когда-то говорили вам, что мы - это вы. Мы были вами. Вы забыли?}
\people{Да, нам это трудно понять…}
\soul{Тогда подумайте, какова теснота.}
\people{Вы, почему-то знаете значительно больше, чем мы…}
\soul{Разве? Мы же говорили вам о сознании. Мы вам говорили об астральном сознании, ментальном и далее, далее более тонком, тонком и тонком. Вот и пожалуйста. У вас одно сознание, а у нас другое сознание. Но, у нас нет, в вашем понятии, ``наивысших'', всех тех сознаний, о которых вы говорите, как о ``семи''. Ну, а вы слишком рано обожествляете. Простите, но у вас уже более седьмого – уже ``божественное''. Вот вам и ответ. Вспомните о Земле. (перепад) }
\soul{…а вы откройте, и мы придём к вам.}
\people{Хорошо. В таком плане вопрос: молитва, настрой, как бы и есть открывание желания войти с вами в контакт?}
\soul{Мы говорили вам, молитвы ваши только для того, чтобы настроить вас. В вашем понятии, гипнотизёр. Есть понятие – ``ключ''. Этим ``ключом'' может являться всё, что угодно лишь бы могло открыть вам. Поймите.  Вы может это сделать молитвой, можете это сделать перед иконой, перед свечкой, перед любой картиной. Может быть, на улице дуновение ветерка уже может открыть вам что-то. Поймите. Вы можете выбрать всё, и всё может являться ключом, если вы верите в него. Если ж вы не верите в молитвы те, - не поможет. Подумайте. К вам, пришёл сам Христос, но неверящие не приняли его. Если уж сам Христос не смог, то, простите, что сможет тогда молитва? Всё зависит от вас. Вы открываете двери. И Христос, потому и Христос, что не будет ломиться в чужую и закрытую дверь. Ибо это уже будет насилие, и нельзя говорить о добре.}
\people{Скажите, если вместо ``переводчика'' положить академика, с его обширными знаниями, словарным запасом – это будет существенное продвижение контактов?}
\soul{Вы говорите в словарном понятии? В словарном,– да.}
\people{Ну, и в познании…}
\soul{В познании?}
  
\people{Ну, не сравнить же, наверное, знания академика…}
\soul{Хорошо. Тогда подумайте и вспомните, мы вам говорили, что мы вам даём больше эмоционально, чем словами. И, подумайте, если вам будут говорить о душе научными словами. Многое ли вы поймёте? Вы ведь, не академик. Академик, знает более вас слов. И, когда он начнёт говорить словами теми, которые не знаете вы, поймёте ли что? Нет. Вы потеряете интерес.}
\people{То есть, у нас есть преимущество сейчас?}
\soul{Нет. Давайте, не будем говорить о преимуществе. Поймите: подобное - подобным. Намного ли отличается ваши словарные запасы друг от друга? Далее. Неужели, вы думаете, что ``переводчик'' не знает ваших слов? Поймите, красиво говорить и говорить правильно – совершенно разные вещи.}
\people{Мы, кстати, восхищаемся умением ``переводчика'', в трансе, говорить на столько, литературным языком, каким он не может.}
\soul{Литературным? }
\people{По крайней мере, связанно.}
\soul{Поймите. Мы даём вам эмоции. Мы находимся в плане эмоций, и потому даём вам эмоции. Слова, всего лишь, только несут те эмоции. Не более.}
\people{Скажите, вот вопрос есть интересный такой для нас, землян: христианская религия очень отрицательно относиться к проблемам секса. Первый вопрос: на сколько права в этом религия? И второй: как эта проблема рассматривается у вас и решается у вас?}
\soul{Давайте скажем так: вы разберитесь у себя, а уж потом интересуйтесь, как это у других.}
\people{Ну, то есть, религия тут не права, что человечество всегда ограничивать?}
\soul{Нельзя говорить так. Нельзя. Ведь многие из вас, могут отрицать или наоборот - впадать в крайности. Поймите, что вы редко находитесь в ``золотой середине''. Редко. Вы, или исключаете полностью, а у вас есть и такие люди, или наоборот - давайте только на это. Вот, ваши крайности. Найдите середину. В конце концов, у вас есть душа. Вы любите душой или чем-то иным? Чем же вы тогда отличаетесь от животных?}
\people{Ну, вы как-то сказали, что той чакре тоже нужна энергия.}
\soul{Да. Но слишком много вы можете ей дать или наоборот, обделить. Вот, ваши крайности. Простите, если вы будете делать ради ``спортивного интереса'' или наоборот, много ли будет от этого?}
\people{Скажите, как влияют на вас различные психические расстройства людей?}
\soul{Многое, многое, в вашем понятии, расстройства – неумение управлять. Неумение. Простите. Есть люди, у которых, в вашем понятии, чакры, отвечающие за эмоции так сильно открыты, что остальные заглушены. Например, чакры сознания, и далее, далее. И что? Кем вы их называете?}
\people{Умалишенные.}
\people{Сверхчувствительные, впечатлительные.}
\soul{Да нет. }
\people{Нет?}
\soul{Далее, возьмите человека, у которого развиты только умственные чакры и больше ничего. Кто он? Как вы называете?}
\people{Академик.}
\soul{Разве? А, может быть ``сухарь''? А теперь возьмите, если развиты только нижние чакры. Как вы его называете?}
\people{Сексуальный маньяк.}
\soul{И вы, можете, можете быть… Вам привести аналогию других чакр, или вы найдёте, всё-таки, сами? Мы просим вас, приведите( пример. Прим.).  Вам, что? Лень?  }
\people{Ну, тогда, может быть и не надо. С кем вам лучше иметь дело с людьми с психическими расстройствами или с нормальными, обычными, средними?}
\soul{“Умеющими слышать шёпот'', как вы называете нас.}
\people{Но, ведь это, чаще всего, как раз люди, которые сидят сейчас в ``жёлтых домах''.}
\soul{Да? Многие, из вас, должны были бы сидеть там, по старым временам. Ну, что ж, идите! Вспомните, старое время.}
\people{Кстати. Вопрос болезненный, сейчас, в некоторых регионах, ну, в Москве, существует группа, которая начинает очень плохо относиться к уфологам, их терроризирует и так далее. И нам было письмо о том, что можно ожидать нечто подобное и в других городах. Скажите насколько это фанатики, или сознательно работают, или их, какие-то силы подталкивают? Насколько это явление неприятное и грозное?}
\soul{Давайте, скажем так: есть враги, в вашем понятии,- идёт борьба. Да, большинство - всего лишь не хотели признать себя виновными. В вашем понятии - всегда должен существовать ``козёл отпущения''. И вы, чтобы снять с себя вину, говорите: ``Нам плохо от того, что чёрные силы пришли и мешают нам''. Вы не виноваты. Нет. Это пришли извне, и  мешают вам. Виноват у вас дьявол, но не вы''. Удобно, не правда ли?}
\people{Ну, это чисто Московская ситуация или…}
\soul{Разве? Более-менее увлекающиеся, в вашем понятии, аномалиями… Подойдите к нему и спросите, кто виноват больше, ваш вождь или ``чёрные силы''? Что он скажет? – ``Вождь виноват. Но, им управляют чёрные силы''.}
\people{Скажите. Находится ли город Волжский в особой геопатогенной зоне? И, как влияет сильная загазованность воздуха на эмоциональную среду волжан?}
\soul{А вы вспомните. Вы вспомните,- мы говорили вам о ``каменном мешке'' и ``клетке''. Как вы думаете? Влияет? Далее, вы можете назвать хотя бы один город, который не имеет в вашем понятии, этих зон?}
\people{Ну, есть предположение, что Волжский, как-то особо… ну, под нами, по крайней мере, большой разлом идёт…}
\soul{Простите, вы вредите себе более, чем те разломы.}
\people{Ещё вопрос тогда, экологический - сейчас очень болезненный. Сейчас, будет происходить в Волжском закачка сточных вод глубоко под землю. Это является очень большой угрозой для человечества, для состояния нашего региона и Земли даже?}
\soul{Да. Это первое. И далее. Ответим вам на прошлый вопрос. Ваш город находится в геопатогенной зоне. Ваши слова? Влияет это или нет? Вы подумайте. По вашим понятиям, когда-то ведь не было города.  Нет. Но зоны были. И раньше, простите, болели меньше теми болезнями, что имеете сейчас. Вот вам - и пожалуйста. Вы вредите себе более, более чем природа. Поймите. Природа кормит вас. А что делаете вы? Что? Вы - убиваете свою мать! Хотя, находитесь ещё в её чреве!}
\people{Понятно. Вы, можете привести аргументированный ответ по поводу вреда, который принесёт закачка сточных вод под землю?}
\people{Какие-нибудь конкретные последствия.}
\people{Может быть, сейчас объявим борьбу против этого. Может ещё не поздно?}
\soul{И что будет? Что будет? Если вы будете так активно бороться, что будете одним из ``экспериментов''.}
\people{Ну, попробуйте, всё-таки, нам подсказать, в чём особенный вред. Может быть, ничего страшного не произойдёт, постепенно земля очистит эти плохие…}
\soul{Да? С такими  понятиями вы начали жить. ``Постепенно очистит''. Вспомните, первый завод! Вспомните! ``Постепенно очистит''! До сих пор чистится!? Что вы думаете, что это куда-то исчезает? Бесследно, не оставляет следа? Вы же открыли закон – ``ничто никуда…''. Что же вы здесь не можете ёго вспомнить? ``Постепенно очистит''?… С загрязнением вод, вы просто мутируете и будете привычны. Привычны к новой Земле. Земле, изменённой вами. А значит, и тело ваше будет изменено. Подумайте и представьте, человека прошлых веков, если возьмёте и привезёте сюда, что будет с ним? Что!? Он погибнет, как погибнете и вы, если вы, дитя химии, придёте в тот мир, – мир чистоты. Вы, погибнете там тоже. В вашем понятии, это будет ``кислородное'' и далее, далее. (теряется) Пусть, один из вас, задаст по одному вопросу. Но, пусть это сделает каждый.}
\people{Скажите. Вы от 16-го числа,- я вот, сегодня прослушал контакт, - сказали насчёт распределения энергии по чакрам. Что вот, у вас одна большая была, а потом, когда вы разбросаете, все станут маленькими. И сравнили это с одной тарелкой супа на всю семью. А тут вы нам говорите опять насчёт того, что все должны быть равны. Так что-то я никак не могу додумать эту мысль.}
\soul{Простите, о части с супом приводите вы, но не мы.}
\people{Ну, ``переводчик'' говорил так.}
\soul{Далее,- мы вам говорили и говорим, и проводили аналогию о каждой чакре. Вы помните? Это было сегодня.}
\people{Да-да.}
\soul{А теперь, представьте, в вашем понятии,- человек, нормальный человек, не имеющий никаких отклонений… Подумайте. Он не сексуальный маньяк и не сухарь. А? Подумайте, какие у него будут чакры? Что? Они у него будут все разные?}
\people{Нормальных не бывает.}
\soul{Не бывает? Или вы не встречали? А может вы не встречали? Ибо всё подобные к подобным. А вы подумайте. Подумайте и посмотрите.}
\people{(…).}
\soul{Подобное к подобным. Это соблюдается. Ибо друзья - подобные вам. С другими вы просто не будете дружить. Вам будет или скучно  или скучно им. Подумайте. В вашем понятии, ``идеальный человек'' - он должен обладать всеми качествами, но, в вашем понятии, во многом, подобен.}
\people{Скажите, закачка вод ударит по нашим детям, внукам или ещё по более дальним поколениям?}
\soul{И далее и далее. Будет и вам, будет и внукам вашим. Но, вы пострадаете менее, ибо вы только начинаете. Другие продолжат. В конце концов, в вашем понятии, существует ``чаша'' и ``последняя капля''. И, кому-то из ваших поколений достанется эта ``чаша''. Но, я могу успокоить и огорчить вас, вы будете и в следующей жизни, достанется и вам. Вы будете, только, другим уже поколением.}
\people{Скажите. Сейчас, что-нибудь я могу сделать для того, чтобы этого времени не наступило?}
\soul{Каждый из вас должен.  Должен делать то. Если вы говорите: ``я делаю, но не вижу, что есть успехи'', значит, вы, не делаете или просто не видите. Поймите. Вы все хотите чуда. Вы хотите сделать так, чтобы всё было сразу. Подумайте. Вы растите детей. Вы формируете понятия о мире. От вас зависит, какими будут ваши дети. Вот, пожалуйста, одно из… (теряется). }
1-2-3…
 
\soul{Вы были невежливы, и перебили. Спрашивайте.}
\people{Я просто хотела спросить, вот вы говорите, что ничего невозможно сделать, что всё равно неизбежно будет падение наше духовное, нравственное, в общем, экологическое…}
\soul{Нет. Мы не говорили вам того. Иначе, мы бы не стали разговаривать с вами. Зачем разговаривать с теми, которые уже упали, и нет возможности подняться?}
\people{А что такое, и как фразу понять – ``знание надмевает''? В христианстве это часто говорят.}
\soul{Как вы понимаете? Что в вашем понятии? Мы говорили вам о середине. Вы, берёте крайность. Крайность…}
\soul{(перепад записи)}
(Конец первой части контакта.)
\soul{Каждый из вас – пройдёт нашу дорогу… и  далее, и  создаст только свою. Спрашивайте.}
\people{Скажите,  переводчику  очень понравилось обстановка здесь, в этом помещении. Насколько она действительно более чиста, более эмоционально светла нежели в том вагончике, где мы были? Где, действительно, могла остаться аура  зла и грязи, и так далее.}
\soul{Подумайте, мы говорили вам, что каждое слово ваше – влияет на вас. Подумайте, как вы спросили? Неужели мы вам скажем: здесь ``лучше'', там `` хуже''? Тогда, есть такие, в вашем понятии, места, которые можно  назвать `` адом''.  И тогда мы будем противоречить себе. Подумайте! Всё зависит от того, что принимаете вы. Ну, подумайте! Что же тогда получается, по вашим понятиям – к вам спустился Христос, чтоб ``испачкаться''? Испачкаться об вас? Вы подумайте,  - небесный мир, и ваш грешный мир… Он пришёл сюда, чтоб ``зарядиться'' вашими грязными эмоциями? Что говорите вы? Это непонимание ваше.}
\people{Но так почувствовал ``переводчик”…}
\soul{Даже в вашем понятии – можно придти в ВАС и остаться святым. Подумайте.  А можно и в раю набраться столько ``грязи”…}
\people{Скажите, у нас, -  у экстрасенса,  (вырезано).. .  что с ней сейчас происходит?}
\people{Откуда?}
\soul{Неужели вы думаете, что мы вам даём ``одежду''?}
\people{Но это у неё положительные эмоции, или же, нехорошие для неё? Для её здоровья.}
\soul{Простите, пусть она отвечает себе сама. Пусть каждый из вас отвечает себе сам. Почему вы всегда хотите о себе  узнать от других? Вы подумайте, что же вы делаете! Вы доказываете свою беспомощность. Вы не знаете себя, что вы спрашиваете у других? Что вы мне скажете?}
\people{Мы спрашиваем не про себя, а про одного человека, который сейчас старается…}
\soul{Вы спросите про себя.  Спросите себя.  Почему вы хотите увидеть своё отражение в других? Вы говорите : ``Я не знаю себя. Со стороны лучше видно''. А может гордость говорит вам? Вы хотите увидеть отражение, своё отражение в других. Вы боитесь заглянуть в себя. Ибо себя обмануть трудно, но очень легко обмануть других. И потому, вы носите ``маски''. Множество масок, которые вы меняете так часто, что уже забываете, какие вы ``настоящие''.}
\people{Тем не менее, может быть человеку надо уже врача вызывать, а вы тут   о боге  говорите.}
\people{Бесполезно.}
\people{Нет, ну, в данный момент вопрос так стоит.}
\soul{Простите, часто, очень часто, вы призываете на помощь, забывая о душе. Вся ваша медицина – физическая. И вы забываете о душе… Почему вы нас спрашиваете о физике, но не спрашиваете о душе? Вы не спрашиваете, что с её душой. Вы хотите физической помощи, но не духовной… Разве сможете помочь? Вы сможете помочь ей? Никто из вас не сможет помочь ей, пока выбудете говорить о физике и о химии, о медицине, что существует и что не признаёт духовное.}
1-2-3-4
\soul{Спрашивайте. }
\people{Скажите, чем отличаются эмоции от чувств? Или  это одно и то же?}
\soul{Нет. Это совершенно разные вещи, это совершенно разные миры. В вашем понятии – это совершенно разные.  Чувственные, астральные и далее и далее… Подумайте. Эмоции… Вам привести аналоги? А вы посмотрите, - кому-то плохо, а вам – радостно. Вам плохо, - а кому-то радостно. Вы видите здесь отличие? Подумайте. Эмоции – одни…  А чувства – разные. Кому-то приносит горе, кому-то – счастье. Ответ один, - эмоции … а чувства разные. А может быть и наоборот, - одни чувства, но разные эмоции… Кто-то плачет от счастья, кто-то плачет от горя. Подумайте.  Найдите в себе. Почему вы ищете подсказки извне, но не слышите себя? Порой и душа ваша кричит…   Кричит, а вы глушите её. Глушите столь жестоко…}
\people{Вы, конечно, очень хорошо разбираетесь в эмоциях, но подскажите, как человеку лучше жить, -  эмоциями или чувствами? Я представляю уже… но от вас хотелось бы услышать.}
\soul{Нет таких рецептов. Поймите, мы можем сказать, какие ваши краски, но не о всех.}
\people{Наверно, жить эмоциями,- это белее ранимый человек…}
\soul{Вы… Вспомните, мы говорили о распределении энергии… Вы помните? И подумайте, в вашем варианте – существуют чакры. Простите,  есть чакры отвечающие за эмоции,  и есть чакры,  отвечающие за чувства.  А мы вам говорили: ``Распределите  поровну''.  Не унижайте. Не унижайте другие и не возвышайте. Мы ответили вам? Поймите, мы говорили вам, что у нас нет эмоций и нет чувств. Почему? Да потому, что у нас нет понятий ``выше'' и ``ниже''. Мы не говорим и не спрашиваем, где лучше жить – в эмоциях, чувствах? Может надо жить и там и там? Может быть, все чакры, в вашем понятии, должны быть едины? Едины. А вы их распределили, растеряли – здесь больше, здесь – меньше.  И ваша ``змейка'' мечется вверх - вниз, вверх -  вниз. Вот вам и ``маятник''.  И что вы получите? Подумайте, мы когда-то говорили о падшем ангеле… Вы помните? Вы помните, вам будет дано дополнительно, в вашем понятии, энергии только тогда, когда  распределите поровну, а не будете унижать одних и возвышать других. Но, получив новую порцию, вы можете опять кого-то ``обеднить'', кого-то ``обогатить''. И тогда опять вам будет ``падший ангел''. Вы помните? Мы же говорим вам: верьте себе, ищите вопросы в себе. В себе…  И никогда не верьте, не верте другим. Но и не говорите, что они лгут. Ибо, каждый – живёт в своём мире, и у каждого – своё понятие, если он един. И в том – различие. Вы – найдёте в себе. Вы никогда не найдёте готовых рецептов, готовых ответов, потому, что вы всегда любой ответ  пропускаете через себя. Мы вам говорили: Вы  ``сито''. Сито. И, чаще, сознание ваше искажает сита ваши. Сознание говорит: То - ложно, а то – правдиво. Ибо вам говорят эмоции – это лучше, это хуже. Ибо, вам говорят чувства - это хуже, а это – лучше. Вот вам – чувства, вы приходите к врачу, вам делают боль. Что говорит ваше чувство? ``Плохо.  Не надо''.  Что же будет, если вы не пойдёте к нему? Вот вам – жить одними чувствами. Хотите привести аналогию, как жить одними эмоциями? Спрашивайте далее…}
\people{А как, -  не искажать? Можно ли, вообще, не допускать?}
\soul{Прежде, научитесь слушать и видеть себя. Вы когда-то спрашивали, как можно развивать чакры. Мы вам говорим: Представьте! Больше, чем она есть, вы не представите. И всё зависит, как вы говорите, от ``фантазии''. Вы можете представить жёлтый цвет…  А может быть красный вам представляется лучше. Значит, у вас красного больше. Потому, что вы представляете его лучше. Вы можете представить себе чакры, как ``шары''. Шары, расположенные на вашем теле. А теперь подумайте, - к вам приходит человек, который не понимает, никогда не слышал о чакрах… Что, они у него не развиты? И если вы ему скажете: ``Представьте'', он их представит в вашем количестве?  И в вашем виде? Он может представить или одно или множество. Ибо он не знает тех правил, которые созданы вами. Вы сказали – у вас семь тел, у вас семь чакр и далее и далее…}
\people{Скажите, вот вы, живёте в мире эмоций, насколько нам известно…}
\soul{Нет! Мы к вам приходим. Приходим  к вам в мир эмоций, но не в мир чувств. Подумайте, если б мы были в мире чувств, чтоб мы делали вам сейчас? Кому-то из вас было б сейчас плохо, кому-то – хорошо. И что б мы делали сейчас? Мы б не могли, не могли отвечать так, чтоб было понятно каждому. Мы же – приходим в мир эмоций, в вашем понятии. Эмоции действуют на вас, и те  чувства - рождаете вы.}
1-2..
\soul{Вы приходите, и эмоции окружающих влияют на вас, потом, рождаете чувства. Ибо чувства рождаете вы, а эмоции – это  извне. Мы же, не можем придти к вам многим. Ибо вы не открываете, а мы не идем силой. Спрашивайте.}
\people{А скажите, эмоции имеют энергетическое какое-то значение для вас?}
\soul{Да. Оно имеет и для вас. Вы эмоциями можете делать многое. Эмоция может убить вас и может дать вам силу.}
\people{А для вас, перепады эмоций сказываются как-то? То есть, вы не набираете энергии или, наоборот, её слишком много  выплёскивается?}
\soul{Нет. Вы никогда не сможете выплеснуть слишком много или слишком мало. Поймите, МИР – ЕДИН. Столь един, что изменение, каждое изменение в вашем эмоциональном плане влияет на общее. Изменение общее – влияет на ваше. И нельзя сказать, кто влияет больше и кто первый. И если б  вы сказали, первый ваше или они, и мы бы тогда уже заговорили о времени. И говорили, что вы или они – главнее. А мы же говорим вам, что всё столь едино, что нельзя говорить, нельзя разделять, -  вы влияете, или они.}
\people{Скажите, существуют ли на Земле среди нас люди, внутренняя суть которых развивалась в других галактиках? Можно ли их считать инопланетянами?}
\soul{Вы считаетесь инопланетянами. Множество. Множество и даже среди ``спрашивающих''. Вы уверены в своём происхождении? Вы столь легковерны, что если мы скажем: вы – инопланетянин,- вы поверите нам. Если мы вам скажем обратное: Нет, вы землянин и всю жизнь были и родились  тут, - вы поверите. Но вы не будете слушать себя. Потому и верите, верите другим. Вы видите в тех красках, которые нравятся вам. Те же, которые не по нраву вам, вы говорите:  Ложь.  Вот вам и ``правда''. Что такое ``правда'', в вашем понятии? Действительность. Действительность , искажённая зеркалом души. Зеркалом, а не душой.}
\people{Вот вы говорите, что пропускать через себя надо и слушать только себя… Тем не менее, мы живём и делаем не только ошибки, но и грехи разные, про которые вы говорили, что мы, рождаясь заново, должны искупать их. Как, вот, связать эти два понятия?}
\soul{Мы говорили о сознании вашем. Вы привыкли и живёте сознанием. Поймите, если б вы жили, в вашем понятии – сердцем, то вы бы не делали множество грехов, но делали бы другие, в вашем понятии, грехи. А в других мирах это был бы не грех. Поймите, всё - относительно. Вы спрашивали о мире иллюзий. Да, в какой-то мере вы правы. Мир иллюзий… Даже здесь вы можете совершить какой-то поступок , и многие, в вашем же мире, примут его по-разному.  Кто-то скажет :'' Благородно'', Кто-то скажет: ``Нет''. Поймите, многие, в вашем понятии, грехи, что творите здесь,  являются  ``святыми'' и  наоборот. Есть и дальние миры, где всё наоборот. И в тех мирах – вы ветвь.  А мы вам говорили о стволе.  Чтобы вы шли, создавали  ствол, а не множество ветвей. Вы не можете понять это? И слава богу, в вашем понятии. Ибо, если б вы это поняли, поняли сознанием, вы бы это поняли сознанием и создали бы ещё множество ветвей. Потому и говорим вам, говорим вместо разума вашего, а для души. Подумайте, почему мы приходим в мир эмоций. Если мы говорим, что мир эмоций – внешний мир. Мы больше даём вам эмоциями, чем словами. Поймите.}
1-2-3…
\people{Скажите, существуют ли планы, где совсем не развита любовь, а только развивается интеллект? К чему это приводит?}
\soul{К чему? Вы много описали таких миров и назвали ``фантастикой''. В вашем понятии – мир роботов. И подумайте, среди вас есть многие, есть даже те, которые знают практически все (науки.прим.)которые имеют очень, в вашем понятии, большую чакру  ума и искусственную чакру сердца. И вы называете их ``’экстрасенсами''. Многие, многие создают искусственно. Вспомните, мы вам говорили, что представьте чакру… Представляйте всё больше и больше… Вы помните? И это был ответ на вопрос, как развить эти чакры. А теперь подумайте, можно развить её столь большую! Но, что толку? Что она вам даст, если она будет больших размеров, но пустое? Подумайте, мы вам говорили о любви. И что можете быть раздавлены. Вы помните? А подумайте, в вашем понятии, есть…  Извините, у вас есть такое понятие -  хороший человек живёт мало… Сердце! Подвело сердце. Вот и подумайте. Подумайте, - чакра открыта… Открыта, - это прекрасно! Но она не имеет сил для защиты… И потому, в вашем понятии, страдает сердце. Вот вам – и влияние на неё…  Вот вам – и несовершенство ваше. Несовершенство ваших  чакр… Вы можете открыть их, но вы должны их открыть так, чтобы не бояться, что будет затоптана та чакра. Если же вы открываете, в вашем понятии – открыл душу, или – ``Я бы открыл… Я бы открыл душу, но боюсь, что в ней натопчут”…Если вы будете думать так, то уж лучше не открывайте. Ибо, действительно,  любой свет будете принимать за грязь. А мы вам говорим, что нет, нет врагов, есть только учителя. Вы же, чаще, принимаете за врагов… Даже друзей вы принимаете за врагов… Мы вам говорили,  вспомните, о монастырях. Вы помните? Чаще, чаще, вы называете геройством то, что вам не доступно, далеко от вас.  Люди. Совершающие геройство – среди вас, на глазах у вас, – вы не принимаете за геройство. В вашем понятии, в ваших глазах – ``нет пророка в своём отечестве''. Вам больше нравятся ``уединённые'' и, как в вашем понятии, ``тайные''. Да, вот он ушёл в монастырь, он бросил детей, он бросил жизнь и отдал себя богу… И вы восхваляете. Рядом с вами товарищ ваш и враг ваш, борется за вас, помогает вам, учит вас… Вы не называете его героем… Чаще, наоборот… Вот и подумайте.}
\people{А есть ли  мир отличный от нашей реальности? Абсолютной реальности мир.}
\soul{Абсолютной реальности? Нет, как это не печально.  Но вы всегда будете искать. Подумайте…  Вы мечтаете об идеальном мире. Вы называете это `` абсолютом''. Неужели вы думаете, что есть абсолют, и ``выше'' нет ничего? Тогда, в вашем понятии - ``там'' кончается эволюция. Нет… Нет пределов. И вы должны не искать ошибки в том, а радоваться.  Ибо вы всегда будете выше, выше и выше… Да, для вас,  абсолют сейчас – очень маленький. Очень маленький, по сравнению с другими. Ибо вы только начали, начали жить.  Подумайте, приведите аналогию, - для многих, в вашем понятии, детей, - первый класс – абсолют. Придя в первый класс, будете мечтать уже о втором, третьем и далее. Вы – согласны?}
\people{Да.}
\soul{Для многих, абсолют – десять. Пройдёт десять, и он будет мечтать учиться далее. И так будет ``повышаться''  абсолют всегда. Даже пройдя все институты и университеты, но будет учиться далее. Если он не будет учиться, что будет? … Ничего. Все знания его пропадут даром, только и всего.}
\people{Получается,  вся жизнь – учёба?}
\soul{Каждый имеет каждый миг – своего учителя. Каждый миг он совершает или верные, или не верные поступки. Мы вам говорили, что вы – человек. И счастье человека в том, что он умеет совершать ошибки. Но это и его беда.}
\people{А верно ли такое выражение, что все события, которые происходят с нами, только потому происходят, что мы их сами притягиваем в свои жизни желаниями нашими?}
\soul{Подобное – к подобному…}
\soul{Так… Земля считается не священной планетой… В чём её отличие от священных планет?}
\soul{Как вы считаете её не священной планетой? Вы, что, можете её сравнить с другими планетами? Вы видели священные?}
\people{Ну, это по легендам, изотерическим там…}
\soul{Н-да? А для более низких миров вы священны. Мы вам говорили только что. Мы вам говорили о первых и десятых классах.  Вы не внимательны и не можете… Мы же говорили вам, что мы вам даём универсальные, в вашем понятии, ответы, что бы вы могли, могли эти ответы привести ко многим вопросам. Поймите,  всё - относительно. Да, для многих, это - ад. Но есть и более низкие планы. Для них, ваша планета – рай. Так, как вам сказать, с какого уровня она священна или нет?}
\people{Скажите, а вашей цивилизации, вашей субстанции, тоже свойственно чувство гордости собой, своим умом?}
\soul{Мы тоже ищем. И у нас тоже есть понятие чувств, эмоций, но другое. Другое.  Далее. Вы говорите о понятии…  Поймите… Вы говорите: Прошлый контакт хуже, чем сейчас… И далее, далее… А может быть проще, - может вы спрашивали иное? Далее…  Почему вы тогда не говорите: ``Детский сад, первый класс и далее, это всё чепуха, ерунда.'' Это  аналогии у вас. Я беру  его слова. (переводчика. Прим.)  Да. А если не выучить вас, кем бы вы были? Вы бы пошли бы в десятый и далее?}
\people{Скажите, пожалуйста, а вы знаете ответы на все вопросы?}
\soul{Вы знаете ответы на все вопросы. На все. Нет такого вопроса, на который вы бы не знали ответа! Потому, нет у нас и вопросов. Но вы не слышите. Не слышите, потому нам и приходится быть посредниками, переводчиками.  Поймите, мы говорили вам , что вы и отвечаете на свои же вопросы.  Перед вами лежит переводчик, но мы являемся переводчиками более кого-то и выше. Тот – выше и более и более, и в итоге – вы слышите своё через множество переводчиков, и каждый из них – искажает.}
\people{Вот, наверно поэтому  к нам бывают упрёки, что наши контактные ситуации, ответы, вполне банальны, на уровне среднего интеллекта, а вовсе не Высший разум, как нам это хочется думать.}
\soul{Давайте представим, - вы хотите новое…  Ну, хорошо. Как вы можете определить, что вам дали что-то новое? Как вы можете определить, если вы не знаете старое? А быть может, то новое, что вы можете найти здесь, для многих,  было старым? Подумайте! Подумайте, что вы делаете? Даже ваши, ваши понятия новых религий, что создаёте сейчас, - ведь они рождались тысячелетия назад…  Литература ваша – прошлого века! Где ваше  новое? Ибо всё  прошлое, - позабытое старое. Вот вам – новое.}
\people{Если мы искажаем ответы, никогда не узнаем правильный ответ.}
\soul{Нет правильных ответов.}
\people{Скажите, а вот новые религии… Мы сами же их создаём, как вы говорите, да? Вот сейчас (1994г. Прим.)во многих  течениях тоже… Вот, Виссарион… Есть предположение, что  это бывшее воплощение Христа. И, во всяком случае, от него идёт тепло  и хочется идти его дорогой. Это надо слушать своё, собственное, да?}
\soul{Слушайте себя.  Не слушайте нас. Не слушайте нас. Верьте. Подумайте, неужели вы будете слушать нас, если мы скажем; это - верно, это - не верно, идите или не идите? Вы не будете слушать нас и сделаете правильно, -  вас есть душа. Пришёл Христос, и есть те…  Не все приняли его. Вы  согласны?}
\people{Да.}
\soul{Подумайте, почему?}
\people{По вере каждому даётся.}
\soul{Подобное – к подобному. Поймите. И слушайте себя. Вы создали множество религий, но когда мы говорим ``создали'',  это не значит, что это выдумано вами, нет. Это, одна из форм выражений. Выражений в том, что вы не можете понять, ибо чувства ваши не имеют слов…   Их имеет только сознание. И сознание принимает чувства ваши. Потому сознание ваше создаёт религии. Религия, если хотите, это – перевод сознания ваших чувств.}
\people{Вот, наверно лёгкий вопрос  у нас, для вас… Чувства, эмоции, ум, интеллект – развиваются по отдельности в человеке, или тут  у нас идёт смесь взаимодействий?}
\soul{Ну, представьте…  Вы же знаете прекрасно ответ. Вы можете представить'' раздельно''?}
\people{Я думаю, что всё это с детства идёт.  Хорошо. А чем интеллект отличается от интуиции?}
\soul{Интеллект, это одно из форм сознания. Интуиция, если хотите, это перевод сознания –“крика'' души.}
\people{Хорошо. А инстинкт? Чем отличается от интуиции? Может более…}
\soul{Инстинкт? В вашем  понятии это что-то плохое. Но вы забываете и не можете понять… Если мы вам скажем, что любовь ваша – проявление инстинкта… Поймите, слово ``инстинкт `` искажено вами. Искажено вашими теориями. Теория о создании человека и далее. А может быть инстинкт - более  лучше, чем вы думаете? Что тогда он там делает в вас, в вашей личности? Что? Подумайте, что заставляет вас создавать, создавать всё, что уже создано вами? Что? Инстинкт самосохранения. Но, ваше сознание… Сознание , подчиняясь инстинкту самосохранения, переводит  и всё делает по-своему, и, в итоге, всё хуже и хуже… Вы становитесь зависимы от вашей же созданной техники. Ранее, человек выжил бы более, чем сейчас, лишившись  ваших ``игрушек''. Ваше сознание отделяет инстинкт. Природа дает вам инстинкт, и всё  сводится к тому. Вы же – отделяетесь от  природы столь упорно и столь быстро, что существовать будете… (вырезан кусочек записи . Прим.) …так сильно, что будете создавать всё искусственное и забудете всё натуральное. И когда отберут у вас те  `` игрушки'',- придёт такое время, - ведь кому-то может надоесть ждать, когда ж вы всё-таки очнётесь, и когда у вас заберут, как вы забираете у детей спички, и тогда, в вашем понятии, – вы погибните… Погибните в этих телах, в этом мире… И вы будете говорить ещё об одной погибшей цивилизации, которая была столь катастрофична, что забыла себя…}
(Конец контакта.)
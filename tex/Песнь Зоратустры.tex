Песнь Зоратустры
Георгий Губин
Радость моя - ЖИЗНЬ!Обращаюсь к тебе,Сердце моё! Пой песню Радости сильней! Пусть она заглушит стоны ига моего Разума! Пусть услышит он песню Свободной Души! Пусть не соблазнится преходящим в мире и познает вечные ценности Бытия всего сущего во всех Мирах и пространствах!
 Радость моя,Жизнь! Пусть звуки твои несутся по великой Бескрайности, озаряя чутких и ободряя уставших в Путях своих!
 Радость моя, Жизнь! Пусть огонь твой согреет окоченевшие умы всех творящих бездушие и тех, кто спит.Коснись их языком пламенным , пусть откроют они сердца свои, прочтут тебя там и скажут: - Радость моя - ЖИЗНЬ!
Нет в сознании Вечности, ибо преходяще сознание, живёт одну жизнь и пытается насытиться ею.Поэтому радуется,когда притянуло к себе своё,невечное, а потому грошовое и бессмысленное.Потому-что придётся однажды расстаться с этим и  память о том рассеется навсегда! Но не той радостью поёт сердце моё!
На что мне тратить драгоценное время? На Тлен? Да разве за тем я пришёл сюда,что бы пройти Путь гружёный скарбом и стать закованным в кандалы безумия, охватившего эту чудесную планету? Я что - сумашедший?
И не в том беда людей,что имеют, а в том ,что имеют слишком и хотят более… В конце-концов пресыщаются, но до некоторых всё-таки доходит когда-то,что в этой гонке они потеряли нечто большее, не успели ощутить саму Жизнь! И потому радость их - тлен!
Но там,куда уйдут они после, не то злато надобно, а иное… И где оно зарыто, знает только чуткое сердце да ясный разум…
Не тот разум, что ворует грошовое, а тот, что бережёт и преумножает нажитое.
И знать должен тот разум, что отдавая - не теряешь. И знать должен тот разум, что приобретая, нужно отдать. И чем больше отдаёшь - тем больше приобретаешь. И кто не понял слова эти - тот не понял, зачем Мир был создан. И прибудут силы в том, кто внял в сердце своём слова сии. На том стоял, стоит, и стоять будет МИР наш!
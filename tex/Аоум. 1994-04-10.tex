Аоум глава 16 10-04-94 г
\people{Скажите, пожалуйста, сегодня проводник  очень тяжело входил в контакт, чем это объясняется? У него не было настроения? Он говорил: у него всё хорошо. Он готов работать''. То есть, не очень хорошо получается это у него? Да? Вы можете более внятно? Почему-то не идёт звук. Попробуйте по воздействию, более внятно.}
\people{1-2-3}
\people{Сегодня 10 апреля. Мы рады с вами встретиться вновь, после длительного перерыва. Мы хотим продолжить  вопросы по судьбе России. Всё-таки она нас тревожит. Там не много вопросов. Вы можете вкратце отвечать? Вы готовы?}
\soul{Спрашивайте.}
\people{Хорошо. Вот, вы говорили, что на отрицательных примерах социализма воспитывались все другие страны. Иного пути не было для такого учения?}
\soul{Разве мы не говорили вам подобное? Мы говорили вам, что вы – учителя.}
\people{На нас был эксперимент, поэтому все другие воспринимали, как не надо жить.}
\soul{В вашем понятии, вся ваша жизнь — эксперимент. Когда же тогда вы живёте?}
\people{Вы правы. Но нам кажется, что капитализм более бездуховен, и вы об этом, нам кажется, говорили. Но  почему же России приходится вновь приходится повторять все ошибки ранней стадии капитализма?}
\soul{Давайте подумаем. Первое, - мы вам не говорили о духовном? Мы вам сказали, при любом строе можно жить. Далее. Вы же, вся ваша жизнь - это кольцо.}
\people{Понятно.}
\soul{Что вы поняли?}
\people{Ну, то есть, повторяются любые ошибки.}
\soul{Нет. Вы ищете, вы пытаетесь найти всё, что угодно, лишь бы вам помогло. Вы забываете о духовном и ищете материальное. И при любом строе, если вы не изменитесь, вы не найдёте лучше.}
\people{Мы убеждаемся, что капиталистические тенденции, которые сейчас насаждаются в России, ведут к ожесточению людей, к замкнутости, к отсутствию взаимопомощи. Почему надо повторять этот явно не христианский путь?}
\soul{Вы хотите сказать, что все остальные страны поддерживают христиан? Кто они? Далее, - почему вы говорите о всех? Расскажите о себе. Что вам дало это? Как вы можете осуждать других, если вы даже не знаете себя?}
\people{Ну, даже на примере близких знакомых я вижу, что стремление к наживе, деньгам, ведёт к их ожесточению.}
\soul{Мы говорили, что пришло время, должно всё всплыть ваше, чтоб вы увидели своё, испугались и, быть может тогда найдёте правильный путь. Вспомните. Посмотрите историю и вы найдёте, что твориться сейчас, и что приводило?}
\people{К революции.}
\people{Но надо стойко переживать эти испытания? Не страдать, не паниковать? Так вы хотите сказать?}
\soul{Что в вашем понятии ``стойко''?}
\people{Ну-у… Как богом дано, так и ``не возражаем''. }
\soul{Это ``стойко''?}
\people{Ну-у, смирение. Может быть, так и надо смиряться, а мы  бунтуем, пытаемся найти лучшие пути, неудачные подчас?}
\soul{Вы хотите быть рабами или хотите жить? Или вам нравиться, когда над вами экспериментируют?}
\people{Нет. Наверно, ни то, ни другое. Мы хотим самостоятельно решать.}
\soul{Как вы можете говорить о ``стойкости'' и о ``смирении''? }
\people{Ну, почему ж тогда Христос говорил: ``не противься злому, если тебя спрашивают, отдай последнюю рубашку''? На той стадии, видимо, это было правильно?}
\soul{Давайте скажем так, - что всё-таки, многое, что вы читаете, не сказано им. Далее,- как вы можете понять, если мы скажем: ``нет добра и зла''? Тогда, вы будете творить всё, что угодно. Вы согласны? Как вы можете понять, делаете вы зло или делаете добро? Мы говорили вам, что вы лжёте. Вы лжёте постоянно. Лжёте себе. Не верите себе, и потому, мы нужны вам. Ваши же ответы, мы должны говорить вам, потому что вы не верите себе.}
\people{Ну, это наше несовершенство. Да.}
 1-2-3
\people{Нам, подчас, авторитетнее услышать ваш ответ, чем даже такой же, и более умный, - моего соседа..}
\soul{В том и беда ваша.}
\people{Да. Скажите, поможет ли возрождение религии в России становлению лучших качеств русской нации?}
\soul{Религия помогает, если не переходит в крайности.}
\people{А эта угроза есть? Да?}
\soul{А разве не было того?}
\people{Когда-то было. Но мы религию вообще отрубили религию. Вырубили.  Наверно не правильно поступили?}
\soul{Вы любите крайности. Вам, или всё, или ничего. Вспомните. Даже вожди…}
1-2-3…
\soul{Спрашивайте.}
\people{Были ли верующими людьми, наши - президент академии наук Келдыш или Андрей Сахаров, великие умы России? Они уже умерли.}
\soul{Они же сами отвечали вам. Вспомните и подумайте. }
\people{Но я не помню прямого заявления.}
\soul{Хорошо, давайте скажем так: да.}
\people{Хорошо. Вот это нас и удовлетворяет. Надо, подчас, и коротко отвечать.}
\soul{Далее. Как вы понимаете – ``верующий''? Стать и говорить с богом? Канонически? Или может быть ``вера'', это что-то другое? Мы говорили вам, - религия, вера и бог -  разные вещи. Нет среди вас неверующих.}
\people{Скажите, какие из передовых стран раньше придут к справедливому строю для всех людей?}
\soul{И что даст вам этот ответ?}
\people{Но мы хотя бы будем ориентироваться.}
\soul{Ориентироваться? Если вам напророчат будущее, вы будете стремиться к такому будущему? Вы будете ждать, сложа руки, ибо оно ``всё равно'' сбудется.}
\people{Мы, скорее всего, скептики.}
\soul{Далее,- мы же говорили вам, мы не прогнозисты. }
\people{Хорошо.}
\soul{И не гадалки. Спрашивайте.}
\people{Если социализм не состоятелен, как общественно политическая формация, то какой строй его заменит? Он имеет название?}
\soul{Мы говорили вам, -  любой строй. Вспомните.}
\people{Но рабовладение не заменит. И феодализм тоже, по-моему, не прогрессивный. }
\soul{А что вы знаете?}
\people{Но рабовладение, это, вообще, эксплуатация одних, другими. Куда же…}
\soul{А что вы делаете сейчас ? Вы, только поменяли название, и не больше.}
\people{Да, верно.}
\people{А кто ж тогда рабы в нашем обществе, а кто рабовладельцы? У нас сейчас нет, допустим, партийной элиты.}
\soul{Вы забыли спросить, кому вы хозяин.}
\people{Кому мы хозяин? Лично мы - никому, пожалуй. Значит, получается, мы – рабы? Ну, тоже не ощущаем.}
\soul{Спрашивайте.}
\people{Будет ли царь на Руси?}
\soul{Нет.}
\people{Переходный период после распада СССР очень болезненно происходит. Это из-за новых ошибок руководства?}
\soul{Вы всегда делали ошибки. Далее,- придёт время, и ваша страна возродится и будет такой, какой была.}
\people{А мы застанем это время в нашей физической оболочке? В сегодняшней.}
\soul{Вас это сильно интересует?}
\people{Да.}
\soul{Вам назвать сроки?}
\people{Ну… Не точно. Ну, хотя бы при нашей жизни в этом теле мы застанем это счастливое время?}
\soul{Мы говорили о ваших телах. И говорили, что ждёт вас.}
\people{Вы говорили о печальной судьбе России до 2000-го  года.}
\soul{Мы говорили о вас. Вы живёте в России. Как может быть она печальна, а вы - нет?}
\people{Понятно. Но кому эта печальная судьба России угодна, Богу или Сатане?}
\soul{Что вы хотите сделать? Чтоб мы были ``гадалками''? Мы отвечали вам – прогнозировать…}
\people{Вдруг вы ошибаетесь о печальной судьбе России?  Всё-таки, Силы какие-то  поднимаются.}
\soul{Поймите, мы пришли рождать сомнения, а вы хотите получить ответы. Когда ж вы научитесь думать сами? Когда вы научитесь видеть? Почему вы ждёте поводыря, почему не снимите вашу ``повязку''?}
\people{Мы используем и те и другие моменты. То есть, как сомнения,  и как подсказки.}
\soul{Так кто ж вы тогда? Не рабы ли здесь?}
\people{Ну, может быть. Да-а… Нам, вот, кажется ошибочна и не совсем человечна нынешняя политика правительства, которое душит налогами любую инициативу людей, не стимулируя их труд. Так происходит потому, что у власти вновь не совсем умные люди, или они выполняют определенную роль, и ведут чью-то политику?}
\soul{Мы вам сказали о печали. Вы забыли о духовном. Вы ищете только материального. В вас нет духовного. И не намного-то вы и иные, по сравнению с другими строями. }
\people{Ясно. (вырезан кусок записи)… повернуть его в ``ту'' сторону.}
\soul{В любом случае, вы наследите. }
\people{Угу… Ясно.}
\soul{Ваша задача не наследить, а оставить следы, по которым пройдут другие.}
\people{Ага. Вот в чём дело. Понятно. Спасибо.}
\people{Верно ли, что из России хотят сделать сырьевой придаток ведущих западных стран, и кто-то охотно выполняет эту задачу?}
\soul{Да. Здесь вы правы.}
\people{Значит, в правительстве есть люди, которые толи недопонимают, толи хорошо понимают выгоды от этого? Да?}
\soul{Вы, иногда, понимаете, что вам нужно, и что нет. Всё равно делаете. Вспомните, вам многое не нравиться, но вы делаете, потому, что вы – рабы обстоятельств. Мы говорили вам, что вы любите подчиняться одному индивиду…}
1-2-3…
\soul{Приведите аналогию.}
\people{Привести аналогию чему?}
\soul{Сейчас говорит один, и вы слушаете. У вас один правитель, и вы его слушаете. Даже если недовольны его указом, вы всё равно выполняете. Так кто же рабы?}
\people{Ну, вот, мы оцениваем, и, может быть, сделаем совсем другие выводы.}
\soul{Вы говорите `` рабовладельческий строй''. Вы живёте в нём! Посмотрите!}
\people{Тогда, надо жить в анархии? Каждый ``сам себе голова''?  Так получается?}
\soul{Разве?}
\people{Ну, а как?}
\soul{Давайте скажем так, вашими словами  - вы можете участвовать в (…) любого? Ведь вы не знаете мир!}
\people{Скажите, правда ли, что в России грядет диктатура?}
\soul{А вы посмотрите историю.}
\people{1-2-3}
\soul{Спрашивайте.}
\people{Какую ``историю'' посмотреть?}
\soul{Посмотрите, это всегда приводило, в вашем понятии, к диктатуре. Далее,- сейчас будет репетиция. Спрашивайте.}
\people{Ясно. Репетиция.}
\soul{И если вы будете говорить с чёрными силами - вы просто хотите уйти от ответа. Мы говорили вам, что всё зависит от вас. Никто извне не придёт, и не будет мешать вам, если вы задумали…}
\people{1-2-3}
\people{Вы говорили, что Ельцин дан российской нации, как экзамен. Этот экзамен уже начался?}
\soul{Он идёт всю жизнь. Разница только в том, в какой ``фазе''.}
\people{1-2-3}
\people{“В какой фазе'' вы сказали? Значит,  ожидаются обострения, коли уже заявлено о возможной диктатуре? Вы это тоже чувствуете по настроению?}
\soul{Поймите, мы не будем отвечать на подобные вопросы. Что вы будете делать, если будете знать правду?}
\people{“Бить в колокола'' будем.}
\soul{Не будете. А если и будете, никто вас не услышит. Как вы можете доказать, что вы человек, что вы - живой, если я не хочу этого знать и в это верить? Вы мне сможете доказать?}
\people{Но мы могли бы, что-то писать в газетах, что-то рассказывать, как-то будировать людей.}
\soul{Теперь представьте, как это будет выглядеть. Я понимаю, если б вы были политик, и вы бы говорили ``так'' и ``так”… А если вы скажете: ``нам сказали в контактах''… На что это будет похоже?}
\people{Понятно. Но, вас только это смущает? }
\soul{Это смущает вас.}
\people{Ну, мы не будем на вас ссылаться. Сделаем выводы. А может быть, и в политику ударимся. В конце концов, не настолько мы погрязли в себялюбии, страхе. }
\soul{Здесь вы ошибаетесь.}
\people{То есть, вы чувствуете, что мы равнодушны к судьбе России? Да?}
\soul{Нет. Мы говорим о вашем эгоизме. Вас больше будет беспокоить не судьба России, а что будете делать с ней, если мы вам ответим.}
\people{Ну, это…вы правы…  Что необходимо предпринимать для объединения наций и наработки новых идеалов? Есть у вас к нам такие пожелания?}
\soul{Мы же говорили вам о духовном. Что вы хотите ещё? Самое главное вы теряете. Что в вашем понятии ``смерть''?}
\people{Мы уже поняли, что это переход в другую фазу, потеря тела - и всё, - жизнь продолжается.}
\soul{Ваша жизнь? Хорошо. Телом двигала душа. А что двигает душу?}
\people{Дух, наверное.}
\soul{Хорошо, что двигает дух?}
\people{Ну, наверно, разум высший. Что-то ``божеское''.}
\soul{Тогда скажите, что двигает высший разум?}
\people{Ну, тут уже за гранью наших предположений. Мы не можем ответить. Ответьте сами.}
\soul{Ищите.}
\people{Вот, Нострадамус предрекал что-то вроде третьей мировой войны в 1999 году. Это начнется на территории России или из-за нас начнётся?}
\soul{Вы вспомните, мы отвечали на этот вопрос. Далее, - имеет ли значение, где будет нажата ваша ``красная кнопка''? Имеет ли это значение, нажата в России или ещё где-то, если не будет более ничего?}
\people{Но она будет нажата? По одним сведениям - Нострадамус очень много ошибок делал. По другим, наоборот, очень точные прогнозы давал. Как вам кажется?}
\soul{Подумайте, если мы скажем ``будет нажата кнопка'', и вы будете публиковать, что ``будет нажата''.  Вы, спровоцируете, и она действительно будет нажата. Ибо кто-то подумает: `` Нажата?  Пусть это буду я''! И подумайте, если мы скажем ``нет''. Что вы будете делать? Будете продолжать так же жить?}
\people{Ну, конечно, вашего авторитета не так уж много, чтоб мы радикально что-то меняли. Но мы начинаем уже считаться с вашим мнением.}
\soul{Если мы придём авторитетом, что вы получите от того? Вы должны быть авторитетом, прежде всего себе и остальным. Почему вы хотите помощь извне? Вы живёте в России и просите помощи. Просите помощи у нас. Россия? А вы - где? Вот вам аналогия. Вот вам  - ``всё во всём''. Вам не хватает всё и вся. }
\people{Минуточку. У нас экстренный этот… Можно переводчика временно из контакта вывести?}
\soul{Выводите. }
(Продолжение сеанса)
\people{Скажите, вам не сложно было в третий раз выходить сегодня на контакт?}
\soul{Сложности только у вас.}
\people{То есть, посредник не очень хорошо воспринимает такие перерывы?}
\soul{Вы можете объяснить, почему не воспринимаете вы?}
\people{Да нет. То есть, и посредник не может этого объяснить? Да?}
\soul{Нет, мы говорили вам, что вы живёте во лжи. Если вы что-то хотите, вы или сможете пользоваться, или выкините.}
\people{Скажите, вы хорошо юмор понимаете наш человеческий?}
\soul{Да.}
\people{Наши анекдоты, шутки, розыгрыши вам тоже понятны и близки?}
\soul{Да.}
\people{Вы сами можете какую-нибудь шутку, анекдот, рассказать про нас?}
\soul{У нас нет ваших игр.}
\people{То есть, это несерьезное занятие? Вы, этим, не собираетесь?(заниматься. Прим.)}
\soul{Мы говорили вам, что мы похожи, но иные.}
\people{Сейчас присутствует у нас тут новый товарищ, который готовил вопросы по религии по сотворению мира, по Библии. Вот, мы хотим поговорить на эту тему. Первый вопрос, - в библии сотворение мира начинается словами: ``В начале сотворил Бог небо и Землю''. Что именно имеется в виду под словом ``Небо'' и ``Земля''?}
\soul{Вы вспомните. У вас  - рождение вселенной. Вспомните. Согласитесь, если верить вашей теория большого взрыва, то не было ничего, и `` небо'' было создано тогда, когда была создана жизнь. Вот вам и `` небо''. Далее вы спросите  - ``вода''.}
\people{Да.}
\soul{Вода, это вы. Вы. Подумайте. Даже химически, вам не надо это доказывать. }
\people{Угу.}
\soul{И подумайте, - порядок; небо, земля, вода, животные и далее - вы. Чего же далее? Здесь не отрица… }
\people{Угу…  А вот, почему было сказано - ``Земля же была безвидна и пуста. Бездонна. И, попросту, вода''. Собрание вод, назвали ``морями'' на третий день. А о какой воде идёт речь в первый день творения и во второй день творения?}
\soul{Согласитесь, что вы можете знать, только зная названия. И потому, дано было название. Вы называли ``водой''.  Может, она лучше, чем огонь? В сущности - и нет. Вам так кажется. Мы же говорим вам: каждое слово, каждый знак несёт имя. Тем более – звук. Названо ``вода''. А мы говорим, что вы из воды. Подумайте, где здесь начало? Можно ли было создать вас, и не имея воды? Где была бы только вода, и если б не было земли? Будьте логичны. Да, есть множество, в вашем понятии, планет и миров, где нет воды. Где вода является небом. Есть жизнь и там. Вода есть везде. Но вы назовите по-другому. Всмотритесь,- в небе - вода.  Даже в камне есть вода. }
\people{В библии сказано - ``дух божий носился над водою''. И сказано это на пятой строчке библии. Что хотели этим передать нам? Почему нет сообщений о передвижениях в другие дни творения?}
\soul{Вы что, хотите, чтоб вам дано было каждое мгновение? Как вы можете представить?  Без движения? Почему?  Вы можете написать, в вашем понятии, и всегда будете говорить, что вы ездили где-то, или будете рассказывать, всё-таки, что вы делали?}
\people{В моем представлении, получилось так, что непосредственно было показано творение материи. После этого происходит появление пространства, времени, (день первый), и чтобы показать пространство, было написано, что ``бог носился''. Тем самым, это показано - это пространство. Верно ли это представление?}
\soul{У вас когда-то была теория, но вы почему-то её забыли, что материя,  это ``искривление времени в пространстве''.  Так что же было первым? Материя? Подумайте. Вами же было сказано – ``небо''.}
\people{Может, это теоретически? Ну, тут у нас Эйнштейн утверждает, что…}
\soul{Небо. Поймите. Небо, для вас, это ``ничто''. Вы его не ощущаете, не видите его. Для вас, это – ничто. И это значит, что не было сперва материи. Небо, это не ``первое''. А мы говорили, что  материя была более. (более, чем мы подразумеваем.прим.)}
\people{Ну, чисто теоретически, если во Вселенной не будет материи, то, значит, не будет ни времени, ни пространства?}
\people{Дух останется только один.}
\soul{В вашем понятии, без материи не существует пространства? Хорошо. Тогда подумайте, где вы гуляете во снах? Там нет ничего материального. У вас была теория о пространстве и времени и, лишь только - материи. И вам скажут - было небо, не было материи, было только понятие о ``пространстве'' и о ``времени''. Вспомните. Далее, - ``И назвали день''. Тогда подумайте, было ли в ``начале'' ``время''?}
\people{Наверно, не было.}
\soul{Подумайте. Посмотрите с этой точки.}
\people{Раз два события отличающихся есть, значит, есть и понятие ``времени'', отрезок времени.}
\people{Да, зарождение времени. }
\people{В нашем понятии,  ``ночь'', это, всё-таки, связано с вращением Земли вокруг Солнца. А если светило было создано богом, как сказано - на четвертый день,- какой тогда отсчёт времени?}
\people{Как же без Солнца шёл отсчёт дней, и что же ``светило''?}
\soul{Тогда подумайте. Ребёнок имеет понятия о времени? Вы можете объяснить ему, ``завтра''? Когда придёт этот день? Как вам объяснить, что ``завтра'' уже наступило?}
\people{Ну, мы ему и говорим: Проснешься  утром, вот и будет ``завтра''. }
\people{Ну, трудно, конечно, ребенку втолковать. }
\people{Ну, а почему говориться в библии: ``и увидел бог, что это хорошо''? }
\people{Нет… ``Увидел, что свет — это хорошо''. }
\people{Да. Непосредственно - свет. Он и про другое там говорит.}
\people{Он не ведал, что творит? Так получается?}
\soul{Нет. Поймите, в вашей же библии сказано, вы не внимательны,-  что вы, сами боги. Вспомните.}
\people{Помню. ``По подобию''.}
\soul{Вспомните более, - ``Кто видит меня…”}
\people{”Кто видит меня, тот видит бога''.}
\people{Да. И что?}
\soul{Он пришёл к вам человеком. Человеком пришёл в вашей плоти. Далее, в вашем понятии, Христос — Бог, и забыли, что прежде всего, он - человек. И потому, всё  не верно вами.(понято. Прим.)  Далее. Вы хотите узнать начало? Мы когда-то вам говорили: сможете ли вы понять, что, будучи далеко по времени, вы вернулись и увидели, что вы не можете жить, ибо не было вас? Что вы будете делать? И вы будете создавать, для того, чтоб родиться в будущем. Мы же говорили вам, помните? Вы плодите миры.}
\people{Ну, по библии, опять же, - там, по первой строчке читать, то вспоминается, что из воды, действительно,  жизнь пошла, а вот душа непосредственно уже из земли была сотворена, якобы. Земля дала душу  живую. Почему такое распределение? В библии, как раз  - так.}
\soul{Может быть тело, ваше тело было создано из земли?}
\people{“То произведёт земля душу живую”- цитирую дословно.}
\people{Что же творило, всё-таки, вода, земля или бог?}
\soul{Тогда, пожалуйста, объясните мне, как (…) Кто его сотворил? Кто его родил? Бог? Или, всё-таки, мать?}
\people{Земля, видимо. Земля, для нас, мать.}
\soul{Нет. Мы говорим именно о матери. А где она взяла составляющие вашего ребёнка, говоря вашим языком? Поймите, задав правильный вопрос, вы уже найдёте ответ. Мы вам говорим: вы не умеете задавать их. Вы говорите: `` приведите пример – ``НЛО''.  - Неопознанный летающий объект. Это только говорит о том, что вы так и не познаете его, если будете так называть его. Мы вам говорили, в каждом слове есть смысл. Далее, вспомните начало. Вспомните.}
\people{Ну, вспомнил.}
\soul{Что вы вспомнили?}
\people{”Аз, буки, веди, глагол''.  Что там дальше… ``Добро”… }
\people{А какой смысл вы в это вкладываете?}
\soul{Вы говорите:  ``небо'', ``земля'', ``начало''. Где вы хотите найти ``начало''? В вечности?}
\people{Ну, что, библия врёт?}
\soul{Разве мы говорили вам то? }
\people{Подводите к этому выводу?}
\soul{Мы говорили, что вы пришли. Вы вспомнили? Вы. Как вы можете объяснить? Логически, это - не объяснить. На веру – вы не примите. Если же вам дать всё ``от и до'', что будет? В вашем понятии, религия – весть, - погибнет. Много ли вы верите в то, что знаете, как оно работает? Далее, - не готовы. Не готовы. Если вам дать все знания, вы потеряете внешнее. У вас множество теорий рождения мира. Если мы вам скажем, что все они верны. Все. Вы поймёте нас? }
\people{Но это не так, наверно, чтоб все они были одновременно верные?}
\soul{А вы подумайте. Подумайте. Аналогия,-  как вы можете объяснить одинаковость какой-нибудь картины? Будут разногласия.  Для кого – нравиться, для кого – нет. Противоположное, но говориться об одном. Как и говорит  теория ваша, что ``противоположности - в одном''. Вы говорите: ``расширяющаяся Вселенная'', и этим хотите убрать божественность.}
\people{Мы готовы согласиться, что это ошибочная теория, о взрыве и расширяющихся Вселенных.}
\soul{Нет. Мы говорим вам, что все ваши теории верны.}
\people{Но они настолько же верны, насколько и неточны? Одновременно.}
\soul{Конечно. Подумайте, как могут быть верны противоположности? Только в том случае, если они не точны. }
\people{Значит, искать надо что-то принципиально что-то новое?}
\soul{Далее. Одни говорят ``расширяющиеся'', другие говорят ``сжимающиеся''. Вы можете представить геометрически? Представьте. И найдите аналогию, когда одновременно эти два пункта будут исполняться. Подумайте.}
\people{Это вибрация будет, наверно.}
\soul{Нет. Давайте возьмем проще, - возьмём шар и начнём его сжимать в эллипс. Согласитесь,  что для одного наблюдателя он будет расширяться, а для другого, сжиматься.}
\people{Да. Смотря где координаты в сетке будут стоять.}
\soul{Подумайте… Мы же, говорим о Большом взрыве. Но, подумайте,- …Что было до него? Ведь не было материи, не было ни пространства, ни времени. Так что же, всё-таки, было?}
\people{"Абсолют'', наверно, был.}
\soul{Тогда, что в вашем понятии ``Абсолют'', и каким законам он подчиняется?}
\people{У нас слабое понятие о том, что было до нас. Совершенно слабое.}
\soul{Поймите, нельзя увидеть тот абсолют, нельзя увидеть то начало. Можно увидеть лишь только его действие. Нет времени - нет пространства. Нет этих понятий, и нет ваших законов. Можно сказать и наоборот, - все законы есть в нём, и все времена и пространства есть в нём. Согласитесь, чтоб разделиться на них, надо иметь их. Так кто же будет прав? Мы говорим вам, что там есть всё, и тут же скажем, что там нет ничего. И там, и там вы будете правы. Много сказано, в вашем понятии, в жизни, для того, чтоб вы научились думать. А вы берете целое и раскладываете  пазлы. Что это приносит вам? Давайте, представьте, - вы найдёте формулу любви, и что? Вы будете любить? Или загубите свою любовь?}
\people{Скорее - загубим.}
\soul{Поймите, если бог будет известен всем вам, то он окажется – не бог. Ибо вы умудряетесь близость принимать за вседозволенность. Если вам будет известно всё, вам будет легко потерять. И потому, бог будет всегда ``тайной''. Тайной, для вас. Бог - это ваши мечты. Подумайте, разве это не так? Вы перекладываете все ваши мечты в ваше понятие  о боге.  И потому, ``бог”…  было тогда, да и сейчас, - у каждого своё понятие о боге. Но вы нашли кривое единство и сказали, что бог один. И стали отрицать язычество. Хотя, все…}
1-2-3-4-5-6-7…
\soul{Хотя, остались язычниками. Вы можете нам объяснить, почему вы поклоняетесь огню?}
\people{Почему - огню?}
\soul{Разве нет?}
\people{Ну, я переспрашиваю.}
\soul{Зайдите и увидите там свечи. Зайдите в любую церковь и увидите там икону. И подобие? (найти в язычестве. Прим.)  Вспомните. У вас множество понятий о боге, множество религий. И отличных и нет. Вдумайтесь. Только в этом вы говорите о боге, об одном, - и имеете множество религий. И не одна не верна.}
\people{Ну, и всё-таки  какая из них должна быть ближе нам?}
\soul{Вспомните аналогию о картинах. Какой взгляд нужнее? Вас – пятеро. Вы смотрите одну картину. У каждого своё мнение. Так как вы хотите, чтоб я вам сказал, чья точнее? Как вы можете сказать? }
\people{Да, для каждого будет своя.}
\soul{Да. Научитесь, хотя бы отстаивать свое мнение.}
\people{Да, но оно ведь заранее не верно, потому что всего не видишь. Я часто сталкивался. Знаю, что это такое. Потом, с другой точки зрения мне доказывают, я смотрю  - да, вроде, не так.}
\soul{Мы говорили об энергетическом питании. Мы говорим вам о фанатизме. Вы же не можете найти границу между фанатизмом и верой.}
\people{Да, трудновато.}
\soul{Далее, - вы спрашивали, если вы христианин, то и будьте им, а то это будет похоже на предательство или попытку к бегству.}
\people{Да, но ведь все религии говорят об одном! Какая, в принципе, разница, кто буддист, кто христианин?  Все говорят об одном же.}
\soul{Есть разница. Вы сначала стали когда-то христианином, а потом начнёте поклоняться Будде. Так что же?  Вы Христа отставили в сторону? Он вам больше не нужен? Не предательство ли это?}
\people{Ну, извините, есть же такая заповедь  ``не сотвори кумира''.  Это, по-моему, Христос и говорил. Ну, по крайней мере, так написано.}
\soul{Тогда, в вашем понятии, любое предательство можно оправдать. ``Не сотвори кумира''. Тогда все ваши жертвы были напрасны. Вы же не видите разницы между фанатизмом и верой. Подумайте, что такое ``кумир''? Фанат!}
\people{Скажите, у меня такой вопрос: в детстве ещё, у меня внутри слышалось такое, что я не родной своим родителям. Такой голос, как со стороны. Не знаю. Может и мой. Я приглядывался,- характер, вроде от матери есть и от отца. А вот, откуда? Это не у одного меня, кстати.  Это у соседа моего тоже.}
\soul{А вы подумайте. Мать дала вам тело, но душу? Кто даёт вам душу?}
\people{А мать, что, без души живёт что ли? Она не даёт свою душу?}
\people{Ну, душа – это отдельное…Она - самостоятельное.}
\soul{Но, вы же, уже спрашивали, и мы отвечали вам, что душа решает где будет жить с кем и когда. }
\people{Да. Я помню.}
\soul{Вы, так  не смотрели?}
\people{Значит, это моя же душа сказала мне? Так получается?}
\soul{Но будьте осторожны. Вы не можете различить, что говорит вам: сознание; подсознание; душа; дух, или  может быть тело и многое, многое другое. Вы не можете знать.}
Конец записи.
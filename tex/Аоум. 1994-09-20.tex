Аоум. глава 20-09-94 г
Георгий Губин
\people{Сегодня 20-е сентября 1994-го года. Мы, конечно, извиняемся, но вы наверно знаете ситуацию, что кассета не записалась. Возможен повтор всего, чего вы сказали?}
\soul{Вы отказались.}
\people{От чего?}
\soul{Мы хотели б, конечно. Что делали вы? Вы смогли повторить? Что должны были сказать вы, вы помните? Вспомните начало, что вы должны были сказать? Вы должны были его восстанавливать. Вы помните? Далее,- вы должны были совершите ту же ошибку и быть так же наказаны. Вы отказались. Спрашивайте. }
\people{Так, значит, повтор невозможен что ль?}
\soul{Спрашивайте.}
\people{Хорошо, я тогда буду задавать те же вопросы. Вы согласны на них отвечать?}
\soul{Спрашивайте. И помните, и знайте всегда, хотя и есть понятие о времени, но нельзя обратить то, что уже было сделано, нельзя повторить. Нельзя войти дважды в одну реку. Всегда будут изменения. Всегда. Даже если вы, в вашем понятии, изобретёте машину времени и вернётесь  - всё равно будут изменения, всё равно. Хотя бы - изменения в вашем сознании. Согласитесь, что вернувшись в прошлое, вы будете другими. Не тем, кем были. Потому и  не будет никогда вам повтора. Единственное, что вы можете увидеть, то - быть посторонним наблюдателем, смотреть ``кино'' в вашем понятии. Только так. Но тогда вы не сможете влиять ни на что впрямую.}
\people{То есть, те же сны - это тоже самое ``кино''?}
\soul{Спрашивайте.}
\people{Так, хорошо. Скажите, пожалуйста, а вот, множество параллельных миров существует… Ну, а вот какой смысл - гулять по этим мирам?}
\soul{Вы хотите повториться?}
\people{Ну, да.}
\soul{Вы хотите услышать ответ в прошлом или другой?}
\people{Ну, желательно, в прошлом.}
\soul{Тогда, вы помните, как задавали вопрос?}
\people{А-а… К сожалению, нет.}
\soul{Вы задавали два. Говорили, что  будем отвечать на ваши вопросы. Вы хотите, чтобы мы сказали – ``есть смысл, и нет его''. Да, смысла нет. Но смысл в том, что вы, больше похожи на потерявшего, который ищет. Ищет без смысла. Просто ищет.  Ищет там, где удобней, в вашем понятии, где ``светлей'', ищет там, где легче. - В том нет смысла. И  мы же говорим вам - есть смысл, ибо приобретаете новое. В вашем понятии – опыт, знания. Сознание ваше познаёт более. Чувства ваши – живут! Вы поняли?}
\people{Что-то понял. Так, хорошо. Скажите, а вот, все эти параллельные миры, вы говорили – можно представить, как один живой единый организм. То есть…}
\soul{Мы не говорили, что `` можно'',  мы вам говорили, что это ``есть''!!! Это разные вещи. Подумайте.}
\people{Понятно. И что более ``высшие'',-  по нашим понятиям ``высшие'', - ну, то есть те, которые могут…более управляемые мыслью, - живут во всём этом,  одновременно, во всех этих параллелях?}
\soul{Ваша ошибка в том, что вы хотите услышать те же, те же ответы, и потому, вы путаете себя. Давайте забудем, и начните снова. Начните снова, опираясь на те знания, но не желая повторять их один к одному. Вы не сможете того сделать, ибо у вас слаба память, но вы запутаете себя, а, значит, других, и потому - задавайте как в начале. Не опирайтесь на прошлое. Ибо, это было вам, и никто не слышал более.}
\people{Постараюсь. Скажите, вы говорили, что общаетесь с ``более'', так сказать, умеющими управлять, более, чем вы. И вы не намного ``выше'' нас. Ну, вы понимаете… Просто, аналога нет. А как это всё понять, вот так вот - в комплексе, где ж  это всё, вот, этот ``организм'' находится?}
\soul{Везде. Ибо, всё, что бы вы не назвали, любая материя, любое существо, или что-то иное,  это есть одно. Это есть один живой организм. Для упрощения вы называете его ``жизнью''. Жизнь. Вы всё называете жизнью: движение камня, полёт ваших искусственных спутников. Всё. Движение ваших мыслей - жизнь. Когда-то ранее, вы знали более, что такое жизнь, но ваше сознание не могло воспринимать, ибо оно было мало, мало. Мало и сейчас, и потому, давайте просто говорить – жизнь. Не будет задумываться что это, ибо вы не сможете понять и больше запутаетесь. Далее. Вспомните,  мы говорили вам, в вашем понятии - на четвертом диалоге,- о том, что мы пронизываем все миры, и в то же время, мы говорили вам в том же диалоге,- и вспомните,- что мы не можем пронизывать полностью ваш мозг, ибо он ставит защиту. Вы помните?}
\people{Да-да-да…}
\soul{Как же вы тогда можете говорить о нас, как о ``высших''? Подумайте.}
\soul{Тогда вопрос стоит - кто же мы?}
\people{Вы? - Путники. Ищущие. Потерявшиеся. Назовите как угодно, в любом случае вы будете правы. В любом. Если не конкретно себя, значит,  других.}
\people{Так…Мы ищем это наше потерянное счастливое прошлое, или  там, счастливое будущее, так получается?}
\soul{Но вы не найдёте. Вы не найдёте прошлое, вы создадите новое будущее.}
\people{Скажите, прошлое, всё-таки, наше живо? Ну, кто-то там… Кто-то ж там остался?}
\soul{А вы подумайте. Мы говорили вам  о том, что живут и `` первые'', и ``вторые'', и ``шестые'' среди вас. Вы помните? Значит, есть те, что только родились, только начали. Неужели вы думаете, что однажды Бог создал вас по подобию и более не делал того?}
\people{Скажите ещё такое, вот, у нас смерть рассматривается, ну, сейчас, с данными ситуациями, как просто переход в параллельный мир. Значит, рождение в этом мире, есть переход из какого-то мира другого? Ну… Не другого, а….}
\soul{Давайте будем точнее, смерть или параллельный мир? Мы говорили вам, о смерти, как об очень маленьком промежутке времени, столь коротким, что его нельзя даже назвать мгновением. И вы никогда не сможете измерить саму смерть. Страшна не сама смерть, а её ожидание. Сама же смерть - столь мало… И, в то же время,  для многих – бесконечно. Многие из вас - могут остаться там, в вашем понятии -  быть мёртвым. Многие - могут проскочить и не заметить, тогда - идут в параллельный мир, тогда скажут вам: ``Нет смерти. Не видел''. Или по ошибке назовёт ``смертью'' свой параллельный мир. Многие  - придут и останутся там, на большое время, они будут знать что такое смерть, уйдя в параллельный мир. Будут знать. Для многих, это может быть и счастье, а может и нет, ибо он будет знать страх. Страх. Подумайте. Подумайте…}
1-2…
\people{Ну, в принципе, получается, как мы воспринимаем мир, такой он для нас и является. Кто воспринимает ``смертью'', для того смерть, кто раем, для того  ``рай'', так получается?}
\soul{Если воспринимаете мир вы, разве может быть какой-то другой?  Другой вы не воспримите. Мир, который вы видите - это ваш мир.  И каждый из вас  видит его по-разному. И потому, каждый из вас живёт в своём мире. В своём. Вы называете это ваши чувства, ваше сознание. Подумайте, вы все разные, и все видите по-разному, думаете -  разное. Вы согласны?}
\people{Да.}
\soul{И притом, имея и живя в разных мирах, вы живёте, всё-таки, в одном.}
\people{Хм. Согласен.}
\soul{Вот и подумайте.}
\people{Так. Ну, если такое, вдруг, случится в этом мире, в параллели в этой телесной,- люди начнут думать в одном ``направлении'', так сказать…}
\soul{Нет, в этом не будет ничего хорошего. Представьте, если все люди будут думать одинаково. Возьмите аналогию. Представьте - компьютеры ваши - одной программы. Много ли они дадут пользы? Да, они могут решить очень быстро одну задачу. А чтоб решить другую, надо перепрограммировать. Но,  кто это будет делать, если все работают по одной программе? Представьте, что это будет. В том и сила ваша, что вы разные, в том и сила ваша, что вы умеете ошибаться и признаёте ошибки свои, и ищете. И ищете…}
1-2…
\soul{Я говорил вам о передаче энергии. Вы помните?}
\people{Да.}
\soul{Решать вам, или вы будете часто прерываться, или вы можете помочь.}
\people{Так…Насчёт эфира хотел бы ещё уточнить. Скажите, эфир, это прото-вещество, из которого сделано всё, я так понял.}
\soul{Вы вспомните. Вспомните, мы говорили вам.}
\people{А-а! Мы должны вернуться к этой силе, да?}
\soul{Вы вспомните классификацию, которую давали вам. Вы помните?}
\people{Да-а… ``Астрал высших'' - следующий этап, да?}
\soul{Давайте, скажем так: Эфир, это ``двойник'' жизни. Или давайте скажем:  эфир – ``мать'' жизни, ибо эфир создал всё. Но эфир не обладает разумом, хотя и содержит всё. Имея и зная эфир, вы будете знать всё и вся. Но подумайте, имеет ли разум ``библиотека''?}
\people{Н-да…Она имеет потенциал какой-то, наверно.}
\soul{Эфир, если хотите -  это Начало. Это - знание всего, но не имеющее сознания. -”Библиотека''. Потому и дано вам сознание, что бы вы могли владеть, а не, просто, иметь. Вы же,- многие из вас,- зная много, владея многим и помня многое,- ваш мозг помнит всё, начиная с рождения, кончая вашей смертью, он помнит всё, до мельчайших подробностей, но пользуетесь ли вы тем? - Нет. То – мёртвый груз. И этот ``груз'' отягощает вас, не даёт вам, не даёт вам  делать то, что вы хотели, то, что вы мечтаете. И отсюда, в вашем понятии, и есть Фрейд.}
1-2….
\soul{Спрашивайте.}
\people{Скажите, а кто дал нам разум?}
\soul{Кто?}
\people{Ну, или ``что''. Вот, допустим, эфир, как вы сказали, из него состоит всё…}
\soul{Если мы вам скажем - Бог,- то не уверены в этом и сами.}
\people{А-а… Вон чё!}
\soul{Хотя, многие хотели бы услышать то. Мы же говорили вам, что мы тоже Ищущие. Разница только в том, что вы обладаете плотью физики, мы же - не имеем таковой. Но в нашем понятии - тоже существуют тела, наши тела. И мы тоже хотим их сбросить, и мы тоже хотим утончиться, и мы тоже хотим улететь. Подумайте. Вы говорите ``совершенный огонь''.  Вы должны достичь огня, стать огнём. Это предел? Нет. Огонь тоже имеет много составляющих. Вы согласны? }
\people{Угу.}
\soul{Если хотите, вашему ``утончению'' нет предела. Нет. И вы, достигнув вершины – ``Абсолюта'' в вашем понятии,- заметите ли то? Если вы, достигнув Абсолюта, станете ``всем'' и будете обладать всеми телами. Всеми, и тяжелыми и малыми. Мы же говорили вам, что в вашем поняти ``Абсолют'' пронизывает всё. Вы согласны?}
\people{Ну, да.}
\soul{Будьте логичны. Значит,  вы придёте к тому же уровню, если рассуждать логически. Но есть и другое суждение. И его вы получите только тогда, когда вы придёте. И тогда вы будете ВСЕМ. Тогда в ВАС  будут создаваться ``боги''. Но это далёкий путь для многих.  А есть и другие, более короткие пути, но их знает только каждый, ибо они индивидуальны. Нельзя дать общий совет. Нельзя. И много ли вы доверяете общим советам?}
\people{Скажите, ну, тогда получается, что, так сказать, более ``утончаясь'' и ``утончаясь'', приобретая всё… Не знаю…свои чувства распространяя на всё, что ли – вот так… Становясь ВСЕМ, - получается - в каждом - бог и… бесконечность?}
\soul{Поймите, трудно привести аналогию. Трудно. В вашем языке нет таких понятий. Ну, подумайте, если я начну говорить не вашим языком, много ли вы поймете не зная перевода? }
\people{Ну, понятно.}
\soul{Далее. Поймите, чем ``тоньше'' становитесь вы, тем, значит, легче вам пронизывать ``более грубые'' миры. Согласны?}
\people{Согласен.}
\soul{И тогда можно сказать, что вы владеете тем грубым. Хотя, всмотритесь!  Всмотритесь, - воздух владеет вами?  Ведь он пронизывает вас. Вы согласны?}
\people{Ну, может он не ``пронизывает'', но, по крайней мере, во мне находится. Явно.}
\soul{Теперь подумайте, воздух более тонок или нет, чем ваша плоть?}
\people{Нет. Из тех же молекул состоит.}
\soul{Вы поняли разницу между ``утончением'' и просто ``большим разбросом''?}
\people{Ну, да.}
\soul{Вот и подумайте. Многие из вас, разбросав себя по миру всему, говорят – ``Я утончился'' и не хотят признать ошибки своей. Спрашивайте.}
\people{Скажите, вот сегодня прозвучало сообщение, что где-то около 2094 года, значит,  люди будут на более, так сказать, духовном развитии. Ну.. ожидается, предполагается там… Теоретически, какая вероятность того, что это так скоро будет?}
\soul{Поймите, никогда не наступит то время, когда все будут…}
\people{Угу, понятно.}
\soul{Иначе – тогда нельзя было б говорить, что уже есть миры, которые выше вас, нельзя говорить о мирах, которые ниже вас. - ``Значит, не наступило то время!''. Вы согласны, что оно должно наступить только в 2094 году? То, значит, тогда - ``до этого'' все одинаковы и нет никаких `` миров'' иных! Будьте здесь логичны. Далее, вы, или подобные вам, могут стать великими. Относительно. Можно создавать новые звёзды, перемещаться куда угодно, владеть временем, и при этом, оставаться просто ``рабами техники'', не иметь никаких, в вашем понятии, способностей. Вы согласны?}
\people{Да.}
\soul{Вот вам и ``технократия''. Можно не обладать никакой техникой,- возьмите свой животный мир,- он не создает техники, что ж получается - он выше вас, ибо обходится без неё? Нет. Можно прожить без техники и не знать об этом, и, при этом, тоже перемещаться когда и куда угодно, и при этом - оставаться низшими. Ибо, вы - технократически, а я - биологически, -  есть в том большая разница, если уже природой дано? }
\people{Угу… Так…}
\soul{И потому, нельзя говорить о ``высших'' и о ``низшем'', если все вы можете перемещаться только техникой, они это могут делать без техники - нельзя говорить, что они духовны. Нет, ни в коем случае. Поймите, духовность - это то, что не меряется ``возможностями'' и ``способностями''. Поймите, можно оставаться ``не умея ничего'', не обладая никакой энергией и, в тоже время, быть духовным. Вспомните ваших героев. Вспомните прикованных к постелям,- много ли они имели ``возможности''? Много ли они имели энергии? Но они оставались людьми. Вы согласны?}
\people{Да.}
\soul{И были те, которым завидовали многие - и что? Были ли они духовны?}
\people{Ну, по крайней мере, люди, которые умеют что-то, более, чем обычный человек, они всё-таки трактуют очень… Ну, в таком стиле - десяти заповедей. Ну, в этом ключе.}
\soul{А теперь представьте, приходит в мир ваш дьявол и будет говорить вам: Убейте! Много ли пойдет за ним? Много ли? Если даже инстинкт ваш - воспротивится! Пойдёте ли вы? - Нет. И потому, он придёт к вам и будет говорить о десяти заповедях. Но будет говорить  искаженно и с многими ``поправками''. Подумайте, не приходит зло открыто, не приходит оно явно. И добро, бывает, приходит к вам, как зло. Подумайте. Подумайте. Мы говорили вам - нет добра, нет зла. Вы помните?}
\people{Да помню отлично.}
\soul{Давайте скажем - любое зло, будь оно принято правильно, - будет добром. Подумайте, оступившийся ребенок упал, ему больно, - зло? Зло.  Но он - учится ходить. Далее, любое добро, если оно принято неправильно, может быть злом.}
\people{Верно. Это - чаще всего.}
\soul{Вспомните. Вспомните – кто дал вам десять заповедей. Разве злом он к вам пришел? Добром! И что же?- И против него превратилось во зло. И многие, сейчас, те  ``десять заповедей'' - торгуют. Торгуют…}
1-2-3-4-5-6-7-8-9…
\soul{Спрашивайте.}
\people{Скажите, вот, если уж мы о десяти заповедях заговорили, - вот, меня всегда поражала такая вещь, по крайней мере - как написано, - я не знаю как точно было, к сожалению, - что Христос проповедовал, вроде, любовь к ближнему и не гневаться… Ну,  все отрицательные эмоции подавлять, так сказать, учиться понимать людей, - в тоже время, написано, что он разбросал у торговцев в храме торгующих, их товары, вроде как накричал на них, - там написано.}
\soul{Он ведь пришёл к вам в плоти. Он пришёл к вам человеком. Почему вы забываете о том, что многое человеческое осталось в нём? Далее. Как воздействовать на вас? Как? Если вы не слышите духовного. Мы говорили вам о том, как вы задаёте вопросы. Вы помните? Вы задаёте их плотью, и потому мы отвечаем вам плотью. Если вы задаёте духовно, мы дадим вам духовно. Или, представьте, если вы не слышите духовно ничего… Как говорить вам? -Вашей плотью, вашим языком.}
\people{Христос, вроде, проповедовал, ну, как сказать… что Бог в нём, он – в боге… Ну как-то это всё… Значит, он, в принципе, знал, что бог есть? }
\soul{Подумайте, что говорите. Подумайте. Мы объясняли вам о рождении Христа, о приходе Христа к каждому из вас. Вы помните?}
\people{Да. Вот, по этому поводу вопрос всё время хочу задать,- забываю, - в1993-м году, в прошлом, вот, осенью, - мы с переводчиком почувствовали, что что-то происходит такое, и, между прочим, не одни мы почувствовали, очень много народа, - один там  объявил себя Христом… Ну, вот, какую-то энергию  почувствовали что ли…Что это было вообще? Просто, люди открылись в этот момент более, или…как понять?}
\soul{У вас существует понятие ``рассвет'', ``день'' и ``заход''?}
\people{Да.}
\soul{Почему вы не можете распространить это на космос? Неужели вы думаете, что только на вашей планете существует понятие ``сезоны''?  Существуют везде. Везде есть ``свет'' и `` мрак''.}
\people{Угу. Вот откуда это солнце сверкнуло? День был. Где это находится?}
\soul{Это день? Вы привыкли всё мерить физически. Неужели, в вашем понятии – физически происходит ``рассвет'' в космосе? Простите, а вы физически получаете радость от рассвета или может, всё-таки, нет?}
\people{Но это откликается на физическом состоянии же? Вот, все вот эти эмоции там?}
\soul{Но если вы же говорили, что душа владеет телом  - почему же нет? Хотя, вы столь плотно спрятали свою душу, что и забыли о ней.}
\people{Но, в принципе, большого греха нет, что мы, вот, живём в таком теле, в плотной материи, так сказать?}
\soul{Мы когда-нибудь обвиняли вас в том, что вы живёте в этом теле? }
\people{Нет, но…}
\soul{Вы можете это вспомнить? }
\people{…косвенно – да.}
\soul{И что вы говорите? Если библия ваша говорит - ``и создал по подобию своему'', что ж и бог низок, что может носить тела ваши? И что, опустился он от того, что стал Христом? Почему вы привыкли мерить в плохом и хорошем. Почему вы решили, вспомните,  мы говорили вам ранее, мы говорили вам сегодня о плотях, и вы уже не помните.}
\people{Но, с другой стороны, вы говорили, что  ``кому нужна ваша физика в мире, где нет физики?''. Вот, как это понять ваше изречение?}
\soul{А вы подумайте, зачем нам ваша физика? Что мы будем делать с ней? Давайте сделаем проще, давайте вы начнёте жить, ну, хотя бы в воде. Без ничего. Вы выживете там? И нужно ли это вам? Далее. Мы говорили, что не нужна ваша физика нам, но не вам. Вы не внимательны.}
\people{Ну, может быть, да. Хотя, вроде как, вы это нам говорили.}
\soul{Вы придаёте слишком много значения своим телам и потому приходиться говорить вам об этом. Если вы живёте в этом мире, так живите в нём. Живите в нём по-человечески. Вы же - ``скорей бы прожить, а там – может и лучше будет!''.}
\people{Некоторые вообще не верят, так… живут лишь бы в свое удовольствие,- есть такая теория,- живи пока здесь, а там..!}
\soul{Каждый получит по заслугам своим, по вере своей. Каждый. Это не будет наказание. Будет ваш выбор. Вы же создаёте миры. Мысли ваши, образы ваши создают миры, в которые вы же и придёте. Если создадите мрачный мир, вы придёте в мрачный. Если же вы будете создавать светлые миры, вы будете жить в светлых. Если вы богаты и мечтаете о богатых  мирах, это не значит, что вы придёте в богатый мир, ибо у вас есть ещё страх потерять то богатство, и потому, вы придёте в мир страха. Не забывайте и о том.}
\people{А если хочется пойти в мир, так сказать, всеобщего братства, всеобщей любви, - ведь тоже, можно сказать, что есть страх потерять это всё. Тогда, тоже получается – придём в мир страха.}
\soul{Поэтому и говорят вам:``не бойтесь''. Вы спрашивали, можно ли летать? }
\people{Да.}
\soul{Мы говорили вам -  вы можете, но вы боитесь. Что мы можем сказать вам, какой дать вам рецепт, чтоб не боятся? Никакой!}
\people{Ясно.}
\soul{Ибо вы боитесь - не мы. И вы должны побороть. Не будете же вы всё время ``ходить'' под-руку? Вы согласны? Если за вас будут решать задачи, сможете ли вы решать их сами? Вы же, чаще, хотите найти готовый ответ. Потому и хотите открыть третий глаз силой.}
\people{Ясно. Так… Ну, что можно сказать… Cпасибо.}
\soul{Вспомните, когда последний раз прерывали вы. }
\people{Так.}
\soul{Можете вспомнить? Что растерялись вы?}
\people{Вспоминаю. Точно не могу припомнить.}
\soul{Что вы хотите вспомнить? Прошлое? Мы же говорили вам, не стоить жалеть.}
\people{Прошлое, может, не стоит жалеть, но…}
\soul{Подумайте. Вы говорите, что ничто не проходит бесследно, и тут же огорчаетесь, что потеряли. Почему вы противоречите себе? Ну, если вы говорите ``не бесследно'', - о чём же тогда жалеете? О чём?}
\people{Да это понятно. Может быть о том жалеем, что сделали, что-то не так, как хотели. Так получилось, где-то что-то не учли.}
\soul{И что же? Ищите, исправляйте ошибки свои. Исправляйте. Чаще, вы совершаете эти ошибки и потом, опять их повторяете, ибо забыли. Забыли, что совершили их. Потому и будут, в вашем понятии, ``наказывать'' вас. Но поймите правильно, вы – себя… }
\people{Да, я понимаю.}
Вы будете повторять эти ошибки до тех пор, пока не научитесь их исправлять. Вы же -  слепой мышонок, который ищет и не может найти. Но, поймите, если придут и откроют глаза ему - что увидит ими, и увидит ли?
\people{Ну, понятно. }
\soul{Поймите, мы же слышим, вы же читали, что не может прийти к вам высший, вы не можете увидеть высшее, ибо ``ослепнете'' и далее и далее. Вы помните?}
\people{Угу.}
\soul{Вот и подумайте про аналогию с мышонком. И вы являетесь им. Но не огорчайтесь. Нам приходиться, к сожалению, говорить вам такое - не вы один.}
\people{К сожалению.}
\soul{Да. Ибо, вы горды. Любые подсказки вы понимаете за подачки, ибо сами судите по себе, по вашей испорченности. Ибо, когда вы кому-то даёте, вы называете это подачкой ``про себя''. И всё, что будет дано вам, будете называть подачкой и вы. Если же вы будете отдавать с чистым сердцем, то всё, что будет даваться вам, вы будете тоже принимать как от чистого сердца, даже если это давал вам подонок и с плохими эмоциями. Спрашивайте.}
\people{Что ж хотел спросить такое интересное… Вы как-то сказали, о конце света. Что если мы представляем физически, физически он и придёт на Землю. То есть, будет сотрясать там…Ну, мысли наши вернутся…. Скажите, от чего зависит вот это время возврата.?}
\soul{От силы. От силы. Если хотите - от массы, от количества.}
\people{Еще вот, такой может личный вопрос… Скажите, ничего дурного в том нету, ну, допустим у меня сейчас на данном этапе затруднения с деньгами, вот. И  у меня такая мысль, эти свои знания улаживаю… в эти деньги. В этом нет ничего греховного, если я смогу почувствовать какие там выпадут номера или ещё что-нибудь?}
\soul{Поймите, грех будет не в том, что вы почувствуете, а в том, когда вы их получите и что будете делать с ними. Если обуяет вас жадность, то будет грех в том, что вы чувствовали.}
\people{Угу, спасибо. Скажите, есть такая теория, что человек вместит в себя ровно столько, сколько он способен, а как же тогда происходит прогресс?}
\soul{Вы же растёте. Почему же вы не можете рассуждать даже логически и здесь? Поймите, если б это было не так, то вы бы могли обладать многим, многим и многим. Вы же,- посмотрите вокруг,- кто-то обладает больше, кто-то меньше. Вы согласны? Ну, и тот, кто обладал меньше, - вспомните себя, - много ли вы знали, и сколько  знаете сейчас. Вы же - растёте, значит, вмещаете больше. Это грубая аналогия, но она верна.}
\people{Скажите вот такое, а люди стоящие у руля у нас, ну, у государства, у заводов… ну, начальники наши, если они добились доверия такого, значит, они…ну, простите, скажем грубо -  выше нас?}
\soul{А вы подумайте, они добились доверия вашей души или сознания? Возьмите и вспомните всех ваших правителей, ваших начальников и далее. Вспомните их сознанием и вспомните их чувствами. И найдёте ли в том разницу? Сознанием легко добиться, ибо сознание легко обмануть, а чувства… Чувства у вас всегда были на втором месте, ибо вам приходится подстраиваться под всех. Этим занимается сознание. И потому, сознание легко ``уговорить''.}
1-2-3-4-5-6.
\soul{Спрашивайте.}
\people{Так… Значит, это только сознание,-  чувства на втором месте. Скажите, а чувства если будут на первом месте, что получится?}
\soul{А мы говорили вам, пусть будут вместе. Почему вы делите на первый-второй? Возьмите и вспомните тех, кто имеет на первом месте сознание, чувства на втором или далее. Кто они? }
\people{Угу.}
А теперь, только тех, кто обладает одними только чувствами и в них мало сознания. Вы поняли, кто они?
\people{Понятно.}
\soul{Говорите о ``золотой середине''? Нет, золотая середина это не то и не другое. Давайте возьмем половину сознания, половину чувств, вот вам и ``золотая середина''. - Нет.}
\people{А где же решение проблемы?}
\soul{Они должны идти вместе. Не должно сознание обижать чувства или чувства сознание.}
\people{Трудно удержаться так.}
\soul{Трудно удержаться, если вы будете делать это сознанием, ибо уже сознание идёт впереди заведомо. И трудно удержаться, если чувства уже заведомо впереди. И чаще, вы правдивы, когда не делаете того специально - вашем понятии - случайно, ``в порыве'', ``осенило'' и далее, и далее…}
(Конец контакта)
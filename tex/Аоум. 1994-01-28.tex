\chapter{28-01-1994г}

\people{…Мы вновь  просим  вашей помощи в ответах на наши вопросы. Мы зададим вопросы по аномальным явлениям. Во время прошлого контакта, 25 января, вами без предупреждения был прерван контакт. Почему это произошло, если можно?}
\soul{Мы говорим, о переводчике.}
\people{Ему было плохо?}
\soul{Нет. Мы дали ему слишком много.}
\people{А это не связано с уровнем наших вопросов?}
\soul{И да, и нет.}
\people{Т.е.  они были неинтересны, или мало квалифицированными для вас?}
\soul{Нет, вы задаёте направление, и смысл вопроса не имеет значение.}
\people{Нам показалось, что мы вторглись в область запретных знаний. Это так?}
\soul{Нет. Мы говорили вам, что мы не дадим вам лишнего.}
\people{А… Спасибо. Стоит ли нам приглашать на сеансы контактов посторонних людей, или это мало продуктивное занятие?}
\soul{С какой целью?}
\people{Чтобы повысить наш интеллект, интеллект нашей группы. Может быть, у них более интересные вопросы,  и ваши  интересные ответы будут.}
\soul{Подумайте.}
\people{О чём?}
\soul{Вы говорите, повысить ваш интеллект? С помощью других?}
\people{Интеллект группы.}
\soul{Нет. Мы говорили вам: главное не вопрос, а его основа.}
\people{Скажите, а следует ли нам заранее, может быть, оговаривать с вами кандидатуры сеансов связи? Вы можете заранее одобрять или отвергать, в приемлемой, конечно, форме  наши кандидатуры? }
\soul{Нет, мы не будем отрицать, и не будем сопротивляться. Это - ваши проблемы.}
\people{Т.е. если мы сочтём нужным кого-то пригласить, вы не будете особо препятствовать?}
\soul{Мы не будем, но мы говорили вам, что «переводчик» слышит или не слышит.}
\people{Скажите, если принять, что молчание – это знак вашего согласие, а слово” задавайте вопросы” – отрицание, то можно в такой форме вести опрос по кандидатурам в контакт?}
\soul{Нет.}
\people{Вы утверждаете, что в мире нет ничего случайного. Поэтому - неудача с записью второй половины кассеты 25 января,  это - наша небрежность или ваше воздействие?}
\soul{Вы спрашиваете, о случайном, и говорите, о небрежности воздействия. Вы можете, в вашем понятии, ``случайно'' сделать и небрежность. Мы говорили вам: мы не влияем физически на вас.}
\people{Т.е. это отказала аппаратура? Так?}
\soul{Это ваши проблемы. Мы воздействуем не физически, – первое. И второе, – и, всё-таки, физически, нам приходится делать. Подумайте сами.}
\people{Скажите, а вы бы не могли продемонстрировать нам свои возможности, восстановив не качественные записи?}
\soul{Мы говорили вам о не физическом воздействии. Мы физически влияем только на переводчика. Точнее, мы говорим с ним на энергетическом уровне, мы пользуемся его  эмоциональным фоном. Мозг создаёт защиту – это движение рук и уход от нас.}
\people{Мы хотели  пригласить на контакт одного приверженца системы Парфирия Иванова  и знатока религии. Следует ли это сделать?}
\soul{Вы спросите его, о Боге. Спросите. Оно отрицает язычество, но говорит о троице. Почему же тогда отрицание язычества? Пусть он ответит на этот вопрос.}
\people{Т.е. он во многом нелогичен или не так  понимает…}
\soul{Мы говорили вам, о единстве мира. Религия говорит, о едином Боге. Вспомните, язычники говорят, о множестве Богов. Ваша религия в ваше время, говорит  о Троице, и говорит, о единстве этой Троицы, не понимая, что же они, всё-таки, говорят. Согласитесь, или один или много. Если говорить о трёх, то можно признать и всех остальных  Богов. Вы согласны?}
\people{Да.}
\soul{Далее, спросите: если Бог всемогущ и всевидящ, почему же он отдал власть падшему ангелу? Это его слабость? Зачем Богом было создано это? Если он ответит на эти вопросы,- пусть придёт. Но мы не дадим ему ничего. Мы дадим  ему только разочарование.}
\people{Я вот, интуитивно чувствую, что это может случиться. Спасибо большое. Ещё вот вопрос об интуиции. Меня, всё же, занимают ваши указания контактного воздействия лично на меня в 1971 и 1977 годах. Скажите, не связано ли это с находкой окаменевшего дерева в те годы? Я вспоминаю эти эпизоды.}
\soul{Мы говорили вам, что мы приходим к вам и вещими. Дело не в корнях, а какие ассоциации приносят они. Вспомните, вспомните своё детство.}
\people{А, что именно в детстве вспоминать? Мои фантазии?}
\soul{Вы не помните рождение, вы не помните первый поцелуй матери, вы не помните боль, которая научила вас дышать. Согласитесь, в вашем понятии, вы не помните. Далее, вспомните вашу первую улыбку, осознанную вами. Вспомните. Далее, вспомните, хотя бы просто дорогу в школу и сколько много видели вы, и вспомните свою первую ложь, и вспомните свою первую встречу со смертью. Согласитесь, вас тогда не интересовало то, что интересует сейчас. Вы просто спрашивали: ``а куда ушёл?'' Вы тогда не верили в смерть. Вы согласны? }
\people{Согласен. Но меня изумляет ваше знание моих… или это в общих чертах так высказано?}
\soul{Понимайте, как хотите. Далее, вы все одинаковы.}
\people{Но вот, скажите, моё путешествие в районе, Баскунчака, в район горы Большое Богдо, вот находка, может быть, некоторых камней и окаменевшего дерева, вот этим именно в те годы я увлекался лекциями Ажажи и стал задумываться, об НЛО. Это не случайны мои увлечения и задумки об НЛО?}
\soul{Хорошо. Вспомните школу и скажите: география была вашим любимым предметом?}
\people{Была. Я стал, потом, путешествовать по стране.}
\soul{Она была с начала вам любимой?}
\people{Да, нет. Скорей всего, после путешествий. Не уверен. Не знаю.}
\soul{Вспомните. Вспомните учителя, и вспомните его смену, и вспомните, о разочаровании.}
\people{Да. Виктор Сергеевич Земляков мне очень  дал, как учитель географии. Потом, следующих учителей я просто не помню.}
\soul{Возьмите аттестат, возьмите сохранившиеся дневники, и вы поймёте, как учились ранее и как учиться стали далее. Спрашивайте.}
\people{Хорошо, сейчас вопросы задаёт Гера.}
\people{Так, у меня вопрос такой,  как понять изречение из Евангелии: «пусть ваша правая рука не знает, что делает левая»? C какой стороны это понимать?}
\soul{Если вы делаете искренне, не задумываясь, что вам вернётся это.}
\people{Т.е. правая - левая – это как “отдача” и “забирание”, так сказать.}
\soul{Не понимайте всё буквально. Можно подать нищему подаяние и думать, - авось пригодиться. А можно это сделать из жалости, а можно сделать, просто, от любви.}
\people{Так, о приборе, вот, который нам давали… Я ему показывал схему… видите её сейчас?}
\soul{Вы не внимательны. Вы спорите о наименованиях. Мы говорили вам, о зарядах, мы говорили вам, о вибрациях и говорим вам, о СВЧ. Согласитесь, это тоже- вибрация.}
\people{Да, согласен.}
\soul{Вибрации, которые принимаете вы и можете назвать их. Мы говорили вам об отрицательных частотах – это тоже вибрации. Весь мир движется вибрациями. Если вам кто-то скажет, что что-то двигается прямо – не верьте. Спрашивайте.}
\people{Так, тогда вопрос такой: частота вот этой СВЧ, не скажете? Потому что диапазон великий…}
\soul{Вам сказать конкретно?}
\people{Ну, если вам не трудно…}
\soul{С какой целью? Мы приводили вам пример отца и сына. Вспомните. Если вы отец, и к вам придёт сын, спросит, как пользоваться спичками? Вы скажете ему?}
\people{В более взрослом виде, скажем.}
\soul{Нет. Маленькому вы скажете: не смей трогать. А когда он вырастет, он не будет вас спрашивать. Он сам научится.}
\people{Ладно. Спасибо. Вопрос такой, - каждый ли попадает в тот мир, в который верит? И от чего это зависит?}
\soul{Здесь вы правы – “каждому по вере его”. Если вы не верите в Бога – вы к нему не придёте. Если вы не верите в потустороннюю жизнь – вы не придёте.}
\people{Т.е. в того Бога, которого данный индивидуум считает Богом, я так понял? Для вас ведь тоже есть Боги? Кстати, раз вы укоряете, что мы не спрашивали о них…}
\soul{Мы вас не укоряем, что вы не спрашиваете, – первое. Второе ,– да, есть Бог, и он един. Но все видят его по-разному, и для них есть множество Богов. Для вас - один Бог, у  другого – другой. Согласитесь, два верующих в одного Бога, видят его по-разному.}
\people{Можно так спросить, а как вы видите своего Бога? Ну, нашего Бога.}
\soul{Мы говорили вам, что вы представляете Бога, как персону. Вы помните?}
\people{Да.}
\soul{Мы говорили вам, что у нас нет персон. Вы помните?}
\people{Да.}
\soul{У нас есть Бог - и он ваш Бог. Вспомните ваше детство, вспомните ваши мысли, вспомните ваши ощущения, вспомните, у вас ранее ощущений было более, чем мыслей, вы воспринимали всё по-другому. Мы к вам, и к вам, приходили и играли с вами. Вы не помните, но мы помним всё. Мы были с вами – вы родились, и с первым плачем, первой улыбкой, вы стали забывать нас. Потом, вы стали рассказывать всем, как вы играете с нами. Вспомните свои игры. Кто играл с вами? Вам стали говорить: “это фантазия, смышлёный мальчик”. Вы выросли, и вам стали говорить, что это ложь. И вы потеряли нас.}
\people{Да мне долго говорили, что я не такой, как все. Да и как-то я ощущал тоже, сам, что жизнь не должна быть такая, как сейчас. Но, честно говоря, я боялся темноты, очень.}
\soul{Вы боялись одиночества.}
\people{Да.}
\soul{Вы помнили, что вы были не одни, но у вас это осталось только в чувствах. Далее спросите у соседа, помнит ли он споры с братом?}
\people{Напротив соседа? Хорошо. Ну, спасибо.}
\soul{Мы говорили, о вас. Спрашивайте.}
\people{Вопрос такой, вот вы говорите, что рисуете ему («переводчику») картины, а он словами описывает.}
\soul{Мы не рисуем.}
\people{Ну, понятно… Даёте…}
\soul{Нет. Он приходит к нам. В вашем понятии. В нашем понятии - он не приходит, он, просто, видит. Мы говорили вам, что все миры в вас. Все миры пронизывают вас, но вы не умеете видеть. И будучи детьми, вы видели многое, и вы называли это фантазией. Далее, мы говорим вам: да, вы глупы. Но вспомните, сколько раз мать говорили вам это? Она говорила это с обидой, со злостью ли? Она это вам говорила любя… Спрашивайте.}
\people{В прошлый раз мы просили  контакт с планетой Вафлем. У вас был?}
\soul{Нет.}
\people{Скажите, пожалуйста, самый вероятный, по вашим вычислениям, конец нашего 2010 года.}
\soul{Вы хотите, чтобы мы решили вашу судьбу? Мы дали вам исчерпывающий ответ.}
\people{Понятно. Спасибо.}
\soul{Далее, вы, каждый, ответственны  за это. Не меряйте величиной, не меряйте должностью.}
\people{Вы говорили, что о своих ошибках относительно себя самих, а относительно нас, какова вероятность ваших ошибок?}
\soul{Вам  хватает своих ошибок. Ошибаясь, мы делаем больно себе, но не вам. Вы же, ошибаясь, уничтожаете многое. Мы должны ненавидеть вас, но как мы можем ненавидеть самих себя? Мы говорили вам, что мы – это вы. Мы – каждая ваша мысль, мы – каждая ваша клетка.}
\people{Нам трудно это постигнуть. Может быть, мы постараемся.}
\soul{Вспомните, мы говорили вам, что мы - энергия. Что движет вами, что даёт вам жизнь?}
\people{Но, не вы же?}
\soul{Мы говорили вам, об энергии. Мы говорим вам, вы – это мы, мы – это вы. Мы находимся в каждой вещи. Мы даём вам постоянно. Если вы не слышите и проходите мимо, то для вас это просто случайность, потеряно. Вы читали “Асгард” (книга “Асгард – город богов” .прим.)Вспомните крещение.}
\people{Да.}
\soul{И подумайте, и вспомните ответ. Если б было не замечаемо, то не было б крещения. Вам проще, вы были с нами, и нам не надо крестить вас. Вы просто отстали, и потеряли. Вспомните, вам говорят: будьте детьми. Мы же говорим вам: будьте детьми, но не разумом. Спрашивайте.}
\people{Дайте, пожалуйста, оценку вопросам по десятибалльной шкале, первых трёх контактов, вот лично со мной, в том году, и самую яркую из всех последующих.}
\soul{Мы говорили вам, что нас волнуют основы вопросов, а не как сказано это. Согласитесь, задавая вопрос, вы задаёте его не словесно, а эмоционально. Далее. Сопоставьте всё вместе. Мы же говорили вам: мы находимся в вашем эмоциональном фоне и всё, ваше любое слово, отдаётся. Далее, вы даёте только направление, вы можете задать любой вопрос, это ваши проблемы. Смотря на эти вопросы, вы будете сами осуждать или хвалить себя. Вы даёте только направление для нас, направление для него. Мы не влияем на вас физически - в прямом… В вашем духовном - мы даём вам сильные изменения. Далее, изменившись духовно, вы изменитесь и физически. Только такая реакция.}
\people{Нам показалось, что «переводчик», что-то получает от вас. В нём пробуждается скрытый талант, он стал писать книгу, мыслит интересными категориями. Мы правы, что это ваше воздействие?}
\soul{Мы говорили вам, что мы в каждой вещи и в каждой клеточке. И мы кричим вам, чтобы вы услышали. Вы многое не слышите или не замечаете. Нет. Мы говорили вам, о писателях. Мы говорили вам, что мир пишет вами. Вы услышали и пишете. Согласитесь, услышать нас не так-то трудно, если прислушаться. Далее. Какая разница, как вы будете свой груз? А мы даём тяжёлый груз. Мы даём груз сомнений. Как вам донести его людям? Одни пишут, другие создают картины. Какая разница? Другие просто живут, и несут свой груз и раздают его. Мы даём вам груз, но мы стараемся дать по силам вашим. Спрашивайте.}
\people{Тогда получается, никакой принципиальной разницы не будет, будет ли создана книга, так сказать ``на бумаге'' или эта книга, образно говоря, будет написана жизнью самой, нашей?}
\soul{Подумайте, как вы можете донести, как вы можете отдать ваш “груз”? Если вы не можете отдать его жизнью, отдайте ”бумагой”!}
\people{Спасибо. Представители, какого мира приходили сюда к людям, в смысле в Португалии в образе Девы Марии? Это на прошлом контакте было уже сказано.}
\soul{Мы говорили вам, что ``к вам приходят'', такой был наш ответ. Мы говорили вам, о инопланетянах. Вы говорили, о животных. Мы скажем вам, что это были не животные, это не были инопланетяне. Если я кажу - это была энергия, любящая и созданная вами,- вы поймёте?}
\people{Мы постараемся понять, теперь уже. У меня вопросы закончены. Спасибо.}
\people{Меня (Белимов) шокировало ваше знание меня с детских… Хотя, я не собирался сегодня задавать вопросы о себе, но я попробую, потому что, может быть, такого случая уже больше не представиться. Скажите, вот мой путь от путешествий по стране к изучению аномальных явлений был предопределён или, всё-таки, это случайность?}
\soul{И – да, и - нет. Вы рождаетесь, и вам дано право выбора. У вас множество путей, и вы выбрали один из них. Как вы назовёте это – ”предопределено” или нет?}
\people{Ясно. Я чувствую, что я, в данный момент, не могу уже свернуть с этого пути, хотя, мне это грозит и семейными неприятностями и непониманием сослуживцев и т.д.}
\soul{Вдумайтесь, мы хотим, чтобы вы сказали: “ Это определённость выбора.”}
\people{Я прихожу уже к тому, что это определённость выбора. Не смотря ни на какие невзгоды, которые преследуют меня.}
\soul{Чем больше вы возьмёте, тем будет труднее. Вспомните, и назовите мне человека, приносящего новое, и живущего прекрасно? Вы можете привести пример?}
\people{Да, это очень мало и редко. Чаще всего, после смерти получает признание. Мы согласны с этим.}
\soul{Да. И спросите его, что он будет делать, если Христос, в его понятии, придёт к нему? Кем он будет, учеником или палачом?}
\people{Кого ``его'' спросить?}
\soul{Вы спрашивали о религии. Вы захотели привести. Ответив на эти вопросы, он может придти. И, только тогда, когда он ответит себе правду. Прийдя с лжой, - мы принесём ему разочарование, но мы не хотим этого, мы идём помогать вам. Порой, нам приходится делать больно вам. Но согласитесь, будучи вы родителем - делаете это.}
\people{Скажите, я оказался не слишком понятливым учеником, если только в мои годы, пришёл только к более-менее осознанному пониманию аномальных явлений и окружающего мира?}
\soul{Мы говорили вам, что мы не говорим о ``выше'' и ``ниже''. Вспомните ваши мечты, вспомните ваши фантазии. Я хочу вас спросить: вы мечтали быть космонавтом?}
\people{Ну… в период, когда полетел Гагарин, может быть, но не принял …(усилий. прим.)}
\soul{Нет. Вспомните.}
\people{Да, скорее всего, так и есть. Я понял, что это недостижимая мечта.}
\soul{Нет. Вы не видели смысла.}
\people{Ну, может быть и так. Скажите, я скептик по натуре и ничего не могу с собой сделать. Это мой недостаток или неизбежность, с которой мне и вам приходится мириться?}
\soul{Поймите, нельзя назвать что-то или достатком или недостатком. Всё приносит пользу или вред.}
\people{Но мне претит бездумная вера. Это, действительно, создаёт…}
\soul{Мы говорили вам, что если мы назовём вам, что это недостаток, мы упустим многие достоинства, полученные этим. Вы согласны? Веря огульно, вы станете фанатом, и тогда, вы будете принимать на веру всё,  и это не принесёт пользы. Далее, вы когда-то, отказались от всего этого и теперь хотите вернуть всё. Вы поняли?}
\people{Нет. От чего - всё? Я, вроде, иду достаточно трудным и длинным путём к моим исследованиям.}
\soul{Вы говорите о снах. Вспомните ваши детские сны. Сны, которые вы видите сейчас – навеяны. Чем навеяны детские сны? Подумайте. Сны сейчас, навеяны прожитой вами жизнью.  Какую жизнь прожил ребёнок? Вы поняли?}
\people{Нет, я не очень, поскольку я сны почти не вижу сейчас, а в детстве были странные сны, фантастические. Я - в каких-то пещерах. Это, наверно, прошлая моя жизнь.}
\soul{Мы говорили вам « вспомните».}  (пошёл счёт)
\soul{Вспомните, что принесла вам первая смерть. Вы должны помнить.}
\people{Помню. Большое горе, жалость.}
\soul{У вас был страх?}
\people{Наверно, был страх.}
\soul{Вы всё не могли понять…}  (запись прерывается)
\soul{…которую причинили вы. Помните?}
\people{Я не помню, кому я причинил боль, именно в детском возрасте.}
\soul{Вспомните. Первый камень и первую боль. И у вас была радость, что вы попали, лишь только потом пришла жалость. Ничто не проходит, всё остаётся. Спрашивайте.}
\people{Можно ли считать, что я не случайно стал руководителем волжской группы по изучению аномальных явлений. Но, насколько удачна моя кандидатура для такой миссии?}
\soul{Мы говорили вам, что от каждого зависит. И мы говорили вам о посте, который вы занимаете. Делайте своё дело. Мы скажем: каждый отвечает за конец или рождение света.}
\people{Всё-таки, я чувствую себя более репортёром, чем по-настоящему глубоким исследователем. Моим учителям и наставникам приходится мириться с этим?}
\soul{Мы не меряемся с вами. Мы не унижаем вас. Да, порой, мы говорим вам унизительные слова, но мы говорим вам это не от обиды. Далее. Мы не меряем - выше и ниже. Мы говорим вам, о направлении. И скажем: даже в этой жизни вы изменитесь. Спрашивайте.}
\people{Это будет трагическое изменение или, всё-таки, положительное для меня?}
\soul{Подумайте, если мы вам скажем “то-то” и” то-то” – у вас есть два варианта: сидеть безвольно и ждать когда это случится. Тогда, мы будем лгунами. Согласны? Если мы вам скажем, будет “то-то” и “то-то”,- у вас есть другой путь – стремиться к этому. Но тогда это будет наша подсказка, и мы будем вести вас, и вы теряете инициативу. Вы пришли сюда, чтобы выбрать самим.}
\people{А чем бы вы могли помочь мне в углублении моих возможностей исследователя?}
\soul{Мы говорили вам о духовных изменениях. Далее. Вы больше похожи на книгу, которая несёт в себе знания, но ничего не знает о них сама. Согласитесь, и в этом - тоже есть польза.}
\people{Я согласен с вашими определениями.}
\soul{Спрашивайте.}
\people{Скажите, можно ли ожидать, что когда-либо в моём окружении друзей появятся способные генераторы идей и целеустремлённые исследователи аномальных явлений?}
\soul{Вы сами были когда-то «генератором идей». Мы не зря напоминаем вам о детстве. Вы вспомните, у вас было множество идей, и вспомните, как они гибли.}  
\people{Наверно я реально осознавал, что они не осуществляться, и я отступал. Видимо, так?}
\soul{Не только. Чаще, вас бросало со стороны в сторону,- сегодня одно, завтра другое. Вы были мальчишкой, и это простительно.}
\people{В наших сеансах мы выяснили вот, что: вы рассказывали о существовании в иных мирах, измерениях наших двойников и даже тройников и т.д. Нам трудно это понять. Неужели есть копии Белимова в иных мирах?}
\soul{Мы говорили вам о “ветвях”. Мы говорили вам, о вероятности. Каждое мгновение создаётся новая цепь, новая ветвь. Вы можете сказать одно слово, или другое, и это даст разные жизни. В вашем понятии - параллельные миры. Многие ветви тупиковы.}
\people{А скажите, двойники имеют физический облик только на Земле или ещё где-то?}
\soul{Поймите, мы говорили о вероятности: может или не может. Да даже в вашем понятии, множество Земель, множество вас, физически, в плоти, и, в той же самой. Но, вы сейчас встанете и пойдёте - в одной реальности, в другом мире вы будете сидеть, и слушать, в третьем - вы будете читать. Всё зависит, от каждого мгновения. Каждое мгновение создаёт множество миров, и мы тоже их создаём. Согласитесь, сказав одно слово или другое, вы меняете вопросы. Согласны? И представьте, в этом мире вы создали один вопрос, в другом вы создали другой, мы говорили вам о множестве контактов.}
\people{Вы говорили, что в других мирах, мой двойник уже, может быть, создал книги какие-то, в этом только предстоит. Есть более удачные варианты моих двойников?}  (пошёл счёт)
\people{И других мирах мой двойник занимается тем же, чем и я?}
\soul{Да.}
\people{Он только ищет ответы на загадки аномальных явлений?}
\soul{Вы невнимательны. Мы говорили вам и не раз, что есть вы, есть ваши труды, они различны или темой или только словами. Вспомните, мы говорим вам уже это третий раз!}
\people{Это как раз очень трудно осознать. Хорошо, более приземлённый вопрос: у него, моего двойника, и в других мирах неурядицы в семейной жизни?}
\soul{Подумайте, мы говорили о множестве вас в других мирах, о множестве реальностей. Конечно, есть множество вас неудачников, и есть миры, где вы удачливы. И есть те миры, где вы уже гораздо выше, и есть те миры, где вы уже Боги и вы обладаете силой и создаёте свои миры. Мы говорим вам: вы должны все соединиться. Все вы должны создавать одну реаль. Есть понятие о нирване. Вы понимаете это, чаще, как просто смерть, больше не рождаться. Мы вам скажем своё понятие: все эти миры должны соединиться и создать мощный ствол, стать вам единым. Вы должны слиться с множеством Белимовых, с множеством удачников и неудачников и с Богом. И тогда, вы пойдёте выше. Но это произойдёт не скоро.}
\people{Да, я чувствую, что это не скоро. А сейчас мне даются семейные трудности и проблемы – это разве не помеха в устремлениях моих на моём пути?}
\soul{Вы называете это – трудности. Теперь представьте, что у вас их нет. И вы скажете, что вы будете работать продуктивней?}
\people{Может быть и нет.}
\soul{Может быть? Вы сомневаетесь даже в этом.}
\people{Не знаю… Не могу оценить это.}
\soul{Мы говорили вам, о всех, несущих новое, об их проблеме. Вы повторяетесь. Спрашивайте.}
\people{Хорошо. Не был бы ли путь Белимова более продуктивным, если бы ему оказывалась поддержка среди родных, а не помехи и отрицание?}
\soul{И да, и нет.}
\people{Т.е. преодолевая, я ищу какие-то доказательства моей правоты, да?}
\soul{Вам приходится делать и это (прервалось). Вы вырабатываете себе, в вашем понятии, терпение. Вы более становитесь упрямей, и даже в вашем понятии, это приносит пользу. Далее, да, вам приходится сомневаться, но согласитесь, сомнение ищет истины.}
\people{Скажите, а будет ли прогресс творчества Белимова на Земле? Что бы вы посоветовали для этого прогресса?}
\soul{Мы уже говорили вам. Если мы скажем то-то и то-то, что вы будете делать? Далее. Мы говорили вам, что здесь вы изменитесь, в этой жизни. Мы достаточно много сказали вам.}
\people{Хорошо, а я буду ждать, и стремиться к этому. Вот интересный вопрос: знакомые экстрасенсы утверждали, что мы с женой связаны кармически и всегда находимся в состоянии противоборства. Как будто бы, это началось тогда, когда я, будучи птицей, съел её, воплощённую в мышь. Это чепуха или действительно - реальность?}
\soul{В прямом смысле, в вашем понятии – чепуха. Вы не были ни птицей и ни мышью, в прямом смысле. Поймите, многое, вы придаёте картинке сами. Вы видите только то, что позволяет ваш мозг. Мозг позволяет увидеть вас мышкой или птичкой, но это не значит, что вы были ими физически. Далее, мы работаем с «переводчиком» и, в отличие от многих контактов, мы даём ему картинки, которые не  воспринимает мозг. Мы не рисуем картинки те, которые рисует мозг, и потому-то он ставит защиту. Он хочет увидеть знакомое и видит иное, видит настоящее, а не нарисованное вами, и потому, пугается и создаёт защиту. Вы поняли?}
\people{Да, но как сберечь нашего «переводчика» от перенапряжения или плохих эмоций…}
\soul{Вы не внимательны, вспомните, на одном из контактов, мы говорили вам, что если мы почувствуем, то мы не придём. Далее, нам придётся сделать перерыв. В вашем понятии, мы уйдём, но ненадолго.}
\people{Это будет в ближайшее время? Нам стоит продолжать? Или мы сами почувствуем, когда период начнётся?}
\soul{Вы не помните, что мы говорили в прошлом. Мы повторим: когда вы ляжете, вы поймёте, “придём” мы или “не придём”.}
\people{Понятно. Действительно ли, что дата рождения даёт возможность узнать, какое это по счёту воплощение на Земле, данного человека? Наверно это распространяется только на последние 2 тысячелетия, потому, что мне кажется, мы воплощаемся множество раз, а не до девяти. }
\soul{Вы правы, мы говорили вам уже, на примере 68-го года(контакт 19-01-94г.прим.) и скажем далее: первые две цифры дают общие, в вашем понятии - о человечестве. Последние две цифры говорят о человеке. Но, не понимайте впрямую, не говорите о 6-й или 9-й реинкарнации. Если вы так говорите, то подумайте, мы вам говорим о вечности, а что же было до этих 9-ти? Вы не логичны здесь. Далее, по датам и по имени вы можете узнать многое, но мы не будем вам говорить, потому что вами уже многое писано и прочитано.}
\people{Хорошо, верно ли что по Сидику Афгану (афганский математик Мохаммед Сидик Афган), что даже по номеру паспорта можно узнать судьбу человека, и паспортные данные отнюдь не случайны для каждого индивидуума?}
\soul{Я кажу вам так: сейчас вы имеете одну национальность, в следующей (жизни. прим.) - другую. И, как правило, вы будете всеми.(то есть, всеми национальностями. прим.)}
\people{Угу. А вот где паспортов нет, значит, человек не кодируется, как считает Сидик Афган?}
\soul{Разве?}
\people{А что у них вместо паспортов?}
\soul{Вы подумайте, что вы говорите. Вы говорите о единстве в мире, и теперь говорите: “ если человек не имеет паспорта, он не кодируется”. Согласитесь, что это напоминает одну из фраз в ваших книгах.}
\people{Какую фразу? Неужели вы это знаете?}
\soul{Мы знаем, что знает переводчик. Далее, мы знаем, что знаете вы. Далее, мы знаем, что знают многие. Мы находимся в мире эмоций, мы говорили вам, и мы не называем имён, чтобы не стать вершителем судеб. Далее, мы хотим говорить со всеми, а не конкретно с кем-то. Согласитесь, здесь вы несёте одну фамилию, в другом мире - вы несёте другую. Если мы будем называть вас по фамилии, то мы не дадим услышать другим. Вы поняли? Если что-то личное надо сказать вам и только вам, то только тогда мы назовём ваше имя.}
\people{Скажите, а можно рассказать о реинкарнациях, вот меня лично, начиная, может быть, с растений? Если вкратце, это можно? Я был когда-то растением?}
\soul{Я говорил вам, что мы были с вами. Далее, вспомните, мы говорили об иерархии: сила ; пра-материя , материя и далее, далее, далее. Вы помните? Мы говорили: человек – животное. Если мы пойдём выше, мы скажем: человек – природа. Далее, природа - астралы, в вашем понятии ``высших''.}
\people{Скажите, по одному из гороскопов, я, в последний раз, рождался в Японии и был странником. Это - так?}
\comment{(пошёл счёт)}
\soul{Мы говорили вам, о множестве реалей. В одной реалии, вы были и японцем, в других - китайцем. Мы говорили вам, что вы были и будете всеми. Вы будет нести все национальности. И мы говорили вам, что вас - множество. Согласитесь, имея множество, есть множество вариантов. В одной жизни вы были монахом, в другой – человеком, гонящим монахов. В другой жизни, вы были проповедником,  и в другой – карающим его. Мы говорили вам, о полярности. Вы не правильно поняли. Вы поняли: добро – в следующей жизни «злой». Нет! Мы говорили только, что вам вернётся, что делали вы.}
\people{Вернётся то, неправильное, да?}
\soul{Всё, что сделали вы, вернётся вам. Вы если в прошлой жизни кого-то обидели, то та же обида вернётся к вам. Далее. Если… Дайте счёт.}
\comment{(пошёл счёт)}
\people{Нам кажется, что любую книгу, если думать о книге, укрепили бы сведения о воплощениях любого человека. Можно ли с вами договориться когда-нибудь, чтобы вы подробней рассказали о чьей-то судьбе из нашей «четвёрки» или «тройки» участников этих контактов? Но не сейчас, а в другое время.}
\soul{Почему же, мы можем сделать и сейчас. Но мы говорили вам, вы не помните?}
\people{О чём именно?}
\soul{Хорошо, спросите соседа, что помнит он.}
\people{Т.е. не стоит такие вопросы задавать?}
\soul{Нет, мы пока просим вас, чтобы вы спросили соседа.}
\people{Ну, как твоё мнение? (спрашивает другого участника, он что-то отвечает). Он уверен, что это принесёт…}
\soul{Вы не внимательны. Мы сказали, что можем ответить и сейчас. Вы просили один из примеров, мы согласны дать, и мы спрашиваем: что помнит сосед? Мы не спрашиваем вашего согласия. Вы уже дали его.}
\people{Я помню 14-й и 1058-й }
\soul{Что вы помните?}
\people{Что, мы были цареубийцами или хотели царя убить… }
\soul{Вспомните. Вспомните и не откидывайте ваши фантазии. Мы говорили вам, что вы не умеете фантазировать, и вся ваша фантазия – это одна из реалий. Вспомните.}
\people{Да. Я, кажется, понимаю, на что вы намекаете. Но тут не будет подобных фокусов… я надеюсь. }
\soul{Мы говорили вам о долгах. Далее. В вашем понятии, мы говорили вам, что вы “расплатились”. Далее. Вы называете даты… Вы называете даты - считаете ли их ключевыми?}
\people{Ну, да, они считают…}
\soul{Я спрашиваю вас, как считаете вы? Что вы помните о 14-м годе?}
\people{Сказать?}
\soul{Говорите.}
\people{Значит, четверо друзей писали книгу, были такие же контакты, и, я так понял, с контакта тоже книга была. Вот, потом, значит, они не сошлись во мнениях по вопросу о царе, и…}
\soul{Теперь подумайте, они пишут о духовном. И люди, пишущие о духовном, -  хотят убить царя? Вы можете понять это? Подумайте, могут ли люди, пишущие, о духовном, пойти убивать? Ответьте мне пока на этот вопрос.}
\people{Да, это нонсенс. Тогда, они либо лгут сами себе, либо заблуждаются.}
\soul{Мы даём вам подсказку. Мы давали, вспомните, да, книга писалась, но вы должны и помнить о человеке убившем, о человеке сжёгшем (книгу. прим.). Вы помните?}
\people{Это одно и то же лицо, кто сжёг и убил?}
\soul{Согласитесь, человек, пишущий о духовном, не будет убивать. Иначе, он предал бы себя. Мы не говорили о предательстве. Вы поняли?}
\people{Не очень.}
\soul{Вспомните всё. Но мы не можем сказать вам конкретно. Мы хотим, чтобы вы вспомнили сами. Если не можете вспомнить, тогда, догадайтесь.}
\people{О чём?}
\soul{Что должны сделать вы сейчас? Мы говорили вам, а вы спрашивали, о кармических связях. Мы говорили вам, и вы поняли, что вы связаны, и  более. Вы помните?}
\people{Да, помним.}
\soul{Вы должны сделать вывод. Поймите, вы умеете убивать не только физически, но и духовно. Далее, мы говорили вам, что в этой жизни вы будете помогать друг другу, в вашем понятии, намёками. Вы поняли?  Согласитесь, - ”убить” или “намёк” – разные вещи. Вы поняли? }
\people{Вещи-то разные, но суть одна, по-моему.}
\soul{Мы хотим сказать, что вы не будете убийцей, ни в каком смысле.}  
\people{Можно ли задать вопрос о нашей догадке? Кем для нас, и для меня лично, Соколов Олег?(Первый, кто вывел переводчика в контакт. прим.)}
\soul{Он никем не был ни для вас, ни для вас. Я достаточно многое сказал вам.}
\people{Т.е. он – не человек четвёрки? Хорошо. Спасибо. Поймёт ли супруга «переводчика» значимость наших контактов или это бесполезно?}
\soul{Это будет долгий путь.}
\people{Ему это будет грозить трудностями?}
\soul{Спросите у него. Он скажет вам лучше.}
\people{Хорошо. Нам кажется, что у переводчика есть писательский талант, но он не востребован. Ему следует обратить особое внимание на воплощение?}
\soul{Мы только что говорили о книгах. И мы говорили, что были книги писаны. И мы скажем вам, что вы были одним из авторов, и мы дали вам понять, кто из вас был поджигателем. Спрашивайте.}
\people{Можно так понять, что у переводчика получится книга, если он  будет этим старательно заниматься, так?}
\soul{Да, книга будет написана, но - вероятность. Далее. Сейчас  не принято сжигать. Далее, виновный, должен помогать более, и вы будете здесь не последним.}
\people{Я буду тоже ему помогать?}
\soul{Далее, мы говорили вам, о писателях. Мы приводили вам пример, кто пишет вами - вы слышите и пишете. Пишете своими словами и пишете нашими эмоциями, и пишете нашими мыслями. Всё зависит от того, как вы слышите.}
\people{Т.е. это следует развивать в нас?}
\soul{У вас, никогда это и не исчезало.}
\people{А что сейчас мешает его писательскому мастерству?}
\soul{Что мешает вам?}
\people{Страх, неуверенность в своих силах.}
\soul{Во многое, что вы пишете, вы не верите сами. Согласитесь, уверовав, вы напишете по-другому.}
\people{А будут ли сейчас удачны мои мысли, в нынешнем воплощении, или они будут столь же малоуспешны?}
\soul{Мы говорили вам, что даже в этой жизни вы изменитесь. Почему нам приходится повторяться?!}
\people{Ну, извините… Будем ждать…}
\people{Минуточку. У меня такой вопрос: а вот изменится он в этой жизни в какой период, длительный или короткий?}
\soul{Мы говорили вам: если скажем то-то и то-то… Вы помните? (то это обязательно случиться. прим.)}
\people{Да. Т.е. надо набраться терпения…}
\soul{Поймите, еcли вас хотят научить ”ходить”, согласитесь, что если вас всё время “вести под руку”, вы никогда не научитесь. Далее, в вашей жизни множество аналогов, которые можно привести на любые ваши вопросы, вы просто не видите. Далее, мы говорим вам не более, чем вы знаете, но ваша память… Спрашивайте.}
\people{Да, память у нас, в данном случае, помехой, просто, служит.}
\soul{Согласитесь, вы знаете очень многое, очень многое! Но мы не можем вам взять и открыть. Вы должны сделать это сами. То, что вы создадите сами, болью, переживаниями, будет ценнее вам, чем, если мы просто принесём вам. Спрашивайте.}
\people{Скажите, что вы посоветуете мне предпринять для духовного развития и совершенствования, как творческой личности?}
\soul{Вы идёте к этому, согласитесь, в вас есть уже изменения. Вспомните ваши ощущения и вспомните и сравните, как вы думали три года назад и сейчас.}
\comment{(прерывается запись)}
\people{Скажите, на Земле находили осколки, в частности на реке Вакша, неземных металлов. Значит ли это, что нас посещают цивилизации,  подобные нашей? Когда и что произошло над Вакшей? Вам это известно?}
\soul{Я скажу вам так: даже в ваших недрах есть множество неизвестных вам металлов. Далее, многие приходят к вам за ними. Вы же - не знаете их цену.}
\people{А в каких годах произошло крушение аппарата над Вакшей?}
\soul{Это ваши проблемы.}
\people{Это не связано с тунгусским метеоритом?}
\soul{Нет. Что вы понимаете под понятием «тунгусский метеорит»? Вы ищете комету? Вы ищете ракету?}
\people{Там много версий. А какая версия ваша?}
\soul{Вы назовите вашу, более близкую вам.}
\people{НЛО  }
\soul{Всего лишь?}  
\people{Ну, мне нравиться гипотеза об энергофоре, что это плазменный сгусток, который оторвался от Солнца и пошёл на Землю…}
\soul{Мы когда-то говорили вам, что для многих, ваша Земля является Солнцем. Почему же Земля туда не может войти?}
\people{Ну, эти исследования ведутся более 35 лет и толку нет. Т.е. тунгусский метеорит…}
\soul{Поймите, ваша ошибка в самом выражении. Вы говорите: «тунгусский метеорит», и одно это тормозит вас.}
\people{Ну, тунгусский эффект, катастрофа, скажем так… это порождение Земли?}
\soul{Что вы понимаете под понятием ``катастрофа"? Обязательно, что-то плохое для вас?}
\people{Я сказал - эффект. Допустим…}
\soul{Это - порождение Земли.}
\people{Порождение Земли?  Там много высвободилось энергии, сравнимо с энергией 200 Хиросим… Но, почему наблюдался полёт?}
\soul{Подумайте,- свинец,- что он для вас? Он является защитой от ядерных лучей. Вы согласны?}
\people{Да.}
\soul{Но есть вещи тяжелее свинца, и вы знаете их. Но вы не можете соединить это понятие и не можете подумать и решить: уран тоже может являться защитой от самого себя. Вы же нашли ответ. Вы нашли - чем тяжелее, тем лучше защита.}
\people{Скажите, а можете ли вы лично вступать с теми физическими цивилизациями в контакт?}
\soul{Мы, физически - не вступаем ни с кем. Как мы можем вступать физически, если мы не имеем физики?}
\people{Есть у вас более развитые партнёры-цивилизации?}
\soul{Подумайте, как мы ответим вам?}
\people{Так прямо и ответите.}
\soul{Унизив вас?}
\people{Ну, мы не слишком горды…}
\soul{Вы ошибаетесь. Вы горды и чрезмерно. И потому рисуете все картины относительно вашей гордости. Вы решили, что весь мир похож должен быть, на вас. И потому делаете спутники, собираете технику и в этой технике хотите увидеть себе подобных. И, тогда, вы не заметите другие формы жизни, ибо вы будете решать – ”это не жизнь, жизнь – это мы”.}
\people{Но мы уже приходим к менее жёстким меркам, что…}
\soul{И вы пришли к полевым формам?}
\people{Ну, взгляд пришёл. Полевая форма жизни тоже…}
\soul{И что вы видите этим взглядом?}
\people{По крайней мере,  мы объясняем некоторые…}
\soul{Вы объясняете это физически. Вы не умеете объяснять духовно, вы объясняете это физически, объясняете в физическом плане.}
\people{Так, вот такой вопрос: вы сказали, что мы, давно, были сверх-цивилизацией, и, значит, взрыв произошёл какой-то, или катастрофа, и мы - осколок этой цивилизации, в теле людей? Ну, или вошли в плоть. Ну,  вот такая теория. Скажите, те же инопланетяне - они осколок этой же цивилизации?}
\soul{Нет.}
\people{Нет? У них была своя?}
\soul{Вас было множество. Вас даже было множество в те времена, как и сейчас. Вы были более на высоком уровне, только и всего. Далее, подумайте, посудите логически, если вы достигли вершины совершенства, разве смогли бы упасть в такую глубину?}
\people{Т.е. не достигли…  Я так понял, этой вершины достичь нельзя. Потому, что это и есть -  бесконечность.}
\soul{Это вы подумайте, как я отвечу. Рассудите логически, если я скажу: нельзя,-  что вы будете делать? Вы скажете: «Зачем мне это нужно? Это недоступно. Всё равно не получится». И не будете делать этого. Если я вам скажу, и лично вам, – можно. Что вы будете делать? Вы скажете: `` У меня множество времени, у меня множество реинкарнаций - в вашем понятии. Когда-нибудь, я приду к этому''.}
\people{У вас такая интересная точка зрения…}
\soul{Нет, это ваша точка зрения, это лично ваша точка.}
\people{Мы так поняли, вы главным образом привязаны и сосуществуете с людьми, с землянами. А другие цивилизации в смысле энергетического питания и общения, вам не доступны?}
\soul{Поймите, мы пронизываем все миры, и они… и они не связаны с вами или с кем-то другим. Мы - просто живём. Живём, во всех мирах.}
\people{Сразу одновременно?}
\soul{Вы живёте, тоже одновременно, во всех мирах, но не замечаете этого и не видите.}
\people{А может ли это быть прогрессом для нас, если мы это увидим?}
\soul{Мы вам три минуты назад отвечали. Мы говорили вам, о вершине. Подумайте.}
\people{Скажем, будет ли прогрессом или регрессом, когда мы будем больше…}
\soul{Хорошо, какая разница, если мы вам говорим о вершине, и если я вам говорю сейчас, а будет ли у вас прогресс? Согласитесь, вопрос тот же, разные слова.}
\people{Хорошо, всё ли на Земле доступно вашему пониманию или вы тоже ограничены в знаниях?}
\soul{Мы не достигли совершенства.}
\people{Если вы понимаете мысли животных, скажите причину, по которой киты, порой стаями, выбрасываются на сушу.}
\soul{Хорошо, тогда вы мне объясните причину, когда вы идёте массово на самоубийство.}
\people{Нас чрезвычайные обстоятельства могут к этому побудить. Значит, у китов тоже создаются такие?}
\soul{Объясните мне, какая чрезвычайность обстоятельств заставляла вас это делать?}
\people{Ну, вера, допустим, церковные догмы самосожжения… Причин много было.}
\soul{И вы называете это ``стадностью''. Почему вы ставите себя ниже, чем вы есть?}
\people{Вы нам говорите, мы горды…}
\soul{Да, вы горды и потому унижаете себя, чтобы возвысить потом. Любой нормальный поступок, который должен быть совершён, вы называете геройством, хотя вы должны были так жить. Вы же называете это геройством, унизив себя.}
\people{В сущности, мы сознаём наше несовершенство, но про китов нам не ясно.}
\soul{Чем они хуже вас или лучше?}
\people{Ну, может быть, это мы их побудили, создали экологию, невыносимую для жизни…}
\soul{Ещё я вам скажу, что, в вашем понятии, «физические», «полевые» и далее - влияют на вас и создают из вас ``стадо'', - я сказал вам ответ? То же влияет и на китов. Вы не лучше и не хуже их, и вы, часто, поддаётесь эмоциям, пришедшим извне. На вас воздействует весь мир, на вас воздействуют поля и «плохие» и «хорошие», в вашем понятии. И они рождают эмоции. Рождаете не вы, рождает ``внешнее'', согласитесь.}
\people{Скажите, где наша свобода вообще что-то делать?}
\soul{Вы должны получить эту свободу, вы должны заработать её. Да, вы в оковах тела, здесь вы правы, но вы в оковах только тогда, когда не боретесь с ним, когда подчиняетесь ему.}
\people{Скажите, а кто сейчас из представителей животного мира особенно нуждается в защите и помощи, но не может нам об этом просигнализировать?}
\soul{Нуждаются все. И вы нуждаетесь более.}
\people{А кто из мира птиц и животных сейчас стоит на грани вымирания? Вы можете назвать?}
\soul{Могу сказать, что ВЫ стоите на грани вымирания.}
\people{Ну, это так безнадёжно? Мы, всё-таки, надеемся, что…}
\soul{Если б было безнадёжно, мы б не разговаривали с вами.}
\people{Надежда умирает последней. Понятно.}
\soul{Нет. Надежда не умирает никогда, умирает только тело. Дайте счёт.}
\comment{(пошёл счёт)}
\people{Нам нравится жизнь, устройства муравьёв, пчёл – это вполне организованная разумом популяция, или такова сила инстинктов?}
\soul{Что, в вашем понятии, инстинкт? Он для вас так мрачен и так пугает вас. Вы приняли: инстинкт – это плохое. Что же - природа создала ”плохое”? Может быть разум ваш плох? Вы слишком много доверяете ему. Вы научились губить свои чувства. Вы делаете то, что полезно вам, но не другим. Если чувство подсказывает другое, вы говорите: болит сердце и принимаете “пилюлю”. Я имел в виду, “пилюлю”  не химическую. Спрашивайте.}
\people{У вас прозвучало однажды, что вас наказывают за, якобы, некорректные контакты с людьми. Кто выполняет такие функции? }
\soul{У нас нет, в вашем понятии, иерархии. Наказываемся сами, ибо вы создаёте  другие мысли, создаёте другие эмоции, создаёте другие планы, которые вредны нам. Подумайте, мы не бессмертны.}
\people{Не бессмертны - в вашем понятии?}
\soul{В вашем понятии. У нас тоже есть понятие “смерть”. Подумайте, если мы говорили вам, что мы живём мгновенно, значит, у нас тоже есть смерть. Подумайте.}
\people{Трудновато понять.}
\soul{Разве? Вы не можете ответить логически? Представьте, мы – это вы. Вы живёте 70, мы живём “доли”. В чём разница? Вы умираете, умираем и мы, но мы рождаемся и помним. Помним всё, что было, и потому, мы разговариваем с вами. Согласитесь, если мы живём мгновение, столь короткое, что у вас нет времени измерить его, как мы тогда разговариваем с вами? Мы разговариваем с вами только потому, потому что помним всё, что прожили. Спрашивайте.}
\people{Вы, конечно, счастливые в этом.}
\soul{Разве? Нет в этом счастья.}
\people{Но, вот, скажите: мы особенно много получаем от вас сведений новых для нас, но тогда кто дозирует крупицы вашей осведомлённости, если вы говорите, что у вас иерархии не существует?}
\soul{Ваши вопросы, ваши эмоции. Мы ответили вам только что.}
\people{Наши эмоции? Как можно больше постигнуть от вас. А полу-ответы, нас, конечно… эмоционально ими подбодряемся.}
\soul{У вас больше жадности, у вас понятие узнать всё основано на жадности, жадности знаний. Далее, мы говорили, и точнее сказали вы: “на знании «переводчика»”. Но, мы же говорили вам, что вы  и «переводчик» знаете всё. Причём тут знания? Мы говорили только его языком.}
\people{Ну, вот, у нас есть ощущение, что мы разговариваем с переводчиком, который начитался много фантастической литературы. И тогда, процесс контактов может быть не интересный при общении.}  
\soul{Это ваши проблемы. Далее, поймите, мы пришли к вам рождать сомнения.}
\people{Скажите, можно ли подсказать нам зону повышенной активности посещений НЛО в Нижнем Поволжье?}
\soul{Нас не интересует это. Мы говорили вам, что инопланетяне, в вашем понятии, не мудрее вас. Мы говорим о духовном.}
\people{Скажите, а существует ли единый закон бесконечности, так скажем? Можно представить так: если кто-то сотворил зло, то это зло к нему вернётся?}
\soul{В очень приближённом,  для вас, варианте  - ``Не навреди''.}
\people{Скажите, всё-таки, пришельцы довольно часто посещают районы (перечисляет какие-то районы), что их особенно там интересует или тревожит?}
\soul{Тревожит ваша смерть.}
\people{Что?}
\soul{Вы рождаете смерть. Подумайте,  ваше же детище - несёт вам смерть. И мы вам не зря говорили, о защите. То, что вредно вам,- они могут защитить вас. Подумайте. Мы говорили вам, о тяжёлых металлах. Мы говорили вам: чем тяжелее, тем выше защита. Вы нашли это, сами. Но вы не можете сделать ещё один шаг. Мы подсказываем его вам. Спрашивайте.}
\people{Так, вопрос такой: вот вы говорите, что физическая смерть – это только смена оболочек, и пока мы не достигнем совершенства, в нашем понятии, мы не уйдём выше, я так понял. Правильно?}
\soul{Да.}
\people{И где эти критерии, когда мы станем ``сами собой"? Какие они, вы можете сказать?}
\soul{Мы же говорили вам: ищИте. Если мы вам дадим ответ, вы не сможете решить задачу в следующем теле, более тонком. Вы останетесь на том уровне, останетесь, потому что мы вас толкнём туда, и не больше. Рано или поздно вы упадёте. Вы упадёте ниже, чем были. И потому, вы должны заработать сами, тогда тот груз, что имеете, будет помогать вам, а не тянуть вниз.}
\people{Не так давно, под  Жирновском  произошло такое событие, что человек пастух, его тело сгорело, а одежда осталась целой. Можно ли…}
\soul{Время.}
\people{Это он попал в аномальную зону времени?}
\soul{Время.}
\people{Понятно. А можно обезопаситься? И как обследовать эту зону?}
\soul{Это его судьба.}
\people{А если мы туда сунемся?}
\soul{Согласитесь, то место, где был он, посещали многие. Или он попал туда, где никто не был?}
\people{Ну, там есть зона, огороженная людьми, видимо давно уже, колёсами от телег. От чего предостерегали людей, когда обкладывали…}
\soul{Вы, многое огораживали, боясь колдовских сил, и что же? В вашу изгородь со временем входили.}
\people{У меня такой вопрос. Вот, если мне, по моему пройденному пути, не суждено умереть, допустим, от радиации, я могу пойти смело в атомный котёл, так скажем, и, буквально, выйти оттуда без потерь?}
\soul{Нет. Даже если вы решили пойти в атомный котёл, значит, то было уже решено. Далее, мы говорили вам о ветвях. Вспомните, мы говорили вам, что есть тупиковые. Вы просто попадёте в одну из тупиковых ветвей, только всего. И, не зайдя в этот реактор, вы бы пошли дальше по стволу, пошли бы в другую ветвь, там было бы что-нибудь иное. Это и есть ваше право выбора.}
\people{А войдя в реактор, мы, значит, умрём физически, и потом вселимся опять в другое тело?}
\soul{Всё зависит от вас. Да, вы - вселитесь. }
\people{Лично?}
\soul{Спрашивайте.}
\people{Мы, всё-таки, собираемся посетить эту зону под Жирновском. Вы можете посоветовать меры безопасности? Стоит ли туда не подготовленным людям идти?}
\soul{Это ваши проблемы. И как вы понимаете – “подготовленными”? Вы, опять, понимаете это физически.}
\people{Т.е. вы не подскажете нам меры безопасности…}
\soul{Мы уже подсказали вам.}
\people{Тяжёлый металл…}
\soul{Мы говорили вам, не физически.}
\people{Ладно, мы ничего не поняли.}
\soul{Не поняли? Хорошо. Человек ложиться на операционный стол. Если он не хочет жить – никакая операция не поможет ему. Если даже операция окажется неудачной, в вашем понятии,-  он останется жив, если он жаждет этого.}
\people{Мы должны веровать,… Т.е. наши желания, они нас ``ведут'', так сказать? Наше тело ведёт наше желание.}
\soul{Это и есть ваша ошибка, что вас ведёт тело.}
\people{Нет, наше тело ведёт желание наше .}
\comment{(запись обрывается)}
\soul{Как вы сможете их победить, не навредив другим? Хорошо, вы решили: я брошу пить, я брошу есть. И что получится? Вы умрёте с голода, только и всего. Ибо вы решили это только физически, духовно вы не готовы. Далее, вы скажете: я бросил любить,- и многие делают это у вас,- и что же? Вы кого-то сделаете несчастным, только и всего. А мы говорим вам:  “локтями” вы в рай не попадёте.(толкаясь локтями. прим.)}
\people{Так, афганский математик высчитал, что 17 мая и 30 июня этого года в зоне Нижнего Поволжья, конкретно мы не знаем, может…}
\soul{С вами это происходит каждый день. Почему вы называете конкретные числа?}
\people{Но там будет контактная ситуация любопытная, о которой многие узнают. Унас есть желание оказаться в этих местах 17 мая и 30 июня. Есть ли шанс у нас, оказаться там?}
\soul{Пожалуйста, это ваши проблемы. Сколько раз вам уже обещали официально признание инопланетян. Вы дождались этого?}
\people{Нет.}  
\comment{(пошёл счёт)}
\soul{Знаете ли вы символ этого знака, что сделали? ( перекрестил переводчика. прим.)}
\people{Лично я? Да.}
\soul{Что обозначает?}
\people{"Спаси и сохрани''.}
\soul{Спрашивайте.}
\people{Вы получается, и видите нас?}
\soul{Мы говорили вам, что тело ваше видит больше, чем видят ваши глаза. Тело ваше реагирует на все потоки излучающиеся.}
\people{Мы не достаточно чувствительны к этим потокам. Мозг я имею в виду.}
\soul{В какой-то мере, это ваше спасение.}
\people{Да, мы бы сгорели со стыда.}
\soul{Мы не говорили о стыде. И говорим вам: представьте, если вы всё время будете понимать всё, что твориться вокруг и многие волны, созданные вами, представьте, какой будет хаос  у вас. Сможете ли вы это выдержать? }
\people{Нам надо быть готовым, что бы этот хаос как-то в порядке держать, да?}
\soul{Вы не умеете избирать. Вы - приёмник. И представьте, если вы расстроитесь до той степени, чтобы  принимать сразу всё, вы что-нибудь поймёте?}
\people{Ну, среднее арифметическое будет какое-то…}
\soul{ Вы не одни. В вашем понятии ``больные'' – действуют так. (сумашедшие.прим.)}
\people{Мне, всё-таки, интересно узнать про зону, где, хотя бы приблизительно, произойдёт контакт…}
\soul{Мы сказали вам, мы не интересуемся этим, и потому мы не знаем того. Мы не интересуемся физически, вы интересуете нас духовно. Далее. Мы разговариваем с вами, и мы не хотим говорить, о них. Далее. Это не принесёт вам пользу.}
\people{Скажите, один писатель написал, что у вас, скажем, с нашей точки зрения, инопланетянами, неприязнь, они вас не принимают, что ли, а почему-то вы обратились именно к людям, землянам?}
\soul{С чего вы решили так? Мы сказали, что мы контактируем с множеством. Далее, вы интересуетесь физически, вы ищете подобных себе, вы ищете на физическом уровне, вы ищете тех, живущих где-то, но похожих на вас. Но вас не интересует душа, того же инопланетянина. Вас интересует больше его развитие, физическое развитие, интересует его тело, интересуют его возможности в физическом плане, техника. О духовном - спрашиваете ли вы?}
\people{Я так понял, есть две ипостаси: мир духовный и мир телесный.}
\soul{Нет. Мы говорили вам, что мир един. Вы песчинка, песчинка  в огромном мире. Вы внутри этой песчинки и считаете песчинку миром, забыв, что песчинка находится в более другом, более большом. Вы не увидите из этой песчинки. Вы живёте в скорлупке, которую создали сами.}
\people{Мне сообщает девочка-контактёр из Михайловки о ряде своих возможностей. В частности, она обладает возможностью переходить на энергетическое питание. Т.е. отказ от пищи. Это действительно, правда?}
\soul{Мы говорили вам, подумайте, если мы вам скажем “да” и не скажем “нет” - мы изменим ваш ход, если мы скажем вам ”нет”, вы не будете верить ей. Если мы вам скажем “да”, вы поверите беспрекословно и будете экспериментировать, чем погубите и себя и её. Далее. Вы, чаще, только говорите: ``я могу'', ``я хочу''. И начав это, встретив трудности, бросаете. Представьте, - девушка, не имеющая питания, в вашем понятии, - даст ли ей это пользу? И сможет ли она это сделать? Чаще, вы фантазируете. Мы говорили вам, о гордости. Это - одно из проявлений.}
\people{Скажите, эта девушка сообщает, чтобы получить ответ на любой вопрос, а она получает их, необходим определённый допуск, а если туда допустили, то нужно обязательно получить знания. Так ли это?}
\soul{Нет.}
\people{Т.е. это, опять - фантазия?}
\soul{Вы должны быть подготовлены. Мы говорили вам,- вы получите знаний больше на основе старых, соединив старое  в узлы. Как вы можете понять новое, не пройдя старое? И конечно, в этом плане она права, но, согласитесь, это знали и вы. Почему же вы не верите себе, но верите ей, и верите нам? Почему вы не хотите увидеть это в себе? Вы  их уже знаете, и ваша  жизнь говорила об этом, - новое познаётся на основе старого. Согласитесь, что вы это знали, и знали уже давно.}
\people{Ну, всё-таки, мы сомневаемся…}
\soul{В чём вы сомневаетесь? Вы сомневаетесь даже в этом? Хорошо, не зная букваря, вы сможете читать?}
\people{Да, не сможем.}
\soul{Ну, вы же знали! Почему же вы ищете ответы у других и спрашиваете у нас? Почему, вы не верите себе, а верите нам? Вы же не знаете нашу природу, и мы не можем вам объяснить, но вы будете верить нам всему, что мы скажем?}
\people{Ну, мы тоже, с поправкой смотрим, не бездумно…}
\soul{Какова поправка? Поправка - на уровень вашего знания, на уровень вашего понимания. Но, вы делаете правильно, но мы пришли к вам рождать сомнения, мы пришли к вам, что бы вы делали поправки. Делали поправку относительно себя, и она будет всё выше и выше, но мы - не несём вам лжи.}
\people{Ну, хорошо. Мы, как раз,  это чувствуем, что наш  потолок знания поднимается, всё-таки, вверх. Но волгоградской  девушке были  продемонстрированы сеансы   лечения людей с помощью НЛО, инопланетян, причём с заменой, трансплантацией внутренних органов. Это фантазии её?}
\soul{Подумайте, вы, открывая новые земли, приносите подарки. Вспомните историю. Проходит время подарков, и вы становитесь палачами. Согласитесь, что, чаще, ``заиметь друга'' в вашем понятии, – одарить его чем-то.}
\people{Т.к. инопланетяне похожи на нас, ничем в принципе не отличаются, они будут действовать, как и мы в подобных ситуациях. Я - прав?}
\soul{Вы правы. Они не выше вас. Они - технически, - да, они умеют лечить. Но умеете и вы! Согласитесь,- лечение в средневековье и лечение сейчас? Теперь, представьте себя в средневековье, и вы будете “инопланетянами”, и вы будете лечить больше, чем другие. То же самое, для вас, - и инопланетяне.}
\people{Но, надо быть человеку, вылеченному благодарным за то, что ему заменили почку или желудок выправили…}
\soul{Чаще, вам делают хорошо, в вашем понятии, но вы не видите, что вас уже “точит червь”. Будьте осторожны, умейте доверять. Умейте. Но умейте доверять душой. Если вам приносят позолоченное блюдо, не радуйтесь ему, посмотрите, нет ли в нём яду.}
\comment{(обрыв записи)}
\people{…Кои имеются в языках, но есть возможность общаться через телевиденье, это очень существенный фактор, в первую очередь можно и в массовом порядке заявить о существовании миров.}
\soul{У вас здесь есть люди, которые могут сделать это сами. Зачем здесь нужны инопланетяне? Согласитесь, что здесь находящиеся могут сделать это сами. Это не предоставит трудности.}
\people{Но инопланетяне… Почему они не воспользуются телевизионной передачей, чтобы заявить о существовании?}
\soul{Мы сказали: многое - ваши фантазии. Но мы не будем говорить конкретно, где вы  фантазируете, а где, правда. И вы знаете причину. Мы говорили вам – вы должны научиться видеть сами.}
\people{Ясно. Но в 70-х годах был случай выхода неизвестных существ на телевизионное вещание на все каналы сразу.}
\soul{Хорошо, в той же Англии был случай «нашествия марсиан». Вспомните. И было ли это нашествием?}
\people{Ну, это были фантазии.}
\soul{Нет. Это была передача, и вы приняли за нашествие.}
\people{А-а-а … Да-да-да. Было такое.}
\soul{Спрашивайте.}
\people{Т.е. в Англии не выходили на всё телевизионное вещание? Это была газетная утка?}
\soul{Мы ответили вам.}
\people{Нас интересует гипотеза экспериментов внеземных цивилизаций на Земле. Существует гипотеза об эксперименте внеземной цивилизации созвездия Гончих псов или Ориона, об эксперименте на Земле по созданию человека. Она верна или совершенно ошибочна?}
\soul{Теперь подумайте. Зачем вы говорите о Гончих Псах, если вы стараетесь это сделать на Земле? Вы сейчас стараетесь создать животное, вы экспериментируете с клетками и стараетесь достигнуть. И зачем вы говорите о Гончих Псах? Если вы, находясь в начале, уже начинаете экспериментировать, почему же выше образованным вам не делать этого? Спрашивайте. Если вы не поняли, мы повторим, не стесняйтесь.}
\people{Я бы хотел продолжить первую нашу тему, о любви. Вы, не против?}
\soul{Спрашивайте.}
\people{Скажите, где грань между любовью и слепой любовью? И как можно любить то, что заранее приносит вред?}
\soul{Поймите, слепая любовь и просто любовь, только существует в вашем понятии. Слепая любовь – это не любовь, это - эгоизм.}
\people{Нет, допустим, любишь не это, а…}
\soul{Нет, это эгоизм и страх потерять. Слепая любовь – это когда вы не замечаете, что-то плохое в объекте любви. А почему вы не замечаете? Потому что вы боитесь потерять, и, поэтому, не замечаете. Закрашиваете своими красками, чтобы не потерять. Подумайте об этом.}
\people{Хорошо. А вот, если, допустим, это плохое я заметил, зная, что в объекте моей любви есть плохое, но у меня есть надежда помочь, чтобы он стал хорошим.}
\soul{Надежда помочь или помочь?}  
\people{Ну, естественно, что что-то делалось. Есть надежда, что этот человек мне не просто так…}
\soul{Если вы видите плохое, это уже не слепая любовь. Слепая, это - “не видеть” . Чаще, она рождена страхом, страхом потерять. Это и есть эгоизм.}
\people{Ясно.}
\soul{Спрашивайте.}
\people{Сейчас в нашем обществе много плохого, но это, же не значит…}
\soul{Вы хотите сказать, что вы сейчас слепо любите?}
\people{Ну, можно сказать, что я надеюсь…}
\soul{Вы не научились любить, а уже говорите о слепой любви.}
\people{А есть кто-нибудь, кто научился любить? Вообще, в вашем взгляде, в общем.}
\soul{Что заставило вас задать этот вопрос? Найдите в начало вашей мысли.}
\people{Но, есть ли, всё-таки, тот, кто узнал все краеугольные камни этой любви? Познал её.}
\soul{Сперва, ответьте на мой вопрос. Что заставило вас задать такой вопрос?  Почему вы задали? И найдите начало, начало, мысль, рождающая этот вопрос.}
\people{Начало мысли такое: возможно ли вообще постичь любовь?}
\soul{Не лгите себе, не лгите и подумайте, и найдёте начало. Если мы вам скажем, что вами говорит страх:  не успеть, не полюбить, вы ж не поверите мне. Вы скажете: нет, это не так. Вы найдите начало.}
\people{Скажите, почему наши позывные в космосе на радио частотах, остаются без ответа, до сих пор?}
\soul{Мы вам говорили. Вы горды так, что хотите весь мир увидеть своими глазами. Вы хотите, чтобы все были подобны вам. Почему вы решили, что все остальные должны работать на ваших частотах? }
\people{Т.е. у них могут совершенно другие быть радиочастоты?}
\soul{Поймите, поймающие эти радиочастоты, будут, даже в вашем техническом плане, не умнее вас.}
\people{А какие рекомендации вы можете дать относительно технического поиска разумных цивилизаций, разумной жизни во вселенной?}
\soul{Ближе к природе.}
\people{Только так?}
\soul{А вы посмотрите, посмотрите, как излучает Земля и повторите. Вы когда-то начали это, и нашли волны водорода и стали излучать. Это - начало разговора с природой. Говорите ей. Мы же скажем: вы нашли только малый кусочек. Сумейте повторить, как говорит Земля. Она разговаривает со всеми звёздами.}
\people{Так значит, всё-таки, есть определенные частоты в электромагнитном диапазоне, которые…}
\soul{Нет. Ну, представьте, как Земля может излучать в определённом диапазоне? Подумайте, может ли она это делать? Согласитесь, определённый диапазон – определённые размеры резонатора. Земля же имеет их множество, она имеет множество элементов и каждый излучает по-своему. Как мы можем говорить об одной частоте? Вы принимаете звук. Можете ли вы сказать, что вы принимаете одну частоту?}
\people{Т.е. исследование…}
\soul{Комплекс.}
\people{Комплекс исследований с Землёй, для нас может быть очень прогрессивным. Мы что-то узнаем, да?}
\soul{Да. Возьмите аналогию со звуком. Только комплекс, в вашем понятии, звуковых частот, позволяет вам понять. Если вы будете говорить на одной частоте, вы даже не услышите этого.}
\people{А с Землёй? У учёных есть уже такие попытки понимания диалога с Землёй?}
\soul{Да.}
\people{Это у нас в стране?}
\soul{И у вас тоже. И, заметьте, в вас опять говорит гордость.}
\people{Это неистребимо видимо в нас, хотя, мы  и сознаем своё несовершенство.}
\soul{Да. Все ваши черты приносят и плохое и хорошее. Нельзя сказать, что у вас то-то и то-то - плохое. Нет, оно приносит и пользу. И мы говорили вам когда-то, что нет таких вещей, которые бы были только плохими или только хорошими. Всё зависит от множества, как воспринимать, использовать и далее.}
\people{Ну, вот допустим, как расшифровать, вот эти вот волны? Смыслокод, какой-то должен быть? }
\people{У нас в звуке есть смыслокод, определённая…}
\soul{В вашем понятии, есть  смыслокод. Но подумайте, будет ли Земля носить смыслокод, который носите вы? Она будет говорить вашими словами? А мы говорили вам, что придя на другую планету, вы будете смотреть себе подобных, и не заметите  непохожих. Скажете: это не жизнь, жизнь – это мы. Вспомните. И потому, вы, имея понятие смыслокода только в вашем понятии, не найдёте его. Но, заметьте, Земля не говорит вашим языком.  Язык - это ваше искусственное, и, потеряв его, вы будете разговаривать гораздо лучше.}
\people{Много споров ведётся об истории Земли. Якобы, история человечества насчитывает…}  (счёт)
\comment{(сеанс окончен)}
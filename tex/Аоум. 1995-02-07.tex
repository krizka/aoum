Аоум. глава 23-я 07-02-1995г
Георгий Губин
 
\people{**}
 (Начинается не сначала)
\people{Да, мы дышим практически одними проблемами. Иногда, бывает радостно, когда день светлый, солнышко греет, всё вроде есть причина жить.}
\soul{Поймите, вы поделили мир на счастье и горе, на добро и зло. И потому, вы видите кусками. Сейчас вы настроились на проблемы. Завтра, вы настроитесь на что-то другое, послезавтра - другое, другое, и вы можете меняться каждую минуту. Вы не можете слышать всё. Сознание ваше слабо. Вы хотите воспринять сознанием. Поэтому, вы - узконаправленный  луч. Проблемы? Значит, только проблемы. И трудно будет отвлечься на иное. }
\people{Отвлечься-то можно, это не трудно, но, всё время тебе напоминает…}
\soul{Представьте, вы видите только проблемы или что-то одно и одно. Хорошо, если вы прыгаете. А если вы только идёте по одному направлению и видите только одни проблемы?}
\people{Да, жизнь в ``копеечку'' покажется.}
\soul{А теперь, представьте, мы, не хотящие делить. И для нас нет понятий проблем, счастья, радости и горя. Мы живём. Мы просто живём и видим сразу всё, и поэтому, не можем сказать, что мы, сегодня радостнее или горестнее. Ибо мы не делим, поэтому у нас нет этих понятий. Вы скажете, и когда-то уже говорили: ``скучен мир ваш, нет искусства у вас'', и далее. Вы помните?}
\people{Да, но это…}
\soul{В вашем понятии, нет у нас искусства. Ибо вы искусство создаёте, или только для слуха, или только для глаза. }
\people{Сейчас комбинированные начали создавать.}
\soul{Комбинированные?}
\people{Да.}
\soul{И что вы из этого получаете?}
\people{Допустим, светомузыка.}
\soul{Да? Назовите мне светомузыку, ту, которая могла бы одна, без музыки, сказать, о чём говорит она. Комбинацией называется лишь только то, если оба компонента имеют равную силу. Вы привели не удачный пример. Вам лучше было бы сказать ``картины и музыка''.}
\people{Да, есть и такое. Я просто не вспомнил вовремя.}
\soul{А тогда подумайте, если вы всматриваетесь в ту картину, слышите ли вы музыку? И когда вы слышите музыку, всматриваетесь ли вы в эту картину? И можете ли вы увидеть картину полностью, целиком? Когда вы слушаете музыку, и вы желаете быть более внимательны, - вы теряете множество, из-за вашей внимательности. Ибо вы начинаете прислушиваться к какому-то одному инструменту.}
\people{Вот, вы, тут, сегодня, противоречите сами себе.}
\soul{Пожалуйста.}
\people{Вам напомнить?}
\soul{Пожалуйста.}
\people{Вы обвинили меня, что я не вдаюсь в подробности, что я ``верхушечник'', как меня мой друг назвал. Я с ним полностью согласен, не обижаюсь. И тут же говорите, что если увлекаться в одно направление, не брать целиком всё, то я просто закопаюсь в этих конкретностях, и целиком ничего не увижу.}
\soul{Вот ваша поверхностность и здесь. А вы подумайте. И давайте, повторим. Вы слушаете музыку. Если вы захотите послушать её внимательно, то вы будете обострять внимание на какой-то отдельный инструмент. Вы уже будете менее слышать другие. А можете ли вы слышать цельно? Можете только тогда, когда вы невнимательны. Но тогда уже - сознание не слышит. Вы в это время можете читать книгу, смотреть фильм и что угодно. Для вас, это только фон. Только для сознания. Вот вам и способы гипноза. Когда вы хотите что-то внушить, но чтобы об этом не знали. Вы поняли меня? И вот вам один из ответов на вопрос, который вы должны решить об обманчивости реальности.  Далее, как вы думаете, в какой мир смотрите вы? В каком мире вы живёте, относительно сознания? В реальном или нет?}
\people{Мне кажется, я вообще… Живу вообще, не в том, где мне хотелось.  }
\soul{Мы вас спрашиваем.}
\people{В реальном.}
\soul{В реальном. Какова ж его реальность, если мы говорили вам, и вы согласны, что вы в этом мире видите столь малое и слышите ещё менее?}
\people{Подождите. А чувства? Их куда деть? Вы говорите ``навевают''. Вот эта мне жизнь навевает не то, что она из себя представляет.}
\soul{А что вы решили ``навевает'' вам? Чувства или ваше сознание?}
\people{Не знаю. Интуиция, может, или, может, сознание.}
\soul{Интуиция?}
\people{Кстати, есть она вообще или, так, - фокус сознания?}
\soul{Чаще, интуиция - это сознание. К сожалению, это чаще. }
\people{Вчера удостоверился.}
\soul{Да, вы не можете видеть всю картину, вы смотрите её маленькие кусочки. И поэтому, вы не можете понять многих положений, что творятся. Вы не можете понять, отчего и почему. Ибо вы только точка в этой картине, и ближайшие точки бьют вас. А вы не знаете почему. А всё просто. Незнание ваше только в том, что сознание ваше слепо и глухо. Об этом говорят даже ваши науки. Далее, О реальности мира. Вы не можете увидеть энергетические поля, нет? Но вы прекрасно используете. Но вы не можете их увидеть. А если б вы их увидели? Представьте, хотя бы на мгновение, что вы будете видеть… ну, давайте больше не будем говорить, о электрических полях. Давайте, просто расширим немножко звуковой диапазон. Хотя бы немножко. На один порядок. Будет ли он слышен для вас?}
\people{Получается, что мы более глухи, чем слепы.}
\soul{Вы можете ответить, почему вы сказали так? }
\people{Более глухи, чем слепы?}
\soul{Здесь не нужно брать что-то мистическое или страшное. Достаточно знать ваши законы физики.}
\people{Нет, я вот, как раз, вычисляю длину волны.}
\soul{Ну, давайте скажем так. Свет – это что? }
\people{Ну, то же самое.}
\soul{То же, что и звук. Вы согласны? Теперь подумайте, если поменять эти вещи местами …}
\people{Мы бы слышали больше, чем видели?}
\soul{Нет. Давайте скажем так: вы будете слышать в видимом диапазоне. И будете видеть в звуковом диапазоне. }
\people{Интересно это попробовать.}
\soul{Интересно? А в природе это существует. И вы сами уже изобрели такие приборы. }
\people{Ну, это будет не понятно для сознания?}
\soul{Да.}
\people{Оно просто не поймёт, что там свистит, кашляет, вспыхивает.}
\soul{Да, потому что вы сразу начнёте сравнивать. Вот и  будете брать за эталон то, что видели и будете сравнивать с тем, что теперь слышите. А теперь представьте, если б изначально у вас были бы поменяны эти чувства, вы бы поняли ту разницу? Вы бы этого не заметили.}
\people{Верно. }
\soul{А почему же тогда сознание это замечает сейчас? }
\people{Потому, что оно привыкло, как вы говорите.}
\soul{Вот вы пришли к понятию привычки. Что в вашем понятии ``привычка''?}
\people{Условный рефлекс…}
\soul{Энергия, и причём, очень грубая. Придут физики и скажут: ``Разве мы можем измерить привычку сознания? Разве можем изменить чувства?''.  Но, простите, ваши же физики говорят, что электрическое поле воздействует на все виды материи. Но почему же тогда они отрицают, что глаз может видеть электрическое поле? Вы можете мне это объяснить? Почему вы противоречите себе?}
\people{Я этого и не утверждал. Что-то я вообще не задумывался.}
 1-2-
\soul{Спрашивайте.}
\people{Скажите, пожалуйста, в вашем мире все едины, но думают по-разному, насколько я помню. Имеют право думать, как хотят, т.е. свободная воля существует?}
\soul{Для вас, трудно объяснить, что мы едины, и, в то же время, мы имеем индивидуумы. Мы не можем вам этого объяснить. Ибо ваше сознание сравнивает… А с чем оно может сравнить совершенно иное?  В вашем понятии, мы не имеем физики. И, в то же время, мы говорим, что мы тоже имеем тела. Но, вашему сознанию это трудно понять, и оно только допускает, что, значит, у нас другие тела – не физические. И вы, опять же, говорите ``Энергетические'', хотя, энергетика – это физика.}
\people{Вы и энергии не имеете, что ли?}
\soul{Простите, в вашем понятии?}
\people{Нет, вообще.}
\soul{Ничего физического мы не имеем ничего. Вы поймите, мы же говорили вам, что даже ваши биополя, что называете, вы сравниваете физикой. И потому видите только семь. Разделили на семь. И причём, последние два - вы придумали только название, и не можете даже представить чтоб они из себя представляли. Ваш мир поделен на 7. }
\people{Ну, да. Семь дней.}
\soul{Мы говорим вам, о семи уровнях. Вы придумали название для всех. Последние два – у вас есть только название, и вы не можете даже представить, что это такое. Где бог… Вы говорите: - ``Энергетика''. Но это же, физика, если вы учите об энергетике в учебниках физики. Разве это не физика? Хорошо, вы скажете:- это более тонкие поля. Но, простите, когда-то вы не знали понятий ``радиоволны'', об электричестве. У вас была физика, теперь вы стали знать, что существуют еще более тонкие поля, по сравнению с теми временами: существует электрическое поле, существует световое и далее, далее. Но, поймите, логически вы должны понять, что, значит, рано или поздно вы откроете более тонкие поля. Вы согласны? Рано или поздно вы начнёте изучать уже поля, которые носите вы. То есть – ваше тело более подробно. И вы, рано или поздно, изучите все эти 7 тел и скажете: ``здесь нет бога'', ибо физика ваша изменится. И изменится сознание ваше. И вы никогда не сможете найти именно те поля, именно те, что называются человеком. Вы человека приравниваете к своей шкуре. Одежде.}
\people{Скажите, как вы относитесь к Иисусу Христу?}
\soul{Мы когда-то говорили вам, он человечнее вас.}
\people{Да.}
\soul{А теперь возьмите и сопоставьте то, что мы сказали сейчас.}
\people{Я что, был Христом?}
\soul{Да. Когда-то вы бы к этому пришли,- вашими словами. Но, это не значит, что были вы.}
\people{Ясно. }
\soul{Далее, мы говорили вам только что, о полях, что существуют в вашем понятии.  Поля – это всего лишь ``одежды''. То, что не }
имеет полей - это и есть Человек. Теперь вы - ваш Иисус, и мы говорим вам, что он более человечен. Почему? Даже логически, ваше сознание может понять, что Иисус не носит, не носит ваших тел. Истина именно в тех полях, которые не может найти физика. И не найдет. Она будет допускать, и будет искать. Будет совершенствоваться и придёт в тупик. И когда-то, кто-нибудь найдет выход из того тупика, но, тогда, будет другая физика. Уже не та. Вы поняли?
\people{Она грубее будет?}
\soul{Нельзя будет говорить о ``грубее'' и ``тоньше''. Это иное, совершенно иное, не имеющее никаких точек, общих с вами. }
\people{Хорошо. Как понять ваше заявление, что мы - это вы, и, вместе с тем, мы  - это не вы?}
\soul{И вы опять думаете сознанием? Вы поверхностны. Знания ваши широки, но вы - поверхностны. }
\soul{(20 сек ничего не слышно, плохая запись.) Но, мы же говорили вам, об истинном человеке, не имеющем этих полей. Почему вы так не внимательны? Почему вы всегда… Вы очень небрежны. Конкретно вы. (20 секунд почти пустоты). }
\people{Таким родился.}
\soul{Таким родился?}
 (10 сек ничего не слышно, плохая запись.)
\soul{Может вы одели одежды, что сделали вас уродами? А теперь, обвиняете бога. }
\people{Ну, почему - бога?}
\soul{Бог сотворил по ``подобию своему''. Через эти слова потеряли веру в бога. Ибо – ``Как я уродлив! Неужели и Бог таков?”}
 1-2-  1
\people{Скажите, вот, тоже, немножко не понятно… Может я, действительно, сознанием всё думаю? Вы как-то сказали, что душа есть у всего и вся.  И что нет бездуховного. Без души. Вы говорили так? По смыслу. Приблизительно.}
\soul{Даже приблизительно, это так.}
\people{Хорошо. А в первых контактах, мне помнится,  вы говорили, что множество ходит людей без души и отказались отвечать на вопрос  ``почему?”}
\soul{Что спрашивало нас?  Только сознание ваше, и потому, мы говорили так, чтобы сознание ваше поняло. Мы говорили вам, что – да, есть те, кто не имеют души. Но, простите, сознание ваше приняло это, как отрицательное. А вы не допускаете…Вы же, сами  произнесли это: ``Но ведь он же, человек''! Вы согласны? А вы про него говорите: у него нет души! Но это не значит, что у него действительно её нет.}
\people{А! Понятно.}
\soul{Просто она столь глубоко спрятана, что её почти и не видно. И вы не ощущаете её и не чувствуете. И мы когда-то говорили, что вы не можете видеть и нас. Вы можете видеть реакцию. Реакцию вашей физики на нас. И вы скажете: ``мы видели вас и измеряли вас'' линейками  и далее, далее. В какой-то мере, для сознания, для этого уровня -  вы правы. Вы можете оценить ту реакцию физики вашей.  Принять за нас. Нельзя сказать, что вы солгали. Ваша физика отреагировала так, и приняв вашу реакцию на нас, за нас, вы видите физикой только маленький кусочек. И измеряя душу, вы измерили реакцию ваших приборов. И если б вы могли измерить столь грубыми приборами, что существуют у вас, ими могли б измерить душу, то как легко было б вам её уничтожить! Это, только реакция. И нам трудно объяснить, что вы одновременно и лжёте и говорите правду. Чтоб это понять, вам надо поумнеть ещё. Да, вы лжёте. Ибо это - всего лишь реакция. Реакция вашей физики. Но вы и правы, ибо, для вас, эта реакция – правда. Можно ли обвинить ребёнка во лжи, если у него о мире понятие другое? И можно ли обвинить того же ребёнка в том, что ваш красный цвет он видит, что это зелёный?}
\people{Почему? }
\soul{А вы не учили его.}
\people{Ну, тогда  получается, что ребёнок видит с другой реакцией?}
\soul{Давайте скажем так:  В чём разница, если с прилавка взяли вы, или взял ребёнок, не имеющий понятия о воровстве? Для одного, это будет воровством.  Можно ли назвать вором младенца, если он не знает понятия о воровстве? Вы тут же сейчас скажете: Первобытный человек не знал, можно убивать или нет и поэтому, его нельзя обвинить в убийстве.}
\people{В принципе, да. Получается так.}
\soul{Рано или поздно вы бы сказали. А вы найдите разницу в том и в том.}
\people{Хорошо. Где-то на островах до сих пор считается хорошими манерами съесть человека.}
\soul{И было бы очень плохой манерой, если бы вы обвинили их в убийствах. Это всё равно, что обвинить ребёнка. Да, по вашим понятиям, это жестоко. Но, простите, вы считаете себя современным, а его диким. И поэтому, нельзя, нельзя сравнивать. Вы можете говорить только о степени дикости, но, вы не можете обвинять его в убийстве.}
\people{Это вы хотите сказать, что мы можем обвинять только своих ``соплеменников'', так сказать?}
\soul{Вы можете только говорить, о совершенстве. В вашем понятии, убивать нельзя. Не дай бог! Но вы поняли, что нельзя убивать, но убиваете и очень успешно. Так простите, кто более виновен, вы или дикарь, который не знает, что этого нельзя делать? Кого более обвинят? Как вы говорите ``в воровстве''? Кого более обвинят? Подумайте. И вы найдёте в Библии ответы.}
\people{Скажите, у вас философия, вообще в почёте? Порассуждать, там…}
\soul{В вашем понятии?}
\people{Да.}
\soul{В вашем понятии – у нас иные понятия.}
\people{Ну, хорошо. Пообщаться - скажем так.}
\soul{А что, мы с вами сейчас не общаемся?}
\people{Нет, я говорю, вы между собой как-то контактируете?}
\soul{Мы не имеем этого.}
\people{Интересно. Интересно, вы что, вы вообще не обмениваетесь ничем?}
\soul{Простите, как вы обмениваетесь меж собой?}
\people{Приходим и говорим.}
\soul{Да? А как мы можем прийти к самому себе? Вы можете ответить? С самим собой? Вы представьте. Вы можете поговорить с собой самим? Вы можете представить, как можно прийти к самому себе в гости?}
\people{Я вообще не представляю. Когда начинаешь в себе копаться, там чего-нибудь, открывать себя ещё заново – ну, вот тут диалог и происходит.}
\soul{Диалог или игра?}
\people{Может и игра. Я ещё не думал об этом.}
 (сбой)
 1-2-3-4
\people{Скажите, а разрыв в нашей беседе откликается на времени  контакта переводчика, или это не зависящее никак? (имелось ввиду – перерыв между 2-мя контактами проведёнными в один день )}
\soul{Нет. Вы можете, не прерывая контакта, выйти. (из транса. Прим.)  Это не изменит. Мы не придём только в том случае, если у вас не появится желание. И в то же время, мы можем сказать, что – да. Ибо вы столь не стабильны, и желания ваши столь поверхностны. Далее. Мы желаем говорить с вами? Мы не желаем того! Вы – желаете! Мы приходим. Далее. Мы говорим с вами почему?}
\people{Да.}
\soul{Нам приходится приходить и разбивать. Разбивать целое, что бы рождать те мгновения, что бы нам и разговаривать с вами. И подобных, в вашем понятии, ``мгновений'' - много.}
\people{А это будет добром?}
\soul{Вы хотите! Мы приходим и пытаемся разбудить вас. Но, к сожалению, мы всё больше убеждаемся, что надежды наши не сбудутся. Не сбудутся ещё долго.}
 (Конец записи)
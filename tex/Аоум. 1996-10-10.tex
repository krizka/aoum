Аоум. глава 10. 10. 1996г
Георгий Губин
***

 (Другое время. Примерно 1648 год) 

 (Монахи) ….и положивши его…

 (Гера)   Положивши его…

 (Монахи) …накрыв опахалом и произнеся молитву, должны услышать ответы на вопросы наши.

 (Ольга)  Кто это?

 (Монахи)  Он же, должен ответить за каждый. И если промолчит – будет наказан. Только так мы можем заставить их говорить им.  Им жалко его, и потому отвечают нам. У нас же, вопросов множество!  Мы хотим быть властелинами и будем ими! Мы уже множество знаем способов убить или узнать любую из тайн. И эти же способы проверяем на нём же. 

 (Ольга)  Мы не хотим с вами иметь дело!

 (Монахи) Им жалко его – и продолжают отвечать нам. 

 (Ольга)  Мы не хотим с вами иметь дело!

 (Монахи) Тогда, когда же проснётся, кормим и поим его, но ничего не рассказываем. И глядя на раны свои, говорим ему: «Это они били тебя!» Он же злится! Тогда приходят они, чтобы снова извинится. А тут - мы! Так было и так будет! И сын его будет таким же!  Сила наша в том, что мы не знаем жалости! Сила наша в том, что мы умеем задавать вопросы и умеем заставить отвечать их!  Многие из нас уже умерли, может быть и с ихней помощью. Но нас много и всех - трудно. Тем более,  они слишком добры.  Тем и прекрасно зло, что имеет кулаки и может постоять за себя!  А добро? Что добро?! Что может оно?! Оно его ласкает, а нам того и надо! И чем больнее делаем, тем быстрее отвечают нам. Говорят нам, к совести призывают: «Вы же, монахи, служители Господа!» Да, мы служители Его! Потому и имеем на множество право! Он не имеет понять руку на нас, потому что он раб наш, а мы рабы Божьи. И если Богу надо – будет! И нам плевать каким способом сделаем это… Мы - монахи, а он - всего лишь отвечающий. (Запись обрывается) 

[щелчек:выключается и включается запись]

(Безымянный)-Кто вы?

(Гера) *Мы?
(Ольга Васильева)*Говори.
(Гера)* Сергей, ты?

(Безымянный) - Кто Вы?[удивленно]

(Гера)* А Вы?
(Ольга Васильева и Девушка) *А Вы - кто?

(Безымянный) - Вы не те, с кем обычно говорили, что доставляют боль мне, но и не Монахи.

(Девушка) * Ах, это…[шепотом]
(Ольга Васильева) *Мы - люди.
(Гера) * Так…А кто ты? Как тебя зовут? Где ты живёшь?
(Ольга Васильева)*Как Вас зовут?

(Безымянный) - У меня нет имени, мне не дают имени.

(Ольга Васильева) *Почему?

(Безымянный) -Потому что, говорят,  я не должен иметь имени, потому, что я ещё не вырос для него. И если я отвечу на все их вопросы, тогда мне дадут имя.

(Гера) *А что они хотят знать?

(Безымянный) - Я не помню, они не говорят. Но, иногда… Иногда, я что-то помню.  

(Гера) * Что?

(Безымянный) - Они спрашивают - как получить тайну, как можно узнать тайну. Как заставить человека говорить… Ещё. Ещё  они спрашивали о каких-то книгах, что где-то у них здесь зарыты книги, и в них таятся знания древние. 

(Гера) * Угу.

(Безымянный) - И будто они[те у кого спрашивают монахи] знают, должны знать где они[книги] закопаны.  Они[монахи] спрашивают, но Те[те у кого спрашивают Монахи] не отвечают. И они, может из-за того, из-за злости, делают мне больно. Когда просыпаюсь я, монахи говорят: Они были жестоки и мучили тебя, и ты должен благодарить нас. 

(Гера)* Брешут.

(Безымянный) -А как я могу их благодарить? Я ложусь, они опять задают вопросы. А кто вы? Вы - люди?

(Ольга Васильева) *Мы - будущее. Мы - будущее.
(Гера) *А какой год сейчас на дворе?

(Безымянный) - 1648 от рождества Христова.

(Гера) *Так, ясно. А…[собирается задать вопрос]
(Ольга Васильева) *И вот у тебя… и у тебя нет имени?

(Безымянный) - Нет. У меня нет имени.

(Гера) *А мать, ты помнишь? Отца?

(Безымянный) - Я всегда жил здесь.

(Гера)* А страна? 

(Безымянный) - Страна? [Задумался]

(Гера) *Да.

(Безымянный) -Для меня, этот монастырь - страна.

(Гера)* Ну, как они называются…?
(Ольга Васильева) *Народ как называется? 
(Гера) * Народ? 

(Безымянный) - Я не знаю. Я живу здесь.

(Гера) * Но, ты живёшь… А как они называют себя там? Ведь какие-то…

(Безымянный) - Я их называю Отцами.

(Гера) * Отцами. Хорошо. А имена есть у вас?

(Безымянный) - Конечно.

(Гера) * Какие?
(Девушка) *Не говорят.

(Безымянный) - Почему же? У них есть монахи, носящие имена женщин, но они почему-то не женщины. У них есть имена, носящие мужчин и зверей! Но они не звери.

(Гера) * Ну, скажи хоть…

(Безымянный) - Я знаю Лисицу.

(Гера) * Угу

(Безымянный) - Я знаю Гору.

(Ольга Васильева) [вздыхает удивляясь][кашлянула]
(Гера) *А Гора, это, случайно, не вождь?

(Безымянный) - Нет, он спрашивающий.

(Ольга Васильева) * Скажи. Они хотят знать…
(Гера) * Что они хотят?
(Ольга Васильева) * А что, зачем они хотят…?
(Гера) * Где закопаны книги в монастыре, да?

(Безымянный) - Я помню, что они спрашивали о книгах. И то, только из-за того, что потом, когда я проснулся, они спрашивают: “Помнишь ли ты о книгах? Они ведь говорили о книгах!?” А я не помню.

(Гера) * Угу

(Безымянный) - Они водят меня по монастырю. Дают мне в руки прутик и смотрят, что буду делать я. Я хожу… Я должен зайти в каждый уголочек, я должен потрогать каждую стену. А знаете, как страшно… как страшно, в подвалах!?

(Гера) * Да-а.

(Безымянный) - И эта веточка должна показать книгу. И когда я прохожу, и если где-то она шелохнулась, они разбивают стену и ищут. Они, никогда не ошибались, но книгу не нашли. 

(Ольга Васильева) * Скажи, а монастырь где? Он в скалах вырублен?

(Безымянный) * Вокруг -  лес.

(Гера) * А села нет никакого?
(Ольга Васильева) * А монастырь…

(Безымянный) - Нет, только монастырь. Я не могу выйти из него.

(Ольга Васильева) * Ты никогда не видел других людей?

(Безымянный) - Нет…

(Гера)* Никто не приходил, не забегал?

(Безымянный) - Нет! Приходят!  И, даже, приезжали! 

(Ольга Васильева) *На лошадях?
(Девушка) * Кто?
(Гера) * На чём?

(Безымянный) - Да, ло-ша-дях. Это какие-то большие сундуки, а в них сидят люди.

(Ольга Васильева)*  Карета называется.
(Девушка) *Карета, да.
(Гера)*  Сани.
(Девушка) * А кто приезжал?

(Безымянный) - Я не знаю, меня не пускали.

(Ольга Васильева) * Скажи, а вообще - чем ты ещё занимаешься?

(Безымянный) - Ничем.

(Ольга Васильева) * А сколько тебе лет?

(Безымянный) - Не знаю.

(Ольга Васильева) * Ты молодой или старый?

(Безымянный) - М-м-м…[думает] Наверное, молодой. Уж я ещё помню детство.

(Девушка) * А какое детство было?
(Гера) * Какое у тебя  было детство? Ты всегда в монастыре был?

(Безымянный) - Да. Я помню, когда мне приносили молоко, и когда я пил его, сыпали порошок, и говорили, что благодаря порошку этому, я буду говорить с другими… 

(Ольга Васильева) * С другими, - с кем?
(Девушка) * С нами?

(Безымянный) - С Вами, с Ними.

(Гера) * С кем  “ с ними”?
(Ольга Васильева) [кашлянула] * С монахами [говорит Гере].

(Безымянный) - Нет!

(Ольга Васильева) * А с кем?

(Безымянный) - С теми, у кого спрашивают монахи.

(Гера и Ольга Васильева)  *А-а-а!
(Ольга Васильева) * А ты ещё с кем-то разговариваешь, кроме нас?

(Безымянный) - С Ними.

(Гера) * И всё?

(Безымянный) - А! И с Вами сейчас.

(Гера) * А ещё с кем-нибудь?

(Безымянный) - И всё.

(Гера) * Всё, да?
(Ольга Васильева) * Нет, а когда раньше. Или раньше когда-то - с кем ты ещё говорил?

(Безымянный) - Нет. Я пил молоко с порошком. Его ещё называл… Гора его называл, Гора его называл - зелье «Оммана».

(Ольга Васильева) *Ага, Амрита. А! Ну, да… Ну…

(Безымянный) - И это зелье, раскроет уста и глаза мои  для других миров.

(Ольга Васильева) * Миров… Скажи, а у вас зима бывает?

(Безымянный) - Да!

(Ольга Васильева и Гера) * Снег, да? 

(Безымянный) - Да, да. И тогда мне холодно.

(Ольга Васильева) *А тебе не дают одежду?

(Безымянный) - Вы знаете… На мне цепь и большой камень. И для того, чтобы разогреться, я начинаю бегать по двору, а камень беру в руки. Тогда мне становиться теплее… Но когда я должен говорить с кем-то, то мне запрещают бегать. Но, правда, отводят в келью, и там я греюсь…

(Ольга Васильева) * А где ты спишь?

(Безымянный) - Потом меня поят молоком, но уже без порошка. И тогда, укрывают меня, ставят свечи во главе и поют молитву. Потом, я не помню ничего. Просыпаюсь от боли.

(Ольга Васильева)*  Где у тебя болит?

(Безымянный) - Там, где бьют. 

(Девушка) *А где бьют? Почему бьют?
(Гера) * По спине?

(Безымянный) - Бьют везде. Потом,  мне, если я плохо отвечал, то меня перевязывают и не дают ходить в туалет, и тогда я мучаюсь.

(Ольга Васильева) * Скажи а у тебя…А ты где спишь то? У тебя есть постель какая-то?

(Безымянный) - Да.

(Ольга Васильева)*  Ведь у тебя есть там хоть чем-то укрываться?

(Безымянный) - Укрываться? 

(Ольга Васильева) * Ну, когда спишь, да.

(Безымянный) - Солома.

(Ольга Васильева) * А… солома, да. А из одежды, что на тебе?

(Безымянный) - То, что износилось, отдают мне. У меня много этого тряпья. 

(Ольга Васильева и Девушка) * А-а-а… А кормят тебя…



(Обрыв пленки) 
      ( На этом месте речь шла о ведьмах и “Святой Инквизиции.”) Прим.

(Безымянный) - …Не знаю.

(Девушка)* За что это всё?

(Безымянный) - Не знаю. Они не признаются. Но благодаря моим ответам, они признаются в этом. И тогда их сжигают и….

(Девушка) *На костре?

(Безымянный) - Да. Ещё -  бросают в воду.

(Ольга Васильева) *В воду.
(Девушка) [вздыхает удивляясь] *А, это женщины, в основном, молодые и очень красивые, да?

(Безымянный) - Да.

(Девушка)* Да…
(Ольга Васильева)* Они всплывают, на воде, да? Если они всплывают, значит, их сжигают…

(Безымянный) - Нет.

(Ольга Васильева) *Нет? Их топят?

(Безымянный) - Их топят!

(Ольга Васильева) *Топят… Вот это “времена” у вас…
(Девушка) *Инквизиция.
(Ольга Васильева) *Ну, так 17 век. Конечно…[Девушке]

(Безымянный) - Но, ведь, поймите,они же – ведьмы! Мне их жалко, но, они же, ведьмы!

(Ольга Васильева) *Они - люди!
(Девушка) *Ну, ведьмы… Из-за чего же такое называние - ведьмы?

(Безымянный) - Но, они же - против бога…

(Ольга Васильева) *Нет. Они не против бога.
(Девушка)* А чем это…

(Безымянный) - Они против монахов.

(Ольга Васильева)* Ой.[тяжело выдыхает] Это монахи тебе так говорят. Они не против бога. Разве,  когда убивают…

(Безымянный) - Б…бэ…[диалог прерывается] 

(Ольга Васильева) *…убивают кого-то именем бога?

(изменилась интонация)

(Случайный) - Это пахнет!

(Гера) *Что пахнет?
(Девушка) *Что?

(Случайный) - Видишь, этот пахнет, это называется “Хрущёвщина”!

(Гера) *Это что – “Хрущёвщина”? Что имеешь в виду?..
(Девушка) *Это кто-то другой.[шепчет]
(Ольга Васильева) *Ага.

(Контакт прерывается, идёт счёт)

(Ольга Васильева, Гера, Девушка)* 4, 5

(Шёпот)  - Первый, пришедший в Астрал, сперва теряется. Им необычен этот мир, они пугаются его. Они видят множество цветов, и видят совершенно всё по-другому. Предметы, которые знали всю жизнь, выглядят совершенно по-другому. Тот же самый шифоньер оказывается вдруг таким объёмным, и в нём можно спрятать множество стран. 

(Девушка)  *Ага.

(Шепот)-  Никогда нельзя было подумать, что в Астрале могут находиться теже предметы, что есть и здесь. Всегда считалось, что Астрал - это, значит души… Души умерших. Или души не рожденных… Или, множество таких, но только души! Оказывается, нет! В этом мире тоже есть горы.  Тоже есть небо. Тоже есть Земля. И, даже, в некоторых кусочках - есть та же самая техника, но только работает немножко по-другому. Удивительный мир! Мир красок. И если в этом мире вы почувствуете боль, это будет не та боль, которую вы видите и слышите здесь. Это будет боль, которая захватит всего Вас. Всего! Это очень страшно. Очень. Но, в этом же мире и есть моменты счастья. Тогда вы будете действительно счастливы, как не могли быть счастливы здесь. Что такое Астрал!? Астрал — Начало. Первая буква говорит о том, что это — Начало. Последняя буква говорит о том, что этот мир — ме-ня-ем! Он не постоянен! Он не может быть постоянен, потому что Астрал создан нами. Нашими… нашими чувствами, нашими желаниями, нашими мечтами, нашими проклятиями, нашими болью и счастьем. И все это – наше! Если сказать, что человек родил Астрал, это будет ложь. Но и нельзя сказать, что Астрал родил Человека. Это столь совместно, что нельзя разделить ни Человека, ни Астрала. Ну, что такое человек? Это та частица астрала, которая более-менее ещё реальна. Почему? Очень просто! Мир Астрала состоит из множества. Из множества элементов. Эти элементы могут быть и фэнтези — ваши фантазии. Это могу быть любые чувства, какие бы они не были. Представляете, какой Хаос твориться в этом мире, и когда вы собираетесь путешествовать по Астралу - какой нужен вам проводник!? Сколько людей, столько мыслей - столько и изменений в этом мире. А сколько в нем уровней. Множество! Вы можете попасть на самый низкий, где существует только,  всего лишь, одна темнота. Именно в этом мире ничего не было, лишь только Дух метался. И именно в этом мире впервые Дух - произнес Слово. И тогда родились новые. Родился Мир, в котором мы сейчас живём. Именно в этом Начале не было ничего. Была пустота и лишь только мятежный Дух носился и искал себе подобного и не найдя его  - создал! Произнес Слово своё, рожденное мечтами его, и появилась Земля. И дал ей имя — Земля. И сделал воды. И в Воды - жизнь поселил. И дал имя жизни — Рыба. И сделал животных он и дал им имя — Животные. Но не увидел себя в них и создал тогда Человека. И дал человеку право назвать свои Имена. Пришёл человек и каждую из рыб назвал, и каждому из животных Имя дал, и себя назвал и детей своих назвал…

(Ольга Васильева) *Один… 

(Контакт прерывается, идёт счёт)


(Шепот) -  Пришли к Нему и спросили - Вы создали Мир этот? И ответил Он: “ Нет, Мечты мои, Желания мои, Любовь моя!” Потому, в каждом из Нас, есть и желание и мечта и любовь. И Он, сделавши и сотворивши Нас, передал и боль свою. И потому мы болеем. Болеем, когда Незнаем, болеем, когда Хотим, болеем, когда Ищем. Имя этой болезни — Жизнь. И Он, мятежник, ушёл от нас, чтобы не мешать нам, чтобы были самостоятельными в счастье и беде своей. Но придёт время, и мы должны вернуться. Вернуться к нему и сказать — “Вот он я, сын твой!” И если узнает — Счастье! Если же скажет – “Нет, ты не мой! Ты чужой!” - уйдешь, и не будет Дома у тебя…  Мир Астрала — Мир любви и страха [тяжело выдыхая]. В нём нельзя быть без проводника. Можно очень легко заблудиться и не вернуться. Сколько множество людей было потеряно в нём, когда, желая выйти в него, терялся и, только тогда, потом, была одна дорога…И тогда, «сошедший с ума» или, как раньше говорили - «одержимые», - терялись в этом мире. Уже ушли из Астрала, но не пришли сюда. Нигде! Нигде! Многие, неумеющие прийти в него, старались ”химией” и любыми другими способами. Что делали сперва… Клали в воду и  топили… Топили до той степени, когда уже терял сознание. И тогда, вытаскивали его,  давали глоток воздуха, и он, умирающий, видел миры Астрала. И представьте, какие миры он мог видеть? Разве мог он увидеть высшие, если с какой печалью, с какой болью ложился он в эту воду?! Было и другое! Была даже наука, как ударить дубинкой, чтобы не убить, но всё же…..

(Ольга Васильева)* Один…

(Контакт прерывается, идёт сёт)


(Шепот) - Каждый из живущих…

(Гера) *Каждый из живущих [шепотом]

(Шепот)  - …держит уголёк. Уголёк в ладони каждого… горит и даёт жизнь. Другая же рука – прикрывает. Прикрывает уголёк тот, чтобы не загасить его, чтобы чужой не отнял его. И так  каждый из вас несёт этот уголёк по жизни. Бывает время, когда вы, от жадности, раскидываете руки и теряете уголёк этот. И тогда, где-то вдалеке от вас, он тлеет.  Вы же… Вы же - болеете и чахнете… Жадность убивает вас. Бывает и другое - когда уголёк ваш жгёт и мучает вас, и вы не можете понять, что происходит с вами - выбрасываете его! Потому, что так легче! Чтобы не мучиться! И это вы называете самоубийством. 

(Ольга Васильева) *Скажите кто Вы?

(Шепот) - И потому, нельзя с хранящими, но не по своей воле терявшими угольки, ложить их рядом. Есть и те, кто держат угольки, но протянув помощь, теряет и его. Теряют, и тоже, умирают. Можно ли назвать его, можно ли обвинить его в потере того уголька? Не-ет! Он сохранил чужой… Он сохранил чужой уголёк… Чужой уголёк, для него, дороже своего! Вот что называется Человеком! Вот… 

(Ольга Васильева) *Скажите… Один[счёт]

(Контакт прерывается, идёт счёт)

(Шепот) - Сперва - зубило.

(Гера) *Сперва - зубило.[шёпотом]

(Шёпот) – Сперва - зубило, сделанное из камня. Дерево и шкура. Потом, научились создавать бумагу и стали на ней писать. Но, прежде всего,  были рисунки, рассказывающие истории. Эти рисунки превращались в буквы. И, наконец, было до а….


(Контакт прерывается, идёт счёт)


(Гера) *Говорят[шепотом]

(Шепот)  - Говорят, что порядок это, всего лишь, -  одно из хаоса. Нет! Это неправда. Что такое Хаос!? Что такое Хаос…и есть ли он? Нет! Пока есть жизнь — нет Хаоса. Умрёт жизнь, когда ни одной души, ни одного огонька не останется — вот тогда будет Хаос. Только тогда. Всё остальное, только не понято и только.  Вы говорите “ энтропия”… А это значит, что постепенно, рано или поздно, угаснет жизнь. Физическая? Да! Она должна это сделать! Ну, представьте, как вы будете расти, если вы будете постоянно в одном и том же теле? Никак! Это не-воз-мож-но! Вы, сперва, потеряете одно, потом другое и дальше. Но, это всё должно происходить своим, и только своим (чередом. Прим.). Что происходит у вас? Вы,- первое,- раса «Асуров»… Что было с ней?

(Ольга Васильева)*Раса Асуров?

(Шепот) - Погибла! Она не имела человеческих тел. Не имела. Она более была похожа на тела насекомых. И вы могли бы быть ими и носить тела их. Но нет! 

(Ольга Васильева) *Это на луне[шепчет]

(Шепот) - Погибла! От чего? Да зазнались! Просто  зазнались! Сказали что - ” Раз мы Боги, значит, можем Всё”. И что? Не смогли остановиться! Пойдёмте далее… Раса - «Впередcмотрящих». Их называли так  - почему? Да потому, что они могли видеть Будущее. И что? Разбили самих себя! Почему? Потому, что они не могли понять, что Мир не может быть “одним”. Есть множество будущего. И все, смотрящие в будущее, увидели его разным. И распри начались, и стали меж собой разбираться, кто прав и кто нет и, не найдя согласия, убили себя! Осталось малое. Хорошо, что к этому времени уже зарождался Человек. Хорошо, что человеку можно было преподнести, что человеку можно было отдать Знания свои. И, хотя человек растерял их, пока стал, наконец, человеком, хоть что-то  осталось в нём. Пойдёмте далее…

(Ольга Васильева) *Раса.
(Девушка) *Раса…
(Ольга Васильева) *Один…


(Контакт прерывается, идёт счёт)


(Шепот) -  И Богов колесницы,

(Гера) *И богов колесницы.[шепчет]

(Шепот) - …и  Богов колесницы… давили людей. И ко…

(Ольга Васильева) *Один…



(Контакт прерывается, идёт счёт)


(Шепот)  - Увидевши звёзды.

(Гера) *Увидевши звёзды[шепчет]

(Шепот)  - Увидевши звёзды, что делаете вы? Что приносят вам звёзды эти? Печаль… Печаль, и ощущение сколь малы вы. Увидевши звёзды, хотите прийти к ним, хотите увидеть, что есть на них. Раньше, когда  были детьми , вы звёзды представляли по-иному. Вы не знали, что звезда - это Солнце. Вы не знали, что многие из звёзд - это планеты и на них есть Жизнь. Для вас была звезда - это "ЧТО-ТО", где были, и могут быть осуществлены ваши мечты. Теперь, вы выросли и узнали, что звезда, это всего лишь сгусток плазмы. Что планета, это всего лишь пустой никому ненужный кусок земли или материи… Как угодно. И там ничего не растёт, только вы здесь, на этой планете, живёте и всё. Теперь вы знаете, что такое Солнечные системы …  Да зачем вам это было знать!? 

(Ольга Васильева) *Скажите кто вы?

(Шепот)  - Зачем?! Чтобы потешить свой ум? И всё? И вы забыли, какими вы были детьми. Вы забыли, как вы летали к этим звёздам. Вспомните, когда вам снились звёзды, когда вы летели к ним… И, когда прилетали и гуляли среди них… те ли это были звёзды, что снятся сейчас? И снятся ли они вам?

(Ольга Васильева) *Кто Вы?

(Шепот) - Печаль… Печаль не в том, что вы тешите себя. Печаль в том, что вы не можете понять этого! Вам говорят - вы не слышите. Вы хотите узнать имя наше? Множество раз слышали его! Множество! И каждый раз отрицали. Множество раз Мы говорили в вас, останавливали вас. И что? Вы не слышали нас. Множество раз вы хотели и пытались спросить, что такое «Шёпот» и когда приходит он, вы не слышите нас…

(Ольга Васильева и Девушка)* Один…



(Контакт прерывается, идёт счёт)
(меняется интонация переводчика)


(Предохранители) - Ваша печаль - и снова нам работать… Вы узнали нас?

(Девушка) *Кто Вы?
(Ольга Васильева)* Да узнали, но вы представьтесь, мы, наверное, не представлялись.

(Предохранители) - Как же вы тогда узнали, если нам представиться?

(Девушка) *Узнали, узнали.
(Гера) *Предохранители?
(Девушка) *Да конечно!

(Предохранители) - Когда-то назвались и так…

(Гера) *А Вы не в курсе кто-кто-кто-кто.[заикается]вот с нами сейчас разговаривал?

(Предохранители) – О, нет! Давайте каждый будет за себя!

(Ольга Васильева) *Вот так!
(Девушка)* Ну, ладно. [кашлянула]

(Предохранители) - Ну, представьте, что мы сделаем? Сделаем ли мы им хорошую услугу, если Они не говорят.?

(Ольга Васильева) *Да.

(Предохранители) - А если быть точнее, они всё-таки сказали, но вы не очень-то и расслышали.

(Гера) *Ну, да.
(Ольга Васильева)* Мы? Да.

(Предохранители) - Потом, слушая кассеты, вы скажете – “Ах! Вот что!”  Что интересно, через год послушав, вы скажете —“ Ах нет, вот Это было! “ И Вы постоянно меняетесь, вы постоянно находите что-то новое. Постоянно. Благодаря чему? Благодаря тому, что вы забываете старое. У вас появляются новые версии, и вы не можете соединить их все вместе. Соберите все версии вместе, и тогда это будет верным. Только Это может быть верным. Вы же, строите маленькие-маленькие кусочки. Вы оторвали кусочек – “Ах, какой красивый, какой замечательный! Он мне подходит.”  Прошло время, этот кусочек уже надоел, вы его выкидываете и берёте новый. Начинаете рассматривать и хвалить его – “Ах Вот как это было! Как же я раньше не догадался?! Вот дурак-то я был такой! ”И что? До следующего раза! Да возьмите, соберите эти кусочки! Понимаете, что делаете вы? Вы разбрасываете Картину! Как вам её потом можно собрать, если эти кусочки вы разбросали в разные стороны и постарались их забыть? Да соберите вы их, и будет вам Картинка. Соберите всё, что было услышано, уведено вами и прочувствовано вами,  и тогда вы поймёте! Вы же, слушаете технику вашу, и не можете вспомнить, какие чувства были у вас во время использования этой техники…

(Девушка) *Ну, да.

(Предохранители) - Вы слушаете интонации, даже иногда восхищаетесь ими, но не можете повторить их сами. Почему? Почему!? Ведь каждый из вас может, может говорить, как Мы, как Они, как любой! А что сделаете вы? Что сделаете вы?  Вы поставите свои границы! Вы возьмёте, нарисуете свою картинку из наших кубичков, слЕпите что-то свое, поставите в свою рамку, поставите на видное место и скажете - Вот это так! Всем остальным вы кажете - это не верно! Правильно? Правильно!  Почему? Очень просто! Вы….

(Ольга Васильева и Девушка) *Один… 


(Контакт прерывается, идёт счёт)
(меняется интонация переводчика)

(Ольга Васильева)* Вот, мы сейчас, так, немножко поспорили между собой,  о шёпоте. Помните? Вы, конечно, всё помните. Вы закончили, мы - вернее, или не знаю, как… контакт закончился на вопросе: спросили у нас, что такое «шёпот». Ну, мы начали тут разные версии выдвигать об этом. Мы не забывали об этом, просто как-то другие вопросы возникали…И… не было времени, может, или чего, не знаю даже, спросить. И вот сейчас нам ответили, о «Шёпоте».

(Первые) - Всего лишь только часть!

(Ольга Васильева) *Ну, часть, да.

(Первые)-  Небольшая часть. Что такое шёпот!? Шёпот — это всё, что говорит в Вас. Именно в Вас, но не сознание Ваше. Шёпот - это когда Вы можете чувствовать, но не можете понять что это. Иначе, если бы поняли, то был бы уже не шёпот, а просто диалог, разговор. Шёпот… И почему именно шёпот? Всё очень просто. Он слишком слаб по сравнению с вашим сознанием. А если быть точнее, он слишком далек от Вас, у Вас слишком много одежд. Слишком много…И чем будет меньше одежд этих, тем больше Вы будете слышать этот шёпот…и тем разборчивее он будет для Вас. Шёпот - можно назвать любой, любой из диалогов произнесенный. Что такое шёпот? Это Ваша совесть. Это Ваши прошлые и будущие воспоминания. Шёпот - это всё, что не познано и не понято Вами, - вот что такое шёпот. Шёпот это то, что и делает вас Человеком. Ведь сознание имеет и машина. Тем более, вы уже научились строить такие машины. И они, вскорости, будут “выше” и умнее вас! Но они не могут слышать шёпот. Шёпот это есть то, что делает вас Человеком. Это и есть ваше - истинное человеческое. Это и есть тот Аоум, это и есть тот Дух божий, что был дан Вам. Вот, что такое “шёпот”. Почему же слаб он? Он не слаб! Вы глухи. Только и всего. Спрашивайте.

(Ольга Васильева) *А, вот, скажите…Вот,  как нам… Мы так поняли - в прошлом…В прошлом воплощении переводчик занимался спиритизмом, - вот как бы нам не скатиться до “этого дела”? Может, - такая опасность может быть? Или нет?

(Первые) - Что делаете вы? Сколько было множество, множество разговоров с вами. Сколько было множество повторов. Хорошо, если бы просто были повторы…Но они постоянны, постоянны в том, что вы не знаете и ни хотите знать о прошлом. Вы жалуетесь на память. А разве память виновата? Простите, когда в школе вас заставляли учить стихотворения, и когда вы приходили и не отвечали на урок, вам ставили плохие оценки, что, память виновата или лень ваша?

(Ольга Васильева) *Да, лень.
(Девушка) *Лень.

(Первые) - Теперь слушайте, что делаете вы - вы постоянно спрашиваете. Вы только набираете, вы только создаёте Количество. Когда же будет, в конце, Качество? И ,вообще - ваша теория “количество переходящее в качество” — это ложь! Это, всего лишь попытка оправдать себя. Попытка оправдать свои множественные ошибки. Да, конечно, в какой-то мере это верно, расшибив лоб можно всё-таки добрать до цели, но ведь можно было бы сделать это и короче, если сделать это с умом, а не просто тыкаться наугад. Сколько Множества было потеряно вами? Сколько множество мы затрагивали тем, и вы теряли их? Теряя только из-за того что Вы были “машиной”, только и всего. И сколько множества будет ещё потеряно. Из-за чего? Потому что Вы не умеете настраиваться. Вы начинаете…С чего начинаете Вы…?

(Ольга Васильева) *Когда приходим?

(Первые) - С чего начинаете вы? С чего?

(Ольга Васильева) *Когда приходим?
(Гера) *Со счёта.

(Первые) - Вы не можете избавиться от своих забот. И вы продолжаете заботы распространять здесь и на других. Пожалуйста - вы можете шутить, вы можете хохотать смеяться или плакать. Делайте что угодно. Не это важно, важно то, что вы приходите с ограничением времени. Вы уже знаете, что Вам будет скоро “пора” –  у вас есть слишком множество дел…И эти внутренние часы постоянно тикают, и этим тиканьем, забивают вас. Вы не можете избавиться от проблем, даже от тех, которые не нужны были  вам. Вы любите сочинять себе. Почему….

(Ольга Васильева и Девушка) *Один…

(Контакт прерывается, идёт счёт)
(меняется интонация переводчика)

(Безымянный) - Почему? Почему я живу, как собака!? Почему? Почему на мне одежда монаха, но после того когда он уже десять лет, извините за выражение,  делал с ней что угодно? И я, теперь, её ношу. Почему, беря лоскуток, я могу узнать о его хозяине? И почему… Почему я не могу раскрыться и должен держать это в себе? Почему? Да потому, что меня завалят всяким тряпьём и будут выяснять — “А скажи об этой тряпочке, а скажи о хозяине об этой, об этой, об этой…” И я только буду занимать тем, то что буду всем предсказывать чем вы болеете, когда вы наконец-то сдохнете, и в этом духе… Ну, почему…

(Безымянный) - Поче..му…

(Контакт прерывается, идёт счёт)
(меняется интонация переводчика)
(голос: мягкий, детский, радостный)


(Ольга Васильева) *А-а, ты! [улыбаясь]

(Мабу) - Подсматривал! [радостно улыбаясь]

(Ольга Васильева) *Ой, молодец, опять подсматривал!
(Девушка)[смеется]
(Ольга Васильева) *Ну. ты только и делаешь, что подсматриваешь, Мабуу. Ну, давай…

(Мабу) – Хитренькие!

(Ольга Васильева) *Мы?

(Мабу) – Я - хитренький!

(Ольга Васильева) *А-а-а-а… А мы? Не хитренькие, а?

(Мабу) - Если вы не подсматриваете, значит, нет.

(Ольга Васильева) *Чё-чё-чё?
(Девушка) [улыбаясь]* А мы тут подсматриваем откуда-то…
(Ольга Васильева) *А мы знаем всё про тебя!

(Мабу) - Чего “знаем”?

(Гера)* У тебя сколько жён?
(Ольга Васильева) *Всё про тебя знаем.
(Девушка) *Ага. У тебя сколько жён, скажи?
(Гера) *Сколько жён, скажи?

(Мабу) - Пять.

(Ольга Васильева)  *А-а-а!
(Гера)* О-о-о! Тогда, у тебя…
(Ольга Васильева)  *А мы знаем, когда у тебя семь будет…

(Мабу) - Когда? [удивленно]

(Ольга Васильева) *Скоро.

(Мабу) - У-у-у, “знаете”! Это и монахи мне говорят “скоро”. Это и я могу сказать "скоро"!

(Ольга Васильева) *Ой, ты какой! Ты гляди, какой ты умный парень, а! Ну.. ну, ладно. Ну, теперь рассказывай опять, что ты подглядывал, и что ты там увидел?

(Мабу) - Меня не побили! [радостно]

(Ольга Васильева) *Не побили?

(Мабу) - Не-а! 

(Ольга Васильева) [смеется].* Монахи?

(Мабу) - Да. 

(Ольга Васильева) *А ты куда подглядывал, ну-ка…

(Мабу) - Я не подглядывал, я «носик» делал! 

(Ольга Васильева)* А-а-а-а…
(Гера) *Монаху или себе?
(Ольга Васильева) *Монаху «носик» делал? 

(Мабу) - Да.

(Гера) *Ну и чего?
(Ольга Васильева)* А тебя похвалили, наоборот, да?

(Мабу)  - Нет, не хвалили.[обиженно]

(Гера) *Просто отпустили?

(Мабу) - Сказали, что так делать нельзя.

(Девушка) *Правильно!

(Мабу) - А я сказал : - Я же не его! Я – “время”.

(Девушка) [смеется]
(Ольга Васильева) *А! Время ударил. [улыбаясь]

(Мабу) - Они смеялись. И тогда сказали - хочешь увидеть Время?

(Гера) *Ну?

(Мабу)  - А я сказал — хочу!  - Только ты, вот, его не бей. Тогда дали.

(Гера) *Ну, и что?  Это как выглядит?

(Мабу) - Не интересно.

(Гера) *А что там?

(Мабу) - Это когда с одного… э-э-э…[подбирает слова] …ну вот, вода течёт! 

(Гера)* А-а-а…
(Ольга Васильева) *А, река. Река.
(Девушка) *Ааа..

(Мабу)  - Не-е-ет!

(Гера) *Песочные часы, наверное.
(Ольга Васильева) *Нет. Вода течёт. Река времени. Да? Нет?

(Мабу) - Нет! Вода течёт.

(Гера) *Ну?
(Ольга Васильева) *Течёт.

(Мабу) - Она течёт. А там, а там… чего-то нарисовано. И, вот, когда вода нарисованная. Э-э-э-э…[задумался]. Тогда, значит, чего-то наступило. 

(Ольга Васильева) *А-а-а!
(Гера) *Ясно.
(Ольга Васильева)* Часы. Часы такие, - водяные часы.

(Мабу) - Потом, воду дают нам. Она считается полезной.

(Ольга Васильева) *А-а-а…

(Мабу)  - Они дают нам её пить, вот. А ещё, после того, как я с вами разговариваю,  мне дают эту воду. И я… э-э-э…

(Ольга Васильева) *Что делаешь?

(Мабу) - Ма… Моюсь.
 
(Ольга Васильева) *Умываешся.
(Девушка)*”Моюсь”.

(Мабу) - Да!

(Ольга)*А! И как? Тебе приятно?

(Мабу) - Вода как вода. Чего это она будет  приятная или неприятная ? Холодная!

(Девушка) [смеется] *Холодная.
(Ольга Васильева) *А-а-а. А ты любишь умываться?

(Мабу) - Чего?

(Ольга Васильева) *Умываться любишь?
(Девушка) *Водой.

(Мабу) - Не-а.

(Ольга Васильева)* А почему?
(Девушка) [Смеется]

(Мабу) - Ну её… 

(Ольга Васильева) [кашлянула] *А когда ты ходишь, когда дождь, ноги у тебя грязные?

(Мабу) - Чего это у меня грязные?

(Девушка) *Когда вода течёт с неба, у тебя грязные ноги?
(Ольга Васильева)* Когда дождик.

(Мабу) - Э-э-э-э! Когда вода с неба течёт, я же не стою там!

(Девушки) *А-а-а-а!

(Мабу) - Пусть течёт, а я спрятался.

(Девушки) *А-а-а…!
(Девушка) *В пещере?
(Ольга Васильева) *В пещере, да?
(Девушка)* Ты не любишь, когда вода течёт?

(Мабу) - Нет.

(Девушка) *А почему?

(Мабу) - Холодная!

(Девушки) *А, холодная. И умываться ты не любишь, и воду ты с неба не любишь… Мыться ты, короче, не любишь.[улыбаясь]
(Ольга Васильева) *Монахи…

(Мабу) - Почему это мыться я не люблю? Люблю, но мало.

(Ольга Васильева)* Мало. Ясно.
(Девушка)* А-а-а…

(Мабу) - Да, когда грязный.

(Ольга Васильева) [смеется] 
(Девушка) [смеется]*Ясно.
(Ольга Васильева) *Мабу, а ты… А у тебя…

(Мабу) -  Мабу-у-у!

(Девушка) [смеется] 
(Ольга Васильева) *Мабу-у-бабу-у-у. Ну, Мабу-у!
(Гера) - Бабу-у…

(Ольга Васильева) *Мы тебя любим Мабу-у.[смеется]. Ты своих пять жён всех любишь одинаково? Тебе все нравятся? 
(Девушка) *Или кто-то…

(Мабу) - Я? Петя,[обращается к Гере] — ты всех жён любишь одинаково?

(Девушка) *У него только одна.
(Гера) *Нет, к сожалению, Мабу-у. 

(Мабу)  - Ну, вот. А почему ” к сожалению”?

(Гера) *А потому, что всех любить одинаково - наверное хорошо…Тогда никто не скажет…

(Мабу) - Ничего хорошего. Я пробовал.  Не получается.

(Гера) *Да!?
(Девушка) *Да? А почему не получается? [улыбаясь]
(Гера) *Ну, ладно.
(Ольга Васильева) *Петя…, ну, что ты не разговариваешь с ним?[говорит Гере]
(Гера) *Ну, а что у тебя получается? Скажи…

(Контакт прерывается, идёт счёт)
(начинается с конца предложения)

(Мабу) -Ауку.

(Ольга Васильева) *А-а-а! Во! Это какая? Старшая жена?

(Мабу) - Не…

(Гера) *Вторая?

 (Мабу) - Старшую не люблю. Она  дюже старая.

(Ольга Васильева) *А-а-а…
(Девушка) *А-а-а, ты молодую, наверное, любишь….
(Ольга Васильева)* Один…Два…

(Контакт прерывается, идёт счёт)


 (Мабу) - Я тебя, Петя, не обижал! [обиженно]

(Девушка) *А что он тебе сделал?
(Гера)*А что… А что, я тебя обидел? Чем? [Удивленно]

(Контакт прерывается)

(Ольга Васильева) *Мабу-у!
(Девушка) *Мабу-у!
(Ольга Васильева)* Мабу!?
(Демушка) *Мабу-у!!

(Контакт прерывается, идёт счёт)
(меняется интонация переводчика)

(Зеркало) - Почему вы не хотите разговаривать с нами?

(Ольга Васильева) *А кто Вы?
(Гера) *А…
(Девушка) *Кто Вы?

(Зеркало) - Мы? Те, кто начинали.

(Ольга Васильева) *Нет, ну, вы представьтесь.

(Зеркало) Мы уже здоровались с вами сегодня.

(Гера) *(…)Мы вам тоже пожелали…
(Ольга Васильева) *Ну, Мы хотим знать, вы здоровались? Ну, кто Вы?
(Девушка) *Скажите, кто Вы?

(Зеркало) - Вы не можете без имен? 

(Ольга Васильева)* Нет, ну, ведь…
(Девушка) *Ну, хотя бы что-то.
(Гера) *Обозначьте, хоть как-нибудь, себя!
(Девушка) *Не обязательно имя, просто…

(Зеркало) - Дать подсказку…

(Гера и девушка) *Ну, да.
(Девушка) *Хотя бы так.

(Зеркало) - А лучше - можно соврать?

(Гера и девушка)* А не важно.
(Девушка)* Ну, соврёте, это будут уже ваши проблемы…
(Гера) *Как соврёте,  так мы вас будем и называть..

(Зеркало) - Давайте скажем так…Более всего наш характер похож на одного из вас.

(Девушка) *На кого?
(Гера) *Наверное, на меня.

(Зеркало)  - Да, да! И мы имеем те же самые привычки, что имеете и Вы.

(Девушка) *А, “вон чего вышло”!

(Зеркало)  - А теперь, решайте, плохие мы или хорошие.

(Девушка) *Ну, ладно. Давай, поговори…
(Гера) *Конечно, хорошие.  Почти родные!
(Девушка) *Да, да![смеется]

(Зеркало) - Вот! Вот ещё один “минусик” в нашу и в вашу пользу.

(Девушка) *Он пошутил!

(Зеркало) - Шутка? Что такое “шутка”. Вы знаете, что такое шутка?

(Гера) *Кстати, что такое шутка, если серьёзно?

(Зеркало) - Если серьёзно? Так задайте этот вопрос серьёзно. 

(Девушка)* Что такое жизнь?

(Зеркало) - Вы же устраиваете спектакль, только и всего. Когда к вам приходит ребёнок, вы начинаете над ним смеяться. 

(Девушка) *Не всегда…

(Зеркало) - Не всегда ли?

(Девушка) *Нет. Не всегда…

(Зеркало) - Тогда, почему же, тогда вы обидели его? Почему?

(Девушка) *Кого? Мабу?

(Зеркало) – Потому, что вы считаете его глупцом по сравнению с вами. А теперь представьте, как должны тогда были-бы обидеться те, что с вами уже разговаривают? Вы можете это представить? Вам трудно. Потому, что вы не хотите себя глупцами назвать. Но Они не скажут вам это прямо. Почему? Да потому, что Они, опять же, жалеют вас. Это Они же  учили убивать других, лишь бы бедного переводчика поменьше били. Вот она, “доброта”. Такую “доброту” мы не признаём.

(Ольга Васильева)* Подождите, подождите… Ещё раз, пожалуйста. Что-то мы не поняли.
(Гера)* Что-то мы в этом вопросе  “плаваем” немножко.

(Зеркало) -  Плаваете?

(Ольга Васильева)* Они же учили…

(Зеркало) - Только что сегодня вы услышали… 

(Гера) *Ну… 

(Зеркало) - …о монахах..

(Гера) *Да…1648-й.. 

(Зеркало) - …что добывают, там, оружие…

(Девушки) *Да, да, да.

(Зеркало)  - …и били, чтобы пожалели те, говорящие. И вы теперь не поняли? Тогда, какой смысл? Сказать честно? Нам жалко, нам жалко тех, кто говорят с вами. Они бьются! Не удивительно, если они расшибут лбы об вас, и всё будет бестолку. Это не удивительно. Потому, что вы, как стояли - так и стоите. Вы хотите новые вопросы, а крутитесь вокруг да около и задаёте те же. Они же, видя ваши попытки, ваши потуги, стараются помочь вам — устраивают целые спектакли… Представьте, сколько времён им пришлось поднять!?

(Гера и Девушка) *Да.

(Зеркало)-  И что это им стоило!? И только для того, чтобы вы “ловили хи-хи”? 

(Гера) *Каюсь.

(Зеркало) – Понимаете, что делаете вы? Какова ваша несерьезность. Как вы готовитесь? Как? Уже говорили вам об этом. Нам повторяться? Или уже понято? Что делаете вы, к вам приходят, и первое, что вы задаете это Имя. Нужно ли Имя вам? Вам нужно не имя, потому что может прийти чёрт и назваться богом. Смотрите по делам его а не по именам. Это - первое. 

(Ольга Васильева) *Поняли.

(Зеркало) - Второе… Множество из того, как говорят и как делают Они - скажем честно - не согласны мы.  Мы считаем, добро должно быть с кулаками и должно отвечать.  Они признают – да, добро должно быть с кулаками. НО только кулаки эти надо пускать в ход не всегда. Может быть. Но, чаще, было бы легче их запустить. Вот через кулаки вы более доходчивы. Давайте, пойдём дальше…

(Ольга Васильева) *А Вы случайно не тот – “пятый”?

(Зеркало) - Не-ет! Зачем мы будем называться тем, кого ещё нет?

(Девушка) *Всё ясно.

(Зеркало) – И, если быть точнее, кого вы уже давно потеряли, вокруг да около крутились…И снова, и снова предали его и снова опять ищете. Вы представляете, сколько было поднято Времен? И во всех временах вы занимались тем-же, чем и сейчас.
 
(Гера) * Угу.

(Зеркало) - И ни в одном из этих времен не было толку. И тот обиженный ребенок дал больше, чем дали все остальные. 

(Девушка) *Это точно!

(Зеркало) - Почему? Да потому что ещё не научился врать. Он только начинал это уметь делать. И делал это столь глупо, что даже сами монахи, которым врал, просто смеялись над ним. А что делаете вы? Вы врёте постоянно! Вы врёте даже сегодня. Не однажды вы уже соврали. Давайте не будем обвинять, вы просто вдумайтесь в себя и заметьте это. Вы же, носящий характер наш, вдумайтесь более, что делаете вы. Вы начинали, и на вас ответственность лежит. Вы начинали, и вы должны вести. Что делаете вы? Вы пускаете всё на самотек и тут же начинаете хамить. То есть, близость вы начинаете принимать за вседозволенность…. 

(Контакт прерывается)
(начинается с конца предложения)

(Зеркало) - И саме страшноетрашное, самое страшное - это обидеть дитя. А вы не заметили этого? Вы, носящий характер наш. Мы можем назвать имена…

(Гера) *Да понятно…

(Зеркало) - В отличие от них, мы можем назвать имена. А почему? Потому, что Мы не столь сильны, и поэтому, нас не так слышно. Поэтому,  можем назвать имя.

(Девушка) *Назовите.

(Зеркало) - Назовут Они[Первые] — услышит Весь Мир.  Потому и  не зовут. Почему? Да потому что Они находятся вне мира Всего. А значит, любое их - замечаемо очень. Мы же живём в этом же мире и в хаосе, что твориться у вас, очень легко затеряться. Назвать Имя, любого из вас? Назвать имена, что носили вы раньше? Или назвать Имена, что будут дальше?

(Ольга Васильева) *Зачем?

(Зеркало) - Зачем? Действительно, зачем?

(Гера) *Честно, а зачем мы тогда хотим знать, кем мы были, кем будем…? 

(Зеркало) - А вы хотите знать? 

(Гера) * Ну, вообще так, люди, в общем.
(Девушка) * Я, вот, хочу знать…

(Зеркало) - Что хочет в вас знать? Всмотритесь в себя! Что именно хочет в вас знать? 

(Девушка) * Любопытство?

(Зеркало) - Чисто, голое любопытство! Разобраться в себе?

(Девушка) * Нет, не голое любопытство, нет…

(Зеркало) - Голое любопытство и не больше. Сознание ваше. Душой вы хотите знать? Душой вы не хотите знать, потому что душа уже и так всё знает! Она прожила всё это, зачем ей вспоминать-то? И до сих пор даже, в этом живёт. В том же ребёнке, и в том же не рождённом, и в будущем она живёт. Ей это не нужно знать, нужно вашему Сознанию, нужно вашему любопытству, чтобы подобрать ключик к душе. И чтобы, просто-напросто, превратить в одну из рабов. Вот что ваше Сознание, и вот, что - ваше любопытство, основанное чисто на Сознании. И чаще всего, чаще всего, именно это любопытство в вас. И потому, это любопытство и мешает всему. Когда же, действительно Душа хочет добиться до Сознания вашего, - она тоже вызывает любопытство. Чтобы сознание полюбопытствовало — а что же там было, и что же там будет!?  Да, это один из её приемов. Это, действительно так. Но, этот крик не превратиться в похабщину, не превратиться во что-нибудь плохое. Если же вы будете удовлетворять чисто собственное любопытство, к вам придут кто угодно, и пятый и десятый… Поверьте нам. Это чисто любопытство, чисто вашего сознания. Множество «низких» будут крутиться вокруг вас. И множество «Низких» будут стремиться поболтать с вами, чтобы просто-напросто дать вам свою Энергию, чужую, «приготовить плацдарм», или забрать у вас. Или, навешать вам “лапшу”. Вы понимаете, что делаете вы, чисто из голого любопытства? Чаще - делаете так. Есть у вас вопросы, множество вопросов, которые, действительно, заданы от души. Здесь мы не можем сказать ничего против…

(Зеркало) - Хорошо заданный вопрос не только хороший ответ, но и защита Вас от других. Все “чёрное” и “тёмное” не сможет прийти к вам, если действительно в вас говорит душа. Только душа может родить доброе. Сознание - не может! Она не знает, что такое – “Доброта”. Она только нарисовала фальшивую картинку и всё… Что такое Сознание дающее Доброту? Это просто попытка купить, только и всего. Вы скажете: “Не правда. Жестоко!” Жестоко! В том и смысл, что Мы хотим сказать Вам правду.  От ДУШИ вы творите добрые дела, не от Сознания. От ДУШИ вы любите, не от сознания. От ДУШИ всё Истинно — остальное ложно. И человек, прежде всего - ДУША, а всё остальное, всего лишь только “шкуры”. Это - просто “кусок мяса” и не больше. Не будь души,  это будет просто ”куском мяса” и не больше. А вы забываете об этой душе. О самом главном вы не помните, не хотите помнить, вы глушите, убиваете её в себе.

(Гера) *Скажите…

(Зеркало) - Зачем? Зачем Вам нужен этот порыв? Сознание подсказывает - «Не-е-ет, надо по-другому! Давайте так; давайте схитрим, давай обманем и так дальше…» Спрашивайте!

 (Гера) *Скажите, вот - а мне тогда вообще не понятно, это сознание присутствует только в этом физическом мире? 

(Зеркало) -  Нет. 

(Гера) *Или в других мирах есть свои уровни сознания?

(Зеркало) - Везде. Везде где есть материя, есть и сознание. Что такое сознание!?

(Девушка) *Да, что такое сознание?

(Зеркало) - И для чего оно вообще нужно? Если так послушать нас,  то можно сказать что сознание не нужно, зачем же нам такой достался “дар”? Оно нужно Вам! Что такое Сознание?  Это ваша физика! Что такое Дух? Это не физика! Что такое Бог? Это не Физика! Это то, именно истинное, духовное, не содержащее ни одной молекулы вашей физики! Так вот, этот вот Дух должен освоить, должен соединиться с Ма-те-ри-ей. 

(Гера) *Ага.

(Зеркало) - А что такое материя!? Представление материи  — только Ваше сознание! 

(Девушка) *Угу.

(Зеркало)  -  Потому и нужно вам Сознание.  Чтобы вы сознанием поняли, что такое Бог. Приходили к вам пророки и говорили о вашем Сознании. Вспомните, вспомните! Они говорили, не мы! И они говорили вам: “Осознайте! Уверуте! Ждите лучиной зажженной!” А что делаете вы? Ничего. То, что вы научились делать, так это просто разжигать костры и распинать. И это время ещё не прошло!  И до сих пор Вы занимаетесь этим! Я не говорю конкретно о вас. Я говорю о всех, о всех вас.  

(Ольга Васильева) *”Ты” говорил.

(Зеркало) - Да, я имею личность. Потому и говорю – “я”.” Я”, именно - “я”. И каждый из вас должен говорить “я”. Если вы скажете “мы”, имея себя и только себя в виду, это уже эгоизм, и не больше. Вы говорите – “Я люблю всех!” Да никого вы не любите, когда говорите так! Никого. Вы только сознанием говорите, произносите слова. Душой же вы никогда не сможете произнести этих слов вслух потому, что душа не имеет привычки хвастаться, и не имеет привычки “набивать” себе “цены”. Простите, за прямоту, но “вокруг да около” это не наш метод. 

(Ольга Васильева) *Мы даже не знаем… Конечно… Вы сказали правду. Тут уж мы, конечно, согласны.
(Девушка) *Прощать не за что.
(Ольга Васильева) *Ну, а с другой стороны…

(Зеркало) - Прощать не за что!? 

(Ольга Васильева) *Ну, а почему, простите, если вы сказали правду, то зачем прощать?

(Зеркало)  - Дорогие мои! Никто никогда бы с вами не разговаривал, если бы не за что было вас прощать! Вас всегда везде прощали. И единственное, о чём просим мы, чтобы вы учились прощать! Чтобы вы были прощаемы и умели прощать сами. Чтобы вы были счастливыми и делали счастливыми других! И никогда и никто, кем бы он ни был, не скажет вам, что вы…

(Гера)* Один…


(Контакт прерывается, идёт счёт)
(меняется интонация переводчика)

(Первые) - Простите, за их горячность, но, к сожалению, они правы. Но они наиболее ближе к вам и им больше накипело…

(Ольга Васильева) *Ну, да.

(Первые) - Мы слишком далеки от вас. Слишком! И, хотя вы были когда-то нашей “дорогой”… всё-таки мы тоже умеем забывать, и по этому не можем прочувствовать полностью. Они же, среди вас, потому, им больнее, потому, они там и не выдерживают. И мы хотим досказать за них. Никогда и никто не может обвинить вас Последними и Падшими. У вас всегда, всегда, пока у вас есть душа, у вас всегда есть шанс подняться. У вас всегда есть та ступенька, на которую вы могли бы подняться, оглянуться. Испугаться. А бывает, что приходится именно страхом, только страхом, чтобы вы испугались самих себя и перестали, перестали жить, как живёте. И потому…

(Девушки)* Один…
(Контакт прерывается, идёт счёт)
(меняется интонация переводчика)

(Зеркало) - Мы надеемся. Хотим надеяться, что вы, всё-таки, не будете в обиде на нас. 

(Гера) *Да какие обиды…

(Зеркало) - Почему? Почему мы говорим так жестко с вами? Устали уж от вас, устали… Хорошо если бы был проблеск какой. Надеялись бы. Да сколько же можно надеяться-то? Потому, мы приходим, кричим вам, толкаем вас, делаем больно вам, чтобы вы проснулись. Рано или поздно проснётесь, но хотелось, чтобы побыстрее бы уж. Уж, пожалуйста, постарайтесь. А пока… Потом мы будем только слушать вас и не будем больше вмешиваться вам…. Ве…

(Гера) *Один, два…

(Контакт прерывается, идёт счёт)
(меняется интонация переводчика)

(Мабу) - Ка-акой он зло-ой! Он так крича-а-ал!

(Девушки)* Кто? Кто?

(Мабу) - Он так кричал!

(Девушки и Гера) *Кто?

(Мабу) - Я напугался-я-я!!! 

(Ольга Васильева) [смеется] *Ты опять подсматривал?

(Мабу) - Чуть-чуть.

(девушка и Гера) *А кто это был?

(Мабу) - Не знаю! [удивленно] 

(Гера) *Ну, ты видел его? 

(Мабу) - Нет. Я глаза закрыл!

(Девушка)* Глаза. Ага.
(Ольга Васильева)* От страха?

(Мабу) - Да.

(Гера) *Мабу!

(Мабу) - Чего? 

(Гера) *Ты это… Ты меня извини там. Я тебе пошутил. Назвал другое имя твоё… 

(Мабу) - Чего? [удивленно].

(Девушка) *Он не помнит.
(Гера) *Чего, ну, ты же обиделся на меня… 

(Мабу) - Чего это я обиделся?

(Девушка)* Он не помнит.

(Мабу) - Чего это я не помню?

(Девушка) *Ты, вроде, на Петю обиделся. Он что-то сказал такое, а может даже не сказал, и…Может подумал..
(Гера) *Да.

(Мабу) - Не-ет.

(Девушка) *Нет? А что?

(Мабу) - Это, наверно, ещё бу-удет!

(Гера) *А-а-а…
(Девушка) *Ух ты, какой умный!

(Мабу)  - Это, может, я когда-нибудь обижусь. А ещё - я не обижался, а мне уже объясняли!

(Гера)* Что?
(Ольга Васильева) *Кто?

(Мабу) - Мне уже говорили! А вот эти… Злые которые.

(Ольга Васильева) *А чего они объясняли?

(Мабу) - Они мне говорили что… И, тут я не понял, как это так?  Что я могу раньше, могу позже… 

(Ольга Васильева) *Да?

(Мабу) - А вы хитренькие! Стали теперь узнавать, сколько у меня жён.

(Девушка) *[Смеется]

(Мабу) – И, тогда, вы уже можете уже сказать, Раньше я или Позже. 

(Девушка)* Ааа…

(Мабу) - Они тут все рассказывают. Да. Вот. А теперь…

(Группа) - Один, два..

(Контакт прерывается, идёт счёт)

(Мабу) - Что Вы сейчас делали?

(Гера) * Сейчас?
(Девушки)* Считали. Мы считать учимся.

(Мабу) - Да? Я столько не умею…

(Ольга Васильева)* Столько не умеешь, как мы, да?

(Мабу) - Нет.

(Девушка) *А до скольки ты умеешь?

(Мабу) - Я? Семь.

(Девушка) *Семь? Ну!
(Ольга Васильева) *Ну-ка, посчитай, посчитай, давай! Посчитай…

(Мабу) - Один. Два. Три. Ещё один.

(Девушка) *Нет! [смеется]

(Мабу) - Ещё два, ещё три…

(Ольга Васильева)* Ну…

(Мабу) - И ещё один.

(Ольга Васильева) *О!
(Девушка) *И будет семь, да?!

(Мабу) - Да.

(Ольга Васильева)* Молодец! Научился до семи, даже, считать.

(Мабу) - Да.

(Ольга Васильева) *Ну… А когда научишься до девяти считать, как мы?

(Мабу) - Не-е-е…

(Ольга Васильева)* Что –“Не”?
(Девушка)* Не научили.

(Мабу) – Сейчас, подумаю.

(Ольга Васильева) *А… ну, подумай, да.

(Мабу) - Нужно будет камушки читать.

(Ольга Васильева) *Ты на камушках считаешь? Ну!

(Мабу) - Один, Два, Три. Ещё один, Два, три… Ещё один… Ещё… Ещё два.

(Девушка)* О-о-о! Что-то много.
(Ольга Васильева) *Ну, и сколько вместе-то будет?

(Мабу) - Ещё… Э-э-э-э… Ещё один.

(Ольга Васильева)* Молодец!

(Мабу) - И будет девять!

(Ольга Васильева)* Правильно!

(Мабу) - Да.

(Ольга Васильева) *О! Молодец какой ты, а!
(Девушка)* Один…

(Контакт прерывается, Конец записи)
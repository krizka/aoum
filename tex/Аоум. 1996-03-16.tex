Аоум. глава 33-я 16. 03. 1996г
Георгий Губин
  
 
Аоум. глава 31-я
Контакт 16.03.96
 Кассета Сторона ``А''
 Аннотация: Запись контакта начинается с конца предложения.
\soul{Вы можете перекреститься, матерясь, а можете перекреститься,  желая истинно добра. Можете перекреститься, вспоминая бога, а можете перекреститься, чтобы сбылось проклятие.}
\people{Кстати, про жесты, если вот так вот. А вот, мы, обычно, накладываем крестные знамения. Вот — лоб, сердце и плечи. А если вот в некоторой, значит, литературе, написано,  что истинное крестное знамение должно быть — лоб, сердце и глаза. Как вы можете на этот счет сказать, что?}
\soul{Идут и спорят,  кто искренен. Лучше бы вы спорили, как искренно молиться. Лучше бы вы спорили, как искренно молиться, а не как делать это.}
\people{А ну… Ну, ведь крестное знамение, для многих, играет большую роль.}
\soul{Тогда, все те, кто не крестятся, будут гореть в аду? Вы устраиваете войны только из-за того, что не поделили, как креститься. }
\people{Это точно.}
\soul{И это вы называете верой в бога? Вы взяли в моду себя распинать. Чтобы почувствовать себя Богом, Христом? Зачем вы лезете на кресты? Неужели вы думаете что, повисев на нём, вы станете святым, и Христос узнает вас? Это, просто, ещё одна роль, и не больше. }
\people{Ну, да. Мученика.}
\soul{И какие мучения принимаете вы? Физические? А Духовные? Вы знаете, что сейчас Вас восхваляют! И Вы знаете, что не будет смерти, и вас снимут. И вы не оскорблены своей любовью. И что, - обезьяна подражающая? И не больше. Простите если бы вы… Вы….}
\people{1, 2, 3, 4, 5..}
\soul{Спрашивайте. }
\people{А можно продолжить? }
\people{Это уже другие. }
\people{Это другие. }
\soul{А Вы подумайте сами! Подумайте сами, что делаете вы? Вы хотите сыграть роль Христа, распинаете себя. И знаете, притом, что не будет смерти, что будет слава[вздыхая]. Слава. Не боль. И, да, вы говорите: ``Cвершилось!'', также, как сказал Христос. Но, что вы  подразумеваете, когда говорите это? ``Свершилось! Наконец-то, сбылась мечта моя. Повисеть. Побывать им. Наконец-то, будут восхвалять меня, показывать пальцем, говорить: ``Вот Он, мученик Христовый”''! Да как Вы можете говорить о вере, если вы мучаете плоть? Плоть, которую дал вам тот же бог. Вы говорите о вере в бога, а сами мучаете её, и этим мучаете и бога. Неужели вы думаете, что можно прийти к богу только через физическую боль?  Вообще, зачем нужна боль, любая боль? Это придумано только Вами, и не больше. Чем ``больнее'', тем, значит, урок будет ``крепче''. А нельзя ли было бы проще?}
\people{Как, например?}
\soul{Как? Ну, вот и ищите!  }
\people{Ну, хоть один пример какой-нибудь…}
\soul{Пример?}
\people{Ну, из жизни, может быть, чьей-то. Вот тех, которые действительно были истинными святыми.}
\soul{И что? У Вас уже был один пример - Распятый Христос! И какой Вы сделали урок? Ходите, распинаете теперь себя! Что вы приняли здесь? Вот вам пример. И скоро, это у вас войдёт в моду. Вот что самое страшное. Да, вы скажете: ``Истинно верую, и хочу, пусть ошибаюсь, но хочу испытать чувства! Чувства и боль Христа''.  Да можете ли Вы испытать её? Прежде всего, вы должны быть Христом, чтобы испытать то же. Прежде всего, вы должны быть изгоем всех! А вы - изгой? Если толпа восторженных поднимает вас на крест и кричит вам ``Виват''! Вы - изгой? }
\people{Да вряд… Не то что, вряд ли,-  а точно, что нет. Потому-то…}
\soul{О какой тогда можно говорить…(искренности. прим.)?}
 (Обрыв кассеты)
\people{Ну, вот, прошлые… На прошлом контакте, переводчик… Вот было такое ощущение, что он как бы связывал нити. Действительно, это вот, связывал -  может, это чисто символически, конечно, такое?}
\soul{Да. Он связывал.  Но, что он связывал?  Вы не можете знать. Не знаем и мы.}
\people{Да? Ну, нам так показалось, что сначала нить была, как бы вот… Ну, как бы вот, от нас. То есть, вот мы сидим, вдвоём были,  и он, как бы, от нас.}
\soul{Поймите же. Поймите. Мы не входим в вас. Иначе - вы были бы роботами, машинами, говорящими, что хотим мы. Мы, просто, примитивно, но - будет точнее - рисуем картинки. Рисуем картинки вашему мозгу. А мозг уже дорисовывает их сам. Верно, или неверно… Если не верно, - мы стараемся нарисовать другую… И, тогда, получаются задержки, вы даёте счёты.}
\people{М-м-м… Вот почему…}
\soul{Но, помните, он [переводчик] будет говорить. Он будет говорить, и , хотя у него словарный запас гораздо больше, чем пользуется… Почему? Потому что мы не затрагиваем глубины мозга. Иначе, тогда мы были бы посторонними для него. Вспомните первый контакт, когда мозг ставил преграду. Вы, помните?}
\people{Да, да.}
\soul{Потому что мы хотели от жадности прийти, да поглубже, да побольше.}
\people{Ну, самое главное вы не с плохими же мыслями хотели это.}
\soul{Да какая разница, если не умеючи? }
\people{Как? }
\people{Жадность.  Просто,  жадность. Да.}
\people{А Вы же сами говорили…}
\soul{Если не умеючи - это не оправдание.}
\people{А вы же…}
\soul{Всегда и везде нужно быть во всем осторожным. Всегда, если не знаете, и какие бы намерения не были у вас, честными или нет, но, если вы делаете наскоком,  это всего лишь только вред, и не больше. Вот вам, одно из расшифровок — ``с благими намерениями путь в ад вымощен''. }
\people{1, 2…[обрыв записи]}
 (Задержка, обрывается контакт, счет)
\people{Восемь. Скажите, а почему Вы говорите, то – ``Спрашивайте''? то -  ``Говорите''?}
\people{Да какая разница?[шепотом ]}
\people{Можно так? Мне любопытно просто.. }
\people{Это Он говорит. [шепотом ]}
\people{Это переводчик так переводит?}
\soul{Нет, зачем же? Нам приходится, нам приходится говорить так.  Ибо, вы уже объявлены ``дичью''. Потому и будем ``скакать''.}
\people{М…мы уже ``дичь'' для кого-то? Все ясно, чтобы нас, вроде бы, не поймали.}
\soul{Уже не раз говорили Вам. И не раз говорили об именах. Не раз говорили вам о настрое. А что делаете вы? Вы, тут же спрашиваете - ``Кто вы?'' И, тут же, называете имена.  Потом, говорите и ссылаетесь, что не знали. Вы говорите, что это  название вируса, название города, название местности… Да какая разница? Достаточно тех имен, которые уже произнесены. И много. Любое слово - это Имя. Любое! Любой звук,  это уже Имя. Уже одно это должно настораживать. Уже одно это должно вас убедить. Пожалуйста, соединитесь, сойдитесь. Сойдитесь, так чтобы можно было говорить без слов. И тогда, уже не будем бросать на ветер. И не будем разбрасывать Следы.}
\people{Ну, скажите, раз мир един и в каждом бог, то в… }
\soul{А Вы знаете об этом? Вы только кричите, что мир един, и не ощущаете того.}
\people{Нет, это… Вы нам говорили,что мир един, то, в принципе, как он  может…  }
\soul{А как мы можем прийти к вам? Силой?}
\people{Нет, конечно.}
\soul{Только силой вы приходите. Только вы, и не больше. Бог к вам не придёт силой, он не будет ломаться в двери ваши. Он не будет пугать вас. Вы очеловечили его, Вы дали ему все те качества, все характеры. И у каждого бог зависит от того, какой характер у него самого. Если он сам жаждет ненависти, - и бог будет ненавидеть, и будет карать. Если же он умеет прощать - и бог будет прощаем. А вы?  А вы - нашли классическую картинку и начинаете её подстраивать под себя. Рисуете, как вам будет выгодно. Потому и говорят вам, что у каждого свой бог.  И вот, и каждый попадёт в свой ад, который сам же и нарисовал. Вы тут же спросите – ``А как же не верящий,  который не верит ни в что, ни в рай ни в ад? А значит, они не будут ни там, ни там?'' Вот это самое страшное, что вы окажитесь в пустоте. Окажитесь один на один - Вы и Вы. И больше никого.  И вот -  истинное мучение.}
\people{Скажите, а это не тогда ли мятежный дух витал, так сказать, над землей и родил этот мир, чтобы не быть одному. Это не тот случай?}
\soul{Не будьте многословным. Множество слов произносите вы, но они пусты. Почему? Потому что вам нравиться само построение слов. Вам нравиться, как они звучат. Вам нравиться их резонанс, физический резонанс. У вас есть множество, множество молитв, а они не помогают. Хотя, звучат и любимы вами. Почему? Потому что вам нравится лишь только звук. Вы не можете прожить, Вы не можете соединиться с этой молитвой. И потому,  ничего не приносит вам. Зато вы можете прекрасно, прекрасно прийти в церковь и, увидев рассерженного попа, прекрасно тут же обвинить и бога. Почему так?  Почему? Почему один человек может разрушить веру, которая копилась у вас веками и в одно мгновение может это все рассыпать и разрушить. Почему? Вот в чём страх, вот чего боится каждый. Вот проявление силы, но не разумной. И эта сила может или погубить вас или поднять на столько высокое. Но вы не знаете, как это сделать. И никто вас этому никогда не научит, и никакие учебники не заставят вас. Никаких инструкций никогда вы не найдёте. Никогда. Вы должны почувствовать эту силу в себе и не отделять её. Не говорить, что это часть ваша. А вы сейчас даже не видите её, не говоря даже о том, чтобы её чувствовать. Вы должны….}
\people{Один}
 (Задержка, обрывается контакт, счёт)
\people{Мы, однажды, говорили о расах. Вот, раса Асуров погибла, раса Впередсмотрящих погибла. Можно продолжить?}
\soul{Ничто не погибает.}
\people{Нет, ну не погибла… Ну, она… как… Ну, более не… Уничтожила себя, как вы сказали.}
\soul{Ничто никогда не погибает. Если Вы хотите сказать о физическом уровне, так и скажите.}
\people{Ну, хорошо, будем говорить… А как ``идея'' они, значит, разбились по миру или как?  Витают где-то? Как сама ``идея''.}
\soul{Среди вас, множество всех. Говорили же вам: и ``первые'' и ``шестые''.}
\people{Ну, да. Да, да, это…Ага. То есть, это, мы , будем говорить -  сейчас, вот - человечество. В человечестве присутствуют все расы?  И те, которые были, и те, которые будут, может. Ну, будут … Ну, как эти…}
\soul{Нет, которые были.}
\people{Которые были ,только в прошлом. Ага. А которые будут, строятся здесь уже из тех, которые были?}
\soul{Следующая раса уже будет включать и вас.}
\people{Угу. Скажите, а вот если прошлое, то есть вот, человек прошёл… ну, не человек, ну, не важно. Мы, короче, - человечество, - прошли отрезок времени, - прошлое это, для нас. И оно было однозначным как бы. Да? Ну, то есть мы, как бы, его уже прошли. А оно, по идее, то есть…}
\soul{Нет.}
\people{Нет?}
\soul{Оно не может быть однозначным. Оно никогда не может  таким быть до тех пор, пока вы ещё говорите только словами. Вы всегда будете находить ложные пути, ложные ветви. Вы всегда будете возвращаться в ложное прошлое, которые не существовало, и которые вы создали сами. И вот тогда вы будете говорить о ``параллельности миров''. К сожалению. К сожалению, ваш путь похож на ломаную линию. Часто, она, если хотите…}
\people{Один…}
 (Задержка, обрывается контакт, счёт)
\soul{Спрашивайте.}
\people{Ну, продолжите по поводу нашего пути, похожего на ломаную линию больше чем…}
\soul{Да вы похожи на бегущего прячущегося зайца. Вот ваш маршрут.}
\people{А… Бежим в противоположную сторону? Туда-сюда бегаем.}
\soul{Именно бегущего, от страха, зайца. Потому, что вы боитесь своего прошлого. Вы не хотите себе в нём признаться. И вы тогда сочиняете множество оправданий: - ``Да это было не так, это было по-другому'', ``да вы меня не поняли'', ``да я хотел не то сказать''. Вот вам и петли заячьи, а вот вам и новые направления. И вот - вас уже ``множество''. Множество ложных. }
\people{Ага, ясно.}
\soul{Да когда ж вы сойдетесь!? Вы не можете толпой самого себя войти в узкие врата рая! Не можете.}
\people{Короче, мы придумываем много для себя. Сами себя оправдываем ,в какой-то мере, наверное. И вот, потом…}
\soul{В какой-то мере? Да вы всегда это делаете. Всегда.}
\people{Это немножко прояснилось вообще-то, по этому поводу. А будущее… В принципе, вот, когда начало было, ведь путь был, то есть, было несколько… Вы говорили ``Вперед смотрящие''… }
\soul{Было и множество начал. По той же причине.}
\people{По той же причине? Даже так. Ну, это, конечно, уже труднее, это понятно, когда очень тяжело что-то…}
\soul{Это только вы говорите, что была теория Большого взрыва и в Начале было только начало.}
\people{А на самом деле этого не было?  Это продолжение, конечно…}
\soul{Было множество, множество различных начал.}
\people{Скажите, вот, мы уже спрашивали, вообще-то. Ну, вот,  многие авторы, особенно автор ``Тайной доктрины'',  допустим, и  ``Агни-йоги'' утверждают, что Луна — это мать Земли. Ну, правда, мы уже задавали этот вопрос. Ну, вот, ещё раз… }
\soul{А мы вам  уже отвечали. }
\people{Да…  Правильно. А вот… }
\people{Не было бы Луны - Земля ушла с орбиты. Земля ушла б с орбиты - не было бы этой физической жизни, допустим. Были бы другие условия. Верно?}
\people{А тогда…}
\soul{Не очень.}
\people{Ну, на физическом плане, верно?}
\soul{Говорили вам о щите, говорили… }
\people{Угу. А, тогда, скажите…}
\soul{И говорили, что придёт время, когда уйдёт этот щит, и тогда вы уже более откроетесь. И тогда вам будет сложнее, потому, что вас будет легче, легче уничтожить, но и зато тогда вы будете больше видеть. А там уж, извините, у кого сколько сил. Каковы направления и желания.}
\people{Ну, а вообще-то, куда все, вот, стремятся, так сказать?}
\soul{А куда Вы стремитесь? Куда лично стремитесь Вы? }
\people{Не знаю. Мне, например, хочется понять вообще суть жизни, что ли.  Или, как… И, вообще, зачем я вот, живу здесь, и вообще, кто я такая?}
\soul{И для чего?}
\people{Ну, как ``для чего''?  Для того, чтобы жить дальше!}
\soul{Да? А Вы что, не живёте дальше?}
\people{Да не то, чтобы дальше. Нет, не жить… Нет, мы живём. Но, дело в том, чтобы понять самую глубинную суть - смысл жизни. Не ``просто так'' жить.}
\soul{Так для того и живёте, чтобы понять глубинную суть.  }
\people{Ну вот, по этому, и цель наша такая.}
\soul{А вы хотите найти её в книжках, найти её в знаниях и, желательно, готовыми и уже разжеванными. Неплохо было б, если бы ещё и проглотили.}
\people{По крайней мере, книги, это, тоже писали люди не глупые. Там тоже очень много есть… Ну, как-то у нас вот…}
\soul{А мы говорили о глупости? Мы говорили о глупости?}
\people{Да нет, конечно.  Вы правы, что в большинстве, конечно, мы своём вот вцепимся за книги за свои какие-то там. Ищем ответы на вопросы.}
\soul{И ищете ответы подтверждающие Ваши, и не больше. А всё остальное - вы скажете: ``Ложны''.}
\people{Да не всегда, конечно, так. Но, дело в том, что ещё человек, может  так устроен. Я вот, во всяком случае, не буду говорить о других, а я вот о себе скажу что… Просто, иногда и не знаешь, права ты, не права. Другие люди как-то вот так поступают, а ты по-другому поступаешь. Их множество… Ну, проявлений. Проявления-то разные. У всех разные проявления. }
\soul{Часто ``Крик души'' Вы путаете с ``Криком желания''. И тогда Вы будете уже говорить: ``Я чувствовал, что надо было сделать так''. И вы подозреваете интуицию, ясновидение. А это было, всего лишь, ``Крик желания'', или ещё что-нибудь. Маски. Вы ещё не научились. Вы ещё  не научились, к сожалению, разбираться в себе. Вы не знаете начало Ваших мыслей  - отчего родилась именно эта мысль, а не другая. Это вы не умеете. Вы не можете произнести свою мысль. И на кончике языка сочиняете множеств её формулировок, и,  наконец-то произносите одну из них и всё равно не довольны. А если ещё приходиться выводить на кончик пера — ещё сложнее. Почему? Почему? Потому, что мы…}
\people{Один, два…}
 (Задержка, обрывается контакт, счет)
\soul{Если хотите, то назовусь Васей. А дальше? Почему Вы молчите?}
\people{Васей? [улыбаясь]}
\soul{Что было дальше?}
\people{Дальше было…}
\soul{Тогда - я Петя! [молчание] …Забыли?}
\people{Да.  Да, нет… Мы, просто, растерялись..}
\soul{Почему вы промолчали?}
\people{Мы, просто, растерялись.., Э… Думали, он там с кем-то сам говорит…}
\soul{Вот вам и память!}
\people{Да нет.}
\soul{А Вы даже думали? А где тогда был ``Крик души''?}
\people{Да! Правы, правы.}
\soul{Неужели вы ничего не почувствовали?}
\people{Почувствовали!}
\people{Почувствовали, конечно.}
\soul{Вы догадались?}
\people{Догадались.}
\soul{Вы вспомнили? }
\people{Да.}
\soul{Но не вспомнили. А сколько много того - ``не вспомненного''? Сколько много желанного и не проявленного? А сколько много проявлено то, что не хотелось бы вам делать вообще. Чаще, - вот Вам роль актера. Чаще -  вы выполняете то, что нужно окружающим, но не себе.}
\people{Да, да…}
\soul{Вы играете! И играете довольно-то успешно. И уже то, что было другому нужно, вы считаете, что это было нужно и вам. А потом ещё  скажете о памяти.  Что прошло много времени,  уже будете доказывать, что это — ``Я! Это мне интуиция подсказала!'' И что? А что делать бедной интуиции, которая в это время кричала совершенно другое?}
\people{Да, спасибо, вообще-то. Вы нам хороший урок преподнесли сейчас. Интересно, ну, тогда это, получается,  если жить и делать то, что ты хочешь, и, так сказать…*  А в этом и есть Жизнь, между прочим. Поэтому и мы…}
\people{Не, ну в этом я согласен. Это правдивая жизнь, так сказать. По крайней мере, не лжёшь сам…}
\soul{А вы смотрите, что Вы хотите. А что Вы хотите? Относительно окружающей среды. Вы хотите приспособиться к окружающей среде. Вы говорите о потребности, именно о желании - ``Истинном желании''. А что ``Истинное желание''? Откуда Вы знаете, что истинно в Вас? Истинно  это желание или другое? Вы хотите хорошо жить. Без бедности. Вы хотите этого! Да, конечно, вы тут же скажете, что это было бы гораздо легче  жить духовно,- ``у меня было бы больше времени'', и так далее, так далее… Не было никаких различных ссор. Не надо было бы подстраиваться. А откуда родилось это желание? Извне! Не в Вас, а извне. Другие. Другие, нищие, кричали вам: ``Вот, если будешь Богатый''… А вы уже считаете, что это Ваше желание.}
\people{Да…}
\soul{И, когда Вы говорите: ``Я правдив и всегда говорю только правду и правду. Я никогда не лгу''. Что? Вы правдивы? Да, Вы говорите правду. Конечно! Но было ли это Вашем желанием, или, опять же, - извне? Где Ваше истинное, рожденное именно Вами! Где? Всё извне… Всё.}
\people{Но, мы же, как человечество, в принципе-то, действительно - зависим друг от друга. Мы же вместе…}
\soul{Вот именно, что зависите, а не живёте! А не просто, живёте дружной семьей.}
\people{Ах, вон как.  Ну, иногда….}
\soul{А Вы, как на базаре. Вся ваша жизнь, это базар. Подешевле купить, и подороже себя продать. Не так?}
\people{Да..[вздыхает]}
\soul{О какой семье тогда говорите Вы? Если… }
\people{Точно.}
\soul{…если у вас только деловые связи. Даже когда идёт речь о любви. А истинна ли она? Родилась ли  она в Вас? Чаще - нет! Чаще, вы подстраиваетесь. Если любимый хочет от вас услышать слово ``Люблю'' или какое-нибудь ещё` наименование, то вы произнесёте его, чтобы  было ``Любимый!'' несущий  это наименование - чтобы было легче.}
\people{Ну, а если не произнесём? Не смотря на то, что…}
\soul{Ну и что же? Начинаются драмы. И опять  - рожденные ``извне''. А вы,  ваше ``истинное Я'' -  спит в это время, спит,  и не больше. И только ``шкуры'' ваши, ваши одежда, ваша физика - начинает играть роль человека. ``Да, я люблю'', или ``да, я ненавижу'' И, всего лишь, только дань окружающему, и не больше. Всего лишь только желание, всего лишь только инстинкт. Инстинкт, и не больше, чтобы выжить в этом мире. А вы говорите о ``Высших'' чувствах. А где они Ваши Высшие чувства? Рожденные в вас - высшие чувства? Да вы их никогда словами то не произнесёте! Вы не найдёте им слов. Вы будете говорить – ``Я не знаю, что со мной твориться.''  Чаще всего, вы говорите так. Хотя, и здесь есть обман. Но реже. Вот когда вы не знаете что делать, когда вы не знаете, что вы хотите, что вы ищете,  зачем вы живете…Что вы вообще здесь делаете… Вы не хотите ни спать, вы не хотите работать, но не хотите лениться. Вы не хотите ничего. Вы хотите писать и не хотите. Одновременно вы хотите, и нет. Вот! Вот тогда Ваша физика растерялась и потеряла, забыла свою роль! К сожалению, это бывает так редко. И хорошо, что бывает редко! Потому, что физика в эти мгновения делает[тяжело выдыхая] - свой вывод. Она решает, что ``пора с этим кончать, всё надоело.'' И тогда, пожалуйста вам — самоубийца. Это что? Душа хотела? Душа хотела убить себя, зная, что она бессмертна? Или, может просто, она нема? Или вы глухи? Что будет точнее? }
\people{Мы глухие, наверное.  Ну, хорошо, тогда где выход, из этого положения?}
\soul{Выход?}
\people{Да. Где же тогда выход, если  мы, выходит, ничего не слышим?}
\soul{А вы ищите! Вы ищите! Говорят же вам, никто, никто никогда не сможет прожить за Вас! Пришёл к вам бог -  вы его распяли. Вы ему не поверили! ЕМУ не поверили! Так ни кому не поверите! Себе не верите! Если вы ``не верите себе'' разве вы будете верить другим? Да найдите себя! Себя в себе. Тогда вы найдёте и бога там же. А вы? А вы,  приходите в ``Святые места'', и, делая различные физические упражнения и произнося длинные монологи, называемые молитвами, - хотите увидеть бога. Почувствовать в себе. И, при этом… И, при этом, не зная себя. Вы себя потеряли…  Вы найдите, прежде, себя, а потом уже ищите бога. А вы же, вы же так глубоко потерялись, так сильно заросли! И хотите, чтобы кто-то пришёл и стал снимать за вас одежды? Или стал подсказывать, как снимать их? Да не получится того. Не полу-чить-ся…  Потому, что это будет тогда — ещё одна роль, и не больше, не больше… Истинное ``Я'' - ``играть'' не будет. На то оно и Истинное!}
 Кассета Сторона ``Б''
 Аннотация: Запись контакта начинается с конца предложения.
\people{…астральное тело, если можно так назвать человека, когда оно будет… Ну, я не говорю, что сейчас это возможно, когда-нибудь, будет ``прозрачным'', если так сказать. То есть,  чистым. Или не будет руководить, так человеком, что ли, если так говорить - физически.}
\soul{И опять, Вы говорите о физике, тут же говорите…}
\people{Нет, но…}
\soul{Тут же говорите:- `` а вот когда физика станет прозрачной?''  Ваш воздух прозрачен. Когда не будет управлять ``нами''? А кем, ``вами”-то? Да вы найдите себя! Вы даже не знаете, о ком говорите! Кем управляют ваши тела? Кем? Вы не знаете этого, а спрашиваете. Вы найдите себя. Найдя себя, вы уже будете знать, что вами управляет. И что вы -  просто марионетки. И, тогда, вы уже, одну за другой, нити  - будете убирать. И если вы нашли действительно истинное ``свое Я'', то Вы никогда, оборвав нити, не создадите новые нити, чтобы сделать марионеткой кого-то другого.}
\people{Скажите, а так вот с человеком всегда было? То есть, вот нагромождения такие вот существовали. Он сам их нагромоздил или это…?}
\soul{А кто же ещё? Конечно, самый первый груз, самую первую ``шкуру'' одел он. Ну, представьте себе общество, как говорите  - человеков, -  одевших,  каждый - по шкуре. А потом, вы стали дарить друг другу ещё шкуры. Ну, простите, если к вам пришли и принесли новые три шкурки, он виноват или вы? Вы же их одели, вы же приняли этот подарок!  Значит вы виноваты,  а не принёсший вам. Вы согласны?}
\people{Да.}
\soul{Вы их напялили, а теперь ими гордитесь. Теперь они вам жмут, и что вы хотите? Вы хотите новые шкурки. Эти снять и получить новые. К сожалению,  это так. К сожалению, множество понятий ваших учений и заключается в том. Вы хотите избавиться от тел, Вы хотите уйти в ``мир огненный''. Как хорошо! Ещё одна ``шкурка'', только из ``огня''.}
\people{Так человек может, допустим, ну если он нашёл себя действительно, то все вот эти его одежды, так сказать, сами собой куда-то исчезнут как бы? Или как?}
\soul{Нет. Просто уже не будут подчиниться этим одеждам.}
\people{Ах! Вон как.}
\soul{А одежды будут подчиняться ему. Неужели вы думаете, что прозрев, он сразу же станет бесплотным, потеряет все эти шкуры, потеряет все одежды? Нет. Просто он уже будет ими управлять. И он, уже будет – ``Когда хочу, тогда и болею. И какой хочу болезнью такой и болею''. А чаще, что? ``Ах!  Живот скрутило!'' - и всё, и настроение поменялось.}
\people{Ну, да.}
\soul{Так что? Живот вам настроение испортил? Или вы ушиблись, - ``Ах как плохо! Я ушибся,- у меня нет настроения!'' Или - на оборот  -  вы терзаете свою плоть с наслаждением. Откуда берётся наслаждение? }
\people{Ну, это уже св…}
\soul{Извне? Извне! Потому, что вы видите одобрение других. ``Вот каков! Какова сила его! Во имя бога может себя распять!''  А во имя бога ли, или во имя славы своей?}
\people{Во имя славы, конечно.  Скажите, а вот люди, тоже страдающие, допустим, такими отклонениями, как мазохизм и садизм, вот у них - тоже какие-то, вот, такие же  приблизительно симптомы тогда?}
\soul{Не ``приблизительно'', а ``так оно и есть''.}
\people{Так же, да? Ну, то есть в… болезнь одна}
\soul{Физическая игра. Игра физики.}
\people{Угу.}
\soul{Когда Вы видите человека только в плоти. И ещё… Редкое, но, к сожалению, страшное. Это когда вы уже увидели себя, оборвали нити свои, но тут же решили создать новые другие нити.  Как  жажда власти. Вы уже хотите, помня, что были марионеткой, хотите отомстить и сделать марионетками других. А заметьте, - самый жестокий властелин — это бывший раб.}
\people{Да, это точно.}
\soul{И вот, бывший раб становиться властелином. Властелином своего тела. Соответственно, ему это мало. Он хочет ещё. Ещё и ещё, и уже множество людей подчиняется ему. И это приносит ему удовольствие. Малейшее неподчинение строго карается, так, как карался когда-то и сам.}
\people{В общем-то, есть такие слова: ``А раб который стал царём, всё раб, всё тот же раб.[вздыхает]. Понятно. Ну, а вот если…}
\soul{Да. Вы правы. Нити оборваны, но не сброшены. И теперь, он окутан, окутан в ещё больший кокон,….}
 (Задержка, обрывается контакт, счет)
\people{Хорошо, скажите тогда [ вздыхает], ответьте на такой вопрос. Если мы, действительно, так себя потеряли, то только тогда, когда мы себя найдём, своё, вот, истинное начало, то только тогда мы научимся и любить по-настоящему. Только тогда?}
\soul{А Вы вспомните порывы любви! Вы должны помнить! Вспомните! Вы не сможете пережить,  но, хотя бы вспомните. Вы тогда разве много рассуждали логически?}
\people{Да вообще не рассуждали!}
\soul{И окружающую обстановку Вы оценивали трезво?}
\people{Нет. Нет, конечно.}
\soul{Вот Вам ``упоение любви''! Но, вы не можете столь долго существовать только в этом состоянии,  а если и будете, то это будет не нормально, это уже будет отклонение, это уже будет болезнь. Почему? Потому, что остались нити. И тогда - это уже просто Хаос. Нити! Нити дергаются беспорядочно, - кукла делает любые механические движения, и не больше. И, хотя где-то в глубине  уже что-то проснулось, зашевелилось, оно только бессильно, оно, всего лишь шепчет вам. А вы же, не разобрав слов, не разобрав того шепота, не понимаете, что твориться с вами. И тогда… Хорошо, если действительно любовь. Хорошо, если так. А бывает - нет. Бывает, что тело увлеклось, столь сильно! Столь сильно  вжилось в роль, а вы это называете увлечением, - и будет тот же эффект любви.  Правда, кратковременным, потому что он не будет поддерживаться Истинным Огнём.}
\people{Вот, поэтому, всё так у нас… }
\soul{Давайте дальше. Никогда никому не верьте, что вы уже не сможете исправиться, что вы так глубоко упали. Никогда никому не верьте. В вас есть зерно, истинное зерно и оно должно проснуться. Любое сомнение в вас - это уже пища для зерна. Конечно, у кого-то оно может расти быстрее, у кого-то  очень медленно, но оно никогда не погибнет. Вы можете представить, чтобы зерно погибло?}
\people{Да нет, конечно.}
\soul{Вы можете представить, чтобы часть, как вы говорите, ``часть божья'' - и вдруг погибла? Что, бога стало меньше? Такого быть не может. Но, только услышав это, многие из вас опускают руки… ``Ха-а! Тогда всё хорошо! Чего мне бояться?'' Церковь ваша решила избавиться от этого и строго отменила реинкарнации, чтобы вы быстрее думали. А что в итоге?  Отменила почти все. Теперь любое проявление считается не нормальным , а дьявольщиной. ЗдОрово, не правда ли?}
\people{Да.}
\soul{Это что? Это, всего лишь, только Власть. }
\people{Да. }
\soul{Много власти, и не больше.}
\people{(Ольга Васильева) Один.}
 (Задержка, обрывается контакт, счет)
\soul{Говорите.}
\people{Вот скажите, опять же, может нетактичный вопрос… Как-то, раньше, я лично боялась его задать, всё-таки,  вот, из всех прошлых, так сказать, контактах, я, например, сделала вывод, что, - я буду о себе говорить,- у меня с переводчиком какая-то, всё равно, существует кармическая связь, что ли… Ну, не знаю. Мы  с ним знакомы, так сказать. Это действительно так? То есть, вот я узнала себя в некоторых персонажах. Я, правда, не  знаю, права или нет.}
\soul{О, нет! Ищите.}
\people{Нет, да?}
\soul{Ищите. Мы не будем ``разжёвывать'' за вас.}
\people{Нет, нет, я не это, не прошу, но, просто…Понятно. А вот, о цветах, если можно так, поговорить о цветах… Вот, допустим, если красный цвет, то, когда, вот, смотришь на него очень долго, появляется, ну, допустим, над предметом, зеленое как бы отражение…Не отражение…. Как сказать….}
\people{Аура.}
\people{Аура что ли, если можно так сказать. У оранжевого — голубая. У желтого, допустим, голубое,  и на оборот.}
\soul{Возьмите учебник физики, и Вы там уже найдёте.}
\people{Там уже всё…}
\soul{Не связывайте это с чем-то. Возьмите учебник и вы найдёте там ответ.}
\people{Ага.}
\soul{Если говорить действительно о цветах, в Вашем понятии, - именно отвечающие за цвета,- то мы уже говорили и повторим вкратце. Нет истинного чистого цвета никакого! Нет истинно красного. Нет! Нет таких аур, чтобы они были истинно красными. И истинно зеленого над  тобой нет. Говорит основной цвет. Относительно - вы правы. Относительно. Почему? Потому, что вы - есть колония живых существ, огромное количество живых веществ! И можете говорить об основном цвете, именно только говоря о каком-то органе. Конкретно, о каком-то участке вашего тела. Тогда вы можете говорить об основном цвете содержащем. Но не обо всём. И если вы будете светиться каким-то одним цветом, как вы говорите - `` обладающим всем телом'' - это уже ненормально. Это уже плохо. Это говорит о том, что ваш орган, или становится ``рабовладельцем'' или ``рабом''. Эффект будет тот же. Возьмите сердце, - у сколь многих оно искусственно? — Доброта… Искусственна, взята из книжек, из фильмов, из желания выглядеть добрым - и сердце уже искусственно. Да, действительно, сердечная чакра будет очень ярка и будет, как вы говорите ``основным цветом''. Но, здесь надо увидеть, что этот цвет искусственный. Если же нет, то вы обманетесь. Если же он станет вашим учителем, тогда вы тоже будете идти по неверной дороге. Вы понимаете это? Дальше. Его цвет имеет множество различных названий, и соответственно, множество различных соотношений с чувствами. Нельзя сказать, что красный цвет это - только Зло. Это Огонь. Да, это действительно огонь. Огонь, который может и сжечь, огонь, который может согреть, огонь, который может осветить. Ну, всмотритесь, всмотритесь в огонь , и вы увидите в нём множество различных других оттенков. И огонь этот, может гореть другими цветами. Зависит от примесей, зависит от того, что горит. Вы согласны?}
\people{Да, да.}
\soul{А вот теперь, представьте, если в Вас горит ненависть…и, в тоже время, вы сильны энергией красного цвета. Какой будет оттенок тогда этого цвета? Какой будет огонь? Огонь  будет окрашен уже ненавистью! Цвет? Цвет — серы.}
\people{Цвет ненависти серый?}
\people{Серы.}
\people{Серый с желтизной.}
\people{Интересно. Скажите, а вот запахи и цвета, они, конечно, связаны, и всё имеет… Ну, допустим, если цветок..}
\soul{Если это вибрация это одна и та же. Просто…}
\people{Да. Ну, вот, если цветок…  ну, одного цвета, допустим, розовый, вот, у него запах тоже соответствует, ну, так сказать, может соответствует розовому или нет? И звук его, если конечно можно его услышать, мы не слышим, конечно… Ну, вот, если другой цветок, другой породы, если так сказать, будет тоже розовым, но у него запах другой, значит, цвета и запахи не связаны как-то? Или это ещё зависит от…}
\soul{Нет, всё  это связано. Но вспомните об огне и примесях. }
\people{Ах, вот как даже. Ну, всё ясно. Угу.}
\soul{Что - цветок? Он более искренен. Искренен тем, что не прячет себя. Он просто растёт. Он просто растёт, имеет цвет и показывает его вам, и этим цветом хочет увлечь вас. Он соблазняет вас, потому и красив.}
\people{И женщины, наверное, так же.}
\soul{А вы, чаще краситесь…}
\people{Да.}
\soul{…краситесь  искусственными  цветами потому, что не хватает своих.}
\people{Нет, потому, что так хотят мужчины,  наверное, и мы выполняем их желания. Только что был на эту тему разговор…}
\soul{Хотят мужчины?}
\people{А разве нет?}
\people{И смех и грех.}
\soul{Так, где же тогда ваш истинный?}
\people{Так вот… В том-то и дело… Оно так и получается….}
\people{А чего вообще появилось это ``крашение''?… }
\people{Почему, действительно, женщины любят себя украшать, и мужчины как-то…}
\soul{Ну, давайте,  договоримся тогда, что сперва, как вы говорите, ``краситься'' начали всё-таки мужчины.}
\people{Татуировки.}
\people{А… ну, да, татуировка. Да.}
\soul{А уже только потом переняли женщины. И вот, эти колебания  весы постоянно перевешивают. Красились мужчины, потом, стали женщины, теперь  - опять мужчины. Женщины уже предпочитают иметь более естественный цвет. Вот вам - `` качели''. Для чего это было? Для устрашения!}
\people{Кого?}
\soul{Для устрашения, врагов. Потому, что первое, что вы сделали,  это нашли себе множество врагов.}
\people{Это точно. Вот, когда смотришь в фильме, я вот тоже, недавно об этом подумала. Фильмы смотришь… Фильмы смотришь…}
\soul{Говорите.}
\people{… о племенах в Африке. И вот, они ещё, видно что, ну, первобытно-общинный строй у них., но уже,  они Воины. В первую очередь, - они Воины. Всё остальное уже потом. Человек, действительно, вот как-то агрессивен, будем говорить. Ну, если не изначально, то очень… Не знаю, как… (сказать. Прим.)}
\soul{Это цвета.}
\people{Ещё многому не научились, а уже убивать научились. Выходит так.}
\soul{В первую очередь. Вы говорите о татуировках. Да, она, действительно, была и есть. И в себе несёт многое. От того, как вы покрасились, можно сказать о вашем характере. И будь то просто губная помада сейчас или у древнего воина, - не имеет значения. Вспомните. Вспомните, как вы говорите,  ``дикаря''. Что было для него красный цвет? Страх! Огонь уничтожающий и сжигающий. Вы помните?}
\people{Да!}
\soul{Соответственно он брал красные краски, чтобы боялись его. И, только потом  он догадался, что ещё  огонь может сохранить ему жизнь. И тогда вождь уже наносил красные краски - одну полоску для устрашения врагов, а другую полоску,  что он будет ``хранителем племени''. Возьмите краску зеленую. Что это?}
\people{Растительность, трава.}
\people{Жизнь.}
\people{Маскировка}
\people{Жизнь,}
\people{Да?}
\soul{Хорошо, пусть будет ``трава''. То тогда Вы должны были помнить, что трава  раньше была не зеленой.}
\people{Раньше? Да!}
\people{Угу.}
\people{Да, да, нам же говорили.}
\people{Да это же какое время-то!? }
\people{Ну, какое. Вот, в то время, когда жил наш друг, допустим, или…}
\soul{Так вот, пусть будет трава. Зеленый цвет. Ярко наносит полоску, что Он хочет жить. Он просит у богов жизни, да побольше, потому и линия будет потолще. И, в тоже время, это пища. И хотя жизнь и пища раньше были соединены, всё же наносится ещё одна полоса. Можно-то жить и голодным, но лучше было бы сытым. Вы согласны? Наносится ещё одна полоса, потом наносится ещё одна, чтобы накормить племя, если это вождь.}
\people{Угу.}
\soul{А теперь, давайте возьмём отрицательную сторону того же зеленого цвета. Припомните сами.}
\people{Страх, если вот, ну..}
\soul{В чём он мог увидеть страх?}
\people{В зелёном цвете страх в чём?}
\people{В лесу том же, страх. Звери там всякие…}
\soul{Правильно! И как часто это связано с неизведанным, ибо, он не видит того зверя и чувствует опасность. Значит, наносится линия, чтобы уберегла от опасности или чтобы спрятаться. Просто спрятаться. Цвет маскировки, который сохранился до сих пор.}
\people{А вот скажите тогда, вот говорили, что было Солнце красное, небо было красное.}
\people{Трава была красная…}
\people{И всё вокруг было красным. То есть, это  воспринимал так глаз?}
\soul{Нет, это физика.}
\people{А она меняется? И ну, вот мы сейчас так же?}
\soul{Меняется! Пока вы(человечество. Прим.) голубые. Но, скоро измениться и этот цвет.}
\people{Интересно, на какой.}
\people{Жёлтый?}
\soul{А вы посмотрите спектр, и тогда, всё будет понятно вам.}
\people{Голубой — синий. Если в цветах, ну, вообще нам говорили же вы, по-моему, что жёлтый будет, жёлтый. Нет?}
\people{Оранжевая раса.}
\soul{Посмотрите, посмотрите спектр: - ``Каждый Охотник Желает Знать, Где Сидит Фазан''.}
\people{Ну, сейчас голубой. Поэтому и небо голубое. Да? Так поэтому мы видим и небо голубым. Так?}
\people{Синий.[шепчет]}
\people{Так!}
\people{И потому, что нами, в общем-то, правит, не… не правит разум… ну, преобладает в сознании. А цвет сознания, да? Это центр, горловой центр. }
\people{То есть, воспроизводит от того центра, который более развит, что ли?}
\people{Нет, как… Да, вот, в общем-то.}
\soul{Давайте скажем так,  что не от центра зависит, а центр зависит.}
\people{Ну, да. От какого-то… Тогда, вот мы уже прошли, ну,  допустим, вот это -  красный и так далее до голубого или нет?}
\soul{Прекрасно… Давайте теперь вспомним дальтоников. А куда же они ``дошли''?}
\people{А почему у них не нормально считается? А может у них нормальное восприятие? Для них, во всяком случае.}
\people{Может, это чисто нервное? То есть, какие-то каналы заблокированы или…}
\people{Не, ну ведь они находятся, допустим,  как если зелёный -  видят, как красный. Может, это связано с чем-то, как в ту эпоху, где жил, допустим, вот, наш друг? Нет? Это не так?}
\soul{И это тоже.}
\people{И это так..[говорит для себя]. Ну, а может, когда-нибудь не…}
\soul{Химия!}
\people{Химия?}
\soul{Химия.}
\people{Тела, да?}
\soul{Тела! }
\people{Ну, так, когда не будет…}
\soul{А вы должны были бы помнить о родословной. Вы должны были помнить о ДНК. А что такое ДНК? Это, тот же самый архив пройденной жизни.}
\people{Ага.}
\people{То есть…}
\soul{Независимо - впервые родился… Первая это реинкарнация или нет. Независимо. Потому, что существуют ещё и родители.}
\people{Ну, родовая связь, да? Скажите, вот все люди,в принципе, на Земле-то и физически, наверное , связаны? Почти все, во всяком случае, или…}
\soul{Да.  Мы уже пытались даже как-то сказать, что человечество произошло от Адама.}
\people{Да нет[улыбаясь]. Не, ну вот, так, если проследить нам, там, кто на ком, как, что, - и получается, в принципе, ну,  очень большие, во всяком случае, вот, группы людей. Может, потому и называется там нациями? Или…народами. Нет?}
\soul{Ну, откуда, по-вашему,  взялись нации и народы? Ну, вот, возьмите и скажите, что русские - это одна семья, китайцы — это другая семья, и вы не много-то и ошибётесь.}
\people{Скажите, а всё-таки, почему вот и цвет тела у всех разный и форма. И как-то вот…}
\people{Физически…}
\people{Ну, вот, да, расы. То есть, это всё зависит от того, в каком он климате живёт?  Люди живут… Или ещё какие-то причины были?}
\soul{Да. Да. Да.}
\people{Если это физика, значит, и от физики зависит.}
\soul{Физика, чисто физика. Здесь не нужны какие-то эзотерические знания. Возьмите учебники физики, химии, биологии и вы найдёте здесь множество  ответов. Вы найдёте их, и они будут верны. Вы очень многое, что связано чисто с физикой, хотите в них найти какие-то несуществующие тайны. Всё в Мире взаимосвязано! Всё! И пока вы открываете не законы, а кусочки проявления этого единства. Если вы говорите о каком-нибудь Законе и формируете его — это, всего лишь вы заметили маленький, маленький кусочек чего-то единого. Проявление на Вас - этого Единого. И, если Земля обладает определенными физическими параметрами  - это только из-за того, что это Единое проявилось именно  той реакцией, которую Вы измеряете. И если Вы придёте на другую планету или другую Вселенную то увидите совершенно другое, увидите совершенно другое. Это и есть другое? Это та же самая физика! Та же самая! Только Вы здесь увидели другие реакции, другие проявления только и всего. То же самое - когда стоят двое и рассматривают картину, и каждый видит своё. Всё, то же самое. И каждый запомнит только то, что хотел запомнить он. Совершенно разное. И, даже когда вам скажут:  ``Посмотрите на этого человека и потом опишите его'' -  вы все его опишите по-разному. Потому, что одни будут обращать внимание  больше на одно, другие - на другое. Правильно?}
\people{Да!}
\soul{Здесь уже можно, конечно, говорить, если далеко зайти, о духовном. Что человек больше смотрит и видит ``другого''. Так же и здесь - проявление одной и той же физики, но в разных условиях.}
\people{Вот, скажите,  элементы, вот, все, химические, которые, допустим, мы знаем это…}
\soul{Это, всего лишь только, проявление, и не больше. Если хотите, то, как вы говорите –'' искривление Пространства во Времени''. Интересная формулировка…}
\people{А выходит, что…}
\soul{Так вот…  Здесь можно взять…  Да - пространство было немножко другое, немножко другое искривление, так под  другим углом, и поэтому, получилось -  здесь - водород, здесь - амиак. Ну, что ж… Ну, примерно, так.}
\people{Угу. Один.[счёт]}
\people{Скажите, а правда что у каждого, вот так иносказательно может быть, мужчины своя женщина на Земле? Вот это  - про две половинки.}
\soul{Правда. Вы сомневаетесь в этом?}
\people{Сомневается. Конечно, сомневается.}
\people{Сомневаюсь.}
\people{Ну, чё смеемся?[говорит девушке]}
\soul{Это жалко. Если Вы сомневаетесь в этом, то, значит, Вам будет очень трудно найти свою половинку…. }
\people{Да, мне кажется, эти половины…}
\soul{Дальше… Дальше, - что такое мужчина и женщина, в вашем понятии? И обязательно ли они должны сойтись? Обязательно ли должны быть вместе? Мужчина и женщина… Вроде бы - и то и другое, - человек.}
\people{Да.}
\soul{А совершенно всё разное, совершенно всё противоположное. А совершенно ли ВСЁ  противоположное?}
\people{Да. Нет, нет, конечно.}
\soul{А что только противоположно?}
\people{Только пол.}
\soul{Только имя!}
\people{Да.}
\soul{Мужчина и Женщина - и не больше. А пол? Пол - второстепенно. Пол уже, даже сейчас, вы научились менять, как хотите.}
\people{Да.}
\soul{А вот Род  каждый из вас - не знает! Он может все жизни прожить женщиной, а Род - мужской. Почему? Да потому, что просто не хочется, и только и всего. А почему?… Не знаем.}
\people{Вы тоже не знаете?}
\people{Вот, между прочим, мне сказали, что, вообще-то, я вот, как бы пол… Род у меня мужской.  А я сейчас – женщина. А вообще, я… У меня были какие-то такие внутренние… }
\soul{"Мне сказали"… ``Мне нагадали''… }
\people{Да, нет.}
\soul{И всегда так!  Вы найдите и почувствуйте в себе, только в себе, а не то, что сказали вам. }
\people{Извините, все…}
\soul{Если мы сейчас придем и скажем вам: `` Вы были тем-то и тем-то…”-  Да не верьте вы нам! }
\people{А кому тогда верить?}
\soul{Не верьте! Найдите в себе. То, что откликнется в вас… Пускай, это даже будет не верно. Пускай, вы, всё-таки  ещё актер… Мы вам говорили, найдите в себе. Если вы будете бояться  – ``ах, я ошибусь, я ошибусь”- это будет всего лишь только ещё одна роль, ещё одни выученные слова для спектакля, и не больше. Да не бойтесь вы ошибок! Вы не бойтесь их совершать, вы бойтесь их не замечать. Не бойтесь, когда вы увидели ошибку, бойтесь, что вам лень её исправить или ``страшно'' исправить. Сколь много сделали вы ошибок? Сколь много? И вместо того, чтоб изменить их, чтоб исправить их, вы уходите от них – ``давай, забудем!'' Ваша любимая фраза – ``Кто старое помянет…'' – это что?}
\people{Ну, может быть, действительно, наоборот - надо выговориться что ли?}
\soul{Выговориться? И что? Вы, если вам позволить, начнёте выговаривать всё, и нужное и ненужное. Вы приплетёте сюда всё что угодно, всё, что придёт вам в мысль. Так и говорите.  Любое слово.  И приплетёте всё, и уже наговорите целую кучу, а потом, когда очухаетесь, будете говорить: ``Неужели всё это я наговорил так? И где я это всё взял?''.  ``Слово – не воробей…'' – сказано вами! Знаете ли вы меру в спорах? Знаете ли вы меру в словах? Когда вы начинаете спорить, вы теряете контроль и вы уже придумываете кучу доказательств… и даже вспоминаете – ``да я где-то читал, сказал такой-то великий…'' , хотя этого ``великого `` даже и не существовало, а если и был этот  ``великий'', то он никогда об этом и не говорил! Правильно?}
\people{Да.}
\soul{Правильно. А вы сочиняете, что бы доказать. Потому, что вы чувствуете, что своего авторитета уже не хватает, и вы тут же цепляетесь за теорию другого. И тогда, получается – ``да, я с вами поспорил, теперь с Коперником  спорьте!''  А вы часто делаете так. Часто. Из-за какой-то мелочи вы раздуваете ссоры, и эта мелочь, как снежный ком, обрастает всем ненужным. Вы вспоминаете всё, что было и не было. Вы сочините целую кучу всего, что вами было не договорено и не понято. Вы, тут же переиграете и сделаете по-своему, и уже будет, как обвинение.  Где вами непонятое станет обвинением. И что? И эта мелочь переросла в отраву.}
\people{Да, нам надо быть  царями.   Мы все считаем это за проявление, так сказать,  умения жить… Или, как там ещё…}
\soul{Да, конечно, вы, чаще, подчиняетесь силе, и хотите силу. В вашем понятии –'' держать в уздах'', ``держать в кулаке'' - Ваши, ваши выражения.}
\people{Но, всё-таки, давайте, немножко продолжим, всё-таки, о расах. Вот, наверно, это заметно, что человечество… Не будем говорить, что только человечество, наверно все, кто на Земле жил, - они повторяют как-то свою историю, будем говорить. Даже если взять то, что нам показали будущее… да? Вот, там люди похожие на муравьёв  - это вроде как из прошлого тоже? Из далёкого, опять же.}
\soul{Мы говорили вам о ``спиралях''. Говорили? Но мы и говорили, что каждый из вас побывал в ``шкуре'' любого. Каждый из вас был  ``негром'' – как вы говорите, и обладал всеми признаками всех рас. Каждый из вас. И потому, (…) нацизм, расизм. Именно потому.}
\people{А почему?}
\soul{Да потому, что вы считаете, что  ``’эта раса выше,  та - ниже''. А вы хотите нам, о расах -  `` Ах, эти черномазые! Они ничего не умеют!''. В вашем понятии, если они не обладают… не ``окультурены'' вашей техникой, значит они глупы, значит, их можно использовать, как рабов. И хотя вы сейчас говорите, что у вас нет рабов, у вас их множество.}
\soul{Да, надо будет сравнить с собой.  Многих рабом увидишь.}
\soul{Ну, и как вы определяете, кто властелин, кто раб? А не замечаете, что вы, властелин рабов, сами же – раб кого-то другого.}
\people{Да…}
\soul{Вы этого не заметите. Нет этого разделения.}
 [Запись обрывается]
 (КОНЕЦ)
Аоум. глава 23-12-95
Георгий Губин
23-12-95
\people{**}
\people{У вас, у кого?}
\people{Рассказывайте. Нам интересно.}
(разрыв плёнки)
\people{Я? Да. Почки, сердце. (Гера)}
\people{А вы можете что-то еще добавить? (Белимов)}
\soul{Вы будете страдать головными болями и давлением. (Белимову)}
\people{А кто это говорит? Кто говорит? (Ольга)}
\soul{Вам же - бояться воды.}
\people{Тебе. (Белимов Ольге)}
\people{Мне? Или тому, кто… чье поле рядом с вами? (Ольга)}
\soul{Вы говорите, сейчас.}
\people{Так, а чего бояться мне, который рядом с ней сидит? Есть у вас такие.. сведения? (Белимов)}
\soul{Разве нет?}
\people{С кем мы разговариваем? Вы можете назваться? Вы можете назваться? (Ольга)}
\people{Маятник! (Гера. Просьба ко всем обратить внимание на изменение махов руки переводчика.)}
\people{Ясно..  другой,  другой… или у него в подсознании или другая, другая сущность. (Белимов)}
\soul{Возьмите…}
\people{Так. Вас за руку? (Белимов)}
\people{За руку вас взять? (Лена)}
\people{Сейчас возьмёт. Только тихонько. (Белимов Лене) Что вы можете сказать?}
\soul{Не долгая жизнь. (очень тихо)}
\people{Недолгая жизнь. (Гера)}
\people{Это из-за болезни или же из-за ситуации в стране? А могу я, задающий вопросы, тоже взяться за руку и что-то получить от вас информацию? (Белимов)}
\soul{Мы же предлагали вам.}
\people{А-аа.. (Белимов взялся за руку переводчика) Такая информация идёт? Пожалуйста, мы ждём.}
\soul{Ассеметрия сердца.}
\people{Да? Но  пока ничего не чувствовал. (Белимов)}
\people{Так с кем мы разговариваем? Кто нам отвечает? Вы можете назваться? (Ольга)}
\people{Сознание переводчика, или какая-то другая сущность? (Белимов)}
\soul{Вы боитесь…}
\people{Ну..  боязнь, не боязнь, но нам интересно, может быть, действительно другие на нас выходят ещё цивилизации. Такие случаи бывали. В других группах. (Белимов)}
\people{А можно спросить, в связи с чем связано? В связи с чем недолгая? (Гера)}
\people{И можно ли это исправить? (Лена)}
\soul{Судьба.}
\people{Судьба?}
\soul{Вы слабы, чтобы бороться с ней.}
\people{И всё-таки, вы можете назваться? Почему вы молчите? (Ольга)}
\soul{А как мы назовёмся? В любом случае, вы можете не поверить. В любом случае мы не будем доказывать вам кто.}
\people{А нам не нужно доказывать. (Ольга)}
\soul{И будете ли вы знать имя наше?}
\people{Ну, не имя, а хотя бы.. из.. это… ну.. иной мир, параллельный мир, инопланетный мир или другое сознание? Что из этих названных верно? (Белимов)}
\soul{Ничто.}
\people{Как? Вы даже … (Белимов)}
\soul{Вы ищите извне, и не хотите найти в себе.}
\people{В себе, да? Ааа..ну значит, подсознание. (Белимов)  }
\soul{Разве?}
\people{Нет? А что? (Белимов)}
\soul{Мир столь велик… И вы не только одни в нём… А вы говорите о параллели.}
\people{Ну, тогда нам трудно представить, с кем мы имеем дело. Внутри нас, что может?  Микробы быть, Это, что…Вирусы, - тоже может быть ``мыслительный процесс''. (Белимов)}
\people{Вы живёте на астральном плане? Ну, как астральный план? (Ольга)}
\soul{Нет. }
\people{Ментальный план? (Ольга)}
\soul{Нет.}
\people{Выше? (Гера)}
\soul{Мы живём среди вас. Мы такие же, как вы, только вы нас не видите. Но нельзя говорить о параллели.}
\people{Вы живёте чувствами? (Ольга)}
\soul{Нет.}
\people{Разумом? (Ольга)}
\soul{Нет.}
\people{Вы имеете материальную оболочку тела? (Белимов)}
\soul{Да.}
\people{Ну, тогда в каком измерении вы живёте? (Белимов)}
\soul{В вашем.}
\people{Так. (Белимов)}
\soul{Среди вас.}
\people{Среди нас. (Белимов)}
\soul{Но вы нас не видите.}
\people{Тогда, спириты, может быть? Спиритические…  Вы - души умерших людей? (Белимов)}
\soul{Нет. Разве души имеют материю?}
\people{Нет, не имеют. Ну, тогда как же вы имеете материю и среди нас? Может, кто-то из нас? (Белимов)}
\soul{Что видите вы? Вы можете видеть движение воздуха? Вы можете увидеть поля?}
\people{Ну, в какой-то мере, да. (Ольга)}
\soul{Да?}
\people{Но вы не воздух? (Гера)}
\soul{Давайте так, - когда вы можете увидеть электрический ток?}
\people{Увидеть? Когда он в глазах засверкает, только тогда. (Смех)}
\soul{Да. Только когда  коснётесь. Вот и мы боимся дотронуться до вас, как и вы до нас.}
\people{Ну, вы… (Ольга)}
\soul{Мы - небольшая частица вашего страха, любопытства. Если хотите, то - мы затрагиваем ваш эмоциональный мир. }
\people{Ну, разве это не та же самая сущность или субстанция, с которой мы вели уже длительное время разговоры? Разве это не то же? (Белимов)}
\people{Мы говорили - не говорить о ``сущности''. И есть и  ``субстанция''. Это ваши слова.}
\soul{Ну, да.}
\soul{И этим вы унижаете себя и нас и, тем более, их, что достаточно много времени потратили на вас. Они говорили вам, сколь быстра их жизнь и  сколько прошло их уже. Разговор идёт.}
\people{А вы… вы тоже участвуете вот в наших беседах? Вы тоже всё знаете, о чём… (Ольга)}
\soul{Давайте скажем так: мы соседи им и вам.}
\people{Ну, в общем-то, мы все вместе, значит. (Ольга)}
\people{Ну, какую информацию вы нам можете подарить? Нам хочется новую информацию иметь, а не пережёвывать старое. На вопросы… Вы можете отвечать на вопросы? И насколько они будут другими по отношению с теми, что мы говорили? (Белимов)}
\soul{Или вы ищете подарков.}
\people{Ну, мы просто любознательны, и мы хотим как бы можно глубже во что-то… }
\soul{Да нет…}
\people{… не подарки. Ну, почему…  (Белимов)}
\soul{Вы боитесь, вы боитесь плавать. Вы набираете нужную информацию, и ничто не меняет вас, ничто. Вас успокаивает то, что у вас память плоха. И не хотят они ругаться. А вы же - просто ленивы. Ленивы и боитесь.}
\people{Наверное. Мы несовершенны. (Белимов)}
\people{Скажите, может, просто поговорить, так… Просто так, без… (Гера)}
\people{Предложите вашу тему, о чем вы можете с нами поговорить, и что вас интересует. (Белимов)}
\people{И хотите ли говорить вообще? (Гера)}
\people{Да. (Белимов)}
\soul{Говорите.}
\people{Значит, вы часть нас, да? Ведь мы часть вас, или всё вместе частями мы, в общем, человек, да? (Ольга)}
\soul{Если говорить о единстве, - о том Великом Единстве, то мы –  есть вы. Здесь они правы. Если же брать только в ваших рамках, то - нет,  мы – не вы.}
\people{У вас тоже есть общество? (Белимов)}
\soul{У нас тоже есть материя. У нас тоже есть тела. И даже более. Как вы считаете – семь тел. }
\people{Но переводчик…  (Ольга)}
\people{Разделение полов есть? (Белимов)}
\soul{Да.}
\people{Семьи есть? (Белимов)}
\soul{Да.}
\soul{Вы тоже размножаетесь? (Белимов)}
\soul{Да.}
\people{Какая у вас цель в жизни? (Белимов)}
\soul{Мы тоже ищем. Разве кто-нибудь может сказать вам о цели? В любом случае мы солжём.}
\people{А мы вас не отвлекаем? (Ольга)}
\people{Может у вас дела какие-нибудь там свои? (Гера)}
\soul{Разве вы силой заставили говорить нас?}
\people{Да нет. (Лена)}
\people{Ну… тогда мы не знаем. Мы готовы продолжить разговоры предыдущих контактов и…  новые вопросы. Вы готовы отвечать? Вы будете отвечать на эти вопросы? (Белимов)}
\soul{Говорите.}
\people{Скажите. Вот можно вопрос задать? Скажите, когда на прошлых контактах вообще-то у меня всегда хорошее состояние физическое было, а в этот раз я почувствовала сердечную боль, вообще, в левой стороне боль почувствовала. Вообще-то я поняла, что… }
\people{Кстати, и я тоже неприятное чувство ощущала. (Лена)}
\people{Нет, это вот, чем объясняется? Вы не можете объяснить? (Ольга)}
\soul{Мы приходим грустью. Мы никогда не приходим к вам, когда вам радостно. Такова наша тяжёлая доля приходить, когда вам грустно и тяжело. Мы, в вашем понятии, предохранители ваши. Вы слишком  часто пользуетесь нами. Ваша боль уничтожает нас.}
\people{А вы дэвы… Вы не… (Ольга)}
\soul{Давайте без терминов.}
\people{Без чего? (Ольга)}
\people{Без терминов. (Белимов)}
\people{Скажите, а вот они как-то говорили, что чтобы с нами разговаривать, они замещают одно из тел, ну…  Вы и есть те, кого они замещали? (Гера)}
\soul{Вы не внимательны.}
\people{Возможно. (Гера)}
\soul{Вы не поняли, что говорили они.  Один из ваших предложил это.}
\people{Так, вы что, получается, все контакты наблюдали и в курсе всех наших вопросов и событий? (Белимов)}
\soul{Мы же живём рядом с вами. Рядом. И мы не знаем тайн.}
\people{То есть, у вас там, где мир открыт, полностью всё можно увидеть и нельзя спрятать ничто? (Гера)}
\soul{Одна из ступенек ваших.}
\people{Можно сказать, что вы наши ангелы-хранители? (Ольга)}
\soul{Нет.}
\people{Вы - наши утешители? (Гера)}
\soul{Предохранители.}
\people{А-аа.. Скажите, а предохранители, вот.. От чего есть грусть? Большая, допустим. Ну.. как? Если грустно кому-то, допустим, должен кто-то… }
\people{Это же временное явление, оно уходит… (Белимов)}
\people{…или кто-то весёлый его должен хоть отвлечь. (Гера)}
\soul{Мы приходим к тем, кто грустит, чтобы не сделал неверного шага. Одни называют, если говорить вами, - `` темными'', другие  – `` светлыми''. Мы же не имеем цветов, мы просто приходим и помогаем, как можем. Иногда, мы приносим большую боль, чем та, что позвала нас, чтобы облегчить у вас или хотя бы отвлечь от дурного шага. Бывает и такое. Но, чаще, мы не можем справиться с вами. Не можем, потому и называем себя ``предохранителями'', которые имеют свойство перегорать.}
\people{Вы умираете? (Ольга)}
\soul{Материально? Да. Но только мы, в отличие от вас, можем выбирать тело, и помним, помним всё.}
\people{Скажите, может, состояние в нашей стране таково, что вы вынуждены к нам прорваться на контакт и прочее… в общем, тут и каждый из нас, лично, не так уж удручён, чтобы пользоваться вашей помощью. Мы не в состоянии пред-самоубийства. Почему вы всё-таки решили, что пора вам объявиться в разговоре с нами? (Белимов)}
\soul{Не пришли бы - и были бы самоубийцы. Мы же говорили вам, что мы ``отёк'' ваш.}
\people{Громоотвод.(Ольга)}
\soul{Пусть будет так.}
\people{А что, кто-то из нас наиболее чувствует себя удрученным? Тяжело… (Белимов)}
\soul{Каждый из вас.}
\people{Каждый? (Белимов)}
\soul{Нет тех людей, которые не имели бы с нами дел. Нет. И даже ваши, ваши, ваши…}
(счёт)
\people{А мы… если нам… можем сейчас? Переводчик способен выйти на…  на, в общем-то, ту прежнюю цивилизацию или..как назвать… на более высокую ступень, с которой мы общались? (Ольга)}
\soul{Вот,  вы меряете ступенями… Зачем же тогда вам  говорили? Нам жалко их, жалко вас.}
\people{Скажите, пожалуйста, а так получается, что мы… Ну, как сказать…  У меня такое ощущение, что мы – далеко не вы, и, как бы так, какие-то ущербные что ли, тут все. (Гера)}
\soul{Нет, ущербных нет. Есть только слепые и глухие. Никого не надо называть ущербным. Разве можно назвать ребёнка калекой, если он не умеет читать или слеп?}
\people{Скажите, вы информацию даёте тоже в картинках переводчику? (Ольга)}
\soul{Да.}
\people{А почему он сейчас может спокойно разговаривать, не отвлекаясь на движения руками? Вы более… (Белимов)}
\soul{Вы должны были заметить, что мы не произносим частого слова их. Вы можете назвать его?}
\people{Ага, да-да.(Гера)}
\people{Каких? (Белимов)}
\people{”Спрашивайте''. (Гера Белимову)  }
\people{Ааа… (Белимов)}
\soul{Почему? }
\people{Да. (Белимов)}
\soul{Потому, что нам не нужна их сила. Нам надо только слышать вас. И всё. Они не материальны, мы – материя, и слышать вас можем.}
\people{Угу. Хорошо.}
\people{А они, вроде бы, говорили, что… как сказать… -  у них есть свои тела, своя, получается, материя, только не в нашем каком-то понятии. В ином. (Гера)}
\people{Ну, более тонкая материя, скажем так. (Ольга)}
\soul{Нет, совершенно другая физика. Поэтому нельзя найти аналоги. Если говорить вашими словами, то иная Вселенная. Совершенно иная. А там нет ничего вашего, и вы не можете попасть туда, как бы ни хотели того.}
\people{А они к нам могут попасть? (Гера)}
\soul{Нет. И они не могут.}
\people{Но, они же говорили, что они – это мы, а мы … (Ольга)}
\soul{А вы были бы внимательны. Они говорили о реакциях.  Вот вам, пожалуйста, на стыке этих Вселенных есть реакции. А где стык? Стык везде. В каждой точке. В каждом из вас.}
\people{А как вы их охарактеризуете? Они говорят, что сопутствуют человечеству 15 тысячелетий. Они правы? И сколько лет вы с нами рядом? Можете ответить на это? (Белимов)}
\soul{15 000 лет, которые могут разговаривать с вами. А раньше… Раньше, вы отказались от них. Они же говорили вам, что вы были подобны им, и пришли в этот мир, и сказали, что справитесь сами. Но, вас - материю, -  и вы забыли, кто вы. Они же говорили вам, а вы забыли. Потому, жалко их и вас. Вы же пришли сейчас,- вы устали. Вы устали. Первое - берущий за руку, вы привыкли обвинять. Вина его.}
 
\people{Первый, берущий за руку… (Ольга)}
\people{Ты… ну, ты… (Лена)}
\people{Ну, а скажите, вы давно с нами, с человечеством, сосуществуете? (Белимов)}
\soul{Да.}
\people{Вы впереди нас немножко по развитию или одинаково, или просто более информированы? (Белимов)}
\soul{Давайте скажем их словами - ``не меряйте''.}
\soul{}
\people{Ну? как мы тогда можем понять? (Белимов)}
\soul{Принимайте на равных.}
\people{А-а… Значит, равные, да? (Белимов)}
\people{Да, но как… (Гера)}
\soul{Вы же читали библию и уже забыли. Вы же читали. Вспомните.}
\people{Читали, читали… А как вы их охарактеризуете? Они, всё-таки… Мы считаем, что они впереди нас. Интеллект, по крайней мере, заметен интересный, ответы…(Белимов)}
\people{Управляют больше своей мыслью лучше, чем мы. (Гера)}
\people{Да. Почему вы говорите, что вам их жалко? (Белимов)}
\soul{Расширьте. Расширьте зрение ваше и слух ваш, расширьте свои чувства – и будете видеть больше, конечно же, будете умнее. Почему жалко? Да чаще уж - разговор у вас не получается. Они же не хотят насилия.}
\people{Это не их несовершенство, а, скорее всего, нас. И мы согласны с этим. (Белимов)}
\people{Скажите, может быть, вообще не нужны вот эти такие беседы, может быть, мы просто из любопытства, ведь… (Ольга)}
\people{Вторгаемся не в те сферы? (Белимов)}
\people{Да, может быть… (Ольга)}
\soul{Когда-то вам надо было сделать первый шаг. И вы будете подготовлены.}
\people{А как набра…(Белимов)}
\soul{Ничего не происходит  пустое. Если они говорят, и мы говорим, что пустые разговоры, то просто хотим дать вам понять, чтобы вы двигались. Ничто не пусто. Ничто.  Любое мгновение двигает вас. Вперед-назад – это только вы привыкли считать - Перед, зад… Нет ни того и ни другого, есть только движение. И относительно кого-то,- да, вы будете или впереди или позади,- но относительно себя… Как вы можете относительно  себя сказать, что вы впереди или позади? Вы есть вы. И каждый из вас и есть эта Вселенная, но только не в том понятии, чтобы гордость ваша играла.}
\soul{1,2..}
\people{Скажите, на прошлом… в прошлый раз, мы,когда встречались, переводчик сказал ``Чужой''. Он имел вас в виду? (Ольга)}
\soul{Нет. Мы же не здоровались с вами?}
\people{Да. (Гера)}
\soul{Не говорили того слова.}
\people{Угу. Так чё это, уже… Вы можете нам… (Белимов)}
\soul{Вы совершаете ошибки. Вы прибегаете и начинаете спрашивать, даже не настроившись, забываете, что перед и после, вам надо хотя бы умыться, чтобы стряхнуть то, что было до и что будет. А вы же с мыслями своими приходите, а в мыслях ваших те же имена. Почему они боятся имен? Имена боитесь вы. А они же говорили вам, что они играют  ваши игры. Они подстраиваются под вас.}
\people{Скажите, вы читаете наши мысли? (Ольга)}
\soul{Мы же говорили вам, что мы материя и нам нужно слышать. То есть, только печаль может заставить нас слышать горе, а то будут уже черные краски. Ваши черные краски, которые ранят нас. Запах. Запах боли, который призывает нас, чтобы мы успокоили её. Но предупреждаем вас, мы можем быть и злы, в вашем понятии, мы можем дать большую боль, усилить её. Ничего… чтобы не только мы перегорели, но и вы. Но, но… мысленно…}
1-2-3
\people{А что вы можете нам посоветовать, как нашей группе идти, в каком направлении? Вы можете нам дать методику изучения этих миров? Уже, мы видим, три мира хотят с нами контактировать. Или мы хотим… Что вы нам посоветуете сейчас, в каком двигаться направлении? Опять вопросы или же какая-то другая диалоговая ситуация может создаться? (Белимов)}
\soul{А что мы посоветуем? Вы же не будете следовать советам. Хотя бы задавайте.}
\people{Скажите, вот вы начали с… в общем-то, будем говорить, с волнующих всех нас вопросов – о здоровье. Вообще-то это не подкуп, случайно ли?(Ольга)}
\soul{Нет, это не подкуп. Берущи за руку, и считывающи информацию, как вы говорите - самое самое - – глаза, и берущий за руку. Всё остальное - ложно. Вы можете не так услышать, не так понять Смотрящий же в глаза - видит душу, берущий же за руку - становится им же.}
\people{Скажите, ну.. вы имеете тело, оно имеет цвет?(Ольга)}
\soul{Как вы различаете цвета? Это ваши цвета. Мы, хотя и среди вас, но в вашем понятии, мы бесцветны, потому что вы нас не видите.}
 
\people{А в своём мире вы имеете цвета? (Ольга)}
\soul{Мы же в вашем мире.}
\soul{Ну, чё, вы живёте прям в наших квартирах, или вы в данной квартире поселились? Мы это, ни чё не поймём вообще-то. Вы присутствуете в нашем? Когда … (Белимов)}
\soul{Давайте, скажем по-другому. Мы, одно из сложений полей.}
\people{Полей…}
\people{Каких полей?(Гера)}
\people{…физических полей?(Ольга)}
\people{Или био?(Белимов)}
\soul{Да.}
\people{Полей планеты или полей человека?(Ольга)}
\soul{И того, и того. Как вы можете различить своё, от Земли?}
\people{Ну, вы, значит, существуете в эфирном плане, да, если мы вас не видим?(Ольга)}
\soul{Нет. Земля дала вам поля, и Земля, а не вы. Всё, что вы видите, что называете биополями –  всё земное, а не ваше. Реакция Земли на вас. Если вы видите чёрные краски, значит, и Земля к вам так относится.}
\people{Вы, случайно, не домовой? Есть такое понятие? Нет? (Белимов)}
\soul{Нет.}
\people{Но они есть? Вот в этой комнате живёт домовой? Вы не можете ответить? Это не ваша епархия отвечать на это, да? (Белимов)}
\people{Не те вопросы?(Гера)}
\people{Он не знает. (Лена)}
\people{Знает, но не это… (Белимов)}
\people{Вы, просто, немножко, можно сказать, находитесь в другой, ну, как бы… вибрации, что ли другие?  Или как? Почему вы вас не видим?(Ольга)}
\soul{Давайте, скажем - изменение фаз.}
\people{А вы нас видите всегда? (ольга)}
\soul{Мы идём на запах боли.}
\people{На запах боли…}
\soul{И на чёрные краски, что излучаете вы.}
\people{И вы это убираете? (Ольга)}
\soul{Мы стараемся это убрать, но это не в наших силах.}
\people{Значит, вы…}
\soul{Мы также бессильны, как и вы. Нас нельзя назвать хранителями, нельзя. Мы всего лишь помощники его.}
\people{Помощники кого? (Гера)}
\people{Хранителя.(Гера)}
\people{Хранителя?(Ольга)}
\people{Да. Скажите, а отрицательные эмоции, вот, у животных наших, это… вам без разницы: человек или животное, в принципе? (Гера)}
\soul{Мы приходим ко всем. Это вы говорите ``низшие'' и ``высшие''. Для нас нет этого. Любой просящий должен получить.}
\people{Скажите, у нас три года назад было более веселое настроение, и мы более оптимистично были настроены? Мы та же самая группа. Почему именно через три года вы вдруг объявились с нами?}
\soul{Просто нам уступили. Это - первое.  И нам придётся повториться, как и им. Берущий за руку - пусть он  посмотрит в себя. Назовет ли он сейчас себя счастливым? И что гнетёт его? Берущий за руку… Переводчик понял.}
\people{Нечего было браться теперь. (Белимов)}
\soul{Зачем вы обвиняете?}
\people{Скажите, вы уступили.. .ой… вам уступили… то есть, сегодня мы не можем выйти на ту… на ту… ну, как… на ту цивилизацию, с которой мы общались? (Ольга)}
\soul{Пожалуйста.}
\people{Можем, да? Они вам уступили? (Ольга)}
\people{Но прежде, давай… вот, кто вас брал первый за руку, наверное, задаст вопросы… Наверное, у него есть, что накопилось…(Белимов)}
\soul{Они говорили о зарядах, а вы уже забыли.}
\people{Что, у нас пониженный нынче заряд? (Белимов)}
\soul{Вы забыли многое.}
\people{Ну, всё-таки… (Ольга)}
\people{Давай, спрашивай чего-нибудь… (Белимов)}
\people{Скажите, а вот, допустим, вас трогает наша боль, да? А наша радость на вас откликается?(Гера)}
\soul{Да, мы отдыхаем.}
\people{Вам лучше становится? (Лена)}
\people{Мы редко радуемся, наверное, да?(Ольга)}
\soul{Почему же? Нет тех, кто только грустит. Есть и минуты счастья, и мы стараемся, чтобы их было больше через свою боль. Потому, что мы не сжигаем, как вы. Удивительно, как вы можете сжигать. Вы глядите, что ничто бесследно не исчезает, а сжигаете. Что же вы себе противоречите? Или вы лжёте себе? Вы создаёте огонь, чтобы сжечь что-то ``тёмное''. А пепел?  Выбрасываете на голову другим? Вот какая у вас ``доброта''.}
\people{А вы, как нашу… Вы - как сжигаете нашу боль? (Ольга)}
\soul{Мы берём её себе.}
\people{А вы можете в наших вот в этих контактах, - мы согласны с вами и дольше общаться,- как-то посоветовать, как вести конкретно людям, отдельным индивидуумам или сообществу людей? В чём наши ошибки? Мы могли бы следовать вашим советам каким-либо.(Белимов)}
\soul{Мы говорили с ними и предупреждали нас о конкретностях. И мы согласны с ними.}
\people{Посоветуйте хотя бы тем, кто здесь присутствует, что надо делать, чтобы вас реже вызывать? И…вам…}
\people{Не надоедать. (Гера)}
\people{…не надоедать. А потом уже можно ``бросаться'' и в общество.(Белимов)}
\soul{Видеть больше светлого и в каждом моменте искать лучшее. А вы же, как лавина. Какой-то пустячок раздражает вас – и вы уже собираете целую кучу, целую кучу. Лавина, боль, разочарование – вот вам и ``сердце''. Вы же, вы же… }
1-2-3
\soul{Для чего нужен счёт?}
\people{Связаться с вами.(Лена)}
\soul{*Отвлечь  сознание, наверное.(Гера)}
\soul{Да. И в первую очередь, ваши.}
\soul{}
\people{Угу.(Гера)}
\people{Так, у вас есть разум? (Ольга)}
\soul{У вас есть разум?}
\people{Да.(Ольга)}
\soul{Да? Вы можете взять его и посмотреть?}
\people{Нет.(Белимов)}
\soul{Так нет у вас разума?}
\people{Ну, если мы не можем его посмотреть, это не значит, что его нет. Мы же не видим свое биополе, это не значит, что его нет. (Ольга)}
\soul{Да? А почему же тогда вы отрицаете многое, что вы не видите? А почему вы тогда здесь верите, а всему остальному - нет?}
\soul{Нет, мы вас просто спрашиваем, у вас есть разум? В этом вопросе нет ни сарказма, ничего, просто интересно. (Ольга)}
\people{Следующий вопрос… (Белимов)}
\soul{А мы у вас спрашиваем, есть ли у вас разум?}
\people{Ах, вон как… (Ольга)}
\people{Понятно, - вопрос какой, такой ответ. Ну, мы вот, например, обучаемся, а вы? (Белимов)}
\soul{Мы спрашиваем, есть ли у вас разум?}
\people{Вы ж вместе с людьми? Вот, допустим, на этой планете много-много миллионов лет назад появилось человечество. Вы, вместе с человечеством появились? (Ольга)}
\soul{Нет.}
\people{Нет? Позже?(Ольга)}
\soul{Земля тоже болеет. И те, первые, тоже болели. И пра-материя  тоже знает боли.}
\people{Скажите, почему вот переводчик обычно с нами разговаривает, у него, ну… довольно внятно, а сейчас как-то так…}
\people{Слабовато.(Гера)}
\people{…прислушиваться к голосу приходится. Он ещё к вам просто ещё не привык или каким образом?(Ольга)}
\soul{Нет. Мы слабы. }
\people{А-а, вы слабы и забираете нашу боль, а мы сильны и ``вешаем'' нашу боль на вас? (Ольга)}
\soul{Скажите, вы только…  Вы только человеческую боль переживаете или боль Земли тоже чувствуете? (Белимов)}
\soul{Да. Мы же говорили, что даже материя, просто материя, как вы говорите ``не имеющая души'', - правда мы  такого ещё не встречали,- тоже знает боль.}
\people{Вот, один предсказатель, южноамериканский, пожилой, предсказывает, почему-то, катастрофу земли и конец света 31 декабря 1999 года. Вы это тоже можете ожидать или подтвердить?(Белимов)}
\soul{Сколько раз вы уже заканчивали жизнь? И что может затронуть больше всего вас? Что? Страх?}
\people{Страх.(Ольга)}
\soul{Для того, чтобы знать, стать знаменитостью, вас надо испугать – и вы уже легче запомните человека, пугающего вас.}
\people{И ему легче будет жить.(Ольга)}
\people{Ну, хорошо, мы это учитываем.(Белимов)}
\people{Значит, Земля, всё-таки… (Гера)}
\soul{Представьте, какое будет его разочарование, обида, если этого не случится. И к нам добавится работа. Потому, что он создаст чёрные краски от того, что он ошибся.}
\people{Ну, мы рады этому. (Белимов)}
\people{Скажите, а…(Ольга)}
\soul{1,2}
\people{Скажите, а вы, вот сейчас разговариваете с нами, и с теми, с кем мы раньше общались, тоже вы свя… ну… }
\people{Разговариваете.(Гера)}
\people{…связаны, связаны? Как, как? То есть, вы их… ну, как…}
\soul{Мы разговаривали с ними.}
\people{Сейчас? Вы, как бы, между нами и ими, как бы ``провод'', да?(Ольга)}
\people{Вы промежуточный уровень? Да? (Гера)}
\soul{Нет, мы были слушателями.}
\people{М-мм.. со стороны? (Гера)}
\soul{Среди контактов,- как говорите вы,- нам часто приходилось присутствовать и очищать и вас, и его. Хотя бы - воспоминания о прошлом. Вы их так боитесь…}
\people{Его? (Гера)}
\soul{Берущий же за руку, вспомнил и о них. Поймите, смотреть в глаза и брать за руку – это очень ответственно. Очень. И вы этим можете помочь и навредить и себе, и ему. Откуда пошло у вас рукопожатие? От того, чтобы показать, что рука пуста? Глупости!}
\people{А откуда? (Гера)}
\soul{Это инстинкт. Инстинкт оставил о себе память.}
\people{О единении.(Ольга)}
\people{Соединяется, то есть, да?(Гера)}
\people{Скажите, а вы имеете такое же тело, как наше?(Ольга)}
\people{Формой.(Гера)}
\soul{Да.}
\people{И рост?(Белимов)}
\soul{Почти.}
\people{И цвет? Кожа у вас…(Ольга)}
\soul{Для вас - мы бесцветны.}
\people{Скажите, если вас… если мы сегодня расстанемся друг с другом и с переводчиком, каждый придёт домой… можно вас как-то вызвать или это уже невозможно? И каким образом вы можете проявиться? Могут быть такие беседы с самим собой?(Белимов)}
\soul{Как бы мы хотели, чтобы их было как можно меньше.}
\people{А-аа… то есть вас не стоит вызывать, да?(Белимов)}
\soul{Мы приходим, когда вы в печали, когда вам трудно. Но переводчик сейчас в том состоянии, что мы можем говорить. Когда же он будет в обычном, или вы в печали, вы не услышите нас.}
\soul{1,2}
\people{Вот, прежде, чем мы ещё вопросы зададим, скажите, как нам вести с переводчиком, чтобы вывести из состояния? Нужен ли ему счёт обратный, такой-то… Расскажите вот эти моменты. Вдруг нам придёт, мы не сможем его вывести.(Белимов)}
\soul{Делайте всё то же.}
\people{Скажите, у вас вот амплитуда меньше, ну, рука, допустим, вы у нас уже спрашивали, или мы у вас сейчас спрашивали. Отчего это, вы не скажете? (Ольга)}
\soul{Мы слабее.}
\people{Скажите ещё, а вот…(Гера)}
\soul{Будьте внимательны, они говорили, говорили мы. Вы должны умываться до и после. Каждый из вас и даже просто увидевши, должен сделать это.}
\soul{1,2,1,2 }
\people{Скажите, а вот мы вас не видим и относительно нас, вы бес… ну… бесцветны, мы не можем ощутить вас. А вот мы относительно вас, вы нас как-то…(Гера)}
\soul{Мы тоже не видим вас. Мы приходим на запах боли.}
\people{Вы нас ощущаете, да?(Ольга)}
\soul{Да.}
\soul{А мы вас? (Ольга)}
\soul{Только вы говорите об эмоциональном мире, но есть другой мир. И вы ему ещё не нашли название – мир, соединяющий множество этих полей. Вы кто? Вы – соединяющая семь ваших полей. Вы согласны?}
\people{Да, да, видимо так. (Белимов)}
\soul{Вы забываете об этом. И они соединяют, и только мы, как вы говорите – ``фаза''. И относительно друг друга, мы не видим.}
\people{Никогда не увидим? (Гера)}
\people{Скажите, вот…(Белимов)}
\soul{Почему же? Можем. Можно увидеть нас.}
\people{Каким зрением, извините?(Гера)}
\soul{Зрением? Вы можете увидеть нас при ударе, потому что нам тогда приходится проявлять себя более, отдавать всего, всего…}
\soul{1,2}
\soul{Но вы, вы видите нас…}
\soul{1…}
\people{Скажите, вы можете ответить, кто из четверых нас наиболее привлёк вас по болевым ощущениям? У всех нынче хватает забот, так что ли? Все равны? (Белимов)}
\soul{Берущий за руку…}
\people{Берущий за руку - кто? Первый? Второй? (Лена)}
\people{А-а, берущий за руку. Хотя бы так разделить, у кого больше сейчас заботы, состояние такое болевое?(Белимов)}
\people{Более печальное.(Лена)}
\people{У первого, кто брал…?(Белимов)}
\soul{У берущего за руку…}
\people{Берущий за руку.(Ольга)}
\people{Кто первый - берущий?(Гера)}
\people{Ну, короче… (Белимов)}
\people{Скажите, вы мне сказали, что я должна бояться воды или остерегаться воды. Это что значит? Я могу утонуть?(Ольга)}
\soul{Мы только можем сказать, что бойтесь воды.}
\people{Да я и так боюсь, если честно.(Ольга)}
\people{Ну и бойся. А у неё не было в прошлой жизни именно такой гибели? У нас же много жизней было.(Белимов)}
\soul{В отличие от них, мы не можем вам сказать, о прошлой жизни.}
\people{Вот это интересно уже.(Белимов)}
\people{Интересно… А как же вы учли, что у ней уже… (Гера)}
\soul{Вы забываете, что мы живём очень коротко в своих материях, потому что вы очень часто нас сжигаете.}
\people{Может быть вы и есть тот Мир Огненный? (Гера)}
\soul{Нет.}
\people{А как мы сжигаем? (Белимов)}
\people{Тихо, тихо…(Гера)}
\people{Состояние боли… всякими отрицательными реакциями, эмоциями.(Ольга)}
\people{Это же энергия всё-таки. (Гера)}
\people{Боли тела или боли, вот…ну…эмоциональные… (Ольга)}
\people{Душевные.(Лена)}
\soul{И боли тела, и боль души. Все боли, все. Если у вас боли тела - что, душа радуется?}
\people{Скажите, а как можем вам… вас от работы избавить, допустим, от излишней? Радоваться больше?(Ольга)}
\people{Больше светлого.(Лена)}
\people{Да?(Ольга)}
\soul{К сожалению, вы не можете этого сделать. Вы всегда будете знать, что такое боль, потому, что у вас есть материя. И беда наша в том, что мы должны остаться здесь. Вы уйдёте ``выше'', как вы говорите, ``сбросите свои оболочки'', а мы останемся здесь.}
\people{Это ваша работа?(Ольга)}
\soul{Да. ``Санитары''.}
\people{А вы бессмертны?(Ольга)}
\soul{Тела наши - нет.}
\people{А остальное?(Ольга)}
\soul{Остальное – то же, как и у вас, только вы убиваете друг друга сами, а мы сами себя - нет, а вы убиваете нас. Но не только вы, любая боль убивает нас.}
\people{А скажите, вот сейчас в городе могут быть преступления. Кто-то из вашего общества там присутствует, значит, вмешивается в это, чувствует боль? Ну, когда кого-то убивают, допустим, или же вы вынуждены сейчас туда мчаться? Как вот это понять?(Белимов)}
\soul{Нам приходится делать это, и это тяжело - быть свидетелями ваших печалей и переживать их.  Вы о них забываете. Время лечит. Нас оно не лечит, уж оно остается у нас, и эта чаша когда-то переполняется.}
\people{Скажите,  этим вы эволюционируете, развиваетесь, что ли, вот такой ценой работы?(Ольга)}
\soul{Мы - рабочие.}
\people{Скажите, а в вашей власти, допустим, не идти за  запах боли, или вы, просто…на протяжении заряда…(Гера)}
\soul{У нас никогда не было с этим проблем. Мы не знаем, что такое не пойти, что такое отказать. Это вы можете пройти мимо, мы же не знаем того, нас никто не заставляет. Могли бы, наверное, но такого не было.}
\people{Интересно, что будет с тем индивидуумом у вас, который просто скажет…(Гера)}
\soul{Мы не знаем, не было таких.}
\people{Значит, будем говорить, вы более гуманны, чем мы, так?(Ольга)}
\soul{Наверное, нет. }
\people{Нет? Но вы делаете свою работу.(Ольга)}
\soul{Потому, что нам приходится делать и больно. }
\people{Ах, вон как… (Ольга)}
\soul{Со временем  станет, конечно, для вас лучше, но то, что мы сделали вам боль  - остается у нас, и ещё более, чем приняли от вас. А вы же ещё и добавите, скажете,  что были темные…  и этим добьёте нас. Хорошо, если мы выживем, а чаще, мы умираем, получаем новые. Мы помним. Помним всё прошлое. Но мы не можем сказать конкретно о ком-то из вас, потому что мы не закреплены за вами. Кто-то, в вашем понятии, живёт полчаса, кто-то год – кому как повезло, но после смерти он приходит и уходит к другому, а мы - к нему. И неужели вы думаете, что когда вы плачете, всего лишь только один приходит к вам. Всем миром приходим и успокаиваем вас. А что делаете вы? А вы нас потихоньку убиваете. Почему? Особенно женщины, они очень любят ``раздувать'', они начинают жалеть себя. Самое страшное – это жалеть себя. Потому, что это обычно делается со слезами.}
\people{Вы очень правы. Мы с этим всё время сталкиваемся, но никак не можем внушить, что вы правы. Скажите, а там, где болевые ощущения наиболее выражены, допустим, в больницах, где операционные, там режут людей, сейчас в Чечне война, вас, наверное, там сонмище? Вы там очень… Вы, что там? Помогаете или подпитываетесь, развиваетесь, размножаетесь при этом? (Белимов) }
\people{А всё-таки вас нельзя, так сказать, термином каким-то назвать, да? В литературе, который есть.(Ольга)}
\soul{Но мы бы не хотели носить его.}
\people{Не хотели?(Ольга)}
\people{Но он назван?(Белимов)}
\soul{Они не произносят имён, по той же причине не произносим и мы.}
\people{Но, хотя бы…(Белимов)}
\soul{А вы ярлыки даете достаточно быстро…}
\people{Но вас назвать можно…(Ольга)}
\soul{…а потом, начинаете расписывать.}
\people{Вас можно назвать ``строителями''?(Ольга)}
\people{Предохранитель, но… Мы так и назовём вас предохранители, допустим, в своих разговорах, но все-таки это предохранители отрицательных эмоций, отрицательных, так, видимо, да?(Белимов)}
\people{Скажите, вот… Можно?… Я, однажды… У меня болело… допустим, печень болела, и мне одна моя знакомая начала энергетически вроде как боль снимать. И я почувствовала, что меня по голове как бы немножко ударили, а ей по руке,- она тоже это почувствовала. Вот, это не вы ли были? Можете так вот…?(Ольга)}
\soul{Может быть, но говорящий с вами - не был.}
\people{Нет, но я имею в виду, что… вот, из вашего…(Ольга)}
\people{Окружения.(Белимов)}
\soul{Может быть. Мы не знаем. Нас же множество.}
\people{Вы… мы можем вас ощущать, да, каким-то образом, значит?(Ольга)}
\soul{Да. Иначе как бы вы перестали плакать, и как бы мы могли залечить вам раны?}
\people{Они были тоже… то есть, не вы одни, из вас… на плечо могут руку положить и погладить. Тоже? (Ольга)}
\soul{Тоже.}
\people{Ну, вас столько же, сколько и людей, то есть, нас равное количество, или вас больше?(Белимов)}
\soul{Мы же говорили, что не только к людям приходим…}
\people{Ага…(Белимов)}
\people{Вас, наверное…(Ольга)}
\soul{… а ко всем, кто рождает боль. И если бы вы могли слышать, как стонет Земля, как плачут деревья, как умирает трава, как кричит срезанная роза… Вы же не слышите это, и это ваше счастье.}
\people{А вы там и слышите, и знаете?(Белимов)}
\soul{Это боль, боль, страдания.}
\people{Скажите, деревья обладают интеллектом? (Белимов)}
\soul{Вас так интересует это?}
\people{Ну, если…(Белимов)}
\soul{Вы даже не ответили нам, есть ли у вас разум.}
\people{Но, если она…  деревья чувствуют боль и даже люди некоторые чувствуют боль спиленных деревьев, крики их – значит, у дерева есть что-то, рассудочное что-то…(Белимов)}
\soul{Мы не встречали ни одной материи, не имеющей души и не имеющей чувств. Мы такого ещё не встречали.}
\people{Скажите, а разум…(Ольга)}
\soul{Даже атом издает боль, а вы научились выдавать её.}
\people{Расщепляя атом, да?(Ольга)}
\soul{Тем самым - убивая. И это вернётся к вам, а мы должны стать стеной и не пропустить. Сколько миров губите… Сколько?  И не знаете… А мы должны страдать. Вот и подумайте, должны мы вас любить или нет?}
\people{Кстати, вы нас, видимо, недоброжелательно… Не любите?  Так, видимо, да?(Белимов)}
\soul{Да, как же?(гера)}
\soul{А за что нас  любить? (Ольга)}
\soul{В вашем понятии, в нас нет любви к вам.}
\people{А к кому есть? (Ольга)}
\soul{У нас нет любви к себе. У нас нет вашего понятия любви, потому что слово, что вы придумали, очень мало. Что такое любовь в вашем понятии? Вы исковеркали и это. Раньше,- ну, возьмем, давайте, те же 15 000,- оно было чище. Говоря же сейчас ``любовь'', нам приходится приходить. Уж произнося его, вы уже создаете тёмные краски, потому что вы уже забыли, что такое чистая любовь.}
\people{А что такое ``чистая'' любовь?(Гера)}
\soul{Учитесь. Мы вам не можем этого показать, этого вам не покажет никто, потому что вы должны сами. А как вы можете увидеть, если вы не знаете, что это – вы просто не узнаете. Как вы можете увидеть то, чего вы не можете ощутить? Вы же будете смотреть опять не душой, а глазами.}
\people{Я считаю, что чистая любовь – это, так сказать, она сама по себе, уже самодостаточна, и ей, в принципе, ничего не надо, как бы, подкреплять её. Она или есть… (Гера)}
\soul{Вы всегда говорили красиво, но почему-то не делаете. А вы больше говорите, но гораздо больше, чем делаете. А когда вы говорите, о высоком, вы говорите столь глупо – нам приходится опять работать.}
\people{Значит, всё, что бы мы ни сделали, –  для вас работа найдётся, так получается? У нас, сейчас всё исковеркано. (Гера)}
\soul{Ну, если вы так устроены…}
\people{Продолжайте,- `` если вы так устроены'', - что?(Гера)}
\soul{Если вы, пустым, говорите о высоком… Вы, разве, можете увидеть это высокое? Вы создадите только один из вариантов пустоты, а дадите, конечно, высокое имя. Вы будете говорить равнодушно о любви, - ну и что? Просто равнодушие вы назвали другим словом  –  любовью – и всё, вот нам и работа: придти и очищать от того, что вы уже наго…. (наговорили?)}
\people{Скажите, вы сказали… у меня будет недолгая жизнь, и что я слаба бороться, а вообще это исправить можно или нет, это не возможно сделать в этой жизни? (Лена)}
\soul{Первое, что вы должны, – перестать бояться. Второе – хотя бы немножечко, хотя бы чуть-чуть, забудьте об эгоизме, что гложет вас.}
\people{“Фиг'' его забудешь…(Белимов)}
\soul{Это и есть тот яд, который столь сладкий, что  уничтожает вас. Научитесь делить не только горе, но и счастье. Научитесь видеть во всем хорошее, а не сразу искать плохое. Из множества вариантов вы выбираете худший. Почему?}
\people{Есть такой грех. (Лена)}
\soul{Потому что вы, вы, вы…}
\soul{1,2}
\soul{Вы не внимательны.}
\people{Так.(Белимов)}
\people{Задаёте вопрос, а сами не слушаете.(Ольга Белимову)}
\soul{Другой? Другой бы боялся. А сейчас он боялся? Сейчас он улыбался.}
\people{Переводчик?(Ольга)}
\soul{Вы невнимательны и не делаете выводов. Говорите…}
\people{Скажите, вы не докончили свою мысль. Продолжите.(Гера)}
\people{Да, вы… не завершили. Сказали что…(Лена)}
\people{А девушке, что ей надо делать?(Белимов)}
\soul{Говорите.}
\people{Спрашивайте.(Ольга)}
\people{Ну, в принципе, я уже всё узнала.(Лена)}
\people{Насчёт… Что там? (Гера)}
\people{То, что это может…(Лена)}
\people{Выбирает она самые…(Гера)}
\people{Выбираю я сама, да?(Лена)}
\people{Самое худшее.(Гера)}
\people{В общем, ситуация такая: будешь эгоистична – рано помрёшь, вот и всё.(Белимов Лене)}
\people{Скажите, ещё вот на такой вопрос, ответьте, пожалуйста, вот, а все отрицательные эмоции, в принципе, я читал, что вырабатывается при этом в организме яд, ну, токсины, и они убивают сам этот организм при отрицательных эмоциях.(Гера)}
\people{Естественно. Империл - знаешь такой яд? (Ольга)}
\people{Да, говорят – это. Это верно, вообще-то?(Гера)}
\soul{Да, - от того, чего мы чистим вас, но мы не справляемся,  вы слишком сильны в этом плане.}
\people{Мы слишком много вырабатываем яда, да?(Ольга)}
\people{Давай, я о вирусах спрошу.(Белимов)}
\people{А слишком большая радость, она ведь тоже, в принципе, губительная для организма, может быть? Очень большой смех, допустим, или, там…(Гера)}
\people{Излишний. (Ольга)}
\soul{Это уже не радость, это уже одержимость.}
\people{Угу. (Гера)}
\people{Значит…(Ольга)}
\people{Скажите, вы, наверное, очень в этом разбираетесь, вот, проблема вирусов. Вам она известна? Что это загадочное, в общем-то, явление, никакая энциклопедия не признаёт.  Природа не знает вирусов. Что это такое, на ваш взгляд?(Белимов)}
\soul{Это чужое.}
\people{Это чужой?(Ольга)}
\soul{Инородное. Оно приходит и хочет завоевать вас, хочет жить в вас.}
\people{Паразиты.(ольга)}
\soul{Есть более хитрые, которые не губят вас, а приживаются к вам. Зачем им губить свой дом? Они придут, и будут жить в вас и даже могут вылечить вас от кое-каких болезней. Для чего? Чтобы получше благоустроиться, и будут потихоньку питаться вами. А есть, как вы говорите, глупые, когда приходят и сразу хотят съесть вас, тогда в болезни вы умираете. Хорошо, если они опомнятся, остановятся, увидят, что они натворили, и тогда они даже могут помочь, и опять будут лечить вас, чтобы восстановить этот дом, чтобы в нём жить. Вот вам и вирус. То же самое, что делаете и вы. Вы пришли на землю, начали пахать, начали губить, теперь испугались – ``так что, нам может и не хватить?'' - и теперь стараетесь всё снова.}
\soul{}
\people{Хорошо. А вирус - это существо или вещество?(Белимов)}
\soul{Мы же говорили вам только что. Разве может - просто вещество?}
\people{Но, а есть предположения, что вирусы… попали к нам из дальнего космоса, так ли это?(Белимов)}
\people{Нет.(Ольга)}
\soul{Да их множество!}
\people{Они есть везде.(Ольга)}
\people{У людей сложилось впечатление, что вирусы сугубо враждебные человечеству, так ли это?(Белимов)}
\people{Только что говорили.(Ольга)}
\soul{Нет. Зачем же? Зачем? Они могут просто придти и жить в вас, просто им надо же где-то жить. А то мы дойдём до того, что Земля,  а вы – вирус на ней.}
\people{Всё правильно.(Белимов)}
\people{Похоже.(Гера)}
\people{Но, значит, есть полезные особи среди вирусов?(Белимов)}
\people{Конечно. Только что сказали же.(Гера)}
\people{Так.(Белимов)}
\people{В принципе, это не страшно, что они живут в нас, если они нам помогают жить? (Гера)}
\soul{Да в вас живёт столько множество!}
\people{А есть утверждение, что вирус меняет состояние ДНК, клетки ДНК, так ли это?(Белимов)}
\soul{Нет. Кем бы вы тогда были?}
\people{А вирусы чужды… чужды всему земному. Это не так?(Белимов)}
\soul{Вы повторяетесь.}
\people{Так. Обнаружено 1500 видов вирусов. Вы знаете приблизительное число, их больше, меньше?(Белимов)}
\soul{Гораздо больше.}
\people{А что вы знаете о вирусе Эбола?(Белимов)}
\people{Имен не называйте. (Ольга)}
\people{Это название местности.(Белимов)}
\people{Это имя. (Ольга)}
\soul{Вы уже назвали.}
\people{А вам тоже имена не желательно упоминать, да?(Белимов)}
\soul{Нам - тоже.}
\people{Они же говорили.(Гера)}
\soul{Удивительно, как вы внимательны! Что - имя? Имя – это дверь, в которую можно придти. И непрошеные придут, и что сделают? Вы уже назвали имена, потому мы пришли и говорим вам. Мы устали, устали от вас. Приходим и предупреждаем вас.}
\people{Вы не хотите с нами сейчас разговаривать?(Ольга)}
\soul{Мы не хотим имён ваших, не хотим своих имён, никаких имён. И хотя без этого нельзя, уж хотя бы слово, любое слово – это уже имя, ну. хотя бы…}
\people{Скажите, есть гипотеза, что вирусы – это, своего рода, биороботы. Так ли это, и как они из слуг превратились в убийц людей?(Ольга)}
\soul{Их множество. Их множество и разные. И среди вас есть биороботы, среди животных есть они и среди… среди всего есть, как вы говорите, биороботы. И всё относительно в этом мире. Относительно бога, если подумать, только логикой вашей, то вы - тоже его биоробот.}
\people{Ну, это да. А среди нас, вот здесь присутствующих, есть такие?(Ольга)}
\soul{Вы уже испугались.}
\people{Ну, конечно. (смех)}
\people{Мы, просто из осторожности спрашиваем. Скажите, а могли ли вирусы-биороботы счесть, что человечество развивается ``неправильно'' и включилось в борьбу с этим человечеством? Мы, вроде, ощущаем такую борьбу.(Белимов)}
\soul{Да нет, вы просто хотите найти виновных в своей гибели. Ну, конечно же, надо взять кого-то другого - пришёл и вы  уничтожаетесь, - вы же не будете признавать, что вы сами себя убиваете, что вы и есть самоубийца или убийца. Как вы живёте? Вы говорите: ``Среди нас нет самоубийц''. Вы же сказали. А вы подумайте, посмотрите, что вы творите? Вы только этим и занимаетесь!}
\people{Ну, да…(Ольга)}
\soul{1,2}
\soul{Снова работа.}
\people{У кого слезы?(Белимов)}
\people{У собаки?(Ольга)}
\people{У переводчика?(Гера)}
\people{Вы можете сказать, что сейчас с ним происходит? Почему? … Он не работает…(Гера)}
\soul{Глаза.}
\people{Глаза?(Лена)}
\soul{Мы же попросим открыть их.}
\people{Ну, мы можем продолжить вопросы?(Белимов)}
\people{Нет?(Ольга)}
\people{Скажите, чем объяснить загадочность вспышек вирусных, вот в частности, нынешний приход гриппа, причём в очень тяжелой форме. Можете это не стандартно объяснить, что случилось?(Белимов)}
\soul{А к чему же приводят ваши ``вспышки'' - ваши войны?}
\people{А это близкое, да? Войны тоже вызывают?(Белимов)}
\soul{Да они такие же, как вы. Они тоже хотят завоевать больше или меньше. Дальше будьте внимательны, -  Солнце  –  помогаем. Боль ваша – помогаем. Вы можете вспомнить то время, когда была счастливей Земля, и были вспышки эпидемий? Было такое?}
\people{Да нет, скорее всего, в несчастьях эпидемии бывают. Значит, сейчас Россия переживает тяжёлые времена и поэтому и болезни? И другие страны - то же самое, да?(Белимов)}
\soul{Вы ослабляете себя, вы становитесь слабыми. Когда вы становитесь слабыми?}
\people{Тьмой?(Ольга)}
\soul{В грусти, печали  – тогда вы становитесь слабыми, и тогда, вам тяжелее бороться с ними. Да, и явления природы тоже влияют на то. К сожалению, многие из вас просто марионетки, которые зависят от внешней среды: село Солнце – и вы уже ``не те''. Похолодало или стало слишком жарко – вы уже ``не те''. О какой же духовности вы можете говорить, если вы болеете из-за каждого движения Земли? Вы скажете: ``Экстрасенс. Чувствительны, значит, руки''. Да нет, просто слишком слабы.}
\people{Скажите, человечество проигрывает войну с вирусами? Может быть окончательная победа вирусов над человечеством?(Белимов)}
\soul{Нет, это даже не выгодно вирусам.}
\people{Так вы хотите сказать, что экстрасенсы – это люди больные, в принципе, ослабленные?(Ольга)}
\soul{Нет, мы как раз говорим противоположное. Это вы говорите, что - вот он чувствителен к изменению погоды – экстрасенс. Нет, он просто слаб, потому чувствителен.}
\people{А надо, значит, чтобы не быть чувствит… Вернее - как?(Гера)}
\soul{Вы должны чувствовать, вы должны всё это замечать, но это не должно переходить в боль или в болезнь. Вы должны замечать, что пришла зима или лето, конечно, должны, но, это же не значит, что вы должны сразу болеть?}
\people{Угу. Ну, значит, если, допустим, я буду относиться к летнему солнцу, так сказать, с весёлыми взглядами, то оно мне не нагреет голову, скажем так?(Гера)}
\soul{А вам надо. Вам надо. Вы боитесь Солнца.}
\people{Конечно, я падаю в обморок и кровь  носом идёт.(Гера)}
\people{Скажите…(Белимов)}
\people{Вы говорите… Как же в такой ситуации, раз оно на меня вот так действует? Я же, в принципе, не выхожу там с проклятьями в его адрес на улице.(Гера)}
\people{Ты должен себя…(Лена)}
\soul{Да? Вы лжёте даже себе и сейчас.}
\people{Нет, я просто боюсь и опасаюсь, что оно нагреет, но, я же не говорю там, чтобы…(Гера)}
\soul{И никогда не ругаете?}
\people{Ну, вообще-то, может, и было, да. Ну, ладно…(Гера)}
\soul{Вы слишком внимательны. Вы слишком внимательны к себе, и это убивает вас.}
\people{Слишком внимателен к себе?(Гера)}
\soul{Да. Когда вы были маленьким, когда вы падали, ушибали коленку, вы потерли, побежали дальше. Подсказка переводчика. И всё. А что делаете сейчас? Создаёте множество проблем: как, как бы, да не то, не да это, и обязательно себе что-нибудь найдёте. А достаточно назвать имя какой-нибудь болезни, и она к вам может придти, потому что вы её позвали, и позвали даже по имени.}
\people{Значит, получается, то, что я вам сказал, вот, вы меня спросили, чем я болен, а… почки и сердце, это, в принципе, уже есть, да? (Гера)}
\people{Вы не сказали.(Ольга)}
\people{Вы не сказали, прав я или нет.(Гера)}
\people{Ты не сказал название болезни.(Ольга)}
\people{Какая разница? (Гера)}
\people{В чём разница?(Ольга)}
\soul{Да, вы правы. Но мы не отрицали того, если бы вы были внимательны, мы просто добавили и головные боли.}
\people{А… Да, да.}
\people{Скажите, вот…}
\people{Минутку. Не связано это с табаком?(Гера)}
\people{Не маши пальцем!(Ольга Гере)}
\soul{Скорее - нет.}
\people{А? Нет?(Гера)}
\soul{Табак действительно опасен для вашего здоровья. Но, благодаря рекламе, он стал более опасен потому, что вы уже занимаетесь самовнушением.}
\people{“Курение опасно для вашего …'' Ясно. (Гера)}
\people{Скажите, вот у меня бывают часто головные боли после всяких различных праздников, когда вот, допустим, несколько дней подряд люди гуляют, пьют, в общем-то, ну, в основном, и после этого у меня очень сильно болит голова. Это как-то… это как-то…(Ольга)}
\people{Это её чувствительность, да? Повышенная? (Белимов)}
Конец контакта.
Обсуждение.
\soul{Ген, ну-ка давай рассказывай, чё было, давай, садись. Помнишь контакт? Ты помнишь что-нибудь?}
\people{Чё-то помню.}
\soul{Что? О чём разговор был?}
\people{Про здоровье чё-то.}
\soul{Про здоровье, правильно.}
\soul{Что-нибудь видел, картинки видел?}
\soul{Тебе картинками шло чё-то или как?}
\soul{Может ему умыться лучше сначала?}
\soul{Сейчас, мы все помоемся.}
\people{Картинки я не видел. Я не помню.}
\soul{Тебе не очень приятно?}
\people{Что-то такое было.}
\soul{Гена, а ты сам не плакал, ничего?}
\soul{Не чувствовал никаких отрицательных эмоций?}
\people{Да нет, как обычно.}
\soul{Как обычно, ты думаешь? У тебя ощущения, что как обычно прошел сеанс?}
\soul{Ничего необычного не замечал?}
\people{Ну, как обычно.}
\soul{Ничего, да, такого?}
- Ничего особенного?
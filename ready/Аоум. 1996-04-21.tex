Аоум. глава 21. 04. 1996г
Георгий Губин
  21.04.1996 

* (Гера) … ну, мне сказала наша знакомая в, принципе, что вы говорили ей

-  Вы нас прослушайте внимательно, вы не найдёте того.

* (Лена) Простите, вы про счёт сказали, да?

-  Мы говорили о счёте, и говорили о трёх. Мы же всегда говорили вам, считайте.

* (Лена) Ну, я тоже так считаю…

-  И не ограничивайте.

* (Лена) Да, если чувствуешь, то, лучше, нужно посчитать.
* (Александр) Это знакомая ошиблась, или она обманулась как-то? 
* (Лена) Ну, это не важно.
* (Гера) Сегодня 21 апреля.
* (Александр) Сегодня праздник Красной Горки (Пасха) прим. Как вы относитесь к этому празднику, к тому, что проводим контакты именно в этот день?

- Мы говорили вам о религии, и говорили о наших богах. И говорили, что у нас нет их. Потому, у нас свои праздники, у вас свои. И даже среди вас не все празднуют, и не все признают.

* (Гера, Лена) Да. 

- Мы уважаем и вас, и ваши праздники. И было бы глупо… а вы часто делаете глупости, когда нарушаете.

* (Лена) Скажите, а всё-таки, такой великий праздник, как день Пасхи - достоин внимания, или, всё-таки, он - один из таких же, как и обычных праздников у нас?

- Издревле он был всегда. Почему же сейчас не стоит? У вас было множество праздников древних, и которых вы уже не справляете. То, что держится веками, то и нужно.

* (Александр) А из чего вышел этот праздник Красной Горки? Что лежит в основе его? Историю можете перелистнуть назад?

- И зачем?

* (Александр) Чтобы лучше разобраться в этих религиях, которые нас окружают сейчас. В этих праздниках.

- Удивительно, вы создали религии, мы их не признаем, и мы должны помочь вам разобраться в них, в ваших же сочинениях!

* (Александр) Значит, не можете.

- Мы не хотим. 

* (Лена Александру) Зачем это нужно?

-  Поймите, праздник для вас будет только тогда праздник, если признаёте его, и не больше. И каким бы он не был великим, и чем бы он ни был ознаменован…(срыв)

* (Лена) 1..
* (Александр) Помогите нам разобраться, всё-таки, где грань между пророчеством и, если так можно говорить, накаркиванием предстоящих событий?

- Никакой.  Гранью является сила. Сила слова.  Если было достаточно много сил – уже пророчество. Если же это было просто сказано вскользь, то это - просто карканье,-  ваше выражение. И хотя это может сбыться, это не будет пророчеством, это будет всего лишь проклятие. И “проклятие” -  не обязательно что-то плохое, можно проклять и на счастье.

* (Гера) Можно спросить, вот, насчет праздников? А, действительно, когда великие такие праздники, люди празднуют, а в душе его нету, этого праздника, то он не ощущается почему-то, не понятно…

- У вас его и нет. 

* (Гера) Правильно, у нас его нет. Так этот праздник может быть в любой день, в принципе, когда у тебя хорошее настроение, поёт всё в душе, так сказать.

- Нет, здесь вы ошибаетесь. 

* (Гера) Да?

- Когда вы будете праздновать один и рядом больше никто, откуда вы возьмёте силы для празднования самого себя?

* (Гера) Почему – самого себя?

- Для чего вы ходите в церковь? Для чего? Вы говорите об энергии, вы говорите об окружающей (обстановке. Прим.), вы говорите о питании себя и души своей.  И теперь представьте, вы приходите в пустую церковь, где нет никого, или вы просто празднуете Пасху в любой день, где вы не видите поддержки. Велик будет этот праздник?

* (Гера) Ну, это зависит от внутреннего ощущения. Почему,- может быть и велик праздник.

- Нет.

* (Гера) Можно и двоим спраздновать так, что…

- Откуда берётся ваше внутреннее ощущение? Только изнутри?

* (Гера) Нет, от внешних факторов каких-то…и от внутренних.

- А что же вы тогда говорите?

* (Гера) Ну, внешние факторы могут не совпадать с праздником, так сказать, Пасхи или…

- Значит, это говорит только о вашей слепоте, или нежелании увидеть, или от вашей «избранности», только и всего. Вы желаете видеть или то, или другое. Если вы не встали с той ноги, то вы будете обвинять весь мир, весь мир, но не себя, что у вас нет сегодня праздника. Но если вокруг вас и в себе вы ощущаете праздник  - это и есть резонанс,- а иначе - односторонняя любовь, не приносящая добра.

* (Лена) То есть, бывает и такая любовь - односторонняя?

- А разве нет?

* (Лена) Ну, я не знаю вообще-то.
* (Александр) Хорошо, я ещё хочу вернуться к своему вопросу, к карканью, к пророчеству. Вы сказали “никакой разницы”. А где грань между пророчеством и ясновидением?

- А мы когда-то говорили о единении, мы говорили о единстве. Мы говорили… а вы не слышали, и вы ищете грань. Что вы называете пророчеством, ясновидением и карканьем? Вы можете сами сформулировать? Найдите аналог слова – пророчество.

* (Гера) Предвидение.
* (Лена) Пророчество – то есть, видеть правду, как бы. Видеть истину. Видеть то, что будет на самом деле, а не… Накаркать, это – во зло что ни будь сделать.

- Разве? А когда вы накаркали, и всё совпало, разве это не было правдой?

* (Лена) Ну, это зависит от того, с каким зарядом…

- От чего это зависит? Если вы пожелали кому-то разбить лоб, и он расшибил, - это была ложь? Он расшибил.

* (Лена) Ну, это от того, что мы так пожелаем. А на самом деле могло этого и не произойти?
* (Гера) То есть, если бы не пожелали, произошло бы это или нет?
* (Лена) Да. Вот это и есть пророчество, может быть?

- Пророчество, или карканье?

* (Лена) Ну, вот и хотим узнать, где грань между....

- Хорошо. Если вы врагу, и всеобщему врагу, пожелаете расшибить этот лоб, и он расшибёт его, то все назовут вас пророком. Если вы расшибёте таким путём лоб кому-то другому, то вас будут обзывать, что вы накаркали. Здесь вы видите грань?

* (Александр)То есть, получается, что…

- Видели вы грань между разведчиком и шпионом?

* (Александр) С какой стороны смотреть.
* (Лена) Да.  
* (Александр) Хорошо. Так, получается - ясновидение, в том смысле - как уметь видеть будущие события. Просто видеть…

- Ясновидение, это просто – ясно видеть.  А как видите вы? Вы одеваете очки, которые называете “науками”, “хиромантией”, и множествами другими путями. И что вы видите? Вы видите, всего лишь отрывки, кусочки, а потом начинаете рассуждать. И когда они не подходят, вы отбрасываете и тут же забываете их. Или ещё хуже,- вы, начитавшись Сур, подбираете к ним время, когда это должно произойти. Вам приводить пример, или вы сами назовёте?

* (Гера) Понятное дело, не за горами.
* (Александр) Значит, просто заглянуть в будущее человек не может? Узнать, какие события будут в будущем.

- Мы когда-то говорили вам, что можно заглянуть в будущее множеством путей.  Множеством! Можно просто арифметикой, чисто математикой и теорией случайностей и не больше. И никакой мистики.  Можно чисто мистически, но вы будете уверены, кто показал вам это будущее? Будете ли вы в этом уверены? Вы можете ощутить это в себе, не обладая ни мистикой, ни той математикой, не обладая никакими талантами. Ребёнок - он чувствует, когда обожжётся, но всё равно обжигается. И вы должны были бы обратить внимание, что ребёнок, перед тем как удариться,- проследите за ним, проследите и вы убедитесь, что он знает, что ему будет сейчас нанесена боль, и всё равно он идёт к этому.  Вы часто любите говорить: если бы, (вернуть прошлое. прим) - я бы прожил также. И вы действительно проживёте также. К сожалению. И, к сожалению, вы это говорите с гордостью. Почему? Потому, что вы боитесь меняться.  И эта боязнь поменяться, не даёт вам действительно истинное будущее. Если вы сейчас “червь”, и вам нравиться им быть, вы никогда не увидите себя другим. Тем более, если вы уже расправили крылья, то не захотите встретить время, когда этих крыльев уже нет. Продолжать ли?

* (Гера) Да.

- Вы говорите о гранях. А знаете ли вы грань? Грань самому себе?

* (Лена) Самое сложное, пока.

- Можете ли вы сказать, что заставило задать ваш этот вопрос? Именно этот вопрос, а не другой, именно этими словами.  Найдите мне корни мысли, которые заставили их прозвучать. Вы сможете их найти? Нет, вы не сможете, ибо эта мысль была рождена множеством другим. И, как вы говорите, «ища» вы придёте к моменту рождения. И тогда придётся, (выяснить прим.) простите за выражение, с каким настроением вы были зачаты.

* (Гера) Ну, а если и это выясниться, то какие ещё могут быть причины?

- Множество. Окружающая среда,- вы не забывайте об этом.  Вы сами говорили, и мы множество раз говорили вам о единстве. Вы говорите об реинкарнации, и множество вы рисуете фальшивых картинок. Многие рождаются впервые, и удивительно, что они могут вспомнить множество прошлых жизней, которых никогда не существовало! Почему? И знаете почему? Потому, что у вас существует понятие моды. Самое страшное - мода, переходящая в привычку. Вы, благодаря моде, увлечению её, стали более религиозны. Пройдёт мода, пройдут года, и вы плюнете на всю религиозность, и плюнете на все философии, и будете опять жить, как и жили прежде. И, простите, ребёнок честнее от того, что он ведёт себя так, как он есть. Пусть он даже плохой ребёнок, но он не врёт самому себе.

* (Лена) Интересно, а где грань, тогда, межу ребёнком и взрослым?

- Где?

* (Лена) Да.

- Это - когда вы начинаете говорить, подражая взрослым.  Это - когда вы мечтаете стать взрослым, не видя ценности детства. И когда вы станете взрослым, вы уже будете жалеть об этом. Разве не было этого? Было и будет всегда! Ребёнок - только тогда ребёнок, когда он ещё не солгал. И первая ложь, это уже первое взросление.  Мы когда-то вам говорили, что ребёнок с первой улыбкой начинает учиться любить. Вы помните?

* (Гера) Угу.

- И с первым криком он забывает более половины. Вы помните?

* (Гера) Угу.

- Вы должны были б тогда и помнить, что первая произнесенная ложь – это уже не ребёнок.

*(Гера) Да, было такое.

- Потому и говорили вам о Царствии Небесном, и о детях.  (имеется в виду слова Христа: пока не станете как дети, не войдете в Царствие Небесное прим.)

* (Гера) Вроде как, получается, уж если ты единожды солгал, так сказать, назад не вернуться?

- Вы же говорили, что за каждое слово будет дан ответ.

* (Гера) Да.
* (Александр) А первый крик, первая улыбка… А скажите, как относиться к первому мату, который произнес тот же ребенок? Что такое мат, и как к нему вообще относиться,- как к взрослому, так и к детскому мату?

- А теперь представьте, теперь представьте,- ребёнок улюлюкает и наконец, произносится что-то матом. Вы будете его за это бить? Это было просто звук и не больше. А когда вы этому звуку придадите значение, вот тогда, и только тогда!

* (Александр) И как плохо это, произносить такие слова сознательным образом?

- А вы же говорили о резонансе, вы говорили о волновых процессах, вы говорили о вибрациях, и вот вам - пожалуйста. Вы когда-то спрашивали о проклятиях, и мы вам отвечали, что - да, они действительно есть, и даже физически, и даже физика и ваши науки скоро докажут это. Это действительно так. Это – колебания. Но они воздействуют только в том случае, если будет сказано осознанно. А осознанно, это значит, им будет дана энергия, а не просто хаос. Направленное излучение, если хотите.

* (Гера) Ну, то есть, нам, как раз сознание и дано было для того, чтобы мы творили не физически, а в таком ракурсе - осмысленно.

- Мы говорили вам, сознание дано вам для того, чтобы вы осознали и умели управлять материей. И вы должны были бы помнить, что мы сказали, что вы, когда пришли сюда, вам понравилось, и вы забыли истинное значение вашего прихода. И теперь, когда к вам приходят, и говорят, и зовут вас, вы не помните, не узнаёте, и смеётесь, и не веруете.

* (Гера) Это вы о снах?

- Не только.

* (Гера) А ещё что может быть?

- Вдохновение, минуты печали не связанные материей.

* (Гера)  А, вот, ну, вот зовут нас, допустим, а как мы можем уйти? Допустим, мы узнали всё. А как мы сможем уйти? Из тела что ли выйти? Или как? Или чего?

- Сколь часто мы говорили вам, и сколь часто предупреждали вас, что не говорим о теле, а говорим о душе. А вы склоняетесь к самоубийству.

* (Гера) Я не о том.

- О том! Именно о том, когда вы говорите о теле - это начало. 

*(Гера) Понятно.

- И скоро вы договоритесь до того… А многие идут к тому.  Идут с верой в Бога, убивая себя. С верой в Бога! А нужны ли вы Ему там - покойнички?

* (Лена) Да, не нужны.
* (Гера) Простите, это вы говорили, что материя - это всего лишь “кусок мяса”? Как-то было не так давно сказано, в этом году.

- В вашем? ( имеется в виду разность времени в разных мирах) Прим.

* (Гера с задержкой) Ну, да, в нашем году, в нашем.

- А вы не помните, что мы говорили, что ваша материя, это колония живых существ? Огромная колония живых существ. 

* (Гера) Да, помню.

- Вы не помните это?

* (Гера) Помню.

- А вы должны были бы помнить, что атом, это уже - семья.  

*(Гера) Да.

- Клетка - это множество семей. Если хотите, пусть будет - “деревня”. Орган - пусть будет “страна”, а ваше тело, пусть будет - “планета”.

* (Гера) Ну, да, можно так сказать.

- А теперь представьте, если это планета, тогда - кто вы? Кто? Вы – ваша душа.

* (Гера)  “Сборная” тогда получается из всего этого?

- Сборная? А мы говорили, что душа материальна?

* (Гера) Нет.

- Это вы когда-то пытались нам доказать о четырех граммах? (имеется ввиду масса души) Прим.

* (Гера) Нет, не я лично. Насколько я помню.

- Так вот, вы должны научиться управлять этой планетой. Вы же учитесь управлять машинами? 

* (Гера) Да.

- Учитесь. И прежде, чем научиться, вы разбиваете их, и разбиваете себя. Рано, или поздно, всё-таки научитесь.

* (Александр) Скажите, к понятию нашему «душа», какая часть человеческая ближе всего к нему?
* (Лена) Сердце, конечно.
* (Александр) Кроме сердца, есть ещё что-то, наверно?

- А мы только что говорили о нефизическом, а вы теперь предлагаете ещё и кусочек мяса. 

* (Лена) Нет, но вы же говорили, что сердцем вы можете понять многое.

- Сердце! Что вы называете “сердцем”? Не те же мышцы и желудки, что качают кровь, вы согласны?

* (Лена) Нет, нет, конечно. (а смысле – согласна. Прим.)

- Так не ищите эту душу, не ищите её домишко где-то в одной из этих клеток. Если хотите уж так найти, так тогда уж ищите во всём. Душа в каждой клеточке, в каждом атоме. Душа это та самая энергия, которая заставляет двигать, двигать осознанно. Помните, мы говорили вам, что когда умирает ваше тело, уходит душа, создаётся новый вид энергии, создаётся новая жизнь? Помните - “червь разложения”, “червь победитель”? 

* (Лена) Да.

- Это - что? Это - душа ваша? Нет, не ваша! Если хотите, это душа того же червя.

* (Гера) Ну, и выход какой? Непонятно.

- Выход?

* (Гера) Научиться управлять материей, допустим, да, чтобы проводить космические линии, чтоб мы там.… Тогда, если мы научимся управлять…

- Вот ваша и ошибка. Вы спрашивали, мы отвечали вам, что любой технократический путь тупиковый.

* (Гера) Ну, ладно, это я согласен. А как же вы говорите, что душа во всём теле? Значит…

- А вы умеете управлять этим телом? Умеете? Почему тело ваше болит, когда вы этого не желаете? Почему? Или вы пожелали это?

* (Гера) Получается, пожелали, только не сознательно.

- Разве? А может это просто хаос заставляет болеть вас, неумение управлять этим телом? Вы много забываете. Когда-то и это говорили вам. Вы спрашивали о болезнях, мы говорили вам.

* (Гера) Согласен.

- Вы помните? 

* (Гера) Помню.

- И о старости. Вы желаете быть старым? И, заметьте, ребенок, чем больше мечтает быть взрослым, тем быстрее он взрослеет.

* (Гера) Ну, да. Скажите, вот… ну, всё-таки, нельзя же оставаться ребёнком всю жизнь, да?  Потому, что ребёнок много чего не знает…

- Душой?

* (Гера) Да даже душой!
* (Лена, удивленно, Гере) Ничего себе!
* (Гера) Нет, ну, в смысле такие вещи как вот, допустим, ребёнок не знает, что есть обман, или там что-то ещё, а в жизни это кругом, сплошь и рядом, и чтобы просто ему выжить, ему надо знать, что он есть. Хотя бы, по крайней мере, это.

- Да, вот ваш закон, «в волчьей стае, по-волчьи выть». Подчинясь этому закону, вы так и живёте. И подчинясь этим законам, вы никогда не научитесь ничему. Придёт время, когда ваши все науки научатся и будут так сильны, что победят все болезни. Но… всё равно человечество исчезнет, без болезней есть множество других путей.

* (Гера)  Ну, всё-таки, в конце концов, мы, как вид животных, обречены, да? Рано или поздно нам на смену…

- Если - как животных, то будьте уверены – да! Ибо долго животных не держит никто. 

* (Гера) А Земля нас пока ещё носит, да?

- (Очень эмоционально) Простите,- «Земля носит». Ага! Прекрасно! А что же вы не откинете свою руку? Что же вы не выкинете?  Выкиньте хоть кусочек вашего тела! Почему вы не хотите этого сделать? Больно? А почему это должна сделать Земля? Мы только что вам говорили, что вы - планета, и вы не можете сопоставить! «Земля нас носит»!

* (Гера) Ну, допустим, у нас вот ногти отращиваются, мы обрезаем - и ничего, никакой боли, всё нормально. Может мы  - те же “ногти” на Земле?

- Хорошо, давайте будем говорить на вашем примитиве.

* (Гера) Давайте.

- А вы не видите в природе “откидывающее ногти”?

* (Гера) Листья, ветки высыхают. Можно сказать, мы будем той самой веткой, которая высохнет рано или поздно.

- А вы не станьте этой веткой. Cтаньте древом. Или это так сложно? Это не сложно, вам просто лень. Вы просто не хотите изменяться, вы привыкли быть такими и будете такими. Вы привыкли в “волчьей стае”, значит, и будете “выть”, и будете тем же “волком”. И если у вас не будет получаться, то вы будете говорить о своей доброте: «я не умею лгать». Да вы просто не умеете толком “выть”! И не можете просто найти себе место в этой “стае”, и об этом сожалеете. Но чтобы оправдать себя, лжёте себе же, говоря о том - «какой я святой»!

* (Александр)  Хорошо. Взрослея, человек, если он останется так сказать, в дитёнках ,- есть у нас такие люди…
* (Лена Александру) Душою.
* (Александр)  Нет. Душой… Ну, душой, ладно, - и развитием своим. То у таких вот убогих…

- А вы, пожалуйста, не путайте: развитие, душу…

* (Александр, уточняя, перебивает) Умственное развитие.

- А мы говорили об уме? Мы говорили о разуме, или о душе?

* (Александр) Хорошо, другой вопрос…

- Тогда, пожалуйста, если вы хотите, чтобы не тянулись наши споры, прослушайте ранее (предыдущие контакты Прим) чтобы не повторяться  множество раз. Мы говорили вам и о разуме, и говорили о душе, и говорили об огромном их различии. И сейчас скажем коротко, мы говорим вам: - оставайтесь душами. Душами - станьте детьми! И ни слова не сказали вам о разуме! Разум же дан вам не для того, чтобы уберегли детским. И что такое разум? Это тот же самый инстинкт. Инстинкт приспособиться к данным условиям. Простите, обезьяночеловеку не нужен был разум, который есть сейчас у вас. Не нужен. Он и так мог жить. А теперь представьте этого обезьяночеловека сюда, здесь, и посадите его за любую из машин. Сможет ли он это сделать? Что такое разум? Это всего лишь, всего лишь – выработал инстинкт. Инстинкт создал этот разум. Здесь - Дарвин. 

* (Александр) ...Прав. Хорошо, я вам вынужден сказать, что вы телепатией не страдаете. Я совсем другой хотел задать вопрос, и вы не смогли его уловить, прервав меня на полуслове. Хорошо. Человек, - я хотел сказать,-  взрослея - остаётся не развитым, называется «убогим, блаженным» и так далее. Почему-то церковь их восхваляет и считает, что именно для них «Царство Божие»? Расскажите на эту тему что-нибудь.

- Вы восхваляете, но не мы.

* (Гера) Это чисто земное, да, так сказать, понятие, ошибочное?

- Что такое “церковь”? Это та же самая “политика”, к сожалению. Это та же самая “борьба за власть”. И любому правителю выгодно, чтобы народ был “овцой”, и послушной “овцой”. Мы ответили на ваши два вопроса?

* (Александр) Да, вы ответили. Правда, я не согласен с вашими ответами.  Я имею право быть не согласен, как вы считаете?

- Безусловно. 

* (Александр) Слава богу. 
* (Гера) Скажите… Получается так…

- Но, к сожалению… К сожалению, мы правы. Вы говорите о Боге, и не то что видеть, но и представить то его не можете. Вы говорите о Боге, и тут же обрисуете его черными красками. Вы говорите о Боге и присуждаете ему роль палача. Разве не так? 

*(Лена) Бывает и так.

- Любовь говорит в вас? Вера говорит в вас? А может это вера, может даже привычка, чтобы был «вождь стада»? Почему вы говорите о Боге, карающем, о Боге всевидящем, о Боге всезнающем, и очень-очень множество прилагательных приводится к тому? И заметьте, все эти прилагательные подходят к «вождю стада». Вы согласны?

* (Гера) Ну, да.

- Далеко вы тогда ушли?

* (Александр) Об этом вы говорите, я говорил о другом. Всё то, чему вы нас учили, жить чувствовать сердцем…

- Мы - учили? Множество раз говорили вам и повторим – мы беседуем.

* (Александр) Хорошо. В том то и дело, во время наших бесед вы призывали, рассказывали, что надо жить сердцем, чувствовать как-то что-то. Короче, именно всё то, что вы нам наговорили, всё это есть в этих понятиях как: «убогий», «блаженный». Эти люди, которые могли очень многие вещи делать, к которым вы нас призываете. Они предсказывали события в будущем, стремились всегда помогать людям, жить для людей, отдавать себя…

- Вы это признаёте? Вы это признаёте,- а почему же тогда насмехаетесь над убогими?

* (Гера) Сами убогие.

- Почему тогда над ними издеваетесь? Почему вы сторонитесь их? Тогда пойдите, заведите с любым убогим дружбу, вы сможете это сделать? И вы всегда будете себя чувствовать выше его. Почему? Вы это признаёте, и нарушаете. Так, где же тогда ваша вера? Это значит, что просто пустые слова! Всё, что вы сказали сейчас, для вас - пустые слова и не больше. А раз они пусты, то о какой телепатии может идти речь? Если они не пережиты вами, а просто произнесены. Простите, если вы это пережили, то тогда вы сможете и жить этим. А если вы говорите – «верующий», и тут же нарушаете, если вы говорите, как прекрасно вы рассказали об «убогих», и тут же насмехаетесь над ними - что, вы верите в «убогих»? И чем вы веруете? Разумом и не больше…

* (Александр) Далеко не каждый убогий становится святым в православной церкви, например. То житие, которое мне доводилось читать про «убогих», действительно, и сердцем и разумом начинаешь, чуть ли не преклоняться перед ними.

- Преклоняться! А что же вы там  же и не прочитали – «не создай себе кумира»? Почему вы тогда, там же, и те же слова произнесшие тем же человеком, и вы сделали из него кумира? (Имеется в виду заповедь «не создай себе кумира и не поклоняйся ему» прим.) Так, где же тогда вера ваша? Где же тогда вы поняли” разумом и сердцем”?

* (Гера) Скажите, получается - как две параллельные жизни идут: одна, вроде, по разуму – то, что мы видим, то, что можно пощупать, поглядеть и так далее пятью чувствами своими, а вторая, значит, так сказать «половина креста» - это за занавесом для нас остаётся пока? То есть, она где-то, когда-то проявляется там, то есть, не проявляется, она есть, конечно, просто мы иногда замечаем, что она вроде бы есть, но большой веры в это нет, так как она спонтанна и …

 - Что удивительно, эти две жизни, эти две параллели - проходят в вас. Удивительно, что вы живёте одновременно и духовно и физически. Удивительно, что реакция души отзывается реакцией физики. А вы не зная того и не веруя в это, думаете, что вот эта реакция и есть настоящее и истинное.

* (Лена) Хм… Реакция души…

- Но, поймите…

* (Гера) Но, по-моему, это же не связано. Оно не может идти, от одного к другому…


- А мы вам и говорим, мы вам говорим, будьте внимательны. Мы же говорим вам, что душа, и реакция на действие души - и есть та реальность, в которой вы живёте. Та самая физика, тот самый нереальный, виртуальный мир, который вы считаете «реалью».

* (Александр) Скажите, наша осознаваемая память находится ближе к нашему физическому, или духовному нашему миру?

- К физическому. 

* (Гера) Скажите, пожалуйста, а-а…

-Что такое осознаваемая память? Это, то же самое “неумение управлять” своей памятью. И хотя вы запоминаете каждое мгновение и даже запоминаете то, что не видите, вы даже можете вспомнить, что у вас находится за спиной, хотя и не смотрели туда. Но вы не умеете управлять, вы не можете найти ту ячейку, вскрыть её и посмотреть, что находится в ней. Это всего лишь ”неумение управлять” и не больше.

* (Александр) А как вы относитесь к тому, что есть такое мнение, что мы сейчас разговаривая с переводчиком, разговариваем с его неосознаваемой памятью?

- А сколь у вас этой памяти? Почему же вы тогда не разговариваете?

* (Александр) Не каждому дано.

- Не каждому дано? И почему не дано? У каждого есть. А кто вам не дал? Что,  кто-то извне пришёл и не даёт вам, запрещает вам? 

* (Гера) Кстати, а кто ж всё-таки блокирует всё это дело?

- Вы кричите, что вы - бог, а оказывается, есть кто-то выше вас, и кто-то вам не даёт! А что вам не даёт? Окружающие, или может быть страх пред окружающими, и рождённый именно в вас?

* (Александр) Значит, вы не против того, что разговаривает переводчик с нами через свои слои подсознания?

- Решайте.

* (Гера) Скажите, пожалуйста, вот, всё-таки происходит, так сказать, физическое исчезновение, разложение тела, мы уходим в землю, «из праха во прах», а вот то, накопленное, так сказать, я имею ввиду чувства там, тыры-пыры, ну,… Они же нефизические? Они произвели…

-  Разве?

* (Гера) Нет, они произвели реакцию на физику, на нашу, скажем так, грубую физику.

- О нет! Мы говорили вам о чувствах, и говорили о разнице их. Вы должны были помнить, что множество чувств – это чисто физическое.

* (Гера) Ну, да, да. Я не о том…

- Да, а теперь, вы должны были бы помнить, что мы когда-то говорили вам, что человек умирает, и умирают его эмоции, но они никуда не исчезают. Произрастает трава, трава пожинается животными, вы пожинаете эту траву. И можно сказать смело, что, значит, вы и “пожали” этого человека и с его эмоциями.

* (Гера) Ну, да. На том мир и стоит, что он сам себя, как говорится,  взаимоокупает, да?

- Пока вы так будете рассуждать, он будет стоять только на самоокупаемости и не больше. Рано  или поздно, это всё развалится.

* (Гера) Ну, почему? Это стопроцентный к.п.д.

- Потому, что вы хотите создать замкнутую систему. Вы, мечтами своими, хотите изолироваться от всего, быть независимыми ни от чего. В вашем понятии, стать человеком, стать Богом – это не зависеть ни от кого и не от чего. Удивительно, не правда ли – быть в единстве, и в то же время - стать кусочком этого единства, а остальное не признавать?

* (Гера) А разве это плохо – быть независимым?

- А вы можете быть независимы? Вы хоть раз в жизни были независимы, хотя бы мгновение?

* (Гера) Ну, да, вообще-то…, Ну, да,  вообще-то, зависим от природы и чего угодно. От погоды…
* (Лена) От всех, от всего зависим.
* (Гера) От жены и любовницы, от врагов и друзей.
* (Лена) Скажите… (сдерживая смех от последней реплики Геры) Хорошо, тогда, значит, разум - это всё ерунда. Ну, тогда - можно ли осознать любовь полностью?

- Мы никогда не говорили вам о разуме и о «ерунде». Это совершенно разные вещи. Мы никогда не оскорбляли ваш разум. Мы всегда говорили вам и повторим: разум нужен вам, нужен чтобы познать, познать «физику». Но вы стараетесь разумом познать и высшее.

* (Гера) А-а, вот в чём беда!

- Беда! Беда в том, что вы хотите войти в те сферы, где нет никакой « физики». Вы хотите разумом, а значит, и словами. Согласитесь, что разум и слово, это синоним. Вы не можете рассуждать без слов, к сожалению. А это значит, что любовь вы хотите превратить в слова!

* (Лена) Ну, откуда мы знаем тогда, что мы любим, если мы не можем её осознать?

- А вы, любив страстно, - сколь ошибались, считая, что это любовь, а оказалось увлечение, и быстро переходящее? Разве нет какого? Вы всегда уверены, что вы любите? Вы всегда уверены, что это любовь? Вы всегда уверены, что это горе? Вы всегда уверены в своих чувствах, или нет?

* (Лена) Нет, конечно, не всегда.

- Не всегда. А если быть точнее - никогда! И вы простите, если бы вы были уверены в любви, вы бы никогда бы не спрашивали нас о ней.

* (Лена) Естественно, поэтому и спрашиваю.

- Значит, вы не уверенны в любви.

* (Лена) Не уверенна. А где взять эту уверенность?

- Почему? А потому, что вы хотите понять её и переложить её на слова.

* (Лена) Нет, хочу осознать.

- Осознать.

* (Лена) Потому, что это, всё-таки, важная часть нашей жизни – любовь. Если не сказать, что это – всё.

-  Осознавайте, только не станьте одним из фанатиков. Простите, проявление фанатизма есть во всём.

- (Лена) Конечно.

- Когда-то говорили вам и повторим: вкушайте небесную пищу, вкушайте, но не забывайте о земной. Вы же, чтобы легче было вкушать эту пищу, уходите от земных дел, считая, что так выгодно Богу. Удивительно, - вы знаете, что выгодно Богу, хотя не знаете самого Бога!

* (Гера) Скажите, пожалуйста… (Лена стала что-то шептать Гере, перебив его вопрос).

- И о шёпоте. Если мы слышим, как идут ваши электронные часы, почему же мы не должны слышать ваш шёпот?

* (Лена, поражённо) Даже это слышите… Они мешают?
* (Гера) Они очень мешают?
* (Александр) Прокомментируйте наш шёпот.
* (Гера Александру) Да хватит ты!

- Зачем? Зачем?

* (Лена) Ну, это незачем…

- Зачем мы ведём множество пустых разговоров?

* (Гера) Н-да.
* (Лена) И сейчас, вы считаете - пустой разговор?

- Если, чтобы мы не сказали, вы всегда будете считать, что мы уходим от ответа, что мы увиливаем, как вы говорите. Что мы пытаемся и хотим закружить вам голову…

* (Лена сквозь смех) Нет…

- ..не даём вам ничего нового. А почему? Да потому, что вы – глухая стена! И брешь найти в ней очень трудно. Вы заложили себя, вы одели на себя колпаки, те самые ночные колпаки, что раньше одевали, чтобы лучше спалось. И, к сожалению, вы забыли, и не хотите их снимать никогда. И когда ветерок начинает шевелить ваш колпак, вы всего лишь только сильнее натягиваете его, чтобы не просквозило.- ”Ай,  не дай Бог, что подумают окружающие, что подумают другие?”  Вы даже завидуете, вы завидуете убогим, потому, что вы чувствуете, что они свободнее вас. Они могут делать, что хотят. И вот эту вот зависть вы скрываете даже от себя. И вы уже говорите себе: что вы завидуете не их свободе, а завидуете тому, что они могут лечить, тому, что они святые, тому, что они могут предсказывать. Этому вы завидуете? Да не этому, вы завидуете их свободе! Вы завидуете тому, что им не окружающее, а своё внутреннее для них главное. А что делаете вы? Что делаете вы? Вы стараетесь этот колпак натянуть аж до пят!

* (Гера) Можно вас спросить? Вы как-то сказали, что вы должны к нам придти, и спросить или взять у нас силу какую-то. Вы не о силе глупости говорили, часом? Потому, что другой силы, чем мы сильны, пока я не вижу.

- О нет! Мы никогда не говорили. И никогда не хотим, и не будем оскорблять вас. Если мы говорим, что вы глупы, то говорим это лишь с надеждой…

* (Гера) Чтобы поумнели.

- Конечно! Разве вы говорите идиоту, что он идиот? Он всё равно этого не поймет. Вы согласны? 

* (Лена) Да.
* (Гера) Почему? Я лично…

- Вы накажете ребёнка только для того, чтобы научить его.

* (Гера) Да, вот у нас, почему-то, очень хорошо получается учить через наказание, через страх.

- Да, к сожалению, это так. Мы говорили вам, что множество, множество чувств замешано на страхе. К сожалению, и любовь замешана на страхе. Именно та любовь, которой вы владеете, а не то – истинное. Вы боитесь, и даже когда вы говорите, что вы не ревнивы,- всё равно есть боязнь. Боязнь потерять, боязнь, что не так поймут, боязнь обидеть, боязнь быть не услышанным. И мы говорим вам: страх и гордость, - вот. А всё остальное, чем вы владеете, это смесь страха и гордости. К сожалению, это так. 

* (Гера) Красное и чёрное.

- И когда вы говорите о любви, то мы видим в вас гордость. Гордость. И чем красивее вы говорите о любви, тем больше этой гордости. Вот что страшно, вот что страшно в вас. Вы возвышаете себя, вы поднимаете себя на фальшивые высоты, которые не существуют. Вы строите себе холмики и считаете, что это горы, на которые вы поднялись. И хотите… 

(конец стороны кассеты)

-…Вы строите себе холмики и считаете что это - ТАКОЕ!- на которое вы поднялись и хотите  поднять других. Страх, страх и гордость. И не имея,- а вы сказали о красках,- и имея эти краски, вы должны были бы помнить, что идёт время действительно понимания любви. Вы же сами говорили: “раскрываются «чакры»” -  Ваше слово.

* (Лена) Это сейчас идёт время?

- «Чакры»

* (Гера) Ну, да, ну, да, это наше слово.

- Сколь глупо слово но, к сожалению, вы пользуетесь им.

* (Александр) Так что, их не существует?

- Слово глупо!

* (Гера) Угу, а как нам…

- Это первое, и второе, они действительно не существуют.

* (Гера) А экстрасенсы видят всё.

- Что они видят?

* (Гера) Ну, какие-то воронкообразные движения…

- А мы говорили о единстве. Вы говорите о семи телах, а мы вам говорили и повторим, что у вас их множество и нельзя их разделить! Просто вам проще. Согласитесь, когда-то, имея палку, для вас это была просто палка, целая палка, а потом вы нанесли на неё метки, и теперь у вас - метр. Правильно? 

* (Все) Да.

- Правильно? Правильно. Так же сделали и вы. Вы взяли тело и расчленили его на тела, расчленили на «чакры», а в итоге, вы так этим увлеклись,- и тем более ещё и наукой, - в итоге, теперь у вас, как вы говорите, эти «чакры» работают отдельно. Удивительно, но та часть тела, как вы выразились: «мясо» работает в одну сторону, а другая - в другую. Вы расстроитесь, а потом удивляетесь, почему всё-таки так плохо, и начинаете поправлять. А если надо в какую-то сторону покрутить, то начинаете крутить в совсем другую. А потом, вы начинаете говорить: чёрт, а здесь ещё «чакры» видны, а ещё ведь есть и тысячи лепестков над головой! А вы  делите свои тела на множество каналов, нервов и так далее, и считаете, что вы познали себя. И что удивительно, вы можете по этим пятнам определить, что это за человек! Ладно – здоровье  уж - здесь ещё как-то можно действительно определить здоровье  человека по чакрам, - но вы умудряетесь – узнать его характер! - “Ах, раз у него такой-то такой-то цвет, так он, значит, такой”! Прекрасно, не правда ли, а? Во-о!- какое пророчество! - Глядя на мольберт с красками, вы уже знаете какова  картина!  Удивительно, не правда ли? А вы и есть тот мольберт с красками. То, что нарисуете, то и будет вам. Придёт к вам экстрасенс, нарисует вас таким-то, таким-то и, к сожалению, он окажется пророком. Как вы говорите: “а где та грань пророчества?”- а та грань в вас – на сколько вы в него поверили! И если вы уверовали в него, как в бога, значит, он будет идеальным художником и нарисует в вас все эти краски, и вы станете подражать всему тому, что рассказал этот экстрасенс. Хорошо, если он сказал что-нибудь хорошее, а если он у вас найдёт кучу болезней, то болезни появятся.  А если он ещё и найдет, что у вас, оказывается, есть изъян какой-то в характере, и если он у вас был хоть чуть-чуть, - ОГО!!!, так увеличите!, так приумножите!,- он у вас будет преобладающим. Всё, как у экстрасенса. Вот вам – «писатель жизни»! А вы говорите - «экстрасенс», вы говорите – «чакры»! Вы ищете душу… Где вы ищете? Вы её везде искали!

* (Лена) Можно спросить?
* (Гера) Можно, конечно.

- Спрашивайте.

* (Лена) Скажите… Конечно же, может немного глупый вопрос, но в последнее время, почему-то, когда – ну, я не буду называть по имени – “переводчик” приходит к нам домой и я, после него, очень часто начинаю писать, хотя раньше этого не делала, просто не умела.
* (Гера) Прозу?
* (Лена) Прозу, да. Я стихи писала, а прозу - нет. Ну, я не пойму вот это. Закономерность уже какая-то  получается.

- Ищите в себе.

* (Лена) В себе?

- Ищите в себе, старайтесь найти в себе.

*(Лена) Но, почему именно…

-  Поймите, поймите, никогда, никому вы никогда не будете верить! И поверите ему лишь только в том случае, если будет совпадать с вашим взглядом. А мы не хотим быть теми же самыми лжепророками, или экстрасенсами. Мы не хотим влазить в ваше истинное «Я», мы не хотим ворошить вас, и вы должны сами разобраться в себе, сами, прежде всего. И мы говорили о силе, и ждём, когда наступит время, что сможем придти и спросить эту силу. У вас есть эта сила. Есть.

* (Гера) Она у всех, наверное, есть.

- Но вы не можете её дать, не из-за того, что вы жадны, а из-за того, что вы просто не можете сами владеть ею. Вы её разбросали.  Хаос.  И поэтому не можете её собрать. И, к сожалению, понятие «силы», у вас сразу переходит в физику. Если вы подразумеваете, что вы станете «божественны», вы выбираете всего лишь из не множество вариантов, - один из них: вы станете “лучезарным”, и у вас не будет тела. Почему-то вы так возненавидели своё тело. Чем же вам навредило-то это тело? Тем, что оно болит? Так это вы виноваты! Тем, что оно не даёт вам подняться в небеса? А кто в этом виноват? А если и подниметесь, что вы будете делать там? Да, хорошо, что вы туда не поднимаетесь! Другой вариант,- и опять же – материальный, - мы когда-то говорили вам, что вы не умеете мечтать, чисто о духовном…

* (Лена) Говорили.

- …и  вы даже когда-то спорили, что « нет, умеем!” А вы попробуйте. И ещё - множество раз повторяли и повторим, не верьте нам, не верьте никому, верьте себе! И если вы читаете Библию, и если в неё уверили и не отходите, не отступаете ни от единого её слова, это говорит всего лишь о тупости барана и не больше. Это говорит лишь только о фанатизме, и не больше! И поверьте, Богу, ждущего вас, такие не нужны! Мы говорили вам, что вы должны так её прочитать, вы должны пережить её так, чтобы вы, не то что могли не переворачивая листа знать, что написано далее,- ибо вы можете заучить её просто, что и делаете,- но могли и прожить! И никогда не пробуйте копировать, не играйте в святых, это будет глупо… И когда вы играете роль Христа, этим вы унижаете его и возвышаете свою гордыню. И никогда не отрицайте другие веры, ибо вы сами-то не верите в свою, а доказываете другим. Простите, если вы будете грудью будете кричать, пылать и будете говорить, что “я спасен”, «Христос спас меня», и это будет говорить мусульманам, то мусульманин с тем же успехом будет стучать по своей груди. И кто из вас прав? А теперь представьте эту ситуацию: теперь «бедному» Христу придётся драться с Мухаммедом. Смешно? А в вашем понятии это и происходит, это происходит постоянно! Раньше у вас так и было – «битвы богов». Было? Было! Теперь же, сейчас битвы богов нет, сейчас идёт битва религий, ибо было бы смешно, что боги до сих пор дерутся. Вот ваше “взросление”!

* (Лена) Хорошо… Скажите, а вот если я вообще никогда не брала, не... ну,брала в руки эту Библию, но, честно сказать, пока что не вижу в ней ничего особенного. Вообще, это плохо?

- Вам судить! Вам решать! Вам жить! Это – первое. И второе - любая книга должна вам что-то приносить, если ничто не принесла, вина не книги, а ваша. И если эта книга очень дурна, всё равно вы должны найти в ней что-то, хотя бы просто понять, что она дурна, и сделать соответствующие выводы. Но не так как вы это делаете,- вы тут же отбрасываете на первой попавшейся странице. Как вы читаете книги? Начало и конец! Как вы читаете роман? Вы заглядываете в конец, - «ах, хорошо кончился»! Вы уже знаете конец и читать его дальше вам почему-то легче. Да? Как вы читаете книги? Если дать вам книгу, или дать прочитать рассказ, но не дать название этой книги или рассказа, - и что удивительно, вам будет сложнее это читать. Почему? Ключ? А вам нужен этот ключ? А зачем тогда нужен этот рассказ? - Тогда достаточно было бы написать название этой книги, и не читать всю книгу.

* (Александр) Хорошо, Можно поинтересоваться такими вещами?- Вы когда-то говорили, что «контакт» происходит постоянно, через сон…Вы говорили, что  приходите к каждому человеку…
* (Лена) Вдохновение, «шёпот»
* (Александр) Да, вдохновение…Всё это от вашего мира к нашему миру. В таком случае, сны которые видит «переводчик», это тоже продолжение контакта?

- А вы видите сны?

* (Александр) Практически… очень слабо вспоминаю, очень редко…

- Мы не говорим о памяти, мы спрашиваем, видите ли вы сны?

* (Александр) Вижу.

- И назовите мне хотя бы одного, кто, как говорите вы, не был «контактёром»? Вы говорите о связи с внешней средой, и тут же спрашиваете о контактах! 

* (Александр) Я хотел…

- Не ужели вам важно имя говорящего с вами?

* (Александр) Я хотел….

- Если вы не знаете имя, то вы не будете разговаривать, вам уже не интересно разговаривать.  Вы не знаете множество всего, но вам нужно имя! И хотя вы не знаете значений этих имён, всё равно вы требуете эти имена. И вы всегда хотите чего-то экзотического. Если вам, что-то приснилось и, если вам не понравился этот сон, просто не понравился,- то ”просто чепуха, сон и всё, и не больше”! Если он вам очень не понравился, вы обязательно вспомните сатану, …….. Если был прекрасный сон, вы обязательно его припишете Богу, Ангелам!

* (Александр) Вы человечество имеете в виду?  Потому, что здесь присутствующие вряд ли будут вспоминать сатану или ангелов.

- (Удивлённо) Разве?

* (Лена) Ну, почему? 

- А какая разница, что вы взяли и дали другое имя? Ну, назовите это - «высшими силами»! Вы всегда любите давать название, и довольно экзотические. А почему? Та же самая гордость! Вам что-то приснилась, ваша душа к вам же, стучится, рвётся, …а  вы? - «мне приснился сон, к чему бы это? Это наверное высшие силы…», или ещё как назовёте. Какая разница как вы это назовёте, всё равно вы обязательно возвеличите себя в этом.

* (Александр) Хорошо. Значит, вы сами сказали:"к вам рвётся ваша душа",- и так же вы говорили, что именно контакт продолжается через сон и так далее. Значит, сейчас мы  «контачим» с душой переводчика?

- Да, но когда-то мы говорили, что и вы отвечаете на те вопросы,- первое. Второе,- мы когда-то вам говорили, что Мир един. Вы забыли это?

* (Александр) Нет, мы помним постоянно.

- Помните?

* (Лена) Постоянно.

- Вы - помните? Что вы помните? Слова? И не больше, к сожалению.

*(Александр) Да. И не больше.

- Так вот, пока вы помните только слова, вы будете биться, вы будете создавать множество теорий, тут же их опровергать, тут же сочинять новые, а ничего не сдвинется,- просто добавится больше словесного мусора и не больше.

* (Александр) Поэтому, мы хотим более конкретно всё-таки. 

-  А более конкретно…

* (Александр) А то всё обобщённо: всё едино, всё едино… Для нас – это слова.

-  Конкретно? А как мы можем конкретно? Как мы можем объяснить о любви, если мы, как вы говорите «не обладающие телепатией», не можем «пробить вашу стену» и передать её другими способами? Понятие о любви, истинной любви, а не словесной!

* (Александр) А вы считаете, что обладаете телепатией?

- Нет. Нет, мы не обладаем «телепатией» и множество раз говорили вам об этом. Ибо мы не обладаем вашей физикой. И зачем нам это, если мы уже живём?  Зачем, нам нужна память, если мы не забываем, и живём во всём и видим и прошлое, и настоящее, и будущее? И, к сожалению, мы не можем дать это вам, только из-за того, что вы не видите и не слышите нас. Вы слышите только слова… Слова, и не больше.

* (Александр) Ну, сейчас поставили вроде бы все точки над «i», а теперь я хочу вернуться к началу своего вопроса. Здесь смысл, что переводчик рассказал об одном своём сне, который приснился ему буквально там три дня назад. Вы можете прокомментировать, что за сила, или что это было, что вошло в него во время принятия им пищи? (отвечающий молчит). Прим. Вам рассказать более подробно сон? (опять молчание, Александр продолжает.Прим.)  Он сидит за столом, причём в детстве себя видит, принимает пищу, и во время этого…
* (Гера подсказывает) Лет четырнадцать…
* (Александр) Лет четырнадцать ему было. После этого он почувствовал, что вошла какая-то в него сила или что-то такое непонятное, что он не мог понять…
* (Гера) Попросился и вошёл…
* (Александр) Попросился вежливо и вошёл. Всё это у него это проявилось,- первую творческую работу он написал, рассказ. Связано ли это между собой?
* (Гера ) Это вы были  вообще? Или вы - ещё раньше? Или вообще – всегда?

- Мы  как-то говорили вам, что никогда не уходили. И мы говорили вам когда-то о тысячи годах.

* (Гера) А-а…  Ну, так что ж ему снилось всё-таки? Кто-то другой, что ли пришёл?
* (Александр) Ну, это был сон такой. Помогите понять.

- Ну, хорошо, давайте снова вернёмся к параллельности миров живущих в вас.

* (Гера) Угу.

- Как вы думаете, чтобы достучаться до вашего разума, что надо сделать? Чтобы ваш разум признал что-то «иное» - что надо сделать?

* (Гера) Ну, показать надо, по крайней мере.

- Что показать? «Нефизику»? А разум не признаёт нефизизку.

* (Гера) А-а… Ну, тогда - да. Тогда наверно надо войти, просто, так сказать, через эмоции…

-  Надо, просто напросто одеть «одежду». Согласны?

* (Гера, Лена) Ну, да.

- Как обмануть ваш разум? Очень просто…, очень просто! «Напялить» себе «одежды»,- любые,- не важно, лишь бы они были физические, и понравились этому разуму и не больше, - и входи, пожалуйста,-  разум пропустит.

* (Александр) А почему  именно сейчас ему этот сон приснился? Как напоминание о прошлых? Какая связь тут? Мож времён или чего?
* (Гера) Может это специально под этот «контакт», так сказать, он рассказ этот вспомнил? Или нет? Раз вы ж не присоединились…

- Нет! Нет. 

* (Гера) Нет – да?

-Всё гораздо проще. Вы всего лишь спрашивали об неуправляемой памяти. Иногда, в вашем понятии, приходят те моменты жизни, которые вы не можете понять, почему вспомнили.

* (Гера) Угу.

- И во всём этом есть закономерность. А именно – в работе души. Но, не осознавая, не понимая и не веря в эту душу, вы принимаете за случайность, за хаос. И потому,  множество воспоминаний кажутся вам хаосом. И что удивительно, бывает, когда вы желаете что-то вспомнить, и чем это будет нужнее вам, тем меньше шансов, что вы вспомните. И лишь только тогда вспоминаете, когда забываете, что надо это вспомнить, когда устаёте вспоминать. Почему?

* (Лена) Точно.

- Почему? Потому, что - душа даёт [свободно. прим.], а вы привыкли бороться, а вы привыкли иметь определённые одежды. И если, в этой формуле, в этом ключе, позволяющем вспомнить, переставлен какой-то значок,- пусть не значительный, пусть не влияющий,- уже не будет принято. Это беда вашего разума. Это беда! И поэтому, те словесные чувства не принимаются вами. Мы говорили вам о словах, о их пустоте, и говорили о любви, когда нет слов. Вы помните?

* (Лена) Да.

- Слов нет, а понимаете прекрасно. Удивительно, -  появляются слова и начинаются распри.

* (Гера) Да-а…
(спрашивающие, о чём-то между собой шепчутся, обсуждая сказанное)
* (Гера) Можно спрашивать, да?

- Говорите.

* (Гера) Скажите, вот вы сейчас говорили, чтобы мы вспоминали, надо начинать забывать. Бывает и такое, да. А бывает, что и специально, осознанно, мы начинаем вспоминать свои, так сказать...Мы говорим именно про это. Настраиваешься, вспоминаешь, именно - задался целью, цель поставил…

- Ну, безусловно, так и должно быть. Мы же не говорим, что у вас полнейший хаос.

* (Гера) Ага, спасибо. (далее о чём-то шепчутся)
* (Александр) Информация из подсознательной памяти приходит в сознательную память?

- Ну, хорошо. Давайте скажем так,- вы можете назвать, ну, хотя бы, номер билета, автобусного билета на котором вы проехали неделю назад?

* (Лена) Нет, конечно!

- Нет, конечно! А вы читали. [смотрели номер]

* (Лена) Читали…

- А что вам мешает вспомнить?

* ( Гера) Ну, наверно…
* (Лена) Мы не придаём значения.

- Не придаёте значения?

* (Лена) Да.

- А-а, интересно! Когда вы высчитываете этот номер, вы придали значение?

* (Лена) Ну, на тот данный момент это было нужно, а сейчас, я считаю, нет.
* (Александр Лене) Это называется “кратковременная память” у нас.
* (Лена) Ну, почему, кратковременная? Я просто считаю, что это…
* (Александр Лене) Это называется “кратковременная память” называется.

- Вы помните всё. Вы помните всё! И под каким-нибудь стрессом,- пусть это будет гипноз, пусть это, будет ещё какое-нибудь стресс, не важно,- вы можете вспомнить. А почему? А потому, что у вас не будет желания вспомнить это разумом! Потому, что в этот момент вы будете просто «машиной». Когда вы загипнотизированы, вас спрашивают: - назовите номер…, и вы назовёте его, не задумываясь, потому, что вы просто «машина» и исполняете приказ и не больше! Машина, которая обладает памятью и не думает, а нужно это выдавать, или нет! А вы, держа этот билет в руках, посмотрели на него и дали установку, что это не нужно запоминать, это лишний сор и не больше. И чем больше наука будет утверждать, что память ваша ограничена и поэтому не засоряйте её «пустяками» - тем больше вы будете забывать. 

* (Лена) Ну, я считаю, что это пустяк, действительно, не обязательно запоминать. Лучше запоминать какие-нибудь…

- Но вы-то запомнили! Но вы запомнили! Вы считаете, что это пустяк. Вы не допустили эту информацию в «рабочую память». Память, которой вы пользуетесь. И вы…

* (Гера) Оперативная память.

- О,  нет! «Оперативная память», это та память, которая работает именно сейчас, которая позволяет вам привести именно эти фразы, именно эти слова. «Оперативная память», это всего лишь та память, которая позволяет «шевелить вашим языком». И когда вы, как вы считаете, думаете,- а, к сожалению, множество людей считают, что они думают, когда шевелят языком “про себя”,-  вот это и есть «оперативная память». Есть ещё и рабочая память. Она довольно-таки обширна. Вы же можете вспомнить телефон, вы можете вспомнить книги, вы можете вспомнить очень многое, вы можете вспомнить детство. Это есть «рабочая память».Есть «хранилище», долговременное, как вы говорите, и она  делится на три фазы.

* (Александр) Назовите.
* (Гера Александру ) Более, более и более.

- Название? Мы не знаем этих названий.

* (Александр) И переводчик не знает?

- Может быть, не знает и переводчик.

* (Гера) По степени надобности, наверно, делится. Да? 

- Давайте скажем так. Давайте назовём – «первичная память». Только, пожалуйста, не вмешивайте сюда память о «реинкорнациях».

* (Лена) Ну, да.

- Память не существует.

* (Лена) А как же мы можем вспомнить прошлые обстоятельства?

- А мы говорили вам о «химии».

* (Лена) Химическая память, да?

- А вы обладаете только химической памятью. И лишь только тогда… Воспоминания «реинкорнаций» будут лишь только тогда, когда вы превратите её в «химию». Это когда к вам придут и наговорят вам,- не важно, сочинят, или скажут правду,- лишь только тогда. Есть «первичная память», и она зарождается вместе с вами. И она зарождается ещё в матери. И она помнит всё. Она помнит деление клеток, но вы этого не можете вспомнить. Не можете! Эта память содержит абсолютно всё: каждый запах, каждое движение. Мы  говорили, что вы можете вспомнить, что у вас находится за спиной, хотя вы не оглядывались. Почему? Потому, что у вас, помимо зрения, существует ещё множество чувств.

* (Гера) Восприятия, да? Кожное зрение…

- Множество.

* (Гера) Даже так?

- Вы говорите о шести, а мы вам скажем – более. И это всё - запоминается. И вот это, чаще всего, вы называете подсознанием… И вы правы, это и есть то самое подсознание. Та самая химия, то самое подсознание, но никак не душа. Итак, вот вам одно название – «Подсознание». И в это подсознание вы и получаете все ваши наклонности, характер… Ибо, как вы говорите, эта память создает «второй фронт», второй  фронт памяти – «рабочий архив». Это сознание…

(Срыв)
(Идёт счёт)

* (Александр)… Хорошо, я вот такой человек, который, вашими словами, «стеной отгородился», «надел колпак до пят» так вот, именно я  уверен,- почему-то, скажем так, - что все контакты происходят именно с подсознанием переводчика. И если бы вы это тут как-то всё-таки подтвердили, то было бы легче разобраться. Может не стоит тут как-то играть в таинственность какую-то такую, которая преподносится порой на контактах?

- А мы разве когда-нибудь таинственно произносили? Вспомните, мы вначале говорили, что переводчик отвечает на ваши вопросы, и вы отвечаете на эти же вопросы, вы сами отвечаете на них. Вы помните? Вы должны были вспомнить тогда, что мы - есть часть ваша. Вы помните?

* (Лена) Да.

- Вы должны были помнить, что мы пришли в мир эмоциональный, чтобы было бы легче говорить его языком. И если хотите назвать нас подсознанием, пожалуйста, называйте. Мы когда-то хотели вам дать понять, что мы - ваше будущее, и в тоже время, мы предупредили вас, чтобы не шли нашей дорогой.

* (Гера) Скажите… Простите…, так что, если мы пойдём другой [дорогой], тогда вы не будете уже нашим будущем, да? То есть, если мы в другую сторону уходим…

- Здесь всё гораздо сложнее.

* (Гера) Угу, нам не понять. Ну, ладно.

- Поймите, в этом мире, в котором вы живёте, нельзя всё объяснить. Нельзя. И в этом мире нельзя понять, что нет понятия «времени». В этом мире есть и прошедшее, и настоящее, и будущее. В этом мире - есть, да. В этом мире есть слова, и поэтому произносим их и мы. В этом мире «есть» всё, в чём вы живёте, и «нет» того, в чём вы живёте, но не видите этого. И, если вы говорите о “бесконечности”, это для вас ещё пустой звук и не больше. Если мы говорим вам, что мы - ваше будущее, но вы не идёте нашей дорогой, нам очень трудно будет объяснить вам, что, всё же, это едино. Нам очень трудно объяснить, что мы, это есть вы,- а вы это говорили, вы помните?

* (Лена) Да.

- И очень трудно вам понять, что вы, это есть и мы. Вам трудно понять, что такое единство, вам трудно понять, что стол и вы - не намного отличаетесь друг от друга. И хотя разумом вы скажете: «да, конечно, материя»… ,(то и другое. прим.)  для вас это всё равно будет пустой звук. Вам очень трудно понять, что заставляет двигать атомом, и в чём же всё-таки разница этих движений. Те же самые движения  атомов и атомов ваших молекул, и атомов молекул стола, -  те же движения, - а сколь большая разница! Вам это очень трудно понять. Вам очень трудно понять, что нет ничего мёртвого, и что смерть - тоже живуча, и что это тоже жизнь. И что жизнь, это тоже - смерть. Это всё - единство, нельзя вам этого объяснить. И когда вы говорите с усмешкой, и соглашаетесь с нами о «колпаке», это говорит всего лишь о вашем упрямстве. Ну что же, мы рады, что вы упрямы, уж рано или поздно, с тем же упрямством, вы будете потом доказывать другим то, что мы хотим объяснить вам сейчас. Мы хотим вам сказать, мы хотим вам напомнить: « От ненависти к любви - один шаг», и от любви к ненависти - тоже один шаг. И чем больше ненавидишь, тем крепче будешь потом любить - и наоборот. И чем больше сил у вас для зла, значит, тем больше сил будет и для добра, если придёте на дорогу его. Но если у вас нет сил ни для чего – ничего и не будет. Мы когда-то вам говорили, о дьяволе и о Боге, и дали вам понять, что и тот и тот - сильны. Вы помните?

* (Лена) Да.
* (Гера) Потому, что это - одно и то же.

- Поймите, просто один выбрал дорогу «зла», другой выбрал дорогу «добра» Давайте будем говорить пока вашими понятиями. И нельзя говорить, кто из них сильней и “никогда никто никого не победит”.

* (Лена) Потому, что это едино. Так?

- Мы пока в вашем мире, тогда давайте пока не будем говорить о единстве. Не зря же говорили, и призывают вас идти к добру. И вы будете идти к добру, вы должны идти к добру! Часть пойдёт ко злу, часть пойдёт к добру. И тот же дьявол, не намного ошибается, говоря: - что если бы не было б зла, то и не было  добра, если не было бы тьмы, то не было бы и света. (далее помехи, часть текста отсутствует) прим.
- … больше будет нравиться, как вы умирали, больше будет только это, пожалуйста, - у вас будет «ад». Вы будете больше видеть, как вы умирали. Если вам больше нравиться, как вы рождались, - вот вам «рай». А вы говорите о “зоопарке”, где вас кормят, и не больше. А вам нравится! Множеству вам - нравится это! Множество вас, из-за лени стремятся к вере, ибо там "не надо трудиться"… (помехи, обрыв).

* (Гера) Вы слышите?...Я боюсь, что нет…

(выходит на «контакт» Мабу)

- … болят… Это когда… один – два, один – два. И всё, я дальше не знаю. И опять, один - два, один – два, и опять… (неразборчиво) Потом старейшина пришёл и говорит – Снова все камни разбросал. Не дам ему посчитать.

* (Александр) Подожди, это что, страницы будущей книги ты всё пытаешься нам рассказать как-то? (помехи)
* (Лена радостно) Мабу, это ты?!
* (Александр) Он, он.
* (Лена) Мабу?!

- Ну, я.

* (Лена) Ну?!

- Что - “ну”?

* (Лена) А я - кто? А? Кто я, помнишь?

- Помню. Только ты не называлась.

* (Лена) Называлась, неправда!

- Не называлась.

* (Лена) Ну, да!
* (Гера) Петина жена она.

- Ну, а как зовут-то?  Я не знаю.

* (Лена) “Здрасте!”… Зови: - “О”, зови –“ А” (напоминает о ранее оговоренных с Мабу условных именах при общении)

- Чего?

* (Лена) Ага, ясно… (о чём-то говорят с Герой прим.)
* (Лена) Значит, у тебя две жены, да?

- Да!

* (Лена) Две жены, ясно.
* (Александр тяжело вздыхая из-за выхода Мабу на «контакт») Ну, что будем делать? Расскажите чего-нибудь. 
* (Лена) Ну, расскажи чего-нибудь. Чего ты там делаешь?
* (Гера) Что за новости у тебя там?
* (Лена) Да, как ты там живёшь?
* (Александр) Дядя Гена, переводчик, книжку начал писать про тебя. Расскажи, что бы было что написать.

- Чего “Гена”?

* (Александр) Да…, пишет книжечку…
* (Гера) Руны про тебя пишет.

- Кто?

* (Гера) Вот, кто говорит.
* (Лена) Чтобы про тебя узнали здесь. Давай, что-нибудь расскажи нам, такое интересное. 
* (Лена) Мабу, а можешь рассказать, как жён выбирал, ещё раз? А то мы, дураки, тогда не записали.

- Чего “ещё раз”?...  (помехи, обрыв) 

* (Лена) …. А ты сейчас во сне нас видишь?

- Нет.

* (Гера) Монах с тобой есть?

- Да.

* (Лена) Интересно, а с ним поговорить можно?

- Вы же уже пробовали.

* (Лена) Ах, мы пробовали, а я уже не помню…
* (Гера) А как внутри выглядит их пещера? Как она…, просто камни есть там? Ничего там не светится на потолке?  (помехи, срыв)
* (Гера) … Может чего сделать? Подержать свечку, или ещё чего?

- Не знаю.

* (Гера) О, “не знаю”… Ну, как они говорят?
* (Лена) Или они ничего не говорят?

- Ничего не говорят.

* (Лена) А-а, ясно. Это по твоим указаниям делает?

- Чего?

* (Лена) Ну, ты просишь её [жену] свечку подержать, вот она и держит.

- Не держит, а поставила!
* (Лена) А-а, поставила. Ну-ну.
* (Гера) Поставить - ты ей сказал, да?

- Да.

* (Гера) А монахи - тебе сказали?

- Нет, я сам догадался.

* (Гера) Ну, они сказали, что сейчас тебя «уложат»?

- Нет!

* (Гера) Нет?
* (Лена) Тоже сам догадался?  (помехи, срыв) прим.
* (Гера) … Скажи [монахам]: - Вам - привет! 
* (Александр) Пожелай им добра от нас, здоровья.
* (Гера) Скажи, чтобы они не болели, не чихали и, это…, не кашляли, да. (помехи, срыв)

Конец сеанса, переводчик вышел из «контактного» состояния, но спрашивающие этого ещё не поняли.
Ведётся разговор с переводчиком.

* (Гера) “Очки”!- ты? (Переводчик носил очки)
* (Лена, не понимая, Гере) … А что это такое?

- Очки? Что за выражение , ёлки-палки?

* (Лена смеясь) Не…, да… мы уж не знаем, как назвать…

- Дали счёт, так чего уж… 

* (Лена) А ты у нас, до этого, ”Мабу” был.
* (Гера) Мабу с нами связывался.

- Раз вы счёт [обратный, на выход] дали, значит, я должен  [выйти], я и разговариваю.

* (Александр) Я так и понял.
* (Лена взявшись за голову) А я, как дура…! (смеётся)
* (Гера) Хочешь сказать, что ты всех тут “наколол” что ли? 
* (Александр) Я, вот, в самом начале спросил тебя: ты что там за книгу пытался писать? Давай, расскажи чего-нибудь о книге.
* (Лена Александру) Слушай, мы не про это начали…

- Не, ну, я этого контакта не помню…, эт самое…, Сам контакт я потом, может, вспомню.

* (Гера) Слушай, ты под Мабу что ли сейчас подделывался сознательно?
* (Лена) Ну, конечно, подделался. Чего уж там!

- Ничего я не помню пока ещё… 

(конец кассеты.)
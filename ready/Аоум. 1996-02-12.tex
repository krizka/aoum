Аоум. глава 12. 02. 1996 г
Георгий Губин
VG - 12.02.1996.


- 12 февраля 1996 года.
- … изготовленное и приготовленное – и вы не знаете смысла. Вы берёте готовые кубички и собираете. Никто из вас ни что не сделал с нуля. Все взято от природы, и плюс ваша фантазия и руки, и не больше.

* Ну, если можно, тогда будем говорить о том, что человечество приобрело на Луне, или с другой планеты. Вы говорили, что это осколок другой планеты какой-то. Вот, почему человек, даже Мабу, который вроде бы ещё ребёнок, он ещё только начинал развиваться, и то - у него уже присутствует обман…

- Вы же говорите о генах?

* Да.

- Простите, это был осколок планеты, и эта планета разрушила себя. Разрушена была пра-людьми. Так  почему она не должна была остаться в памяти?

* Это такая глубокая память у человека?

- Она гораздо глубже! Гораздо глубже, чем вы можете представить. Давайте начнём так: каждый из вас помнит начало мира, каждый из вас присутствовал в этом начале, и каждый из вас строил это начало, хотя вас ещё и не было - в вашем понятии. 

* Да.

- А теперь представьте, что - ваша память? Мы часто говорим: вспомните будущее.

* Да, оно, наверное, уже существует.

- Только в вашем понятии есть время! Только в вашем! Там, где есть материя, там всегда будет время. Всегда! Мы же неоднократно говорили вам о нематериальности вашей. И все ваши, так называемые «чакры», это материя. И все вами выдуманные миры, это материя. Пусть более тонкая, но, всё-же, материя. Здесь можно было бы вспомнить о законах физики. Вспомните! Хотя вы скажите, что не знаете. Что в тонких плёнках уже имеет значение не только волновая, но и корпускулярная теория.

* Ну, да.

- Вы об этом не вспоминаете! Частица света, которой являетесь и вы, уже обладает и материальными свойствами и волной. Физик… Сбой.

* 1..,2.. 

- 1906 год. В одном из экспериментов в Англии был обнаружен очень интересный эффект. Шла работа с тонкими плёнками. Но эффект нельзя было повторить, и потому был забыт. Ибо вы привыкли называть вещами только то, что можно повторить и множество раз проверить. 1956 год. Было повторено то же самое, но была ссылка на невнимательность лаборанта и опять вы ушли. 1968 год. Было повторено в нескольких лабораториях. Тест прошёл, как случайность. Попытались повторить эту случайность, и что? Вы открыли один из законов, но совсем не тот. 

*  Один из каких законов?  Не тот, который подразумевался в начале, так?

- Вы не могли понять суть, что произошло. И хотя это привело к множествам последствий, в вашем понятии хорошим последствиям, вы уже стали обладать многим возможностями тонких плёнок, но забыли, не смели догадаться, что это относится и к вам. Ведь вы называете свои тела тонкими, так почему же тогда  не примените её здесь физику? И хотя мы против этих применений, вы к этому придёте, к сожалению и нет. Нельзя сказать, что вы плохие, или хорошие. Любое ваше открытие приводит вас и уводит. Любая ваша ошибка не только губит, но и движет вами. Потому мы говорили и хотим сказать снова: ищите, ошибайтесь, и не бойтесь ошибок, бойтесь не заметить её.

* Спасибо. Я бы хотела спросить. У нас произошли некоторые заминки: некоторые участники нашей группы, почему то испугались, видят сны плохие по поводу контактов, на счёт наших внутренних органов какие- то подозрения есть. Ваше мнение на этот счёт, если конечно можно? Как вы относитесь к этому? Мы наверно вам надоели?

- Вы находитесь в мире иллюзий. И скажем вам: надо будет - не услышите более! Рано или поздно, будет так, но сроки зависят от вас.

* То есть…

 - Давайте дальше. Чтобы вы не делали, всё к лучшему, ваша пословица.

* Да.

- Но только и воспринимайте к лучшему! Только в этом случае! Дальше. Вы осторожны, до той степени, что переходите в глупость.

* Ну, да. 

- Почему? Потому, что занимаетесь обманом! Самообманом! И давайте вспомним историю. Историю прошлого и будущего. К сожалению, пока ещё будущее – повторение прошлого. Что в вашем понятии “реинкорнация”? Что в вашем понятии”жизнь”? Что в вашем понятии ”следствие и причина”? Это цепь. Это цепь, состоящая из следствий и причин. И каждый порождает новое звено.  Хорошо, если это звено составляет прямую цепь. Здесь можно сказать, что вы идёте прямой дорогой. Чаще - это спираль, это круто закрученная спираль, и чем больше её шаг, тем прямее. Вы согласны?

* Да.

- Но, витки бывают столь плотно сжаты, что переходя на новый виток, вы можете вернуться на старый. А значит, вы не заметите, как начнёте повторять прошлую жизнь, один к одному. Хорошо, если вы придёте на более высший виток, но чаще, чаще бывает наоборот, вы уходите ниже. И за одну жизнь вы можете прожить множество жизней. Когда-то вы спрашивали: может ли человек вернуться в камень, в растение?

* Да.

- Может. Сколько трудов приходится сделать, но вы умудряетесь сделать это. Когда количество витков и шаг их столь мал, что вы не замечаете, как вы перескальзываете на более ниже и ниже. Вы их называете ступенями, пусть будет так. Теперь представьте прямую. Касательную прямую этих витков. И точка соприкосновения – это событие. И так как это была прямая, то значит, говорит об одних и тех же событиях. Только на более низком, или на более высоком уровне. На более низких, или более высоких витках. Вы понимаете?

* Ну, так.

 - Теперь представьте, что вы должны сделать, чтобы не повторить, чтобы не вернуться на нижнее звено? Вы должны не повторить, а изменить. Из жизни в жизнь к вам приходят одни и те же события. Из жизни в жизнь вы ищете на них ответы, и если они будут одинаковые, это не будет никакого движения. Если в прошлой жизни вы были врагами и в этой жизни останетесь ими – продвинулись ли вы? Нет! Вы остались на том же витке, или не дай Бог ещё ниже. Вы же жалуетесь на память: “не помню, потому и повторяю”. Не помните? Простите, и совесть ваша не помнит ничего? Как она кричит вам?  Вы не слышите! Когда вам что-то кричит, вы говорите пы…(Сбой)
1,2..

* Такой, может, глупый несколько вопрос. Когда мы просили Мабу посчитать нас, он всегда считал на одного больше, а потом переводчик говорил, что он видел, и вместе с нами считал и собаку мою. Вот как это понять?

- Мабу ещё не умеет различать и животных и людей. Для него - это живое. Это самое главное.

* Ну, да. Мы, в общем, так и поняли, но, с другой стороны ведь он видит у монахов животных, и он говорит, что это животное. Он же различает и животного и себя.

- Он видит их, но не знает о них ничего, и он считает, что это слуги господни, слуги монахов. О монахах нельзя говорить, как о живых, иначе это было бы для Мабу -  что он, что монахи. И хотя, он когда-то станет одним из них… Сейчас  монахи для него – боги. И когда станет он одевать одежды монахов, он не станет богом, он останется тем же наивным мальчишкой. Но, ему будет трудно, и он поймёт трудность, что от него зависит многое. Это его пугает, потому он и вышел на вас. Вы скажете: как же так, в первый раз он вышел, когда у него было всего только две жены и не могло быть речи о монашестве.

* Да.

- Ребёнок видит и будущее, и прошлое, и настоящее. Он уже знает понятие времени, хотя тем, что это связано с кормлением, с режимом укладывания спать, с физиологическими проблемами. У него уже есть понятие  времени но, ещё сознанием не понятое, а значит, хоть и туманно, хоть и плохо он может увидеть будущее. И тревога, эта тревога передастся в него, и потому, он будет искать защиту. Он приходит к вам. Вы говорите интерес? Нет! Он ищет себе подобных. Только и всего.

* Ищет себе подобных? 
* Скажите, можно такой вопрос? Почему вы сказали - нельзя,-  раньше я может не слышала,- меньше трёх и не больше семи? Здесь что-то с энергией связано, с зарядом каким-то?

- Да, здесь есть множество причин. Вас должно быть не менее трёх. Даже у вас сказано, “где вас трое”…  Вы помните?( Прим. Имеется в виду слова Христа: - где двое или трое соберутся во имя моё, там и Я среди вас.)

* Да, у нас много таких поговорок.

- И хотя мы не претендуем на роль Бога и  храма, но всё-таки соблюдайте. Почему не более семи? Потому что более - вы уже потеряете внимательность, а значит, будет не просто контакт… (сбой).
1,2..3..,

- Спрашивайте.

* Так, о чем же я хотела ещё… Скажите, вот нам постоянно говорят, намекают, и санитары и те которых вы назвали нашими чакрами, что мы далее бесед не идём. Может быть мы можем эксперимент какой-нибудь  провести? Можете вы что-нибудь предложить, или мы только должны предлагать?

- Уже идёт, уже идёт большой эксперимент, и идёт он, к сожалению уже века!

* Ну, да. Вы знаете, иногда, когда мы планируем проводить контакт, когда подготавливаем вопросы, когда идёт настройка на разговор, я, как бы уже знаю некоторые ответы на то, что мы зададим.

- Мы, кажется, говорили вам и повторим снова – ничего нового мы не дали вам. Ничего! Мы говорим только то, что знаете вы, но забыли об этом.

* Скажите, можно ли заниматься экспериментами на себе, с эфирным телом допустим и нужно ли это делать? Ну, к примеру, я кундалини поднимаю.

- Давайте скажем так, если этого хочет ваш мозг, и только лишь, то не советуем. Вы должны бы помнить, что сила кундалини, это не только сила светлого, но и чёрного.

* Это да.

- Многие сексуальные маньяки, именно начинали с этого.

* Да, теперь понимаю.

- Давайте пойдём дальше. Если хочет ваш мозг, ваше сознание, многого вы не достигните, больше будет обмана. И этот обман ударит вас же. Было бы лучше, если бы вы не делали всё это искусственно. А искусственное открытие третьего глаза, это что?  Это уродство, и не больше.

* Н-да.

- А что делаете вы сейчас? Что делаете вы все? Вы создаёте множество реклам, вы создаёте множество сект по открытию экстрасенсорного чувства, и что это? Это всего лишь попытка уничтожить себя! И хотя вы скажете: нет, мы хотим подняться духовно. О какой духовности здесь говорится? Какой духовность, если не…(непонятно). Вспомните, как пришёл и сказал Иисус вам: здесь нет Бога! И это он говорил в храме! И вы повторяете снова.

* Ну, так…вообще-то это даже видно, что у нас сейчас некоторая вакханалия что ли, иногда становится и страшно, куда нас это заведёт. Но, может, в этом есть хоть зерно какое-нибудь?

- Конечно же, конечно! И среди плевел можно найти зёрна, но было бы лучше, если бы вы старались и ухаживали за урожаем, а не забросили его.

* Вы знаете, я одна здесь, поэтому я, может, про себя буду говорить. У меня некоторое время, может год, назад возникло такое ощущение, что слишком увлеклась всеми этими книгами, всякими такими экспериментами и т.д., и у меня пришло такое ощущение, что надо искать в самой жизни суть.

- А мы говорили вам: живите и более ничего не нужно вам.

* Ну, вот…

- Но живите! Именно живите, а не существуйте. Что такое жизнь? Это не когда плоть живёт! Когда душа живёт, в мире, а не спряталась за шкурами и одеждами. Когда-то говорили вам, что очень легко поменять одежды очень легко! Один из вас сразу же пожелал их поменять. Для чего? Чтобы просто убедиться в нашей правдивости. А что делаете многие сейчас вы? Вы меняете даже пол, а для чего? Вы ссылаетесь, что в прошлой жизни я была женщиной, а сейчас я мужчина, я хочу остаться женщиной, у меня осталась память. Да глупости это всё! Просто вы заблудились в собственной жизни и хотите измениться, думая, что изменится всё вокруг. Но, нельзя же меняться физически! Физически вы меняетесь всегда, что толку от того!?  Нельзя меняться духовно, ибо духовно вы не можете измениться. Всё, что есть в вас, уже есть! Но вы должны найти в себе это «есть», открыть в себе! Если вы будете делать это только сознанием, получите только ложь!  Многие из вас только и занимаются тем, что лгут! А теперь представьте, что отъявленный лгун начинает всех лечить. 

* Представляю.

- Что происходит? Что происходит? Кто-то придёт и скажет: вы вылечили меня. Он обрадовался, ложь подкрепилась. Но наступит время, когда он поймёт, что это было ложь, и, как правило, это время приходит именно тогда когда уже поздно! И тогда, умирая, вы попадаете в мир лжи. В мире лжи вы не сможете найти истинную дорогу и снова вы будете притянуты к земле, но ещё в более худшем состоянии. Давайте поговорим… 
1,2,3…

- На какой букве мы остановились?

* На букве «Щ»

- Хорошо, тогда напомните, что  буква «Ш»?

* Это три единицы…

- Смысл?

* А, смысл? Сейчас уже трудно вспомнить… Опять жалуемся на память.

- Хорошо, давайте по-другому, похоже ли это на трезубец?

* Да, Нептуна, да?

- Прекрасно. Теперь давайте вспомним три силы.

* Три силы.

- Составляющие вашу жизнь, вспомните.

* Ну, да.

- Так назовите их.

* Любовь, разум и…

- О, нет.

* А чего?

- Давайте вернемся к более древнему, и пока не будем трогать духовность, вспомним, что говорили философы: огонь, вода… и что ещё?

* Да, и земля. Но вообще – четыре. Ещё воздух. Обычно, всего четыре.

- Разве? Изначально было всего лишь три. Вам этого хватало.

* Но, теперь уже четыре.

- Представьте, вы добавляете хвостик, что это значит? Это попытка удержать этот трезубец, завладеть им и держать его. Ножечка столь маленькая, столь кривая, потому что она сделана вами, и вы сделали её, довольно-то, не умело, потому трезубец чаще переворачивается и бьёт вас же. Перевернём, и что получится?

* «Т»

- Ваша «т» прописная. А помните, мы говорили о прописном, что это значит?

* Да…как-то…

- Да, вы уже не помните. Хорошо, давайте следующую букву.

* Так, «Ш», «Щ», что там дальше я уже потеряла. «Ъ», «Ь».

- Хорошо, давайте скажем так. Почему буквы расставлены именно так, а не по-другому?

* Ну, в этом, наверное, есть особый порядок свой.

- Как вы думаете, в чём здесь порядок… (сбой)
1,2,3,4…9

- Интересно, не правда ли? Давайте возьмём  сравним руки и ноги. Большая ли у них разница?

* Не очень. А вы..?  Простите..

- Не очень, даже пальцы и то, почти одинаковые. Разница только в том, что ногами вы меньше можете работать, чем руками. А теперь представьте насекомое, много ли у них разниц?

* В ногах и руках? У них нет же ног и рук, только ноги.

- Ну, пускай будет так: между передними и задними конечностями.

* Нет, почти никакой.

- Почти никакой! Почему мы тогда называем ноги и руки? Почему не наоборот?

* Мы же люди.

- Представьте, теперь представьте, что вы будете работать ногами так же, как руками, постоянно будете тренироваться, тренироваться, тренироваться. Из года в год. И что? У вас будет четыре руки, только и всего. А теперь представьте наоборот. Вам нужно только нажимать и нажимать, кнопочки, кнопочки, кнопочки. Вам уже не нужно будет работать руками, как вы делаете сейчас, и что? У вас будет четыре ноги. И что? Были насекомыми и опять  станете насекомыми. Интересно?

* Интересно.
1.2,3…9

- Спрашивайте.

* Скажите, это кто сейчас был? О насекомых говорили.

- Спрашивайте, спрашивайте другое. Разбирайтесь сами.

* Хорошо. Мы про зомби как-то разговаривали. Вот,  чем зомби отличается от человека внешне?

- Внешне - ничем.

* Ничем, да. 

- А внутри пустота. Именно тот самый сосуд, который напичкан чем угодно, но только не душой. Зомби – это чисто материя, только чистая материя,  и вы здесь не сможете найти то божественное. Откуда берутся теперь зомби? Очень просто. Это когда человек не хочет расставаться с землёй. Он до того к ней привык, он до того в неё влюбился, сами понимаете в каком смысле, что он не хочет с ней просто расстаться, ему не за чем этого делать. Он не хочет никакие миры, ему подавай то, что уже есть. Ну, представьте, этот человек умирает. И что? 

* Да ничего…

- Оболочка, созданная им, его фантазиями, его желаниями, чисто оболочка приходит на землю и всё! Душа ушла, как говорится, в чистилище или ещё куда-нибудь, и пришла только оболочка. Это один вариант рождения зомби. Давайте рассмотрим другой. Жил человек, пришёл кто-то и сделал его зомби. Ну, или скажем так, что так не бывает. Так не бывает. Это уже не зомби. В этом человеке остается душа, но она закупорена, и чужая власть руководствуется этим телом. Так что, пожалуйста, не путайте. При жизни стать зомби,  представляете, что это должно быть?! Это должно быть только в двух случаях. Когда человек насильно, сам по своей воле, прощается с жизнью – самоубийца. Но, ему не хватает сил убить себя, в прямом смысле слова, и тогда происходит всё внутренне, внутренне погибает человек, уходит душа. Вот вам самоубийство, и тогда - это зомби. Вы поняли смысл?

* Да.

- Давайте по-другому…(сбой)
1,2,3…9

- … от трёх до семи.

* Ах, да.

- И вы снова.

* Ой, простите, тогда может сейчас прекратить всё это?

- Давайте представим, что мы пришли и тут же уйдём. Будет ли это скромно с вашей и с нашей стороны?

* Да, как-то, вроде, как продолжение, поэтому мы, может быть, действительно не поняли. Не то, что не поняли, наверное, упрямство что ли наше… Я даже не знаю, как это объяснить. Вам трудно, наверное, сейчас, да?

- Спрашивайте.

* Скажите, может ли человек после смерти, физической смерти в нашем мире, попасть в мир санитаров или предохранителей, как они себя называют?

- Нет.

* Никогда не смогут?

-  Вы же подумайте, это частица вас. Она находится в вас. Как вы можете всё целое превратить в маленькую частицу, а куда уйдёт всё остальное? Что такое “смерть” в вашем понятии? Каковы дороги ваши после смерти? Ну, давайте, скажем так: мы говорили, что вы не можете определить сам переход от живого к мёртвому, именно это мгновение вы не можете найти. И этого мгновения действительно нету, ничто не исчезает бесследно и сразу, всё происходит постепенно. Хотя вами принято, что более пяти минут человека нельзя оживить, это не правда. За трое суток вы можете оживить, и мозгу это не повредит. Первое, что повреждается, не мозг, а потеря тех ваших колебаний. Тех ваших колебаний, которые создают и поддерживают  энергию жизни. Итак, запомните: вы можете до трёх суток ещё оживить человека, а это значит, что человек умирает медленно именно эти трое суток. И хотя кто-то из вас, когда то утверждал, что видел и, даже сумел замерить параметры души, уходящей из тела, это всё ложь. Вы можете замерить реакцию, реакцию материи на уход, но предупредим вас, ничто сразу не уходит. Теперь давайте представим: мгновенная смерть. В каких либо катастрофах, человек умирает мгновенно в вашем понятии. Что первым делом происходит? Первым делом умерший видит себя со стороны, и ещё не может понять, что он уже умер. Давайте упустим возможность этих трёх суток, представим, что их пока нету, а просто посмотрим реакцию умершего. Он попал в катастрофу, он наблюдает её со стороны, и он не может ещё понять, что произошло с ним. В это мгновение, в это время он может, и будет присутствовать рядом, он же ещё не ушёл, он ещё не признал себя мёртвым - он будет слышать и ощущать вас. Он не сможет воздействовать на вас никак, ибо тело его лежит, но он будет видеть и помнить всё, и если в этот момент  всё-таки он вернётся к жизни, он сможет вам рассказать, что было, и это не удивительно. Давайте дальше. Смерть от болезни, когда вы долго и мучительно умираете. Здесь уже совершенно другая техника. Человек знает, что он умрёт, он уже настраивается. Здесь уже срабатывает память, память прожитого. Вы это говорите…(Сбой)
1,2,3…9

- …Вот только одно но, что вы можете превратиться в них снова. Почему? ... (Сбой)

(Мабу)
- Кто?

* Ну, вот, наши такие…

- Монахи?

* Ну, да, монахи. Как у тебя дела?

- Хорошо.

* Чем занимаемся?

- Ничего не делаю, отдыхаю.

* Сейчас?

- Да.

* А вообще?

- Что вообще?

* Ну, вообще, вчера, чем занимался?

- Вчера, это когда?

* Ну,  когда ты ночевал, переспал, а ещё до этого, когда ты не спал. Понял?

- Не-а.

* Вот, смотри, сейчас солнышко, сейчас день, да?  А вот когда ты спал, была ночь. Так?

- Не знаю, я же спал, откуда я знаю, что там было?

* Ну, вот, когда ты спишь, это ночь, а когда ты не спишь, это день называется.

- Ну, это не правильно.

* Почему?

- А когда я сплю, а чё они не спят? У них день, а у меня ночь?

* Они, наверное, не спят, потому, что у них много работы.

- Если я сейчас лягу спать, солнышко сядет?

* Нет. Ты ляжешь спать, когда солнышко садится, а не наоборот. Ты понял?

- Н-е-ет.

* А как?

- Когда хочу, тогда и ложусь.

* А, ну это тоже правильно, а ведь, когда солнышко садиться, ты ж больше хочешь спать, чем, когда днём солнышко светит? 

- Ну и чё?

* Ну, вот, когда солнышка нет, темно, тогда это называется ночь. А когда солнышко светит, это называется день. Теперь ты понял?

- Не понял. А если я не буду спать, а солнышка нету, это тоже день?

* Нет, это тоже ночь, потому что темно, но ты просто не будешь спать и всё. Потому, что день и ночь зависят не от тебя, а от солнышка и от Земли.

- Запутала ты меня.

* А?

- Запутала.

* Запутала, да? Мабу, а кто я такая? Почему ты со мной? Тебе же нельзя с чужими жёнами разговаривать, а ты со мной разговариваешь?

- А кто знает?

* Никто не знает?

- Никто.

* Всё ясно. А как меня зовут, тоже не знаешь?

- Не знаю.

* Не знаешь? А хочешь узнать?

- Не.

* Почему?

- Не знаю.

* Нельзя, да? Правильно, молодец. Сколько у тебя жён теперь, нука  говори?

- Почему теперь? Две.

* Две? А остальные где?

- Какие?

* У тебя ещё не было?

- Почему у меня должно быть?

* А земля где твоя? Тебе монахи дали землю?

- Дали чуть-чуть.

* Чуть-чуть?

- За Петю.

* За Петю. Ты без монахов сейчас? Монахов нет?

- Нет, стоит.

* Стоит монах?

- Да.

* А может, он хочет что-то спросить у нас… у меня, вернее. Я теперь одна. У нас?

- Не знаю.

* А ты у него спроси. Может, он что-то спросит у нас.

- А как я у него спрошу? Если я у него начну спрашивать, я тебя не услышу.

* Нет, ты сначала его спроси, к нему обратись, он тебя спросит, а потом ты мне скажешь, а я тебе скажу, а потом ты ему скажешь. Понял?

- Не понял, повтори.

* Ну, смотри, спроси сейчас у него, хочет ли он со мной говорить. Если он скажет: хочу, ты тогда уже мне скажешь. Понятно?

- Попробую.

* Попробуй. 
(Сбой) или смена кассеты.

* Твёрдый знак.

- Когда-то говорили вам о твёрдости знака. 

*Да.

Вспомните. Когда вы исключили его.

* Да, да помню.

- И что стало с миром вашим? 

* Ну, будем говорить так…

- Вы потеряли твёрдость.

* Да.

- Вы потеряли опору и, даже в ваших лозунгах, звучало: мы разрушим старый мир до основания. И на этих обломках  вы решили построить новый. Прекрасная жизнь, не правда ли? Почему вы тогда были слепы и видели на обломках, заметьте, на обломках новый мир. Простите, а из чего вы будете строить этот новый мир? Из тех же обломков? И что же? Получится тот же старый мир. Вы согласны?

* Согласна.

- Но вы почему-то его не видели, но наконец-то пришло время, когда опять всё возвращается. Почему? Потому, что вы построили всё-таки на обломках тех же, не используя ничего нового. (Сбой)

(Мабу)
- Я пробовал.

* Ну, и  как?

- Не получается.

* Не получается? Ну, ладно, потом получится, мы знаем. Мы знаем, что потом у тебя всё получится.

- Когда “потом”?

* О, скоро, когда ты семь жён получишь, тогда узнаешь. Вот тогда у тебя получится.

- Ой, врунишка ты!

* Нет, я не врунишка, вот посмотришь.

- У нас все хотят столько жён, только старейшина может, а я старейшиной не буду.

* Будешь.

- Почему?

* Мы знаем.

- Чего вы знаете?

* Ты очень будешь стараться, ты будешь с нами часто-часто разговаривать, и тебе дадут много жён, аж целых семь, и ты будешь старейшиной.

- Не-ет, вы гадаете. У меня жена так умеет гадать. Она берёт камушки на ладошку, потом говорит: закрывай глаза и выбирай. Я выбираю камешек. А какая в них разница? Они все одинаковые. Брешет наверное.

*Аха-ха!

 - А потом говорит: вот этот камешек, это, значит, ты будешь богатый, будешь … это б…

* Кто?

- Большой.

* А-а, большой!

- Ага. Ну, ладно, я тогда беру другой камешек, она говорит: ой, жизнь плохая, ты умрёшь, так и останешься, и все свои последние две жены  потеряешь. Дурочка какая-то!

* Ну, почему дурочка?

- Ну, пусть врёт одинаково! Не хорошо врёт. Не нравится, когда мне говорят, что будет большое.

* Ну, я тебя не обманываю! Мабу, правда, я тебя не обманываю! Посмотришь, прощения у меня тогда попросишь, ладно? Когда получишь семь жён, тогда прощения попросишь.

- Не попрошу. ( так и не попросил. Прим.)

* Ну, ладно, не надо. Ну, вспомнишь тогда, ладно, что я тебе говорила? Вспомнишь?

- А если не получится?

* Получится, это я тебе говорю.

- А не будет, если я тогда палку возьму.

* Ну, ладно, так и быть, если не получишь это, тогда побьёшь меня палкой.

- Да, и опять будет, как моя жена.

* Ну и чё?


- Я её палкой побью, она сразу говорит: вот хорошие камешки, хорошие камешки, и у меня всё хорошее, а как не бью, начинается всё плохое. Она, наверное, любит, когда её бьют.

* Любит?

- Наверное. А ещё, Ауба сказала, что…
 1,2,3

- Не, хитрая какая.

* Ну, почему же хитрая?

- Раз сказала, что всё можешь, значит отгадывай.

* Ну, значит, сейчас у тебя две жены.

- Правильно.

* Правильно?

- А как ты догадалась?

* Ну, как, как? Очень просто.

- А-а, а я тебя обманул(радостный смех) - три жены.

* Мабу, ну, разве можно так? Ты уже обманывать научился, да?

- Нет. Я проверял.

* Меня?

- Да.

* О, проверял, ты, значит…

- Я не буду палку брать.

* Вот теперь ты меня обманул, да?

- Не!.

* Как нет? Ты проверил, да?

- Сколько у меня жён?

* Чего?

- Сколько у меня жён сейчас?

* Сейчас?

- Да.

* Две.

- Не, три.

* Ну, ладно.

- Глупая.

* Ну, ладно, я глупая.

- Я же не могу знать, что такое “три”, если у меня будет только две жены.

* Ну, почему, разве до трёх ты не умеешь считать?

- Не. Сколько жён могу считать.

* Ну, сколько у тебя жён?

- Три.

* Ну, почему три, если ты до трёх не умеешь считать?

- Умею.

* А когда тебе третью жену дали? Скажи, сколько времени прошло? Сколько тьмы прошло?

- Тьмы – это много.

* Много?

- Конечно.

* Ну, тогда, сколько времени прошло? Примерно. 

- У старейшины…

* Урожай уже был?

- Два..

* Ну, вот, два урожая, и тебе третью жену дали.

- Да..

* Значит, уже много времени прошло, всё понятно.

- Не много.

* Это не много?

- Не-а.

* Всё ясно. Мабу, а ты скажи, вот как вы пшеницу выращиваете, как жёны твои выращивают, как она растёт? Из камней?

- А мне это не нужно знать.

* Почему не нужно?


- Ну, пусть жены и выращивают.

* А вдруг у тебя все жёны помрут?

- Ну да – помрут! Я им “помру”!

* А ты будешь один тогда. Тебе придётся одному тогда.

- Не буду. 

* Не будешь?

-  Ещё пойду возьму.

* А-а.

- Мне их не жалко.

* Не жалко?

- Они меня сегодня обидели.

* Чего они тебе сделали?

- Палки спрятали.

* А, ты, наверное, часто их бьёшь?

- Не часто, поэтому и спрятали.

* Хотели, чтобы ты забыл про палку, совсем.

- Хитрые какие.

* Хитрые! Ха-ха!

-Я всё их всё равно нашёл.

* А ты сейчас один, или с монахом?

- Один.

* Один? Это тебе уже интересно?

- Чего интересно?

* Со мной разговаривать?

- Не знаю, я лёг, уснул и вы теперь мне снитесь.

* А-а, я тебе приснилась, да?

- Да.

* А ещё кто-нибудь кроме меня есть?

- Не, не вижу.

* Не видишь? А ещё есть, ты его всегда считал, вот он там немножко подальше сидит.

- Чего считал?

* Считал.

- Чего?

* А вот там он сидит.

- Кто там сидит?

* А вот там вот. Вот там, в другой пещере.

- Не вижу.

* Не видишь? Всё ясно. А ты просто заснул, или тебе жена огонь принесла, чтобы ты на него смотрел?

- Нет, уснул.

* Просто уснул. Ты во сне теперь со мной разговариваешь?

- Да, ты мне снишься.

* И теперь ты меня видишь, да? Ты же видишь, а?

- Вижу.

* Ну и расскажи, что ты видишь ещё?

- Чего, чего? Тебя вижу.

* А ещё что? Пещера какая?

- Пещера?

* Да.

- Пещеру я не вижу, тебя вижу.

* Только одну меня видишь?

- Да.

* А что у меня, какая я? Расскажи мне, какая я.

- Такая!

* Красивая или не красивая, а?

- Не знаю.

* Ну, почему?

- Травы много.

*  На голове?

- Да.

* У нас вот так. У вас травы нет, а у нас есть трава. Мы такие!

- Ещё…

* Ещё чего?

- Одетая.

* Да, правильно, ещё одетая! Мы тоже в одежде ходим, как монахи, только у нас…

- А я что, голый что ли хожу?

* А ты в повязке ходишь.

- А это что, не одежда?

* Ну, одежда, только у нас больше одежды.

- У меня лучше одежда.

* Конечно удобней, да?

- Да!

*  А если я встану, ты меня будешь видеть? Ну, посмотри. Ты меня видишь так?

- Не вижу, чтоб вставала.

* Не видишь? Просто меня видишь и всё, да?

- Да.

* А я как, выше тебя ростом, или ниже?

- Не знаю. Не надо вставать, а то я тебя не вижу.

* Ну, ладно. А ещё на мне что одето?

- Э-э-э…Не знаю, как называется.

* Где?

- Оно  звенит.

* Звенит?

- Да.

* А где звенит? Я не слышу.

- Ну, звенит.

* Где звенит то?

- У тебя.

* У меня?

(Конец 1-й кассеты)


* (Ольга) Это знаешь, как называется? Это такое устройство…, как в пещере. Ты же в пещере был же у монахов, или ещё не был? 

- Нельзя туда.

* А-а. Ну, ещё нет. Ну, ты знаешь, это такой…, ну, скажем… ну, как… 
(Словосочетание Ольги – «Ну как», Мабу воспринял как название устройства (магнитофона), и далее называет его «нукак». Ольга же не сразу поняла, почему Мабу «нукакает». Прим.)

- «Нукак»?

* Это священный такой камень. Понимаешь?

- Ну, да.

* Понимаешь?

- Понимаю.

* Вот, он умеет говорить, он умеет говорить, этот камень. Понял?

- Понял.

* Ну, вот так.

- Что он говорит?

* А он, когда  мы с тобой разговариваем, всё здесь записывает, всё запоминает, всё, что мы говорим. А потом, когда ты уходишь, я нажимаю на другой камень, который находится на нём, на большом камне, и он начинает говорить твоим голосом. И я всё опять слышу, как мы с тобой разговаривали. Понял?

- А можно я монахам скажу, что у вас «нукак» есть?

* Скажи. Как называется?

- «Нукак».

* Нука?

- «Нукак», ты сама сказала.

* Хм …,  ну ладно, говори. (И ждёт, пока скажет) прим.

- Потом.

* А-а, потом. Ну да, я же забыла, что ты во сне, что ты спишь.

- Почему у меня такого нету?

* Потому, что тебе ж монахи обещали, землю дали.

- Дали.

* Ну, вот, а это нельзя давать, камень такой.

- Я ещё «нукак» хочу.

* Нет, он священный, его нельзя всем давать, только монахам можно.

- Но я не все!

* (смеётся) А почему ты не все?

- Почему это я все?

* Ну, ты не все, конечно, ну…, ты ещё маленький…

- Ну, да!

* Ну, почему “ну, да”?

- Ты большая, а я маленький, да? «Нукак» хочу!

* (Смеётся) Ну, ладно. Получишь. У монахов спросишь, может они тебе дадут.

- Я сам сделаю.

* Ну, сам…

- У меня тоже камни есть.

* Ну, ладно, сделаешь сам.
 (срыв)

- …относится к насекомым.

* Разве?

- Всё, что ползучее, всё для вас гадкое и противное. Интересно, не правда ли? 
(срыв)

* Вот, о насекомых говорили, так сказать, это не наши ли далёкие-далёкие предки как-то вмешиваются, те которые были, ну, как называют их…«селенитами» может быть?

- Думайте, думайте, ищите и догадывайтесь сами.

* Хорошо.

- И помните, что мы говорили вам, не бойтесь ошибиться, бойтесь не заметить ошибки. Это самое главное. Вы же…
 (срыв).
(Вышел на контакт Мабу.)

- Что здесь такое горячее?

* Где?

- Вот, то, что было!

* Горячее?

- Горячее.

* О, это огонь, наверное.

- Не-е.

* А что?

- Горячее!

* Какие…? Горячее, чем огонь? (Вздыхая) Ой…, ну, это как огонь, только который внутри у нас, далеко, далеко внутри.

- Чего внутри?

* Ну, у нас, как огонь горит… внутри.

- Э-э…, не понял. Ну, ладно.

* Поймёшь когда-нибудь.  (Смеётся) Мабу?

- Да.

* А сколько у тебя сейчас жён?

- Да говорил… – три!

* Три...?  Ты это опять, ты опять во сне меня  видишь? Это ты всё ещё спишь?

- В смысле всё  сплю? Ты что…? Я прощался с тобой что ли? 

* Да конечно, я уже с другими поговорить, успела.

- С кем?

* С кем….

- С «нукак»?

* Да.

- Жадина.

* Жадина….(смеётся) Ох…,  ну, ладно. Ну, спроси что-нибудь, Мабу, у меня.

- Что у тебя спросить? Я у тебя спрашиваю, а ты не даёшь, а мне больше ничего не надо.

* О-о, вот так! Тебе больше…. Как я могу тебе во сне…?

- Я вот проснусь и сделаю себе тоже «нукак», и не буду с вами разговаривать.

* Ладно. Хорошо. Мы принимаем твои обиды. Ты на меня не обиделся, а?

- Обиделся….

* Обиделся.… Да ну, Мабу, не обижайся. Ну, нельзя мне тебе дать. Понимаешь? Ты же во сне, а во сне как можно дать? Ты проснёшься, а у тебя этого нет. Это только, когда ты не во сне, тогда только можно тебе дать. Понял меня?

- Что дать?

* А?

- Всё равно жадина.

* Всё равно жадина? Ну, ладно, побей меня палкой, ты же обещал меня палкой побить. А?

- У меня палки нет, вот, в следующий раз лягу спать, и возьму  палку. Вот тогда посмотрим, как ты мне не дашь.

* Ну, ладно. (смеётся) А как ты думаешь, я старше тебя или моложе?

- Чего?

* Ну, вот я - старая или молодая?

- Чего?

* Ну, как вот - я - старая или молодая?

- Чего это такое?

* Ну, вот у тебя одна жена старая, а другая молодая, да? Нет?

- Да.

* Ну, вот. А вот я как - старая или молодая? А?

-  Как третья.

* Третья? 

-Да.

* А третья - это как? И не старая и не молодая?

- Вредная.

* А-а, вредная? А как третья, вредная…! (смеётся) Всё понятно, вредная. Вот так. А почему я вредная, потому что я тебе не даю?

- А ещё ты жадина!

* И жадина, да? И вредная.

- Да.

* А как её зовут?

- Аубу.

* А-а, Ауба, а ты её не любишь, да?

- Чего это я не люблю?

* Любишь?

- (застенчиво) Люблю...

* А-а, ну вот видишь. Значит я тоже не такая уж плохая.

- Причём тут это?

* Как причём?

- Можно и плохую любить. Я же люблю старую…

 * А-а…

- А может, не люблю. Я ещё и сам не знаю.

* Сам не знаешь? Сомневаешься, да?

- Чего?

* Сомневаешься.

- Чего “мняешься”?

* Ну, сомневаешься – это когда не знаешь, любишь, или не любишь, хочешь, или не хочешь, это вот так. Понял?

- Не понял.

* Ну, вот когда точно не знаешь, чего ты хочешь. Вот смотри, ты хочешь кушать...?

- Ну, тогда я ничего не знаю.

* А-а….

- Я ничего не знаю, чего я хочу. Я всё время чего-то хочу, а чего не знаю.

* Это значит, “сомнение” называется. А чего ты хочешь? Ты даже не знаешь, чего ты хочешь?

- Я всё хочу.

* А “всё” - это как?

- Всё - всё хочу.

* Всё, всё, всё?

- Да.

* Это “всё” - это как?

- Пещеру хочу, много жён хочу…

* И много жён, и старейшиной хочешь, да?

- И вообще, жён не хочу, надоели мне…

* Надоели…

- Не-е…хочу.

* Ты с ними ругаешься, наверное…?

- Нет...  Хочу, чтобы…, вот… когда хочу - были, когда не хочу – небыли. Во!

* А-а... (смеётся).

- Во!

* Всё ясно.

- «Нукак» хочу. Много!

* О-о! А зачем тебе много то?

- А когда жён не хочу, пусть они вместо жён разговаривают.

* А-а, всё ясно. (смеётся)

- Что я ещё хочу?

* Чего?

- Что ещё хочу?

* Ещё чего? Ну-ка говори, чего ещё, вспомни?

- Огонь хочу.

* Огонь?

- Да.

* Какой? В пещеру?

-  Старейшина, то даст, то не даст, паразит….

* А-а, значит, ты себя плохо ведёшь.

- Ну, да!

* Он тебя наказывает. За что он тебя наказывает?

- Я запомню.

* Запомнишь? Ему? А потом, что сделаешь?

- Не знаю.

* Тоже ему огонь не дашь?

- Обижусь.

* Обидишься? И разговаривать с ним не будешь?

- Как это я с ним разговаривать не буду?

* Ну, ты же обидишься.

- А как я у него огонь буду брать, глупая?!

* Ну, ты придёшь и так…, покажешь, что тебе нужно и всё, а говорить ничего не будешь, и он поймёт, что тебе нужен огонь.

- Ну, да!

* Не поймёт?

- Я должен произнести тайную речь, которую знаю только один я.

* А-а.

- Тогда я только могу взять огонь. А если я на него сильно обижусь, как я буду говорить?

* А-а, ну, тогда не обижайся совсем. Зачем тебе обижаться?

- Вот это и есть “мнение”, да? Когда хочу и не хочу.

* Да, да, это и называется “сомнение”.

- Я хочу обидеться и не хочу обидеться. Плохое мнение.

* (смеётся) Всё ясно с тобой. Ну, ты что, ещё не собираешься проснуться?

- Надоел?

* Тебе надоел? Мне надоел?

- Да.

* Нет, не надоел, но ты у меня ничего не спрашиваешь, всё время…

- Что я буду спрашивать?

* Ну, я же всё время у тебя спрашиваю….

- Ты спроси, и я у тебя спрошу.

* Ну, вот, я у тебя спрашиваю: « Мабу, ну, как у тебя дела?»

-  Мабу, как…Ой-ё-ёй…. Как у тебя дела?

* Как у меня дела? У меня хорошо дела. Вот, я работаю….

- Чего?

* Ну, как жёны у тебя работают?

- Ага…

* Да.

- Кормишь….

* Нет, я себя кормлю, у меня мужа нет.

- Чего нет?

* Мужа. У меня мужа нету - я одна живу в пещере, и работаю сама, и сама себя кормлю. Понял?

- Не, не понял.

* Не понял?

- Как можно жить одна?

* Ну, вот так получилось. Видишь у меня на голове, сколько травы много?

- Почему говоришь -  «хорошо»?

* Что хорошо живу?

- Да. Что в этом хорошего?

* Ну, мне нравится так.

- Странно….

* Почему?

- У нас так не бывает.

* Не бывает у вас так?

- Да.

* А у нас бывает.

- Ты, наверное, далеко живёшь?

* Да, далеко-далеко живу от тебя.

 (срыв)

- И как бы вы не пытались и не травили их, они всегда будут вокруг вас и рядом с вами. Подобное к подобному… 
(срыв)

-  Первое - это что? Это потеря собственных колебаний, это потеря восприятия энергии извне. И хотя в каждом из вас есть своя энергия, вы всё-таки работаете как аккумулятор, хватает ненадолго – это и есть время разложения вашей…

* Плоти.

- Плоти. И это есть - время отделения оболочек ваших. Вы привыкли говорить о семи телах, здесь неверно именно то, что нет разделений. Их столь много, столь много, что нельзя сосчитать. Но, вы не далеко ушли от истины, если всё-таки сказали, что “семь”,  смотря на их основные изменения. Давайте возьмём так, цвет глаз. О чём говорит цвет глаз? Он говорит о том, какой цвет, или, как вы говорите, какая «чакра» больше владеет вами. И если вы имеете глаза голубые, значит, в вас более  присущ голубой цвет. Что приносит он? Вы смотрите внимательно, более чаще - голубые глаза носят именно те люди, которые обладают, или жестоким нравом – холодом,- или наоборот - очень добродушные. Видели вы глаза красного цвета?

* Нет. Вот я хотела спросить, на фотографиях, на цветных, иной раз, у людей получаются красные глаза. Это о чём-то говорит, или это дефект?

- Ну, давайте продолжим. Видели ли вы красные глаза?

* Да нет, я ни разу…

- Нет. Вы скажете, что таких глаз не бывает, и всё же вы говорите, что глаза “налились кровью”.

* Да.

- Теперь, давайте представим, вы живой человек, или  ваша копия, запечатлённая на бумаге. В чём разница?

* Ну, тут разница во многом…. Здесь чисто внешне может быть…, внешность запечатлена?

- О нет! Хороший фотоаппарат зафиксирует всю вашу внешность. Ну, давайте скажем так. Первое, что вы можете сказать - у этой фотографии нет души. Хорошо, прекрасно…Тогда почему же многие могут определить по фотографии состояние здоровья, жив он, или не жив, и даже прошлое и будущее? Как вы это объясните? Присутствием души?

* Нет.

- Фотографией? Нет.

* Информация заложена, наверное…

- Информация именно в материи, в материи! Так что такое смерть? Это потеря информации.

* Об этом мире, да, где живёшь?

- О нет, зачем же. Мы же говорили вам, что смерть – это потеря внешней энергии, иными словами - той же самой информации. Тело просто забывает, как должно работать, - «информационный голод». И хотя в нём   остались ещё воспоминания, и здесь можно применить законы инерции, - вступает новая стадия, когда в вас уже просыпаются иные колебания, если быть точнее, - в вашей плоти. Именно колебания,  заставляющие  вашему телу распадаться. Проявляется новая жизнь, а если быть точнее – активируется, потому что она  была у вас всегда. Вы согласны?

* Да.

- И вот вам – «победитель червь».  Теперь возьмём «священных» людей, не поддающихся гниению. Это говорит о чём? Это говорит о том, что не была потеряна информация. Можно сказать, что она была заморожена. Можно сказать, что тело не поддаётся гниению, а значит, не создаёт низших форм жизни, поглощающих это тело. Вот вам и “святость”! Давайте пойдём дальше. Что болезнь, что такое болезнь? Это опять - забывчивость материи. Это опять «асинхронизм» колебаний извне и внутри. Давайте дальше. Что такое “заговор”? Что такое “сговор”? И что такое “сглазить”? Что? Это как раз произношение тех колебаний, которые собьют ваши и принесут болезни. И здесь рано или поздно учёные придут к выводу, что существует наговор, именно из-за того, что изменения физических полей приносят другое…

*Один, два… (счёт)

-(Мабу)  “Три” сейчас скажешь, я знаю. 

*  Я?

- Да.

* Правильно. А ты до скольки уже научился считать?

- Четыре.

* До четырёх? Ну- ка, посчитай.

- Один, два, три… и один.

* Правильно, молодец. ….

(Вырезан кусок. Разговаривают уже другие. Прим.)

- …слишком сильное притяжение у неё.

* Да.

- Они не могут попасть ни в какой  другой мир, и тогда, они носятся здесь, - когда-то мы вам объясняли, о полтергейсте,- и одним из видов его являются “иные”. Это те, которых не отпускают родные…. Что значит “держать”?

* Да я знаю, примерно.

- Хорошо.  Это те, кто сами не хотят уйти, это те, кто хотели, стремились быстрее покинуть Землю любым путём. Самоубийцы – одни из них. Существуют, в вашем понятии, семь уровней, если хотите, семь уровней иного бытия, относящегося именно к смерти. Каждый уровень имеет довольно-то жёсткие границы. Заметьте, мы говорили о теле, не имеющем границ, и тут же говорим о жёстких границах. Эти границы очень трудно пересечь, и первые, первые три - будут вашими воображаемыми мирами. Именно те религии [в которые вы верите] отражаются в этих мирах. Именно те [ваши] фантазии отражаются именно в этих мирах. Если вы…
(сбой)
*Один, два, три, четыре, пять, шесть, семь, восемь, девять…

(Вышел Мабу.)

- Теперь я стал одним из [них] и должен раздавать не только огонь, но и учить их. Теперь я должен провожать их в пещеру, где ждёт «время»  и берёт себе. Ещё я должен учиться читать книги све… (не может выговорить) прим.

* (Ольга подсказывает) Священные…

- Священные. И выучив их, и поняв их, и увидев их, только тогда я смогу одеть одежды эти. Не имея…
 (срыв) Ольга не поняла и уточняет

* Не имея чего? …. (срыв, или обрыв плёнки. прим.)

(Продолжают говорившие перед Мабу.)

- … это когда вы переходите на низкие…, или как вы говорите «адовые планы». Представьте, религия говорит о боге столь карающем, что за малейшую ошибку вы будете сожжены. Что произойдёт? Вы попадёте именно к этому богу, именно этот бог встретит вас, и именно этот бог сожгёт вас. И будет сжигать до тех пор, пока вы не поймёте, что это иллюзия. И когда вы поймёте, что этот сжигающий бог есть и ВЫ…, ВЫ и есть этот бог…, только тогда вы сможете пересечь границу. Раньше же - вы будете постоянно, неоднократно сжигаемы, и вы будете чувствовать все боли, какие могли бы чувствовать на Земле, имея плоть. И любые вами придуманные казни тут же будут осуществляться, пока вы не поймёте, что это мир иллюзий, и вы и есть тот бог, что карает вас. Итак, поняв, что вы и есть тот бог, вы переходите на другой уровень, на более высокий, но он же является всё же миром иллюзий.  Простите, в любой религии есть и добро и зло.

* Да.

- Хорошо. Вы перейдёте в «план добра», где все ваши лучшие желания будут выполняться. Вы это назовёте «Раем», и будете в нём жить до тех пор, пока не поймёте, что это тоже иллюзии, и что Ангелы, прилетающие к вам, это вы же сами. И когда вы соединитесь с ними, вы придёте опять в новый «план», но этот план - тех же иллюзий, но на более высоком уровне. Здесь иллюзии бы создавались уже вашей Душой. То были планы сознания - теперь план сознания Души, когда оба владействуют этим уровнем. Теперь представьте, что такое план сознания души. Представьте этот сплав! Это и есть ваши сны. Можно назвать этот план – снами…, но с натяжкой. Здесь вы можете совершать любые подвиги, здесь вы можете быть владыкой всего, но над вами будет вставать кто-то, который будет руководствоваться вами. И до тех пор, пока вы не поймёте, что этот владыка есть и вы…,-  мир иллюзий – не забывайте, и когда вы поймёте это, тогда вы перейдёте дальше. Если же не смогли понять того, то, рано или поздно, вам это будет надоедать. Рано или поздно, вы уже будете получать обратное. Вас этот мир уже будет пугать. Вас этот мир будет пугать, и вы, не зная, что этот мир создан вами…, вы пожелаете уйти из него. Или другими словами – вы убежите, убежите от этого «владыки» - от самого себя! Куда? Конечно же ниже.

* На Землю опять?

- Зачем же на Землю…, На «нижний план». И так вы можете делать это до бесконечности. Где-то было сказано, что от смерти до рождения нового – сорок девять дней. Ну что же, есть и такие цифры. Давайте пойдём дальше. Четвёртый план, в вашем понятии, это когда уже нет сознания, но есть душа ваша. Душа, но сознание всё же исказило понятие о душе. Тот же самый мир иллюзий, но более и более обманчивый, потому, что здесь трудно увидеть себя, увидеть обман. Здесь идёт “честная игра” - в вашем понятии, но эта “честная игра” идёт по вашим правилам. По вашим правилам. И когда вы это поймёте, когда вы поймёте, что все боги, которые встретятся вам здесь - это и есть и вы, вы сами, только тогда вы сможете перейти на другой уровень… 
(срыв)
*Один, два….
(Необычная ситуация: Вышли на «контакт» назвавшиеся хранителями «времени Юга». Притом получилось, что Ольга оказалась для них - у них, в сопровождении «мальчика». В диалоге с их стороны идёт выяснение личности Ольги, и мотивов её «прихода» к ним. Ольга сначала не понимает сложившейся ситуации.)

-… мы вас видим. Кто вы, и зачем пришли сюда?

* Зачем я пришла? А вы зачем пришли?

- Мы были здесь, вы пришли.

* Я здесь живу…

- Вы не можете жить в этом мире, у вас есть плоть.

* А почему я не могу жить в этом мире? (Думая, что говорят про наш мир. Прим.)

- Вы пришли сюда. Мы не знаем вашей цели, но мы хотим узнать, как вы пришли сюда.

* А-а…?

- Какая дверь осталась открытой?

* А о каком мире вы говорите?

- Мы говорим о нашем мире. 

*  (Растерянно) Ну… как..?

- Вы хотите сказать, что вы заблудились?

* Может быть…

*О каком мире вы говорите?

-Мы говорим о нашем мире. Вы хотите сказать, что вы заблудились.

*Может быть.

-Хорошо, тогда куда же вы шли?

*Мы шли к нашим друзьям.

-И где находятся ваши друзья?

*Наши друзья находятся на стыке вселенных.

-Вселенных или Вселенной?

*Вселенных.

-Вселенных? Вы хотите сказать, ваши друзья столь сильны? Как же вы тогда заблудились? Неужели они не дали вам огня?

*Наверное, я в чём-то виновата.

-Хорошо, как вы пришли сюда?

*Давайте познакомимся.

-Давайте.

*Я человек, а вы?

-Мы видим, что вы человек, мы тоже не звери.

*Мы понимаем, конечно, но можете как-то себя обозначить?

-Давайте договоримся, вы не называете имени, но даете понять.

*Я должна дать понять? Кто я?

-Да. Но пожалуйста, без имен.

*Я - женщина.

-Это мы видим. Хотя мы могли бы здесь поспорить с вами.

*Ну, может быть… Я женщина вот в этом мире, в своём мире, где я живу.

-Хорошо, кто рядом с Вами?

*Рядом со мной? Переводчик.

-Нет. Кто рядом с вами?

*Со мной? Я одна.

-Одна. Хорошо, кого вы держите за руку?

*Это прибор. Инструмент искусственный, созданный людьми. 
(Ольга не правильно поняла их. Она думала, что речь идёт о магнитофоне.прим.)

-Или вы не можете увидеть себя, - действительно заблудились,- или вы хотите нас обмануть.

*Наверно, я заблудилась, если вы так считаете.

-Вы не видите, кого вы держите за руку?

*Сейчас?

-Да, сейчас.

*Нет. Переводчика?

-Хорошо, опишите себя.

*Внешне?

-Да.

*Рост? Сказать рост?

-Зачем нужен ваш рост?

*Но вы сказали…

-Опишите себя.

*Внешность? Ещё раз спрашиваю.

-Да, внешность.

*Я шатенка, вам это что-то говорит? У меня темные волосы.

-Похоже, вы действительно заблудились или очень легко нас обманываете.

*Нет, я вас не обманываю. Вы просите описать меня свою внешность. Так? Внешность, которую я имею сейчас в своем мире. Так?

-Нас интересует внешность сейчас - в нашем мире.

*А, в вашем мире. И вы от меня защищаетесь?

-Мы хотим понять, как вы смогли прийти сюда.

*Да я и сама не пойму.

-Хорошо, вы не можете описать себя?

*Кто я есть? Вы видите лучше меня. Вы судите, кто я такая -  враг вам или друг, или может быть просто я, действительно, заблудилась.

-Мы не можем раскрыть вас без надобности.

*Ну, тогда не стоит. Как вы меня воспринимаете? - Как враждебную агрессию, нейтральную, или ещё что?

-Множество красок. Поэтому мы и хотели спросить, какую же вы предпочитаете одежду? У вас их слишком много.

*Одежд слишком много?

-Да.

*А вы не имеете этих одежд?

-Итак, вы не знаете, кого вы держите за руку. Вы не знаете, куда вы пришли. Вы не можете описать себя. Значит, вы не знаете себя?

*Наверно, нет.

-Нет? Прекрасно. Хорошо, ваши друзья на стыке вселенных - и не дали вам огня?

*Вы знаете…

-Как назывались они?

*Они назывались… Они не называли своего имени. Этого нельзя делать.

-И мы у вас не просим имени. Но они, должны же были дать понять, кто они. Или вы дружите с кем попало, не зная их?

*Нет. Они сказали, что они одна из наших реакций. 

-Простите, у вас множество одежд, множество реакций. Какую из них вы говорите?

*Какую из них?

-Если Вы заблудились, то, как же могли ваши реакции попасть на стыки вселенных, а не вселенной?

*Так мы, я так поняла, что…

-Что Вы тогда знаете?

*О чём?

-Вы не видите себя, вы не знаете своих друзей, вы не знаете, куда вы пришли.

*Мы разговариваем через переводчика. Они, обычно, нам… Вы знаете, что такое переводчик? Я сейчас тоже с переводчиком разговариваю,-  с вами через переводчика.

-Не тот ли, кого Вы держите за руку?

*Да, наверно, если вы считаете, что мы держим...

-Тогда, значит, он привел Вас?

*Наверно.

-Хорошо, тогда мы будем спрашивать его. ( 4 сек. тишины. прим.)

*Вы меня слышите?

 (5 сек. тишины. Прим.)

-Вы не слышите сейчас переводчика?

*Слышу.

-Вы слышите его разговор… 

*Да.

-…с нами?

*Нет, я только чувствую.

- Вы только в начале? ( в начале развития . прим.)

( обрыв записи)

-Слышите?

*Нет, мы ещё настолько несовершенны, что можем считать себя глупыми, в какой-то мере,- в этом, в своём мире, может быть.

-“Глупости” не бывает нигде,-  не в нашем и не в вашем мирах. Есть только незнание. И всего лишь.

*Да, наверно, так правильней.

-Давайте назовёмся.

*Давайте.

-Итак, мы хранители времени Юга.

*Юга?

-Это ничто не говорит Вам? Только не путайте со сторонами света.

*Нет, я не путаю. Это у нас есть в Индии - Кали Юга, Сатья Юга. Так? Нет?

-Нет, но Вы будете рядом. Удивительно, как Вы смогли прийти сюда!

*Я не знаю. Я только чувствую что то, но не понимаю это. И переводчик тоже там?

-Мы видим, что достаточно чисты. Переводчик хранит тайну.

*Переводчик хранит…?

-Что ж, это его право.

*Конечно. Вы слышали сейчас звук какой-то?( собака лаяла. У Ольги жил пёс Микки.прим.)

-Да, мы слышим.

*А вы знаете, что это животное? Оно вам не помешает?

-Нет, Вам мешают животные?

*Нет, наоборот.

- Вы нас успокаиваете.

*Давайте поговорим тогда.

-Что Вас интересует?

*Я не знаю. Я вот просто думаю, не повредит ли переводчику наше такое общение?

-Если мы не будем трогать его тайну, то не должно.

*Значит, вы хранители Кали… Как вы сказали?

-Нет.

 *Время Кали. Ой, - Время Юга.

-Нет, времени Юга.

*Вот так, время Юга.

-Времени Юга.

*Времени Юга. Вы может по-подробней что-то объяснить по этому поводу?

-Можно. Есть понятие - о множестве времени. Как мы поняли, вы находитесь во времени 3 целых.

*14 сотых.

-14?

*А сколько?

 - Да нет. Хорошо, скажите, как, в вашем понятии, можно находиться вне целостности времён?

*Как? Вне…?

-Вне целостности времён.

*Вы знаете… мы об этом мало знаем, если не сказать больше.

-Хорошо, Вы не знаете, но пользуетесь этим. Если бы у вас было ровно 3, вы бы не смогли прийти в этом мир, и вы не имели бы понятия   каких-нибудь о внутренних сил. И как Вы говорите “14 сотых” это и есть то, что позволяет вам овладеть иными мирами. Вот вам и понятие времени, вот вам и понятие этих цифр. Мы  находимся во времени Юга, мы говорили уже вам об этом. 

*Да.

-Что это значит?

*Да.

-Это значит, что у нас, в ваших цифрах, это 0,001. Это одно из составляющих времён.

*Времён. А сколько всего? Если можно так, - в цифрах, допустим,-  предположить?

-Это зависит от вселенных. В каждой вселенной свои меры, и даже на каждых планетах они другие. Если Вы переместитесь на другую планету, то Вы можете там оказаться неожиданно волшебником, если это будет более вашего времени – 4,5,6. А может быть и наоборот.

*Так, значит, мы живём в одной вселенной?

-Нет, вы ощущаете одну вселенную. Внутри вас, их уже - множество.

*Ну, это да. Это мы знаем. Подозреваем, вернее, скорее всего. Ну, вы не выяснили, с кем мы общаемся, да? То есть, кто наши друзья, с кем мы имеем … эксперимент, если можно так, “проводим” сказать? Вы не выяснили всё-таки.

-Ну, почему же.

*Или вы догадываетесь?

-Мы можем вам нарисовать геометрически, чтоб было понятно вам. Это прямая линия. Цепь. Цепь, имеющая много звеньев. В одном из звеньев находитесь вы.

*Да. Мир един, мы это часто произносим, но мало об этом знаем. Вернее - не можем понять, как это. Вы понимаете это? Если вы хранители времени.

-У вас много слов. Очень много слов не значащих ничего. То, что вы сейчас произнесли, не очень-то  сильно изменило картину, не очень-то сильно вы “распахнули одежды”. Вы не можете описать себя, вам не хватает тех “сотых”, чтоб увидеть себя. Вами уже созданы зеркала? Отвечайте.

*Зеркала? Если говорить о зеркалах, которые могут увидеть нашу внутреннюю суть, то я бы не сказала, что они могли быть чистыми, или не совсем кривыми.

-У вас ещё нет этих зеркал. Значит, вы ещё только в начале.

*Да. Наверно, так. Вы в нашем мире никогда не были?

-Как мы можем быть в вашем мире? Мы и есть в нём, и нет. 

*Вот вы  впервые, будем говорить так…?

-Мы впервые видим вас.

*А!  Это я попала в ваш мир!

-Опять вы не правы. 


*Опять не правы?

-Вы должны были бы быть внимательны. Мы были удивлены, из какого времени вы – значит, всё-таки, мы не контролировали и не были в вашем мире. Но если мы скажем, что мы не знаем о нём,  то мы солжём.  Мы говорили вам, о тысячных. 0,1 тысячная, вы должны помнить это.

*Да-да.

- Что это значит? Это значит - отсутствие каких либо времён?

*Нет, почему же?

-Вот именно. К сожалению, мы ещё имеем понятие о времени – к сожалению.  Тем более вы.

*А, понятно. Когда исчезнет эта единица…

-Давайте скажем так, когда ещё не было, в вашем понятии,  ничего - был “0”. Мы - в начале рождения всех миров  - 0,001. 

*Ах, вон как!

-Вы же, по сравнению с нами,-  очень молоды.

*Я поняла. Значит, мы попали с переводчиком наверно, да?. Или я одна?

-Мы не знаем, называть его “переводчиком” или нет.

*У нас так принято.

-Давайте, мы вам его опишем, только не вашими понятиями.

*Давайте. Если можно.

-Хорошо. Вы носите цвета близкие к холодному.

*Вы обо мне говорите или о переводчике?

-Нет, о Вас.

*Да, я подозревала это.

-Возьмёмте ваш цвет - жёлтый. Скорее всего, он и позволил пройти вам сюда. Правда, мы не можем понять одного, как это сделал переводчик? Как он мог ухватить ваш ключ, если, простите, он не знает даже своего? Или здесь вмешались “друзья”…?

*Да вряд ли. Они никогда не вмешиваются.

-…или мы в чём-то ошибаемся.

*Я не думаю, что они вмешиваются. Они не склоны к насилию. Вы меня слышите?

(На связь вышли “предыдущие” .прим.)

-…мнения в вас, что допустили подобное.

*Нет, я поняла. У меня было…была такая мысль. Но потом, если честно говоря, я, так , немножко, поверила.

-Мы потеряли Вас. 

*Потеряли?!

-Хорошо, что нашли и нашли вовремя. Понимаете, что такое “прийти в начало”, прийти в начало с вашим маленьким “багажом”?

*Понимаю.

-Это страшно. Хорошо, что они не стали предпринимать ничего.  Они, или вы, всё-таки смогли убедить, что вы действительно заблудившиеся, а не ищете. Множество приходит, чтоб украсть. Они поверили вам, или вы не успели себ….
(сбой. Идёт счёт Ольги)

-Что удерживает вас дать больше счёт?

*Мы никогда так не делаем. Мы считаем до 9 или после 11.

-Мы просили вас после 11, и почему вы не стали этого делать?

*Это вы, которые “хранители времени”?

-Ответьте, почему вы не сделали этого?

*Дело в том, что у нас ещё не было такого никогда, я, например, в очень большой растерянности.

-В растерянности вы отказали?

*Вам?

-Вы отказали дать счёт. Тогда вспомните, что говорили вам об этом счёте. Тогда вспомните, когда было время, когда он уходил в прошлое, но только для того, чтоб защититься. И Вы мне скажете, что не знали об этом? Если не знали, то зачем же тогда  делаете это? Потом, они всё-таки сумели найти его и в прошлом, и прошлое стало помогать ему. Что произошло сейчас? Сейчас ему надо было вспомнить старую защиту, уйти в то прошлое, именно в то прошлое, что было в начале ваших контактов, - вернуться в начало контактов. Вернуться к началу ваших контактов, чтобы он потерял, сбросил багаж всех этих контактов. Для чего?

*Да.

- Да чтоб они не могли найти его. Сколько имён было названо вами?

*Имён?

-Множество.!Вы говорили, что ошибаетесь – “ простите, мы не будем этого делать”, и тут же это повторяли, и тут же это делали! Вы же говорили - “Это же название города!”.

*Этого нельзя делать? И город нельзя называть?

-И, спрашивая о вирусах – называли имена. Простите, этого нельзя делать. И если только, как вы говорите “переводчик” называет имя – так и пользуйтесь этими именами, но не называйте своих. Сколько назвали вы? Благодаря этому теперь к вам может прийти кто угодно. Тем более, если вы сумели прийти в начало, представляете, сколько сейчас будет объявлено на вас охоты? (вероятно предостережение о том, что теперь по этим “вешкам” могут прийти души с низлежащих планов . прим.)

*Да. Что нам делать?

-Дайте больший счёт.

*Вы объясните – зачем, сначала. Объясните, зачем?

-Мы только что делали, только что говорили о  начале контактов.

(конец 2-й кассеты и контакта.)
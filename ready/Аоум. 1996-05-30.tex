Аоум. глава 35-я 30-05-1996г
Георгий Губин
Контакт 30 05 96


Места отмеченные «(…)» 
нуждаются в прояснении  

* (Белимов)Сегодня 30 мая 1996 года, мы вновь собрались на сеанс контакта, причем спрашивающий давно уже не был и сомневался, состоится ли он. Нам удалось выйти на Вас. Как Вы оцените состояние переводчика, он способен работать?

- (Говорун) Атоэнэсхил (неразборчиво).

*(Белимов) Ещё. Не расслышали.

- (Говорун) Атоэнэсхил (неразборчиво).

* (Белимов) Вы говорите на и иностранном языке?
* (Гера) Повторите, пожалуйста.
* (Белимов)  Астенес хил… Ну, мы так разобрали. Но мы не уверенны, что правильно поняли наречие.

- (Говорун) Ноу (неразборчиво)

*(Белимов) Вы знаете?..

- (Говорун) Атоэнэсхил. (неразборчиво)

* (Гера) Here(англ) – здесь. “Кто здесь”,- да? Слушай, мы, наверное, на англичанина вышли.
* (Девушка) Что?
* (Белимов) What  is your name?

- (Говорун) Сноус (неразборчиво)

* (Гера) How?
* (Белимов) No, no. Don't remember you.

- (Говорун) I am’ fill foo life. Groopo.(неразборчиво)

* (Белимов) Прекрасно, прекрасная речь. Вы можете еще более...
* (Ольга) Do you speak Russia?

- (Говорун) No.

*(Белимов) Do you speak English?

- (Говорун) No.

*(Ольга) Do you speak France?

*(Гера) Franch.

(тишина)

*(Гера) Parlie vuos france?
 *(Белимов) “NO” – это, как раз, инглиш.
*(Ольга)Мы не знаем такого языка.
*(Гера)Le france pu ve vues?

- (Говорун) Это мертвый язык.

* (Ольга) Французский.
* (Гера) Нет, вот, это латинский?
*(Белимов) А-а. Вы говорили на латинском?

- (Говорун) Латинский более свежее.

* (Гера) Ещё древней латинского.
*(Белимов) А откуда у переводчика эти знания?

- (Говорун) А Вы не знаете?

* (Гера) Ещё с прошлой жизни, наверное.
* (Ольга) Да, знали, конечно.
* (Белимов) Догадываемся.

- (Говорун) Мы, сначала здоровались?

* (Белимов) ..Спасибо. От Вас каждый раз, новые сюрпризы! Так почему Вы решили, что сегодня надо так с нами пообщаться?

- (Говорун) Много ли вы знаете слов, как вы говорите, иностранных?

* (Белимов) К сожалению, нет. У нас нет практики.

- (Говорун) Хорошо!  Хотя бы, сумейте поздороваться три раза на разных языках.

* (Белимов) [вздыхает]. Good day.  Chiao Bambino – это до свидания… Нет, не можем. 
* (Гера) Chiao — это, здравствуйте. По-итальянски - здравствуйте.
* (Белимов) Здравствуйте? Ну, да, больше не помню.
* (Гера) Salut!

- (Говорун) Ну, хорошо! Вы изучали немецкий. Скажите - спасибо?

* (Белимов) Я - английский. Я немецкий не изучал.
* (Гера)  Я - английский.
* (Ольга) Немецкий я не изучала.
* (Гера)  Ваш ключ.
* (Белимов) “Thank you” по-английски, это я помню.

- (Говорун) Один из вас изучал немецкий.

* (Белимов) А! Это, наверное, переводчик. Я, лично, не изучал.

- (Говорун) А вы уже делите?

* (Белимов) А? 
* (Девушка) Делите?

- (Говорун) Тогда назовите себя.

* (Белимов) Ну, была просьба не называть!

- (Говорун) Кто из Вас «переводчик»?

* (Гера)  Вот, который с нами говорит.
* (Белимов) С нами говорит. Через которого вы говорите.

- (Говорун) Но мы не видим отдельно.

* (Гера) Простите.[кашлянул] А вы, извиняюсь, кто?

- (Говорун) Имя?

* (Гера) Нет. Э-э-э… Вы, те же, кто снами всегда говорили? С самого начала? Личности теже? Сущности.
* (Белимов) Вы впервые с нами общаетесь?

- (Говорун) А вы знаете те сущности?

* (Гера) Ну, Они, по крайней мере, знают себя. Они могут себя идентифицировать, допустим, себя “те”, с себя “этими”. Они, хотя бы, так, знают, что это были Они говорили мысленные посылки. Разговор происходит на энергетическом, наверно, уровне. На мысленном.

- (Говорун) Мы просили вас назваться.

* (Гера) Мы - жители...

- (Говорун) Давайте договоримся, - сколько Вас? Если Вы уже говорите…

* (Гера) Ага! Вообще, в данном помещении - четыре человека.

- (Говорун) Кто из них переводчик?

* (Гера) Переводчик - тот который лежит.
* (Белимов) Его имя Геннадий.
* (Ольга) Не надо говорить.
* (Гера) Имя не надо говорить.
* (Белимов) Имя тоже нужно иметь. Нас хотят узнать.
* (Гера) Нет. Вот, как сказать… Вы...

- (Говорун) Почему Вы боитесь представиться?

* (Гера) Нам, вообще-то, сказали, что нельзя.
* (Ольга) Имена нельзя называть.
* (Белимов) Нам, почему-то запрещали  другие, кто к нам приходил.
* (Гера) А может быть, и те же.
* (Белимов) А Вы? Какое имя...?

- (Говорун) Знаете ли Вы значения имён?

* (Белимов) Смутно. Но хотелось бы, хотелось бы узнать больше. Если Вы нам сейчас что-то разъяснили, мы были бы благодарны за это. Вот, допустим, ”Геннадий”, уже названо имя. Можете сказать его значение?

- (Говорун) Да, можем.

* (Белимов) Так.

- (Говорун) А Вы?

* (Белимов) Ну? мы знаем, что…

- (Говорун) Вы даже уже интересовались.

* (Белимов) ...в переводе – “Благородный”. Ну, там есть ещё и другие значения. Твердо не помню.

- (Говорун) Хорошо. Слово “Благородный”, как будет звучать на нашем языке?

*(Белимов) На каком?
*(Ольга) На Вашем?  На Вашем?

- (Говорун) Пусть это будет латынь, если вы хотите.

*(Белимов) Ну, вы о нас большого мнения, наверное, хорошего. Мы эти языки не знаем. Мы не изучали их.
*(Ольга) А на.., на чём – “на вашем” языке?
*(Белимов) Ну, на латыни.
*(Ольга) Нет, э...
*(Гера) У них свой, наверное.
*(Ольга) Так, значит, Вы - личность?

- (Говорун) Безусловно.

*(Белимов) Ну, мы-то разговаривали с энергетическими субста… сущностями. Они себя идентифицируют.

- (Говорун) На эту тему мы Вам ничего не говорить.

*(Гера) Скажите, вы - житель Земли?

- (Говорун) Да.

*(Гера) А в каком времени вы живёте? В вашем исчислении.

-(Говорун) Точка отсчёта?

*(Белимов) Ну, от Рождества Христова хотя бы.
* (Гера) Да.

-(Говорун) Мы не знаем, о чём вы говорите.

* (Гера) А! Значит, до того как.(До времени Христа. Прим.)Ну, хорошо...
* (Ольга) Простите, а Вы - человек?

-(Говорун) Да.

* (Гера) Вы, наверное, на двух ногах ходите, да?

-(Говорун) Человек.  На двух ногах.

*(Ольга) [пытается задать вопрос]
*(Гера) Ну, погоди.(Ольге) Э-э-э… Скажите, Вас зовут не Ор?

-(Говорун) Нет.

*(Ольга) А вы кто по профессии?

-(Говорун) Я? По вашему,  это будет “говорун”.

*(Гера) Ага! Понятно, а у вас там как? Племя, город? Как Вы живёте? Общественный строй?

-(Говорун) Вы включили много нам не знакомых слов.

*(Гера) Ну... Вы живёте по отдельности, или общиной, там… Ловите рыбу вместе там, чем-то занимаетесь, плоды собираете, пшеницу сеете или как? 

-(Говорун) А как вы живёте? Неужели человек может жить отдельно?

*(Белимов) Нет, мы тоже в обществе. У нас в Солнечной системе.

-(Говорун) Вы решаете, что если мы в вашем понятии “Древние”, то, наверное, живём в пещерах? Нет, есть и города!

*(Девушка) А Вы как свою эпоху называете?

-(Говорун) Как её можно назвать?

*(Ольга) Ну, вот так… Мы живём во время, допустим, во время...
*(Белимов) После Рождества Христова, там.
*(Ольга) Да. Там, мы – палеозойская, там, называем.

-(Говорун) Когда-то мы поняли  -  “Рождество Христово” вы называете точкой отсчёта. 

*(Гера) Нет. Условно.
*(Белимов) Ну, условно, конечно. Уже несколько тысяч лет оно длится. 
*(Ольга) А у Вас, имя Христа вам не знакомо?

-(Говорун) Нет.

*(Гера) А какие у Вас имена самые известные?
*(Девушка) Это...
*(Белимов) Это, может, Атланты, Атлантида?  Вам что-то говорит это?

-(Говорун) Нет. Мы называем себя Аудэани. 

-(Гера) Ауданяне?

-(Говорун) Вы не можете повторить.

*(Белимов)Не можем. Первый раз слышу, даже, это.
*(Гера) Плохо расслышали,  наверное.
*(Белимов)Ещё раз повторите?

-(Говорун) Хорошо. Тогда, я назову свое имя — Асьолла.


*(Белимов)Да… У нас такие имена не используются.
*(Ольга) У вас какой цвет…

-(Говорун) А мы слышим даже, что у Вас нет  и множества звуков, владеимых(нами?).

*(Белимов) Есть. Альола…  Мы даже с трудом это повторяем это слово.
*(Ольга) У вас Солнце какого цвета?

-(Говорун) Солнце?

*(Ольга) Да.

-(Говорун) Оно может менять цвета?!

*(Гера) В середине периода, да. Вернее, не Солнце...
*(Ольга) В зависимости от эпох. От эпохи. У нас, например, Солнце жёлтого цвета, а у Вас?
*(Гера) От преломления спектра.
*(Белимов) Мягкого жёлтого цвета. А у Вас, какого?

-(Говорун) Представьте его!

*(Белимов) Яркий. Глаза болят. Диск,эээ...

-(Говорун) Я говорю, представьте только его. Тогда, мы поймём.

*(Гера) Давайте, представим его.
*(Белимов) Я представляю. Неужели вам слово “Солнце” ничего не говорит? А ”Луна”?

-(Говорун) Нам не говорит, обозначение ваших цветов.

*(Гера) А!
*(Белимов) А Луна? Есть у вас такое понятие, если вы цивилизация, человек?

-(Говорун) Луна?

*(Белимов) Да. Она светит ночью.

-(Говорун) И что это?

*(Ольга) Это ночью светило, ночью.
*(Белимов) Ночное светило, спутник Земли.

-(Говорун) Ночь?

*(Ольга) Ночь, это когда темно.
*(Белимов) Темно — чёрный свет.
*(Гера) Когда вы выходите на улицу и не видите у себя под ногами.
*(Белимов) У вас же есть города,  вы  как-то живёте? Не постоянно же на свету?
*(Ольга) Вы, когда спите… У вас сон есть?
*(Гера) Давайте по-одному. Тут, потом, каши… не расхлебаем.

-(Говорун) Луны — нет.

*(Гера) Так.

-(Говорун) Солнце? 

*(Ольга) Днём. 

-(Говорун) Цвет. Если мы вас правильно поняли, то, наверное, то же. 

*(Ольга) То же, да? А у вас есть численность? Вы считаете числа? Один, два, три… Вам о чём-нибудь это говорит?

-(Говорун) Вы считаете нас за дикарей?

*(Ольга) Да, нет.
*(Белимов) Мы хотим понять, с кем имеем дело. 
*(Гера) Какое у вас развитие там вашей культуры, цивилизации?

-(Говорун) Я, всего лишь, в вашем понятии, говорун.

*(Гера) Но, вы-то живёте в обществе, вы знаете, чем оно...(дышит. Прим.)

-(Говорун) И я тоже не знаю, с кем говорю.

*(Гера) Ага.
*(Ольга)  Мы - ваше будущее.

-(Говорун) И я только учусь видеть вас. И  потому, не могу сказать сколько вас. Вы назвали два имени.

*(Гера) Нет, тут ещё два. [Намек, молчать]Ааа..
*(Ольга) Два имени, да.
*(Белимов) Можно ещё назвать.
*(Ольга) Не надо!

-(Говорун) А вас… ещё…(помимо двух человек.Прим.)

*(Ольга) Скажите, вы нас слышите?

-(Говорун) ...трое?

*(Ольга) Вы нас… Да! Вы нас слышите?

-(Говорун) Я же говорун, значит, и слышу.

*(Ольга) А, и учитесь видеть, да?

-(Говорун) Только учусь.

*(Ольга) Скажите, а имя “Мабу” вам о чём-нибудь говорит? Есть какое-нибудь воспоминание?

-(Говорун) Мабуу?

*(Ольга) Да.

-(Говорун) Нет, у нас нет таких грубых имен.

*(Ольга) Грубо,.. У вас имена более мягкие, да? Значит, по отношению к нам, вы живёте… Если это прошлое… Если прошлое, по отношению к вам — если это прошлое, то вы живёте, как мы понимаем, в эпоху Атлантиды. То есть, у нас - мы так называем эту эпоху.
*(Гера) Скажите, а у вас есть самодвижущиеся телеги там, есть, да?

-(Говорун) Да, да.  Только когда вы спрашиваете, пожалуйста, представляйте.

*(Гера) Ага, вот вы как. Значит, у вас телепатические, так сказать, это, развиваются. А много таких, как вы? Вот, с такими задатками у вас?(телепатическими.Прим.)

-(Говорун) Говорунов?

*(Гера) Ну, Ведунов, Говорунов, Знахарей.

-(Говорун) Ведун? Вы знаете это?

*(Гера) Да. А что такое?

-(Говорун) Что они делают?

*(Гера) Они ведают!

-(Говорун) Именно?

*(Ольга) Знают будущее!
*(Гера) Что вы говорите?

-(Говорун) Знают будущее???

*(Гера) Да.

-(Говорун) Нет, тогда у нас другое понятие Ведунов.

*(Гера) А у вас - как? Он знает, куда вести, там, людей?

-(Говорун) Нет. Это люди, которые не имеют права быть среди нас.

*(Гера) А чего же они такое заслужили?
*(Белимов) Они талантливы же.

-(Говорун) Может это, просто, совпадение слов?

*(Гера) Ну, да. Может, мы называем разные вещи одним и тем же именем.
*(Ольга) Скажите, их изгоняют, или они просто сами отдельно живут?

-(Говорун) Нет, они служат нам.

*(Ольга) И они живут отдельно, да?

-(Говорун) Да.

*(Гера) На окраине города? 

-(Говорун) Почему же? Тогда им было бы неудобно служить.

*(Гера) Ну, приходили бы и служили.
*(Ольга) Там где-то храм, да? В храмах? В храме?

-(Говорун) Храм?

[шепчутся между собой]

*(Гера) У вас вообще там, как на счёт поклонения какому-нибудь верховному божеству? Есть там такое? Бог есть, вообще, в вашем понятии? 

-(Говорун) Да.

*(Гера) А как его зовут?

-(Говорун) Атьсола

*(Гера) Ацеола?

-(Говорун) Атьсола. У вас очень грубый язык. У вас грубые звуки.

*(Ольга) У вас много больше гласных, чем согласных.
*(Белимов) Таких названий… таких....
*(Девушка) Скажите, вы что-нибудь… понятия имеете о звёздах?

-(Говорун) Конечно! Я должен иметь эти понятия. Иначе, я не был бы Говоруном.

*(Ольга) А почему же вы о Солнце тогда так...Или… А!... Это вы - о цвете. А что вы знаете о звёздах? Можете нам рассказать?

-(Говорун) Называйте любую звезду.

*(Ольга) Ну, например, Сириус.

-(Говорун) Сириус?

*(Ольга) Да. Мы так зовём. Может у вас по-другому?

-(Говорун) Тогда, представьте! 

*(Ольга) Её?

-(Говорун) Да.

*(Гера) Вот, я сейчас представляю её, Вы видите?
*(Белимов) Солнце ещё можно представить, а Сириус? Вряд ли.

-(Говорун) Сатурн?

*(Гера) Сатурн.
*(Белимов) Есть, да, планета Сатурн. Она с кольцами, её представить можно. Это планета.
*(Гера) Венеру можно представить, Марс можно.
*(Белимов) Неужели вам эти не известны? Названия этих звёзд.

-(Говорун) Цсиор.

*(Ольга) Так называете вы, да?

-(Говорун) Цсиор? Если мы правильно поняли, то это Сатурн.

*(Ольга) А Венера? 
*(Гера) Вторая планета.
*(Ольга)Ещё её называют “Утренняя звезда”.
*(Белимов) Она самая яркая на плане.. на небосводе.
*(Ольга)Утром и вечером, она самая яркая. Знаете такую звезду?

-(Говорун) Ваал. 

*(Гера) Ваал?
*(Ольга) Это у нас такой есть Бог — Ваал. Мифический.

-(Говорун) Бог?

*(Ольга) Да.

-(Говорун) Это богиня.

*(Ольга) Бог.

-(Говорун) Нет. 

*(Ольга) А у вас - богиня.

-(Говорун) У нас богиня.

*(Ольга)М-мм. А у нас её зовут, просто Венера, богиня Венера. Или Изида.
*(Гера) Богиня любви.

-(Говорун) И она, для нас, значит многое.

*(Гера) Почему?

-(Говорун) Пример?

*(Ольга) Да.

-(Говорун) Если первый, кто увидел её, значит,  день будет счастливый.

*(Ольга) Скажите, а вот у вас одежда какая? Вы можете…  В чём вы ходите? (кому что, а женщины о нарядах; прим.)

-(Говорун) Но мы не знаем ваших имен.

*(Гера) Ну, хорошо.
*(Ольга) Нет, мы имеем ввиду…
*(Гера) У вас жарко, холодно, как? Снег идёт или Солнце постоянно греет? Растет трава, там, растительность какая-нибудь есть?

-(Говорун) Что значит Эдемов Сад?

*(Гера) Эдемов?
*(Ольга) Эдемов Сад?
*(Гера) Райский сад.

-(Говорун) Кто из Вас это подумал?

*(Гера) Я не думал.
*(Ольга) Кто?
*(Белимов) Наверное, у меня мелькнуло что-то.
*(Ольга) А вы знаете, что это такое?

-(Говорун) Но вы так представили… И мы согласились.

*(Гера) А-аа.
*(Ольга) Да? Это хорошо. Скажите, а какие у вас ещё боги? Вот… По именам назовите, пожалуйста.

-(Говорун) Каждая звезда имеет своего правителя, и каждый город имеет защиту.

*(Гера) От чего?
*(Ольга) Угу.

-(Говорун) Ваш город должен иметь свою звезду, а значит, и своего Бога.

*(Ольга) Понятно. А скажите, у вас понятие об Астрологии есть? Звездочеты есть? Мы так называем их.

-(Говорун) Говоруны.

*(Ольга) Говоруны, ага.
*(Гера) А у вас говоруны техническими вооружены, или они чисто своими возможностями видят звёзды? Внутренними, так сказать, взглядами.
*(Белимов) Приборы есть у вас?
*(Гера) Или техника? Технические приборы, телескопы?

-(Говорун) Очень много слов, и мы не поняли.

*(Гера) Хорошо… Вот, у вас глаза есть, чем вы видите? В вашей жизни там. 

-(Говорун) Человек!

*(Гера) Человек, да? Значит, полностью есть тело человеческое. Ну, а если у кого плохое зрение, кто плохо видит, -  есть такие? Старики там, допустим, или больные. Есть?

-(Говорун) Да.

*(Гера) А на глаза им что-нибудь одеваете, чтобы они лучше видели?

-(Говорун) Нет.

*(Гера) Нет?

-(Говорун) Зачем?

*(Гера) Ну, чтобы человек не страдал.
*(Белимов) Улучшить зрение.


-(Говорун) За него видят другие. Это так сложно?

*(Гера) Нет, ну, не всегда же другие находятся рядом.

-(Говорун) Разве? Вы можете оставить его одного?

*(Белимов) Надо спрашивать о детях, о семье…
-(Ольга) Вот, скажите, пожалуйста, а как вам вот, правительство? У вас есть правители? Как вы его называете? Не имя, а просто, как вы его называете? Правители, и всё, да? Или Царь… Царь, допустим, король, может какой-нибудь другой?
*(Гера) Главный над вами кто-нибудь есть, которого вы все слушаете?  

-(Говорун) Альио вельс мико. Это Вы говорите - верховный правитель.

*(Ольга) Угу.
*(Гера) А, правительство, всё-таки, сборное, так сказать. Да?
*(Ольга) Верховный. У вас нет одного, да? У Вас.
*(Гера) Совет!
*(Ольга)  Совет.

-(Говорун) Избираются. 

*(Ольга) В совете избирается правитель, да?

-(Говорун) Семь по семь.

*(Ольга) Семь по семь.
*(Гера) Семь лет по семь правителей, что ли?
*(Ольга) Скажите, а вот, как у Вас семьи устроены? Вот, семья есть? Муж и жена.

-(Говорун) Да.

*(Ольга) А сколько бывает? Много жён или нет? Или у вас это не принято?

-(Говорун) Много жён?

*(Ольга) Да.

-(Говорун) У вас принято это?

*(Гера) Ну, в некоторых странах принято, а некоторых - нет.

-(Говорун) Это вы называете “падением нравов”?

*(Гера) Нет, просто у каждой страны свои, так сказать а.. 
*(Ольга) Обычаи.
*(Гера) Обычай.

-(Говорун) Обычай?

*(Гера) Ну, устоявшаяся…

-(Говорун) Обычай выше закона?

*(Гера) Ну, свои законы. На основе обычаев делаются законы. Потому что...

-(Говорун) Даже Боги подчиняются законам.

*(Гера) Хорошо, какие у вас законы?

-(Говорун) А Вы - свои? Семь по семь!

*(Ольга) Скажите, вот, а говоруны имеют жён?

-(Говорун) Одни.

*(Ольга) А, правители имеют жён?

-(Говорун) Имеют? Если бы они имели жён, как бы они тогда могли править?(не чета нашим:-)

*(Белимов) А детей у вас не бывает что ли?

-(Говорун) Почему?

*(Белимов) Ну, и по сколько у вас в семье детей? В среднем.

-(Говорун) Я не имею!

*(Гера) Ну, а кто имеет. Вы же знаете людей.
*(Белимов) Ну, а у других? Двое, трое детишек?

-(Говорун) Да.

*(Белимов) Так?
*(Ольга) Скажите, а чем, вот, занимаетесь? Вот, какие у вас, допустим..
*(Белимов) Основные занятия. 
*(Ольга) Да.
*(Белимов) Вот вы, вот — Говорун, Астролог.
*(Ольга) Да, а другие?
-(Белимов) А другие?

-(Говорун) Вы говорите о нас или о ведунах?

*(Ольга) Нет, мы говорим о людях.
*(Гера) Ведуны – люди, надеюсь? Не собаки, не свиньи, ни что-то там...

-(Говорун) Люди.

*(Гера) Угу.
*(Ольга) Вот, ну, простые люди, вот, которые живут в городе.  Или у вас только говоруны и только ведуны?

-(Говорун) О, нет.

*(Ольга) Нет.

-(Говорун) Поняли.

*(Ольга) Поняли, ага. Простые люди.

-(Говорун) Обычные. Обычная семья. Обычно все имеют семьи. И лишь только я не имею, Агоба не имеет, Маула не имеет и не будет иметь. Лишь только в том случае, если они отдадут и откажутся говорить.

*(Ольга) Скажите, а вы сейчас один, или с вами кто-то есть?

-(Говорун) Двое.

*(Гера) Кто ещё?

-(Говорун) Ауба.

*(Ольга) И?

-(Говорун) Я.

*(Ольга) Все, вас двое, да? А Ауба,  это кто?
*(Гера) Женщина?

-(Говорун) Женщина.

*(Ольга) А скажите, а вы живёте отдельно? В храме где-то?

-(Говорун) Мы служим!

*(Ольга) Вы служите. Ну, а отдельно живёте?

-(Говорун) Нет.

*(Ольга) Ну, в смысле, так прям… Так, как все? В городе, да? Так же живёте?

-(Говорун) Да.

*(Гера) У вас свой дом?

-(Говорун) В Храме нельзя жить! В него можно только приходить и служить! И приходит время, и ты должен покинуть Храм.

*(Ольга) Скажите, а вот то, что вы, вот, с кем-то разговариваете - потом куда это? Кому вы рассказываете или записываете? У вас письменность есть?

-(Говорун) Здесь есть ещё два Ведуна,  и они пишут.

*(Ольга) А Ведуны старше вас или нет?
*(Гера) Кто кому подчиняется?

-(Говорун) Вы издеваетесь?

*(Гера) Нет, ну, спрашиваем. У нас просто такая система - все кому-то подчиняются.

-(Говорун) Они служат Нам.

*(Ольга) Ах, вот как!
*(Белимов) Они записывают на бумаге наш разговор?

-(Говорун) Что - “бумага”?

*(Гера) А на чём записывают, и как они..
*(Ольга) Как вы называете, на чём записывают?

-(Говорун) Агоб.

*(Ольга) А у нас “бумага” называется. А как вот это, из чего вы делаете это?

-(Говорун) Агоб?

*(Ольга) Да.

-(Говорун) То - листья…

*(Белимов)Угу.
*(Ольга) Дерева да?

-(Говорун) Наверное. Меня это не интересовало.

*(Ольга) А скажите…
*(Гера) Сколько Вам лет?

-(Говорун) Мне?

*(Гера) Да.

-(Говорун) В Вашем?(летоисчислении.Прим.)

*(Гера) Нет, ваших лет. У вас циклы у природы есть?

-(Говорун) Мне 186 …в вашем.

*(Ольга) В нашем понятии. А в вашем?

-(Говорун) В нашем?

*(Ольга) Да. Чтобы мы сумели посчитать, как вот со временем, может быть.
*(Белимов) Они уже перевели.  186.
*(Ольга) Нет, а в вашем… В вашем времени?

-(Говорун) Две Азы и Пять лёко.

*(Ольга) Две Азы …это что означает?
*(Белимов) Две тысячи?

-(Говорун) Мы только что вам сказали.

*(Гера) Сколько у вас пальцев на руке? Скажите, по-вашему.

-(Говорун) Один Лё.

*(Гера) Один Лёк?

-(Говорун) Один Лё!

*(Ольга) Один Лок. Один Лё.
*(Гера) Значит…т. е. На двух руках у Вас один Лё пальцев?
*(Белимов) Ты про одну руку сейчас спросил.
*(Гера) Нет, я “на руках” спросил.

-(Говорун) Одна рука - один Лё, 

*(Ольга) Понятно! [говорит Гере].
*(Гера) Ага.
*(Девушка) Вот, скажите...

-(Говорун) Всего у меня пальцев  - Лё – гоно.

*(Белимов) 10.
*(Ольга) Всего, да?
*(Гера) Десять, значит. Понятно! Надеюсь, вы не трехпалые, так сказать.
*(Ольга) Скажите, вот, а вы свою внешность сможете описать?
*(Белимов) Здрасьте… Так! Тихо!

[*** Интонация переводчика изменилась ***]

-(Первые) Вы почему-то всегда не готовы?! И Вы теряетесь, не сумеете задать те вопросы, которые могли бы Вам помочь, хотя бы определиться, с кем вы разговариваете.

*(Ольга)[Кашлянула]
*(Белимов)[Кашлянул]
*(Гера) Да… как-то, с бухты-барахты…

-(Первые) Разве это было в первый раз?

*(Гера) Не в первый.  Но всё…всё равно…

-(Первые) Может, вы уже привыкли? Тогда, хотя бы готовьтесь. Готовьтесь заранее к этим ситуациям. Коррелируйте их. Представьте, вы попадаете в другую страну. Какие вопросы, помимо вопросов, вы должны задать, чтоб узнать об этой стране как можно больше? Вы не знаете этого? Тогда вспомните самый первый контакт. И когда у вас спросили о городах. У вас спросили : ”что это значит”. Вы помните?

*(Гера) Да, да. Помним.

-(Первые) И как быстро было всё освоено.

*(Белимов) И кто на нас сейчас выходил, вы можете подсказать? Какие эпохи? Будущее или прошлое?

-(Первые) Почему будем отвечать мы?

*(Ольга)  Да, мы сами должны, конечно!

-(Первые) Задавайте вопросы Им. Вы с ними говорили. Мы не хотим быть третьими. Вы любите, когда вмешивается третий?

*(Белимов) Да! [одобряюще]
*(Гера) Ну, если по делу вмешивается, - ради бога.

-(Первые) По делу?

*(Гера) Ну, в смысле, по теме. Там, помочь, или что-то или как-то где. А так, чтобы там…

-(Первые) А много ли вы воспринимаете помощи?

*(Гера) Ну, это наша конечно беда что, частенько воспринимаем всё “насмарку”, так сказать.

-(Первые) Вы очень удивительны. К вам приходит помощь, вы благодарите этого человека. А потом, проходит время, и за эту же помощь вы его корите, обвиняете – ” Если бы он не дал этого совета, то не было бы того-то и того-то”. Как часто у вас это бывает!?

*(Гера) Да, но…
*(Белимов) Ну, а сейчас можно задать вопрос, который, может быть, повлияет и на мои решения?  Я чувствую, мы разговариваем с нашими прежними друзьями. Сейчас что...

-(Первые) Мы никогда не назывались вам друзьями. Старайтесь держать дистанцию. Не из-за того, что мы горды или что-то... Вы должны держать дистанцию. Ибо, сколько раз мы вас уже обвиняли во вседозволенности?

*(Гера) Было такое, да.

-(Первые) Что делаете вы? Прежде чем говорить с нами, вы готовитесь? Нет! Вы привыкли, как вы говорите  – русский “авось”, так и действуете. И тогда, у вас получается просто “авось”.

*(Белимов) Ну, сегодня я готовил вопросы, но вышла совсем не та ситуация. И с ними было бы бесполезно разговаривать по нашим вопросам. Давайте сейчас вернёмся к тому, что мы готовили. Перед сеансом действительно было длительное размышление  про что разговаривать. Вы готовы поддержать такое?

-(Первые) Эволюционист.

*(Белимов) Не готовы?
*(Ольга) Вы простите, можно вас вот спросить? Вы, как-то, на одном из контактов совсем недавно сказали, что вы устали от нас. Может, действительно, мы такие настойчивые, что ли? Может, нам…

-(Первые) Нет, вы не настойчивые. Вы упрямые.

*(Девушка) Даже упрямые… Да. Может, действительно, вам всё-таки… скажите нам правду, - отдохнуть от нас?

-(Первые) Когда есть открытая дверь, а вы ломитесь в стену рядом. Вот она, правда. Вместо того чтобы сделать шаг в сторону и войти в эту дверь. Иногда, к сожалению, случайно, вы становитесь на русло,  на ту дорогу, которая ведёт именно к этой двери. Но подойдя к ней, вы стараетесь открыть её в другую сторону, и, тем больше клините её, а потом возмущаетесь.

*(Белимов) Ну, нельзя ли более понятно объяснить, где эта дверь и как в неё входить?

-(Первые) Нельзя... Нельзя. Потому, что вы должны пойти этой дорогой  и без поводыря. Вы должны прозреть, увидеть эту дорогу узнать её, чтобы однажды вернувшись на неё опять, не пойти по ней. А если вас просто проведут? Всмотритесь! Вы идёте с собеседником по дороге, разговариваете и не замечаете, как уже прошли эту дорогу. И когда у вас спросят, что было на этой дороге - вы не вспомните! Ибо вы были заняты беседой. Вы сможете потом повторить этот путь? Ну, представьте: вы шли по этой дороге и что-то обронили.  Пусть будет - ключ. И, когда вы попытаетесь его найти, вы будете упорно вспоминать, а этой ли дорогой вы шли? Вроде бы эта, а вроде бы и нет. Вы не обращали внимания. Разве мы не правы?

*(Белимов) Ну, такие мы - люди.

-(Первые) И тогда, этот поводырь  приведёт вас, конечно, безусловно, приведёт. Но вы будете рабами. Те самые Ведуны, о которых вы только что слышали.

*(Ольга) А! Ведуны - это от слова “вести”, да?
*(Белимов) Поводырь.
*(Ольга) Ведуны, это “ведать”. То есть, от слова “знать”.

-(Первые) О, нет. Первое значение его было – “Человек, которого ведут”. Его ведут, он  просто идёт. Он просто идёт. Он не знает дороги. Ему говорят: Иди сюда, - он идёт сюда. Его останови -  он будет стоять. И отсюда, - вот вам и раб. А вы, чаще всего, действуете, как рабы. Вас ведут! А мы не хотим. Мы не хотим, чтобы вы были нашими рабами. Мы не хотим, чтобы вы шли той дорогой, которую показываем мы.

*(Ольга) Скажите, пожалуйста, вот однажды вы сказали, что мы были вашей дорогой  когда-то. Вот, мы, конечно,  может быть…
*(Гера) Вы проходили человеческую стадию?
*(Ольга) Нет.
*(Гера) Плотную, скажем так.
*(Ольга) Как вот можно, вот… чуть-чуть, если по подробнее. Хоть намёк какой-то, как наши дороги.

-(Первые) Ну, давайте тогда поговорим так. Что в вашем понятии “душа”?

*(Гера) Ну, это опыт, наверное, который заложился  в течение всех жизней.

-(Первые) Хорошо! Тогда представьте, - вы сейчас человек. В следующей жизни вы превращаетесь в собаку. А зачем тогда было весь этот опыт, как вы говорите -“душа”- нужен собаке? Нужен ли он ей сейчас? Умение держать паяльник, умение читать газету, умение смотреть телевизор и рассуждать о политике? Это нужно собаке? Это что, исчезло бесследно? 

*(Ольга) Да это просто может быть наносное, вот это умение, вот, что-то делать…

-(Первые) Сколько раз мы говорили вам, и придется опять повторить, что ваше сознание, это всего лишь, и не больше....
[ диалог прерывается на полуслове длинная пауза]

*(Гера) Да, мы слушаем.

[*** Интонация переводчика изменилась ***]

-(Говорун) Дай ци тор. Вы слышите?

*(Ольга) Да. Мы слушаем. Ну, мы немножко так… Вернее, нам дали понять немножко, вот э-мм.., как с вами беседовать. Вы, всё-таки, расскажите о том, как вы живёте? Что у вас вообще в жизни главное? Вот у вас, у Говорунов, допустим. 

-(Говорун) Умение правильно услышать. Чтобы не обмануть, чтобы не сделать ошибок. И ещё, я должен следить за здоровьем.

*(Ольга) Кого?
*(Белимов) Себя.
*(Гера) Своим здоровьем или за здоровьем других?

-(Говорун) Здоровьем других.

*(Ольга) А! Это у нас ещё называется - целители, да? Врач.
*(Белимов) Врачеватель, целитель.
*(Ольга) Доктор, ещё у нас называется или целитель.
*(Гера) Не кричите [обращаясь к спрашивающим]

-(Говорун) Доктоо.

*(Ольга) Доктор.

-(Говорун) Доктоо.

*(Ольга) Да, да, yes. Скажите, а вот простые люди, вот семьи, да, есть. Они, дети, ходят в школу куда-то? Чем они занимаются? Дети.
*(Гера) Обучают их чему-нибудь?
*(Ольга) Чем они занимаются? Их только родители воспитывают или ещё кто-то?

-(Говорун) Храм.

*(Ольга) В Храме, да?
*(Гера) Это… все дети, да?

-(Говорун) Это школа.

*(Ольга) Школа - храм[говорит Гере].
*(Гера) Угу.
*(Ольга) А вот сейчас рядом с вами стоит Аула, женщина. У вас мужчина и женщина как-то, вот, ну...
*(Гера) Могут…
*(Ольга) Подчиняются, допустим, женщина мужчине?

-(Говорун) Нет.

*(Ольга)Нет?
*(Гера) Равноправие, да?

-(Говорун) Она, прежде всего, человек!

*(Ольга)Да, да, правильно.
*(Гера) Да.

-(Говорун) И почему она должна быть больше или меньше? Единственное, что она не может, - это стоять и быть правителем.

*(Ольга) Угу. А почему?
*(Гера) Ну, как? А почему? Да, кстати.

-(Говорун) У неё другой удел.

*(Ольга) Правильно. Ну, в детях же[говорит Гере]. А вот у нас, просто...

-(Говорун) Дело не в детях!

*(Ольга) А в чём?

-(Говорун) Её предназначение - быть Женщиной! Это - первое.

*(Гера) А для чего, вот, женщина нужна?

-(Говорун) Для чего нужна женщина?[удивленно]

*(Гера) Рожать детей.[утвердительно]

-(Говорун) Среди вас есть женщина. Пусть ответит сама.

*(Гера) Да.
*(Ольга) Так.
*(Гера) Вот, почему? Для чего нужна женщина? [спрашивает у Ольги]
*(Ольга) Ну, мы потом поговорим.

-(Говорун) А для чего нужен мужчина?

*(Ольга) Всё правильно.

-(Говорун) Вы знаете это?

*(Гера) Ну, как вы считаете?
*(Белимов) Сам для себя.
*(Ольга) Почему у нас есть вот такое, - разделение пола. Мужской и женский? Вы об этом знаете?

-(Говорун) Об этом мы учим детей! [удивленно] Они должны это знать. И не просто знать, но и жить по этому! Это вы говорите, что у вас в каждом го... городе свои законы.

*(Ольга) Да нет, у нас не города, а страны. Вы знаете, что такое страны?
*(Гера) Это - много-много городов и сёл.

[*** Интонация переводчика изменилась ***]
Обратите внимание диалог продолжается с того места с которого был прерван!!

-(Первые) Адаптация, всего лишь, и не больше! А Вы говорите – сознание. Сознание – это  адаптирование к окружающей среде, на данный момент. И поэтому, легко объясняется, почему у собаки одно, а у человека другое сознание. А что такое Душа? А мы говорили вам о нематериальности её.

*(Гера) Да.
*(Ольга) Да.

-(Первые) Помните?

*(Ольга) Да.
*(Гера) Это, наверное, вот это, вот.,

-(Первые) И в тоже время, мы говорим, что она воздействует.

*(Ольга) На всё. Ну, да.
*(Гера) Может быть...

-(Первые) И в тоже время, она не имеет никакой памяти. А зачем ей нужна память?

*(Гера) А зачем ей нужно воздействие?

-(Первые) Память нужна только материи. И всё. И не больше. Представьте, Душа создала ещё одну тройную оболочку. Со стороны собаки, какой была Душа такой и осталась. Ей память не нужна. Лишь только на эту материю воздействует окружающая среда, чтобы получилось Сознание, и, с другой стороны, Душа, которая не воздействует в вашем физическом смысле. Это, всего лишь, если хотите, алгоритм. Работа. И поэтому, это есть Личность! Именно та самая, истинная личность, а не личность собаки.

*(Ольга) Скажите, вот, вообще-то, люди говорят, что личность, это совокупность, вот, качеств именно - не теряется. Т.е. это физическое, астральное и ментальное, т. е. составляющие человека как бы. Ну, то есть… Но…

-(Первые) К сожалению, да. 

*Так, да?

-(Первые) К сожалению, так. Вы, ваша личность, ваш характер, конечно же, зависят от строения тела, а не на оборот. Лишь только потом, став взрослым, вы уже(...) на то и другое. Конечно же, если вы сейчас поменяете фигуру, то соответственно у вас поменяется характер. Даже если на вас одеть одежду дикаря, вы уже будете себя чувствовать дикарём.

*(Гера) Да. 

-(Первые) Если же… И поверьте, вы будете чувствовать больше вседозволенности! Вы же - дикарь! Мы не имеем в виду, что вы превратитесь в дикаря.

*(Ольга) Ну, да. 

-(Первые) Именно, просто, одежду дикаря одеть вам. Ну, вам захочется помахать каменным топором, поиграть. Если же одеть на вас строгий смокинг, то, соответственно, вы будете вести себя по-другому. Даже просто купив простейшую блузку, вы уже придаёте ей значение. У вас уже меняется и походка, умение держать фигуру. Уже меняется. Всмотритесь в себя внимательно!

*(Ольга) Да, да. Мы заметили. 

-(Первые) Спрашивайте.

*(Ольга) Вот, скажите, вот, всё-таки, в чём-то есть вот смысл, что ли? Допустим, ну, внешность человека определяет и, ну, будем говорить так, уровень его развития. Ну, если грубо так сказано.

-(Первые) Болезнь Дауна.

*(Ольга) А?

-(Первые) Болезнь Дауна.

*(Ольга) Ну…

-(Первые) Это, уже ответ?

*(Гера) М-мм.Да.
*(Ольга) Да, да. Скажите, а вот у таких людей, вообще-то, с душой как? Вот, болезнь Дауна, и другие есть такие психические болезни, можно их зомби назвать?

-(Первые) Нет! Ни в коем случае!

-(Ольга) Нет, да? Они, просто, очень…

-(Первые) Это, всего лишь только, мало сознания. Просто не адаптировались, и всего лишь. Отличий может быть множество. Физические, химические, любые. Назовите, как вы говорите“ рыхлое тело”. Твердость тела — твердость характера. И причём, вы не строго, а....

СЧЕТ 1, 2, 3, 4, 5

-(Первые) ...в сознании, которое мы сейчас имеем, получить машину времени. Сколько всего вы переиначили. Просто из-за того, что не вовремя пришёл домой, давай время повернем в спять? (...) А то, что будет разрушено пол-мира, это будет не заметно, потому что это будет разрушено до изменения. А сейчас, в этом понятии, я всегда жил в разрушенном мире. Представляете машину времени? Что это такое? Это изменение всего и себя. И при этом, даже не будете знать, что изменилось. Можно превратиться в лягушку и будешь считать, что всю жизнь жил лягушкой. И обязательно найдётся прошлое, когда ты вспомнишь, что ты вылупился из личинки.

*(Белимов) В связи с вашими словами, можно задать следующий вопрос? Возможно ли перемещение людей во времени? Из прошлого в настоящее, из настоящего в прошлое?

-(Первые) Можно! Конечно. В своё – можно.

*(Гера) А кто-нибудь делал это?

-(Первые) Но, поймите! Уйдя в прошлое, нельзя ничего изменить. Нельзя. Потому что будущее уже предусмотрело этот вариант. И если вы считаете, что придя из будущего в прошлое, вы измените так-то и так-то, и тут же считаете, что, значит, в будущем будет теперь то-то и то-то. Потому что это уже это было предусмотрено. Потому, что это было, в конце концов, прошлое. Вы согласны, что прошлое создаёт будущее? Если вы из этого будущего ввалились в это прошлое и изменили… но это же было в прошлом! А значит и, соответственно, будущее осталось неизменным. Прекрасно. Но! Естьи другое, есть и другое. Сторонний наблюдатель не увидит разрушений! В замкнутой системе — да! В каком будущем вы были в такое же и вернулись. Раздавили вы бабочку или нет, вы все равно не превратитесь в динозавра. А извне, для другого наблюдателя, это будет уже похоже на мыльные пузыри. Представляете ваш мир, вашу Вселенную? Всего лишь только мыльный пузырь, который может лопнуть в любое мгновение, только из-за того, что кто-то изнутри хотел попробовать стенку и проколол её. Так вот, все Миры, все Миры, и те в которых вы живёте, и те которые вы даже не можете представить, ибо они совсем непохожи на вас, - это, всего лишь, множество пузырьков.

*(Гера) И между ними есть стенки!

-(Первые) Между ними есть стенки и переходы. Но, эти переходы, как правило, чаще, закупорены. По той простой причине, что вы не можете сообщаться. По той причине, что эти миры могут быть Антимирами, как вы говорите. Или совершенно другими, не подходящие ни под этот Мир, ни под Антимир. Совершенно всё иное и другое. Но обязательно существует какая-то тонкая плёнка, та плёнка, на которой держатся, как вы говорите, эти пузыри.

*(Гера) Так. И вы живёте, как раз, на той тонкой пленке?

-(Первые) Никто не может жить на тонкой плёнке. Это, всего лишь только переход.

*(Белимов) Ребята, дайте я задам. Как раз становится...
*(Гера) Скажите, а… Минуточку.[Белимову]. А значит, переход в другой Мир, всё-таки, возможен! Можно ли перейти, не разрушая, допустим, это тело?

-(Первые) Можно! Конечно можно! Но, для этого вы должны на время перехода, как вы говорите “этой пленки”,  должны измениться в тот Мир. Если вы здесь были человек и переходите в другой, пусть будет Антимир, вы должны уже стать подобны тому Миру, в который пришли. А представьте, что делаете вы! Вы запускаете спутник. Хорошо, вы прилетели на Луну, поглядели, посмотрели, полетели дальше. И чем дальше вы летите, тем будет больше ошибок. Потому, что вы остались такими как были! Вы пришли и видите безжизненную планету. И вы видите, что эти звезды обладают источником планеты. А мы говорили вам, что ваша звезда -  для кого-то, просто планета, а для кого-то – это, просто, мёртвое. Всё зависит от точки отсчёта, как вы говорите.

*(Гера) А скажите, вот как, допустим, изменять? Сознанием что ли надо измениться или чем, или как? Понять, что есть другое, что-то там?

-(Первые) О! Когда вы сумеете это сделать, вы уже не будете задавать этот вопрос.

*(Гера) Ну, это безусловно. Но, пока-то не умеем , поэтому, задаём.

-(Первые) А это глупо! 

*(Гера) Ну… Пусть будет глупо.

-(Первые) А это - глупо! Это все равно, что мы посадим вас сейчас в самолёт и полетим. Но мы вам объясним, конечно же, объясним, -  дадим вам множество инструкций и множество бумаг, книг, и вы будете сидеть за штурвалом и читать, зевать, и будете долго искать эту страницу, где же она находится? Даже если вы научитесь и выучите наизусть все эти строки вряд ли вы взлетите хорошо! Вы согласны?

*(Ольга) Да. Ну, конечно.

-(Первые) Потому что нельзя полностью всё учесть,  Всего лишь ньюансы, А вся ваша жизнь только ньюансы. И смотрите - если два близнеца, будут выращиваться, а это можно только сказать так - выращиваться - в одинаковых абсолютно условиях, они останутся разными. Представление будет разным. А почему?

*(Гера) Ничего повторяемого в мире нет, - дважды в одну реку не войдёшь.

-(Первые) О нет! Всё повторимо. Всё!

*(Гера) Па.. В общем – повторимо. То есть, в каких-то проявлениях....

-(Первые) Нет. Всё повторимо. Всё.

*(Ольга) До тонкостей?

-(Первые) До тонкостей, это одно из явлений вечности. А представьте, представьте, вы берете стакан воды. И в нём содержится столько-то, столько-то молекул.

*(Гера) Угу.

-(Первые) И Вы уже можете сосчитать, сколько будет вариантов. Согласны? 

*(Гера) Да.

-(Первые) Но, это же будет конечное число?

*(Гера) Да. Безусловно.

-(Первые) Конечно. И через этот, скажем, период, снова будет тот же самый стакан воды. Но простите, пройдет столь множество времени, перебора этих всех вариантов, что для вас это уже будет совершенно Новое, и вы уже не будете думать, что это когда-то было. Вот вам и понятие Времен. Когда-то давным-давно, уже нельзя даже говорить о числах, была та же Земля, были те же вы, и, когда-то, будет опять. И до тех пор, пока вы, наконец, не поймёте, что бегаете по кольцу и привязаны колышком. А когда вы это поймете… вот тогда можно говорить о разбегании Галактик в истинном смысле, а не просто. И вот тогда,  вы будете освобождены. А так - бег по кругу.

*(Белимов) Скажите, у наших друзей произошла трагедия, погиб талантливый очень, способный мальчик...

-(Первые) И вы предлагаете - давайте вернёмся?

*(Белимов) Они предлагают.

-(Первые) Давайте вернемся, и восстановим, сделаем так чтобы этот человек не погиб. Вы так хотите? А когда-то мы вам говорили, что нас не интересует даже ваше здоровье. Вы помните это?

*(Белимов) Ну, мы думаем, что вы лукавите. Мы...

-(Первые) Нет, мы не лукавим.

*(Белимов) Мы так к вам не относимся. Если у вас будет расстроено здоровье,  мы постараемся вам чем-то помочь.

-(Первые) И чем вы нам поможете? А как вы помогаете друг другу? Ну, хорошо… Как вы можете помочь ближнему человеку, если у него расстроено здоровье? Химией?

*(Белимов) И химией и экстрасенсорикой. Мы не остаёмся равнодушными.
*(Гера) По крайней мере, воздействовать на сознание можно.

-(Первые) И что?

*(Белимов) Мы стараемся помочь, у нас есть медицина и просто свои методы.

-(Первые) И, чаще, получается – вы, просто ухудшаете ситуацию. Чаще - так.

*(Белимов) Но благие намерения были первичны, значит, мы хотим, как лучше. 

-(Первые) А благими намерениями путь в ад вымощен, 

*(Белимов) Ну, понятно. Но, так вот скажите..
*(Гера) Стоп. Минуточку. Вы сказали:“благими намерениями путь в Ад выслан”. Хорошо. Вы так же сказали, в начале… Я, вот, давно хотел у вас так… спросить, - что, значит, не деяние, а мысль… От мысли зависит качество поступков, так сказать.

-(Первые) А вы умеете управлять своей мыслью?

*(Гера) Нет. Ну, если я, допустим, человеку говорю…

-(Первые) Если, если у вас заболел ближний человек, в первую очередь вам это лично мешает, что он болеет. А лишь только потом вы переживаете. А начало было мысли, что Он заболел и мешает Вашим делам. У вас новые проблемы. Вот ваши первые мысли, а вы их, потом, прячете… благородными. Не так?

*(Белимов) Бывает, но не всегда.

-(Первые) Не всегда?

*(Ольга) Да, да, да…

-(Первые) Да чаще, - чаще, вы так и делаете. Мы говорили вам, ищите начало мысли. А вы смотрите, вы… сколько раз прослышали от нас о лжи, и вы опять продолжаете лгать. Вы это делаете нарочно, потому, что у вас вошло в привычку, это инстинкт. Это одно из основ сознания, чтобы выжить в этом Мире. Представляете? Вы выживаете только из-за того, что лжёте друг другу. Вы никогда не скажите прямо человеку то, что хотите сказать ему. А почему? Множество причин. И при этом, как вы говорите “к сожалению”, что этот человек заболел, в первую очередь, к сожалению, это мешает Вам. Вы это считаете, как помеха, лишь только потом вы уже вспоминаете, что это всё- таки,  близкий человек  и надо было помочь. А посторонний человек заболел, - так он даже на нервы действует. Не так?

*(Белимов) Ну, у Вас совсем не так, видимо? Да?

-(Первые) Вы все такие. Вы все, к сожалению.

*(Белимов) Не.., а у вас по-другому? Вы нашли оптимальные пути да?

-(Первые) Мы? Конечно же, нет. 

*(Ольга) Ну, вот, мне так кажется...

-(Первые) Мы же - не боги.

*(Гера) Ну, тогда и говорите “Мы все такие”.
*(Белимов) А от нас что-то много слишком...

-(Первые) Хорошо, мы скажем: Мы все такие. И этим вы подтвердили только что сказанное нами.

*(Гера) Нет, ну, если и вы такие же, как и мы, тогда можно и обобщить, по-моему.
*(Ольга) Вы уж простите его, он иногда, по-моему, эмоционален слишком.

-(Первые) Дело не в этом. Дело совершенно не в том. Эмоции  всего лишь только помогают ещё легче разговаривать с вами. Лишь бы они небыли бурными, неуправляемыми.

*(Гера) У меня не достаточно… В смысле – нормальные эмоции “буйности”, так сказать?  “Буйности”, так сказать, нормально?

-(Первые) А что Вы считаете “буйностью” мысли?

*(Гера) Ну, для вас - не усиленно?

-(Первые) Нет, скажите, что вы считаете “буйностью” мысли?

*(Гера) Буйностью мысли?

-(Первые) Да.

*(Гера) Но и буйность эмоций?

-(Первые) Мысли.

*(Ольга) Но, скорее всего, хаос мысли.
*(Гера) Меня спросили вообще-то.[Ольге]. Буйность мысли - это когда, наверное, что-то тревожит гораздо “буйнее”, скажем. Когда ты считаешь это главным, ну, одним из главных, тогда, эмоции, действительно, выпирают, и уже нельзя спокойным голосом что-либо такое сказать.

-(Первые) А первая мысль была точнее — Хаос. Помните“Девятый вал”?

*(Ольга) Да.

-(Первые) А теперь перерисуйте его для себя. (Айвазовский.- ”Девятый вал”. Прим.)

*(Ольга) Да. Можно представить.
*(Гера) А-а, вот что…  Это этого художника, да? Ясно.

-(Первые) И при этом, этот вал вам помогает, - одно из способов защиты сознания,- как Вы говорите “не сойти с ума”. И когда вы зациклились на одной мысли, и особенно когда вы потеряли близкого человека и вы не верите этому, вы создаёте множество фантазий, множество фантастических миров и, рано или поздно, не оставив это дурное занятие,  вы станете просто одержимым, и не больше. Вы должны, в первую очередь, отпустить этого человека. Что вы делаете? Что вы делаете? Вы, сперва, за человека переживаете, переживаете не о нём, а о том, что потеряли его. Ваш эгоизм. Не больше. А потом, вы множество раз говорили, “есть миры, иные, он не умер”. И тут же в это не верите, потому что вы не допускаете того. И вы держите при себе… Держите этого человека, а потом удивляетесь - откуда у нас здесь столько много ”шарабашек”?

*(Белимов) Угу. Надо отпустить?

-(Первые) А вы их мучаете! Рождаете своей какой-то жалостью, ведущей...

*(Ольга) Ещё называют это любовью, к сожалению.
*(Белимов) Да.

-(Первые) Любовь? Простите, что вы назвали “любовью”? Любовью - если только рядом. Вы можете представить, что вы говорите?

*(Ольга) Да, мы можем.

-(Первые) Два близких человека… Если это истинная любовь, то расстояние не имеет значения. А вы? “Я скучаю по любимой, я скучаю по любимой, приезжай быстрей ко мне”! Глупо! О какой тогда любви вы говорите? О физической? Почувствовать рядом, вот оно - мое! Нет разве?

*(Ольга) Так.
*(Белимов) А что, вот, сейчас посоветуете людям, которые потеряли сына, что лучше всего для них сейчас? Метод успокоения.

-(Первые) Отпустите его! Отпустите. Пусть идёт дальше! Зачем он должен здесь крутиться вокруг вас? Что вы держите его при себе? “Вычеркнуть из памяти”.  А вы можете его просто отпустить? Зачем, зачем вы рисуете картинки?- “Ах если бы, ах если бы… и тут же -  Ах! Если бы он не погиб, он бы сейчас закончил институт, он был бы сейчас на танцах там-то”. Вот вам - иллюзии! Иллюзии, ведущие вас в Ад, а его к пыткам! Как вы его пытаете? Как вы говорите - жалостью. Да поймите же вы! Поймите, в конце концов,  и поверьте. И живите, потому,  что есть множество Миров. И есть время, когда надо оставить, оставить то, что вы здесь нажили. Бросить. А вы, как говорите, - привычкой. А ещё было бы лучше, -  жадностью,-  не хотите покинуть этот мир. Вы верите, что есть иное, а боитесь смерти. Тут же вы, если решите прыгнуть и разбиться и будете при этом обманывать себя: - Ах, вот мне надоело! Ах, я пойду в иной мир и надо избавляться от физических тел. - Здорово? Отрезвляет память?

*(Белимов) Скажите. Вы так говорите, как будто знаете эту ситуацию очень хорошо. И, может быть, Вы сейчас общаетесь с этим умершим человеком?

-(Первые) Слишком хорошо!

*(Белимов) Вы общаетесь с ним? Он через вас, может, просит родителей отпустить его? Это не так?

-(Первые) Нет, он не просит… Он не просит. Почему? Да потому что вы не слышите!

*(Белимов) Но, мы можем передать…

-(Первые) Он не покидает. Он не покинет вас, и не может покинуть, пока вы будете держать. Он будет стучаться к вам в душу, но она будет закрыта вашей “благородностью”. И вы, не поняв, что он находится около вас и мучается, будете сочинять новые иллюзии. И эти иллюзии будут губить, губить и его и вас. И, в конце концов, если этот человек уже был...

[*** Интонация переводчика изменилась ***]

-(Переводчик) с 11 дали счёт?

*(Гера) Нет.
*(Белимов) А Вы хотите?
*(Ольга) Это - Он.
*(Гера) Да, это Он, я же говорил, не говорите числа! Мы если…
*(Ольга) Давай, поговорим?
*(Гера) ..от одного до девятнадцати, то потом - от девятнадцати до одиннадцати. С одиннадцати.
*(Ольга) Поговорим?
*(Белимов) Вот, кстати, я давно хочу поговорить с самим переводчиком.
*(Ольга) Ну, давай, слушаем.
-(Белимов) Скажите…
*(Гера) Да это Он! Что-нибудь скажите.(сознание. Личность переводчика. Прим.)
*(Ольга) Ты слышишь?
*(Белимов) Ну, понятно. С кем мы сейчас контактируем с сознанием или подсознанием переводчика?[переводчику]
*(Гера) Да это он! Сознание![Белимову]
*(Белимов) Ну, и помолчи.[Гере]
*(Гера) Нет, ну… Вы не разговариваете что ли…

-(Переводчик) По-моему, всё-таки, вместе, - и сознание и подсознание. Только подсознание уже тоже, вроде как… ну, не совсем на уровне сознания, но, уже.

*(Белимов) Но оно подключено? Ведь в обычном состоянии мы подсознание почти не слышим. Сейчас это объединились, ты сказал?

-(Переводчик) Ну, почти - да.

*(Белимов) Поэтому, сейчас из подсознания, наверное, можно извлекать многие знания, которые там заложены. Они включают опыт разных, других жизней и прочее?

-(Переводчик) Нет.

*(Ольга) Скажи, ты нас видишь?
*(Белимов) Нет? Как ты нас чувствуешь?

-(Переводчик) Нет. Только слышу. Слышу и пятна. 

*(Ольга) Чувствуешь. А скажи, у нас кто-нибудь есть тут новый? Ты можешь это вот?

-(Переводчик) Ну… я пятна вижу. 

*(Белимов) Сколько пятен?
*(Ольга) Общие, да?

-(Переводчик) Ну… как…  Я никак не вижу, чтобы именно. Общие.

*(Ольга) Общие. А какой цвет?

-(Переводчик) Серый.  В общем, такой, типа стали. 

*(Белимов) Ну, это хороший цвет в твоем понимании или не очень?

-(Переводчик) Нет, не очень.

*(Белимов) Не очень, да? А есть ли среди нас светлое пятно, допустим?

-(Переводчик) Для меня это - одно большое пятно, Но есть, но можно, наверное, вот так вот. Хотя, не знаю… ну в принципе, это, наверное, просто… Ну, общий фон серый. Типа стали, что ли… Красного много. Были какие-то бурные эмоции. Похоже, так.

*(Ольга) Да, да.

-(Переводчик) В основном, что… - Боль.

*(Ольга) Боль?

-(Переводчик) Да. У кого-то что-то болит.

*(Белимов) У переводчика болит зуб.
*(Гера) Физическая боль?

-(Переводчик) Да. Нет, у меня, по-моему, как раз, и не болит сейчас.

*(Ольга) У кого-то да? Это боль или?

-(Переводчик) Да, у кого-то боль. Сейчас, болит. Да нет, чисто такой… физическая по- моему.

*(Ольга) Физическая?
*(Белимов) А как ты это улавливаешь? Через телепатический? Или же какими-то… по цвету, или что-то такое?

-(Переводчик) Да, нет. По цвету, наверное.

*(Белимов) По цвету? Ну, как же, мы ожидали, что в таком состоянии при счёте на выход на тебя - на переводчика, мы многое узнаем о том, о подсознании. Что оно раскрывается.

-(Переводчик) Подсознание  - оно, в принципе, не знает больше сознания, оно помнит больше. Архив памяти, просто, больше и всё.

*(Белимов) Так вот, то есть мы…

-(Переводчик) А прошлую жизнь, как я могу помнить подсознанием? Если у меня сейчас совершенно другое сознание и подсознание.

*(Белимов) Так… А может быть это блеф, что мы живём много жизней? Может быть это выдумки? Нами мало чем подтверждаются. А сейчас ты говоришь, что подсознание не помнит прошлых жизней. Как ты думаешь?

-(Переводчик) Ну, у меня в подсознании есть сейчас вариант, но, опять же, - все варианты установлены, всё связано, именно с контактом. Получается… Ощущение, что это и есть мои прошлые жизни.

*(Гера) Слушай, а может…

-(Переводчик) Потому, что они уже, вроде как проиграны у меня в сознании. Соответственно, и в подсознании.

*(Гера) Слушай, а может это ложная реинкарнация, как ты говорил, “фентези”? Ты их просто так сказать...или тебе их внушили?

-(Переводчик) Может быть, запросто. Но, понимаешь смысл в чём? Сознание - это вот именно что у меня сейчас вот есть - какой-то кусочек памяти, которым я владею. 

*(Гера) Угу.

-(Переводчик) А памяти - довольно маленький сейчас объем. Это вот немного прошлого, как говориться, чуть-чуть помню, и вот сейчас вот. А подсознание - оно помнит все абсолютно. Такой массив, мощный массив. Получается,- сознание пользуется маленьким объёмом памяти, а  подсознание – всем. Всем, что было за твою всю жизнь, начиная от рождения и так дальше.

*(Гера) Как на компьютере, оперативная память и постоянно запоминающая. Да?

-(Переводчик) И, получается, - подсознание. Почему получается так, что, вот, неосознанно что-то, не естественно сознанию? Ну, потому, что подсознание помнит даже те мелочи, на которые вроде как не обращали внимания сознательно. Например. Ну, скажем, шел я по дороге, да. Вроде бы, я и не смотрел на эту дорогу, но я все равно помню каждый кустик, каждый листочек, каждое дуновение ветерка. Потому что, ну… Ну, физически-то, я всё это ощутил же!

*(Гера) Угу.

-(Переводчик) Подсознание всё запомнило. И, соответственно, оно потом на своём уровне там, что-то даст, какое-то воспоминание или что. Ну, какое-то действие. А  я-то сознанием  не помню той дороги же!

*(Гера) Ну, да.

-(Переводчик) И, по этому, получается вроде как –“ О, подсознание, какое могучее”! 

*(Белимов) Ну, тогда, может сейчас, в этом состоянии ты вспомнишь, ну, время своего рождения? Вечер был или день? Как ты почувствовал свет? Можешь вспомнить? Или даже до рождения? Зачатие. Или это нельзя? Ты же говоришь, что эту жизнь только помнишь. Ну-ка, вспомни.

-(Переводчик) Ну, не знаю.  Надо попробовать. 

*(Гера) Давай, попробуй.
*(Белимов) Ну, давай, что было в период зарождения вот, соединение клеток? Как ты ощущаешь это всё? Вспоминай. Вспышка, боль или что?

-(Переводчик) Туннель.

*(Белимов) Туннель? Надо же! Как и при смерти.
*(Ольга) Падаешь?
*(Белимов) Падаешь или что?

-(Переводчик) Да, падаю.

*(Ольга) И, потом, раз…и ничего не помнишь!?
*(Белимов) Скажи, пожалуйста, когда ты был, так.. 

-(Переводчик) Почему “не помню”?[удивленно]

*(Ольга) Помнишь? Ну...

-(Переводчик) Туннель. Очень-очень быстро двигаюсь.

*(Ольга) Угу.
*(Белимов) Так…

-(Переводчик) А потом, какое-то яркое что-то.

*(Белимов) Уже есть вспышка, да?

-(Переводчик) Не вспышка! Просто, какой-то яркий огонь.

*(Ольга) А! Где? Куда летишь?

-(Переводчик) Ну, внизу в туннеле. Да. Или, ну, просто в туннеле, в конце туннеля.

*(Белимов) Скажи, пожалуйста, когда…

-(Переводчик) Нет. Ни вспышек, ничего. Просто, я вроде как - уже.

*(Ольга) Уже пришёл.

-(Переводчик) Да. Вроде, как уже и пришёл.

*(Ольга) И чё дальше?

-(Переводчик) А я даже и не понял, когда пришёл.

*(Ольга) Ну, а дальше что?
*(Белимов) А когда душа…

-(Переводчик) Просто - свет.

*(Ольга) И всё? И ты себя ощущаешь, осознаешь. Что ты - это ты.
*(Гера) А вокруг ничего?

-(Переводчик) А даже не поймешь, честно говоря. Я не ощущаю себя что, я - вроде как Свет и есть, этот свет. А, вроде, как и я - в свету просто. Не знаю, как...(выразить. Прим.)

*(Ольга) А свет, какого цвета?

-(Переводчик) Белый… Да, белый.

*(Ольга) С желтизной? Бело-жёлтый. Зелёного нет?

-(Переводчик) Да нет, нет. Просто, я  не знаю, какой цвет. Но яркий, очень яркий такой. Чистый цвет, скажем.

*(Ольга) Чистый, да?
*(Белимов) А когда ты начнешь себя ощущать мальчиком, допустим?
*(Ольга) А дальше? Дальше иди.
*(Белимов) Рассказывай дальше. Как в утробе?  Что ты там… Что там чувствуешь?
*(Ольга) Ты слышишь, что мать-то говорит? Мать, что-нибудь говорит? Слышишь? Что она чувствует? Что она думает?
*(Белимов) Есть ли у тебя страх?

-(Переводчик) Страх? Нет, не страх.

*(Ольга)Страха нет? А любовь? 
*(Гера) Приятное чувство или нет?

-(Переводчик) Нет, Я слышу разговор.

*(Ольга) Какой? Говори.

-(Переводчик) Мужчины и женщины.

*(Ольга) Ну, говори, что они говорят?
*(Белимов) А словами - по смыслу - не понимаешь, да?

-(Переводчик) Нет, я слышу просто разговор, но я не знаю, о чём они говорят.

*(Ольга) А… просто разговор. А эмоции присутствуют?

-(Переводчик) Да.

*(Ольга) Положительные, отрицательные? Или, просто спокойный разговор?

-(Переводчик) Да нет, не спокойный.

*(Ольга) Отрицательный, да?

-(Переводчик) Нет.

*(Ольга) Наоборот, хорошо? Они, наверное, говорят, что вот, скоро у нас кто-то родится.

-(Переводчик) Ну, не знаю. Я слов не понимаю.

*(Ольга) Нет, но, вот эмоции как? Эмоции тебе… как вот?

-(Переводчик) Ну-у…хорошие.

*(Ольга) Хорошие, да? Радостные. Потом, дальше. Давай дальше. Дальше. Ты - родился, рождаешься, вернее.
*(Белимов) Ну, подожди. До рождения ещё довольно много. А это, когда ты почувствовал свою душу, допустим? Это на третьем месяце происходит? Или в самом раннем?  Раньше период? Как ты думаешь?

-(Переводчик) Ну, я не знаю.

*(Ольга) У каждого -  по-своему.

-(Переводчик) Как по месяцам? У меня здесь нет времени.

*(Белимов) А!
*(Ольга) Да-да-да.
*(Гера) Слушай, раз времени нету, слушай, скажи, а как, до того, как ты пришёл, вот,  и в туннель упал, что ты там видел?
*(Ольга) Ты помнишь?
*(Белимов) Ты помнишь, что было?

-(Переводчик) До туннеля?

*(Белимов) Да! Как ты появился?
*(Гера) Откель пришёл?
*(Белимов) К тебе не приходили мужчина и женщина со свитком и не говорили никакие имена? Что ты должен родиться?
*(Гера) Установки не давайте! Пусть сам скажет. [спрашивающим]
*(Ольга) А скажи..

-(Переводчик) Нет. У меня…

*(Белимов) Так.

-(Переводчик) У меня голубой камень.

*(Ольга) Голубой. Ага, понятно у тебя, какой он? Какой? Опиши его. Он имеет вот, камень…как… вот, камень,- ты ощущаешь его, как? Ну, камень… прозрачный, там, - какой? 
*(Белимов) Он ограненный?
*(Ольга) Ты просто чувствуешь, что это?

-(Переводчик) Я не могу сказать какой. Просто, камень и я знаю, что он голубой. 

*(Девушка) И всё?

-(Переводчик) Да.

*(Гера) А кто тебе его дал? Дальше иди тогда.
*(Белимов) Откуда он у тебя появился в руке?

-(Переводчик) Нельзя!

*(Белимов)Нельзя? Нельзя говорить?
*(Гера) Э… погоди, а почему? Кто сказал “нельзя”? Это конец твоей памяти, сознательной, так сказать, или...

-(Переводчик) Нет. Нельзя!

*(Ольга) Хорошо. Давай тогда - к рождению ближе.
*(Гера) Ну, ладно. Не будем.
*(Ольга) Так, к рождению ближе давай, давай. Как ты рождался?
*(Белимов) Боялся уйти из этого мира?

-(Переводчик) Я падаю в туннель.

*(Ольга) Ага!  Падаешь в туннель. А камень с тобой?

-(Переводчик) Я - камень!

*(Ольга) Ты - камень? Ага. А сколько у тебя граней?

-(Переводчик) Не знаю.

*(Ольга) Много?

-(Переводчик) Не знаю!

*(Белимов) Камень, может, не гранёный. [Ольге] Ну, и как рождение происходило? Ты боялся? Страх был? Ты не хотел уходить из того мира где ты был, пробыл девять месяцев? Или ты с радостью шёл?

-(Переводчик) Нет. Боль.

*(Белимов) Боль? Понятно.

-(Переводчик) Умираю.

*(Ольга) А-а! Там умираешь, а здесь рождаешься!?
-(Белимов) Ощущение, что ты умираешь? Ясно. Ты своё имя помнил?

-(Переводчик) Имя?

*(Белимов) Да.
*(Ольга) Такое, каким назовут тебя именем. Знаешь? Кем ты должен родиться, как тебя назовут - ты знаешь это?

-(Переводчик) На камне написано.

*(Белимов) Что написано? А ты помнишь его? Может…

-(Переводчик) Нет, не знаю.

*(Белимов и Ольга) Не знаешь?

-(Переводчик) Я знаю, что там написано мое имя, но я не знаю… (какое.Прим.)

*(Ольга) Как? Какое имя?

-(Переводчик) Там написано Всё.

*(Ольга) Всё. И что ты должен делать, да?
*(Белимов) И твоя судьба написана?

-(Переводчик) Там написано всё.

*(Ольга) Как ты должен прожить жизнь, да?

-(Переводчик) Всё написано!

*(Ольга) И Всё, даже. Ага, всё ясно.
*(Гера) Погоди.
*(Белимов) Но ты никак не можешь вспомнить? Хотя бы отрезки. Как ты проживёшь жизнь? Когда умрёшь? Вот это, не помнишь? 
*(Гера) Не надо - “когда умрешь”.

-(Переводчик) Я не хочу.

*(Белимов) Ясно, не надо.
*(Ольга) Не надо, не надо. А ты вспомни, т.е. Не “вспомни”, а давай, - рождаешься. Как ты вот, ну, перед тем, как родиться, какие у тебя ощущения? Перед тем как родиться здесь в этом мире. Что там... Как, вот ты захотел, или кто “захотел”? Или, как это происходит?
*(Белимов) Что-то выталкивает?

-(Переводчик) Просто, стало неудобно.

*(Ольга) Неудобно, да? А неудобно - это как? Тесно или что?

-(Переводчик) Просто, неудобно.

*(Ольга) И что? Неудобно и нужно куда-то уходить, или выходить?

-(Переводчик) Нет.

*(Белимов) Тебе кто-то подсказывал, что надо выходить? С тобой кто-то общался?

-(Переводчик) Нет. Просто… Просто, меня выкидывают.

*(Ольга) Выкидывают?
*(Гера) А кто?

-(Переводчик) Не знаю.

*(Гера) Ну, и дальше?

-(Переводчик) Боль.

*(Белимов) А потом? Свет? Яркий свет, да?

-(Переводчик) Нет.

*(Ольга) Наоборот?  Темно, да?

-(Переводчик) Нет. Темно.

*(Ольга и Белимов) Темно.
*(Ольга) Да-да.

-(Переводчик) Пятно… Человек… Это пятно превращается в человека.

*(Ольга) Ага.
*(Белимов) Врача?
*(Ольга) Нет.[Белимову]

-(Переводчик) В женщину.

*(Ольга) Мама.

-(Переводчик) Нет, в женщину.

*(Белимов) Ну, это врач. Ну, кто принимает-то роды.
*(Ольга) Ага, превращается в человека. Дальше? Она что говорит?

-(Переводчик) Она через шесть лет умрёт.

*(Гера) Через шесть лет?
*(Ольга) Она?
*(Белимов) Ты её знаешь судьбу? 

-(Переводчик) Я сейчас знаю.

*(Белимов) Имя ты её не помнишь, не знаешь, да?

-(Переводчик) Имя?

*(Белимов) Да. 

-(Переводчик) Наталья Васильевна.

*(Белимов) Так.
*(Гера) И фамилию можно узнать?

-(Переводчик) Синицина.

*(Белимов) Синицина? Это что, она работала в роддоме? 
*(Ольга) Акушер? 

-(Переводчик) Да, наверное.

*(Белимов) Так. А вот ты сейчас знаешь, в каком городе ты родился?

-(Переводчик) Нет.

*(Белимов) Тебе ударили по попке? Для чего? Ты помнишь этот момент, как ты вздохнул в первый раз? 

-(Переводчик) Нет. Никто не бил.

*(Белимов) Никто не бил?
*(Ольга) А ты кричишь?

-(Переводчик) Нет. 

*(Белимов) А как ты вздохнул? Тебе больно было легкие раскрывать? Ты помнишь это?

-(Переводчик) Такое ощущение, что я тону.

*(Белимов) Тонешь?
*(Девушка) Тебе страшно?
*(Гера) Захлёбываешься что ли? Да?

-(Переводчик) Тону и больно.

*(Белимов) А потом? Когда стало легко?

-(Переводчик) Когда закричал.

*(Белимов) Ага, закричал. Ну, вот, легкие прорвал - это.
*(Ольга) Маму, маму… ну, помнишь? Маму.  Это – молоко, молоко захотел. Кушать захотел. Помнишь? Ну-ка, вспомни!

-(Переводчик) Я не хотел.

*(Белимов) Как маму звали? Когда узнал первый раз имя её? Как её зовут?

-(Переводчик) Я всегда знал!

*(Гера) Даже когда шёл?
*(Ольга) А отца? А отца?

-(Переводчик) Я всегда знал!

*(Ольга) А отца?

-(Переводчик) У меня же был камень, и я был этим камнем, а там было всё написано.

*(Белимов) Угу.

-(Переводчик) А я закричал!  

*(Ольга) И забыл.

-(Переводчик) Нет, не всё забыл.

*(Ольга) А отца как зовут?

-(Переводчик) Многое забыл.

*(Белимов) Многое забыл, но кое-что помнишь, всё же, да?
*(Гера) Да не стреляйте вы  в него вопросами [спрашивающим]

-(Переводчик) Многое забыл…  Дедушка… Я сплю... Просыпаюсь… Дедушка умирает.

*(Белимов) Сколько тебе лет?

-(Переводчик) Не знаю, я только маленький.

*(Ольга) Маленький? Маму любишь?... Счёт?
*(Гера) Ты слышишь?
*(Белимов) Одинадцать, двенадцать…
*(Гера) Попробуй, а если обратный [шепчет Белимову о прекращении]
*(Белимов) Так не… ну, может ещё поддержать? Чё, сразу выходить?
*(Гера) 11, 12, 13, 14, 15, 16 ,17, 18, 19
*(Белимов) Гена, ты слышишь нас?
*(Ольга) Назад, назад, назад, назад!
*(Белимов) Подожди. Но мы мало поговорили.
*(Ольга) Ты хотел уйти?[переводчику]

-(Переводчик) Я не хочу об этом говорить.

*(Ольга) Понятно.
*(Белимов) Вот, скажи, когда дедушка умирал, ты маленьким был...

-(Переводчик) Я не хочу об этом.

*(Гера) Ага, хорошо. Тебя вывести назад?

-(Переводчик) Нет. Просто - давайте не будем об этом.

*(Белимов) Давайте, о другом, хорошо…
*(Девушка) Вот, что-нибудь, вспомни, как, вот это, ты первый раз… мать тебя грудью накормила. Молоком. Ну-ка, вспомни, какие у тебя ощущения были?

-(Переводчик) Отвратительные.

*(Ольга и Гера) Отвратительные? Правда?
*(Гера) Ты есть, не хотел сказал, да?
*(Ольга) А ты мать свою любил, как вот, ну, в смысле - ощущения?
*(Белимов) Ну, спрашивайте вы..

-(Переводчик) Я в первый раз принимаю пищу. Именно такую пищу. Ну, конечно, для меня будет отвратительно, я же привык не так питаться.

*(Гера) А как? А как?
*(Ольга) Ну, всё понятно

-(Переводчик) Ну, естественно, я укусил свою мать сразу же. Потому, что не умею. А уж только потом я буду знать, и потом мне уж станет вкусненько-то. А сейчас, мне пихают что-то, а я даже не знаю, что это такое, и для чего это нужно.

*(Ольга) Кошмар[улыбаясь]
*(Белимов) Ясно. Ну, тебя правильно назвали твоим именем? Ты это имя ожидал? Которое на камне было? Или я тут…

-(Переводчик) Я не знаю.

*(Ольга) Скажи, как в школу ты пошёл? В школу пошёл, в первый класс, в первый раз. Вспомни, приятно тебе было? Ну-ка, как ты ждал? Не ждал этого дня?

-(Переводчик) Пять тридцать утра - я вскочил. Родители спят. Я решил, что они проспали, и бегом в школу. Я пришёл, а там - никого.

*(Ольга) [смеется]
*(Белимов) Так и было! [удивляясь]

-(Переводчик) Ну, я долго стоял, носился вокруг этого забора и всё возмущался, когда же начнутся уроки, я же сегодня первый день, а они никто не торопятся. Мне было очень обидно. Потом, пришли родители, они смеялись и даже не стали меня ругать.

*(Ольга) Ну, а за что ругать!? Конечно! А когда десятый класс заканчивал, выпускной помнишь?

-(Переводчик) Помню!

*(Ольга) Ну-ка, расскажи! Приятные были воспоминания или как? 

-(Переводчик) Наконец-то отмучился!

*(Ольга и Гера) [смеются] Так, а когда...

-(Переводчик) Это было, вроде как бы“ отмучился”, и, в тоже время, как-то жалко уходить. Привычка. Вроде, сколько я уже проучился. И такой страх, “а что же дальше- то будет”? А с другой стороны, получается, вроде как – “Ой! Свобода”! В общем, ощущения такие… не очень-то. Поэтому я, наверное, и поспорил на бутылку-то водки, что выпью.

*(Ольга) Ничего себе! И ты выпил, да? [смеясь и удивляясь]
*(Белимов) И ты выпил? [удивляясь]

-(Переводчик) Выпил!

*(Ольга) И чего было?

-(Переводчик) Ничего. Я шёл домой и, действительно, всё кругом шаталось, и я говорил, что - это не брёх собаки --  болтается, всё шатается.

*(Ольга и Белимов) [смеются]
*(Белимов) Правильно![смеясь] Скажи… Ага...

-(Переводчик) Потом, я позвонил в дверь. А я весь испугался, что мать увидит, что я пьяный. И я так…” настроился”. Мать? Мать не заметила. Потом, я лег спать.

*(Ольга) Ничего себе!
*(Белимов) Ну, всё правильно, поди.
*(Ольга) А ещё? У тебя какие мечты были, вот, такие вот, после школы? Вот, ты закончил школу, мечты какие-то были? Ну, сокровенные-то уж не говори, конечно, ну, такие вот…
*(Белимов) Ну, вот, выбор профессии для всех юношей был всегда тяжёл. Он совпал с твоими увлечениями?

-(Переводчик) Нет. Для меня это не было тяжёлым. Для меня было тяжёлым только устроиться на работу и не больше. Я носился по всему городу, и пытался пройти медкомиссию. Потому, что у меня падало зрение и никуда не пропускали.

*(Белимов) Угу. А ты сразу радиотехникой…

-(Переводчик) Я, с Андреем, поехали на химию в поликлинику. Выходя из трамвая — щенок! Я взял этого щенка. И, благодаря этому щенку, я прошёл комиссию.

-(Ольга) Почему?

-(Переводчик) Потому, что я этого щенка подарил медсестре. Она должна была поставить печать.

*(Ольга) А! И ей он понравился! [улыбаясь]

-(Переводчик) Да. Она и забрала его.

*(Гера) Слушай, Гена. Геныч. Один важный момент. Вспомни ту газету с дельфинами. 

-(Переводчик) Помню.

*(Гера) Вспомни всё, что ты хотел знать о ней. Ты меня о чём-то спрашивал, я забыл, о чём ты спрашивал, но тебе эта газета была важна. Что-то она тебе говорила, такое.

-(Переводчик) Эта газета, да это была газета, «Пионерская правда». Там была фотография мальчишки и дельфина. Довольно-то коротенькая статья, просто описание этой фотографии. И я…”Пионерская правда”, - одна копейка. Я купил.

*(Гера) Число какое и месяц?

-(Переводчик) Не знаю. Я…н-да…  Не очень-то хорошо я сделал. Я украл рубль у родителей и купил на этот рубль все газеты. Все сто газет.

*(Ольга) Из-за этой статьи?

-(Переводчик) Нет, из-за этой фотографии.

*(Ольга) Ну, фотографии…
*(Гера) А чего она тебе сказала?

-(Переводчик) Не знаю.

*(Гера) Вспомни, что она тебе тогда говорила? Ощущения вспомни, когда ты смотрел. Что напоминала?
*(Белимов) Другую жизнь, может быть?
*(Гера) Может, ты был дельфином?

-(Переводчик) Нет, не знаю. Нет, по-моему,  просто я завидовал.

*(Гера) А-аа..
*(Ольга) Что мальчик с дельфином.

-(Переводчик) Там был… Нет! Там была статья. Там говорилось, что кто-то держит террариум…Дельфинарий. 

*(Гера) Ну, ну?
*(Белимов) Угу.

-(Переводчик) И вот… фотография, как его сынишка, катается с дельфином. Может, он связан с отцом? Не знаю. Я сам не могу понять, что это было за чувство. Я сейчас не могу. Я просто хорошо помню, но я не помню, чтоб именно чувствами.

*(Гера) А-аа.
*(Белимов) Вот ты интересовался радиотехникой в школе, а скажи, насколько тебя  литература интересовала?
*(Гера) Художественная, да?[Белимову]
*(Белимов) Ну… Писать самому, может быть.

-(Переводчик) 16 Марта семьдесят шестого года  я отослал рассказа в «Юный техник»

*(Белимов) Угу.
*(Девушка) Напечатали?

-(Переводчик) Это был рассказ о голубе и кошке. Как они подружились. Я получил ответ.

*(Ольга) Ну, и чё в этом ответе было?

- В этом ответе было написано, что всё это чепуха, ерунда и такого быть не может.

*(Белимов) Ну, этот сюжет ты всё же выдумал? Или, всё-таки…

-(Переводчик) Сюжет был выдуман, конечно.

*(Белимов) А-аа..Ну, да.
*(Ольга) А я тебе, вот, потом,  расскажу про эпизод, который я видела. Видела своими глазами. Как ты говорил? Про кошку и голубя, да?

-(Переводчик) В пионерской правде. Как говорится, первая моя публикация.

*(Белимов) Была?

-(Переводчик) “Турнир смекалистых” - я участвовал в нём, и там был один из вопросов, об уголовном праве. Я пошёл в библиотеку и один к одному переписал ответ. Ну и с остальными ответами отослал. И вот этот ответ в “шапке” статьи был напечатан. И было подписано  - Геннадий Харитонов.

*(Белимов) Угу.
*(Гера) Сдул, короче [улыбаясь]
*(Ольга) Ясно.[улыбаясь]

-(Переводчик) Ну, почему “сдул”? Вопрос был чисто о праве, и всё.

*(Белимов) Он правильно использовал это.[улыбаясь]
*(Гера) Слушай, ещё такой вопрос, Геныч. Ты можешь вспомнить сны? Ты помнишь сны? Память охватывает этот массив?

-(Переводчик) Ты хочешь спросить о часах? 

*(Гера) Нет. И о часах тоже! Часы помнишь? Цифры можешь называть, какие ты зубрил всю ночь?

-(Переводчик) Часы… 

*(Гера) Повтори.

-(Переводчик) Я помню полностью этот сон.

*(Гера) Повтори цифры.

-(Переводчик) Четверка.

-(Гера) По-порядку.

-(Переводчик) Две ”четверки” - первые. Потом, идёт “восемь”.

*(Гера) Восемь?

-(Переводчик) Нельзя.

*(Ольга) Всё правильно. Ладно.
*(Гера) Почему нельзя?
*(Ольга) Ну, значит нельзя. [Гере]

-(Переводчик) Не знаю… Это я должен сейчас по-другому вспомнить.

*(Белимов) Гена, ты вспоминай такие вещи…
*(Гера) Слушай, ещё вопрос, Геныч. Ты во сне видел книгу. Ты её писал, читал…или что там. Вспомни этот сон. Что ты там…? Хотя бы предложение. Хотя бы.

-(Переводчик) 22 сентября шестьдесят седьмого года, когда я впервые начал читать эту книгу.

*(Гера) Впервые?
*(Ольга) Название?

-(Переводчик) Нет. На... Впервые, вообще, во сне стал читать эту книгу

*(Ольга) А-аа.
*(Гера) Как она называется?

-(Переводчик) Да… Но я читал её… Я же не умел читать? Но я читал её [удивляясь]

*(Ольга) Во сне. Ну, во сне же всё можно, ты же нам сам говорил.
*(Гера) Ну, во сне… Слушай, кассета кончается! Ты давай, это… Чего можешь вспомнить? Предложение или как начинается, кто написал? Дата издания, если есть?

-(Переводчик) Нет, меня это не интересовало. Я читал саму книгу.

*(Гера) Хорошо, что там было?

-(Переводчик) Я помню, что писал и читал. Хорошо, я сейчас попробую вспомнить.

*(Белимов) Да.
*(Гера) Скажешь, когда попробуешь.

-(Переводчик) «Сколь множество раз мы мечтали о встрече, и столько же  уходили от неё, только для того, чтобы снова мечтать. И так прошло множество жизней, ивсё оставалось на своих местах. Однажды, услышав голос и произнеся его вслух, отныне стал го...» “Говорушкой”, что ли?

*(Ольга) Говоруном. Да?

-(Переводчик) Да, наверное, говоруном.

*(Ольга) Ну...
*(Белимов) Переплетается.

-(Переводчик) «И сперва чувствуя игру и забаву, и способ убить время, стал разглядывать и сочинять свои картинки. И если бы не его друзья...»

*(Ольга) ...То?

-(Переводчик) Всё. Надо страницу переворачивать. 

*(Ольга) Переворачивай.


-(Переводчик) Но как? 

*(Гера) Погоди, погоди. Заново! Тебе друзья дают эту книгу?

-(Переводчик) Да!

*(Белимов) Дедушка? Так он же умер!     

-(Переводчик) Дедушка… Он в белом.

*(Белимов) Так.

-(Переводчик) Ну, так, в белой накидке. Ну, как…

*(Белимов) Угу.
*(Гера) Понятно, хитон.

-(Переводчик) Да. Он сам весь белый. В одной руке у него шар. Типа, как у царей держава.

*(Белимов) Угу.
*(Гера) Угу.

-(Переводчик) А в другой - перо. Гусиное перо, большое.

*(Гера) Он Бог пера.
*(Белимов) Диктует это он тебе, да?

-(Переводчик) Нет, нет. И мужчина и женщина.

*(Гера) Это с того сна, который, когда они к тебе приходили, шар давали, да?

-(Переводчик) Да.

*(Гера) Те же самые!?

-(Переводчик) Да.

*(Белимов) Они тебе приятны? Мужчина и женщина.

-(Переводчик) Ну, они же друзья.

*(Белимов) Друзья, да?
*(Гера) Слушай, а как ты идентифицируешь? Они где-то живут? Ну, с какой-то звезды?

-(Переводчик) Никак. Я просто радуюсь, когда  они приходят, и всё. Я никогда не задавал им этого вопроса. И, по-моему, они даже довольны этим.

*(Белимов) А ты чувствуешь - это пришельцы?

-(Переводчик) Я никогда не задумался, кто они. Просто, друзья и всё. Я никогда не задумывался, откуда они берутся и для чего они нужны. Просто друзья.

*(Белимов) Ну, в реальности ты их не встречал? В реальности, а?

-(Переводчик) В реальности?

*(Белимов)Да.
*(Ольга) А что такое - Реальность? [Белимову]
*(Белимов) Ну, - не во сне.

-(Переводчик) В реальности… А в реальности - я иду среди леса, а ко мне выходят животные и прощаются со мной.

*(Белимов) Х-м.
*(Гера) Это другой сон. Ты рассказывал.

-(Переводчик) Это не сон, это реальность!

*(Белимов) Ну, разве так бывает, чтобы выходили животные? Они же боятся человека.

-(Переводчик) Нет, это другое. Это совершенно другой мир.

*(Белимов) А-аа...

-(Переводчик) Они выходят, а я им обещаю, что я вернусь. 

*(Белимов) А может быть ты сам…

-(Переводчик) Выходит дед и говорит, что мне не нужно этого делать. Я говорю, что я сильный, и я смогу справиться и смогу вернуться. Он говорит: -  Вернуться-то ты вернёшься, но это будет слишком долго. Слишком долго ждать, когда ты, наконец-то поймёшь, что самое родное, именно то, что ты, чаще всего, не хочешь иметь. Тебе кажется, что мы уже надоели, и тебе хочется приключений только из-за того, что здесь не можешь найти друзей. А друзей ты не можешь найти только из-за того, что тебе просто лень. Лень дружить. И ещё… Ты хочешь множество знаний для того, чтобы хоть как-то выделиться среди других. Ты хочешь всё пощупать. А зачем, когда это можно все увидеть, и не надо щупать? Но, я - всё равно… я прощаюсь со всеми и куда-то улетаю.

*(Белимов) Скажи, а когда ты почувствовал необычные способности в себе, типа ясновидения, предвидения?


-(Переводчик) Мне было два года.

*(Белимов) Так. Что в это время произошло? Подробно вспоминай.

-(Переводчик) Мать разозлилась на меня и откинула. Я стукнулся о  батарею, заплакал. Мать испугалась. У меня пошла кровь. Мать испугалась. И я почему-то увидел, ну… как бы… увидел её по-другому.

*(Белимов) И понял, что она будет наказана за это?

-(Переводчик) Да нет, зачем? Просто я… Вы спрашиваете, как я первый раз почувствовал? Вот, я и ответил.

*(Гера) Ясно видеть начал.
*(Белимов) Ясно. А потом, ещё были какие-нибудь необычные способности? В школе, допустим. Допустим, ты мог узнать билет, который тебе попадется? Или что-то. Какие-то были ещё случаи?

-(Переводчик) Для меня всё это не было проблемой.

*(Белимов) Как? Как это “не было”?

-(Переводчик) Я всегда… У меня была одна тактика - заучивать первый и последний билет.

-(Белимов) Так, и они тебе, как раз, и попадались?

-(Переводчик) Попадались. Но я, почему-то всегда умудрялся сделать так, что экзаменатор считал, что мне попался именно первый билет. Или последний. Или тот, который я выучил

*(Ольга) [смеется]
*(Белимов) Так, может ты вот даже гипнозом внушал это?

-(Переводчик) Нет, я просто вытаскивал билет и говорил — 23-й. Двадцать третий.

*(Белимов) И рассказывал ему, а он был, как раз, последний? Да?

-(Переводчик) Нет. Это был совершенно другой номер, и я его ложил. И ложил, причём,  очень нагло. Я считал, что наглость - это как раз помогает убрать лишние подозрения. Но, я тут же давал ему в руки тот билет...

(обрыв кассеты.на полу слове)

-(Переводчик) ...что вроде, как-то не удобно. Я-то даже не смотрел на него и ложил на стол, и он говорил: - Отвечай. Или – “Иди готовься”.

*(Белимов) Гена, ну, это мудрость твоя так проявлялась.
*(Ольга) Какая мудрость? Хитрость! [улыбаясь]
*(Белимов) Ну, мудрость,..хитрость…
*(Гера) Гена.. 

-(Переводчик) Бесплатно... Я ездил бесплатно в автобусах потому, что меня не замечает кондуктор. А я играл в это время роль невидимки.

*(Гера) А, и у меня тоже было.

-(Переводчик) У меня не было денег, и я думал про себя “я - невидимка”. Стою среди друзей, у друзей забирают билеты, у меня - нет. 

*(Белимов) И получалось? И ни разу не “залетел”?

-(Переводчик) Получалось. Но это получалось, когда я, именно делал это спонтанно. Когда я тут же старался  специально повторить, то уже - нет. 

*(Белимов) Ага. А ещё какие были случаи, действительно?
*(Гера) Слушай! Ты входил в состояние такое, как будто отрешенности от этого?

-(Переводчик) Нет. Я просто желал, чтобы это было, и всё. Играл. Играл, просто, играл.

*(Гера) Угу, ясно.
*(Белимов) А потом? А взрослым - ты не пробовал это делать?

-(Переводчик) Ну, наверное, сейчас – то же самое, что с моим проездным.

*(Белимов) Т.е. Это проездной не реальный?
*(Гера) Да “липа”!

-(Переводчик) Нет, он не реальный, но ещё в этом никто не сомневался.

*(Белимов) Видишь, как пользуется! То есть в тебе, видимо, есть задатки очень необычного  характера.

-(Переводчик) А у меня, скорее всего, вид такой. Вроде бы, как простенький, вроде не обманывает. А может быть, опять,  та же самая, в принципе, наглость.

*(Гера) Слушай, Геныч, насчёт сна, вот, твоего, когда ты уходил из леса там, прощался со всеми… Да? Ты помнишь?

-(Переводчик) Помню.

*(Гера) Слушай, у меня наверное аналогичная ситуация была во сне, когда я ле... Почти что  аналогичная, ты слышишь сейчас?

-(Переводчик) Слышу.

*(Гера) Помнишь, я тебе рассказывал,.. что планета,.. там какой-то электронный голос передает координаты, я там… ну, всё о ней, короче. А я всё лечу, а эта тарабарщина, значит...

-(Переводчик) Бред.

*(Гера) А?

-(Переводчик) Бред. Я не смогу так долго.

*(Ольга) Ага. Назад счёт, да? А что?

-(Переводчик) Нет. Просто задавайте больше вопросов или что. Я не смогу так долго быть. Меня уже ждут.

*(Ольга) Ага. Ну, давай, назад счёт дадим.
*(Гера) Ну, давай дадим. Ладно, я прекращаю.
*(Белимов) Не-е..  давайте, мне интересно. Гена, как ты вышел на написание парапсихологический свой томик, вот этих… что ты записываешь в общей тетради? Как это пришло к тебе в голову? Как это даются тебе вот эти афоризмы?

-(Переводчик) Очень просто.

*(Белимов) Как? Ну-ка.

-(Переводчик) Гера пришёл и стал хвастаться, что он видел передачу, как экстрасенсы, которые с помощью гипноза куда-то впадают. И он так это горячо рассказывал, что я тоже этого захотел.

*(Ольга) М-мм, всё ясно.
*(Белимов) И получилось?
*(Гера) Это детонатор был, да?
*(Ольга) Нет. [Гере]

-(Переводчик) Ну, наверное, я был уже, похоже, готов к этому. Просто, мне нужна была какая-то причина, толчок, чтобы это сделать.

*(Белимов) И вот, ты так, наверное, вышел и на контактную ситуацию? Так? Ты так же вышел?

-(Переводчик) На контактную ситуацию?

*(Белимов) Да, как вот, как у вас получилось это?
*(Гера) О! Я помню, даже без него.
*(Белимов) Нет, ну, я хочу от него [Гере] Гена, ну, вспоминай!

-(Переводчик) Да, я помню. Мы стояли около вагончика с колёсами. Гера, когда уходил от меня,..просто болтали, пустые разговоры… Он уходил от меня через дырку в заборе. И когда он вышел за забор, он  - “Да! Кстати!”  Я подошёл к забору, и он стал мне рассказывать о передаче по телевидению. И потом, он поспешил, ему было некогда. А на следующий день он пришёл, и мы уже пробовали.

*(Ольга) Гена! Ген, скажи, кто тебя сейчас зовёт? Я вот тоже чувствую, вот уже какое-то время почувствовала что-то такое. Как будто прорывается что-то, кто-то. Или это тебя зовут, да?

-(Переводчик) Да. Меня кто-то зовёт, точнее – говорят, чтобы я был осторожен.

-(Ольга и Гера) А-а.
*(Гера) Ну, это,..ну…
*(Белимов) А чего? Тут же твои друзья сидят. От кого эта осторожность?
*(Гера) Мы же не ведаем, что мы творим. Мы можем спросить вопрос, а ему будет плохо.

-(Переводчик) Дело в том, что я не должен много знать.

*(Ольга) Ага,-  многое говорить.

-(Переводчик) И сам. Дело в том, что если я смогу сейчас прочитать всё что на камне, то это будет довольно трудно. А теперь я хорошо понимаю на себе, что значит -  если предсказатель участвует в предсказанных событиях, то  эти события уже не сбудутся. Поэтому, по этой причине, и поэтому всегда всё говорилось притчами и “вокруг да около”. Только намёками! Только из-за этого, чтобы не нарушить ход событий. 

*(Ольга) Ты знаешь, ты сейчас говоришь, как Сергей Иванов говоришь. Тебе имя это что-нибудь говорит? 

-(Переводчик) Сергей Иванов?

*(Ольга) Да!

-(Переводчик) Да.

*(Ольга) Что говорит?

-(Переводчик) Я знаю трёх Сергеев Ивановых. Один Сергей Иванов - мы с ним вместе работали на телеграфе. Потом он уволился. работал де… Ну, дальше не хочу уже…

*(Ольга и Гера) Ну, да.

-(Переводчик) Второй Иванов. В садике.

*(Ольга) В садике?

-(Переводчик) Да. А я его не знаю… Не должен знать.

*(Белимов) Ты вспомнил, хорошо. Третий Иванов?

-(Переводчик) Так, подожди… Я не должен его знать! Значит, я прочитал тогда. Прочитал от мальчишки. От сына.  Ещё Сергей Иванов…Что-то старое…старое. 

-(Девушка) Угу.
*(Гера) Что  старое?
*(Белимов) Вспоминай получше. Где он живёт? Какая между вами связь была?

-(Переводчик) Так. Ну, это что-то… похоже, это тот Сергей Иванов, о котором-то и… Ну, вот до революции… Дореволюционные времена. 

*(Ольга) М-мм, ну, да.
*(Гера) Ты помнишь о нём чего-нибудь?

-(Переводчик) Я вижу его!

*(Гера и Белимов) Какой он из себя?

-(Переводчик) Молодой, прыщавый. Худой.

*(Ольга) А волосы?

-(Переводчик) Ну, волосы…волосы... 

*(Гера) Коричневые, темные?


-(Переводчик) Нет, белые.

*(Белимов) Белые? Светловолосый.
*(Девушка) Седые.
*(Белимов) Или седые?

-(Переводчик) Да, пожалуй, седые.

*(Белимов) У него была кличка? “Седой”, допустим?
*(Гера) Слушай...[пытается задать вопрос]

-(Переводчик) “Лысый”!

*(Белимов) “Лысый”? [удивляясь] Ты же говорил, что он седой?
*(Ольга) Он старый? [намекает на возраст]

-(Переводчик) Нет, “Лысый”! 

*(Белимов) Молодой, прыщавый.

-(Переводчик) Да, но у него кличка “Лысый”.

*(Гера) Ну, ладно, бог с ним. Скажи...
*(Белимов) А! Кличка “Лысый”! (дошло наконец-таки! Прим.)Хорошо, скажи он на(...) Не помнишь?  Или..(...)

-(Переводчик) Не знаю. Я, просто, вижу его. Я же не знаю, где я его вижу!

*(Белимов) А! Ещё не знаешь.
*(Гера) А комсомольский билет… Ты не видишь его? Номер можешь вспомнить?

-(Переводчик) Нет. Я, просто вижу его, и всё.

*(Девушка) Ну, всё.
*(Гера) М-мм.. номер комсомольского… 
*(Белимов) Гена, скажи, пожалуйста, когда у тебя зародилось, вот, желание писать афоризмы твои? Почему ты их так… такие красивые получаешь тексты? Откуда они идут?
(5 секундное молчание)
*(Ольга) Нельзя.[Белимову]
*(Белимов) Нельзя?

-(Переводчик) Не знаю. Это будет точнее.

*(Ольга) Давай.
*(Белимов) Ну, может быть это твой путь. Так. Тогда, вот мы, может, скоро закончим сеанс? Скажи, пожалуйста, мы какие делаем ошибки при контактах с другими через тебя? Мы не подготовлены или чего?
*(Гера) Это, вообще-то, нужно войти в контакт, а то они, по-моему, наверное,  не знают, как от нас...
*(Белимов) Кто эти четверо, которые должны объединиться?

-(Переводчик) Я этого ничего не скажу. А интонация у вас очень сильно заметна. Например, вот когда… ну, вот сегодня взять. Вы ” девять”  сказали с таким удовольствием что: “Ну, наконец-то”!

*(Гера) “Свершилось”! Да? [улыбаясь]

-(Переводчик) Э… Да, вот. 

*(Гера) Помягче надо быть, да?

-(Переводчик) Нет, просто чувствуется, вот, в интонациях ваших, вот в этих вот словах чувствуются Ваши эмоции. Вот, вот вы, например, с таким удовольствием, что вот типа: “Ну, наконец-то”! В принципе, то же самое, что, как что “девять”, что эта фраза - один к одному. Картинки – то же самое.

*(Гера) Угу, Эмоции дают. Ну, понятно. Слушай...

-(Переводчик) А Гера, очень здорово, когда он… Когда не просто, вот, понимаете, не просто сухой счёт, а когда, типа–“ ну,  давай, давай… ещё немножко”. Ну... ну, такого вот, чистого человеческого, как сегодня ты это сделал. 

-(Белимов) Ну, мы, под-час, не отдаём себе, конечно, отчёта. Ну…Хочется энергетику.. 

-(Переводчик) Не хватает какой-то причастности. Сухость… вот это вот - серой краской, стальной цвет… или цвет.

*(Белимов) Скажи, Гена, я тебе это сегодня говорил на кухне, что у меня важный… ну… переворот в жизни. То есть, переход на другую профессию и прочее…

-(Переводчик) Не-е… Не буду.

*(Белимов)Ты сейчас не будешь?

-(Переводчик) Не. Не буду.

*(Белимов) Ах, то есть, это грозит мне какими-то жизненными трудностями?
*(Ольга) Ну, нельзя будущее знать. [Белимову]

-(Переводчик) Нет. В первую очередь это будет грозить мне больше, а потом только - вам!

*(Белимов) Почему? Ты же в моей судьбе не больно-то и участвуешь!
*(Ольга) Ну, нельзя говорить. Зачем…

-(Переводчик) Ну, как это я ” не участвую”? Участвую! И участвую на прямую!

*(Белимов) Да?

-(Переводчик) Ну, а подумайте. Вон сколько времени было убито! Хотя бы, на те же самые контакты. Вы согласны, что уже изменили многое?

*(Гера) Да.
*(Белимов) Да не сказал бы…
*(Гера) Уже вериться, что душа бессмертна.[ напомнил Белимову его слова. Прим.]

-(Переводчик) Вот, пускай… Хорошо. Пускай… Нет, дело не в душе, не в этом. Хотя и в этом, конечно, есть что-то. А вот, именно… Хотя бы - то самое количество часов! Сколько мы могли бы, сделать многого другого, и изменить какое-то…

*(Белимов) Да нет. Это, я считаю, это одно из самых полезных наших занятий.

-(Переводчик) Я не говорю о вредности. Я говорю о том, что изменилось, изменилось очень многое!

*(Гера) Слушай, Геныч, вот ты скажи, такой вопрос – вот, если кто-то подправляет, так сказать, не дает разгласить то, что должно быть, значит, оно уже где-то у кого-то созрело, и он этот план приносит на Землю?

-(Переводчик) Толчок. 

*(Белимов) Ну, это от Бога идёт?

-(Переводчик) Ой… От Бога… Это сложный вопрос. В общем, честно говоря, как говорить - от бога, от дьявола… В принципе, здесь немножко не верно. Потому, что… ну, как сказать…. Даже самого святого человека услышит голос дьявола. Он просто чутьём понимает. Вот, как мы можем определить - это мужчина это женщина, это ребёнок, а это взрослый человек? Мы даже не можем объяснить толком же, правильно? как мы это понимаем. Вот, так же и он. То же самое - святой человек, допустим, он понимает, что это вот - от дьявола, а это - от бога. 

*(Гера) Так всё едино! Я вот в этом...

-(Переводчик) О нет! Никак не едино. Не едино. Да! Действительно, это Едино, но в Нас сейчас нет этого единства. Кусочки мозаики, всего лишь. Душа - где-то в одном месте, а мы - где-то здесь шляемся. Вот, получается, в душе, в принципе-то, как “материи” нет совсем. Вот, у нас ищут в теле души, а её там нету. Это извне она где-то находится. Причём, находится так далеко… Ну, как далеко? Ну, нельзя сказать даже… и далеко и близко она. Потому, что там другие меры.

*(Белимов) Ну, то есть не внутри.
*(Ольга) Скажи, а вот есть люди…

-(Переводчик) Даже не знаю, не знаю, как сказать. Она, можно сказать что далеко, и в тоже время, можно сказать - именно в душе и живу. Понимаете? Другая мера совершенно. Я, лично, не знаю...

*(Ольга) Ген, скажи, а вот есть люди, которые, как душа, живут. То есть, они чувствуют в себе душу. То есть,  действительно, вот, которая живёт в них. Есть такие люди? Ты встречал?
*(Гера) Такие тут не живут.[Ольге]
*(Ольга) Живут.

-(Переводчик) Да, к сожалению, я не встречал. Потому, что мы все, материально всё- таки, связаны. Она… Понимаете, как получается, в принципе, вот, сколько много, ну, можно сказать, что это всё относится больше всего к монахам, да? А всё равно у них есть, всё равно, у них есть… Хотя бы взять то - чтобы отпраздновать Крещение надо заплатить 40 000, -  это уже говорит, о, всё-таки, материальном. И это, один из способов заработать, всё-таки, деньги. А отлучение церкви… от церкви… это, всё-таки, борьба за власть. 

*(Гера) Политика. Понятное дело.
*(Белимов) Гена, ты сейчас сам...

-(Переводчик) Это даже дело не только в политике. Понимаешь… Мы… дело в том, что вот идёт оно – распятие… да… И, считает, что он будет...

*(Гера) Страдать?

-(Переводчик) Испытает “муки божьи” -  муки Иисуса. И он испытает их. Он даже физически не сможет их испытать, потому что он идёт во славе всего народа, который ведёт его к этому кресту. Понимаешь? И все кричат “Слава”! Наконец-то! Молодец, какой он сильный! И так далее. Ну-у… всё понятно.  Поэтому, даже физически, вот эта эйфория даже физическую боль не даст ему прочувствовать, какая была у Христа. А смысл-то не в физической боли.

*(Гера) Естественно, да.

-(Переводчик)А духовного - он никак не сможет прочувствовать, потому, что духовно  его никто не оплёвывал. 

*(Белимов) Гена, а вот оценивая божественное направление и дьявольское, ты сейчас в этом состоянии сможешь сказать, - увлечение моё контактной ситуацией, это от бога или от дьявола? Нас же обвиняют в этом, что это дьявольские игры.

-(Переводчик) Да. Но видите, как получается -  Церковь обвиняет. Она говорит о движении души, говорит о том, что мы должны духовно расти. А любое движение души она тут же  объявляет бесовщиной. Понимаете? Всё это дьявольщина, дьявольщина. Можно же изучать, мы должны всё изучать, в конце концов, мы должны познать бога. А как мы познаем бога? Изучая ту же физику, химию, те же самые обычные науки. Вот это тоже - познание бога. А и даже когда тот же самый врач… ну, как раньше врачей гоняли? Говорят, что помешал “промыслу божьему”. Бог решил, что он должен был умереть. А ты его лечишь! А потом, наконец-то дошло. Так это что же получается, тогда получается, что врач сильнее бога что ли? Бог решил, чтобы ему умереть, а врач взял и его вылечил. И тоже, - быстро взяли и политику поменяли. Чисто человеческое такое – разрешили, чтобы не было, - убрать вот эти камни преткновения. Поменьше неверия. Чтобы не было, и тоже - убрали метод. И теперь, он помогает божьему промыслу!

*(Гера) Кстати, ” о птичках“!  Вот, действительно, где грань? 
*(Белимов) Гена, а скажи...
*(Гера) Что, запретная тема, да?
*(Белимов) Нельзя? То есть, мы занимаемся богоугодным делом в познании?
*(Гера) Да он, по-моему, ” завис”.
*(Ольга) Назад? Счёт назад?
*(Белимов) Да почему сразу назад? Интересный разговор идёт!
*(Ольга) Ну, он  - всё,- не говорит.
*(Белимов) А потом, нужно обратно.

-(Переводчик) Э-э, слушаю.

*(Белимов) Гена! Но если ты почувствуешь, что пора прекращать разговор, мы можем дать обратный счёт. Не надо? Скажи.  Объясни, что нам делать?
*(Гера) Что там с тобой?
*(Белимов) Мы в твоё состояние первый раз ввели тебя.
*(Гера) Ну, мы-то не первый раз. При вас - в первый.

-(Переводчик) Шесть раз.

*(Белимов) Шесть раз уже было? Хорошо.
*(Ольга) А вот, скажи, счёт не давали с 11-ти до 19-ти, а ты всё равно в это состояние вышел. Ты сам захотел этого?

-(Переводчик) Нет. Это вы давали счёт. Ну, вы же хотели.

*(Белимов) Я хотел, я хотел.
*(Ольга) И я хотела.

-(Переводчик) Вы хотели. А раз вы хотели, значит, вы уже и, значит, его сказали.

*(Белимов) Я хотел тоже, потому, что с теми уже не интересно было беседовать.
*(Гера) Слушай, такой вопрос… Вот эти, Первые, с которыми мы с начала беседовали, они, я так понял, это… специально других просят поговорить, чтобы уж не надоедали мы им?

-(Переводчик) Нет. Просто не имея контроля, как бы. Просто, различный фон эмоций. Идут различные толчки, а значит, различные направления.

*(Гера) То есть, мы тебя “толкаем”?

-(Переводчик) То есть, я, получается, лечу то туда, то сюда. У меня есть желание выйти, просто побывать в той дали.

*(Гера) А! Просто – выйти… А конкретно…

 - (Переводчик) А конкретно, именно что -  куда, - я не знаю. Если я буду даже сейчас  и вы скажете : “ О! Давай попробуем, Ген, так. Давай, ложись и думай – “с первыми, с первыми, с первыми” - я всё равно не выйду. Нет, Степаныч, это бесполезный способ.  Вы уже подумали сейчас об этом... 

*(Белимов) Угу. Да. Ну, я подумал о том, чтобы - как бы ты вышел на Сергея Иванова, потому, что там есть к нам…к ним вопросы. 

-(Переводчик) Ну, вот видите! А я вам отвечаю, что не получится так. Потому, что это будет чисто так вот - сознанием именно это всё твердить. А… ну, не знаю, как это называть.

*(Гера) Слушай, может тебе там видно, как сознание соединить с подсознанием? Как вот это?

-(Переводчик) А как соединить? Оно было всю жизнь соединено. Кто же рассоединял-то?

*(Гера) Нет, ну, как осознать, что ты осознаешь неосознанное? Скажем так. То есть - подсознательное. То есть, сознательно знать, чем ты оперируешь. 

-(Переводчик) Сознание это не допустит, сознание не допустит. Вот почему оно сейчас так говорит? Потому что, просто, ну как вот тот же самый алкоголик, вот.  Он напился, у него нет сейчас тормозов. Он теряет этот контроль, то бишь -  он становиться, относительно подсознания… он становится более....

*(Ольга) Открытым!

-(Переводчик) Не открытым. Почему? Нет, а - более слабым. То есть, притупляется сознание. Притупляется вот эта реакция защиты. Сознание должно себя постоянно защищать. Вот как оно выглядит. Действительно защищает. Чаще всего, мы… не чаще, а всегда, - защищаем себя ложью, потому, что мы ведём себя относительно окружающей среды. В автобусе - мы одни, здесь - мы другие. Мы постоянно лжём и забыли уже, где есть наше настоящее, честно говоря.

*(Ольга) Слушай.

-(Переводчик) А когда начинаешь вспоминать наше настоящее,  мы себе места не находим. Мечемся по квартире, не можем понять, чего же мы хотим.

*(Гера) Да, есть такое.
-(Ольга) Слушай, может поэтому, иногда, люди, ну… говорят - расслабиться надо. То есть, вот это сознание настолько  нас уже…
*(Гера) Завоевало.
*(Ольга) Да. Что человек уже не выдерживает и напивается?

-(Переводчик) Ну, как раз, конечно, это один из самых простых способов расслабиться. Очень просто, но только надо понимать, дело в том, что придёт время отрезвления. 

*(Ольга) Ну, да. А слушай… Ну, как… А вообще, - ещё какие способы есть? 

-(Переводчик) Да множество способов, чтобы… Возьми любое искусство, возьми любую книгу, просто отвлечься именно от этого. А у нас смотри, как получается, - вот у нас засела какая-то мысль, и мы её целыми днями долбим, долбим, мы просыпаемся с этой мыслью, пока “то да сё”, мы постоянно перебираем именно эту мысль, а всё остальное притупилось.

*(Ольга) Слушай, а вот...
*(Гера) Не, ну, нельзя. Погоди.[девушке]. Ну, нельзя же одновременно держать внимание на всех вещах?

-(Переводчик) Ну, нельзя же зацикливаться на одном! А у нас так и получается. Вот, проблема,-  надо подготовиться к лекции. Ложишься спать, проверишь эту лекцию, а так или так это. Просыпаешься, и первая же, первая же мысль - это лекция. А потом, всё остальное - поесть, умыться и так далее.

*(Белимов) Ну, так не надо? Так не надо?

-(Переводчик) Так, получается, и ночью вы работали именно с этой лекцией и больше ни с чем. 

*(Белимов) Угу.

-(Переводчик) А в итоге, получается, вы дали установку, вот эта прямая дорога. Любая попытка шаг в сторону — расстрел. Прыжок в высоту - это попытка улететь. 

*(Ольга) Ген, слушай, вот, кроме тех искусств, которые мы сейчас знаем: живопись, скажем, музыка, культура и так далее, вот, есть ещё какие-то сейчас?

-(Переводчик) Да, для любого  мера своя. Для кого – рыбалка.

-(Девушка) Ну, ясно, ясно. То есть, вот такое, какое-то новое, есть вот направление в творчестве, которые мы ещё не знаем, или, вот, может быть, только начинаем, как бы открывать для себя?

-(Переводчик) Конечно есть.

*(Ольга) Нельзя говорить? Да?
*(Гера) Нет, почему? Да их много! Ты чего? На стыке наук!
*(Белимов) Ну, каких, каких?

-(Переводчик) О, не-е! А причём здесь стык наук?

*(Ольга) Науки или искусства?
*(Белимов) Ну, вроде создания компьютерной графики, виртуальная реальность, так называемая.

-(Переводчик) Это всё к одному относится.

-(Белимов) Видео искусству, да?

-(Переводчик) Да, к одному. Да, конечно, сейчас зарождается очень много. Оно, в принципе, всегда-то  и было так. Но дело в том, что. Ну, как объяснить… Если я сейчас попытаюсь объяснить вот этот вид искусства, то мне придется сравнивать это с другими искусствами, которые уже вроде как похожи, а в тоже время нарисую вам не правильно. Это - первое, а второе - для меня он не подвластен, этот вид искусства. 

*(Белимов) Ну, скажи, психография, - она будет занимать все больше позиции или это случайно...?

-(Переводчик) Нет, она будет всё меньше и меньше завоевывать, потому что, в конце концов, надо больше учиться самому всё больше и больше писать. Самому, а не просто на печатной машинке.

*(Белимов) Ну, ведь на нас кто-то выходит через психографию? 

-(Переводчик) А чаще, выходим сами же и мы.

*(Белимов) Сами?
*(Гера) То есть, подсознательно?

-(Переводчик) А чаще, получается, ну, можно сказать - да, подсознательно.

*(Ольга) Слушай...
*(Белимов) Ну, может,  это нашим умом…

-(Переводчик) Грубо, это грубо говорить, что это именно подсознательно, это… Вы понимаете,  всё вот это, всё вот это, весь наш мир… Человек, - он живёт в своем мире. Получается, - сколько у нас людей, столько и миров. 

*(Гера) Ну, да, в принципе.

-(Переводчик) Да, и получается, вот этот вот мир. Это, всего лишь, только реакция души. Понимаете? Движение, всего лишь только души. Реакция души. И если вот эта реакция, как говориться, совпала, или как-то не была искажена внешним… Почему монахи уходили-то!? Чтобы не исказить эту реакцию! Не исказить её бытовыми проблемами там. И так они уединялись. Как говорили: ” уединиться с богом”.  А в итоге, они, просто-напросто, освобождались от других. Как говориться: «не думать постоянно о лекциях». Чтобы не дать это направление. 

*(Ольга)  Просто, о насущном - грубо так говорить, да? Где-то работает, как что-то... 

-(Переводчик) Да, и в итоге, они становятся, как говориться, одержимыми одной идеей, в данном случае - один говорит, мечтает, постоянно думает о лекциях, как их сделать, а другой  мечтает и думает о боге.  Разницы абсолютно никакой! Что там что там. Просто у этого человека лекция, в конце концов, просто-напросто может даже присниться. А другому,  приснится, просто, бог.

*(Белимов) Гена, ну, я считаю, что психография, это проявление, как раз, иных миров на на нашем земном плане. 

-(Переводчик) Мы слишком много хотим сослаться на иные миры. Себя-то уж не признаём.

*(Белимов) Ну, это не так? Я ошибаюсь, значит?

-(Переводчик) Да нет, зачем, же. Ну, не будем говорить о процентах, но получается и то и другое. Есть да, есть иные миры, а есть своё.  А чаще всего, это вот, как раз, одно из направлений нового искусства. Как покрасивее... вот, как объяснить?

*(Гера) Прибрехать.

-(Переводчик) Да, да! Себя показать. Ну, ”прибрехать” здесь - грубо, а просто, выразить себя. Пускай, даже ложно, но, всё-таки, выразиться. И большинство, конечно, из психографии если вот заметить, что, в принципе, идёт повтор. Повтор всего, что было! И при своей жизни, как говориться, он где-то это, если не сознанием, так подсознанием, и где-то это услышал, где-то это видел. Вот, пожалуйста. Потом, у него это всё  перерабатывается. Но нельзя сказать, что это -  просто переписал, - нет, он переработал, он пережил это, и выдал. Я не говорю о чистом плагиате. Чистый плагиат - это всё, просто, взял, просто списал и переживать ничего не надо. 

*(Белимов) Угу.

-(Переводчик) Было же так, что были написаны две абсолютно одинаковые книги.

*(Гера) Как?
*(Белимов) Да? Было?

-(Переводчик) Было! И что? Нельзя же обвинить их в плагиатстве? Нет! Удивительно стало. А сколько открытий было одинаковых!?

*(Гера) Ну, да. Вообще-то, открытия были.

-(Переводчик) Ну, можно сейчас сказать, - ”да какой - “науки”? Он, просто, первый заметил и всё, и не больше”. Потому, что все науки, как говориться, продвигались к этому моменту. Если бы не было... 

*(Гера) Да,  телефон изобрели с разницей в час. Запатентовали. В один час...
*(Белимов) Гена, а ты не помнишь, кто сочинил эти одинаковые книги? Не можешь сказать, о чём они? Это очень интересное замечание.
*(Ольга) Нет. Почему...

(Обрыв)

*(Гера) Слушай, я у тебя хотел что-то спросить... Господи, забыл… Что-то я хотел спросить у тебя такое…

-(Переводчик) У тебя сейчас что-то такое тёмное проскочило.

*(Гера) Ко мне? Или от меня?

-(Переводчик) От тебя - что-то тёмное. 

*(Белимов) Темная мысль, наверное.
*(Гера) Нет. Да я, вообще-то, что-то такое…
*(Ольга) Ну, то есть…

-(Переводчик) Нет, не мысль… Какое-то, э-мм… какая-то уверенность  на каком-то эгоизме, такое вот, замешанное на каком-то, или эгоизм, замешанный на самоуверенности. Ну, что-то такое.

*(Гера) А… Ну, это я говорил, наверное…  Когда говорил, что с тобой всё в порядке.

-(Переводчик) О, я не знаю. Я не знаю, где сейчас был.

*(Белимов) А почему ты сейчас завис? Ты не понял?

-(Переводчик) Нет, этого я не знаю.

*(Ольга) Ген, скажи, вот я такой вопрос хочу задать… Вот мы, когда попали с тобой к Хранителям Времён, помнишь?

-(Переводчик) Я их видел даже!

*(Ольга) Да ты что!?
*(Белимов) Как они выглядели? Это не люди?
*(Ольга) Это не те Двенадцать, этих…
*(Ольга и Гера) Старцев.

-(Переводчик) О, нет. 

*(Ольга)Нет?

-(Переводчик) Ну, как “видеть” сказать?.. Ощущал, скорее всего. Это что-то, какая-то… чернота, может быть, такое...объемное что-то. Ну, я не знаю, как… 

*(Гера) Может - громада какая-нибудь,  или мощность?

-(Переводчик) Да, нет. Знаешь, какая… Бесконечность какая-то. Именно вот…

*(Ольга) Черная дыра?

-(Переводчик) Да, наверное, черная дыра. Какая-то такая бесконечность, такая вот, и причём, вот, вроде бы темнота, но темно всё… не мгла. Не видно ничего, но это всё живое! Дышит! И всё это - такое огромнейшее! И, кстати, “темнота” - вот эта мысль не верна. Сразу - что это сразу - что это злое.

*(Ольга) Да, да, да, да. Ведь всё произошло, по индийским всем этим Ведам, что всё произошло из Тьмы. И Они на контакте нам тоже не раз говорили...
*(Гера) А ты слышала: знание Свет, а не знание – Тьма?
*(Девушка) Ну, это уже другое!

-(Переводчик) А изначально, она имела совершенно другой смысл. Незнаний – тьма. Имелось в виду, - множество не знаем чего. А мы уже переименовали в другое. 

*(Ольга) Да, потому что тьма - это “много”. Вот Мабу нам что говорил? Что такое тьма? Это - много времени. Да, ведь? 

-(Переводчик) Да. И, в этом случае, подходит, просто, теряет первоначальное значение.

*(Белимов) Угу.
*(Ольга) Слушай, Ну, вот, как бы вот… Ты не знаешь, чтобы мне туда не попадать больше? Зачем мне, если я...

-(Переводчик) А ничего страшного нет, что ты туда попадаёшь. Страшно не это, а страшно то, что боишься. 

*(Ольга) Ах, вот оно как.

-(Переводчик) А если боишься, то уже чувствуешь какой-то своей вины. Это - первое. И второе, - что вы очень много сочинили, чего можно, чего нельзя. Понимаете, вы сочинили, что больше трех раз счёт нельзя давать. Этого никогда нигде не было сказано. 

*(Ольга) Нет, говорили, говорили на контакте. Лично мне говорили.

-(Переводчик) Нет, было сказано не так. Было сказано, что “вы не должны останавливать счёт посредине, а вы это сделали уже больше трёх раз”. И, причём, тут же в этом сеансе.  

*(Ольга) А…Ну, я не так поняла.
*(Гера) Ну, вот! А мы…

-(Переводчик) То есть, надо всегда завершать счёт до конца, даже если я проснулся, и было, всего лишь, только “восемь”, - всё равно, счёт нужно вести до единицы постоянно. Или на оборот, надо всегда завершать. Если я начал говорить, “Спрашивайте” там, говорю уже, - всё равно, надо очень быстро, но закончить счёт. 

*(Гера) А мысленно можно закончить?
*(Ольга) Нет.

-(Переводчик) Нет. Ты же работаешь сейчас не мысленно со мной.

*(Ольга) Всё ясно. А вот Они нам дали когда понять, - когда мы досчитали, тогда, по-моему, до семи, а потом назад счёт - с семи.
*(Гера) Так Они, иногда, меня сами и прерывали так: “Достаточно”, и “Продолжить”.

-(Переводчик) Но это сделали они, - не сами, а вы, очень много придумали своих правил. Вот она, - это чёрное/тёмное. Вот это, вот, что я тебе говорил в тебе. Вот эта уверенность, что” надо так”. А она ничем не обоснована эта уверенность. Конечно, вы сейчас скажете, что самая первая мысль, она верна, эта мысль. Это не правильно.

*(Гера) Как!?

-(Переводчик) Это не верно! Это не верно. Потому, что самую первую мысль может дать и сознание!

*(Гера) У-у! 

-(Переводчик) Что –“ну”? Вспомни, в библии, как сказано? 

*(Гера) Да не ” ну”,  я  у-у”... говорю…

-(Переводчик) Самая первая реакция была Ильи-то, что этого “не может быть”. Самая первая мысль, и тогда Иисус объявил его, что он и есть тот дьявол, который соблазняет. А мы-то не знаем причин, мы не знаем, что это мысль была, именно первая мысль.  Это первая мысль, которая у нас первая начала на языке ворочаться. 

*(Гера)А что ей ворочать помогло?

-(Переводчик)А это, всего лишь только уже, как говориться, чисто физический процесс идёт.

*(Гера)То есть, сознание. Согласно сознания.

-(Переводчик)Да. А вот, истину, что мы думаем, она же происходит внутри нас, и мы это никак не ощущаем. Мы её можем ощутить только чувствами, и не больше.

*(Ольга)Ген, а ты принимал участие в Библии? В написании?
*(Гера)Нет, мы же жили в России, нам же сказали.

-(Переводчик) Нет.

*(Ольга)Нет? Не было такого? Ну, там, монах когда... Ещё до... Ну, то есть, выходили на контакт когда...То есть, был этим, в монастырях... ну, и так далее. Я имею в виду в таком плане. (имелось в виду - не были ли наши контакты как-то причастны к написании Библии. прим.)

-(Переводчик)Ну, я сейчас нахожусь, вообще-то, в подсознании. А подсознание - это что? Просто полный объём памяти, но именно за эту жизнь и не больше.

*(Гера)А глубже капнуть?

-(Переводчик) А если взять... как... Ну, вот я говорил про Сергея Иванова, да? Но, в принципе, ведь это могло быть и ложное, благодаря тому, что это было в контактах.

*(Гера)Да.

-(Переводчик)Понимаете...Химически... Химически - я в прошлой жизни-то не жил же. Правильно? У меня сейчас же совсем другая химия.

*(Белимов)Ну, да, соответственно.

-(Переводчик)И поэтому, я не могу точно вам сказать, что именно реинкарнации, допустим, существует. Хотя я чувствую, я знаю, что они есть, но я не могу это ничем доказать. И, что самое интересное, это не должно иметь доказательств.

*(Ольга)Всё правильно. Да, верно.

-(Переводчик)Потому, что мы тогда потеряем интерес и совершенно изменим жизнь, и причем в худшую.

*(Гера)К лучшему? Значит, мы всё-таки по лучшему идём? 

-(Переводчик)К худшему! К худшему!

*(Ольга)Мы ещё раз будем, и всегда будем. Ещё нам "работать"...

-(Переводчик)Да, нам сейчас нельзя об этом... И поэтому, всё, любое вот, какие-то... Да, приходят доказательства. Есть доказательства, неопровержимые доказательства, что прошлая жизнь существует. Но, конечно же, оно нужно это. Поддерживать в нас нужно эту искорку. Понимаете? Огонь. Огонь, но чтобы не был большим, чтобы не жечь нас.

*(Гера)Угу.

-(Переводчик)И... поэтому, тут же, как говориться, подливается “вода” - что, мол, нету реинкарнаций. Вот, для этого и существуют люди верующие и не верующие, потому, что тот же самый фанатик - он не нужен никому, и тому же Богу он не нужен, потому, что фанатик - это тот же самый одержимый, тот же самый "больной". И тогда, для этого, получается, нужен и неверующий, сомневающийся. Потому, что идёт борьба.

*(Белимов) Нет, ну , тут уже идёт движение, действительно.
*(Гера)Не... Ну, как можно, не веря - строить чего-то? Понимаешь?
*(Белимов) Нет, он стимулирует того, кто верит, находить аргументы и всё прочее.
*(Гера)А-а, для поддержки.

-(Переводчик) Конечно. 

*(Белимов)У нас борьба сейчас. Моя жена не верит, а я стараюсь ей доказать.
*(Гера)Так получается, что каждый из вас - фанатик, а Богу они не нужны.
*(Белимов)Я ищу...
*(Ольга)Тихо, тихо!

-(Переводчик) Сейчас ты начинаешь говорить, даже не поняв ещё что - ты начинаешь уже говорить. Понимаешь? А в итоге получается - или срыв контакта, или переход на совершенно другой разговор и причём прежний остаётся не законченным. Вот ты не правильно понял сейчас - тут же это надо разворачивать, приходится поправлять. А чаще - это бывает просто сбой, потому, что у тебя самая первая реакция происходит - это...